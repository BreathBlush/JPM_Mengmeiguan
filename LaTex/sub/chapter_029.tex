\includepdf[pages={57,58},fitpaper=false]{tst.pdf}
\chapter*{第二十九囬 \\吳神仙貴賤相人 潘金蓮蘭湯午戰}
\addcontentsline{toc}{chapter}{第二十九囬 吳神仙貴賤相人 潘金蓮蘭湯午戰}
\markboth{{\titlename}卷之三}{第二十九囬 吳神仙貴賤相人 潘金蓮蘭湯午戰}


\begin{myquote}
百年秋月與春花,展放眉頭莫自嗟!

吟幾首詩消世慮,酌二盃酒度韶華;

閒敲棋子心情樂,悶撥瑤琴興趣賒:

人事與時俱不管,且將詩酒作生涯。
\end{myquote}

話説到次日,潘金蓮早起,打發西門慶出門,記掛着要做那紅鞋。㧱着針線筐兒,往花園翡翠軒臺基兒上坐着,那裏描畫鞋扇,使春梅請了李瓶兒來到。李瓶兒問道:「姐姐,你描畫的是甚麽?」金蓮道:「要做一雙大紅光素緞子白綾平底鞋兒,鞋尖兒上扣綉『鸚鵡摘桃』。」李瓶兒道:「我有一方大紅十樣錦緞子,也照依姐姐描恁一雙兒,我要做高底的罷。」於是取了針線筐,兩個同一䖏做。金蓮描了一隻,丢下説道:「李大姐,你替我描這一隻,等我後邊把孟三姐叫了來。他昨日對我説,他也要做鞋哩!」一直走到後邊。玉樓房中倚着護炕兒,手中也衲着一隻鞋兒哩。金蓮進門,玉樓道:「你早辦?」金蓮道:「我起的早,打發他爹往門外與賀千戶送行去了。教我約下李大姐,花園裏趕早涼做些生活。等住囬日頭過,熱了做不的。我纔描了一隻鞋,教李大姐替我描着,徑來約你同去,喒三個一答兒哩好做。」因問:「你手裏衲的是甚麽鞋?」玉樓道:「是昨日你看我開的那雙玄色緞子鞋。」金蓮道:「你好漢,又早衲出一隻來了!」玉樓道:「那隻昨日就衲了,這一隻又衲了好些了。」金蓮接過看了一囬説:「你這個到明日使甚麽雲頭子?」玉樓道:「我比不得你們小後生,花花黎黎。我老人家了,使羊皮金緝的雲頭子罷。週圍㧱紗綠線鎖出白山子兒,上白綾高底穿好不好?」金蓮道:「也罷。你快收拾,咱去來,李瓶兒那裏等着哩!」玉樓道:「你坐着,咱喫了茶去。」金蓮道:「不喫罷,咱㧱了茶那裏喫去來。」玉樓吩咐蘭香:「炖下茶送去。」兩個婦人手拉着手兒,袖着鞋扇,逕往外走。吳月娘剛上房穿廊下坐,便問:「你們那去?」金蓮道:「李大姐使我替他呌孟三兒,去與他描鞋。」説着,一直來到花園内。

三人一處坐下,㧱起鞋扇,你瞧我的,我瞧你的,都瞧了一遍。先是春梅㧱茶來喫了,然後李瓶兒那邊的茶到,孟玉樓房裏蘭香落後纔㧱茶至。三人喫了,玉樓便道:「六姐,你平白又做平底子紅鞋做甚麽?不如高底鞋好看。你若嫌木底子響脚,也似我用毡底子,卻不好?走着又不響。」金蓮道:「不是穿的鞋,是睡鞋。也是他爹,因我不見了那隻睡鞋,被小奴才兒偷了,弄油了我的,吩咐教我従新又做這雙鞋。」玉樓道:「又説鞋哩!這個也不是舌頭,李大姐在這裏聽着。昨日因你不見了這隻鞋,來昭家孩子小鐵棍兒怎的花園裏拾了,後來不知你怎的知道了,對他爹説,打了小鐵棍兒一頓。説把他猴子打的鼻口流血,躺在地下死了半日,惹的一丈青好不在後邊海罵。罵那個淫婦王八羔子學舌,打了他小廝。説他小廝一點尿不曉孩子,曉的甚麽?便唆調打了他恁一頓。早是活了,若死了,淫婦王八羔子也不得清潔!俺再不知罵淫婦王八羔子是誰?落後小鐵棍兒進來,他大姐姐問他:『你爹為甚麽打你?』小廝纔説;『因在花園裏耍子,拾了一隻鞋,問姑父換圈兒來。不知甚麽人對俺爹説了,教爹打我一頓。我如今尋姑夫,問他要圈兒去也。』説畢,一直往前跑了。原來罵的王八羔子是陳姐夫。早是只李嬌兒在傍邊坐着,大姐沒在跟前。若聽見時,又是一塲兒。」金蓮問:「大姐姐沒説甚麽?」玉樓道:「你還説哩!大姐姐好不説你哩!説:『如今這一家子亂世為王,九條尾狐狸精出世了,把昏君祸亂的貶子休妻。想着去了的來旺兒小廝,好好的従南邊來了,東一帳,西一帳,説他老婆養着主子,又説他怎的㧱刀弄杖,成日做賊哩,養漢哩,生生兒祸弄的打發他出去了。把個媳婦又逼臨的吊死了。如今為一隻鞋子,又這等驚天動地反亂。你的鞋好好穿在脚上,怎的教小廝拾了?想必喫醉了,在那花園裏和漢子不知怎的餳成一塊,纔掉了鞋!如今沒的摭羞,㧱小廝頂缸,打他這一頓,又不曾為甚麽大事。』」金蓮聽了道:「沒的那扯𣭈淡!甚麽是大事?殺了人是大事了,奴才沒刀子要殺主子!」向玉樓道:「孟三姐,早是瞞不了你,咱兩個聽見來興兒説了一聲,唬的甚麽樣兒的。你是他的大老婆,倒説這個話!你也不管,我也不管,教奴才殺了漢子纔好!老婆成日在你那後邊使喚,你縱容着他,不管教他,欺大滅小,和這個合氣,和那個合氣。各人寃有頭,債有主,你揭條我,我揭條你,吊死了你還瞞着漢子不説!早是花了錢,好人情説下來了,不然怎了?你這時推乾淨,説面子話兒!右右是左右,我調唆漢子也罷。若不教他把奴才老婆漢子一條提攆的離門離戶也不算,恆屬人挾不到我井裏頭!」

玉樓見金蓮粉面通紅,惱了,又勸道:「六姐,你我姊妹都是一個人,我聽見的話兒有個不對你説?説了,只放在你心裏,休要使出來。」金蓮不依他,到晚等的西門慶進入他房來,一五一十告西門慶説,來昭媳婦子一丈青怎的在後邊指罵,説你打了他孩子,要邏楂兒和人嚷。這西門慶不聽便罷,聽了記在心裏。到次日,要攆來昭三口子出門,多虧月娘再三攔勸下。不容他在家,打發他往獅子街房子那裏看守,替了平安兒來家看守大門。後次月娘知道,甚惱金蓮,不在話下。正是:事不三思終有悔,人逢得意早回頭。

卻説西門慶在前廳打發來昭三口子,搬移獅子街看守房屋去。一日,正在前廳坐,忽有看守大門的平安兒來報:「守備府周爺差人送了一位相面先生,名喚吳神仙,在門首伺候見爹。」西門慶喚來人進見,遞上守備帖兒,然後道:「有請。」須臾,那吳神仙頭戴青布道巾,身穿布袍草履,腰繫黄絲雙穗縧,手執龜殼扇子,自外飄然進來。年約四十之上,生的神清如長江皓月,貌古似太華喬松,威儀凛凛,道貌堂堂。原來神仙有四般古怪:身如松,聲如鐘,坐如弓,走如風。但見他:

\begin{myquote}
能通風鑑,善究子平。觀乾象能識陰陽,察龍經明知風水。五星深講,三命秘談。審格局,決一世之榮枯;觀氣色,定行年之休咎。若非華嶽修眞客,定是成都賣卜人。
\end{myquote}

西門慶見神仙進來,忙降階迎接,接至廳上。神仙見西門慶,長揖稽首,禮畢就坐。須臾茶罷,西門慶動問神仙高名雅號,仙鄉何䖏,因何與周大人相識。那吳神仙坐上欠身道:「貧道姓吳名奭,道號守眞。本貫浙江仙遊人。自幼従師天台山紫虚觀出家。雲遊上國,因往岱宗訪道,道經貴䖏。周老總兵相約,看他老夫人目疾,特送來府上觀相。」西門慶道:「老仙長會那幾家陰陽?通那幾家相法?」神仙道:「貧道粗知十三家子平,善曉麻衣相法,又曉六壬神課。常施薬救人,不愛世財,隨時住世。」西門慶聽言,益加敬重,誇道:「眞乃謂之神仙也!」一面令左右放桌兒,擺齋管待神仙。神仙道:「周老總兵送貧道來,未曾觀相造,豈可先要賜齋!」西門慶笑道:「仙長遠來,一定未用早齋。待用過,看命未遲。」

於是陪着神仙喫了些齋食素饌,擡過桌席,拂抹乾淨,討筆硯來。神仙道:「請先觀貴造,然後觀相尊容。」西門慶便説與八字:「屬虎的,二十九歲了,七月二十八日子時生。」這神仙暗暗掐指尋紋,良久説道:「官人貴造丙寅年,辛酉月,壬午日,丙子時,七月廿三日白露,已交八月算命。月令提剛辛酉,理取傷官格。子平云:傷官傷盡復生財,財旺生官福轉來。立命申宫,是城頭土命:七歲行運辛酉,十七行壬戌,二十七癸亥,三十七甲子,四十七乙丑。官人貴造,依貧道所講,元命貴旺,八字清奇,非貴則榮之造。但戊土傷官,生在七八月,身忒旺了。幸得壬午日干,子中有癸水,水火相濟,乃成大器。丙子時,丙合辛生,後來定掌威權之職。一生盛旺,快楽安然,發福遷官,主生貴子。為人一生耿直,幹事無二,喜則和氣春風,怒則迅雷烈火。一生多得妻財,不少紗帽戴。臨死有二子送老。今歲丁未流年,丁壬相合。目下丁火來尅。若你尅我者為官鬼,必主平地登雲之喜,添官進祿之榮。大運現行癸亥,戊土得癸水滋潤,定見發生。目下透出紅鸞天喜,熊羆之兆。又命宫馹馬臨申,不過七月必見矣。」西門慶問道:「我後來運限何如?有災沒有?」神仙道:「官人休怪我説,但八字中不宜陰水太多,後到甲子運中,常在陰人之上;又是多了年流星打攪,又把個壬午日衝破了,不出六六之年,主有嘔血流膿之災,骨瘦形衰之病。」西門慶問道:「於今如何?」神仙道:「目今流年,至多日逢破敗五鬼在家炒鬧,些小氣惱,不足為災,都被喜氣神臨門衝散了。」西門慶道:「命中還有敗否?」神仙道:「年趕着月,月趕着日,實難矣。」西門慶聽了,滿心歡喜。便道:「先生,你相我面何如?」神仙道:「請尊容轉正,貧道觀之。」西門慶把座兒掇了一掇。神仙相道:「夫相者,有心無相,相逐心生。有相無心,相隨心滅。吾觀官人,頭圓項短,必為享福之人;體健觔強,決是英豪之輩;天庭高聳,一生衣祿無虧;地閣方圓,晚歲榮華定取。此幾樁兒好䖏。還有幾樁不足之䖏,貧道不敢説。」西門慶道:「仙長但説無妨。」神仙道:「請官人走兩步看。」西門慶眞個走了幾步。神仙道:「你行如擺柳,必主傷妻;魚尾多紋,終湏勞碌。眼不哭而淚汪汪,心無慮而眉縮縮,若無刑尅,必損其身。妻宫尅過方可。」西門慶道:「已刑過了。」神仙道:「請出手來看一看。」西門慶舒手來與神仙看。神仙道:「智慧生於皮毛,苦楽勸乎手足;細軟豐潤,必享福逸楽之人也。兩目雌雄,必主富而多詐;眉抽二尾,一生常自足歡娛;根有三紋,中年必然多耗散;奸門紅紫,一生廣得妻財;黄氣發於高廣,旬日内必定加官;紅色起於三陽,今歲間必生貴子。又有一件不敢説:淚堂豐厚,亦主貪花;谷道亂毛,號為淫杪。且喜得鼻乃財星,驗中年之造化;承漿地閣,管末世之榮枯:

\begin{myquote}
承槳地閣要豐隆,準乃財星居正中。

生平造化皆由命,相法玄機定不容。」
\end{myquote}

神仙相畢,西門慶道:「請仙長相相房下衆人。」一面令小廝:「後邊請你大娘出來。」於是李嬌兒、孟玉樓、潘金蓮、李瓶兒、孫雪娥等衆人都跟出來,在軟屏後潛聽。神仙見月娘出來,連忙道了稽首,也不敢坐,在傍邊觀相,「請娘子尊容轉正。」那吳月娘把面容朝看廳外。神仙端詳了一囬説:「娘子面如滿月,家道興隆;唇若紅蓮,衣食豐足。山根不断,必得貴夫而生子;聲響神清,必益夫而發福。請出手來。」月娘従袖口中,露出十指春蔥來。神仙道:「乾姜之手,女人必善持家;照人之鬢,坤道定須秀氣。這幾樁好䖏。還有些不足之䖏,休道貧道直説。」西門慶道:「仙長但説無妨。」神仙道:「淚堂黑痣,若無宿疾必刑夫;眼下皺紋,亦主六親若氷炭。

\begin{myquote}
女人端正好容儀,緩步輕如出水龜。

行不動塵言有節,無肩定作貴人妻。」
\end{myquote}

相畢,月娘退後。西門慶道:「還有小妾輩請看看。」於是李嬌兒過來。神仙觀看良久,「此位娘子,額尖鼻小,非側室必三嫁其夫;肉重身肥,廣有衣食而榮華安享。肩聳聲泣,不賤則孤;鼻梁若低,非貧即夭。請走幾步我看。」李嬌兒走了幾步。神仙道:

\begin{myquote}
「額尖露臀並蛇行,早年必定落風塵。

假饒不是娼門女,也是屏風後立人。」
\end{myquote}

相畢,李嬌兒下去。吳月娘呌:「孟三姐,你也過來相一相。」神仙觀看,「這位娘子,三停平等,一生衣祿無虧;六府豐隆,晚歲榮華定取。平生少疾,皆因月孛光輝;到老無災,大抵年宫潤秀。請娘子走兩步。」玉樓走了兩步。神仙道:

\begin{myquote}
「口如四字神清徹,溫厚堪同掌上珠。

威媚兼全財命有,終主刑夫兩有餘。」
\end{myquote}

玉樓相畢,叫潘金蓮過來。那潘金蓮只顧嬉笑,不肯過來。月娘催之再三,方纔出見。神仙擡頭觀看這個婦人,沉吟半日,方纔説道:「此位娘子,髮濃鬢重,兼斜視以多淫;臉媚眉彎,身不搖而自顫。面上黑痣,必主刑夫;人中短促,終須壽夭。

\begin{myquote}
舉止輕浮惟好淫,眼如點漆壞人倫。

月下星前長不足,雖居大廈少安心。」
\end{myquote}

相畢金蓮,西門慶又叫李瓶兒上來教神仙相一相。神仙觀看這個女人,「皮膚香細,乃富室之女娘;容貌端莊,乃素門之德婦。只是多了眼光如醉,主桑中之約無窮;眉靨漸生,月下之期難定。觀臥蠶明潤而紫色,必産貴兒;體白肩圓,必受夫之寵愛。常遭疾厄,只因根上昏沉;頻遇喜祥,盖謂福堂明潤。此幾樁好䖏。還有幾樁不足䖏,娘子可當戒之;山根青黑,三九前後定見哭聲;法令綳纏,鷄犬之年焉可過!愼之,愼之!

\begin{myquote}
花月儀容惜羽翰,平生良友鳳和鸞。

朱門財祿堪依倚,莫把凡禽一樣看。」
\end{myquote}

相畢,李瓶兒下去。月娘令孫雪娥出來相一相。神仙看了,説道:「這位娘子,體矮聲高,額尖鼻小,雖然出谷遷喬,但一生冷笑無情,作事機深内重。只是喫了這四反的虧,後來必主兇亡。夫四反者,唇反無稜、耳反無輪、眼反無神、鼻反不正故也。

\begin{myquote}
燕體蜂腰是賤人,眼如流水不廉眞。

常時斜倚門兒立,不為婢妾必風塵。」
\end{myquote}

雪娥下去,月娘教大姐上來相一相。神仙道:「這位女娘,鼻梁仰露,破祖刑家;聲若破鑼,家私消散。面皮太急,雖溝洫長而壽亦夭;行如雀躍,䖏家室而衣食缺乏。不過三九,當受折麽。

\begin{myquote}
惟夫反目性通靈,父母衣食僅養身;

狀貌有拘難顯達,不遭惡死也艱辛。」
\end{myquote}

大姐相畢,教春梅也上來教神仙相相。神仙睜眼兒見了春梅,年紀不上二九,頭戴銀絲雲髻兒,白線挑衫兒,桃紅裙子,藍紗比甲兒,纏手縛脚出來,道了萬福。神仙觀看良久,相道:「此位小姐,五官端正,骨格清奇。髮細眉濃,禀性要強;神急眼圓,為人急燥。山根不断,必得貴夫而生子;兩額朝拱,主早年必戴珠冠。行步若飛仙,聲響神清,必益夫而得祿,三九定然封贈。但喫了這左眼大,早年尅父;右眼小,周歲尅娘。左口角下只一點黑痣,主常霑啾唧之災;右腮一點黑痣,一生受夫愛敬。

\begin{myquote}
天庭端正五官平,口若塗硃行步輕;

倉庫豐盈財祿厚,一生常得貴人憐。」
\end{myquote}

神仙相畢,衆婦女皆咬指以為神相。西門慶封白銀五兩與神仙,又賞守備府來人銀五錢,拏拜帖囬謝。吳神仙再三辭卻,説道:「貧道雲遊四方,風餐露宿,化救萬道,周總兵送將過來,可一時之情耳,要這財何用?決不敢受。」西門慶不得已,㧱出一疋大布:「送仙長做一件大衣何如?」神仙方纔受之,令小童接了,收在經包内,稽首拜謝。西門慶送出大門,揚長飄然而去。正是:柱杖兩頭挑日月,葫蘆一個隱山川。

西門慶送神仙出,囬到後廳問月娘衆人:「所相何如?」月娘道:「相的也都好,只是三個人相不着。」西門慶道:「那三個人相不着?」月娘道:「相李大姐有宿疾,到明日生貴子。他現今懷着身孕,這個也罷了。相喒家大姐到明日受折磨,不知怎的折磨?相春梅後日也生貴子,或者只怕你用了他,各人子孫,也看不見。我只不信説他春梅後來戴珠冠,有夫人之分。端的喒家又沒官,那討珠冠來?就有珠冠,也輪不到他頭上!」西門慶笑道:「他相我目下有平地登雲之喜,加官進祿之榮,我那得官來?他見春梅和你們站在一䖏,又打扮不同,戴着銀絲雲髻兒,只當是你我親生養女兒一般,或後來匹配名門,招個貴婿;故説有珠冠之分。自古算的着命,算不着好。相逐心生,相隨心滅。周大人送來,喒不好囂他的頭,教他相相除疑罷了。」説畢,月娘房中擺下飯,打發喫了飯。

西門慶手㧱芭蕉扇兒,信步閒遊,來花園大捲棚内聚景堂内,週圍放下簾櫳,四下花木掩映。正値日當午時分,只聞綠陰深䖏一派蟬聲,忽然風送花香,襲人撲鼻。有詩為證:

\begin{myquote}
綠樹陰濃夏日長,樓臺倒映入池塘。

水晶簾動微風起,一架薔薇滿院香。

别院深沉夏簟清,石榴開遍透簾明,

槐陰滿地日卓午,時聽新蟬噪一聲。
\end{myquote}

西門慶坐於椅上以手扇搖凉,只見來安兒畫童兒兩個小廝來井上打水,㧱澆冰安放盆内。西門慶道:「呌一個來。」來安兒忙走向前,西門慶吩咐:「到後邊對你春梅姐説,有梅湯提一壶來,放在這冰盤内湃着。」來安兒應諾去了。半日,只見春梅家常露着頭,戴着銀絲雲髻兒,穿着毛青布褂兒,桃紅夏布裙子,手提一壺蜜煎梅湯,笑嘻嘻走來,問道:「你喫了飯了?」西門慶道:「我在後邊上房裏喫了。」春梅説:「嗔道不進房裏來。把這梅湯放在冰盤内湃着你喫?」西門慶點頭兒。春梅湃上梅湯,走來扶着椅兒,取過西門慶手中芭蕉扇兒替他打扇,問道:「頭裏大娘和你説甚麽話來?」西門慶道:「説吳神仙相面一節。」春梅道:「那道士平白説戴珠冠。教大娘説『有珠冠只怕輪不到他頭上』。常言道:凡人不可貌相,海水不可斗量。従來旋的不圓砍的圓,各人裙帶上衣食,怎麽料得定?莫不長遠只在你家做奴才罷!」西門慶笑道:「小油嘴兒,自胡亂!你若到明日有了娃兒,就替你上了頭。」於是把他摟到懷裏,手扯着手兒頑耍。問他:「你娘在後邊在屋裏?怎的不見?」春梅道:「娘在屋裏,教秋菊熱下水要洗浴。等不的,就在床上睡了。」西門慶道:「等我喫了梅湯,等我摑混他一混去。」於是春梅向冰盆倒了一甌兒梅湯與西門慶,呷了一口,湃骨之涼透心沁齒,如甘露洒心一般。

須臾喫畢,搭伏着春梅肩膀兒,轉過角門,來到金蓮床房中。掀開簾櫳進來,看見婦人睡在正面一張新買的螺鈿床上。原是因李瓶兒房中安着一張螺鈿厰廳床,婦人旋教西門慶使了六十兩銀子,也替他也買了這一張螺鈿有欄杆的床。兩邊槅扇,都是螺鈿攢造,樓臺殿閣,花草翎毛,三塊梳背,安在床内,都是松竹梅歲寒三友。裏面掛着紫紗帳幔,錦帶銀鉤,兩邊香毬吊掛。婦人赤露玉體,止着紅綃抹胸兒,蓋着紅紗衾,枕石鴛鴦枕,在涼蓆之上睡思正濃。房裏異香噴鼻。西門慶一見,不覺淫心頓起,令春梅帶上門出去。悄悄脱了衣褲,上的床來,掀開紗被,見他玉體互相掩映。戲將兩股輕開,按麈柄徐徐插入牝中。比及星眸驚閃之際,已抽拽數十度矣。婦人睜開眼,笑道:「怪強盜,三不知多喒進來?奴睡着了就不知道。奴睡的甜甜兒,鬼混死了我!」西門慶道:「我便罷了。若是有個生漢子進來,你也推不知道罷!」婦人道:「我不好罵的,誰人七個頭八個膽,敢進我這房裏來?只許了你恁沒大沒小的罷了。」

原來婦人因前日西門慶在翡翠軒誇獎李瓶兒身上白淨,就暗暗將茉莉花蕊兒攪酥油定粉,把身上都搽遍了。搽的白膩光滑,異香可掬,使西門慶見了愛他,以奪其寵。西門慶於是見他身體雪白,穿着新做的兩隻大紅睡鞋。一面蹲踞在上,兩手兜其股極力而提之,垂首觀其出入之勢。婦人道:「怪貨,只顧端詳甚麽?奴的身上黑,不似李瓶兒的身上白就是了。他懷着孩子,你便輕憐痛惜;俺們是拾兒,由着這等掇弄!」西門慶問道:「説你等着我洗澡來?」婦人問道:「你怎得知道來?」西門慶把春梅告訴他話説了一遍。婦人道:「你洗,我教春梅掇水來。」不一時,把浴盆掇到房中,注了湯,二人下床來,同浴蘭湯,共效魚水之歡。當下添湯換水,洗浴了一囬。西門慶乘興把婦人仰臥在浴板之上,兩手執其雙足,跨而提之,掀騰𢵞幹,何止二三百囬;其聲如泥中螃蟹一般,響之不絶。婦人恐怕香雲拖墜,一手扶着雲鬢,一手扳着盆沿,口中燕語鶯聲,百般難述。怎見這場交戰,但見:

\begin{myquote}
華池蕩漾波紋亂,翠幃高捲秋雲暗;才郎情動要爭持,稔色心忙顯手段。一個顫顫巍巍挺硬鎗,一個搖搖擺擺輪鋼劍。一個捨死忘生往裏鑽,一個尤雲殢雨將功幹。撲撲鼕鼕皮鼓催,蹕蹕礴礴鎗付劔;𥐙𥐙へへ弄響聲,砰砰ぺぺ成一片。下下高高水逆流,洶洶湧湧盈清澗;滑滑溜溜怎住停,攔攔濟濟難存站。一來一往□□□,一衝一撞東西探。熱氣騰騰妖雲生,紛紛馥馥香氣散。一個逆水撑船將玉股搖,一個艄公把舵將金蓮揝;一個紫騮猖獗逞威風,一個白面妖嬈遭馬戰。喜喜歡歡美女情;雄雄糾糾男兒願;翻翻覆覆意歡娛,鬧鬧挨挨情摸亂。你死我活更無休,千戰千贏心膽戰;口口聲聲呌殺人,氣氣昂昂情不厭。古古今今廣鬧爭,不似這番水裏戰。
\end{myquote}

當下二人水中戰鬧了一囬,西門慶精泄而止。搽抹身體乾淨,撤去浴盆,止着薄纊短襦,上床安放炕桌菓酌飲酒。婦人教秋菊:「取白酒來與你爹喫。」又向床閣板上方盒中㧱菓餡餅與西門慶喫,恐怕他肚中飢餓。只見秋菊半日㧱上一銀注子酒來,婦人纔待斟在鍾上,摸了摸,冰凉的,就照着秋菊臉上只一潑,潑了一頭一臉。罵道:「好賊少死的奴才!我吩咐教你篩了來,如何㧱冷酒與爹喫?你不知安排些甚麽心兒!」呌春梅:「與我把這奴才採到院子裏跪着去!」春梅道:「我替娘後邊捲裹脚去來,一歇兒沒在跟前,你就弄下硶兒了!」那秋菊把嘴谷都着,口裏喃喃呐呐説道:「每日爹娘還喫冰湃的酒兒,誰知今日又改了腔兒。」婦人聽見,罵道:「好賊奴才,你説甚麽?與我採過來!」教春梅每邊臉上打與他十個嘴巴。春梅道:「皮臉沒的打汚濁了我手!娘只教他頂着石頭跪着罷。」於是不由分説,拉到院子内,教他頂着塊大石頭跪着。不在話下。婦人従新教春梅煖了酒來,陪西門慶喫了幾鍾。掇去酒桌,放下紗帳子來,吩咐拽上房門,兩個抱頭交股體倦而寢。正是:若非羣玉山頭覓,多是陽臺夢裏尋。

畢竟未知後來何如,且聽下囬分解。

