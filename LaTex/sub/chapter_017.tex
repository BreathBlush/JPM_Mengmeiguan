\includepdf[pages={33,34},fitpaper=false]{tst.pdf}
\chapter*{第十七囬 \\宇給事劾倒楊提督 李瓶兒招贅蔣竹山}
\addcontentsline{toc}{chapter}{第十七囬 宇給事劾倒楊提督 李瓶兒招贅蔣竹山}
\markboth{{\titlename}卷之二}{第十七囬 宇給事劾倒楊提督 李瓶兒招贅蔣竹山}


\begin{myquote}
記得書齋乍會時,雲踪雨跡少人知。

晚來鸞鳳棲雙枕,剔盡銀燈半吐輝。

思往事夢魂迷,今宵幸得效于飛。
\end{myquote}

話說五月二十日,帥府周守備生日。西門慶那日封五星分資、兩方手帕,打選衣帽齊整,騎着大白馬,四個小廝跟隨,往他家拜壽。席間也有夏提刑、張團練、荆千户、賀千户,一般武官兒飲酒。鼓樂迎接,搬演戲文,又是四個唱的遞酒。玳安接了衣裳,囬馬來家。到日西時分,又騎馬接去。走到西街口上,撞見馮媽媽。問道:「馮媽媽那裏去?」馮媽媽道:「你二娘使我來請你爹來。顧銀匠整理頭面完備,今日拿盒送來,請你爹那裏瞧去。你二娘還和你爹説話哩。」玳安道:「俺爹今日都在守備府周老爹處吃酒,我如今接去。你老人家囬罷,等我到那裏對爹説就是了。」馮媽媽道:「累你好歹説聲,你二娘等着哩。」這玳安打馬逕到守備府,衆官員正飲酒在熱鬧處。玳安走到西門慶席前説道:「小的囬馬家來時,在街口撞遇馮媽媽,二娘使了來説,顧銀匠送了頭面來了,請爹瞧去;還要和爹説話哩。」西門慶聽了,拿了些點心湯飯與玳安吃了,就要起身。那周守備那裏肯放,攔門拿巨盃相勸。西門慶道:「蒙大人見賜,寜可飲一盃。還有些小事,不能盡情,恕罪恕罪!」於是一飲而盡,作辭周守備上馬,逕到李瓶兒家。婦人接着,茶湯畢,西門慶吩咐玳安囬馬家去,明日來接。玳安去了。

李瓶兒呌迎春盒兒内取出頭面來,與西門慶過目。黄烘烘火焰般一付好頭面,收過去,單等二十四日行禮,出月初四日准娶。婦人滿心歡喜,連忙安排酒來,和西門慶暢飲開懷。吃了一囬,使丫鬟房中搽抹凉蓆乾淨,兩個在紗帳之中,香焚蘭麝,衾展鮫綃,脱去衣裳,並肩疊股,飲酒調笑。良久,春色横眉,淫心蕩漾。西門慶先和婦人雲雨一囬,然後乘着酒興坐於床上,令婦人横軃於袵席之上,與他品簫。但見:

\begin{myquote}
紗帳香飄蘭麝,蛾眉輕把簫吹。雪白玉體透簾幃,禁不住魄颺魂飛。一點櫻桃小口,兩隻手賽柔荑。才郎情動囑奴知,不覺靈犀味美。
\end{myquote}

西門慶於是醉中戲問婦人:「當初有你花子虚在時,也和他幹此事不幹?」婦人道:「他逐日睡生夢死,奴那裏耐煩和他幹這營生!他每日只在外邊胡撞,就來家,奴等閒也不和他沾身。況且老公公在時,和他另在一間房睡着,我還把他罵的狗血噴了頭。好不好,對老公公説了,要打躺棍兒也不算人。甚麽材料兒,奴與他這般頑耍,可不砢硶殺奴罷了!誰似寃家這般可奴之意,就是醫奴的薬一般。白日黑夜,敎奴只是想你。」兩個耍一囬又幹了一囬。傍邊迎春伺候下一個小方盒,都是各樣細巧果仁肉心、鷄鵝腰掌、玫瑰菊花餅兒。小金壶兒,滿泛瓊漿。従黄昏掌上燈燭,且幹且飲,直耍到一更時分。只聽外邊一片聲打的大門響,使馮媽媽開門瞧去,原來是玳安來了。西門慶道:「我吩咐明日來接我,這早晚又來做甚麽?」因呌進房來問他。那小廝慌慌張張走到房門首,西門慶與婦人睡着,又不敢進來,只在簾外説話,説道:「姐姐姐夫都搬來了。許多箱籠在家中。大娘使我來請爹,快去計較説話哩。」這西門慶聽了,只顧猶豫:「這早晚端的有甚緣故?湏得到家瞧瞧。」連忙起來。

婦人打發穿上衣服,做了一盞暖酒與他吃,打馬一直來家。只見後堂中秉着燈燭,女兒女婿都來了,堆着許多箱籠牀帳家活,先吃了一驚,因問:「怎的這咱來家?」女婿陳經濟磕了頭,哭道:「近日朝中俺楊老爺被科道官參論倒了。聖旨下來,拿送南牢問罪。門下親族用事人等都問擬枷號充軍。昨日府中楊幹辦連夜奔走,透報與父親知道。父親慌了,敎兒子同大姐和些家活箱籠,就且暫在爹家中寄放,躲避些時。他便起身往東京我姑娘那裏打聽消息去了。待的事寜之日,恩有重報,不敢有忘。」西門慶問:「你爹有書沒有?」陳經濟道:「有書在此。」向袖中取出,遞與西門慶拆開觀看。上面冩道:

\begin{myquote}[\markfont]
\hspace*{4em}「眷生陳洪頓首書奉

大德西門親家見字。餘情不敍。茲因北虜犯邊,搶過雄州地界,兵部王尚書不發人馬,失誤軍機,連累朝中楊老爺,俱被科道官參劾太重。

聖旨惱怒,拿下南牢監禁,會同三法司審問。其門下親族用事人等,俱照例發邊衛充軍。生一聞消息,擧家驚惶,無處可投。先打發小兒令愛,隨身箱籠家活,暫借親家府上寄寓。生即上京,投在家姐夫張世廉處,打聽示下。待事務寜帖之日囬家,恩有重報,不敢有忘。誠恐縣中有甚聲息,生令小兒另具銀五百兩,相煩親家費心處料。容當叩報,沒齒不忘。燈下草草不宣。

\raggedleft{仲夏二十日洪再拜。」}
\end{myquote}

西門慶看了,慌了手脚。敎吳月娘安排酒飯,管待女兒女婿。就令家下人等,打掃廳前東廂房三間,與他兩口兒居住。把箱籠細軟都收拾月娘上房來。陳經濟取出他那五百兩銀子,交與西門慶打點使用。西門慶呌了吴主管來,與了他五兩銀子,敎他連夜往縣中孔目房裏抄錄一張東京行下來的文書邸報。上面端的寫的是甚言語?

\begin{myquote}[\markfont]
「兵科給事中宇文虚中等一本,懇乞宸斷,亟誅誤國權奸,以振本兵,以消虜患事。臣聞夷狄之祸,自古有之。周之玁狁,漢之匈奴,唐之突厥,迨及五代而契丹浸強,又我

皇宋建國,大遼縱横中國者已非一日。然未聞内無夷狄,而外萌夷狄之患者。諺云:霜降而堂鐘鳴,雨下而柱礎潤。以類感類,必然之理。譬猶病夫在此,腹心之疾已久,元氣内消,風邪外入,四肢百骸,無非受病,雖盧扁莫之能救,焉能久乎?今天下之勢,正猶病夫尪羸之極矣。君,猶元首也;輔臣,猶腹心也;百官,猶四肢也。

陛下端拱於九重之上,百官庶政各盡職於下,元氣内充,榮衛外扞,則虜患何由而至哉!今招夷虜之患者,莫如崇政殿大學士蔡京者:本以憸邪奸險之資,濟以寡廉鮮耻之行,讒諂面諛,上不能輔君當道,贊元理化;下不能宣德布政,保愛元元。徒以利祿自資,希寵固位,樹黨懷奸,蒙蔽欺君,中傷善類;忠士為之解體,四海為之寒心。聯翩朱紫,萃聚一門。邇者河湟失議,主議伐遼;内割三郡,郭薬師之叛,燕山失陷;卒致金虜背盟,憑陵中夏。此皆誤國之大者,皆由京之不職也。王黼貪庸無賴,行比俳優。蒙京汲引,薦居政府,未幾謬掌本兵,惟事慕位苟安,終無一籌可展。迺者張達殁於太原,為之張皇失措。今虜之犯内地,則又挈妻子南下,為自全之計。其誤國之罪,可勝誅戮?楊戩本以紈袴膏粱,叨承祖癊,憑籍寵靈,典司兵柄,濫膺閫外。大奸似忠,怯懦無比。此三臣者,皆朋黨固結,内外蒙蔽,為

陛下腹心之蠱者也。數年以來,招災致異,丧本傷元,役重賦煩,生民離散。盜賊猖獗,夷虜犯順。天下之膏腴已盡,國家之紀綱廢弛。雖擢髮不足以數京等之罪也。臣等待罪該科,備員諫職,徒以目擊奸臣誤國而不為

皇上陳之,則上辜君父之恩,下負平生所學。伏乞

宸断,將京等一干黨惡人犯,或下廷尉,以示薄罰;或置極典,以彰顯戮;或照例枷號;或投之荒裔,以禦魑魅。庶天意可囬,人心暢快。國法已正,虜患自消。天下幸甚!臣民幸甚!奉

聖旨。蔡京姑留輔政。王黼楊戩便拿送三法司會問明白來説。欽此欽遵!續該三法司會問過,並黨惡人犯王黼楊戩,本兵不職,縱虜深入,荼毒生民,損兵折將,失陷内地,律應處斬。手下壞事家人、書辦官掾親黨:董升、盧虎、楊盛、龐宣、韓宗仁、陳洪、黄玉、賈廉、劉成、趙弘道等,查出有名人犯,俱問擬枷號,一個月滿日發邊衛充軍。」
\end{myquote}

西門慶不看萬事皆休,看了耳邊廂只聽颼的一聲,魂魄不知往那裏去了。就是:驚損六葉連肝肺,唬壞三毛七孔心。即忙打點金銀寳玩,馱裝停當。把家人來保來旺叫到臥房中,悄悄吩咐:「如此如此,這般這般,僱頭口,星夜上東京打聽消息。不消到爾陳親家老爹下處。但有不好聲息,取巧打點停當,速來囬報。」又與了他二人二十兩盤纏。絶早五更,僱脚夫起程上東京去了,不在話下。

西門慶通一夜不曾睡着。到次日早,吩咐來昭賁四,把花園工程止住,各項匠人都且囬去,不做了。每日將大門緊閉,家下人無事亦不敢往外去,隨分人叫着不許開。西門慶只在房裏動彈,走出來,又走進去,憂上加憂,悶上添悶,如熱地蚰蜒一般,把娶李瓶兒的勾當丢在九霄雲外去了。吴月娘見他每日在房中愁眉不展,面帶憂容,便説道:「他陳親家那邊為事,各人寃有頭債有主,你平白焦愁些甚麽?」西門慶道:「你婦人知道些甚麽!陳親家是我的親家,女兒女婿兩個業障搬來咱家住着,這是一件事。平昔街坊鄰舍,惱咱的極多。常言:機兒不快梭兒快,打着綿羊駒驢戰。倘有小人指戳,拔樹尋根,你我身家不保。」正是:関着門兒家裏坐,祸従天上來!這裏西門慶在家納悶,不題。

且説李瓶兒等了一日兩日,不見動靜,一連使馮媽媽來了兩遍,大門関得鐵桶相似,就是樊噲也撞不開。等了半日,沒一個人牙兒出來,竟不知怎的。看看到廿四日,李瓶兒又使馮媽媽送頭面來,就請西門慶過去説話。叫門不開,立在對過房簷下等。少頃,只見玳安出來飲馬,看見便問:「馮媽媽,你來做甚麽?」馮媽媽説:「你二娘使我送頭面來。怎的不見動靜?請你爹過去説話哩。」玳安道:「俺爹連日有些小事兒,不得閒。你老人家還拿囬頭面去,等我飲馬囬來對俺爹説就是了。」馮媽媽道:「好哥哥,我在這裏等着,你拿進頭面去和你爹説去。你二娘那裏好不惱我哩。」這玳安一面把馬拴下,走到裏邊。半日出來道:「對俺爹説了,頭面爹收下了。敎你上覆二娘,再待幾日兒,我爹出來往二娘那裏説話。」這馮媽媽一直走來囬了婦人話。婦人又等了幾日,看看五月將盡,六月初旬時分,朝思暮盼,音信全無。夢攘魂勞,佳期問阻。正是:

\begin{myquote}
懶把蛾眉掃,羞將粉臉匀。

滿懷幽恨積,憔悴玉精神。
\end{myquote}

婦人盼不見西門慶來,每日茶飯頓減,精神恍惚。到晚夕孤眠枕上,輾轉躊躕。忽聽外邊打門,彷彿見西門慶來到。婦人迎門笑接,携手進房,問其爽約之情,各訴衷腸之話;綢繆繾綣,徹夜歡娱。鷄鳴天曉,頓抽身囬去。婦人恍然驚覺,大叫一聲,精魂已失。慌了馮媽媽,進房來看視。婦人説道:「西門慶他剛纔出去,你關上門不曾?」馮媽媽道:「娘子想得心迷了,那裏得大官人來?影兒也沒有。」婦人自此夢境隨邪,夜夜有狐狸假名抵姓,來攝其精髓。漸漸形容黄瘦,飲食不進,臥牀不起。

馮媽媽向婦人説,請了大街口蔣竹山來看。其人年小,不上三十,生的五短身材,人物飄逸,極是個輕浮狂詐的人。請入臥室,婦人則霧鬢雲鬟,擁衾而臥,似不勝憂愁之狀。勉強茶湯已罷,丫鬟安放褥墊。竹山就牀診視脉息畢,因見婦人生得有姿色,便開言説道:「小人適診病源,娘子肝脉絃出寸口而洪大,厥陰脉出寸口久上魚際,主六慾七情所致,陰陽交爭,乍寒乍熱,似有鬱結于中而不遂之意也。似瘧非瘧,似寒非寒,白日則倦怠嗜臥,精神短少;夜晚神不守舍,夢與鬼交。若不早治,久而變為骨蒸之疾,必有屬纊之憂矣。可惜,可惜!」婦人道:「有累先生俯賜良劑,奴好了重加酬謝。」竹山道:「小人無不用心。娘子若服了我的薬,必然貴體痊安。」説畢起身。這裏使薬金五星,使馮媽媽討將薬來。婦人晚間吃了他的薬下去,夜裏得睡,便不驚恐。漸漸飲食加添,起來梳頭走動。那消數日,精神復舊。

一日,安排了一席酒餚,備下三兩銀子,使馮媽媽請過竹山來相謝。這蔣竹山従與婦人看病之時,懷覬覦之心,已非一日。於是一聞相請,即具服而往。延之中堂,婦人盛粧出見,道了萬福。茶湯兩換,請入房中。酒饌已陳,麝蘭香藹。小丫鬟綉春在傍,描金盤内托出三兩白金。婦人高擎玉盞,向前施禮,説道:「前日奴家心中不好,蒙賜良劑,服之見效。今粗治了一盃水酒,請過先生來知謝知謝。」竹山道:「此是小人分内之事,理當措置,何必計較!」因見三兩謝禮,説道:「這個學生怎麽敢領?」婦人道:「些湏微意,不成禮數,萬望先生笑納。」辭讓了半日,竹山方纔收了。婦人遞酒,安了坐次。飲過三巡,竹山席間偸眼睃視婦人,粉粧玉琢,嬌豔驚人。先用言以挑之,因説道:「小人不敢動問,娘子青春幾何?」婦人道:「奴虚度二十四歲。」竹山道:「又一件,似娘子這等妙年,生長深閨,處於富足,何事不遂?而前日有此鬱結不足之病?」婦人聽了,微笑道:「不瞞先生,奴因拙夫去世,家事蕭條,獨自一身,憂愁思慮,何得無病?」竹山道:「原來娘子夫主殁了,多少時了?」婦人道:「拙夫従去歲十一月得傷寒病死了,今已八個月來。」竹山道:「曾吃誰的薬來?」婦人道:「大街上胡先生。」竹山道:「是那東街上劉太監房子住的胡鬼嘴兒?他又不是我太醫院出身,知道甚麽脈!娘子怎的請他?」婦人道:「也是因街坊上人薦擧請他來看。還是拙夫沒命,不干他事。」竹山又道:「娘子也還有子女沒有?」婦人道:「兒女俱無。」竹山道:「可惜娘子這般青春妙齡之際,獨自孀居,又無所出,何不尋其別進之路?甘為幽鬱,豈不生病。」婦人道:「奴近日也講着親事,早晚過門。」竹山便道:「動問娘子,與何人作親?」婦人道:「是縣前開生薬舖西門大官人。」竹山聽了道:「苦哉,苦哉!娘子因何嫁他?小人常在他家看病,最知詳細。此人專在縣中把攬説事,擧放私債;家中挑販人口。家中不算丫頭,大小五六個老婆;着緊打躺棍兒,稍不中意,就令媒人領出賣了。就是打老婆的班頭,坑婦女的領袖。娘子早是對我説,不然進入他家,如飛蛾投火一般,坑你上不上,下不下,那時悔之晚矣。况近日他親家那邊為事干連他,在家躲避不出。房子蓋的半落不合的都丢下了。東京行下文書,坐落府縣拿人。到明日他蓋這房子,多是入官抄沒的數兒。娘子沒來由嫁他則甚?」一篇話把婦人説的閉口無言。況且許多東西,丢在他家,尋思半晌,暗中跌脚:「怪嗔道一替兩替請着他不來,原來他家中為事哩!」又見竹山語言活動,一團謙恭,「奴明日若嫁得恁樣個人也罷了,不知他有妻室没有?」因問道:「既蒙先生指教,奴家感戴不淺。倘有甚相知人家親事,擧保來説,奴無有個不依之理。」竹山乘機請問:「不知要何等樣人家?小人打聽的實,好來這裏説。」婦人道:「人家倒也不論乎大小,只像先生這般人物的。」這蔣竹山不聽便罷,聽了此言,喜歡的勢不知有無。於是走下席來,雙膝跪在地下,告道:「不瞞娘子説,小人内幃失助,中饋乏人,鰥居已久,子息全無。倘蒙娘子垂憐見愛,肯結秦晋之緣,足稱平生之願。小人雖啣環結草,不敢有忘!」婦人笑以手携之,説道:「且請起。未審先生鰥居幾時?貴庚多少?既要做親,湏得要個保山來説,方成禮數。」竹山又跪下哀告道:「小人行年二十九歲,正月二十七日卯時建生。不幸去年荆妻已故,家緣貧乏,實出寒微。今既蒙金諾之言,何用氷人之講?」婦人聽言笑道:「你既無錢,我這裏有個媽媽,姓馮,拉他做個媒證。也不消你行聘,擇個吉日良辰,招你進來,入門為贅。你意下若何?」這蔣竹山連忙倒身下拜:「娘子就如同小人重生父母,再長爹娘!宿世有緣,三生大幸矣。」一面兩個在房中各遞了一盃交歡盞,已成其親事。

竹山飲至天晚囬家。婦人這裏與馮媽媽商議,説:「西門慶家如此這般為事,吉兇難保。況且奴家這邊没人,不好了一場,險不丧了性命。為今之計,不如把這位先生招他進來,過其日月,有何不可?」到次日,就使馮媽媽通信過去,擇六月十八日大好日期,把蔣竹山倒踏門招進來,成其夫婦。過了三日,婦人凑了三百兩銀子與竹山,打開門面兩間,開店煥然一新。初時往人家看病只是走,後來買了一疋驢兒騎着,在街上往來搖擺,不在話下。正是:一窪死水全無浪,也有春風擺動時。

畢竟未知後來何如,且聽下囬分解。

