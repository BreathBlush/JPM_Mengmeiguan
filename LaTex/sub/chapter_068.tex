\includepdf[pages={135,136},fitpaper=false]{tst.pdf}
\chapter*{第六十八囬 \\鄭月兒賣俏透密意 玳安殷勤尋文嫂}
\addcontentsline{toc}{chapter}{第六十八囬 鄭月兒賣俏透密意 玳安殷勤尋文嫂}
\markboth{{\titlename}卷之七}{第六十八囬 鄭月兒賣俏透密意 玳安殷勤尋文嫂}


\begin{myquote}
雪壓殘紅一夜凋,曉來簾外正飄飄。

數枝翠葉空相對,萬片香魂不可招。

長楽夢回春寂寂,武陵人去水迢迢。

欲將玉笛傳遺恨,苦被東風透綺寮。
\end{myquote}

話説西門慶與李瓶兒燒紙畢,歸潘金蓮房中歇了一夜。到次日,先是應伯爵家送喜麵來;落後黄四領他小舅子孫文相,宰了一口猪,一壜酒,兩隻燒鵝,四隻燒鷄,兩盒菓子,來與西門慶磕頭。西門慶再三不受,黄四打旋磨兒跪着說:「蒙老爹活命之恩,救出孫文相來,擧家感激不淺。今無甚孝順,些微薄禮,與老爹賞人罷了,如何不受?」推阻了半日,西門慶止受猪酒:「留下送你錢老爹,也是一樣。」黄四道:「旣是如此,難為小人一點窮心無處所盡,只得把羹菓擡回去。又請問老爹,幾時閑暇?小人問了應二叔,裏邊請老爹坐坐。」西門慶道:「你休聽他,哄你哩!又費煩你,不如不年下了。」那黄四和他小舅子千恩萬謝出門。這裏西門慶賞擡盒錢,打發去訖。

到十一月初一日,西門慶往衙門中囬來,又往李知縣衙内喫酒去;月娘獨自一人,素粧打扮,坐轎子往喬大戶家與長姐做生日,都不在家。到後晌,有庵裏薛姑子,聽見月娘許下他到初五日李瓶兒断七,敎他請八衆尼僧來家念經,拜血盆懺。於是悄悄瞞着王姑子,買了兩盒禮物來見月娘。月娘不在家,李嬌兒孟玉樓留下他,陪他喫茶説:「大姐姐不在家,往喬親家與長姐做生日去了。你須等他來見他,他還和你說話,好與你寫法銀子。」那薛姑子就坐住了。潘金蓮因想着玉簫告他說,月娘喫了他的符水薬纔坐了胎氣,自従李瓶兒死了,又見西門慶在他屋裏把奶子也要了,恐怕一時奶子養出孩子來,攙奪了他寵愛。於是把薛姑子讓到前邊他房裏,無人處悄悄央薛姑子,與他一兩銀子,替他配坐胎氣符薬喫,尋頭男衣胞,不在話下。到晚夕等的月娘來家,留他住了一夜。次日,問西門慶討了五兩銀子經錢寫法與他。這薛姑子就瞞着王姑子大師父,不和他說。到初五日早,請了八衆女僧,在花園捲棚内建立道場,各門上貼歡門吊子,諷誦《華嚴》、《金剛經》咒,禮拜《血盆寳懺》,灑花米,轉念《三十五佛名經》。晚夕設放焰口,施食。那日請了吴大妗子、花大嫂、官客吴大舅、應伯爵、溫秀才喫齋。尼僧也不打動法事,只是敲木魚、擊手磐念經而已。

那日伯爵領了黄四家人,具帖初七日在院中鄭愛月兒家置酒,請西門慶。西門慶見帖兒笑了,說:「我初七日不得閑,張西材家喫生日酒。倒是明日空閑。」問:「還有誰?」伯爵道:「再沒人,只請了我、李三哥相陪。又費事呌了四個女兒唱《西廂記》。」西門慶吩咐與黄四家人齋喫了,打發囬去。伯爵便問:「黄四那日買了分甚麽禮來謝你?」西門慶如此這般:「我不受他的,再三磕頭禮拜,我只受了猪酒,添了兩疋白鷴紵絲、兩疋京緞、五十兩銀子,謝了龍野錢先生。」伯爵道:「哥,你不接錢儘夠了,這個是你落得的。少說四疋尺頭值三十兩銀子,那二十兩那裏尋這分上去?便益了他,救了他父子二人性命!」當日坐至晚夕方散。西門慶向伯爵說:「你明日還到這邊。」伯爵說:「我知道。」作别去了。八衆尼僧,直亂到一更多時分,方纔道塲圓滿,焚燒箱庫散了。

至次日,西門慶早往衙門中去了。且說王姑子打聽得知,大清早晨走來西門慶家,說薛姑子攬了經去,要經錢。月娘怪他:「你怎的昨日不來?他說你往王皇親家做生日去了。」王姑子道:「這個就是薛家老淫婦的鬼。他對着我說,咱家挪了日子,到初六念經。經錢他都㧱的去了,一些兒不留下?」月娘道:「這咱哩!未曾念經,經錢寫法都找完了與他了。早是我還與你留下一疋襯錢布在此。」教小玉,連忙擺了些昨日剩下的齋食與他喫了。把與他一疋藍布。這王姑子口裏喃喃呐呐罵道:「我敎這老淫婦獨喫!他印造經,賺了六娘許多銀子。原說這個經兒咱兩個使,你又獨自掉攬的去了。」月娘道:「老薛說你接了六娘血盆經五兩銀子,你怎的不替他念?」王姑子道:「他老人家五七時,我在家請了四個師父,念了半個月哩。」月娘道:「你念了,怎的挂口兒不對我題?你就對我説,我還送些襯施兒與你。」那王姑子便一聲兒不言語,訕訕的坐了一囬,往薛姑子家嚷去了。看官聽說:似這樣緇流之輩,最不該招惹他。臉雖是尼姑臉,心同淫婦心。只是他六根未凈,本性欠明,戒行全無,廉耻已喪。假以慈悲為主,一味利慾是貪;不管墮業輪迴,一味眼下快楽。哄了些小門閨怨女,念了些大戸動情妻;前門接施主檀那,後門丢胎卵濕化;姻緣成好事,到此會佳期。有詩為證:

\begin{myquote}
佛會僧尼是一家,法輪常轉度龍華。

此物只好圖生育,枉使金刀剪落花。
\end{myquote}

却説西門慶従衙門中回來,喫了飯,應伯爵又早到了,盔的新緞帽,沉香色𧜽褶,粉底皂靴,向西門慶聲喏說:「這天也有晌午,咱也好去了。他那裏使人邀了好幾遍了,休要難為人家。」西門慶道:「咱今邀葵軒走走。」使王經:「往對過請你溫師父來。」王經去不多時,回説:「溫師父不在家,望朋友去了。畫童兒請去了。」伯爵便說:「咱等不的他。秀才家,知道有要沒緊望朋友多咱來?倒沒的誤了勾當!」西門慶吩咐琴童:「備黄馬與應二爹騎。」伯爵道:「我不騎。你依我,省的搖鈴打鼓。我先走一步兒,你坐轎子慢慢來就是了。」西門慶道:「你說的是,你先行罷。」那伯爵擧手先走了。西門慶吩咐玳安、琴童、四個排軍,收拾下暖轎跟隨。纔待出門,忽平安兒慌慌張張従外㧱着雙帖兒來報說:「工部安老爹來拜。先差了個吏送帖兒,後邊跟着便來也。」慌的西門慶吩咐家中廚下備飯,使來興兒買攢盤點心伺候。

良久,安郎中來到,跟従許多人。西門慶冠冕出來迎接。安郎中穿着粧花雲鷺補子員領,起花萌金帶,進門拜畢,分賓主坐定,左右㧱茶上來。茶罷,敍其間闊之情。西門慶道:「老先生榮擢失賀,心甚缺然。前日蒙賜華札厚儀,生正值喪事匆匆,未及奉候起居為歉。」安郎中道:「學生有失吊問,罪罪。生到京也曾道達雲峯,未知可有禮到否?」西門慶道:「正是,又承翟親家遠勞致賻。」安郎中道:「四泉一定今歲恭喜在即。」西門慶道:「在下才微任小。豈敢過于非望?」又說:「老先生此今榮擢美差,足展雄才大略。河治之功,天下所仰。」安郎中道:「蒙四泉過譽。一介寒儒,叨承科甲,處在下僚。辱蔡老先生抬擧,備員冬曹,謬典水利。奔走湖湘之間,一年以來,王事匆匆,不暇安跡。今又承命修理河道,况此民窮財盡之時。前者皇船載運花石,毀閘折壩,所過倒懸,公私困弊之極;而今瓜州、南旺、沽頭、魚臺、徐沛、呂梁、安陵、濟寜、宿遷、臨清、新河一帶,皆毀壞廢圯;南河南徙,淤沙無水,八府之民皆疲弊之甚;又兼賊盜梗阻,財用匱乏,大覃神輸鬼役之才,亦無如之何矣!」西門慶道:「老先生自有才猷展布,不日就緒,必大陞擢矣。」因問:「老先生勅書上有期限否?」安郎中道:「三年欽限,河工完畢,聖上還要差官來祭謝河神。」說話之間,西門慶令放桌兒。安郎中道:「學生實告,還要往黄泰宇那裏拜拜去。」西門慶道:「旣如此,少坐片時,教跟従者喫些點心。」不一時,放了桌,就是春盛案酒,一色十六碗,都是炖爛下飯:鷄蹄、鵝鴨、鮮魚、羊頭、肚肺、血臟、鮓湯之類;純白上新軟稻粳飯,用銀鑲甌兒盛着,裏面沙糖、榛、松、瓜仁拌着飯。又小金鍾暖斟羙釀。下人俱有攢盤點心酒肉。安郎中席間,只喫了三鍾,就告辭起身説:「學生容日再來請教。」西門慶款留不住,送至大門首,上轎而去。囬到聽上,解去了冠帶,換了巾幘,止穿紫絨獅補直身。使人問:「溫師父來了不曾?」玳安囬説:「溫師父未回家哩。有鄭春和黄四叔家來定兒來邀,在這裏半日了。」

西門慶即出門上轎,左右跟隨,逕往院中鄭愛月兒家來。比及進院門,架兒行頭都躲過一邊,只該日俳長兩邊站立,不敢跪接。鄭春與來定兒先通報去了。應伯爵正和李三打雙陸,聽見西門慶來,連忙收拾不及。鄭愛月兒、愛香兒,戴着海獺臥兔兒,一窝絲杭州攢,翠重梅鈿兒,油頭粉面,打扮的花僊也似的,都出來門首迎接。西門慶下了轎,進入客位内。西門慶吩咐不消吹打,止住鼓楽。先是李三黄四見畢禮數,然後鄭家鴇子出來拜見了,纔是愛月兒姊妹兩個插燭也似磕了頭。正面安設兩張交椅,西門慶與應伯爵坐下。李智、黄四,與鄭家姊妹兩個打横。玳安在傍稟問:「轎子在這裏?囬了家去?」西門慶令排軍和轎子都回去。吩咐琴童:「到家,看你溫師父家裏來了,㧱黄馬接了來。」琴童應喏去了。伯爵因問:「哥怎的這半日纔來?」西門慶悉把工部安郎中來拜留飯之事,説了一遍。須臾,鄭春㧱茶上來。愛香兒㧱了一盞遞與伯爵。愛月兒便遞西門慶,那伯爵連忙用手去接,說:「我錯接,只説你遞與我來。」愛月兒道:「我遞與你?沒修這樣福來!」伯爵道:「你看這小淫婦兒,原來只認的他家漢子,倒把客人不着在意裏。」愛月兒笑道:「今日輪不着你做客人,還有客人來。」喫畢茶,收下盞托去。須臾,四個唱《西廂》妓女,都花枝招颭、綉帶飄飄出來,與西門慶磕頭,一一都問了名姓。西門慶對黄四說:「等住回上來唱,只打鼓兒,不吹打罷。」黄四道:「小人知道。」只見鴇子上來説:「只怕老爹害冷!」敎鄭春放下暖簾來,火盆獸炭頻加,蘭麝香靄。只見幾個青衣圓社,聽見西門慶老爹進來在鄭家喫酒,走來門首伺候,探頭舒腦,不敢進去。有認的玳安兒,向玳安打恭,央及作成作成。玳安悄悄進來替他禀問,被西門慶喝了一聲,唬的衆人一溜煙走了。

不一時,收拾菓品案酒上來,正面放兩張桌席,西門慶獨自一席,伯爵與溫秀才一席,留空着溫秀才坐位在左首。傍邊一席李三和黄四,右邊是他姊妹二人。端的盤堆異品,花插金瓶。鄭奉鄭春在傍彈唱。纔遞酒安席坐下,只見溫秀才到了。頭戴過橋巾,身穿綠雲襖,脚穿雲履絨襪,進門作揖。伯爵道:「老先生何來遲也?留席久矣。」溫秀才道:「學生有罪,不知老先生呼喚。適往敝同窗處會書,來遲了一步。」慌的黄四一面安放鍾筯,與伯爵一處坐下。不一時,湯飯上來,黄芽韮燒賣,八寳攢湯,薑醋碟兒。兩個小優兒彈唱一回下去。端的酒斟綠蟻,詞歌金縷。四個妓女纔上來唱了一摺「遊藝中原」。只見玳安來説:「後邊銀姨那裏,使了吴惠和蠟梅送茶來了。」原來吴銀兒就在鄭家後邊住,止隔一條巷。聽見西門慶在這裏喫酒,故使送茶。西門慶喚入裏面,吴惠蠟梅先磕了頭說:「銀姐使我送茶來與爹喫。」揭開盒兒,斟茶上去,每人一盞瓜仁栗絲鹽笋芝麻玫瑰香茶。西門慶問:「銀兒在家做甚麽哩?」蠟梅道:「姐兒今日在家沒出門。」西門慶喫了茶,賞了他兩個三錢銀子。即令玳安同吳惠:「你快請銀姨去。」鄭愛月兒急俐便就敎鄭春:「你也跟了去,好歹纏了銀姨來。他若不來,你就說我到明日就不和他做夥計了。」應伯爵道:「我倒好笑,你兩個原來是販𣭈的夥計!」溫秀才道:「南老好不近人情。自古同聲相應,同氣相求;本乎天者親上,本乎地者親下。同他做夥計,一般了。」愛月兒道:「應花子,你與鄭春他們都是夥計,當差供唱,都在一處。」伯爵道:「儍孩子,我是老王八,那咱和你媽相交,你還在肚子裏!」

說笑中間,廚下割獻豕蹄一領,又是四碗下飯,羊蹄黄芽、臊子韮、肚肺羹、血臟之類。妓女上來唱了一套「半萬賊兵」。西門慶呌上唱鶯鶯的韓家女兒,近前問:「你是韓家的?」愛香兒說:「爹,你不認的,他是韓金釧侄女兒,小名消愁兒,今年纔十三歲。」西門慶道:「這孩子到明日成個好婦人兒!擧止伶俐,又唱的好。」因令他上席遞酒。黄四下湯下飯,極盡殷勤。

不一時,吴銀兒來到。頭上戴着白縐紗䯼髻、珠子箍兒、翠雲鈿兒,周圍撇一溜小簪兒,耳邊戴着金丁香兒;上穿白綾對衿襖兒,粧花眉子;下着紗綠潞紬裙,羊皮金滚邊;脚上墨青素緞雲頭鞋兒。笑嘻嘻進門,向西門慶磕了頭,後與溫秀才等各位都道了萬福。伯爵道:「我倒好笑了,來到就教我惹氣:俺們是後娘養的,只認的你爹?與他磕頭,望着俺們擩一拜。原來你這麗春院小娘兒這等欺客!我若有五棍兒衙門,定不饒你!」愛月兒呌:「應花子,好沒羞的孩兒!那裏哥兒你行頭不怎麽的,光一味好撇。」一面安座兒,讓銀姐坐。就在西門慶桌邊坐下,連忙放鍾筯。西門慶見他戴着白䯼髻,問:「你戴的誰人孝?」吴銀兒道:「爹故意又問,今兒與娘戴孝一向了。」西門慶一聞與李瓶兒戴孝,不覺滿心歡喜,與他側席而坐,兩個說話。須臾,湯飯上來,愛月兒下來與他遞酒。吴銀兒下席,說:「我還沒見鄭媽哩。」一面走到鴇子房内,見了禮出來。鴇子呌:「月姐讓銀姐坐,只怕冷,教丫頭燒個火籠兒與銀姐烤手兒。」隨即添換熱菜,打發上來。吴銀兒在傍,只喫了半個點心,呵了兩口湯,放下筯兒,和西門慶攀話。因㧱起鍾兒來說:「爹,這酒寒些。」従新折了,另換上暖酒。鄭春上來,把伯爵衆人等酒都斟上,行過一巡。吳銀兒便問:「娘前日断七念經來?」西門慶道:「五七多謝你們茶。」吴銀姐道:「好説,俺們送了些粗茶,倒敎爹又把人情囘了,又多謝重禮,教媽惶恐了不的。昨日娘断七,我會下月姐和桂姐,也要送茶來,又不知宅内念經不念。」西門慶道:「断七那日,胡亂請了幾衆女僧,在家拜了拜懺,親眷一個都沒請,恐怕費煩。」飲酒說話之間,吴銀兒又問:「家中大娘、衆娘們都好?」西門慶道:「都好。」吴銀兒道:「爹乍沒了娘,到房裏孤孤兒的,心中也想?」西門慶道:「想是不消說。前日在書房中,白日夢見他,哭的我了不的。」吴銀兒道:「熱突突沒了,可知想哩。」伯爵道:「你們說的只情說,把俺們這裏只顧旱着。不説來遞鍾酒,也唱個兒與俺聽。俺們起身去罷。」慌的李三黄四連忙攛掇他姐兒兩個上來遞酒。安下楽器,吴銀兒也上來,三個粉頭一般兒坐在席傍,躧着火盆,合着聲音,啟朱唇,露皓齒,詞出佳人口,唱了套〔中呂·粉蝶兒〕:「三弄梅花」,端的有裂石流雲之響。

唱畢,西門慶向伯爵説:「你落索他姐兒三個唱,你也下來酬他一盃兒。」伯爵道:「不打緊,死不了人。等我打發他,仰𢵞着,直舒着,側臥着,金鷄獨立,隨我受用。又一件,野馬䠕場,野狐抽絲,猿猴獻菓,黄狗溺尿,僊人指路,靠背將軍柱,面對木伴哥,隨他揀着耍。」愛香道:「我不好駡出來的,汗邪了你這賊花子,胡説亂道的!」這應伯爵用酒碟安三個鍾兒,説:「我兒,你們在我手裏喫兩鍾;不喫,望身上只一潑。」愛香道:「我今日忌酒。」愛月兒道:「你跪着月姨兒,教我打個嘴巴兒,我纔喫。」伯爵道:「銀姐,你怎的説?」吴銀兒道:「二爹,我今日心内不自在,喫半盞兒罷。」那愛月兒道:「花子,你不跪,我一百年也不喫。」黄四道:「二爺,你不跪,顯的不是趣人。也罷,跪着不打罷。」愛月兒道:「不,他只教我打兩個嘴巴兒,我方喫這鍾酒兒。」伯爵道:「溫老先兒在這裏看着,怪小淫婦兒,只顧趕盡殺絶!」於是奈何不過,真個直撅兒跪在地下。那愛月兒輕揎彩袖,款露春纖,駡道:「賊花子,再敢無禮傷犯月姨兒不敢?高聲兒答應,你不答應,我也不喫。」那伯爵無法可處,只得應聲道:「再不敢傷犯月姨了。」這愛月兒一連打了兩個嘴巴,方纔喫那盃酒。伯爵起來道:「好個沒仁義的小淫婦兒,你也剩一口兒我喫。把一鍾酒都喫的淨淨兒的!」愛月兒道:「你跪下,等我賞你一鍾酒。」於是滿滿斟上一盃,笑望伯爵口裏只一灌。伯爵道:「怪小淫婦兒,使促挾灌撒了我一身酒。我老道只這件衣服,新穿了纔頭一日兒,就汚濁了我的。我問你家漢子要!」亂了一囬,各歸席上坐定。

看看天晚,掌燭上來。下飯添換,都已上完。下邊玳安、琴童、畫童、應寳,都在鴇子房裏放桌兒,有湯飯點心酒餚管待。須臾,㧱上各樣菓碟兒來。那伯爵推讓溫秀才,只顧不住手拈放在口裏,一壁又往袖中褪。西門慶吩咐取個骰盆兒來,先讓溫秀才。秀才道:「豈有此理?還従老先兒那邊來。」於是西門慶與吴銀兒用十二個骰兒搶紅。下邊四個妓女,㧱楽器彈唱呌呵酒。飲過一巡,吴銀兒却轉過來與溫秀才伯爵搶紅,愛香兒却來西門慶席上遞酒猜枚。須臾過去,愛月兒近前與西門慶搶紅,吴銀兒却往下席遞李三黄四酒。原來愛月兒旋往房中新粧打扮出來,上着煙裏火迴紋錦對衿襖兒,鵝黄杭絹點翠縷金裙,粧花膝褲,大紅鳳嘴鞋兒。燈下海獺臥兔兒,越顯的粉濃濃雪白的臉兒,猶賽美人兒一般。但見:

\begin{myquote}
芳姿麗質更妖嬈,秋水精神瑞雪標。

鳳目半彎藏琥珀,朱唇一顆點櫻桃。

露來玉笋纖纖細,行步金蓮步步嬌。

白玉生香花解語,千金良夜實難消。
\end{myquote}

這西門慶一見,如何不愛?喫了幾鍾酒,半酣上來,因想着李瓶兒夢中之言:「少貪在外夜飲。」一面起身,後邊淨手。慌的鴇子連忙呌丫鬟點燈,引到後邊。解手出來,愛月隨即也跟來伺候,盆中淨手畢,拉着他手兒同到房中。房中又早月窗半啟,銀燭高燒,氣暖如春,蘭麝馥郁。牀畔則斗帳雲横,鮫綃霧設。於是脱了上蓋,底下白綾道袍,兩個在牀上,腿壓腿兒做一處。先是愛月兒問:「爹今日不家去罷了。」西門慶道:「我還去。今日一者銀兒在這裏,不好意思;二者我居着官,今年考察在邇,恐惹是非,只是白日來和你坐坐罷了。」又説:「前日多謝你蚫螺兒。你送了去,倒惹的我心酸了半日。當初有過世六娘他會揀,他死了,家中再有誰會揀他!」愛月道:「揀他不難,只是要㧱的着筋節兒便好。那日我胡亂整治了不多兒,知道爹好喫,教鄭春送來。那瓜仁都是我口裏一個個兒嗑的,汗巾兒是我閒着用工夫撮的穗子。瓜仁只説應花子倒撾了好些喫了。」西門慶道:「你問那訕臉花子頭,我見時他早兩把撾去,喃了好些,只剩下沒多,我喫了。」愛月兒道:「倒便益了賊花子,恰好只孝順了他。」又説:「多謝爹的衣梅。媽看見,喫了一個兒,喜歡的了不的。他要便痰火發了,晚夕咳嗽,半夜把人聒死了。常時口乾,得恁一個在口内噙着,他倒生好些津液。我和俺姐姐喫了沒多幾個兒,連罐兒他老人家都收了在房内,早晚喫,誰敢動他。」西門慶道:「不打緊,我明日使小廝再送一罐來你喫。」愛月又問:「爹連日會桂姐來沒有?」西門慶道:「自従孝堂裏到如今,誰見他來?」愛月兒道:「六娘五七,他也送茶去來?」西門慶道:「他家使李銘送去來。」愛月道:「我有句話兒,只放在爹心裏。」西門慶問:「甚麽話?」那愛月又想了想,説:「我不説罷。若説了,顯得姊妹們恰似我背地説他一般,不好意思的。」西門慶一面摟着他脖子説:「怪小油嘴兒,甚麽話?說與我,不顯出你來就是了。」 兩個正説得入港,猛然應伯爵走入來,大呌一聲:「你兩個好人兒,撇了俺們,走在這裏説梯己話兒!」愛月兒噦道:「好個不得人意怪訕臉花子!猛可走來,唬了人恁一跳。」西門慶駡道:「怪狗才,前邊去罷,丢的葵軒和銀姐在那裏,都往後頭來了。」這伯爵一屁股坐在牀上説:「你㧱胳膊來,我且咬口兒我纔去。你兩個在這裏儘着㒲搗。」於是不由分說,向愛月兒袖口邊勒出那賽鵝脂雪白的手腕兒來,帶着銀鐲子,猶若羙玉,尖溜溜十指春葱,手上籠着金戒指兒,誇道:「我兒,你這兩隻手兒,天生下就是發ぎぐ的肥一般。」愛月兒道:「怪刀攮的,我不好罵出來的!」被伯爵拉過來,咬了一口,走了。咬的老婆怪呌,罵:「怪花子,平白進來鬼混人死了!」便呌:「桃花兒,你看他出去了,把籠道子門関了!」

一面関上門,愛月便把李桂姐如今又和王三官兒子女一節説與西門慶:「怎的有孫寡嘴、祝麻子、小張閑、架兒于寬、聶鉞兒、踢行頭白囬子、向三,日逐標着在他家行走。如今丢開齊香兒,又和秦家玉芝兒打熱。兩下裏使錢,使沒了,包了皮襖,當了三十兩銀子;㧱着他娘子兒一副金鐲子,放在李桂姐家算了一個月歇錢。」西門慶聽了,口中罵道:「恁小淫婦兒,我吩咐休和這小廝纏,他不聽,還對着我賭身發咒,恰恰只哄我!」愛月兒道:「爹也别要惱。我説與爹個門路兒,管情敎王三官打了嘴,替爹出氣。」西門慶把他摟在懷裏,用白綾袖子兜着他粉項,揾着他香腮,他便一手㧱着銅絲火籠兒,内燒着沉速香餅兒,將袖口籠着燻爇身上,便道:「我説與爹,休敎一人知道。就是應花子也休望他提,只怕走了風。」西門慶問:「我的兒,你告我説,我儍了,肯敎人知道。端的甚門路兒?」鄭愛月悉言:「王三官娘林太太,今年不上四十歲,生的好不喬樣,描眉畫眼,打扮狐狸也似。他兒子鎮日在院裏,他專在家,只送外賣,假托在個姑姑庵兒打齋,但去就在說媒的文嫂兒家落脚。文嫂兒單管與他做牽兒。只説好風月。我説與爹,到明日遇他遇兒也不難。又一個巧宗兒:王三官兒娘子兒,今纔十九歲,是東京六黄太尉姪女兒,上畫般標緻,雙陸棋子都會。三官常不在家,他如同守寡一般,好不氣生氣死,為他也上了兩三遭吊,救下來了。爹難得先刮剌上了他娘,不愁媳婦兒不是你的。」當下被他一席話,説的西門慶心邪意亂,摟着粉頭説:「我的親親,我又問你,怎的曉的就裏?」這愛月兒就不説常在他家唱,只説:「我一個熟人兒,如此這般和他娘在其處會過一遍,也是文嫂兒說合。」西門慶問:「那人是誰?莫不是大街坊張大户姪兒張二官兒?」愛月兒道:「那張懋德兒好㒲的貨!麻着七八個臉彈子,密縫兩個眼,可不砢硶殺我罷了!只好樊家百家奴兒接他,一向董金兒也與他丁八了。」西門慶道:「我猜不着,端的是誰?」愛月兒道:「敎爹得知了罷,是原梳籠我的那個南人。他一年來此做買賣兩遭。正經他在裏邊歇不的一兩夜,倒只在外邊常和人家偸猫遞狗,幹此勾當。」這西門慶聽了,見粉頭所事合着他的板眼,一發歡喜,說:「我兒,你既貼戀我心,每月我送三十兩銀子與你媽盤纏,也不消接人了,我遇閒就來。」愛月兒道:「爹,你有我心時,甚麽三十兩二十兩,月間掠幾兩銀子與媽,我自恁懶待留人,只是伺候爹罷了。」西門慶道:「甚麽話!我決然送三十兩銀子來。」

説畢,兩個上牀交歡。牀上鋪的被褥約一尺高,愛月道:「爹脱衣裳不脱?」西門慶道:「咱連衣耍耍罷,只怕他們前邊等咱。」一面扯過下枕來,粉頭解去下衣,仰臥枕畔,裏面穿着紅潞紬底衣,褪下一隻膝褲腿來。這西門慶把他兩隻小小金蓮扛在肩頭上,解開藍綾褲子,那話使上托子。但見花心輕拆,柳腰款擺,正是:

\begin{myquote}
花嫩不禁揉,春風卒未休。花心猶未足,脉脉情無極。低低喚粉郎,春宵楽未央。
\end{myquote}

那當下兩個至精欲洩之際,西門慶幹的氣喘吁吁,粉頭嬌聲不絶,鬢雲拖枕,滿口只呌道:「親達達,慢着些兒。」良久,楽極情濃,一泄如注。雲收雨散,各整衣裙,於燈下照鏡理容。西門慶在牀前盆中淨手,着上衣服,兩個携手來到席上。吴銀兒便守着伯爵,愛香兒挨近葵軒,正擲色猜枚,觥籌交錯,耍在熱鬧處。

衆人見西門慶進入,都立起身來讓坐。伯爵道:「你也下般的,把俺們丢在這。你纔出來,㧱酒兒且扶扶頭着。」西門慶道:「俺們說句話兒,有甚這閑勾當?」伯爵道:「好話,你兩個原來說梯己話兒!」當下伯爵㧱大鍾斟上暖酒,衆人陪西門慶喫,四個妓女㧱楽器彈唱。玳安在傍掩口説道:「轎子來了。」西門慶𢫓了個嘴兒與他,那玳安連忙吩咐排軍打起燈籠,外邊伺候。這西門慶也不坐,陪衆人執盃立飲。吩咐四個妓女:「你再唱個『一見嬌羞』我聽。」那韓消愁兒説:「俺們會唱。」於是㧱起琵琶來,款放嬌聲,㧱腔唱道:

\begin{myquote}
「一見嬌羞,雨意雲情兩意投。我見他千嬌百媚,萬種妖嬈,一捻溫柔。通書先把話兒勾,傳情暗裏秋波溜。記在心頭,心頭,未審向時成就。」
\end{myquote}

唱了一個詞兒,吴銀兒遞西門慶酒,鄭香兒便遞伯爵,愛月兒奉溫秀才。李智黄四都斟上。又唱道:

\begin{myquote}
「問爾丫鬟,欲鑄黃金拜將壇。莫通明曉寄與書生,雲雨巫山。重門今夜未曾拴,深閨特把情郎盼。夜靜更闌,更闌,偸花妙手今番難按。」
\end{myquote}

喫畢,西門慶令再斟上,鄭香兒上來遞西門慶,吴銀兒遞溫秀才,愛月兒遞伯爵。鄭春在傍捧着菓菜兒。又唱道:

\begin{myquote}
「夢入高唐,相會風流窈窕娘。我與他同携素手,共入羅幃,永結鸞鳳。靈犀一點透膏肓,鮫綃帳底翻紅浪。粉汗凝香,凝香,今宵一刻人間天上。」
\end{myquote}

唱畢,又呌呵酒。愛月兒却轉過捧西門慶酒,吴銀兒遞伯爵,愛月兒遞溫秀才,並李三、黃四,従新斟酒。又唱第四個:

\begin{myquote}
「春暖芙蓉,鬢亂釵横寳髻松。我為他香嬌玉軟,燕侣鶯儔,意羙情濃。腰肢無力眼矇朧,深情自把眉兒縱。兩意相同,相同,百年恩愛和偕鸞鳳。」
\end{myquote}

唱畢,都飲過,西門慶起身。一面令玳安向書袋内取出大小十一包賞賜來:四個妓女,每人三錢;呌上廚役,賞了五錢;吴惠、鄭奉、鄭春,每人三錢;攛掇打茶的,每人二錢;丫頭桃花兒,也與了他三錢。俱磕頭謝了。黄四再三不肯放,道:「應二叔,你老人家說聲,天還早哩。老爹大坐坐,也盡小人之情,如何就要起身?我的月姨兒,你也留留兒!」愛月兒道:「我留他,他白不肯坐。」西門慶道:「你們不知,我明日還有事。」一面向黄四李三作揖,道:「生受,打攪。」黄四道:「惶恐!没的請老爹來受餓。又不肯久坐,還是小人沒敬心。」説着,三個唱的都磕頭,説道:「爹到家,多頂上大娘和衆娘們,俺們閒了,會了銀姐,往宅内看看大娘去。」西門慶道:「你們閒了去坐上一日來。」一面掌起燈籠,西門慶下臺磯,鄭家鴇子迎着道萬福,説道:「老爹,大坐囬兒,慌的就起身,嫌俺家東西不羙口?還有一道米飯兒未曾上哩。」西門慶道:「夠了。我不是還坐囬兒,許多事在身上。明日還要起早,衙門中有勾當。教應二哥,他沒事,教他大坐回兒罷。」那伯爵就要跟着起來,被黄四死力攔住,説道:「我的二爺,你若去了,就沒趣死了。」伯爵道:「不是,你休攔我。你把溫老先生有本事留下,我就算你好漢!」那溫秀才奪門就走,被黄家小廝來定兒攔腰抱住。西門慶到了大門首,因問琴童兒:「溫師父有頭口在這裏沒有?」琴童道:「備下驢子在此,畫童兒看着哩。」西門慶向溫秀才道:「旣有頭口,也罷,老先兒你陪應二哥再坐坐,我先去罷。」於是都送出門來。

那鄭月兒拉着西門慶手兒,悄悄捏了一把,臨上轎,一徑揚聲説道:「我頭裏説的話,爹你在心裏,法不傳六耳!」西門慶道:「知道了。」愛月又道:「鄭春,你送老爹到家,多上覆娘們。」那吴銀兒也説:「多上覆大娘。」伯爵道:「我不好説的,賊小淫婦兒們,都攙行奪市的捎上覆;偏我就沒個人兒上覆!」愛月道:「你這花子過一邊兒!」那吴銀兒就在門首作辭了衆人並鄭家姐兒兩個,吴惠打着燈回家去了。鄭月兒便呌:「銀姐,見了那個流人兒,好歹休要説。」吴銀兒道:「我知道。」衆人囬至席上,重添獸炭,再泛流霞。歌舞吹彈,歡娱楽飲,直耍了三更方散。黄四擺了這席酒,也與了他十兩銀子。西門慶賞賜了三四兩,俱不在話下。當日西門慶坐轎子,兩個排軍打着燈,逕出院門,打發鄭春回家。一宿晚景題過。

到次日,夏提刑差答應的來,請西門慶早往衙門中審問賊情等事,直問到晌午。來家喫了飯,早時沈姨夫差大官沈定㧱帖兒送了個後生來,在緞子舖煮飯做火頭,名喚劉包。西門慶留下了,正在書房中㧱帖兒與沈定回家去了。只見玳安在傍邊站立,西門慶便問道:「溫師父昨日多咱來了?」玳安道:「小的舖子裏睡了好一回,只聽見畫童兒打對過門,那咱有三更時分纔來了。我今早晨問,溫師父倒沒酒,應二爹醉了,吐了一地。月姨恐怕夜深了,使鄭春送了他家去了。」西門慶聽了,呵呵笑了,因呌過玳安近前,説道:「舊時與你姐夫說媒的文嫂兒在那裏住?你尋了他來,對門房子裏見我,我和他説話。」玳安道:「小的不認的文嫂兒家,等我問了姐夫去。」西門慶道:「你喫了飯,問了他,快去。」玳安到後邊喫了飯,走到舖子裏問陳經濟。經濟道:「尋他做甚麽?」玳安道:「誰知他做甚麽?猛可敎我找尋他去。」經濟道:「出了東大街,一直往南去,過了同仁橋牌坊,轉過往東,打王家巷進去,半中腰裏有個發放巡捕的廳兒,對門有個石橋兒,轉過石橋兒,緊靠着個姑姑庵兒,傍邊有個小衚衕兒;進小衚衕往西走,第三家豆腐舖隔壁上坡兒,有雙扇紅封門兒的,就是他家。你只叫文媽,他就出來答應你。」這玳安聽了,説道:「再沒了?小爐匠跟着行香的走——瑣碎一浪湯。你再説一遍我聽,只怕我忘了。」那陳經濟又說了一遍。玳安道:「好近路兒,等我騎了馬去。」一面牽出大白馬來,搭上替子,兜上嚼環,躧着馬臺,望上一騙,打了一鞭,那馬跑踍跳躍一直去了。出了東大街,逕往南,過同仁橋牌坊,由王家巷進去,果然中間有個巡捕廳兒,對門就是座破石橋兒,裏首半截紅牆,是大悲庵兒,往西是小衚衕,北上坡挑着個豆腐牌兒,門首只見一個媽媽晒馬糞。玳安在馬上便問:「老媽媽,這裏有個說媒的文嫂兒?」那媽媽道:「這隔壁封門兒就是。」玳安到他門首,果然是兩扇紅封門兒,連忙跳下馬來,㧱鞭兒敲着門兒呌道:「文媽在家不在?」只見他兒子文ら兒開了門,便問道:「是那裏來的?」玳安道:「我是縣門前提刑西門老爹來請,教文媽快去哩。」文ら聽見是提刑西門大官府家來的,便讓家裏坐。

那玳安把馬拴住,進入裏面他明間内,見上面供養着利市紙,有幾個人在那裏會茶,祈祀罷進香算帳哩。半日,㧱了鍾茶出來,説道:「俺媽不在了。來家説了,明日早去罷。」玳安道:「驢子現在家裏,如何推不在?」側身逕往後走。不料文嫂和他媳婦兒,陪着幾個道媽媽子正喫茶,躲不及,被他看見了。説道:「這個不是文媽?剛纔就囬我不在家了,教我怎的囬俺爹話?惹的不怪我!」文嫂笑哈哈與玳安道了個萬福,説道:「累哥哥,你到家囬聲兒,我今日家裏會茶。不知老爹呼喚我做什麽?我明日早往宅内去罷。」玳安道:「只吩咐我來尋你,誰知他做甚麽?原來不知你在這咭溜搭剌兒裏住,教我找尋了個不發心。」文嫂兒道:「他老人家這幾年宅内買使女、説媒、用花兒,自有老馮和薛嫂兒王媽媽子走跳,希罕俺們?今日忽剌八又冷鍋中荳兒爆,我猜見你六娘没了,一定敎我去替他打聽親事,要補你六娘的窝兒。」玳安道:「我不知道。你到那裏見了俺爹,他自有話和你説。」文嫂兒道:「哥哥,你略坐坐兒,等我打發會茶人去了,同你去。」玳安道:「原來等你會茶?馬在外邊沒人看,俺爹在家緊等的火裏火發,吩咐了又吩咐,教你快去哩。和你說了話,如今還要往府裏羅同知老爹家喫酒去哩。」文嫂道:「也罷,等我㧱點心你喫了,同你去。」玳安道:「不喫罷。」文嫂因問:「你大姐生了孩兒沒有?」玳安道:「還不曾見哩。」這文嫂一面打發玳安喫了點心,穿上衣裳,説道:「你騎馬先行一步兒,我慢慢走。」玳安道:「你老人家放着驢子,怎不備上騎?」文嫂兒道:「我那討個驢子來?那驢子是隔壁豆腐舖裏驢子,借俺院兒裏喂喂兒,你就當我的驢子?」玳安道:「我記得你老人家騎着匹驢兒來,往那去了?」文嫂兒道:「這咱哩,那一年吊死人家丫頭,打官司,為了場事,把舊房兒也賣了,且説驢子哩。」玳安道:「房子倒不打緊處,且留着那驢子和你早晚做伴兒也罷了。别的罷了,我見他常時落下來好個大鞭子。」那文嫂哈哈笑道:「怪猴兒,短壽命!老娘還只當好話兒,側着耳朵聽你什麽好物件兒。幾年不見,你也學的恁油嘴滑舌的,到明日還教我尋親事哩。」玳安道:「我的馬走得快,你步行,知道挨磨到多早晚?惹的爹説。你上馬,咱兩個疊騎着罷!」文嫂兒道:「怪小短命兒,我又不是你影射的。街上人看着,怪剌剌的。」玳安道:「再不,你備豆腐舖子裏驢子騎了去。到那裏等我打發他錢就是了。」文嫂兒道:「這等還許説。」一面敎文ら將驢子備了,帶上眼紗,騎上。玳安與他同行,逕往西門慶宅中來。正是:欲向深閨求艷質,全憑紅葉是良媒。有詩為證:

\begin{myquote}
誰信桃源有路通,桃花含露笑春風。

桃源只在山溪裏,今許漁郎去問津。
\end{myquote}

畢竟未知後來如何,且聽下回分解。

