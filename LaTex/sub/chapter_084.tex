\includepdf[pages={167,168},fitpaper=false]{tst.pdf}
\chapter*{第八十四囬 \\吳月娘大鬧碧霞宫 宋公明義釋清風寨}
\addcontentsline{toc}{chapter}{第八十四囬 吳月娘大鬧碧霞宫 宋公明義釋清風寨}
\markboth{{\titlename}卷之九}{第八十四囬 吳月娘大鬧碧霞宫 宋公明義釋清風寨}


\begin{myquote}
冬夏長青不世情,乾坤妙化屬生成。

清標不染塵埃氣,貞操惟持泉石盟。

凡節通靈無並品,孤霜釀味有餘馨。

世人欲問長生術,到底芳姿益壽齡。
\end{myquote}

話說一日,吳月娘請將吳大舅來商議,要徃泰安州頂上與娘娘進香,——西門慶病重之時,許的願心,吳大舅道:「旣要去,须是我同了你去。」即時吳大舅保定,備辦香燭紙馬祭品之物,玳安來安兒跟隨,僱了頭口騎,月娘便坐一乘暖轎子。吩咐孟玉樓、潘金蓮、孫雪娥、西門大姐:「好生看家,同奶子如意兒衆丫頭,好生看孝哥兒。後邊儀門,無事早早關了,休要出去。」外邊又吩咐陳經濟:「休要那去,同傅夥計大門首看顧。我約莫到月盡就來家了。」十五日早晨燒紙通信,晚夕辭了西門慶靈,與衆姊妹置酒作别。把房門各庫房門鑰匙交付與小玉拏着:「前後仔細!」次日早五更起身,離了家門。一行人僱了頭口,衆姊妹送出大門而去。

那秋深時分,天寒日短,一日行兩程,六七十里之地,未到黄昏,投客店村坊安歇,次早再行。一路上秋雲淡淡,寒雁啛啛,樹木凋落,景物荒凉,不勝悲愴。有詩單道月娘為夫主遠涉關山答心願為證:

\begin{myquote}
平生志節傲冰霜,一點眞心格上蒼。

為夫遠許神州願,千里關山姓字香。
\end{myquote}

話休饒舌。一路無詞,行了數日,到了泰安州。望見泰山,端的是天下第一名山。根盤地脚,頂接天心,居齊魯之邦,有巖巖之氣象。吳大舅見天晚,投在客店,歇宿一宵。次日早起上山,望岱嶽廟來。那岱嶽廟就在山前,乃累朝祀典,歷代封禪,為第一廟貌也。但見:

\begin{myquote}
廟居岱嶽,山鎭乾坤;為山嶽之至尊,乃萬福之領袖。山頭倚檻,直望弱水蓬萊;絶頂攀松,都是濃雲薄霧。樓臺森聳,金烏展翅飛來;殿宇稜層,玉兔騰身走到。雕梁畫棟,碧瓦朱簷。鳳扉亮槅映黄紗,龜背繡簾垂錦帶。遙觀聖像,九旒冕舜目堯眉;近觀神顔,衮龍袍湯肩禹背。九天司命,芙蓉冠掩映絳綃衣;炳靈聖公,赭黄袍偏襯藍田帶。左侍下玉簪朱履,右侍下紫綬金章。闔殿威儀,護駕三千金甲將;兩廊勇猛,勤王十萬鐵衣兵。蒿里山下,判官分七十二司;白騾廟中,土神按二十四氣。管火池鐵面太尉,日日通靈;掌生死五道將軍,年年顯聖。御香不断,天神飛馬報丹書;祭祀依時,老幼望風皆獲福。嘉寜殿祥雲香靄,正陽門瑞氣盤旋。正是:萬民朝拜碧霞宫,四海皈依神聖帝。
\end{myquote}

吳大舅領月娘到了岱嶽廟,正殿上進了香,瞻拜了聖像,廟祝道士在傍,宣念了文書。然後兩廊都燒化了錢紙。喫了些齋食,然後纔領月娘上頂,登四十九盤,攀藤攬葛上去。娘娘金殿在半空中雲煙深處,約四十五里,風雲雷雨,都望下觀看。月娘衆人従辰牌時分岱嶽廟起身,登盤上頂,至申時已後方到娘娘金殿上,有朱红牌匾,金書「碧霞宫」三字。進入宫内,瞻禮娘娘金身。怎生模樣?但見:

\begin{myquote}
頭綰九龍飛鳳髻,身穿金縷絳綃衣。藍田玉帶曳長裾,白玉圭璋擎彩袖。臉如蓮蕚,天然眉目映雲鬟;唇似金朱,自在規模瑞雪體。猶如王母宴瑤池,卻似嫦娥離月殿。正大僊容描不就,威嚴形像畫難成!
\end{myquote}

月娘瞻拜了娘娘僊容。香案邊立着一個廟祝道士,約四十年紀。生的五短身材,三溜髭鬚,明眸皓齒;頭戴簪冠,身披絳服,足穿云履。向前替月娘宣讀了還願文疏,金爐内炷了香,焚化了紙馬金銀,令左右小童收了祭供。

原來這廟祝道士,也不是個守本分的。乃是前邊岱嶽廟裏金住持的大徒弟,姓石,雙名伯才,極是個貪財好色之輩,趨時攬事之徒。這本地有個殷太歲,姓殷,雙名天錫,乃是本州知州高廉的妻弟。常領許多不務本的人,或張弓挾彈,牽架鷹犬,在這上下二宫,專一睃看四方燒香婦女,人不敢惹他。這道士石伯才,專一藏姦蓄詐,替他賺誘婦女,到方丈任意姦淫,取他喜歡。因見月娘生的姿容非俗,戴着孝冠兒,若非官戶娘子,定是豪家閨眷;又是一位蒼白髭鬚老子,跟隨兩個家童,不免向前稽首,收謝神福:「請二位施主方丈一茶!」吳大舅便道:「不勞生受,還要趕下山去。」伯才道:「就是下山,也還早哩。」不一時,請至方丈,裏面糊的雪白,正面芝蔴花坐牀,柳黄錦帳,香几上供養一軸洞賓戲白牡丹圖畫,左右一聯,淡濃之筆大書:「攜兩袖清風舞鶴,對一軒明月談經。」問吳大舅上姓。大舅道:「在下姓吳,名鎧。這個就是舍妹吳氏,因為夫主來還香願,不當取擾上宫。」伯才道:「旣是令親,俱延上坐。」他便主位坐了,便呌徒弟守清守禮看茶。原來他手下有個徒弟,一個呌郭守清,一個名郭守禮,皆十六歲,生的標致。頭上戴青緞道髻,用紅絨䋲扎住總角,後用兩根飄帶,身穿青絹道服,脚上凉鞋淨襪,渾身香氣襲人。客至則遞茶遞水,斟酒下菜;到晚來,背地來掇箱子,拏他解纔塡餡。明雖為廟祝徒弟,實為師父大小老婆。更有一件不可說,脱了褲子,每人小腹裏夾着一條大手巾。看官聽說:但凡人家好兒好女,切記休要送與寺觀中出家,為僧作道,女孩兒做女冠姑子,都趁他男盗女娼,十個九個都着了道兒。有詩為證:

\begin{myquote}
琳宫梵刹事因何?道即天尊釋即佛。

廣栽花草虚清意,待客迎賓假做作。

羙衣麗服裝徒弟,浪酒閒茶戯女娥:

可惜人家嬌養子,送與師父作老婆。
\end{myquote}

不一時,兩個徒弟守清守禮房中安放桌兒,就擺齋上來。都是羙口甜食,蒸煠餅饊,减碟春饌,各樣菜蔬,擺滿春臺。白定磁盞兒,銀杏葉匙,絶品雀舌甜水好茶。喫了茶,收下家伙去,就擺上案酒,大盤大碗餚饌,都是鷄鵝魚鴨葷菜上來。□□壺□斟琥珀,銀鑲盞滿泛金波。吳月娘見酒來,就要起身,叫玳安近前,用紅漆盤託出一疋大布,二兩白金,與石道士作致謝之禮。吳大舅便說:「不當打攪上宫。這些微禮,致謝僊長。不勞見賜酒食,天色晚來,如今還要趕下山去。」慌的石伯才致謝不已,説:「小道不才,娘娘福蔭,在本山碧霞宫做個住持,仗賴四方錢糧,不管待四方財主,作何項下使用?今聊備粗齋薄饌,倒反勞見賜厚禮,使小道卻之不恭,受之有愧。」辭謝再三,方令徒弟收下去。一面留月娘吳大舅坐:「好歹坐片時,畧飲三盃,盡小道一點薄情而已。」吳大舅見欵留懇切,不得已和月娘坐下。

不一時,熱下飯上來。石道士吩咐徒弟:「這個酒不中喫。另打開昨日徐知府老爹送的那一罎透瓶香荷花酒來,與你吳老爹用。」不一時,徒弟另用熱壺篩熱酒上來。先滿斟一盃,雙手遞與月娘。月娘不肯接。吳大舅說:「舍妹他天性不用酒。」伯才道:「老夫人連路風霜,用些何害?好歹淺用些。」一面倒去半鍾,遞上去與月娘,接了。又斟一盃遞與吳大舅說:「吳老爹,你老人家試嘗此酒,其味何如?」吳大舅飲了一口,覺香甜絶羙,其味深長。說道:「此酒甚好。」伯才道:「不瞞你老人家說,此是青州徐知府老爹送與小道的酒。他老夫人、小姐、公子,年年來岱嶽廟燒香建醮,與小道相交極厚。他小姐衙内又寄名在娘娘位下,見小道立心平淡,殷勤香火,一味志誠,甚是敬愛小道。常年,這岱嶽廟上下二宫錢糧,有一半征收入庫。近年多虧了我這恩主徐知府老爹題奏過,也不征收,都全放常住用度,侍奉娘娘香火,餘者接待四方香友。」這裏說話,下邊玳安、來安,跟従轎夫,下邊自有坐處,湯飯點心,大盤大碗酒肉,都喫飽了。

看官聽說:這石伯才窝藏殷天錫,賺引月娘到方丈,要暗中取事,豈不加意奉承?飲了幾盃,吳大舅見天晚,要起身,伯才道:「日色將落,晚了,趕不下山去。倘不棄,在小道方丈權宿一宵,明早下山従容些。」吳大舅道:「爭奈有些小行李在店内,誠恐一時小人囉唣。」伯才笑道:「這個何須挂意!如有絲毫差遲,聽得是我這裏進香的,不拘村坊店道,聞風害怕。好不好把店家拏來本州夾打,就教他尋賊人下落。」吳大舅聽了就坐住了。伯才拏大鍾斟上酒,吳大舅見酒利害,遂偸酒在懷,推醉了更衣,要徃後邊閣上觀看隨喜。伯才便教徒弟守清引領,拏鑰匙開門,教大舅觀看去了。這月娘覺身子乏困,便要牀上側側兒。這石伯才一面把房門拽上,外邊坐去了。

也是合當有事,月娘方纔牀上歪着,忽聽裏面響亮了一聲,牀背後紙門内跳出一個人來:淡紅面貌,三柳髭鬚,約三十年紀;頭戴滲青巾,身穿紫錦袴衫。雙關抱住月娘,說道:「小生姓殷,名天錫,乃高太守妻弟。久聞娘子乃官豪宅眷,天然國色,思慕已久,渴欲一見,無由得會。今既接英標,乃三生有幸,死生難忘也!」一面按着月娘在牀上求歡。月娘唬的慌做一團,高聲大呌:「清平世界,朗朗乾坤,没事把良人妻室強𢺞攔在此做甚!」就要奪門而走。被天錫死命攔擋不放,便跪下說:「娘子禁聲。下顧小生,懇求憐允!」那月娘越高聲呌的聲緊了,口口大呌「救人」。來安玳安聽見是月娘聲音,慌慌張張走去後邊閣上呌大舅說:「大舅快去,我娘在方丈和人合口哩!」這吳大舅兩步做一步奔到方丈,推門,那裏推得開?只見月娘高叫:「清平世界,攔燒香婦女在此做甚麽!」這吳大舅便呌:「姐姐休慌,我來了!」一面拏石頭把門砸開。那殷天錫見有人來,撒開手,打牀背後一溜煙走了。原來這石道士牀背後都有出路。吳大舅砸開方丈門,問月娘道:「姐姐,那廝玷污不曾?」月娘道:「不曾玷污。那廝打牀背後走了。」吳大舅尋道士,那石道士躱去一邊,只敎徒弟來支調。被大舅大怒,喝令手下跟隨玳安來安兒,把道士門窻戶壁都打碎了。一面保月娘出離碧霞宫,上了轎子,便趕下山來。

約黄昏時分起身,走了半夜,投天明趕到山下客店内。如此這般,告店小二說。小二叫苦連聲說:「不合惹了殷太歲!他是本州知州相公妻弟,有名殷太歲。你便去了,把俺開店之家,他遭塌凌辱,怎肯干休!」吳大舅便多與他一兩店錢,取了行李,保定月娘轎子,急急奔走。後面殷天錫氣不捨,率領二三十閒漢,各執腰刀短棍,趕下山來。吳大舅一行人,兩程做一程,約四更時分,趕到一山凹裏。遠遠樹木叢中有燈光,走到跟前,卻是一座石洞,裏面有一老僧,秉燭念經。吳大舅問:「老師,我等頂上燒香,被強人所趕,奔下山來,天色昏黑,迷蹤失路至此。敢問老師,此處是何地名?従那條路囬家去?」老僧道:「此是岱嶽東峯,這洞名喚雪澗洞,貧僧就呌雪洞禪師,法名普靜,在此修行二三十年。你今遇我,實乃有緣。休徃前去,山下狼蟲虎豹極多。明日早行,一直大道,就是你清河縣了。」吳大舅道:「只怕有人追趕。」老師把眼一觀,說:「無妨,那強人趕至半山,已囬去了。」因問月娘姓氏。吳大舅道:「此乃吾妹,西門之妻。因為夫主來此進香。得遇老師搭救,恩有重報,不敢有忘!」於是在洞内歇了一夜。次日五更,月娘拏出一疋大布謝老師。老師不受,說:「貧僧只化你親生一子,作個徒弟,你意下如何?」吳大舅道:「吾妹止生一子,指望承繼家業,若有多餘,就與老師作徒弟出家。」月娘道:「小兒還小,今纔不到一周歲兒,如何來得?」老師道:「你只許下我,如今不問你要,過十五年纔問你要哩。」月娘口中不言,「過十五年,再作理會。」遂許下老師。看官聽說:不當今日許老師一子出家,後來十五年之後,天下荒亂,月娘㩦領孝哥孩兒,徃河南投奔雲離守就婚去,路遇老師,度化在永福寺,落髮為僧。此事表過不題。

次日,月娘辭了老師,徃前所進。走了一日,前有一山攔路。這座山名喚清風山,生的十分險惡。但見:

\begin{myquote}
八面嵯峨,四圍險峻。古怪喬松盤翠蓋,槎枒老樹挂藤蘿。瀑布飛來,寒氣逼人毛髮冷;巔崖直下,清光射目夢魂驚。澗水時聞,樵人斧響;峯巒倒卓,山鳥聲哀。麋鹿成羣,狐狸結黨;穿荆棘徃來跳躍,尋野食前後呼號。佇立草坡,一望並無商旅店;行來山徑,週迴盡是死屍坑。若非佛祖修行處,定是強人打劫塲。
\end{myquote}

原來這山喚作清風山,山上有座清風寨,寨中有三個強寇。一名錦毛虎燕順,一名矮脚虎王英,一個白面郎君鄭天壽。手下聚五百小嘍囉,專一打家劫道,放火殺人,人不敢惹他。當下吳大舅一行人騎頭口,簇擁着月娘轎子,進入山來。那時日色已落,天色昏黑,不見村坊店道,正在危懼之際,不防地下拋去一條絆馬索子,把吳大舅頭口絆落倒,跌落塹坑内,閃出一夥小嘍囉,將月娘轎子,搶上山來。原来山下小嘍囉,見吳大舅一行人,騎着馱垜,逕入山來,報與三個強寇。

吳大舅一行人都被㧱到寨前。三個強寇在寨上,正陪山東及時雨宋江飲酒。宋江因殺了娼婦閻婆惜,逃躱至此,三人留他寨中住幾日。宋江看見月娘頭戴孝髻,身穿縞素衣服,擧止端莊,儀容秀麗,斷非常人妻子,定是富家閨眷,因問其姓氏。月娘向前道了萬福:「大王,妾身吳氏之女,千戶西門慶之妻,守節孤孀。因為夫主病重,許下泰山香願。先在山上,被殷天錫所趕,走了一日一夜,要囬家去。不想天晚,悞従大王山下所過。行李馱垜,都不敢要,只是乞饒性命還家,萬幸矣。」宋江因見月娘詞氣哀惋動人,便有幾分慈憐之意,乃便欠身向燕順道:「這位娘子乃是我同僚正官之妻,有一面之識。為夫主到此進香,因被殷天錫所趕,悞到此山所過,有犯賢弟清蹕,他是個烈婦,看我宋江的薄面,放他囬去,以全他名節罷!」王英便說:「哥哥,爭奈小弟沒個妻室,讓與小弟做個押寨夫人罷!」遂令小嘍囉把月娘擄入他後寨去了。宋江向燕順鄭天壽道:「我恁説一塲,王英兄弟就不肯教我做個人情?」燕順道:「這兄弟諸般都好,只喫了有這些毛病,見了婦人女色,眼裏火就愛。」那宋江也不喫酒,同二人走到後寨。見王英正摟着月娘求歡。宋江走到跟前,一把手將王英拉着前邊,便說道:「賢弟旣做英雄,犯了『溜骨髓』三字,不為好漢。你要尋妻室,等宋江替你做媒,保一個室女好的,行茶過水,娶來做個夫人。何必要這再醮做甚麽?」王英道:「哥哥,你且胡亂權讓兄弟這個罷。」宋江道:「不好。我宋江久後决然替賢弟完娶一個好的。不爭你今日要了這婦人,惹江湖上好漢耻笑。殷天錫那廝,我不上梁山便罷,若上梁山,决替這個婦人報了仇。」看官聽說:後宋江到梁山做了寨主,因為殷天錫奪了柴皇城花園,使黑旋風李逵殺了殷天錫,大鬧了高唐州。此事表過不題。

當日燕順見宋江說此話,也不問王英肯不肯,喝令轎夫上來,把月娘擡了去。吳月娘見放了他,向前拜謝宋江說:「蒙大王活命之恩!」宋江道:「阿呀,我不是這山寨大王,我是鄆城縣客人。你只拜這三位大王便了。」月娘拜畢,吳大舅保着,離了山寨,上了轎子,過了清風山,徃清河縣大道前來。正是:撞碎玉籠飛彩鳳,頓開金鎖走蛟龍。有詩為證:

\begin{myquote}
世上只有人心歹,萬物還教天養人。

但教方寸無諸惡,狼虎叢中也立身!
\end{myquote}

畢竟未知後來何如,且聽下囬分解。

