\includepdf[pages={133,134},fitpaper=false]{tst.pdf}
\chapter*{第六十七囬 \\西門慶書房賞雪 李瓶兒夢訴幽情}
\addcontentsline{toc}{chapter}{第六十七囬 西門慶書房賞雪 李瓶兒夢訴幽情}
\markboth{{\titlename}卷之七}{第六十七囬 西門慶書房賞雪 李瓶兒夢訴幽情}


\begin{myquote}
終日思卿不見卿,數聲寒角未堪聞。

匣中破鏡收殘月,篋裏餘衣歛断雲。

寒鴉揀枝栖不定,征鴻断字嘆離群。

玉釵敲断心難碎,想像傷心記未眞。
\end{myquote}

話説西門慶歸後邊,辛苦的人,直睡至次日日色高還未起來。有來興兒進來說:「搭綵匠外邊伺候,請問拆棚。」西門慶罵了來興兒幾句,說:「拆棚敎他拆就是了,只顧問怎的?」搭綵匠一面外邊七手八脚,卸下蓆䋲松條,拆了送到對門房子裏堆放不題。玉簫進房說:「天氣好不陰的重!」西門慶令他向煖炕上取衣裳穿,要起來。有吳月娘便説:「你昨日辛苦了一夜,天陰,大睡回兒起來,慌的老早就爬起去做甚麽?就是今日不徃衙門裏去也罷了。」西門慶道:「我不徃衙門裏去。只怕翟親家那人來討書,好打發回書與他。」月娘道:「旣是恁説,你起去。我呌丫頭熬下粥等你來喫。」這西門慶也不梳頭洗臉,蓬頭披着絨衣,戴着毡巾,逕走到花園裏藏春閣書房中。

原來自従書童去了,西門慶就委王經管花園兩邊書房門鑰匙,春鴻便收拾打掃大廳前書房。冬月間,西門慶只在藏春閣書房中坐。那裏燒下的地爐煖炕,地平上又安放着黄銅火盆,放下梅梢月油單絹煖簾來。明間内擺着夾枝桃,各色菊花,清清瘦竹,翠翠幽蘭。裏面筆硯瓶梅,琴書瀟洒。牀炕上茜紅毡條,銀花錦褥,枕横鸂鶒,帳挂鮫絹。西門慶歪在牀上,王經連忙向桌上象牙盒内炷爇龍涎於流金小篆内。西門慶使王經:「你去呌來安兒請你應二爹去。」那王經出來,吩咐來安兒請去了。只見平安走來對王經説:「小周兒在外邊伺候。」那王經走入書房,對西門慶說了。西門慶叫進小周兒來,磕了頭,説道:「你來得好,且與我篦篦頭,捏捏身上。」因說:「你怎一向不來?」小周兒道:「小的見六娘沒了,忙,沒曾來。」西門慶於是坐在一張醉翁椅上,打開頭髮,敎他整理梳篦。只見來安兒請的應伯爵來了,頭戴毡帽,身穿綠絨襖子,脚穿一隻舊皂靴,棕套,掀簾子進來,唱喏。西門慶正篦頭,説道:「不消聲喏,請坐。」伯爵拉過一張椅子來,就着火盆坐下了。西門慶道:「你今日如何這般打扮?」伯爵道:「你不知,外邊飄雪花兒哩,好不寒冷!昨日家去晚了,鷄也呌了。你還使出大官兒來拉,俺們就去不的了。我見天陰上來,還討了個燈籠,和他大舅一路家去了。今日白爬不起來。不是來安兒去呌,我還睡哩。哥,你好漢,還起的早!若着我,成不的。」西門慶道:「早是你看着,我怎得個心閒?自従發送他出去了,又亂着接黄太尉,念經,直到如今,心上是那樣不遂。今早房下説:你辛苦了,大睡囬起去。我又記挂着只怕翟親家人來討囬書,又看着拆棚。二十四日又打發韓夥計和小价起身,打包,寫書帳。丧事費勞了人家,親朋罷了,士夫官員,你不上門謝謝孝,禮也過不去。」伯爵道:「正是。我愁着哥謝孝這一節。少不的也謝,只摘撥謝幾家要緊的,胡亂也罷了。其餘相厚,若會見,告過就是了。誰不知你府上事多,彼此心照罷。」

正說着,只見王經掀簾子,畫童兒用綵漆方盒銀鑲雕漆茶鍾,㧱了兩盞酥油白糖熬的牛奶子。伯爵取過一盞,㧱在手内,見白瀲瀲鵝脂一般酥油飄浮在盞内,説道:「好東西!滚熱。」呷在口裏,香甜羙味。那消費力,幾口就呵沒了。西門慶直待篦了頭,又敎小周兒替他取耳,把奶子放在桌上,只顧不喫。伯爵道:「哥,且喫些不是?可惜放冷了。像你清晨喫恁一盞兒,倒也滋補身子。」西門慶道:「我且不喫。你喫了,停會我喫粥罷!」那伯爵得不的一聲,㧱在手中一吸而盡。畫童收下鍾去。西門慶取畢耳,又呌小周兒㧱木滚子㨰身上,行按摩導引之術。伯爵問道:「哥滚着身子,也通泰自在些麽?」西門慶道:「不瞞你說,像我晚夕身上常時發酸起來,腰背疼痛。不着這般按捏,通了不得。」伯爵道:「你這胖大身子,日逐喫了這等厚味,豈無痰火?」西門慶道:「昨日任後溪常說:老先生雖故身體魁偉,而虚之太極。送了我一罐兒百補延齡丹,說是林眞人合與聖上喫的,教我用人乳常清晨服。我這兩日心上亂亂的,也還不曾喫。你們只説我身邊人多,終日有此事;自従他死了,誰有甚麽心緒理論此事!」

正說着,只見韓道國進來,作揖坐下,說:「剛纔各家都來會了,船已僱下,准在二十四日起身。」西門慶吩咐甘夥計攢下帳目,兌了銀子,明日打包。因問:「兩邊鋪子裏賣下多少銀兩?」韓道國說:「共凑六千餘兩。」西門慶道:「兑二千兩一包,着崔本徃湖州買紬子去。那四千兩,你與來保徃松江販布,過年趕頭水船來。你每人先㧱五兩銀子,家中收拾行李去。」韓道國道:「又一件,小人身従鄆王府,要正身上直,不納官錢,如何處置?」西門慶道:「怎的不納官錢?像來保,一般也是鄆王差事,他每月只納三錢銀子。」韓道國道:「保官兒那個,虧了太師老爺那邊文書上註過去,便不敢纏擾;小人此是祖役,還要勾當餘丁。」西門慶道:「旣是如此,你寫個揭帖,我央任後溪到府中替你和王奉承說,把你官身註銷,常遠納官錢罷!你每月只委付家下一個的當人打米就是了。」那韓夥計作揖謝了。伯爵道:「哥,你這一趟替他處了這件事,他就去也放心。」少頃,小周滚畢身上,西門慶徃後邊梳頭去了,吩咐打發小周兒喫了點心。良久,西門慶出來,頭戴白絨忠靖冠,身披絨氅,賞了小周三錢銀子。又使王經:「請你溫師父來。」不一時,溫秀才峨冠博帶而至。叙禮已畢,左右放桌兒,㧱粥上來,四碟小菜,一碗頓爛蹄子,一碗黄芽韮𤆑驢肉,一碗鮓𤆑餛飩鷄,一碗炖爛鴿子鶵兒,四甌軟稻粳米粥兒,安放四雙牙筯。伯爵與溫秀才上坐,西門慶関席,韓道國打横。西門慶吩咐來安兒再取一盞粥,一雙快兒,「請你姐夫來喫粥。」不一時,陳經濟來到,頭戴孝巾,身穿白紬道袍,蔥白緞氅衣,蒲鞋絨襪,與伯爵等作揖,打横坐下。須臾,喫了粥,收下家伙去,韓道國起身去了。只有伯爵、溫秀才,在書房坐的。西門慶因問溫秀才:「書可寫了不曾?」溫秀才道:「學生已寫稿在此,與老先生看過,方可謄眞。」一面袖中取出,遞與西門慶觀看。其書曰:

\begin{myquote}[\markfont]
\hspace*{4em}「寓清河眷生西門慶端肅書復

大碩德柱國雲峯老親丈大人先生台下:自従京邸邂逅,叙語之後,不覺違越光儀,倐忽半載。生以不幸,閨人不祿,特蒙親家遠致賻儀,兼領誨教,足見為我之深且厚也。感刻無任,而終身不能忘矣。但恐一時官守責成,有所疎陋之處,企仰門墻,有負薦拔耳。又賴在

老爺鈞前常為錦覆,則生始終蒙恩之處,皆親家所賜也。今因便鴻,謹候起居,不勝馳戀,伏惟炤亮,不宣。外具揚州縐紗汗巾十方,色綾汗巾十方,揀金挑牙二十付,烏金酒鍾十個,少將遠意,希笑納。」
\end{myquote}

西門慶看畢,即令陳經濟書房内取出人事來,同溫秀才封了,將書謄付錦箋,彌封停當,御了圖書。另外又封五兩白銀,與下書人王玉,不在話下。

一囬見雪下的大了,西門慶留下溫秀才在書房中賞雪。搽抹桌兒,㧱上案酒來。只見有人在煖簾外探頭兒,西門慶問:「誰?」王經説:「鄭春在這裏。」西門慶呌他進來,那鄭春手内㧱着兩個盒兒,擧的高高的跪在當面,上頭又擱着個小描金方盒兒。西門慶問:「是甚麽?」鄭春道:「小的姐姐月姐,知道昨日爹與六娘念經辛苦了,没甚麽,送這兩盒兒茶食兒來與爹賞人。」揭開:一盒菓餡頂皮酥,一盒酥油泡螺兒。鄭春道:「此是月姐親手自家揀的,知道爹好喫此物,敬來孝順爹。」西門慶道:「昨日又多謝你家送茶,今日你月姐費心,又送這個來。」伯爵道:「好呀,㧱過來,我正要嚐嚐。死了我一個女兒會揀泡螺兒,如今又是一個女兒會揀了。」先捏了一個放在口内,又撚了一個遞與溫秀才,說道:「老先兒,你也嚐嚐。喫了牙老重生,抽胎換骨。眼見稀奇物,勝活十年人!」溫秀才呷在口内,入口而化,說道:「此物出於西域,非人間可有。沃肺融心,實上方之佳味。」西門慶又問:「那小盒兒内是什麽?」鄭春悄悄跪在西門慶跟前,揭開盒兒,說:「此是月姐捎與爹的物事。」西門慶把盒子放在膝蓋兒上,揭開纔待觀看,一邊伯爵一手撾過去,打開,是一方迴紋錦雙攔子細撮穗古碌錢同心方勝結,桃紅綾汗巾兒,裏面裹着一包親口磕的瓜仁兒。這伯爵把汗巾兒掠與西門慶,將瓜仁兩把喃在口裏,都喫了。比及西門慶用手奪時,只剩下沒多些兒,便罵道:「怪狗才,你害饞癆饞痞?留些兒與我見見兒,也是人心!」伯爵道:「我女兒送來,不孝順我,再孝順誰?我兒,你尋常喫的夠了!」西門慶道:「溫先兒在此,我不好罵出來。你這狗才,忒不像模樣!」一面把汗巾收入袖中,吩咐王經把盒兒掇在後邊去。

不一時,盃盤羅列,篩上酒來。纔喫了一巡酒,玳安兒來説:「李智黄四関了銀子,送銀子來了。」西門慶問:「多少?」玳安道:「他説一千兩,餘者再一限送來。」伯爵道:「你看這兩個天殺的,他連我也瞞了,不對我説。嗔道他昨日你這裏念經他也不來,原來徃東平府関銀子去了。你今收了,也少要發銀子出去了;這兩個光棍,他攬的人家債也多了,只怕徃後,後手不接。昨日北邊徐内相發狠,要親徃東平府自家擡銀子去。只怕他老牛箍嘴箍了去,卻不難為哥的本錢了!」西門慶道:「我不怕他。我不管甚麽徐内相李内相,好不好我把他小廝提溜在監裏坐着,不怕他不與我銀子。」一面敎陳經濟:「你㧱天平出去,收兑了他的,上了合同就是了;我不出去罷。」

良久,陳經濟走來囬話,説:「銀子已兑足一千兩,交入後邊大娘收了。黄四說,還要請爹出去説句話兒。」西門慶道:「你只説我陪着人坐着哩。左右他只要揭合同的話,敎他過了二十四日來罷。」經濟道:「不是。他有樁事兒要央煩爹,請爹出去,親自對爹説。」西門慶道:「甚麽事,等我出去?」一面走到廳上。那黄四磕頭起來,說:「銀子一千兩,姐夫收了,餘者下單找還與老爹。有小人一樁事兒,今央煩老爹……」説着,跪在地下哭了。西門慶拉起來道:「端的有甚麽事,你說來。」黄四道:「小的外父孫清,搭了個夥計馮二,在東昌府販綿花。不想馮二有個兒子馮淮,不守本分,要便鎖了門出去宿娼。那日把綿花不見了兩大包,被小人丈人説了兩句,馮二將他兒子打了兩下。他兒子就和俺小舅子孫文相廝打,攘起來,把孫文相牙打落了一個,他亦把頭磕傷,被客夥中解勸開了。不想他兒子到家,遲了半月,破傷風身死。他丈人是河西有名土豪白五,綽號白千金,專一與強盜作窝主,敎唆馮二,具狀在巡按衙門朦朧告下來,批雷兵備老爹問。雷老爹又伺候皇船,不得閒,轉委本府童推官問。白家在童推官處使了錢,敎隣勸人供狀,説小人丈人在傍喝聲來。如今童推官行牌來提俺丈人,望乞老爹千萬垂憐,討封書對雷老爹説,寜可監幾日,抽上文書去,還見雷老爹問,就有生路了。他兩人廝打,委的不関小人丈人事;又係歇後身死,出于保辜限外。先是他父馮二打來,何必獨賴在孫文相一人身上?」西門慶看了説帖,寫着:「東昌府現監犯人孫清、孫文相,乞青目。」因說:「雷兵備前日在我這裏喫酒,我只會了一面,又不甚相熟,我怎好寫書與他!」那黄四就跪下,哭哭啼啼哀告說:「老爹若不可憐見,小的丈人子父兩個就都是死數了。如今隨孫文相投去罷了,只是分豁小人外父出來,就是老爹莫大之恩。小人外父今年六十歲,家下無人。冬寒時月,再放在監裏,就死罷了!」西門慶沉吟良久,說:「罷,我轉央鈔関錢老爹和他說說去;與他是同年,都是壬辰進士。」那黄四又磕下頭去,向袖中又取出一百石白米帖兒遞與西門慶,腰裏就解兩封銀子來。西門慶不接,説:「我那裏要你這行錢!」黄四道:「老爹不稀罕,謝錢老爹也是一般。」西門慶道:「不打緊,事成我買禮謝他。」

正説着,只見應伯爵従角門首出來,說:「哥,休替黄四哥說人情,他閒時不燒香,忙時走來抱佛腿。昨日哥這裏念經,連茶兒也不送,也不來走走兒,今日還來說人情?」那黄四便與伯爵唱喏,説道:「好二叔,你老人家殺人哩!我因這件事整走了這半月,誰得閒來?昨日又去府裏與老爹領這銀子。今日李三哥起早打卯去了,我竟來老爹這裏交銀子,就央說此事,救俺丈人。老爹再三不肯收這禮物,還是不下顧小人。」伯爵看見是一百兩雪花官銀放在面前,因問:「哥,你替他去說不說?」西門慶道:「我與雷兵備不熟,如今又轉央鈔関錢主政替他説去。到明日我買分禮謝老錢就是了,又收他禮做甚麽!」伯爵道:「哥,你這等就不是了。難道他來説人情,哥你賠出禮去謝人?也無此道理。你不收,恰似你嫌少的一般,倒難為他了。你依我,收下他這個禮。雖你不稀罕,明日謝錢公,也是一個樣兒。黄四哥在這裏聽着:看你外父和你小舅子造化,這一回求了書去,難得兩個都沒事出來,你老爹他恆是不稀罕你錢,你在院裏老實大大擺一席酒,請俺們耍一日就是了。」黄四道:「二叔,你老人家費心,小人擺酒不消說,還教俺丈人買禮來磕頭酬謝你老人家。不瞞你老人家,我為他爺兒兩個這一場事,晝夜上下替他走跳,還尋不出個門路來。老爹再不可憐,怎了?」伯爵道:「儍瓜,你摟着他女兒,你不替他上緊,誰上緊?」黄四道:「房下在家只是哭,俺丈人便躱了,家中連送飯人也沒一個兒。」

當下西門慶被伯爵説着,把禮帖收了,禮物還令他㧱回去。黄四道:「你老人家沒見,好大事,這般多計較!」就徃外走。伯爵道:「你過來,我和你説,你書幾時要?」黄四道:「如今緊等着救命,老爹今日下顧,有了書,差下人,明早我使小兒同去走遭。」於是央了又央:「差那位大官兒去?我會他會。」西門慶道:「我就替你寫書。」因呌過玳安來,吩咐:「你明日就同黄大官一路去。」那黄四見了玳安,辭西門慶出門,走到門首,問玳安要盛銀子搭褳。玳安進入後邊,月娘房裏正與玉簫小玉裁衣裳,見玳安站着等要搭褳,玉簫道:「使着手,不得閒騰,敎他明日來與他就是了。」玳安道:「黄四緊等着,明日早起身東昌府去,不得來了。你騰騰與他罷!」月娘便説:「你㧱與他就是了,只敎人家等着。」玉簫道:「銀子還在牀地平上掠着不是!」走到裏間,把銀子徃牀上只一倒,掠出搭褳來,説:「㧱去了,怪囚根子!那個喫了他這條搭褳,只顧立虰螞蝗的要。」玳安道:「人家不要,那個好來後邊取來?」於是㧱出,走到儀門首,還抖出三兩一塊蔴姑頭銀子來。原來紙包破了,怎禁玉簫使性那一倒,漏下一塊在搭褳底内。玳安道:「且喜得我拾個白財!」於是褪入袖中,到前邊遞與黄四搭褳,約會下明早起身。

且説西門慶囬到書房中,即時敎溫秀才修了書,付與玳安,不題。一面覷那門外雪,紛紛揚揚,猶如風飄柳絮,亂舞梨花相似。西門慶另打開一罈雙料麻姑酒,教春鴻用布甑篩上來。鄭春在傍彈箏低唱,西門慶令他唱一套「柳底風微」。正唱着,只見琴童進來説:「韓大叔教小的㧱了這個帖兒與爹瞧。」西門慶看了,吩咐:「你就㧱徃門外任醫官家,替他説説去,教他明日到府中承奉處替他說說,註銷差事。」琴童道:「今日晚了,小的明早去罷。」西門慶道:「是了。」不一時,來安兒用方盒㧱了八碗下飯:一碗黄熬山薬鷄,一碗臊子韮,一碗山薬肉圓子,一碗炖爛羊頭,一碗燒猪肉,一碗肚肺羹,一碗血臟湯,一碗牛肚兒,一碗爆炒猪腰子;又是兩大盤玫瑰鵝油盪麵蒸餅兒,連陳經濟共四人喫了。西門慶教王經㧱盤兒,㧱兩碗下飯,一盤點心與鄭春喫,又賞了他兩大鍾酒。鄭春跪禀:「小的喫不的。」伯爵道:「儍孩兒!冷呵呵的,你爹賞你不喫,你哥他怎的喫來?」鄭春道:「小的哥喫的,小的本喫不的。」伯爵道:「你喫一鍾罷,那一鍾敎王經替你喫。」王經道:「二爹,小的也喫不的。」伯爵道:「你這孩兒,你就替他喫些兒也罷。休說一個大分上,自古長者賜,少者不敢辭。」一面站起來,説:「我好歹教你喫這一盃。」那王經捏着鼻子,一吸而飲。西門慶道:「怪狗才,小行貨子他喫不的,只恁奈何他喫!」還剩下半盞,教春鴻替他喫了,令他上來排手唱南曲。西門慶道:「咱們和溫老先兒行個令,飲酒之時敎他唱便有趣。」於是叫王經取過骰盆兒,就是溫老先兒先起。溫秀才道:「學生豈敢僭?還従應老翁來。」因問:「老翁尊號?」伯爵道:「在下號南坡。」西門慶戲道:「老先生,你不知,他家孤老多,到晚夕桶子掇出屎來,不敢在左近倒,恐怕街坊人罵,敎丫頭直掇到大南首縣倉牆底下那裏潑去,因起號呌做『南潑』。」溫秀才笑道:「此『坡』字不同,那『潑』字,乃是點水邊之發,這『坡』字卻是『土』字傍邊着個『皮』字。」西門慶道:「老先兒倒猜的着,他娘子鎭日着皮子纏着哩!」溫秀才笑道:「豈有此説!」伯爵道:「葵軒,你不知道,他自來有些快傷叔人家。」溫秀才道:「自古言不褻不笑。」伯爵道:「老先兒,悞了咱們行令,只顧和他說甚麽?他快屎口傷人,你骰就在手,不勞謙遜。」溫秀才道:「擲出幾點,不拘詩詞歌賦,要個雪字,就照依點數兒上。説過來,飲一小盃;說不過來,喫一大盞。」當下溫秀才擲了個幺點,説道:「學生有了,雪殘鸂鶒立多時。」推過去該應伯爵行,擲出個五點來,伯爵想了半日,想不起來,說:「逼我老人家命也。」良久説道:「可怎的也有了!」說道:「雪裏梅花雪裏開。好不好?」溫秀才道:「老翁說差了,犯子兩個雪字,頭上多了一個雪字。」伯爵道:「頭上只小雪,後來下大雪來了。」西門慶道:「這狗才單管胡説。」敎王經斟上大鍾。春鴻拍手唱南曲〔駐馬廳〕:

\begin{myquote}
「寒夜無茶,走向前村覓店家。這雪輕飄僧舍,密酒歌樓,遙阻歸槎。江邊乘興探梅花,庭中歡賞燒銀蠟。一望無涯,有似灞橋柳絮滿天飛下。」
\end{myquote}

伯爵纔待拏起酒來喫,只見來安兒後邊㧱了幾碟菓食:一碟菓餡餅,一碟頂皮酥,一碟炒栗子,一碟晒乾棗,一碟榛仁,一碟瓜仁,一碟雪梨,一碟蘋婆,一碟風菱,一碟荸薺,一碟酥油泡螺,一碟黑黑的團兒,用橘葉裹着。伯爵撚將起來,聞着噴鼻香,喫到口,猶如飴蜜,細甜美味,不知甚物。西門慶道:「你猜!」伯爵道:「莫非是糖肥皂?」西門慶笑道:「糖肥皂那有這等好喫?」伯爵道:「待要説是梅蘇丸,裏面又有核兒。」西門慶道:「狗才,過來我說與你罷。你做夢也夢不着,是昨日小价杭州船上捎來,名喚做衣梅。都是各樣藥料,用蜜煉製過,滚在楊梅上,外用薄荷橘葉包裹,纔有這般羙味。每日清晨,呷一枚在口内,生津補肺,去惡味,煞痰火,解酒尅食,比梅蘇丸甚妙。」伯爵道:「你不説,我怎的曉的?」因說:「溫老先兒,咱再喫個兒。」教王經:「㧱張紙兒來,我包兩丸兒,到家捎與你二娘喫。」又㧱起泡螺兒來問鄭春:「這泡螺果然是你家月姐親手揀的?」那鄭春跪下說:「二爹,莫不小的敢說謊?不知月姐費了多少心,揀了這幾個兒來孝順爹。」伯爵道:「可也虧他,上頭紋溜就像螺螄兒一般,粉紅純白兩樣兒。」西門慶道:「我見此物,不免又使我傷心。惟有死了的六娘他會揀,他沒了,如今家中誰會弄他!」伯爵道:「我頭裏不説的,我愁甚麽,死了一個女兒會揀泡螺兒孝順我,如今又鑽出個女兒會揀了!偏你也會尋,尋的都是妙人兒!」西門慶笑的兩眼沒縫兒,趕着伯爵打,說:「你這狗才,單管只胡説!」溫秀才道:「二位老先生可謂厚之至極!」伯爵道:「老先兒你不知,他是你小姪人家。」西門慶道:「我是他家二十年舊孤老兒了。」陳經濟見二人犯言,就起身走了。那溫秀才只是掩口而笑。須臾,伯爵飲過大鍾,次該西門慶擲骰兒,於是擲出個七點來。想了半日,説:「我打〔香羅帶〕一句唱:『東君去意切,梨花似雪。』」伯爵道:「你説差了,此在第九個字上了,且喫一大鍾。」於是流沿兒斟了一銀衢花鍾,放在西門慶面前,教春鴻唱,說道:「我的兒,你肚子裏棗核解板兒——能有幾句兒!」春鴻又排手唱前腔:

\begin{myquote}
「四野彤霞,囬首江山白無涯。這雪輕如柳絮,細似鵝毛,白勝梅花。山前曲徑更添滑,村中魯酒偏增價。疊墜天花,疊墜天花,濠平溝滿令人驚訝。」
\end{myquote}

看看飲酒至昏,掌燭上來。西門慶飲過,伯爵道:「姐夫不在,溫老先生你還該完令。」這溫秀才㧱起骰兒,擲出個幺點,想了想,見書房牆上挂着一幅吊屏,泥金書一聯:「風飄弱柳平橋晚,雪點寒梅小院春。」説了未後一句,伯爵道:「不算,不算。不是你心上發出來的,該喫一大鍾。」春鴻斟上。那溫秀才不勝酒力,坐在椅上只顧打盹,起來告辭。伯爵只顧留他不住。西門慶道:「罷罷,老先兒他斯文人,喫不的。」令畫童兒:「你好好送你溫師父那邊歇去。」溫秀才得不的一聲,作别去了。伯爵道:「今日葵軒不濟。喫了多少酒兒,就醉了!」於是又飲夠多時,伯爵起身,說:「地下黑,我也酒夠了。」因說:「哥,明日你早教玳安替他下書去。」西門慶道:「你不見我交與他書?明日早去了。」伯爵掀開簾兒,見天陰地下滑,旋要了個燈籠,和鄭春一路去。西門慶又與了鄭春五錢銀子,盒内囬了一罐衣梅,捎與他姐姐鄭月兒喫。臨出門,西門慶因戲伯爵:「你哥兒兩個好好去。」伯爵道:「你多說話,父子上山,各人努力。好不好,我如今就和鄭月兒那小淫婦兒答話去。」說着,琴童送出門去了。

西門慶看收了家伙,扶着來安兒,打燈籠入角門,従潘金蓮門首所過,見角門関着。悄悄就徃李瓶兒房門首彈了彈門,有綉春開了門,來安就出去了。西門慶進入明間,見李瓶兒影,問:「供養了羹飯不曾?」如意兒就出來應道:「剛纔我和姐供養了。」西門慶入房中,椅上坐了,迎春㧱茶來喫了。西門慶令他解衣帶,如意兒就知他在這房裏歇,連忙收拾伸鋪,用湯婆熨的被窝暖洞洞的,打發他歇下。綉春把角門関了,都在明間地平上支着板櫈,打鋪睡下。西門慶要茶喫,兩個已知科範,連忙攛掇奶子進去和他睡。老婆脫了衣服,鑽入被窝内。西門慶乘酒興服了薬,那話上使了託子,老婆仰臥炕上,架起腿來,極力鼓搗,沒高低𢵞磞,𢵞磞的老婆舌尖冰冷,淫水溢下,口中呼達達不絶。夜靜時分,其聲遠聆數室。西門慶見老婆身上如綿瓜子相似,用一雙胳膊摟着他,令他蹲下身子,在被窝内咂ぎぐ,老婆無不曲體承奉。西門慶説:「我兒,你原來身體皮肉也和你娘一般白淨,我摟着你,就如同和他睡一般。你須用心伏侍我,我看顧你。」老婆道:「爹沒的説,將天比地,折殺奴婢,㧱甚麽比娘?奴婢男子漢已沒了,早晚爹不嫌醜陋,只看奴婢一眼兒就夠了。」西門慶便問:「你年紀多少?」老婆道:「我今年屬兔的,三十一歲了。」西門慶道:「你原來小我一歲。」見他會説話兒,枕上又好風月,心下甚喜。早晨起來,老婆先起來伏侍㧱鞋襪,打發梳洗,極盡慇勤,把迎春綉春打靠後。又問西門慶討蔥白紬子,做披襖兒與娘穿孝,西門慶一一許他。敎小廝舖子裏㧱三疋蔥白紬來,「你們一家裁一件。」以此見他兩三次打動了心,瞞着月娘,背地銀錢、衣服、首飾,甚麽不與他。

次日,潘金蓮就打聽得知,西門慶在李瓶兒房内和奶子老婆睡了一夜。走到後邊對月娘說:「大姐姐,你不説他幾句?賊沒廉耻貨,昨日悄悄鑽到那邊房裏,與老婆歇了一夜。餓眼見瓜皮,甚麽行貨子,好的歹的攬搭下!不明不暗,到明日弄出個孩子來算誰的?又像來旺兒媳婦子,徃後教他上頭上臉,甚麽張致!」月娘道:「你們只要栽派教我說!他要了死了的媳婦子,你們背地都做好人兒,只把我合在缸底下一般。我如今又做儍子哩!你們説只顧和他說,我是不管你這閒帳!」金蓮見月娘這般説,一聲兒不言語,走回房去了。

西門慶起早,見天晴了,打發玳安徃錢主事䖏下書去了。徃衙門囬來,平安兒來禀:「翟爹人來討回書。」西門慶打發書訖,因問那人:「你怎的昨日不來取?」那人說:「小的又徃巡撫侯爺那裏下書來,躭擱了兩日。」説畢,領書出門。西門慶喫了飯,就過對門房子裏,看着兌銀、打包、寫書帳。二十四日燒紙,打發韓夥計、崔本、來保,並後生榮海、胡秀五人,起身徃南邊去。寫了一封書,捎與苗小湖,就謝他重禮。

看看過了二十五六,西門慶謝畢孝,一日早晨,在上房喫了飯坐的。月娘便說:「這出月初一日,是喬親家長姐生日,咱也還買分禮兒送了去。常言:先親後不改。莫非咱家孩兒沒了,断了禮不送了!」西門慶道:「怎的不送?」於是吩咐來興買兩隻燒鵝,一副豕蹄,四隻鮮鷄,兩隻燻鴨,一盤壽麵,一套粧花緞子衣服,兩方綃金汗巾,一盒花翠,寫帖兒教王經送去。這西門慶吩咐畢,就徃前邊花園藏春閣書房中坐的。只見玳安下了書回來,回話説:「錢老爹見了爹帖子,隨即冩書,差了一吏,同小的和黄四兒子到東昌府兵備道下與雷老爹。老爹旋行牌問童推官催文書,連犯人提上去,従新問理。連他家兒子孫文相都開出來,只追了十兩燒埋錢,問了個不應罪名,杖七十,罰贖。復又到鈔関上囬了錢老爹話,討了回帖纔來了。」西門慶見玳安中用,心中大喜。拆開回帖觀看,原來雷兵備回錢主事帖子都在裏面。上寫道:

\begin{myquote}[\markfont]
「來諭悉已䖏分。但馮二已曾責子在先,何況與孫文相忿毆,彼此俱傷;歇後身死,又在保辜限外:問之抵命,難以平允。量追燒埋錢十兩,給與馮二。相應發落,謹此回覆。

\raggedleft{{\kaishu(下書)}年侍生雷起元再拜。」}
\end{myquote}

西門慶看了歡喜,因問:「黄四舅子在那裏?」玳安道:「他出來,都徃家去了,明日同黄四來與爹磕頭。黄四丈人與了小的一兩銀子。」西門慶吩咐置鞋脚穿。玳安磕頭而出。

西門慶就歪在牀炕上眠着了。王經在桌上小篆内炷了香,悄悄出來了。良久,忽聽有人掀的簾兒響:只見李瓶兒驀地進來,身穿糝紫衫,白絹裙,亂挽烏雲,黄懨懨面容,向牀前呌道:「我的哥哥,你在這裏睡哩!奴來見你一面。我被那廝告了我一狀,把我監在獄中,血水淋漓,與穢汚在一處,整受了這些時苦。昨日蒙你堂上說了人情,減了我三等之罪。那廝再三不肯,發恨還要告了來㧱你。我待要不來對你説,誠恐你早晚暗遭他毒手。我今尋安身之處去也,你須防範來!沒事,少要在外喫夜酒。徃那去,早早來家。千萬牢記奴言,休要忘了!」説畢,二人抱頭放聲而哭。西門慶便問:「姐姐,你徃那去?對我説。」李瓶兒頓然撒手,卻是南柯一夢。西門慶従睡夢中直哭醒來,看見簾影射入書齋,正當卓午,追思起由不的心中痛切,正是:花落土埋香不見,鏡空鸞影夢初醒。有詩為證:

\begin{myquote}
殘雪初晴照紙窻,地爐灰燼冷侵牀。

個中邂逅相思夢,風撲梅花斗帳香。
\end{myquote}

不想早晨送了喬親家禮,喬大戶娘子使了喬通來送請帖兒,請月娘衆姊妹。小廝説,爹在書房中睡哩,都不敢來問。月娘在後邊管待喬通。潘金蓮説:「㧱帖兒,等我問他去!」於是驀地進書房。潘金蓮上穿黑青迴紋錦對衿衫兒,泥金眉子,一溜㩟五道金三川鈕扣兒;下着紗裙,内襯潞紬裙,羊皮金滚邊。面前垂一雙合歡鮫綃鸂鶒帶;下邊尖尖趫趫錦紅膝褲下顯一對金蓮;頭上寳髻雲鬟,打扮如粉粧玉琢,耳邊帶着青寳石墜子。推開書房門,見西門慶歪着,他一屁股坐在椅子上,説:「我的兒,獨自個自言自語,在這裏做甚麽?嗔道不見你,原在這裏好睡也!」一面説話,口中嗑瓜子兒,因問西門慶:「眼怎生揉的恁紅紅的?」西門慶道:「我控着頭睡來。」婦人道:「倒只像哭的一般。」西門慶道:「怪奴才,我平白怎的哭?」金蓮道:「只怕你一時想起甚心上人兒來是的。」西門慶道:「沒的胡説,有甚心上人、心下人!」金蓮道:「李瓶兒是心上的,奶子是心下的。俺們是心外的人,入不上數!」西門慶道:「怪小淫婦兒,又六説白道起來!」因問:「我和你說正經話,前日李大姐裝綁,你們替他穿了甚麽衣服在身底下來?」金蓮道:「你問怎的?」西門慶道:「不怎的,我問聲兒。」金蓮道:「你問必有個緣故。上面他穿兩套遍地金緞子衣服,底下是白綾襖,黄紬裙,貼身是紫綾小襖、白絹裙、大紅緞小衣。」西門慶點了點頭兒。金蓮道:「我做獸醫二十年,猜不着驢肚裏病!你不想他,問他怎的?」西門慶道:「我纔方夢見他來。」金蓮道:「夢是心頭想,涕噴鼻子痒。饒他死了,你還這等念他。像俺都是可不着你心的人,到明日死了苦惱,也沒那人題念。——此是想的你這心裏胡油油的!」西門慶向前一手摟過他脖子來,就親了個嘴,說:「怪小油嘴,你有這些賊嘴賊舌的。」金蓮道:「我的兒,老娘猜不着你那黄貓黑尾的心兒!」一面把嗑了的瓜子仁兒,滿口哺與西門慶喫。兩個又咂了一囬舌頭,自覺甜唾溶心,脂香滿唇,身邊蘭麝襲人。西門慶於是淫心輒起,摟他在牀上坐。他便仰靠梳背,露出那話來,教婦人品簫,婦人眞個低垂粉項,吞吐裹沒,徃來嗚咂有聲。西門慶見他頭上戴金赤虎分心,香雲上圍着翠梅花鈿兒,後鬢上珠翹錯落,興不可遏。正做到羙處,忽聽來安兒隔簾說:「應二爹來了。」西門慶道:「請進來。」慌的婦人沒口子叫來安兒:「賊,且不要呌他進來,等我出去着。」來安兒道:「進來了,在小院内。」婦人道:「還不去敎他躱躱兒?」那來安兒走去説:「二爹且閃閃兒,有人在屋裏。」這伯爵便走到松牆傍邊看雪培竹子。王經掀着軟簾,只聽裙子響,金蓮一溜煙後邊走了。正是:雪隱鷺鷥飛始見,柳藏鸚鵡語方知。

伯爵進來,見西門慶唱喏,坐下。西門慶道:「你連日怎的不來?」伯爵道:「哥,惱的我了不的在這裏!」西門慶問道:「又怎的惱?你告我説。」伯爵道:「不好告你說。緊自家中沒錢,昨日俺房下那個,平白又桶出個孩兒來!但是人家白日裏還好撾撓,半夜三更,房下又七痛八病,少不得爬起來收拾草紙被褥,陸續看他,叫老娘去。打緊應寳又不在家——俺家兄使了他徃莊子上馱草去了,百忙撾不着個人。我自家打着燈籠,叫了巷口兒上鄧老娘來。及至進門,養下來了。」西門慶問:「養個甚麽?」伯爵道:「養了個小廝。」西門慶駡道:「儍狗才,生了兒子倒不好,如何反惱!是春花兒那奴才生的?」伯爵笑道:「是你春姨人家。」西門慶道:「那賊狗掇腿的奴才,誰敎你要他來,呌呌老娘還抱怨?」伯爵道:「哥,你不知,冬寒時月,比不的你們有錢的人家;家道又有錢,又有偌大前程官職,生個兒子出來,錦上添花,便喜歡。俺如今自家還多着個影兒哩,要他做甚麽?家中一窝子人口要喫穿盤纏。只這兩日,忙巴劫的魂也沒了!應寳逐日該操,當他的差事去了。家兄那裏是不管的。大小姐便打發出去了,天理在頭上,多虧了哥你!眼見的這第二個孩子又大了,交年便是十三歲。昨日媒人來討帖兒,我說:早哩,你且去着。緊自焦的魂也沒了,猛可半夜又鑽出這個業障來!那黑天摸地,那裏活變錢去?房下見我抱怨,沒計奈何,把他一根銀插兒與了老娘,發落去了。明日洗三,嚷的人家知道了,到滿月㧱甚麽使?到那日我也不在家,信信拖拖徃那寺院裏且住幾日去罷。」西門慶笑道:「你去了,好了和尚,卻打發來好趕熱被窝兒。你這狗才,到底占小便益兒!」又笑了一回。

那應伯爵故意把嘴谷都着不做聲。西門慶道:「我的兒,不要惱。你用多少銀,一發對我說,等我與你處。」伯爵道:「有甚多少!」西門慶道:「也夠你攪纏是的。到其間不夠了,又㧱衣服當去?」伯爵道:「哥若肯下顧,二十兩銀子就夠了,我寫個符兒在此。費煩的哥多了,不好開口的,又不敢塡數兒,隨哥尊意便了。」那西門慶也不接他文約,說:「沒的扯淡!朋友家,什麽符兒。」正說着,只見來安兒㧱茶進來。西門慶呌小廝:「你放下盞兒,喚王經來。」不一時,王經來到,西門慶吩咐:「你徃後邊對你大娘説,我裏間牀背閣上,有前日巡按宋老爹擺酒兩封銀子,㧱一封來。」王經應諾,去不多時,㧱銀子來。西門慶就遞與應伯爵説:「這封五十兩,你都㧱了使去,省的我又拆開他。原封未動,你打開看看。」伯爵道:「忒多了。」西門慶道:「多的你收着。眼下你二令愛不大了?你可也替他做些鞋脚衣裳,到滿月也好看。」伯爵道:「哥説的是。」將銀子拆開,都是兩司各府傾就分資,三兩一錠,松紋足色,滿心歡喜,連忙打恭致謝,說道:「哥的盛情,誰肯!眞個不收符兒?」西門慶道:「儍孩兒,誰和你一般計較?左右我是你老爺老娘家。不然,你但有事來,就來纏我?這孩子也不是你的孩子,自是咱兩個合養的。實和你說,過了滿月,把春花兒那奴才叫了來,且答應我些時兒,只當利錢,不算兑了帳。」伯爵道:「你春姨這兩日瘦的像你娘那樣哩!」兩個戯了一回。伯爵因問:「黄四丈人那事怎樣兒?」西門慶把玳安徃返的事告說了一遍:「錢龍野書到,雷兵備旋行牌提了犯人上去,従新問理,把孫文相父子兩個都開出來了,只認十兩燒埋錢,打了杖罪,沒事了。」伯爵道:「造化他了。他就點着燈兒,那裏尋這人情去?你不受他的,乾不受他的,雖然你不希罕,留送錢大人也好。別要饒了他,教他好歹擺一席大酒,裏邊請俺們坐一坐。你不説,等我和他說。饒了他小舅一個死罪,當别的小可事兒?」不説兩個在書房中說話。

且說月娘在上房㧱銀子與王經出來,只見孟玉樓走入房來,說他兄弟孟鋭在韓姨夫那裏,如今不久又起身,徃川廣販雜貨去,「今來辭辭他爹,在我屋裏坐着哩,爹在那裏?姐姐使個小廝對他爹說聲兒。」月娘道:「他在花園書房,和應二坐着哩。又說請他爹哩,頭裏潘六姐倒請的好他爹!喬通送帖兒來,等着問他爹去,就討他個話兒,到明日咱們好收拾了去。我便把喬通留下,打發喫茶。長等短等不見來,熬的喬通也去了。半日只見他従前邊走將來,教我問他:『你對他說了不曾?』他沒的話回,説:『噦,我就忘了和他說。一囬,應二來了,我就出來了。誰得久停久住和他説話來?』帖子還袖在袖子裏。教我說脆幫根兒咬:『早是沒甚緊勾當,教人只顧等着。你原來恁個沒尾巴行貨子,不知在前頭幹甚麽營生,那半日纔進來,恰好還不曾説!』乞我訌了兩句,徃前去了。」少頃,來安進來,月娘使他請西門慶,説孟二舅來了。西門慶便起身,留伯爵:「你休去了,我就來。」走到後邊,月娘先把喬家送帖來請說了。西門慶說:「那日只你一人去罷。熱孝在身,莫不一家子都出來?」月娘說:「他孟二舅來辭辭你,一兩日起身徃川廣去也,在那邊屋裏坐着哩。」又問:「頭裏你要那封銀子與誰?」西門慶悉言:「應二哥房裏春花兒,昨晚生了個兒子,問我借幾兩銀子使。告我說,他第二個女兒又大,愁的了不的。借助幾兩銀子使罷了。」月娘道:「好好!他恁大年紀,也纔見這個兒子,應二嫂不知怎的喜歡哩!到明日,咱也少不的送些粥米兒與他。」西門慶道:「這個不消説。到滿月,不要饒花子,奈何他好歹發帖兒,請你們徃他家走走去,就瞧瞧春花兒怎麽模樣!」月娘笑道:「左右和你家一般樣兒,也有鼻兒有眼兒,莫非别些兒!」一面使來安下邊請孟二舅來。

不一時,玉樓同他兄弟來拜見,叙禮已畢,西門慶陪他叙了回話,讓至前邊書房内與伯爵相見,吩咐小廝後邊看菜兒。於是放桌兒,篩酒上來,三人飲酒。西門慶敎再取雙鍾筯:「對門請溫師父陪你二舅坐。」來安不一時回説:「溫師父不在,望倪師父去了。」西門慶説:「請你姐夫來坐坐。」良久,陳經濟來,與二舅見了禮,打横坐下。西門慶問:「二舅幾時起身?去多少時?」孟鋭道:「出月初二日准起身。定不的年歲,還到荆州買紙,川廣販香蠟,着緊一二年也不定。販畢貨,就來家了。此去従河南陝西漢中去,囬來打水路,従峽江荆州那條路來,徃囬七八千里地。」伯爵問:「二舅貴庚多少?」孟鋭道:「在下虚度二十六歲。」伯爵道:「虧你年小小的,曉的這許多江湖道路。似俺們虚老了,只在家裏坐着。」須臾,添換上來,盃盤羅列。孟二舅喫至日西時分,告辭去了。

西門慶送了回來,還和伯爵喫了一囬。只見買了兩座箱庫來,西門慶委付陳經濟裝庫,問月娘尋出李瓶兒兩套錦衣,攪金銀錢紙裝在庫内。因向伯爵說:「今日是他六七,不念經,替他燒座庫兒。」伯爵道:「好快光陰,嫂子又早沒了個半月了。」西門慶道:「這出月初五日,是他断七,少不的替他念個經兒。」伯爵道:「這遭哥念佛經罷了。」西門慶道:「大房下說,他在時因生小兒,許了些《血盆經懺》;許下家中走的兩個女僧做首座,請幾衆尼僧,替他禮拜幾卷懺兒。」說畢,伯爵見天晚,說道:「我去罷,只怕你與嫂子燒紙。」又深深打恭說:「蒙哥厚情,死生難忘!」西門慶道:「難忘不難忘,我兒,你休推夢裏睡裏。你衆娘到滿月那日,買禮都要去哩。」伯爵道:「又買禮做甚!我就頭着地,好歹請衆嫂子到寒家光降光降。」西門慶道:「到那日,好歹把春花兒那奴才收拾起來,牽了來我瞧瞧。」伯爵道:「你春姨他說來,有了兒子,不用着你了。」西門慶道:「别要慌,我見了那奴才,和他答話。」伯爵佯長笑的去了。西門慶令小廝收了家伙。走到李瓶兒房裏,陳經濟和玳安已把庫裝封停當。那日玉皇廟永福寺報恩寺都送疏:道家是寳肅昭成眞君像,佛家是冥府第六殿變成大王。門外花大舅家,送了一盒匾食,十分冥紙。吳大舅子家也是如此。西門慶看着迎春擺設羹飯完備,下出匾食來,點上香燭,使綉春請了後邊吳月娘衆人來。西門慶與李瓶兒燒了紙,擡出庫去,教經濟看着大門首焚化,不在話下。正是:芳魂料不隨灰死,再結來生未了緣。

畢竟未知後來如何,且聽下囬分解。

