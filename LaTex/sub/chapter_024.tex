\includepdf[pages={47,48},fitpaper=false]{tst.pdf}
\chapter*{第二十四囬 \\陳經濟元夜戲嬌姿 惠祥怒詈來旺婦}
\addcontentsline{toc}{chapter}{第二十四囬 陳經濟元夜戲嬌姿 惠祥怒詈來旺婦}
\markboth{{\titlename}卷之三}{第二十四囬 陳經濟元夜戲嬌姿 惠祥怒詈來旺婦}


\begin{myquote}
銀燭高燒酒乍醺,當筵且喜笑聲頻。

蠻腰細舞章臺柳,檀口輕歌上苑春。

香氣拂衣來有意,翠花落地卻無聲。

不因一點風流趣,安得韓生醉後醒。
\end{myquote}

話説一日,天上元宵,人間燈夕。西門慶在家,廳上張掛花燈,鋪陳綺席。正月十六,合家歡楽飲酒。正面圍着石崇錦帳圍屏,掛着三盞珠子吊燈,兩邊擺列着許多紗燈椅桌。西門慶與吳月娘居上坐,其餘李嬌兒、孟玉樓、潘金蓮、李瓶兒、孫雪娥、西門大姐,都在兩邊列坐。都穿着錦綉衣裳、白綾襖兒、藍裙子,——惟有吳月娘穿着大紅遍地金通袖袍兒、貂鼠皮襖,下着百花裙。頭上珠翠堆盈,鳳釵半卸。春梅、玉簫、迎春、蘭香,一般兒四個家楽,在傍ち箏敲板,彈唱燈詞。獨於東首設一席,與女婿陳經濟坐。一般三湯五割,食烹異品,菓獻時新。小玉、元宵、小鸞、綉春,都在上面下菜斟酒。

那來旺兒媳婦宋惠蓮不得上來,坐在穿廊下一張椅兒上,口裏嗑瓜子兒。等的上邊呼喚要酒,他便揚聲叫:「來安兒、畫童兒,娘上邊要熱酒,快𤓎酒上來!賊囚根子,一個也沒在這裏伺候,都不知往那裏去了!」只見畫童盪酒上去。西門慶就罵道:「賊奴才,一個也不在這裏伺候,往那裏去來?賊少打的奴才!」小廝走來説道:「嫂子,誰往那去來?就對着爹説,吆喝教爹駡我!」惠蓮道:「上頭要酒,誰叫你不伺候?関我甚事!不駡你罵誰?」畫童兒道:「這地上乾乾淨淨的,嫂子嗑下恁一地瓜子皮,爹看見又駡了。」惠蓮道:「賊囚根子!六月債兒還得快。掃就是,甚麽打緊,教你彫佛眼兒?便當你不掃,丢着,另教個小廝掃。等他問,我只説得一聲。」畫童兒道:「耶嚛嫂子!將就些兒罷了,如何和我合氣!」於是取了苕帚來,替他掃瓜子皮兒。這宋惠蓮外邊嗑瓜子兒,不題。

卻説西門慶席上,見女婿陳經濟没酒,吩咐潘金蓮去遞一巡兒。這金蓮連忙下來滿斟一盃酒,笑嘻嘻遞與經濟,説道:「姐夫,你爹吩咐,好歹飲奴這盃酒兒。」經濟一壁接酒,一面把眼兒不住斜溜婦人,説:「五娘請尊便,等兒子慢慢吃!」婦人一徑將身子把燈影着,左手執酒,剛待的經濟用手來接,右手向他手背只一捏。這經濟一面把眼瞧着衆人,一面在下戲把金蓮小脚兒上踢了一下。婦人微笑,低聲道:「怪油嘴,你丈人瞧着待怎的?」看官聽説:兩個只知暗地裏調情頑耍,卻不知宋惠蓮這老婆只自一個兒在槅子外,窗眼裏被他瞧了個不亦楽乎。正是:當局者迷,傍觀者清。雖故席上衆人倒不曾看出來,卻被他向窗隙燈影下觀得仔細。口中不言,心下自思:「尋常時在俺們跟前,倒且是精細撇清,誰想暗地卻和這小夥子兒勾搭。今日被我看出破綻,到明日再搜求我,自有話説。」正是:

\begin{myquote}
誰家院内白薔薇,暗暗偷攀三兩枝。

羅袖隱藏人不見,馨香惟有蝶先知。
\end{myquote}

飲酒多時,西門慶忽被應伯爵差人請去賞燈吃酒去了。吩咐月娘:「你們自在頑耍,我往應二哥家吃酒去來。」玳安平安兩個小廝跟隨去了。

月娘與衆姊妹吃了一囬,但見銀河清淺,珠斗斕斑,一輪團圓皎月,従東而出,照得院宇猶如白晝。婦人或有房中換衣者,或有月下整粧者,或有燈前戴花者;惟有玉樓、金蓮、李瓶兒三個并惠蓮,在廳前看經濟放花兒。李嬌兒、孫雪娥、西門大姐,都隨月娘後邊去也。金蓮便向二人説道:「他爹今日不在家,咱對大姐姐説,往街上走走去。」惠蓮在傍説道:「娘們去,也携帶我走走。」金蓮道:「你既要去,你就往後邊問聲你大娘去,和你二娘,看他去不去。俺們在這裏等着你。」那惠蓮連忙往後邊去了。玉樓道:「他不濟事,等我親自問他聲去罷。」李瓶兒道:「我也往屋裏穿件衣裳去,這回來冷,只怕夜深了。」金蓮道:「李大姐,你有披襖子,帶出件來我穿着,省得我往屋裏去走一遭。」那李瓶兒應諾去了。獨剩着金蓮一個,看着經濟放花兒。見無人,走向經濟身上捏了一把,笑道:「姐夫原來只穿恁單薄衣裳,不害冷麽?」只見大家人來昭兒子小鐵棍兒笑嘻嘻在跟前,舞旋旋的且拉着經濟,問姑夫要炮𤍤放。這經濟恐怕打攪了事,巴不得與了他兩個元宵炮𤍤,支的他外邊耍去了。於是和金蓮打牙犯嘴,嘲戲説道:「你老人家見我身上單薄,肯賞我一件衣裳兒穿也怎的?」金蓮道:「賊短命,得其慣便了!頭裏躡了我的脚兒,我不言語;如今大胆又來問我要衣服穿!我又不是你影射的,何故把與你衣服穿?」經濟道:「你老人家不與也罷,如何扎筏子來唬我?」婦人道:「賊短命,你是城樓子上雀兒,好耐驚耐怕的蟲蟻兒?」正説着,見玉樓和惠蓮出來,向金蓮説道:「大娘因身上不方便,大姐不自在,故不去了。敎娘們走走,早些來家。李嬌兒害腿疼,也不走。雪娥見大姐姐不走,恐怕他爹來家嗔他,也不出門。」金蓮道:「都不去,罷!只咱和李大姐三個去罷。等他爹來家,隨他駡去!再不,把春梅小肉兒和上房裏玉簫,你房裏蘭香,李大姐房裏迎春,都帶了去,等他爹來家問,就教他答話。」小玉走來道:「俺奶奶也是不去,我也跟娘們走走。」玉樓道:「對你奶奶説了去,我前頭等着你。」良久,小玉問了月娘,笑嘻嘻出來。

當下三個婦人,帶領着一簇男女。來安畫童兩個小廝,打着一對紗吊燈跟隨。女婿陳經濟躧着馬點放煙火花炮與衆婦人瞧。宋惠蓮道:「姑夫,你好歹畧等等兒,娘們携帶我走走!我到屋裏搭搭頭就來。」經濟道:「俺們如今就行。」惠蓮道:「你不等,我就是惱你一生!」於是走到屋裏,換了一套綠閃紅緞子對衿襖兒,白挑線裙子。又用一方紅銷金汗巾子搭着頭,額角上貼着飛金,三個香茶翠面花兒,金燈籠墜子,出來跟着衆人走百病兒。月色之下,恍若僊娥,都是白綾襖兒,遍地金比甲。頭上珠翠堆滿,粉面朱唇。經濟與來興兒左右一邊一個,隨路放慢吐蓮、金絲菊、一丈蘭、賽月明。出的大街市上,但見香塵不断,遊人如蟻,花炮轟雷,燈光雜彩,簫鼓聲喧,十分熱鬧。左右見一隊紗燈引導一簇男女過來,皆披紅垂綠,以為出於公侯之家,莫敢仰視,都躱路而行。那宋惠蓮一囬叫:「姑夫,你放個桶子花我瞧。」一囬又道:「姑夫,你放個元宵炮𤍤我聽。」一囬又落了花翠,拾花翠;一回又掉了鞋,扶着人且兜鞋;左來右去,只和經濟嘲戲。玉樓看不上,説了兩句:「如何只見你掉了鞋?」玉簫道:「他怕地下泥,套着五娘鞋穿着哩!」玉樓道:「你叫他過來我瞧,眞個穿着五娘的鞋?」金蓮道:「他昨日問我討了一雙鞋,誰知成精的狗肉,他套着穿!」惠蓮於是摟起裙子來,與玉樓看,看見他穿着兩雙紅鞋在脚上,用紗綠線帶兒扎着褲腿,一聲兒也不言語。

須臾,走過大街,到燈市裏。金蓮向玉樓道:「咱如今往獅子街李大姐房子裏走走去。」於是吩咐畫童來安兒打燈先行,迤邐往獅子街來。小廝先去打門,老馮已是歇下,房中有兩個人家賣的丫頭,在炕上睡。慌的老馮連忙開了門,讓衆婦女進來,旋戳開爐子炖茶,挈着壺往街上取酒。孟玉樓道:「老馮,你且住,不要去打酒,俺們在家,酒飯吃的飽飽來,你們有茶,倒兩甌子來吃罷!」金蓮道:「你既㽞人吃酒,先飣下菜兒纔好。」李瓶兒道:「媽媽子,一瓶兩瓶取了來,打水不渾的,夠誰吃?要取一兩壜兒來。」玉樓道:「他哄你,不消取;只看茶來罷。」那婆子方纔不動身。李瓶兒道:「媽媽子,怎的不往那邊去走走,端的不知你成日在家做些甚麽。」婆子道:「奶奶,你看丢下這兩個業障在屋裏,誰看他?」玉樓便問道:「兩個丫頭是誰家賣的?」婆子道:「一個是北邊人家房裏使女,十三歲,只要五兩銀子;一個是汪序班家出來的家人媳婦,家人走了,主子把䯼髻打了,領出來賣,要十兩銀子。」玉樓道:「媽媽,我説與你,有一個人要,你赚他些銀子使。」婆子道:「三娘,果然是誰要?告我説。」玉樓道:「如今你二娘房裏只元宵兒一個,不夠使,還尋大些的丫頭使喚。你倒把這大的賣與他罷。」因問:「這丫頭十幾歲?」婆子道:「他今年屬牛,十七歲了。」説着,拿茶來,衆人吃了茶。那春梅玉簫並惠蓮都前後瞧了一遍,又到臨街樓上推開窗子瞧了一遍。陳經濟催逼説:「夜深了,看了快些家去罷。」金蓮道:「怪短命,催的人手脚兒不停住,慌的是些甚麽!」於是叫下春梅衆人來,方纔起身。馮媽媽送出門,李瓶兒因問:「平安往那裏去了?」婆子道:「今日這咱還没來,教老身半夜三更開門閉户等着他。」來安兒道:「今日平安兒跟了爹往應二爹家去了。」李瓶兒吩咐:「媽媽子,早些関了門,睡了罷!他多也是不來,省的悞了你的睡頭。明日早來宅裏伺候。你是石佛寺長老——請着你就張致了。」婆子道:「誰是老身主兒,老身敢張致?」李瓶兒道:「媽媽休得多言多語,明日早與你二娘送丫頭來。」説畢,看着他関了大門,這一簇男女方纔囬家。

走到家門首,只聽見賃房子的韓囬子老婆韓嫂兒聲喚。因他男子漢答應馬房内臣,他在家跟着人走百病兒去了,醉囬來家,説有人夜晚剜開他房門,偷了狗,又不見了些東西,坐在當街上撒酒風駡人。衆婦人方纔立住了脚。金蓮使來安兒:「你去叫韓嫂兒,等俺們問他個端的。」不一時,把韓嫂兒叫到當面:「你為甚麽來?」韓嫂兒不慌不忙,扠手向前拜了兩拜,説道:「三位娘在上,聽小媳婦従頭兒告訴——」唱〔耍孩兒〕為證:「太平佳節元宵夜」云云。玉樓等衆人聽了,每人掏袖中些錢果子與他,叫來安兒:「你叫你陳姐夫送他進屋裏。」那陳經濟且顧和惠蓮兩個嘲戲,不肯搊他去。金蓮使來安兒扶到他家中,吩咐敎他明日早來宅内漿洗衣裳,「我對你爹説,替你出氣。」那韓嫂兒千恩萬謝,回家去。

玉樓等剛走過門首來,只見賁四娘子穿着紅襖,玄色緞比甲,玉色裙,勒着銷金汗巾,在門首笑嘻嘻向前道個萬福,説道:「三位娘那裏走了走?請不棄到寒家獻茶。」玉樓道:「方纔因韓嫂兒哭,俺站住問了他聲。承嫂子厚意,天晚了,不到罷。」賁四娘子道:「耶嚛!三位娘上門怪人家,就笑話俺小家人家茶也奉不出一盃兒來?」生死拉到屋裏。原來外邊供養觀音八難並関聖賢,當門掛着雪花燈兒一盞。掀開門簾,他十四歲女兒長姐在屋裏。桌上兩盞紗燈,擺設着春臺菓酌,與三人坐。連忙教他長姐過來,「與三位娘磕頭遞茶!」玉樓金蓮每人與了他兩枝花兒;李瓶兒袖中取了方汗巾,又是一錢銀子,與他買瓜子兒嗑。喜歡的賁四娘子拜謝了又拜。款留不住,玉樓等起身。到大門首,小廝來興在門首迎接。金蓮就問:「你爹來家不曾?」來興道:「爹未回家哩。」三個婦人,還看着陳經濟在門首放了兩筒一丈菊和一筒大煙蘭,一個金盞銀臺兒,纔進後邊去了。西門慶直至四更來家。正是:醉後不知天色瞑,任他明月下西樓。

卻説陳經濟因走百病兒,與金蓮等衆婦人嘲戲了一路兒,又和來旺媳婦宋惠蓮兩個言來語去,都有意了。次日早晨梳洗畢,也不到舖子内,逕往後邊吳月娘房裏來。只見李嬌兒金蓮陪着吳大妗子坐的,放着炕桌兒,纔擺茶吃。月娘便往佛堂中燒香去了。這小夥兒向前作了揖,坐下。金蓮便説道:「陳姐夫,你好人兒!昨日敎你送送韓嫂兒,你就不動,只當還敎你小廝送去了。且和媳婦子打牙犯嘴,不知甚麽張致!等你大娘燒了香來,看我對他説不説!」經濟道:「你老人家還説哩,昨日險些兒子腰累㿚瘑了哩!跟了你老人家走了一路兒,又到獅子街房裏囬來,該多少里地?人辛苦走了,還教我送韓囬子老婆!教小廝送送也罷了。睡了多大囬就天亮了,今早還爬不起來。」正説着,吳月娘従佛堂燒了香來,經濟作了揖。月娘便問:「昨日韓嫂兒為甚麽撒酒風罵人?」經濟把因走百病被人剜開門,不見了狗,坐在當街哭喊罵人,「今早他漢子來家,一頓好打的,這咱還没起來哩。」金蓮道:「不是俺們囬來,勸的他進去了。一時你爹來家撞見,甚麽樣子!」説畢,玉樓、李嬌兒、大姐,都到月娘屋裏吃茶,經濟也陪着吃了茶。後次大姐囬房,駡經濟:「不知死的囚根子!平白和來旺媳婦子打牙犯嘴,倘忽一時傳的爹知道了,淫婦便沒事,你死也沒處死!」幾句説的經濟睜睜的。

那日西門慶在李瓶兒房裏宿歇,起來的遲。只見荆千户——新陞一處兵馬都監——來拜。西門慶纔起來,旋梳頭,包網巾,整衣出來,陪荆都監在廳上説話。一面使平安兒進來後邊要茶。宋惠蓮正和玉簫小玉在後邊院子裏撾子兒,賭打瓜子,頑成一塊。那小玉把玉簫騎在底下,笑罵道:「賊淫婦,輸了瓜子,不教我打!」因叫惠蓮:「你過來,扯着淫婦一隻腿,等我㒲這淫婦一下子。」正頑着,只見平安走來,叫玉簫:「姐,前邊荆老爹來,使我進來要茶哩。」那玉簫且和小玉廝打頑耍,不理他。那平安兒只顧催逼説:「人坐下來這一日了。」宋惠蓮道:「怪囚根子,爹要茶,問厨房裏上竃的要去,如何只在俺這裏纏?俺這後邊只是預備爹娘房裏用的茶,不管你外邊的帳。」那平安兒走到廚房下,那日該來保妻惠祥,惠祥道:「怪囚,我這裏使着手做飯,你問後邊要兩鍾茶出去就是了,巴巴來問我要茶!」平安道:「我到後頭來,後邊不打發茶。惠蓮嫂子説,該是那上竃的首尾,問那個要。他不管哩!」這惠祥便罵道:「賊潑婦,他認定了他是爹娘房裏人,俺天生是上竃的來?我這裏又做大家夥裏飯,又替大娘子炒素菜,幾隻手?論起就倒倒茶兒去也罷了,巴巴坐名兒來尋上竃的,上竃的是你叫的!誤了茶也罷,我偏不打發上去。」平安道:「荆老爹來坐了這一日,嫂子快些打發茶,我拿上去罷。遲了又惹爹駡!」當下這裏推那裏,那裏推這裏,就躭誤了半日。比及又等玉簫取茶菓茶匙兒出來,平安兒拿出茶去,那荆都監坐的久了,再三要起身,被西門慶留住。嫌茶冷不好吃,喝罵平安來,另換茶上去吃了,荆都監纔起身去了。西門慶進來問:「今日茶是誰炖的?」平安道:「是竃上炖的茶。」西門慶囬到月娘上房,告訴月娘:「今日炖這樣茶去與人吃,你往廚下查那個奴才老婆上竃?採出來問他,打與他幾下。」小玉道:「今日該惠祥上竃哩。」慌的月娘説道:「這歪辣骨,待死!越發炖恁樣茶上去了。」一面使小玉叫將惠祥,當院子跪着,問他要打多少?惠祥答道:「因忙做飯,炒大娘子素菜,使着手,茶畧冷了些。」被月娘數罵了一囬,纔饒了他起來。吩咐:「今後但凡你爹前邊人來,教玉簫和惠蓮後邊炖茶,竃上只管大家茶飯。」

這惠祥在廚下忍氣不過,剛等的西門慶出去了,氣狠狠走來後邊,尋着惠蓮,指着大駡:「賊淫婦,趂了你的心了罷了!你天生的就是有時運的爹娘房裏人;俺們是上竃的老婆來!巴巴使小廝坐名問上竃要茶,上竃的是你呌的?你我生米做成熟飯,你識我見的!促織不吃癩蝦蟆肉——都是一鍬土上人。你恒數不是爹的小老婆就罷了;是爹的小老婆我也不怕你!」惠蓮道:「你好沒要緊,你炖的茶不好,爹嫌你,管我甚事?你如何走來拿人撒氣?」惠祥聽了此言,越發惱了,駡道:「賊淫婦!你剛纔調唆打我幾棍兒好來,怎的不教打我?你在蔡家養的漢數不了,來這裏還弄鬼哩!」惠蓮道:「我養漢,你看見來?沒的扯臊淡哩!嫂子,你也不是什麽清淨姑姑兒!」那惠祥道:「我怎不是清淨姑姑兒?蹺起腳兒來,比你這淫婦好些兒。我不説你罷,漢子有一拿小米數兒!你在外邊,那個不吃你嘲過?你説你背地幹的那營生兒,只説人不知道。你把娘們還放不到心上,何况以下的人!」惠蓮道:「我背地説甚麽來?怎的放不到心上?隨你壓我,我不怕你!」惠祥道:「有人與你做主兒,你可不怕哩!」

兩個正拌嘴,被小玉兒請的月娘來,把兩個都喝開了:「賊臭肉們,不幹那營生去,都拌的是些甚麽?教你主子聽見又是一場兒。頭裏不曾打得成,等住囬卻打得成了!」惠蓮道:「若打我一下兒,我不把淫婦口裏腸抅了也不算!我破着這命擯兑了你,也不差甚麽。咱大家都離了這門罷!」説着,往前去了。後次這宋惠蓮越發猖狂起來。仗西門慶背地和他勾搭,把家中大小都看不到眼裏。逐日與玉樓、金蓮、李瓶兒、西門大姐、春梅,在一䖏頑耍。

那日馮媽媽送了丫頭來,約十三歲,先到李瓶兒房裏看了,送到李嬌兒房裏,李嬌兒用五兩銀子買下,房中伏侍,不在話下。正是:梅花恣逞春情性,不怕封夷號令嚴。有詩為證:

\begin{myquote}
外作禽荒内色荒,連沾些子又何妨。

早晨跨得雕鞍去,日暮歸來紅粉香。
\end{myquote}

畢竟未知後來何如,且聽下回分解。

