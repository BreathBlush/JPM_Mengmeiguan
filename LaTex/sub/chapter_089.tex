\includepdf[pages={177,178},fitpaper=false]{tst.pdf}
\chapter*{第八十九囬 \\清明節寡婦上新墳 吳月娘悞入永福寺}
\addcontentsline{toc}{chapter}{第八十九囬 清明節寡婦上新墳 吳月娘悞入永福寺}
\markboth{{\titlename}卷之九}{第八十九囬 清明節寡婦上新墳 吳月娘悞入永福寺}


\begin{myquote}
風拂煙籠錦旆揚,太平時節日初長。

能添壯士英雄膽,善解覊人愁悶腸。

三尺繞垂楊柳岸,一竿斜插杏花旁。

男兒未遂平生志,且楽高歌入醉鄉。
\end{myquote}

話説吳月娘次日備辦了一張祭桌,猪首三牲、羹飯冥紙之類,封了一疋尺頭,教大姐收拾一身縞素衣服,坐轎子,薛嫂兒押着祭禮先行。來到陳宅門首,只見陳經濟正在門首站立。那薛嫂把祭禮敎人擡進去,經濟便問:「那裏的?」薛嫂道了萬福說:「姐夫,你休推不知。你丈母家來與你爹燒紙,送大姐來了。」經濟便道:「我ぎぐ㒲的纔是丈母!正月十六日貼門神——遲了半月。人也入了土,纔來上祭!」薛嫂道:「好姐夫,你丈母説,寡婦人沒脚蟹,不知你這裏親家靈柩來家,遲了一步,休怪。」正說着,只見大姐轎子落在門首。經濟問是誰,薛嫂道:「再有誰?你丈母心内不好,一者送大姐來家,二者敬與你爹燒紙。」經濟罵道:「趁早把淫婦擡囘去。好的死了萬萬千千,我要他做甚麽!」薛嫂道:「常言道:嫁夫招主,怎的說這個話?」經濟道:「我不要這淫婦了,還不與我走!」那擡轎的只顧站立不動,被經濟向前踢了兩脚,罵道:「還不與我擡了去,我把花子腿砸折了,把淫婦鬢毛都薅淨了!」那擡轎子的見他踢起來,只得擡轎子往家中走不迭。比及薛嫂呌出他娘張氏來,轎子已擡的去了。薛嫂兒沒奈何,敎張氏收下祭禮,走來囬覆吳月娘。把吳月娘氣的一個發昏,說道:「恁個沒天理的短命囚根子!當初你家為了官事,躱來丈人家居住,養活了這幾年,今日反恩將仇報起來了!只恨死鬼,當初攪下的好貨在家裏,弄出事來,到今日敎我做臭老鼠,敎他這等放屁辣臊。」對着大姐說:「孩兒,你是眼見的,丈人丈母那些兒虧了他來?你活是他家人,死是他家鬼,我家裏也難以留你。你明日還去,休要怕他,料他挾不到你井裏。他好膽子,恆是殺不了人。難道世間沒王法管他也怎的!」當晚不題。

到次日,一頂轎子,教玳安兒跟隨着,把大姐又送到陳經濟家來。不想陳經濟不在家,往坟上替他父親添土疊山子去了。張氏知禮,把大姐留下,對着玳安說:「大官到家,多多上覆親家:多謝祭禮,休要和他一般兒見識。他昨日已有酒了,故此這般。等我慢慢說他。」一面管待玳安兒,安撫來家。至晚,陳經濟坟上囘來,看見了大姐,就行踢打,罵道:「淫婦,你又來做甚麽?還自說我在你家雌飯喫!你家收着俺許多箱籠,因此起的這大産業,不道的白養活了女婿!好的死了萬千,我要你這淫婦做甚!」大姐亦罵:「沒廉耻的囚根子,沒天理的囚根子!淫婦出去喫人殺了,沒的禁拿我煞氣!」被經濟採過頂髮,儘力打了幾拳頭。他娘走來解勸,把他娘推了一跤。他娘呌罵哭喊說:「好囚根子,紅了眼,連我也不認的了!」到晚上,一頂轎子把大姐又送將來,吩咐道:「不討將寄放粧奩箱籠來家,我把你這淫婦活殺了!」這大姐害怕,躱在家中居住,再不敢去了。有詩為證:

\begin{myquote}
相識當初信有疑,心情還似永無涯。

誰知好事多更變,一念翻成怨恨謀。
\end{myquote}

這裏西門大姐在家躱住,不敢去了,不題。一日,三月清明佳節,吳月娘備辦香燭金銀冥紙、三牲祭物酒餚之類,擡了兩大食盒,要往城外五里原新坟上與西門慶上新坟祭掃。留下孫雪娥和着大姐衆丫頭看家,帶了孟玉樓和小玉,並奶子如意兒抱着孝哥兒,都坐轎子,往坟上去。又請了吳大舅和大妗子老公母二人同去。出了城門,只見那郊原野曠,景物芳菲,花紅柳綠,仕女遊人不斷頭的走的。一年四季,無過春天,最好景緻:日謂之麗日,風謂之和風,吹柳眼,綻花心,拂香塵:天色暖謂之暄,天色寒謂之料峭;騎的馬謂之寳馬,坐的轎謂之香車,行的路謂之香徑,地下飛起土來謂之香塵;千花發蕊,萬草生芽,謂之春信。春忒然好,有首詞曰:

\begin{myquote}
韶光淡蕩,淑景融和。小桃深粧臉妖嬈,嫩柳嬝宫腰細膩。百囀黄鸝驚囘午夢,數聲紫燕説破春愁。日舒長煖澡鵝黄,水渺茫浮香鴨綠。隔水不知誰院落,鞦韆高掛綠楊蔭。
\end{myquote}

端的春景,果然是好!到的春來,那府州縣道與各處村鎭鄉市都有遊翫去處。有詩為證:

\begin{myquote}
清明何處不生煙,郊外微風掛紙錢。

人笑人歌芳草地,乍晴乍雨杏花天。

海棠枝上綿鶯語,楊柳堤邊醉客眠。

紅粉佳人爭畫板,綵䋲搖曳學飛僊。
\end{myquote}

卻説吳月娘等轎子到五里原坟上,玳安押着食盒又早先到,厨下生起火來,厨役落作整理不題。月娘與玉樓、小玉、奶子如意兒抱着孝哥兒,到於莊院客坐内,坐下喫茶,等着吳大妗子不見到。玳安向西門慶坟上祭臺上,擺設桌面三牲羹飯祭物,列下紙錢,只等吳大妗子。原來大妗子僱不出轎子來。約巳牌時分纔同吳大舅僱了兩個驢兒騎將來。月娘便說:「大妗子僱不出轎子來,這驢兒怎麽騎?」一面喫了茶,換了衣服,走來西門慶坟前祭掃。那月娘手撚着五根香:一根香他拿在手内,一根香遞與玉樓,一根遞與奶子如意兒替孝哥兒上,那兩根遞與吳大舅、大妗子。月娘插在香爐内,深深拜下去說道:「我的哥哥,你活時為人,死後為神。今日三月清明佳節,你的孝妻吳氏三姐、孟三姐,同你周歲孩童孝哥兒,敬來與你坟前燒一陌錢紙。你保佑他長命百歲,替你做坟前拜掃之人。我的哥哥,我和你做夫妻一場,想起你那模樣兒並說的話來,是好傷感人也!」玳安把紙錢點着,有哭〔山坡羊〕為證:

\begin{myquote}
「燒罷紙,小脚兒連跥。奴與你做夫妻一場,並沒個言差語錯。實指望同諧到老,誰知你半路將奴抛卻。當初人情看望,全然是我。今丢下銅斗兒家緣孩兒又小,撇的俺子母孤孀怎生遣過?恰便似中途遇雨半路裏遭風來呵,拆散了鴛鴦,生揪断異菓!呌了聲,好性兒的哥哥,想起你那動靜行藏,可不嗟嘆殺我!」{\marktext\small〔带步步嬌〕}:

「燒的紙灰兒團團轉,不見我兒夫面。哭了聲年少夫,撇下嬌兒閃的奴孤單。咱兩無緣,怎得和你重相見!」
\end{myquote}

玉樓向前插上香,深深拜下,哭唱〔前腔〕:

\begin{myquote}
「燒罷紙,滿眼淚墮。呌了聲人也天也,丢的奴無有個下落。實承望和你白頭廝守,誰知道半路花殘月没。大姐姐有兒童他房裏還好,閃的奴樹倒無陰跟着誰過?獨守孤帷怎生奈何?恰便似前不着店後不着村裏來呵,那是我葉落歸根,收圓結果?呌了聲,年小的哥哥!要見你只除非夢兒裏相逢,卻不想念殺了我!」{\marktext\small〔带步步嬌〕}:

「哭來哭去哭的奴癡獃了,你一去了無消耗。思量好無下梢,無下梢!你正青春奴又多嬌。好心焦,清减了花容月貌!」
\end{myquote}

玉樓上了香,奶子如意抱着哥兒,也跪下上香,磕了頭。吳大舅大妗子都炷了香,行畢禮數,同讓到莊上捲棚内,放桌席擺飯,收拾飲酒。月娘讓吳大舅大妗子上坐,月娘與玉樓打横,小玉和奶子如意兒,同大妗子家使的老姐蘭花,都兩邊打横列坐,把酒來斟。按下這裏喫酒不題。

卻表那日周守備府裏也上坟。先是春梅隔夜和守備睡,假推做夢,睡夢中哭醒了。守備慌的問:「你怎的哭?」春梅便說:「我夢見我娘向我哭泣,說養我一場,怎地不與他清明寒食燒紙兒?因此哭醒了。」守備道:「這個也是養女一場,你的一點孝心。不知你娘坟在何處?」春梅道:「在南門外,永福寺後面便是。」守備說:「不打緊,永福寺是我家香火院。明日咱家上坟,你敎伴當擡些祭物,往那裏與你娘燒分紙錢,也是好處。」至次日,守備令家人收拾食盒酒菓祭品,逕往城南祖坟上,那裏有大莊院、廳堂、花園去處,那裏有享堂、祭臺,大奶奶、孫二娘並春梅,都坐四人轎,排軍喝路,上坟耍子去了。

卻說吳月娘和大舅大妗子喫了囬酒,恐怕晚來,吩咐玳安、來安兒,收拾了食盒酒菓,先往那十里長堤杏花村酒樓下,揀高阜去處、人煙熱鬧那裏,設放桌席等候。又見大妗子沒轎子,都把轎子擡着,後面跟隨不坐,吳月娘領定一簇男女,——吳大舅牽着驢兒壓後同行——踏青遊翫。三里抹過桃花店,五里望見杏花村,只見那隨路上坟遊翫的王孫士女,花紅柳綠,鬧鬧喧喧,不斷頭的走。偏襯着日煖風和,尋芳問景,不知又多少。正走之間,也是合當有事,遠遠望見綠槐影裏一座庵院,蓋造得十分齊整。但見:

\begin{myquote}
山門高聳,梵宇清幽。當頭敕額字分明,兩下金剛形勢猛。五間大殿,龍鱗瓦砌碧成行;兩廊僧房,龜背磨磚花嵌縫。前殿塑風調雨順,後殿供過去未來。鐘鼓樓森立,藏經閣巍峨。旛竿高峻接青雲,寳塔依稀侵碧漢。木魚横掛,雲板高懸。佛前燈燭熒煌,爐内香煙繚繞。幢幡不斷,觀音殿接祖師堂;寳蓋相連,鬼母位通羅漢院。時時護法諸天降,歲歲降魔尊者來。
\end{myquote}

吳月娘便問:「這座寺呌着甚麽寺?」吳大舅便說:「此是周秀老爺香火院,名喚永福禪林。前日姐夫在日,曾捨幾十兩銀子在這寺中,重修佛殿,方是這般新鮮。」月娘向大妗子說:「咱也到這寺中看一看。」於是領着一簇男女,進入寺中來。不一時,小沙彌看見,報與長老知道。見有許多男女,便出方丈來迎,「請施主菩薩隨喜。」但見這長老,怎生模樣:

\begin{myquote}
一個青旋旋光頭新剃,把麝香松子匀搽。一領黄烘烘直裰初縫,使沉速旃檀濃染。山根鞋履,是福州染到深青;九縷絲縧,係西地買來眞紫。那和尚光溜溜一雙賊眼,單睃趁施主嬌娘;這秃廝羙甘甘滿口甜言,專説誘喪家少婦。淫情動䖏,草庵中去覓尼姑;色膽發時,方丈内來尋行者。仰觀神女思同寢,每見嫦娥要媾歡。
\end{myquote}

這長老見吳大舅吳月娘,向前合掌道了問訊,連忙喚小和尚開了佛殿:「請施主菩薩隨喜遊翫,小僧看茶。」那小沙彌開了殿門,領月娘一簇男女,前後兩廊參拜,觀看了一囘,然後到長老方丈。長老連忙點上茶來,雪錠般盞兒,甜水好茶。吳大舅請問長老道號,那和尚笑嘻嘻說:「小僧法名道堅。這寺是恩主帥府周爺香火院,小僧忝在本寺長老,廊下管百十僧衆,後邊禪堂中還有許多雲遊僧行,常川坐禪,與四方檀越答報功德。」一面方丈中擺齋,讓月娘:「衆菩薩請坐,小僧一茶而已。」月娘道:「不當打攪長老寳刹。」一面拿出五錢銀子,敎大舅遞與長老:「佛前請香燒。」那和尚笑吟吟打問訊謝了,說道:「小僧無甚管待,施主菩薩少坐,略備一茶而已,何勞費心賜與布施。」不一時,小和尚放了桌兒,拿上素菜齋食餅饊上來。那和尚在旁陪坐。擧筯兒纔待讓月娘衆人喫時,忽見兩個青衣漢子,走的氣喘吁吁,暴雷也一般報與長老說道:「長老還不快出來迎接,府中小奶奶來祭祀來了!」慌的長老披袈裟戴僧帽不迭,吩咐小沙彌連忙收了家活:「請列位菩薩且在小房避避,打發小夫人燒了紙,祭畢去了,再欵坐一坐不遲。」吳大舅告辭,和尚死活留住,又不肯放。

那和尚慌的鳴起鐘鼓來,出山門迎接,遠遠在馬道口上等候。只見一簇青衣人,圍着一乘大轎,従東雲飛般來,轎夫走的個個汗流滿面,衣衫皆濕。那長老躬身合掌說道:「小僧不知小奶奶前來,理合遠接,接待遲了,萬勿見罪!」這春梅在簾内答道:「起動長老。」那手下伴當,又早向寺後金蓮坟上擡將祭桌來,擺設已齊,紙錢列下。春梅轎子來到,也不到寺,逕入寺後白楊樹下金蓮坟前,下了轎子,兩邊青衣人伺候。這春梅不慌不忙,來到坟前插了香,拜了四拜,說道:「我的娘,今日龐大姐特來與你燒陌紙錢,你好䖏生天,苦䖏用錢。早知你死在仇人之手,奴隨問怎的,也娶來府中,和奴做一䖏。還是奴躭悞了你,悔已是遲了!」說畢,令左右把紙錢燒了。這春梅向前放聲大哭,有哭〔山坡羊〕為證:

\begin{myquote}
「燒罷紙,把鳳頭鞋跌綻。呌了聲娘,把我肝腸兒呌斷。只因你逞風流,人多惱你疾發你出去。被仇人纔把你命兒坑陷。奴在深宅怎得個自然?又無親,誰把你掛牽?實指望和你同牀兒共枕,怎知道你命短無常,死的好可憐!呌了聲,不睜眼的青天!常言道好物難全,紅羅尺短。」
\end{myquote}

這裏春梅在金蓮坟上祭祀哭泣,不題。卻說吳月娘在僧房内,只知有宅内小夫人來到,長老出去山門迎接,又不見進來。問小和尚,小和尚說:「這寺後有小奶奶的一個姐姐,新近葬下,今日清明節,特來祭掃燒紙。」孟玉樓便道:「怕不就是春梅來了,也不定的。」月娘道:「他又那得個姐來,死了葬在此處?」又問小和尚:「這府裏小夫人姓甚麽?」小和尚道:「姓龐氏。前日與了長老四五兩經錢,敎替他姐姐念經,薦拔生天。」玉樓道:「我聽見爹說,春梅娘家姓龐,呌龐大姐,莫不是他?」正說話,只見長老先走來,吩咐小沙彌:「快看好茶。」不一時轎子擡進方丈二門裏纔下轎。月娘和玉樓衆人打僧房簾内望外張看怎樣的小夫人。定睛仔細看時,卻是春梅!但比昔時出落長大身材,面如滿月,打扮的粉粧玉琢,頭上戴着冠兒,珠翠堆滿,鳳釵半卸,穿大紅粧花襖兒,下着翠藍縷金寬襴裙子,带着玎璫禁步,比昔不同許多。但見:

\begin{myquote}
寳髻巍峨,鳳釵半卸。胡珠環耳邊低掛,金挑鳳鬢後雙插。紅綉襖偏襯玉香肌,翠紋裙下映金蓮小。行動䖏,胸前搖響玉玎璫;坐下時,一陣麝蘭香噴鼻。膩粉粧成脖頸,花鈿巧貼眉尖。擧止驚人,貌比幽花殊麗;姿容閒雅,性如蘭蕙温柔。若非綺閣生成,定是蘭房長就。儼若紫府瓊姬離碧漢,蕊宫僊子下塵寰。
\end{myquote}

那長老一面掀簾子,請小夫人方丈明間内坐,上面獨獨安放一張公座椅兒。春梅坐下,長老參見已畢,小沙彌拿上茶。長老遞茶上去,說道:「今日小僧不知宅内上坟,小奶奶來這裏祭祀,有失迎接,恕罪小僧。」春梅道:「外日多有起動長老,誦經追薦。」那和尚没口子說:「小僧豈敢。有甚殷勤補報恩主?多蒙小奶奶賜了許多經錢襯施,小僧請了八衆禪僧,整做道場,看經禮懺一日,晚夕又多與他老人家裝些廂庫焚化。道場圓滿,纔打發兩位管家進城,宅裏囘小奶奶話。」春梅喫了茶,小和尚接下鍾盞來。長老只顧在旁一遞一句與春梅說話,把吳月娘衆人攔阻在内,又不好出來的。月娘恐怕天晚,使小和尚請下長老來要起身。那長老又不肯放,走來方丈禀春梅說:「小僧有件事,稟知小奶奶。」春梅道:「長老有話,但説無妨。」長老道:「適間有幾位遊翫娘子,在寺中隨喜,不知小奶奶來。如今他要囘去,未知小奶奶尊意如何。」春梅道:「長老何不請來相見?」那長老慌的來請。吳月娘又不肯出來。只說:「長老,不見罷。天色晚了,俺們告辭去罷。」長老見收了他布施,又沒管待,又意不過,只顧再三催促。吳月娘與孟玉樓吳大妗子推阻不過,只得出來。春梅一見便道:「原來是二位娘與大妗子!」於是先讓大妗子轉上,花枝招颭磕下頭去。慌的大妗子還禮不迭,説道:「姐姐今非昔日比,折殺老身!」春梅道:「好大妗子,如何說這話?奴不是那樣人!尊卑上下,自然之理。」拜了大妗子,然後向月娘孟玉樓插燭也似磕頭去。月娘玉樓亦欲還禮,春梅那裏肯,扶起磕了四個頭說:「不知娘們在這裏,早知也請出來相見。」月娘道:「姐姐,你自従出了家門,在府中一向奴多缺禮,沒曾看你,你休怪。」春梅道:「好奶奶,奴那裏出身,豈敢說怪?」因見奶子如意兒抱着孝哥兒,説道:「哥哥也長的恁大了。」月娘説:「你和小玉過來,與姐姐磕個頭兒。」那如意兒和小玉二人,笑嘻嘻過來,亦與春梅都平磕了頭。月娘道:「姐姐,你受他兩個一禮兒。」春梅向頭上拔下一對金頭銀簪兒來,插在孝哥兒帽兒上。月娘説:「多謝姐姐簪兒。還不與姐姐唱個喏兒?」如意兒抱着哥兒,眞個與春梅道了個喏,把月娘喜歡的了不得。玉樓説:「姐姐,你今日不到寺中,咱娘兒們怎得遇在一䖏相見?」春梅道:「便是。因俺娘他老人家,新埋葬在這寺後。奴在他手裏一場,他又無親無故,奴不記掛着替他燒張紙兒,怎生過得去?」月娘說:「我記的你娘沒了好幾年,不知葬在這裏。」孟玉樓道:「大娘還不知龐大姐說話!説的潘六姐死了,多虧姐姐如今把他埋在這裏。」月娘聽了,就不言語了。吳大妗子道:「誰似姐姐這等有恩,不肯忘舊,還葬埋了他,逢節令題念他,來替他燒錢化紙。」春梅道:「好奶奶,想着他怎生擡擧我來!今日他死的苦,是這般抛露丢下,怎不埋葬他?」説畢,長老敎小和尚放桌兒,擺齋上來。兩張大八僊桌子,蒸酥煠□餅饊點心,各樣素饌菜蔬,堆滿春臺。絶細金芽雀舌,甜水好茶。衆人喫了,收下家活去。吳大舅自有僧房管待,不在話下。

孟玉樓起身,心裏要往金蓮坟上看看,替他燒張紙,也是姊妹一場。見月娘不動身,拿出五分銀子,教小沙彌買紙去。長老道:「娘子不消買去,我這裏有金銀紙,拿幾分燒去。」玉樓把銀子遞與長老,使小沙彌領到後邊白楊樹下金蓮坟上,見三尺坟堆,一堆黄土,數柳青蒿。上了根香,把紙錢點着,拜了一拜,説道:「六姐,不知你埋在這裏!今日孟三姐悞到寺中,與你燒陌錢紙,你好處生天,苦處用錢!」一面取出汗巾兒來,放聲大哭。有哭〔山坡羊〕為證:

\begin{myquote}
「燒罷紙,淚珠兒亂滴。呌六姐一聲,哭的奴一絲兒兩氣。想當初咱二人不分個彼此,做姊妹一場並無面紅面赤。你性兒強我常常兒的讓你,一面兒不見不是你尋我我就尋你。恰便像比目魚,雙雙熱黏在一處。忽被一陣風咱分開來嚛,共樹同栖,一旦各自去飛!呌了聲六姐,你試聽知:可惜你一段兒聰明,今日埋在土裏!」
\end{myquote}

那奶子如意兒見玉樓往後邊,也抱了孝哥兒來看了看。月娘在方丈内和春梅說話,敎奶子:「休抱了孩子去,只怕唬了他。」如意兒道:「奶奶不妨事,我知道。」徑抱到坟上,看玉樓燒紙哭罷囘來。春梅和月娘匀了臉,換了衣裳。吩咐小伴當將食盒打開,將各樣細菓甜食餚品點心攢盒,擺下兩桌子,布甑内篩上酒來,銀鐘牙筯,請大妗子月娘玉樓上坐,他便主位相陪。奶子小玉老姐兩邊打横。吳大舅另放一張桌子在僧房内。正飲酒中間,忽見兩個青衣伴當,走來跪下,禀道:「老爺在新莊,差小的來請小奶奶,看雜耍調百戯的。大奶奶二奶奶都去了,請奶奶快去哩!」這春梅不慌不忙,說:「你囬去,知道了。」那二人應諾下來,又不敢去,在下邊等候,且待他陪完。大妗子月娘便要起身,說:「姐姐,不再打攪!天色晚了,你也有事,俺們去罷。」那春梅那裏肯放,只顧令左右將大鍾來勸道:「咱娘兒們會少離多,彼此都見長着,休要斷了這門親路。奴也沒親没故,到明日娘好的日子,奴往家裏走走去!」月娘道:「我的姐姐,說一聲兒就夠了,怎敢起動你?容一日,奴去看姐姐去。」飲過一盃,月娘說:「我酒夠了。你大妗子沒轎子,十分晚了,不好行的。」春梅道:「大妗子沒轎子,我這裏有跟隨小馬兒,撥一疋與妗子騎,送了家去。」一面收拾起身。春梅呌過那長老來,令小伴當拿出一疋大布五錢銀子與長老。長老拜謝了,送出山門。春梅與月娘拜别,看着月娘玉樓衆人上了轎子,他也坐轎子,兩下分路,一簇人跟隨,喝着道往新莊上去了。正是:樹葉還有相逢䖏,豈可人無得運時。

畢竟未知後來如何,且聽下囬分解。

