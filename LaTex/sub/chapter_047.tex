\includepdf[pages={93,94},fitpaper=false]{tst.pdf}
\chapter*{第四十七囬 \\王六兒説事圖財 西門慶受贓枉法}
\addcontentsline{toc}{chapter}{第四十七囬 王六兒説事圖財 西門慶受贓枉法}
\markboth{{\titlename}卷之五}{第四十七囬 王六兒説事圖財 西門慶受贓枉法}


\begin{myquote}
風擁狂瀾浪正顛,孤舟斜泊抱愁眠。

離鴻叫徹寒雲外,驛鼓清分旅夢邊。

詩思有添池草緑,河船無約晚潮昇。

憑虚細數誰知己,惟有故人月在天。
\end{myquote}

此一首詩,單題塞北以車馬為常,江南以舟楫為便。南人乘舟,北人乘馬,蓋可信也。話説江南揚州廣陵城内,有一苗員外,名喚苗天秀。家有萬貫資財,頗好詩禮。年四十歲,身邊無子,止有一女,尚未出嫁。其妻李氏,身染痼疾在牀。家事盡託與寵妾刁氏,名喚刁七兒,原是揚州大馬頭娼妓出身,天秀用銀三百兩娶來家,納為側室,寵嬖無比。忽一日,有一老僧在門首化緣,自稱是東京報恩寺僧,因為堂中缺少一尊鍍金銅羅漢,故雲遊在此,訪善結緣。天秀聞之,不吝,即施銀五十兩與那僧人。僧人道:「不消許多,一半足以完備此像。」天秀道:「吾師休嫌少,除完佛像,餘剩可作齋供。」那僧人問訊致謝,臨行,向天秀説道:「員外左眼眶下有一道白氣,乃是死氣,主不出半年,當有大災殃。你有如此善緣與我,貧僧焉可不預先説與你知?今後隨有甚事,切勿出境。戒之,戒之!」言畢,作辭天秀而去。

那消半月,天秀偶遊後園,見其家人苗青,——平日是個浪子,正與刁氏在亭側相倚私語,不意天秀猝至,躱避不及。看見,不由分説,將苗青痛打一頓,誓欲逐之。苗青恐懼,轉央親鄰,再三勸留得免,終是記恨在心。不期有天秀表兄黄羙,原是揚州人氏,乃擧人出身,在東京開封府做通判,亦是博學廣識之人也。一日,差人寄了一封書來揚州與天秀,要請天秀上東京,一則遊翫,二者為謀其前程。苗天秀得書,不勝歡喜,因向其妻妾説道:「東京乃輦轂之地,景物繁華所萃,吾心久欲遊覽,無由得便。今不期表兄書來相招,實有以大慰平生之意。」其妻李氏便説:「前日僧人相你面上有災厄,囑你不可出門。且此去京都甚遠,況你家私沉重,抛下幼女病妻在家,未審此去前程如何,不如勿徃為善。」天秀不聽,反加怒叱,説道:「大丈夫生于天地之間,桑弧蓬矢,不能遨遊天下,觀國之光,徒老死牖下無益矣!況吾胸中有物,囊有餘資,何愁功名之不到手?此去表兄必有羙事於我,切勿多言!」天秀於是吩咐家人苗青收拾行李衣裝,多打點兩箱金銀,載一船貨物,帶了個安童,並苗青,來上東京,取功名如拾芥,得羙職猶唾手。遺囑妻妾守家,擇日起行。

正値秋末冬初之時,従揚州馬頭上船,行了數日,到徐州洪,但見一派水光,十分險惡:

\begin{myquote}
萬里長洪水似傾,東流海島若雷鳴;

滔滔雪浪令人怕,客旅逢之誰不驚!
\end{myquote}

前過地名陝灣,苗員外看見天晚,命舟人泊住船隻。也是天數將盡,合當有事,不料搭的船隻,卻是賊船,兩個艄子皆是不善之徒。一個姓陳,名喚陳三,一個姓翁,乃是翁八。常言道:不着家人,弄不得家鬼。這苗青深恨家主苗天秀,日前被責之仇,一向要報無由,口中不言,心内暗道:「不如我如此如此,這般這般,與兩個艄子做一路,難得將家主害了性命,推在水内,盡分其財物。我這一回去,再把病婦謀死。這分家私,連刁氏都是我情受的。」正是:花枝葉下猶藏刺,人心怎保不懷毒!這苗青由是與兩個艄子密密商量説道:「我家主皮箱中還有一千兩金銀,二千兩緞疋,衣服之類極廣。汝二人若能謀之,願將此物均分。」陳三翁八笑道:「汝若不言,我等不瞞你説,亦有此意久矣!」是夜天氣陰黑,苗天秀與安童在中艙睡,苗青在艪後。將近三鼓時分,那苗青故意連叫有賊。苗天秀従夢中驚醒,便探頭出艙外觀看,被陳三手持利刀,一下剌中脖下,推在洪波蕩裏。那安童正要走時,乞翁八一悶棍打落於水中。三人一面在船艙内打開箱籠,取出一應財帛金銀並其緞貨衣服,點數均分。二艄便説:「我等若留此貨物,必然有犯。你是他手下家人,載此貨物到於市店上發賣,沒人相疑。」因此二艄盡把皮箱中一千兩金銀並苗員外衣服之類分訖,依前撑船囬去了。這苗青另搭了船隻,載至臨清馬頭上,鈔關上過了税,裝到清河縣城外官店内卸下。見了揚州故舊商家,只説:「家主在後船,便來也。」這個苗青在店發賣貨物不題。

常言人便如此如此,天理未然未然。可憐苗員外平昔良善,一旦遭其従僕之害,不得好死。雖則是不納忠言之勸,其亦大數難逃。不想安童被艄子一棍打昏,雖落水中,幸得不死,浮沒蘆港,得上岸來,在於堤邊號泣連聲。看看天色微明之時,忽見上流有一隻漁船撑將下來。船上坐着個老翁,頭頂箬笠,身披短簑。只聽得岸邊蘆荻深䖏有啼哭,移船過來看時,卻是一個十七八歲小廝,滿身是水。問其始末情由,卻是揚州苗員外家童在洪上被劫之事。這漁翁帶下船,撑回家中,取衣服與他換了,給予飲食。因問他:「你要囬去乎?卻同我在此過活?」安童哭道:「主人遭難,不見下落,如何囬得家去?願隨公公在此。」漁翁道:「也罷,你且隨我在此,等我慢慢替你訪此賊人是誰,再作理會。」安童拜謝公公,遂在此翁家過其日月。

一日,也是合當有事。年除歲末,漁翁忽帶安童正出河口賣魚,正撞見陳三翁八在船上飲酒,穿着他主人衣服,上岸來買魚。安童認得,即密與漁翁説道:「主人之寃當雪矣!」漁翁道:「如何不具狀官司處告理?」當下領安童將情具告到巡河周守備府内,守備見沒贜證,不接狀子。又告到提刑院,夏提刑見是強盜劫殺人命等事,把狀批行了。従正月十四日,差緝捕公人,押安童下來拏人。前至新河口,把陳三翁八獲住,到於案,責問了口詞。二艄見安童在傍執證,也沒得動刑,一一招承了,供稱:「下手之時,還有他家人苗青同謀,殺其家主,分贓而去。」這裏把三人監下,又差人訪拏苗青,拏到一起定罪。因節間放假,提刑官吏一連兩日沒來衙門中問事。早有衙門首透信兒的人,悄悄把這件事兒報與苗青。苗青慌了,把店門鎖了,暗暗躱在經紀楽三家。

這楽三就在獅子街石橋西首,韓道國家隔壁,門面一間,到底三層房兒居住。他渾家樂三嫂,與王六兒所交極厚,常過王六兒這邊來做伴兒坐。王六兒無事,也常徃他家行走,彼此打的熱鬧。這楽三見苗青面帶憂容,問其所以。説道:「不打緊,間壁韓家,就是提刑西門老爹的外室,又是他家夥計,和俺家交徃的甚好,凡事百依百隨。若要保得你無事,破多少東西,教俺家過去和他家説説。」這苗青聽了,連忙就下跪説道:「但得除豁了我身上沒事,恩有重報,不敢有忘!」於是寫了説帖,封下五十兩銀子,兩套粧花緞子衣服。楽三教他老婆㧱過去,如此這般,對王六兒説。王六兒喜歡的了不的,把衣服和銀子並説帖都收下。單等西門慶,不見來。

到十七日日西時分,只見玳安夾着毡包,騎着頭口,従街心裏來。王六兒在門首叫下來問道:「你徃那裏去來?」玳安道:「我跟了爹走了個遠差,徃東平府送禮去來。」王六兒道:「你爹如今在那裏,來了不曾?」玳安道:「爹和賁四先徃家去了。」王六兒便叫進去,和他如此這般説話,拏帖兒與他瞧。玳安道:「韓大嬸,管他這事?休要把事輕看了。如今衙門裏監着那兩個船家,供着只要他哩。拏這幾兩銀子來,也不夠打發脚下人的哩。我不管別的帳。韓大嬸和他説,只與我二十兩銀子罷!等我請將俺爹來,隨你老人家與俺爹説就是了。」王六兒笑道:「怪油嘴兒,要飯喫,休要惡了火頭!事成了,你的事甚麽打緊?寜可我們不要,也少不了你的。」玳安道:「韓大嬸,不是這等説。常言:君子不羞當面。先断過,後商量。」王六兒當下預備幾樣菜,㽞玳安喫酒。玳安道:「喫的紅頭紅臉,咱家去爹問,卻怎的囬爹?」王六兒道:「怕怎的?你就説在我這裏來。」於是玳安只喫了一甌子就走了。王六兒道:「你到家好歹累你説,我這裏等着哩。」

玳安一直上了頭口來家,交進毡包後邊,立等的西門慶房中睡了一覺出來,在廂房中坐的。這玳安慢慢走到跟前附耳説:「小的囬來,韓大嬸叫住小的,要請爹快些過去,有句要緊話和爹説。」西門慶説:「甚麽話?——我知道了。」説時,正値劉學官來借銀子,打發劉學官去了,西門慶騎馬,帶着眼紗小帽,便叫玳安琴童兩個跟隨,來到王六兒家,下馬進去,到明間客位坐下。王六兒出來拜見了。那日韓道國因前邊舖子裏該上宿,沒來家。老婆買了許多東西,叫老馮廚下整治,等候西門慶。一面丫鬟錦兒拏茶上來,婦人遞了茶。西門慶吩咐琴童把馬送到對門房子裏去,把大門關上。婦人且不敢就題此事,先只説:「爹家中連日擺酒辛苦。我聞得説哥兒定了親事,你老人家喜呀!」西門慶道:「只因舍親吳大妗那裏説起,和喬家做了這門親事。他家也只這一個女孩兒。論起來也還不搬陪,胡亂親上做親罷了。」王六兒道:「就是和他做親也好,只是爹如今居着恁大官,會在一處,不好意思的。」西門慶道:「説甚麽哩!」説了一囬,老婆道:「只怕爹寒冷,徃房裏坐去罷。」一面讓至房中,一面安着一張椅兒,籠着火盆,西門慶坐下。婦人慢慢先把苗青揭帖拏與西門慶看,説:「他央了間壁經紀楽三娘子過來對我説,這苗青是他店裏客人,如此這般,被兩個船家拽扯,只望除豁了他這名字,免提他。他備了些禮兒在此謝我,好歹望老爹怎的將就他罷。」西門慶看了帖子,因問:「他拏了那禮物謝你?」王六兒向箱中取出五十兩銀子來與西門慶瞧,説道:「明日事成,還許兩套衣裳。」西門慶看了笑道:「這些東西兒,平白你要他做甚麽?你不知道,這苗青乃揚州苗員外家人,因為在船上與兩個船家商議,殺害家主,攛在河裏,圖財謀命。如今現打撈不着屍首。又當官兩個船家招尋他,原跟來的一個小廝安童,又當官三口執證着要他。這一拏過去,穩定是個凌遲罪名。那兩個,都是眞犯斬罪。兩個船家現供他有二千兩銀貨在身上。拏這些銀子來做甚麽?還不快送與他去。」這王六兒一面到廚下使了丫頭錦兒,把楽三娘子兒呌了來,將原禮交付與他,如此這般對他説了去。

那苗青不聽便罷,聽他説了,猶如一桶水頂門上直灌到脚底下。正是:驚駭六葉連肝膽,唬壞三魂七魄心。即請楽三一處商議道:「寜可把二千貨銀都使了,只要救得性命家去。」楽三道:「如今老爹上邊即發此言,一些半些,恆屬打不動兩位官府,須得凑一千貨物與他。其餘節級原解緝捕再得一半,纔得夠用。」苗青道:「况我貨物未賣,那討銀子來?」因使過楽三嫂來和王六兒説:「老爹就要貨物,發一千兩銀子貨與老爹。如不要,伏望老爹再寬限兩三日,等我倒下價錢,將貨物賣了,親徃老爹宅裏進禮去。」王六兒拏禮帖復到房裏與西門慶瞧。西門慶道:「既是恁般,我吩咐原解且寬限他幾日拏他,教他即便進禮來。」當下楽三娘子得此口詞,囬報苗青,苗青滿心歡喜。

西門慶見間壁有人,也不敢久坐,喫了幾鍾酒,與老婆坐了囬房,見馬來接,就起身家去了。次日,到衙門早發放,也不提問這件事。吩咐緝捕:「你休捉這苗青。」苗青就託經紀楽三,連夜替他會了人,攛掇貨物出去。那消三日,都發盡了,共賣了一千七百兩銀子。把原與王六兒的不動,另加五十兩銀子,又另送他四套上色衣服。

且説十九日,苗青打點一千兩銀子,裝在四個酒壜内,又宰一口猪,約掌燈巳後時分,擡送到西門慶門首。手下人都是知道的。玳安平安書童琴童四個禁子,與了十兩銀子纔罷。玳安在王六兒這邊,梯己又要十兩銀子。須臾,西門慶出來,捲棚内坐的,也不掌燈,月色朦朧纔上來,擡至當面,苗青穿青衣,望西門慶只顧磕着頭,説道:「小人蒙老爹超拔之恩,粉身碎骨,死生難報!」西門慶道:「你這件事情,我也還沒好審問哩。那兩個船家甚是攀你。你若出官,也有老大一個罪名。旣是人説,我饒了你一死。此禮我若不受你的,你也不放心。我還把一半送你掌刑夏老爹,同做分上。你不可久住,即便星夜囬去。」因問:「你在揚州那裏?」苗青磕頭道:「小的在揚州城内住。」西門慶吩咐後邊拏了茶來。那苗青在松樹下立着喫了,磕頭告辭囬去。又叫回來問:「下邊原解的,你都與他説了不曾説?」苗青道:「小的外邊已説停當了。」西門慶吩咐:「旣是説了,你即回家。」那苗青出門,走到楽三家收拾行李,還剩一百五十兩銀子。苗青拏出五十兩來,並餘下幾疋緞子,都謝了楽三夫婦。五更替他僱長行牲口,起身徃揚州去了。正是:忙忙如喪家之狗,急急似漏網之魚。

不説苗青逃出性命,不題。單表西門慶夏提刑従衙門中散了出來,並馬而行。走到大街口上,夏提刑要作辭分路。西門慶在馬上擧着馬鞭兒説道:「長官不棄,降到舍下一叙。」把夏提刑邀到家來。門首同下了馬,進到廳上敍禮,請入捲棚内寬了衣服,左右拏茶上來喫了。書童玳安上來,安放桌席擺設。夏提刑道:「不當閒來打攪長官。」西門慶道:「豈有此理。」須臾,兩個小廝用方盒拏了小菜,就在傍邊擺下各樣鷄蹄鵝鴨鮮魚下飯,就是十六碗。喫了飯,收了家伙去,就是喫酒的各樣菜蔬出來,小金把鍾兒,銀臺盤兒,金鑲象牙筯兒。飲酒中間,西門慶慢慢提起苗青的事來:「這廝昨日央及了個士夫,再三來對學生説,又餽送了些禮在此。學生不敢自專,今日請長官來,與長官計議。」於是把禮帖遞與夏提刑。夏提刑看了,便道:「任憑長官尊意裁䖏。」西門慶道:「依着學生,明日只把那個賊人眞贓送過去罷,也不消要這苗青。那個原告小廝安童,便收領在外,待有了苗天秀屍首,歸給未遲。禮還送到長官䖏。」夏提刑道:「長官此意就不是了。長官見得極是,此是長官費心一場,何得見讓於我?决然使不得!」彼此推辭了半日,西門慶不得已,還把禮物兩家平分了,裝了五百兩在食盒内。夏提刑下席來忙作揖謝道:「既是長官見愛,我學生再辭,顯的迂闊了。盛情感激不盡,實為多愧!」又領了幾盃酒,方纔告辭起身。這裏西門慶隨即就差玳安拏了盒,還當酒擡送到夏提刑家。夏提刑親在門上收了,拏回帖,又賞了玳安二兩銀子,兩名排軍四錢,俱不在話下。

常言道:火到猪頭爛,錢到公事辦。且説西門慶夏提刑已是會定了,次日到衙門裏陞廳,那提控節級並緝捕觀察,都被楽三替苗青上下打點停當了。擺設下刑具,監中提出陳三翁八,審問情由,只是供稱:「跟伊家人苗青同謀。」西門慶大怒,喝令:「左右與我用起刑來!你兩個賊人,專一積年在江河中假以舟楫裝載為名,實是劫幫鑿漏,邀截客旅,圖財致命。現有這個小廝供稱,是你等持刀戮死苗天秀波中,又將棍打傷他落水。現有他主人衣服存證,你如何抵賴別人?」因把安童提上來,問道:「是誰刺死你主人,推在水中來?」安童道:「某日夜至三更時分,先是苗青呌有賊,小的主人出船艙觀看,被陳三一刀戮死,推在水中來。小的便被翁八一棍打落水中,纔得逃出性命。苗青並不知下落。」西門慶道:「據這小廝所言,就是實話。汝等如何展轉得過?」於是每人兩夾棍、三十榔頭,打的脛骨皆碎,殺猪也似叫動。他一千兩贓貨已追出大半。餘者花費無存。這裏提刑連日做了文書,點過贓貨,申詳東平府。府尹胡師文,又與西門慶相交,照依原行文書,疊成案卷,將陳三翁八問成強盜殺人斬罪。只把安童保領在外聽候。——有日安童走到東京,投到開封府黄判通衙内,具訴苗青情奪了主人家事,「使錢提刑,除了他名字出來。主人寃讐,何時得報?」黄通判聽了,連夜修書,並他訴狀封在一䖏,與他盤費,就着他徃巡按山東察院裏投下。這一來,管教苗青之祸,從頭上起,西門慶徃時做過事,今朝沒興一齊來!有詩為證:

\begin{myquote}
善惡従來畢有因,吉兇祸福並肩行。

平生不作虧心事,夜半敲門不喫驚!
\end{myquote}

畢竟未知後來何如,且聽下囬分解。

