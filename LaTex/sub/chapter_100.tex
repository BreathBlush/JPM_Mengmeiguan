\includepdf[pages={199,200},fitpaper=false]{tst.pdf}
\chapter*{第一百囬 \\韓愛姐湖州尋父 普靜師薦拔羣寃}
\addcontentsline{toc}{chapter}{第一百囬 韓愛姐湖州尋父 普靜師薦拔羣寃}
\markboth{{\titlename}卷之十}{第一百囬 韓愛姐湖州尋父 普靜師薦拔羣寃}


格言
\begin{myquote}
人生切莫恃英雄,術業精粗自不同。

猛虎尚然遭惡獸,毒蛇猶自怕蜈蚣。

七擒孟獲奇諸葛,兩困雲長羡呂蒙。

珍重李安眞智士,高飛逃出是非門。
\end{myquote}

話説韓道國與王六兒歸到謝家酒店内,無女兒,道不得個坐喫山崩,使陳三兒去又把那何官人勾來續上。那何官人見地方中沒了劉二,除了一害,依舊又來王六兒家行走。和韓道國商議:「你女兒愛姐,已是在府中守孝,不出來了。等我賣盡貨物,討了賒帳,你兩口跟我往湖州家去罷,省得在此做這般道路。」那韓道國說:「官人下顧,可知好哩!」一日賣盡了貨物,討上賒賬,僱了船,同王六兒跟往湖州去了。

卻表愛姐在府中,與葛翠屏兩個持貞守節,姊妹稱呼,甚是合當着。白日裏與春梅做伴兒在一䖏。那時金哥兒大了,年方六歲;孫二娘所生玉姐,年長十歲;相伴兩個孩兒,便沒甚事做。誰知自従陳經濟死後,守備又出征去了,這春梅每日珍饈百味,綾錦衣衫,頭上黄的金,白的銀,圓的珠,光的寳,無般不有,只是晚夕難禁獨眠孤枕,慾火燒心。因見李安一條好漢,又因打殺張勝,巡風早晚十分小心,有意勾搭。一日,冬月天氣,李安正在班房内上宿,忽聽有人敲後門,忙問道:「是誰?」只聞叫道:「你開門則個。」李安連忙開了房門,卻見一個人搶入來,閃身在燈光背後。李安看時,卻認的是養娘金匱。李安道:「養娘,你這早晚來有甚事?」金匱道:「不是我私來,裏邊奶奶差出我來的。」李安道:「奶奶教你來怎麽?」金匱笑道:「你好不理會得!看你睡了不曾,教我把一件物事來與你。」向背上取下一包衣服:「把與你!包内又有幾件婦女衣服,與你娘。前日多累你押解老爺行李車輛,又救得奶奶一命,不然,也乞張勝那廝殺了。」說畢,留下衣服出門。走了兩步,又囘身道:「還有一件要緊的!」又取出一錠五十兩大元寳來,撇與李安,自去了。

當夜過了一宿,次早起來,逕拿衣服到家與他母親。做娘的問道:「這東西是那裏的?」李安把夜來事說了一遍。做母的聽言呌苦:「當初張勝幹壞了事,一百棍打死,他今日把東西與你,卻是甚麽意思?我今六十以上年紀,自従沒了你爹爹,滿眼只看着你。若是做出事來,老身靠誰?明早便不要去了。」李安道:「我不去,他使人來叫,如何答應?」婆婆說:「我只說你感冒風寒病了。」李安道:「終不成不去,惹老爺不見怪麽?」做娘的便說:「你且投到你叔叔山東夜叉李貴那裏,住上幾個月,再來看事故何如。」這李安終是個孝順的男子,就依着娘的話,收拾行李,往青州府投他叔叔李貴去了。春梅以後見李安不來,三囘五次使小伴當來叫。婆婆初時答應家中染病,次後見人來驗看,纔說往原籍家中討盤纏去了。這春梅終是惱恨在心,不題。

時光迅速,日月如梭,又早臘月盡陽日囬,正月初旬天氣。統制領兵一萬二千,在東昌府屯住已久,使家人周忠捎書來家,教搬取春梅孫二娘並金哥玉姐家小上車,止留下周忠:「東莊上請你二爺看守宅舍。」原來統制還有個族弟周宣在莊上住。周忠在府中,與周宣葛翠屏韓愛姐看守宅舍。周仁與衆軍牢保定車輛,往東昌府來。這此一去,不為身名離故土,爭知此去少囬程。有詞一篇單道這周統制果然是一員好將材,當此之時,中原板蕩,志欲吞胡。但見:
\begin{myquote}
四方盜起如屯蜂,狼煙烈焰薰天紅。

將軍一怒天下息,腥膻掃盡夷従風。

公爾忘私願已久,此身許國不知有。

金戈抑日酬戰征,麒麟圖畫功為首。

鴈門關外秋風烈,鐵衣披張臥寒月。

汗馬辛勤二十年,贏得斑斑鬢如雪。

天子明見萬里餘,幾番勞勣來旌書。

肘懸金印大如斗,無負堂堂七尺軀。
\end{myquote}

有日周仁押家眷車輛到於東昌。統制見了春梅孫二娘金哥玉姐衆丫鬟家小都到了,一路平安,心中大喜,就在統制府衙後廳居住。周仁悉把「東莊上叫了二爺周宣來宅,同小的老子周忠看守宅舍」,說了一遍。周統制又問:「怎的李安不見?」春梅道:「又題甚李安!那廝我因他捉獲了張勝,好意賞了他兩件衣服與他娘穿。他到晚夕巡風,進入後廳,把他二爺東莊上收的籽粒銀一包五十兩,放在明間桌上,偸的去了。幾番使伴當呌他,只是推病不來。落後又使人呌去,他躱的上青州原籍家去了。」統制便道:「這廝我倒看顧他,原來這等無恩!等我慢慢差人拿他去。」這春梅不題起韓愛姐之事。過了幾日,春梅見統制日逐理論軍情,幹朝庭國務,焦心勞思,日中尚未暇食,至於房幃色慾之事,久不霑身。因見老家人周忠次子周義,年十九歲,生的眉清目秀。眉來眼去,兩個暗地私通,就勾搭上了。朝朝暮暮,兩個在房中下棋飲酒,只瞞過統制一人不知。

一日,不想北國大金皇帝滅了遼國,又見東京欽宗皇帝登基,集大勢番兵,分兩路寇亂中原:大元帥黏沒喝,領十萬人馬,出山西太原府井陘道,來搶東京;副元帥斡離不,由檀州來搶高陽關。邊兵抵擋不住,慌了兵部尚書李綱,大將種師道,星夜火牌羽書\endnote{火牌羽書——火牌,明代軍中符信之一。兵丁至各地傳命令,憑此牌向各驛站支領口糧。羽書,徵調軍隊的文書上插有鳥羽,以示緊急。},分調山東山西河南河北關東陝西,分六路統制人馬,各依要地防守截殺。那時陝西劉延慶,領延綏之兵;關東王禀,領汾絳之兵;河北王煥,領魏博之兵;河南辛興宗,領彰德之兵;山西楊惟忠,領澤潞之兵;山東周秀,領青兖之兵。卻說周統制見大勢番兵來搶邊界,兵部羽書火牌星火來催,連忙整率人馬,全裝披掛,兼道進兵。比及哨馬到高陽關上,金國斡離不率人馬已搶進關來,殺死人馬無數。正値五月初旬,交陣堵截,黄沙四起,大風迷目。統制提兵進趕,不防被斡離不兜馬反攻,沒鞦一箭,正射中咽喉,墮馬而死。衆番將就用鉤索搭去。被這邊將士向前,僅搶屍首馬載而還。所傷軍兵無數。可憐周統制一旦陣亡,亡年四十七歲。正是:捨家為國忠良將,不辨賢愚血染沙。古人意不盡,作詩一首以嘆之曰:
\begin{myquote}
勝敗兵家不可期,安危端自命為之。

出師未捷身先丧,落日江流不勝悲。
\end{myquote}

又〔鷓鴣天〕一首:
\begin{myquote}
定國安邦羙丈夫,心存正道氣吞胡。

謨謀國事如家事,運用《陰符》佩虎符。

胡騎盛,武功弛,兵不用命將驕癡。

可憐身死沙場内,千載英魂恨未舒。
\end{myquote}

巡撫張叔夜,見統制折於陣上,連忙鳴金收軍,查點折傷士卒,退守東昌,星夜奏朝廷,不在話下。部下卒載屍首還到東昌府,春梅合家大小號哭動天,合棺木盛殮,交割了兵符印信。一日,春梅與家人周仁,發丧載靈柩歸清河縣不題。

話分兩頭,單表葛翠屏與韓愛姐,自従春梅去後,兩個在家清茶淡飯,守節持貞,過其日月。正値春盡夏初天氣,景物鮮明。日長針指困倦,姊妹二人,閒中徐步到西書院花亭上。見百花盛開,鶯啼燕語,觸景傷情。葛翠屏心還坦然;這韓愛姐一心只想念男兒陳經濟大官人,凡事無情無緒,睹物傷悲。口是心苗,形吟咏者,有詩數首為證。

翠屏先道:
\begin{myquote}
「花開靜院日初晴,深鎖重門白晝清。

倒倚銀屏春睡醒,綠槐枝上一聲鶯。」
\end{myquote}

愛姐道:
\begin{myquote}
「春事闌珊首夏時,弓鞋欵欵出簾遲。

晚來悶倚粧臺立,巧畫蛾眉為阿誰?」
\end{myquote}

翠屏又道:
\begin{myquote}
「紅綿掩鏡照窻紗,畫就雙蛾八字斜。

蓮步輕移何䖏去,堦前笑折石榴花。」
\end{myquote}

愛姐道:
\begin{myquote}
「雪為容貌玉為神,不遣風流涴此身。

顧影自憐還自惜,新粧好好為何人?」
\end{myquote}

翠屏道:
\begin{myquote}
「莎草連綿厚似毡,楡莢遍地亂如錢。

誰知蕩子多輕薄,沉醉終朝花下眠。」
\end{myquote}

愛姐道:
\begin{myquote}
「亂愁依舊鎖翠峯,為甚年來憔悴容?

離别終朝魂耿耿,碧霄無路得相逢。」
\end{myquote}

姊妹兩個吟詩已畢,不覺潸然淚下。二爺周宣走來勸道:「你姊妹兩個少要煩惱,須索解嘆着過罷。我連日做得夢,有些不吉。夢見一張弓,掛在旗竿上,旗竿折了。不知是兇是吉?」韓愛姐道:「倒只怕老爺邊上有些說話。」正在猶疑之間,忽見家人周仁,掛着一身孝,慌慌張張走來,報道:「祸事!老爺如此這般,五月初七日在邊關上陣亡了。大奶奶二奶奶家眷載着靈車都來了。」慌了二爺周宣,收拾打掃前廳乾淨,停放靈柩,擺下祭祀,合家大小哀號起來。一面做齋累七,僧道念經。金哥玉姐披麻带孝,弔客往來,擇日出殯,安葬於祖塋,俱不必細說。

卻說二爺周宣,引着六歲金哥兒,行文書申奏朝廷,討祭葬,襲替祖職。朝廷明降,兵部覆題引奏:「已故統制周秀,奮身報國,沒於王事,忠勇可嘉。遣官諭祭一壇,墓頂追封都督之職。伊子照例優養,出幼襲替祖職。」

這春梅在内頤養\endnote{頤養——保養。}之餘,淫情愈盛,常留周義在香閣中,鎭日不出。朝來暮往,淫慾無度,生出骨蒸癆病症。逐日喫薬,减了飲食,消了精神,體瘦如柴,而貪淫不已。一日,過了他生辰,到六月伏暑天氣,早晨晏起,不料他摟着周義在牀上,一泄之後,鼻口皆出凉氣,淫津流下一窪窪\endnote{窪窪——量詞。表示聚集地液體數。},就嗚呼哀哉,死在周義身上,亡年二十九歲。這周義見没了氣兒,就慌了手脚,向箱内抵盗了些金銀細軟,带在身邊,逃走在外。丫鬟養娘不敢隱匿,報與二爺周宣得知。把老家人周忠鎖了,押着找尋周義。可霎作怪,正走在城外他姑娘家投住,一條索子拴將來。周宣已知其情,恐揚出醜去,金哥久後不好襲職,拿到前廳,不由分說,打了四十大棍,即時打死。把金哥與孫二娘看養。一面發喪於祖塋,與統制合葬畢。房中兩個養娘並海棠月桂,都打發各尋投向嫁人去了。止是葛翠屏與韓愛姐,再三勸他,不肯前去。

一日,不想大金人馬搶了東京汴梁,太上皇帝與靖康皇帝,都被擄上北地去了。中原無主,四下荒亂,兵戈匝地,人民逃竄,黎庶有塗炭之哭,百姓有倒懸之苦。大勢番兵已殺到山東地界,民間夫逃妻散,鬼哭神號,父子不相顧。葛翠屏巳被他娘家領去,各逃生命,止丢下韓愛姐,無處依倚,不免收拾行裝,穿着隨身慘淡衣衫,出離了清河縣,前往臨清找尋他父母。到臨清謝家店,店也關閉,主人也走了。不想撞見陳三兒。三兒說:「你父母去年時就跟了何官人,往江南湖州去了。」這韓愛姐一路上懷抱月琴,唱小詞曲,往前找尋父母。隨路饑餐渴飲,夜住曉行,忙忙如丧家之犬,急急似漏網之魚,弓鞋又小,萬苦千辛。行了數日,來到徐州地方。天色晚來,投在孤村裏面。一個婆婆,年紀七旬之上,頭綰兩道雪,鬢挽一窝絲,正在竃上杵米造飯。這韓愛姐便向前道了萬福,告道:「奴家是清河縣人氏,因為荒亂,前往江南投親,不期天晚,權借婆婆這裏投宿一宵。明早就行,房金不少。」那婆婆只顧觀看這女子,不是貧難人家婢女,生的擧止典雅,容貌非俗。但見:
\begin{myquote}
烏雲不整,惟思昔日家豪;眉歛遠山,為憶當年富貴。此夜月朦雲霧瑣,牡丹花被土沉埋。
\end{myquote}

婆婆道:「旣是投宿,娘子請炕上坐。等老身造飯,有幾個挑河伕子\endnote{挑河伕子——挖河的民工。}來喫。」那老婆婆炕上柴竃,登時做出一大鍋稗稻插荳子乾飯,又切了兩大盤生菜,撮上一把鹽。只見幾個漢子,都蓬頭精腿,褌褲兜襠,脚上黄泥流,進來放下荷筐鍬鐝,便問道:「老娘,有飯也未?」婆婆道:「你們自去盛喫。」當下各取飯菜,四散正喫。只見内一人,約三十四五年紀,紫面黄髮,便問婆婆:「這炕上坐的是甚麽人?」婆婆道:「此位娘子,是清河縣人氏,前往江南尋父母去。天晚在此投宿。」那人便問:「娘子,你姓甚麽?」愛姐道:「奴家姓韓,我父親名韓道國。」那人向前扯住問道:「姐姐,你不是我姪女韓愛姐麽?」那愛姐道:「你倒好似我叔叔韓二。」兩個抱頭相哭做一處。因問:「你爹娘在那裏?你在東京,如何至此?」這韓愛姐一五一十,従頭說了一遍:「因我嫁在守備府裏,丈夫没了。我守寡到如今。我爹娘跟了何官人往湖州去了,我要找尋去。荒亂中又沒人帶去,胡亂單身唱詞,覓些衣食前去。不想在這裏撞見叔叔!」那韓二道:「自從你爹娘上東京,我没營生過日,把房兒賣了,在這裏挑河做伕子,每日覓碗飯喫。旣然如此,我和你往湖州,尋你爹娘去。」愛姐道:「若是叔叔同去,可知好哩!」當下也盛了一碗飯,與愛姐喫。愛姐喫了一口,見粗飯不能下咽,只喫了半碗,就不喫了。

一宿晚景休題過。到次日天明,衆伕子都去了。韓二交納了婆婆房錢,領愛姐作辭出門,望前途所進。那韓愛姐本來嬌嫩,弓鞋又小,身邊帶着些細軟釵梳,都在路上零碎盤纏,將到淮安上船,迤邐望江南湖州來。非止一日,找尋到湖州何官人家,尋着父母,相會見了。不想何官人巳死,家中又沒妻小,止是王六兒一人,丢下六歲女兒,有幾頃水稻田地。不上一年,韓道國也死了。王六兒原與韓二舊有楂兒,就配了小叔,種田過日。那湖州有富家子弟,見韓愛姐生的聰明標致,多來求親。韓二再三教他嫁人,愛姐割髮毀目,出家為尼姑,誓不再配他人。後年至三十二歲,以疾而終。正是:貞骨未歸三尺土,怨魂先徹九重天。後韓二與王六兒成其夫婦,情受何官人家業田地,不在話下。

卻說大金人馬,搶過東昌府來,看看到清河縣地界。只見官吏逃亡,城門晝閉,人民逃竄,父子流亡。但見煙生四野,日蔽黄沙。封豕長蛇。互相吞併;龍爭虎鬥,各自爭強。皂幟紅旗,布滿郊野;男啼女哭,萬戶驚惶。番軍虜將,一似蟻聚蜂屯;短劍長鎗,好似森林密竹。一䖏䖏死屍骸骨,横三豎四;一攢攢折刀斷劍,七斷八截。個個㩦男抱女,家家閉戶關門。十室九空,不顯鄉村城郭;獐奔鼠竄,那存禮樂衣冠!正是得多少宫人紅袖泣,王子白衣行。那時西門慶家中吳月娘見番兵到了,家家都關鎖門戶,亂攛逃去,不免也打點了些金珠寳翫,带在身邊。那時吳大舅已死,止同吳二舅玳安兒小玉,領着十五歲孝哥兒,把家中前後都倒鎖了,要往濟南府投奔雲離守,一來那裏避兵,二者與孝哥完就其親事去。一路上只見人人慌亂,個個驚駭。可憐這吳月娘穿着隨身衣裳,和吳二舅男女五口,雜在人隊裏挨出城門,到於郊外,往前所行,到於空野十字路口。只見一個和尚,身披紫褐袈裟,手執九環錫杖,脚級芒鞋,肩上背着條布袋,袋内裹着經典,大移步迎將來,與月娘打了個問訊,高聲大叫道:「吳氏娘子,你到那裏去?還與我徒弟來!」唬的月娘大驚失色,說道:「師父,你問我討甚麽徒弟?」那和尚又道:「娘子,你休推睡裏夢裏!你曾記的十年前,在岱嶽東峯,被殷天錫赶到我山洞中投宿?我就是那雪洞老和尚,法名普靜。你許下我徒弟,如何不與我!」吳二舅便道:「師父出家人,如何你不近道理?此是荒亂年程,亂攛逃生,他有此孩兒,久後還要接代香火,他肯捨與你出家去?」和尚道:「你眞個不與我去?」吳二舅道:「師父,你休閒說,悞了人去路兒。後面只怕番兵來到,朝不保暮!」和尚道:「你旣不與我徒弟,如今天色已晚,也走不出路去。番人且來不到此䖏,你且跟我到這寺中歇一夜,明早去罷。」吳月娘問:「師父,是那寺中?」那和尚用手只一指道:「那路旁便是。」和尚引着,不想來到永福寺。吳月娘認的是永福寺,曾走過一遍。比及來到寺中,長老僧衆都走去大半,止有幾個禪和尚在後邊禪堂中打坐。佛前點着一大盞琉璃海燈,燒着一爐香。此時日啣山時分。但見:

\begin{myquote}
十字街熒煌燈火,九曜廟香靄鐘聲。一輪明月掛青天,幾點疎星明碧落。六軍營内,嗚嗚畫角頻吹;五鼓樓頭,點點銅壺正滴。四邊宿霧,紛紛罩舞榭歌臺;三市沉煙,隱隱閉綠窻朱戶。兩兩佳人歸繡閣,雙雙士子掩書帷。
\end{myquote}

當晚吳月娘與吳二舅玳安小玉孝哥兒,男女五口兒投宿在寺中方丈内,小和尚有認的,安排了些飯食與月娘等喫了。那普靜老師跏趺在禪堂牀上,敲木魚,口中念經。月娘與孝哥兒小玉在牀上睡,吳二舅和玳安做一䖏。着了慌亂辛苦了的人,都睡着了,止有小玉不曾睡熟,起來在方丈内打門縫内看那普靜老師父念經。看看念至三更時,只見金風凄凄,斜月朦朦,人煙寂靜,萬籟無聲。覷那佛前海燈,半明不暗。這普靜老師,見天下荒亂,人民遭劫,陣亡横死者數極多,發慈悲心,施廣惠力,禮白佛言世尊,唸解寃經咒。薦拔幽魂,解釋宿寃,絶去掛碍,各去超生,再無㽞滯。於是誦念了百十遍解寃經咒。少頃隂風凄凄,冷風颼颼,有數十輩焦頭爛額、蓬頭泥面者,或斷手折臂者,或有刳腹剜心者,或有無頭跛足者,或有弔頸枷鎖者,都來悟領禪師經咒,列於兩傍。禪師便道:「你等衆生,寃寃相報,不肯解脫,何日是了!汝當諦聽吾言,隨方托化去罷。偈曰:

\begin{myquote}
勸爾莫結冤,寃深難解結,

一日結成寃,千日解不徹。

若將恩報寃,如湯去潑雪。

若將寃報寃,如狼重見蝎。

我見結寃人,盡被寃磨折。

我今此懺悔,各把性悟徹。

照見本來心,寃愆自然雪。

仗此經力深,薦拔諸惡業。

汝當各托生,再勿將寃結。

改頭換面輪迴去,來世機緣莫再攀!」
\end{myquote}

當下衆人都拜謝而去。小玉竊看,都不認的。少頃又一大漢進來,身七尺,形容魁偉,全裝貫束,胸前關着一矢箭,自稱:「統制周秀,因與番將對敵,折於陣上。今蒙師薦拔,今往東京托生,與沈鏡為次子,名為沈守善去也。」言未已,又一人素體榮身,口稱是「清河縣富戶西門慶,不幸溺血而死。今蒙師薦拔,今往東京城内,托生富戶沈通為次子沈鉞去也。」小玉認的是他爹,唬的不敢言語。已而又有一人,提着頭,渾身皆血,自言是「陳經濟,因被張勝所殺。蒙師經功薦拔,今往東京城内與王家為子去也。」已而又見一婦人,也提着頭,胸前皆血,自言:「奴是武大妻、西門慶之妾潘氏是也,不幸被仇人武松所殺。蒙師薦拔,今往東京城内黎家為女,托生去也。」已而又有一人,身軀矮小,面皆青色,自言是「武植,因被王婆唆潘氏下毒薬喫而死。蒙師薦拔,今往徐州落鄉民范家為男,托生去也。」已而又有一婦人,面皮黄瘦,血水淋漓,自言:「妾身李氏。乃花子虚之妻、西門慶之妾,因害血山崩而死。蒙師薦拔,今往東京城内袁指揮家托生為女去也。」已而又一男,自言「花子虚,不幸被妻氣死。蒙師薦拔,今往東京鄭千戶家托生為男。」已而又見一女人,頸纏脚帶,自言「西門慶家人來旺妻宋氏,自縊身死。蒙師薦拔,今往東京朱家為女去也。」已而又一婦人,面黄ざ瘦,自稱「周統制妻龐氏春梅,因色癆而死。蒙師薦拔,今往東京與孔家為女,托生去也。」已而又一男子,裸形披髮,渾身杖痕,自言是「打死的張勝,蒙師父薦拔,今往東京大興衛貧人高家為男去也。」已而又有一女人,項上纏着索子,自言:「西門慶妾孫雪娥,不幸自縊身死。蒙師薦拔,今往東京城外貧民姚家為女去也。」已而又一女人,年小,項纏脚帶,自言:「西門慶之女、陳經濟之妻西門大姐是也,不幸自縊身死。蒙師薦拔,今往東京城外與番役鍾貴為女,托生去也。」已而又見一小男子,自言「周義,亦被打死。蒙師薦拔,今往東京城外高家為男,名高留住兒,托生去也。」言畢,各恍然不見。

小玉唬的戰慄不已:「原來這和尚,只是和這些鬼說話!」正欲向牀前告訴與月娘,不料月娘睡得正熟,一靈眞性\endnote{一靈眞性——即靈魂。},同吳二舅衆男女,身带着一百顆胡珠、一柄寳石縧環,前往濟南府投奔親家雲離守那裏避兵,就與孝哥完成親事。

一路饑食渴飲,夜住曉行,到於濟南府,問一老人:「雲參將住所在於何處?」老人指道:「此去二里餘地,名靈壁寨,一邊臨河,一邊是山。這靈壁寨就在城上,屯聚有一千人馬,雲參將就在那裏做知寨。」月娘五口兒到寨門,通報進去,雲參將聽見月娘送親來了,一見如故,叙畢禮數。原來新近没了娘子,央浼隣舍王婆婆來陪待月娘,在後堂酒飯,甚是豐盛。吳二舅玳安另在一䖏管待。因說起避兵來就親之事,因把那百顆胡珠寶石縧環,交與雲離守權為茶禮。雲離守收了,並不言其就親之事。到晚又教王婆陪月娘一䖏歇臥,將言說念月娘,以挑探其意,說:「雲離守雖是武官,乃讀書君子。従割衫襟之時,就留心娘子。不期夫人沒了,鰥居至今。今據此山城,雖是任小,上馬管軍,下馬管民,生殺在於掌握。娘子若不棄,願成伉儷之歡,一雙兩好。令郎亦得諧秦晉之配。等待太平之日,再囘家去不遲。」月娘聽言,大驚失色,半晌無言。這王婆囘報雲離守,次日晚夕,置酒後堂,請月娘喫酒。月娘只知他與孝哥兒完親,連忙來到席前叙坐。雲離守乃言:「嫂嫂不知,下官在此,雖是山城,管着許多人馬。有的是財帛衣服,金銀寳物,缺少一個主家娘子。下官一向思想娘子,如渴思漿,如熱思凉,不想今日娘子到我這裏與令郎完親,天賜姻緣,一雙兩好,成其夫婦,在此快活一世,有何不可!」月娘聽了,心中大怒,罵道:「雲離守,誰知你人皮包着狗骨!我過世丈夫,不曾把你輕待,如何一旦出此犬馬之言?」雲離守笑嘻嘻,向前把月娘摟住,求告說:「娘子,你自家中,如何走來我這裏做甚?自古上門買賣好做。不知怎的,一見你,魂靈都被你攝在身上。沒奈何,好歹完成了罷!」一面拿過酒來,和月娘喫。月娘道:「你前邊叫我兄弟來,等我與他說句話。」雲離守笑道:「你兄弟和玳安兒小廝已被我殺了。」即令左右:「取那件物事與娘子看!」不一時,燈光下血瀝瀝提了吳二舅玳安兩顆頭來,唬的月娘面如土色,一面哭倒在地。被雲離守向前抱起:「娘子不須煩惱,你兄弟已死,你就與我為妻。我一個總兵官,也不玷辱了你!」月娘自思道:「這賊漢將我兄弟家人害了命,我若不従,連我命也喪了。」乃回嗔作喜,說道:「你須依我,奴方與你做夫妻。」雲離守道:「不拘甚事,我都依!」月娘道:「你先把我孩兒完了房,我卻與你成婚。」雲離守道:「不打緊。」一面叫出雲小姐來,和孝哥兒推在一䖏,飲合巹盃,綰同心結,成其夫婦。然後拉月娘和他雲雨。這月娘卻拒阻不肯。被雲離守忿然大怒,罵道:「賤婦,你哄的我與你兒子成了婚姻,敢笑我殺不得你的孩兒?」拔劍向牀砍去,頭隨手而落,血濺數步之遠。正是:三尺利刀着項上,滿腔鮮血濕模糊!

月娘見砍死孝哥兒,不覺大叫一聲。不想撒手驚覺,卻是南柯一夢。唬的渾身是汗,遍體生津。連道:「怪哉,怪哉!」小玉在旁,便問:「奶奶怎的哭?」月娘道:「適間做得一夢不祥。」不免告訴小玉一遍。小玉道:「我倒剛纔不曾睡着,悄悄打門縫見那和尚,原來和鬼說了一夜話!剛纔過世俺爹、五娘、六娘,和陳姐夫、周守備、孫雪娥、來旺兒媳婦子、大姐,都來說話,各四散去了。」月娘道:「這寺後現埋着他們,夜靜時分,屈死淹魂,如何不來!」娘兒們也不曾睡,不覺五更鷄叫。

吳月娘梳洗面貌,走到禪堂中禮佛燒香。只見普靜老師在禪牀上高呌:「那吳氏娘子,你如今可省悟得了麽?」這月娘便跪下參拜:「上告尊師,弟子吳氏,肉眼凡胎,不知師父是一尊古佛。適間一夢中,都已省悟了。」老師道:「既已省悟,也不消前去。你就去,也無過只是如此,倒沒的丧了五口兒性命。合你這兒子有分有緣遇着我,都是你平日一點善根所種,不然定然難免骨肉分離。當初你去世夫主西門慶造惡非善,此子轉身托化你家,本要蕩散其財本,傾覆其產業,臨死還當身首異處。今我度脫了他去,做了徒弟。常言一子出家,九祖升天。你那夫主寃愆解釋,亦得超生去了。你不信,跟我來,與你看一看。」於是扠步來到方丈内,只見孝哥兒還睡在牀。老師將手中禪杖向他頭上只一點,教月娘衆人看,——忽然翻過身來,卻是西門慶,項帶沉枷,腰繫鐵索。復用禪杖只一點,依舊還是孝哥兒,睡在牀上。月娘見了,不覺放聲大哭,原來孝哥兒即是西門慶托生!良久,孝哥兒醒了,月娘問他:「如今你跟了師父出家。」在佛前與他剃頭,摩頂受記\endnote{摩頂受記——佛教授教儀式。摩頂,本為釋迦牟尼以大法付囑摩訶薩時,用右手摩其頭頂。後佛教授戒時,將摩受戒者頂傳為定式。受記,指受戒者頭頂上燙香洞。}。可憐月娘扯住慟哭了一場,乾生受養了他一場,到十五歲指望承家嗣業,不想被這個老師幻化去了!吳二舅小玉玳安亦悲不勝。

當下這普靜老師,領定孝哥兒,起了他一個法名,喚做明悟,作辭月娘而去。臨行,吩咐月娘:「你們不消往前途去了。如今不久,番兵退去,南北分為兩朝,中原已有個皇帝。多不上十日,兵戈退散,地方寜靜了,你們還囘家去安心度日。」月娘便道:「師父,你度化了孩兒去了,甚年何日,我母子再得見面?」不覺扯住,放聲大哭起來。老師便道:「娘子休哭,你見那邊又有一位老師來了!」哄的衆人扭頸囘頭,當下化陣清風不見了。正是:三降塵寰人不識,倏然飛過岱東峯。

不說普靜老師幻化孝哥兒去了。且說吳月娘與吳二舅衆人,在永福寺住了那到十日光景,果然大金國立了張邦昌,在東京稱帝,置文武百官。徽宗欽宗兩君北去;康王泥馬度江,在建康即位,是為高宗皇帝。拜宗澤為大將,復取山東河北,分為兩朝,天下太平,人民復業。後月娘歸家,開了門戶,家産器物都不曾疎失。後就把玳安改名做西門安,承受家業,人稱呼為西門小員外。養活月娘到老,壽年七十歲,善終而亡。此皆平日好善看經之報也!有詩為證:

\begin{myquote}
閒閱遺書思惘然,誰知天道有循環。

西門豪横難存嗣,經濟顛狂定被殲。

樓月善良終有壽,瓶梅淫佚早歸泉。

可怪金蓮遭惡報,遺臭千年作話傳!
\end{myquote}

\begin{myquote}
{\kaishu\small\color{gray}(金瓶梅詞話卷終)}
\end{myquote}

\theendnotes

\bookmarksetup{startatroot}

