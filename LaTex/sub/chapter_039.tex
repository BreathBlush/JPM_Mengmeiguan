\includepdf[pages={77,78},fitpaper=false]{tst.pdf}
\chapter*{第三十九囬 \\西門慶玉皇廟打醮 吴月娘聽尼僧説經}
\addcontentsline{toc}{chapter}{第三十九囬 西門慶玉皇廟打醮 吴月娘聽尼僧説經}
\markboth{{\titlename}卷之四}{第三十九囬 西門慶玉皇廟打醮 吴月娘聽尼僧説經}


\begin{myquote}
漢武清齋夜築壇,自斟明水醮僊官。

殿前玉女移香案,雲際金人捧露盤。

絳節幾時還入夢,碧桃何處更驂鸞!

茂陵煙雨埋弓劍,石馬無聲蔓草寒。
\end{myquote}

話説當日西門慶在潘金蓮房中歇了一夜。那婦人恨不的鑽入他腹中,在枕畔千般貼戀,萬種牢籠,淚搵鮫鮹,語言温順,實指望買住漢子心。不料西門慶外邊又刮剌上了韓道國老婆王六兒,替他獅子街石橋東邊,使了一百廿兩銀子,買了一所門面兩間,倒底四層房屋居住。除了過道,第二層間半客位。第三層除了半間供養佛像祖先,一間做住房,裏面依舊鑲着炕床,對面又是燒煤火炕,收拾糊的乾淨。第四層除了一間廚房,半間盛煤炭,後邊還有一塊做坑厠。俱不必細説。自従搬過來,那左近街坊鄰舍,都知他是西門慶夥計,又見他穿着一套兒齊整絹帛衣服,在街上搖擺;他老婆常插戴的頭上黃熀熀,打扮喬模樣,在門前站立;這等行景,不敢怠慢,都送茶盒與他,又出人情慶賀。那中等人家,稱他做韓大哥、韓大嫂;以下者趕着以叔嬸呼之。西門慶但來他家,韓道國就在舖子裏上宿,教老婆陪他自在頑耍。朝來暮往,街坊人家也都知道這件事。懼怕西門慶有錢有勢,誰敢惹他。見一月之間,西門慶也來行走三四次,與王六兒打的一似火炭般熱,穿着器用,均比前日不同。

看看臘月時分,西門慶在家亂着,送東京并府縣軍衛本衛衙門中節禮。有玉皇廟吴道官,使徒弟送了四盒禮物:一盒肉、一盒銀魚、兩盒菓餡蒸酥;並天地疏、新春符、謝灶誥。西門慶正在上房吃飯,玳安兒拿進帖來,上寫着:「玉皇廟小道吴宗嚞頓首拜。」西門慶揭開盒兒看了,説道:「出家人,又教他費心,送這厚禮來!」吩咐玳安,連忙教書童兒封一兩銀子拿回帖與他。月娘在旁因話題起:「一個出家人,你要便的年頭節尾常受他的禮,倒把前日李大姐生孩兒時,你説許了多少醮願,就教他打了罷。」西門慶道:「早是你提起來,我許下一佰廿分醮,我就忘死了!」月娘道:「原來你這個大謅答子貨!誰家願心是忘記的?你便有口無心許下,神明都記着。嗔道孩子成日恁啾啾唧唧的,原來都這願心壓的他,此是你幹的營生!」西門慶道:「旣恁説,正月裏就把這醮願在吴道官這廟裏還了罷。」月娘道:「昨日李大姐説,這孩子有些病痛兒的,要問那裏討個外名。」西門慶道:「又往那裏討外名?就寄名在吴道官這廟裏罷。」因問玳安:「他廟裏有誰在這裏?」玳安道:「是他第二個徒弟應春跟了禮來。」

西門慶一面走出外邊來,那應春兒連忙跨馬磕頭,説:「家師父多拜上老爹,沒什麽孝順,使小徒來送這天地疏,並些微禮兒,與老爹賞人。」西門慶止還了半禮,説道:「多謝你師父厚禮。」讓他坐。説道:「小道怎麽敢坐?」西門慶道:「你坐,我有話和你説。」那道士頭戴小帽,身穿青布直裰,下邊履鞋淨襪,謙遜數次,方纔把椅兒挪到旁邊坐下。西門慶喚茶來吃了。説道:「老爹有甚鈞語吩咐?」西門慶道:「正月裏,我有些醮願,要煩你師父替我還還兒,在你本院。也是那日,就送小兒寄名。不知你師父閑不閑?」徒弟連忙立起身來,説道:「老爹吩咐,隨問有甚人家經事,不敢應承。請問老爹,訂在正月幾時?」西門慶道:「就訂在初九爺旦日那個日子罷。」徒弟道:「此日又是天誕。《玉匣記》上就講:『律爺交慶,五福駢臻。』修齋建醮甚好。那日開大殿與老爹鋪壇。請問老爹,多少醮款?」西門慶道:「也是今歲七月,為生小兒,許了一百廿分清醮。一向不得個心淨,趂着正月裏還了罷!就把小兒送與你師父,向三寳座下討個外名。」徒弟又問:「請問,那日延請多少道衆?」西門慶道:「教你師父請十六衆罷。」説畢,左右放桌兒待茶,先封十五兩經錢,另外又封了一兩酬答他的節禮。又説:「道衆的襯施,你師父不消備辦。我這裏連阡張香燭一事帶去。」喜歡的道士屁滚尿流,臨出門,謝了又謝,磕了頭兒又磕。

到正月初八日,先使玳安兒送了一石白米,一擔阡張,十斤官燭,五斤沉檀馬牙香,十二疋生眼布做襯施;又送了一對京緞,兩罈南酒,四隻鮮鵝,四隻鮮鷄,一對豚蹄,一脚羊肉,十兩銀子,與官哥兒寄名之禮。西門慶預先發帖兒,請下吴大舅、花大舅、應伯爵、謝希大,四位相陪。陳經濟騎頭口先到廟中,替西門慶瞻拜。到初九日,西門慶也沒往衙中去,絶早冠帶,騎大白馬,僕従跟隨,前呼後擁,逕出東門,往玉皇廟來。遠遠望見結綵的寳旛,過街榜棚,進約不上五里之地,就是玉皇廟。至山門前下馬,睜眼觀看,果然好座廟宇,天宫般蓋造。但見:

\begin{myquote}
青松鬱鬱,翠柏森森。金釘朱户,玉橋低影軒宫;碧瓦雕簷,繡幕高懸寳檻。七間大殿,中懸勅額金書;兩廡長廊,彩畫天神帥將。祥雲影裏,流星門高接青霄;瑞霞光中,鬱羅臺直侵碧漢。黃金殿上,列天帝三十二尊;白玉京中,現毫光百千萬億。三天門外,離婁與師曠猙獰;左右階前,白虎與青龍猛勇。寳殿前僊妃玉女,霞帔曾獻御香花;玉陛下四相九卿,朱履肅朝丹鳳闕。九龍床上,坐着個不壞金身萬天教主玉皇張大帝:頭戴十一冕旒,身披衮龍青袍。腰繫藍田帶,按八卦九宫;手執白玉圭,聽三皈五戒。金鐘撞處,三千世界盡皈依;玉磬鳴時,萬象森羅皆拱極。朝天閣上,天風吹下步虚聲;演法壇中,亱月常聞僊珮響。只此便為眞紫府,更于何䖏覓蓬萊!
\end{myquote}

西門慶由正門而入,見頭一座流星門上,七尺高朱紅牌架,列着兩行門對,大書:

\begin{myquote}
「黄道天開,祥啟九天之閶闔,迓金輿翠蓋以延恩;

玄壇日麗,光臨萬聖之旙幢,誦寳笈瑤章而闡化。」
\end{myquote}

到了寶殿上,懸着二十四字齋題,大書着:「靈寳答天謝地,報國酬恩,九轉玉樞,酬盟寄名,吉祥普福齋壇。」兩邊一聯:

\begin{myquote}
「先天立極,仰大道之巍巍,庸申至悃;

昊帝尊居,鑒清修之翼翼,上報洪恩。」
\end{myquote}

西門慶進入壇中香案前,旁邊一小童捧盆巾盥手畢,鋪排跪請上香,鋪毡褥,行禮叩壇畢。原來吳道官諱宗嚞,法名道眞,生的魁偉身材,一臉鬍鬚,襟懷洒落,廣結交,好施捨。現作本宫住持,以此高貴達官多往投之,做醮設席甚齊整,迎賓待客一團和氣。手下也有三五個徒弟徒孫,一呼百諾。西門慶會中常在此建醮,每生辰節令,疏禮不缺。何況西門慶又做了刑名官,來此做好事,送公子寄名,受其大禮,如何不敬?那日就是他做齋功,主行法事,頭戴玉環九陽雷巾,身披天青二十八宿大袖鶴氅,腰繫絲帶,忙下經筵來與西門慶稽首道:「小道蒙老爹錯愛,迭受重禮,使小道却之不恭,受之有愧!就是哥兒寄名,小道禮當叩祝三寳,保安增延壽命,尚不能以報老爹大恩;何以又叨受老爹厚賞許多厚禮,誠有愧赧!經襯又且過厚,令小道愈不安。」西門慶道:「厚勞費心辛苦,無物可酬,薄禮表情而已!」

敍禮畢,兩邊道衆齊來稽首。一面請去外方丈,三間厰廳,名曰松鶴軒,多是朱紅亮槅,那裏自有坐䖏待茶。西門慶見四面粉牆,擺設湖山瀟洒,堂中椅桌光鮮;左壁掛「黃鶴樓白日飛昇」,右壁懸「洞庭湖三番渡過」;正面有兩幅吊屏,草書一聯:「引兩袖清風舞鶴,對一方明月談經。」西門慶剛坐下,就令小廝棋童兒:「拿馬接你應二爹去。只怕他沒馬,如何這咱還沒來!」玳安道:「有姐夫騎的驢子,還在這裏。」西門慶道:「也罷。」吩咐棋童:「快騎接去。」那棋童従山門裏面牽出來騎了,一直去了。

吴道官誦畢經,下來遞茶,陪西門慶坐,敍話:「老爹敬神,一點誠心,小道怎敢惹罪。各道衆都従四更起來,到壇諷誦諸品僊經,并玉皇參行醮經。今日三朝九轉玉樞法事,都是整做。將官哥兒的生日八字,另具一宗文書,奏名於三寶面前,起名叫做吴應元,太乙司命,合延桃康,壽齡永保,富貴遐昌。小道這裏又添了二十四分答謝天地,十二分慶讚上帝,二十四分薦亡,共列一百八十分醮款。」西門慶道:「多有費心!」不一時,打動法鼓,請西門慶到壇看文書。西門慶従新換了大紅五彩獅補吉服,腰繫蒙金犀角帶。到壇,有絳衣表白在旁,先宣念齋意:

\begin{myquote}[\markfont]
「大宋國山東清河縣縣牌坊居住,奉道祈恩酬醮保安信官西門慶,本命丙寅年七月廿八日子時建生,同妻吴氏,本命戊辰年八月十五日子時建生,……」{\kaishu(表白道:「還有寶眷,小道未曾添上。」西門慶道:「你只添上個李氏,辛未年正月十五日申時建生。」)}「同男官哥兒,丙申年七月廿三日申時建生。領家眷等,即日投誠,拜干洪造。言念慶一介微生,三才末品。出入起居,每感龍天之護佑;迭遷寒暑,常蒙神聖以匡扶。職列武班,叨承禁衛。沐恩光之寵渥,享符祿之豐盈。蒞任刑名,每思圖報。恭逢盛世,仰賴帡幪。是以修設清醮,共廿四分位,答報天地之洪恩,酬祝皇王之巨澤。又修設清醮十二分位,茲逢天誕,慶讚帝眞。介五福以遐昌,迓諸天而下邁。良願于去歲七月二十三日,因為側室李氏生男官哥兒時,慶要祈坐蓐無虞,臨盆有慶。恭將男官哥兒寄於三寳殿下,賜名吴應元,期在出幼圓滿。另行請祈天地位下,告許清醮一百廿分位,續箕裘之胤嗣,保壽命之延長。附薦西門氏門中,三代宗親等魂:祖西門京良,祖妣李氏;先考西門達,妣夏氏;故室人陳氏,及前亡後化、昇墜罔知,是以修設淨醮廿四分位,恩資道力,均證生方。共列僊醮一百八十分位,仰干化覃,俯賜勾銷。謹以宣和三年正月初九日,天誕良辰,特就大慈玉皇殿,仗延官道,修建靈寳答天謝地、報國酬盟、慶神保安、寄名轉經、吉祥普滿大齋一晝夜。延三境之司尊,迓萬天之帝駕。日近清光,出入金門而有喜;時加美秩,褒封紫誥以增榮。一門長叨均安,四序公和迪吉。統資道力,介福方來。謹意。」
\end{myquote}

宣畢齋意,鋪設下許多文書符命,表白一一請看。揭開第一張説道:「此是奕世功果影發文書。申請三天三境上帝、十極高眞、三官四聖、泰玄都省,及天曹大皇萬滿眞君、天曹掌醮司眞君、天曹降聖司眞君,到壇證監功德的奏疏。」又揭起第二張:「此是申請東嶽天齊大生神聖帝、子孫娘娘、監生衛房聖母元君,并當時許還願日受禱之神,今日勾銷頃願典者,祠家侍奉長生香火,三教明神,勾銷老爹昔日許的願款,及行下七十五司地府冥官案吏主者,到壇來受追薦,護送亡人生天。此一票,是玉女靈官、天神帥將、功曹符使、土地等神,捧奏三天門運遞關文。此一張,玉清總召萬靈眞符,高功發遣公文,受事官符。此一張,是召九斗陽芒流星火全紾大將,開天門的符命。」看畢此䖏,又到一張桌上,揭起頭一張來:「此是早朝開啟請無佞太保康元帥,九天靈符監齋使者,嚴禁齋儀,監臨廚所。此一張,是請正法馬、趙、溫、關,四大元帥;崔、盧、竇、鄧,四大天君,監臨壇門。及玄壇四靈神君,九鳳破穢大將軍,淨壇蕩穢,以格高眞。此一宗,是早朝啟五師箋文,晚朝謝五師箋文。此一宗,是開闢二代捲簾化壇眞符。此一宗,是請神霄辟非大將軍鳴金鐘陽牒;神霄禁壇大將軍擊玉磬陰牒。此一宗,是安鎭五方眞文雲篆:東方九炁鎭天玉字眞文,南方三炁鎭天玉字眞文,西方七炁鎭天玉字眞文,北方五炁鎭天玉字眞文,中央一炁鎭天玉字眞文,請五老上帝安鎭壇垠,證監功德。俱是按五方顔色彩畫的。此一宗早朝頭一遍轉經,高上神霄玉眞王南極長生大帝;第二遍轉經,高上碧霄東極青華生大帝;第三遍轉經,高上青霄九天應元雷聲普化天尊;午朝第四遍轉經,高上玉霄九天雷祖大帝;第五遍轉經,高上琅霄太一大天帝;第六遍轉經,高上泰霄六天洞淵大帝;晚朝第七遍轉經,高上紫霄深波天主帝君;第八遍轉經,高上景霄青城益算可韓司丈人眞君;第九遍轉經,高上絳霄九天採訪使眞君。九道表箋,掠剩、報應、幽枉、積逮,啟四司、謝四司箋。此又一宗,是午朝高功捧奏拜進三天玉陛,黃籙朱表,并遣旨、介直直、符醮吏者,同當日受事功曹,護送章表殿遞云盤關文。此一宗,是三天持寳籙大將軍,并金龍、茭龍驛吏、火府賫簡童子,靈寳諸符命,不可細數。此一宗,是晚朝謝恩誠詞都疏,及一百八十表醮經醮,雲鶴馬子,俵分錢馬滿散關文。」又一桌案上:「此是哥兒三寳蔭下寄名,外一家文書符索牒劄。」其餘不暇細覽:「請謝高功老爹今日十分費心!」西門慶於是洞案前炷了香,畫了文書,左右捧一疋尺頭與吴道官畫字。固辭再三,方令小童收了。然後一個道士,向殿角頭𥑮碌碌擂動法鼓,有若春雷相似。合堂諸衆,一派音楽響起。吴道官身披大紅五彩雲織法氅,脚穿雲根飛舄朱履,手執牙笏,關發文書,登壇召將。兩邊鳴起鐘來。鋪排引西門慶進壇裏,向三寳案左右兩邊上香。西門慶於是睜眼觀看,果然鋪設齋壇齊整。但見:

\begin{myquote}
位按五方,壇分八級。上層供三清四御、八極九霄、十極高眞、雲宫列聖;中層山川嶽瀆、社會隍司、福地洞天、方輿博厚;下層冥宫幽壤、地府羅酆、江河湖海之神、水國泉扃之衆。兩班醮筵森列,合殿官將威儀。香騰瑞靄,千枝畫燭流光;花簇錦筵,百盞銀燈散彩。天地亭,左右金童玉女,對對高張羽蓋;玉帝堂,兩邊執盂捧劍,重重密布幢旛。風清三界步虛聲,月冷九天乘沆瀣。金鐘撞處,高功進表奏虚皇;玉珮鳴時,都講登壇朝玉帝。絳綃衣,星辰燦爛;芙蓉冠,金碧交加。監壇神將猙獰,直日功曹猛勇。道衆齊宣寳懺,上瑤臺酌水獻花;眞人密誦靈章,按法劍踏罡步斗。青龍隱隱來黃道,白鶴翩翩下紫宸。
\end{myquote}

西門慶剛遶壇拈香下來,被左右就請到松鶴軒閣兒裏,地鋪錦毯,爐焚獸炭,那裏坐去了。不一時,應伯爵謝希大來到。唱畢喏,每人封了一星折茶銀子,説道:「實告,要送些茶兒來,路遠,這些微意,權為一茶之需。」西門慶也不接,説道:「奈煩!自恁請你來陪我坐坐,又幹這營生做什麽?吳親家這裏點茶,我一總都有了,不消拿出來了。」那應伯爵連忙又唱喏説:「哥,眞個?俺們還收了罷?」因望着謝希大説道:「都是你幹這營生。我説哥不受,拿出來,倒惹他訕兩句好的!」良久,吴大舅花子由都到了,每人兩盒細茶食,來點茶。西門慶都令吴道官收了。吃畢茶,一同擺齋,放了兩張桌。桌上堆的鹹食齋饌,點心湯飯,甚是豐潔。西門慶寬去衣服,同吃了早齋。原來吴道官叫了個説書的,説西漢評話《鴻門會》。

吴道官發了文書,走來陪坐,問:「哥兒今日來不來?」西門慶道:「正是,小頑還小哩,房下恐怕路遠,唬着他,來不的。到午間,拿他穿的衣服來三寳面前攝受過,就是一般。」吴道官道:「小道也是這般計較,最好。」西門慶道:「別的倒也罷了,他是有些小膽兒。家裏三四個丫鬟連養娘輪流看視,只是害怕,貓狗都不敢到他跟前。」吴大舅道:「孩兒們好容易養活大!」正説着,只見玳安進來説:「裏邊桂姨銀姨,使了李銘吴惠送茶來了。」西門慶道:「叫他進來。」李銘吴惠兩個拿着兩個盒子跪下,揭開,都是頂皮餅、松花餅、白糖萬壽糕、玫瑰搽瓤捲兒,西門慶俱令吴道官收了。因問李銘:「你們怎得知道今日我在這裏打醮?」李銘道:「小的今早晨路見陳姑夫騎頭口,問來,纔知道爹今日在此做好事。歸家告訴桂姐,三媽説:『還不快買禮去!』旋約了吴銀姐,纔來了。多上覆爹,本當親來,不好來得。這盒粗茶兒與爹賞人罷了。」西門慶吩咐:「你兩個等着吃齋。」吴道官一面讓他二人下去,自有坐䖏,連手下人都飽食一頓。

話休饒舌,到了午朝拜表畢,吴道官預備了一張大插桌,簇盤定勝,高頂方糖菓品,各樣托葷蒸煠醎食素饌,點心湯飯,又有四十碟碗;又是一罈金華酒。哥兒的一頂黑青緞子銷金道髻,一件玄色紵絲道衣,一件綠雲緞小襯衣,一雙白綾小襪,一雙青潞紬納臉小履鞋,一根黃絨線縧,一道三寳位下的黃線索,一道子孫娘娘面前紫線索,一付銀項圈條脱,刻着「金玉滿堂,長命富貴」。一道朱書辟非黃綾符,上書着「太乙司命,合延桃康」八字,就扎在黄線索上,都用方盤盛着。又是四盤羹果,擺在桌上。差小童經袱内包着宛紅紙經疏,將三朝做過法事,一一開載節次,請西門慶過了目方纔裝入盒擔内,共約八擡,送到西門慶家。西門慶甚是歡喜,快使棋童兒家去,賞了道童兩方手帕,一兩銀子。

且説那日是潘金蓮生日,有吴大妗子、潘姥姥、楊姑娘、郁大姐,都在月娘上房坐的。見廟裏送了齋來,又是許多羹果插桌禮物,擺了四張桌子還擺不下,都亂出來觀看。金蓮便道:「李大姐,你還不快出來看哩,你家兒子師父廟裏送禮來了!又有許多他的小道冠髻、道衣兒。噫,你看!又是小履鞋兒。」孟玉樓又走向前,拿起來手中看,説道:「大姐姐,你看,道士家也精細的!這小履鞋,白綾底兒,都是倒扣針兒,方勝兒鎖的;這雲兒又且是好。我説他敢有老婆!不然,怎的扣納的恁好針脚兒?」吴月娘道:「沒的説!他出家人那裏有老婆?想必是僱人做的。」潘金蓮接過來,説:「道士有老婆,像王師父和大師父會挑的好汗巾兒,莫不是也有漢子?」王姑子道:「道士家,掩上個帽子那裏不去了?似俺這僧家,行動就認出來。」金蓮説道:「我聽得説,你住的觀音寺,背後就是玄明觀。常言道:男僧寺對着女僧寺,沒事也有事!」月娘道:「這六姐好恁六説白道的!」金蓮道:「這個是他師父與他娘娘寄名的紫線索,又是這個銀脖項符牌兒,上面銀打的八個字,帶着且是好看。背面墜着他名字,『吴』什麽『元』?」棋童道:「此是他師父起的法名:『吴應元』。」金蓮道:「這是個『應』字!」叫道:「大姐姐,道士無禮!怎的把孩子改了他姓了?」月娘道:「你看不知禮!」因使李瓶兒:「你去抱了你兒子來,穿上這道衣,俺們瞧瞧好不好?」李瓶兒道:「他纔睡下,又抱他出來?」金蓮道:「不妨事,你揉醒他。」那李瓶兒眞個去了。

這潘金蓮識字,取過紅紙袋兒,扯出送來的經疏看,上面西門慶底下同室人吴氏,傍邊只有李氏,再没別人,心中就有幾分不忿,拿與衆人瞧:「你説,賊三等兒九格的強人,你説他偏心不偏心?這上頭只寫着生孩子的,把俺們都是不在數的,都打到贅字號裏去了!」孟玉樓問道:「有大姐姐沒有?」金蓮道:「沒有大姐姐倒好笑!」月娘道:「也罷了,有了一個,也都是一般。莫不你家有一隊伍人,也都寫上,惹的道士不笑話麽?」金蓮道:「俺們都是劉湛兒鬼兒麽?比那個不出材的?那個不是十個月養的哩!」

正説着,李瓶兒従前邊抱了官哥兒來,孟玉樓道:「拿過衣服來,等我替哥哥穿。」李瓶兒抱着,孟玉樓替他戴上道髻兒,套上項牌和兩道索,唬的那孩子只把眼兒閉着,半日不敢出氣兒。玉樓把道衣替他穿上。吴月娘吩咐李瓶兒:「你把這經疏,納個阡張頭兒,親往後邊佛堂中自家燒了罷。」那李瓶兒去了。金蓮見玉樓抱弄孩子説道:「穿着這衣服,就是個小道士兒。」金蓮接過來説道:「什麽小道士兒,倒好像個小太醫兒!」被月娘正色説了兩句,便道:「六姐,你這個什麽話!孩兒們上快休恁的!」那金蓮訕訕的不言語了。一回,那孩子穿着衣服害怕,就哭起來。李瓶兒走來,連忙接過來,替他脱衣裳時,就拉了一抱裙奶屎。孟玉樓笑道:「好個吴應元,原來拉屎也有一托盤!」月娘連忙教小玉拿草紙替他抹。不一時,那孩子就磕伏在李瓶兒懷裏睡着了。李瓶兒道:「小大哥原來困了,媽媽送你到前邊睡去罷。」

吴月娘一面把桌面都散了,請大妗子楊姑娘潘姥姥衆人出來吃齋。看看晚來。原來初八日,西門慶因打醮不用葷酒,潘金蓮晚夕就没曾上的壽,直到今晚來家就與他遞酒,來到大門站立。不想等到日落時分,只見陳經濟自騎頭口來家。潘金蓮問:「你爹來了?」經濟道:「爹怕來不成了。我來時,醮事還未了,纔拜懺。怕不弄到起更。道士有個輕饒素放的?還要謝將吃酒!」金蓮聽了,一聲兒沒言語,使性子回到上房裏,對月娘説:「賈瞎子傳操——乾起了個五更;隔牆掠肝花——死心塌地。兜肚断了帶子——沒得絆了!剛纔在門首站了一回,只見陳姐夫騎了頭口來了,説爹不來了,醮事還未了,先打發他來家。」月娘道:「他不來罷,咱們自在。晚夕聽大師父王師父説因果,唱佛曲兒。」

正説着,只見陳經濟掀簾進來,已帶半酣兒,説:「我來與五娘磕頭。」問大姐:「有鍾兒,尋個兒篩酒,與五娘遞一鍾兒。」大姐道:「那裏尋鍾兒去?只恁與五娘磕個頭兒,到住回等我遞罷。你看他醉腔兒!恰好今日打醮,只好了你!吃的恁憨憨的來家。」月娘便問道:「你爹眞個不來了?玳安那奴才沒來?」陳經濟道:「爹見醮事還沒了,恐怕家裏沒人,先打發我來了,留下玳安在那裏答應哩。道士再三不肯放我,強死強活拉着,吃了兩三大鍾酒纔來了。」月娘問:「今日有哪幾個在那裏?」經濟道:「今日有大舅,和門外花大舅、應二叔和謝三叔、李銘,又有吴惠、兩個小優兒。夜黑不知纏到多早晚。今日只吴大舅來了,門外花大舅教爹留住了,也是過夜的數。」金蓮沒見李瓶兒在跟前,便道:「陳姐夫,連你也叫起花大舅來,是那門兒親?死了的知道罷了!你叫他李大舅纔是,怎叫他花大舅?」經濟道:「五娘,你老人家鄉裏姐姐嫁鄭恩——睜着個眼兒,閉着個眼兒罷。早是兒子不知他什麽帳兒,只是夥裏分錢就是了。」大姐道:「賊囚根子!快磕了頭,趂早與我外頭挺去,又口裏恁汗邪胡説了!」陳經濟於是請金蓮轉上,踉踉蹌蹌磕了四個頭,往前邊去了。

不一時,房中掌上燈燭,放下桌兒,擺上菜兒,請潘姥姥楊姑娘大妗子與衆人來了。金蓮遞了酒,打發坐下,吃了麵。吃到酒闌,收了家活,擡了桌出去。月娘吩咐小玉把儀門關了,炕上放下小桌兒衆人圍定,兩個姑子在正中間,焚下香,秉着一對蠟燭,都聽他説因果。先是大師父説道:

\begin{myquote}[\markfont]
「蓋聞《大藏經》中,講説一段佛法,乃是西天第三十二祖下界,降生東土傳佛心印。昔日唐高宗天子咸亨三年,中夏諸事不題,却説嶺南鄉泡渡村有一張員外,家豪大富,廣有金銀,呼奴使婢。員外所取八個夫人,朝朝快楽,日日奢華。貪戀風流,不思善事。忽的一日出門遊玩,見一夥善人,馱載香油細米等物,人人稱念佛號。向前便問:『你這些善人何往?』内中一人答曰:『一者打齋,二者聽經。』員外又問:『你等打齋聽經,有何功德?』衆人言説:『人生在世,佛法難聞,人身難得。《法華經》上説的好:若人有福,曾供養佛。今生不捨,來生榮華富貴従何而來?古人云:龍聽法而悟道,蟒聞懺以升天,何况人乎?』張員外到家,便叫安童:『去後房請出你八個奶奶來。』不一時,都到堂前。員外説:『婆婆,我今黄梅寺修行去,把家財分作八份,各人過其日月。想你我如今只顧眼前快楽,不知身後如何,若不修行,求出火坑,定落三塗五苦。』有夫人聽説,便道:「員外,你八寶羅漢之體,有甚業障?比不的俺女流之輩,生男長女,觸犯神祇。俺們業重,你在家裏修行,等俺八個替你躭罪,你休要去罷!』正是:

婆婆將言勸夫身,員外冷笑兩三聲。」
\end{myquote}

大師父説了一囬,該王姑子接偈。月娘、李嬌兒、孟玉樓、潘金蓮、孫雪娥、李瓶兒、西門大姐,并玉簫都齊聲接佛。王姑子念道:

\begin{myquote}[\markfont]

\kaishu{
「説八個,衆夫人,要留員外;告丈夫,休遠去,在家修行。

你如今,下狠心,撇下妻子;痛哭殺,兒和女,你也心疼!

閃得俺,姊妹們,無處歸落;好教我,一個個,怎過光陰?

従小兒,做夫妻,相隨到老;半路裏,丢下俺,倚靠何人?

兒扯爺,女扯娘,搥胸跌脚;一家兒,大共小,痛哭傷情。」
}

\hspace*{4em}\marktext{〔金字經〕}

\kaishu{
「夫人聽説淚不乾,苦勸員外莫歸山。顧家園,兒女永團圓;休遠去,在家修行都一般。」
}

\hspace*{4em}\marktext{(白文)}

員外便說:『多謝你八個夫人,我明白死在陰司,你們替我耽罪。我今與你們遞一鍾酒,明日好在閻王面前承當。』飲酒中間,員外設了一計:『夫人與我把燈剔一剔。』員外哄的夫人剔燈,一口把燈吹死。唬的八個夫人失色,連忙呌梅香:『快點燈來!』員外取出鋼刀劔,唬殺八個衆夫人。

\hspace*{4em}\marktext{又偈:}

\kaishu{
老員外,喚梅香,把燈點起;將鋼刀,拿在手,指定夫人:

那一個,把明燈,一口吹死?圖家財,害我命,改嫁別人。

若不説,一劍去,這頭落地!一個個,心害怕,倒在埃塵。

有八個,老夫人,慌忙跪下;告員外,你息怒,饒俺殘生。

你分明,一口氣,把燈吹死;吃幾鍾,紅面酒,拿劍殺人。

你若還,殺了俺,八個夫人;到陰司,告閻君,取你眞魂!
}

\hspace*{4em}\marktext{(白文)}

員外冷笑,便叫八個夫人:『你哄我,當身吹燈不認,如何替我陰司躭罪?八個女流之輩倒哄男身,笑殺年高有德人!』説的八個夫人閉口無言。員外想人生富貴,都是前生修來,便叫安童:『連忙與我裝載數車香油米麵,各樣菜蔬錢財等物,我往黄梅山裏打齋聽經去也。』」

\hspace*{4em}\marktext{〔金字經〕}

\kaishu{
「夫人聽我説根源,梵王天子棄江山。不貪戀,要結萬人緣;都全捨,萬古標名在世間。員外今日修行去,親戚鄰人送起程。」
}
\end{myquote}

念了一囬,吴月娘道:「師父餓了,且把經請過,吃些甚麽?」一面令小玉安排了四碟素菜兒、兩碟醎食兒、四碟兒糖薄脆、蒸酥、菊花餅、扳搭饊子,請大妗子、楊姑娘、潘姥姥,陪着二位師父用一個兒。大妗子説:「俺們不當家的,都剛吃的飽。教楊姑娘陪個兒罷。他老人家又吃着個齋。」月娘連忙用小描金碟兒,每樣揀了個點心,放在碟兒裏,先遞與兩位師父,然後遞與楊姑娘,説道:「你老人家陪二位請些兒。」婆子道:「我的佛爺,不當家!老身吃的可夠了。」又道:「這碟兒裏是燒骨朵,姐姐你拿過去。只怕錯揀到口裏。」把衆人笑的了不得。月娘道:「奶奶,這個是頭裏廟上送來的托葷鹹食,你老人家只顧用,不妨事。」楊姑娘道:「既是素的,等老身吃。老身乾淨眼花了,只當做葷的來!」正吃着,只見來興兒媳婦子惠秀走來。月娘道:「賊臭肉,你也來做什麽?」惠秀道:「我也來聽唱曲兒。」月娘道:「儀門關着,你打那裏進來了?」玉簫道:「他在廚房封火來。」月娘道:「嗔道恁弄的鼻兒烏嘴兒黑的,成精鼓搗,來聽什麽經!」

當下衆丫鬟婦女圍定兩個姑子,吃了茶食,收過家活去,搽抹經桌乾淨。月娘従新剔起燈燭來,炷了香。兩個姑子打動擊子兒,又高念起來。從張員外在黄梅山寺中修行,白日長跪聽經,夜晚參禪打坐。四祖禪師觀見他不是凡人,定是個眞僧出世,問其鄉貫住處,姓甚名誰。員外具説前因一遍:弟子把家財妻子棄了,實為生死出家。四祖收留座下,做了徒弟。白日教他栽樹,夜晚舂米。六年苦行已滿,驚動護法韋馱尊天,驚覺四祖,教他尋安身立命之處;與了他三樁寳貝:斗蓬、簑衣、彎棗棍,往南去濁河邊投胎奪舍尋房兒居住,三百六十日正果圓成:「你如今年紀高大,房兒壞了,傳不得眞妙法,度脱不得衆生。」直說到千金小姐姑嫂兩個在濁河邊洗濯衣裳,見一僧人借房兒住,不合答了他一聲,那老人就跳下河去了。潘金蓮熬的磕困上來,就往房裏睡去了。少頃,李瓶兒房中繡春來呌說官哥兒醒了,也去了。只剩下李嬌兒、孟玉樓、潘姥姥、孫雪娥、楊姑娘、大妗子,守着聽到河中漂過一顆大僊桃來,小姐不合喫了,歸家有孕,懷胎十月。王姑子唱了一個〔耍孩兒〕:

\begin{myquote}
「一靈眞性投肚内,這個消息誰得知?人人不識西來意,呀的一聲孕男女。認的娘生鐵面皮,纔得見光明際。崑崙頂上轉大千世界,古彌陀分南北東西。」

\marktext{説:「千金小姐來到嫂子房中説,『咱兩個曾在濁河邊洗衣,見了那老人,問咱借房兒住,他如何跳在河内,唬的我心中驚怕。又吃了一個僊桃,我如今心頭膨悶,好生疑悔,腹中成其身孕!』正是:

十月腹中母懷胎,千金小姐淚盈腮。}
\end{myquote}

\begin{myquote}
千金説,在繡房,成其身孕;心中悔,無可奈,忍氣吞聲。

一個月,懷胎着,如同露水;兩個月,懷胎着,纔却朦朧。

三個月,懷胎着,纔成血餅;四個月,懷胎着,骨節纔成。

五個月,懷胎着,纔分男女;六個月,懷胎着,長出六根。

七個月,懷胎着,生長七竅;八個月,懷胎着,着相成人。

九個月,懷胎着,看看大滿;十個月,母腹中,准備降生。

五祖投胎在母腹中,因為度衆生。裟婆男女不肯回心,古佛下界轉凡身。借胎出殼,久後度母到天宫。

\hspace*{4em}\marktext{〔□□□〕}

五祖一佛性,投胎在腹中。權住十個月,轉凡度衆生。」
\end{myquote}

念到此處,月娘見大姐也睡去了,大妗子歪在月娘裏間床上睡着了,楊姑娘也打起欠呵來,桌上蠟燭也點盡了兩根。問小玉:「這天有多早晚了?」小玉道:「已是四更天氣,鷄鳴叫。」月娘方令兩位師父收拾經卷。楊姑娘便往玉樓房裏去了。郁大姐在後邊雪娥房裏宿歇。只有兩個姑子,月娘打發大師父和李嬌兒一處睡去了。王姑子和月娘在炕上睡。兩個還等着小玉炖了一甌子茶吃了纔睡。大妗子在裏間床上和玉簫睡。月娘因問王姑子:「後來這五祖長大了,怎生成了正果?」王姑子道:

「這裏爺娘見他有身孕,敎他哥哥祝虎,把千金小姐趕將出去,要行殺害。多虧祝龍慈心,放他逃生。走在垂楊樹下自縊,驚動天上太白李金星,教他尋茶討飯,隨緣度日。不覺十月滿足,來到僊人莊神廟裏,降生下五祖。紫霞紅光,罩滿了廟堂。小姐見孩兒生下就盤膝端坐,心中害怕,不比尋常。後又到天喜村王員外家場裏宿歇。場中火起,㧱起見員外。見小姐顔色,就要留下做小。子母兩個下拜,登時把員外夫人都拜死了。家奴院公拿住子母。後員外甦省過來,説道:『只怕是好人。』留在家中養活。六歲五祖方説話,不由為母的,一直走到濁河邊枯樹下,取了三樁寳貝,逕往黄梅寺聽四祖説法,遂成正果。後還度脱母親生天。」

月娘聽了,越發好信佛法了,有詩為證:

\begin{myquote}
聽法聞經怕無常,紅蓮舌上放毫光。

何人留下禪空話,留取尼僧化稻粱!
\end{myquote}

畢竟未知後來如何,且聽下囬分解。

