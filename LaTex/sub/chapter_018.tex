\includepdf[pages={35,36},fitpaper=false]{tst.pdf}
\chapter*{第十八囬 \\來保上東京幹事 陳經濟花園管工}
\addcontentsline{toc}{chapter}{第十八囬 來保上東京幹事 陳經濟花園管工}
\markboth{{\titlename}卷之二}{第十八囬 來保上東京幹事 陳經濟花園管工}


\begin{myquote}
堪歎人心毒似蛇,誰知天眼轉如車;

去年妄取東鄰物,今日還歸北舍家;

無義錢財湯潑雪,儻來田地水推沙。

若將奸狡爲活計,恰似朝雲與暮霞。
\end{myquote}

話分兩頭。不説蔣竹山在李瓶兒家招贅,單表來保來旺二人上東京打點。朝登紫陌,暮踐紅塵,饑餐渴飲,带月披星。有日到東京,進了萬壽城門,投旅店安歇。到次日,街前打聽,只聽見過路人風裏言風裏語,多交頭接耳,街談巷議,都説兵部王尚書昨日會問明白,聖旨下來,秋後處決。止有楊提督名下親屬人等未曾拿完,尚未定奪,且待今日便有次第。

這來保等二人,把禮物打在身邊,急來到蔡府門首。舊時幹事來了兩遍道路久熟。立在龍德街牌樓底下,探聽府中消息。少頃,只見一個青衣人,慌慌打太師府中出來,往東去了。來保認的是楊提督府裏親隨楊幹辦。待要叫住問他一聲事情何如,因家主不曾吩咐招惹他,以此不言語,放過了他去了。遲了半日,兩個走到府門前,望着守門官深深唱了個喏:「動問一聲,太師老爺在家不在?」那守門官道:「老爺不在家了,朝中議事未囬。你問怎的?」來保又問道:「管家翟爺請出來小人見見,有事禀白。」那官吏道:「管家翟叔也不在了,跟老爺出去了。」來保道:「且住。他不實説與我,一定問我要些東西。」於是袖中取出一兩銀子遞與他。那官吏接了,便問:「你要見老爺,要見學士大爺?老爺便是大管家翟謙禀,大爺的事便是小管家高安禀,各有所掌。況老爺朝中未囬,止有學士大爺在家。你有甚事,我替你請出高管家來,有甚事引你禀見大爺,也是一般。」這來保就借情道:「我是提督楊爺府中,有事禀見。」官吏聽了,不敢怠慢,進入府中,良久,只見高安出來。來保慌忙施禮,遞上十兩銀子,説道:「小人是楊爺的親,同楊幹辦一路來見老爺討信。因後邊吃飯來遲了一步,不想他先來見了,所以不曾趕上。」高安接了禮物,説道:「楊幹辦只剛纔去了,老爺還未散朝。你且待待,我引你再見見大爺罷。」一面把來保領到第二層大廳傍邊,另一座儀門進去。坐北朝南三間敞廳,綠油欄杆,朱紅牌額,石青塡地,金字大書,天子御筆欽賜「學士琴堂」四字。

原來蔡京兒子蔡攸也是寵臣,現為祥和殿學士兼禮部尚書提點太一宫使。來保在門外伺候。高安先入,説了出來,然後喚來保入見,當廳跪下。廳上垂着朱簾,蔡攸深衣軟巾,坐於堂上,問道:「是那裏來的?」來保禀道:「小人是楊爺的親家陳洪的家人,同府中楊幹辦來禀見老爺討信。不想楊幹辦先來見了,小人趕來後見。」因向懷中取出揭帖遞上。蔡攸見上面寫着「白米五百石」,叫來保近前,説道:「蔡老爺亦因言官論列,連日迴避。閣中之事,并昨日三法司會問,都是右相李爺秉筆;你楊老爺的事,昨日内裏消息出來,聖上寬恩,另有處分了。其手下用事有名人犯,待查明問罪。你還徑到李爺那裏説去。」來保只顧磕頭道:「小的不認的李爺府中,望爺憐憫俯就,看家楊老爺分上。」蔡攸道:「你去到天漢橋迤北高坡大門樓處,問聲當朝右相、資政殿大學士兼禮部尚書名諱邦彦的——你李爺,誰是不知道!也罷,我這裏還差個人同你去。」即令祇候官呈過一緘,使了圖書,就着管家高安同去見李老爺,如此這般替他説。

那高安承應下了,同來保出了府門,叫了來旺,帶着禮物,轉過龍德街,逕到天漢橋李邦彦門首。正値邦彦朝散纔來家,穿大紅縐紗袍,腰繫玉帶,送出一位公卿上轎而去。囬到廳上,門吏禀報説:「學士蔡大爺差管家來見。」先叫高安進去,説了囬話。然後喚來保來旺進見,跪在廳臺下。高安就在傍邊遞了蔡攸封緘,幷禮物揭帖。來保下邊就把禮物呈上。邦彦看了説道:「你蔡大爺分上,又是你楊老爺親,我怎麽好受此禮物?況你楊爺,昨日聖心回動,已沒事。但只是手下之人,科道參語甚重,一定問發幾個。」即令堂候官取過昨日科中送的那幾個名字與他瞧,上寫着:

\begin{myquote}[\markfont]
「王黼名下書辦官董昇、家人王廉、班頭黄玉;楊戩名下壞事書辦官盧虎、幹辦楊盛、府掾韓宗仁、趙弘道、班頭劉成、親黨陳洪、西門慶、胡四等;皆鷹犬之徒,狐假虎威之輩。揆置本官,倚勢害人;貪殘無比,積獘如山,小民蹙額,巿肆為之騷然!乞勅下法司,將一干人犯,或投之荒裔,以禦魑魅;或置之典刑,以正國法;不可一日使之留於世也!」
\end{myquote}

來保等見了,慌的只顧磕頭,告道:「小人就是西門慶家人,望老爺開天地之心,超生性命則個!」高安又替他跪稟一次。邦彦見五百兩金銀,只買一個名字,如何不做分上?即令左右擡書案過來,取筆將文卷上西門慶名字改作「賈慶」;一面收上禮物去。邦彦打發來保等出來,就拿回帖囬蔡學士,賞了高安來保來旺一封五十兩銀子。

來保路上作辭高管家,囬到客店,收拾行李,還了店錢,星夜囬到清河縣來。早到家見西門慶,把東京所幹的事,従頭説了一遍。西門慶聽了,如提在冷水盆内,對月娘説:「早是使人去打點,不然怎了!」正是:這回西門慶性命,有如落日已沉西嶺外,卻被扶桑喚出來。於是一塊石頭方纔落地。過了兩日,門也不関了,花園照舊還蓋,漸漸出來街上走動。

一日玳安騎馬打獅子街所過,看見李瓶兒門首開個大生薬舖,裏邊堆着許多生熟薬材。朱紅小櫃,油漆牌面,吊着幌子,甚是熱鬧。歸來告與西門慶説,還不知招贅竹山一節,只説:「二娘搭了個新夥計,開了個生薬舖。」西門慶聽了,半信不信。一日,七月中旬時分,金風淅淅,玉露泠泠。西門慶正騎馬街上走着,撞見應伯爵謝希大兩人,叫住,下馬唱喏。問道:「哥,一向怎的不見?兄弟到府上幾遍,見大門関着,又不敢叫,整悶了這幾日。端的哥在家做甚事?嫂子娶過來不曾?也不請兄弟們吃酒?」西門慶道:「不好告訴的。因舍親家陳宅那邊為些閒事,替他亂了幾日。親事另改了日期了。」伯爵道:「兄弟們不知哥吃驚。今日既撞遇哥,兄弟二人肯空放了?如今請哥同到裏邊吴銀姐那裏吃三盃,權當解悶。」不由分説,把西門慶拉進院中來。玳安平安牽馬,後邊跟着走。正是:

\begin{myquote}
歸去只愁紅日短,思鄉猶恨馬行遲。

世財紅粉歌樓酒,誰為三般事不迷?
\end{myquote}

當日西門慶被他二人拉到吴銀兒家,吃了一日酒。到日暮時分,已帶半酣,纔放出來。打馬正望家走,到於東街口上,撞見馮媽媽從南來,走得甚慌。西門慶勒住馬,問道:「你往那去?」馮媽媽道:「二娘使我往門外寺裏盂蘭會,替過世二爺燒箱庫去來,趕進門來。」西門慶醉中道:「你二娘在家好麽?我明日和他説話去。」馮媽媽道:「兀得大官人還問甚麽好也來?把個現現成成做熟了飯的親事兒,吃人掇了鍋兒去了。」西門慶聽了,失驚問道:「莫不他嫁人去了?」馮媽媽道:「二娘那等使老身送過頭面,往你家去了幾遍不見你,大門関着。對大官兒説進去,敎你早動身,你不理。今敎別人成了,你還説甚的?」西門慶問是誰,馮媽媽悉把半夜三更婦人被狐狸纏着,染病着,看看至死;怎的請了大街上住的蔣竹山來看,吃了他的薬怎的好了;某日怎的倒踏門招進來,成其夫婦:「現今二娘拿出三百兩銀子,與他開了生薬舖。」從頭至尾,説了一遍。這西門慶不聽便罷,聽了氣的在馬上只是跌脚。呌道:「苦哉!你嫁別人,我也不惱。如何嫁那矮王八!他有甚麽起解?」於是一直打馬來家。

剛下馬進儀門,只見吳月娘孟玉樓潘金蓮并西門大姐四個在前廳天井内月下跳百索兒耍子。見西門慶來家,月娘玉樓大姐三個都往後走了,只有金蓮不去,且扶着庭柱兜鞋。被西門慶帶酒駡道:「淫婦們閒的聲喚,平白跳甚麽百索兒?」趕上金蓮踢了兩脚。走到後邊,也不往月娘房中去脱衣裳,走在西廂稍間一間書房,要了舖蓋,那裏宿歇。打丫頭,罵小廝,只是没好氣。

衆婦人站在一處,都甚是着恐,不知是那緣故。吴月娘甚是埋怨金蓮:「你見他進門有酒了,兩三步扠開一邊便了。還只顧在跟前笑成一塊,且提鞋兒,卻敎他蝗蟲螞蚱一例都罵着!」玉樓道:「罵我們也罷,如何連大姐姐也駡起淫婦來了?沒槽道的行貨子!」金蓮接過來道:「這一家子只我是好欺負的!一般三個人在這裏,只踢我一個兒。那個偏受用着甚麽也怎的?」月娘就惱了,説道:「你頭裏何不敎他連我也踢不是?你沒偏受用,誰偏受用?恁的賊不識高低貨!我倒不言語,你只顧嘴頭子嗶哩礴喇的!」那金蓮見月娘惱了,便轉把話兒來摭,説道:「姐姐,不是這等説。他不知那裏因着甚麽由頭兒,只拿我煞氣。要便睜着眼望着我叫,千也要打個臭死,萬也要打個臭死!」月娘道:「誰敎你又要嘲他來?他不打你,卻打狗不成?」玉樓道:「大姐姐,且叫了小廝來問他聲,今日在誰家吃酒來?早晨好好出去,如何來家恁個腔兒?」不一時把玳安呌到跟前,問他端的。月娘罵道:「賊囚根子!你不實説,敎大小廝來吊拷你,和平安兒每人都是十板子。」玳安道:「娘休打,待小的實説了罷。爹今日和應二叔們都在院裏吴家吃酒,散的早,出來在東街口上撞遇馮媽媽,説花二娘等爹不去,嫁了大街住的蔣太醫了。爹一路上惱的了不的。」月娘道:「信那没廉恥的歪淫婦;浪着嫁了漢子,來家拿人煞氣!」玳安道:「二娘沒嫁蔣太醫,把他倒踏門招進去了。如今二娘與了他本錢,開了好不興的大薬舖。我來家告爹説,爹還不信。」孟玉樓道:「論起來,男子漢死了多少時兒,服也還未滿,就嫁人,使不得的。」月娘道:「如今年程,論的甚麽使的使不的。漢子孝服未滿,浪着嫁人的,纔一個兒?淫婦成日和漢子酒裏眠酒裏臥底人,他原守的甚麽貞節!」看官聽説:月娘這一句話,一棒打着兩個人。孟玉樓與潘金蓮都是再醮嫁人,孝服都不曾滿。聽了此言,未免各人懷着慚愧歸房,不在話下。正是:不如意處常八九,可與人言無二三。

卻説西門慶當晚在前邊廂房睡了一夜。到次日,把女婿陳經濟安他在花園中,同賁四管工記帳;換下來昭來,敎他看守大門。西門大姐白日裏便在後邊和月娘衆人一處吃飯,晚夕歸前邊廂房中歇。陳經濟每日只在花園中管工,非呼喚不敢進入中堂,飲食都是小廝内裏拿出來吃。所以西門慶手下這幾房婦女都不曾見面。一日,西門慶不在家,與提刑所賀千户送行去了。月娘因陳經濟搬來居住,一向管工辛苦,不曾安排一頓飯兒酬勞他酬勞,向孟玉樓李嬌兒説道:「待要管,又説我多攬事。我待欲不管,又看不上。人家的孩兒在你家,每日起早睡晚辛辛苦苦,替你家打勤勞兒,那個興心,知慰他一知慰兒也怎的?」玉樓道:「姐姐,你是個當家的人,你不上心誰上心?」月娘於是吩咐廚下,安排了一桌酒餚點心,午間請經濟進來吃一頓飯。

這陳經濟撇了工程,敎賁四看管,逕到後邊參見月娘。作畢揖,旁邊坐下。小玉拿茶來吃了,安放桌兒,拿蔬菜案酒上來。月娘道:「姐夫每日管工辛苦。要請姐夫進來坐坐,白不得個閒。今日你爹不在家,無事,治了一盃水酒,權與姐夫酬勞。」經濟道:「兒子蒙爹娘擡舉,有甚勞苦?這等費心!」月娘遞了酒,經濟傍邊坐下。湏臾,饌餚齊上。月娘陪着他吃了一囬酒。月娘使小玉:「請大姑娘來這裏坐。」小玉道:「大姑娘使着手,便來。」少頃,只聽房中抹的牌響。經濟便問:「誰人抹牌?」月娘道:「是大姐與玉簫丫頭弄牌。」經濟道:「你看没分曉,娘這裏呼喚不來,且在房中抹牌。」不一時,大姐掀簾子出來,與他女婿對面坐下,一同飲酒。月娘便問大姐:「陳姐夫也會看牌也不會?」大姐道:「他也知道些香臭兒。」當時月娘只知經濟是個志誠的女婿,卻不道這小夥子兒詩詞歌賦、雙陸象棋、拆白道字,無所不通,無所不曉。有〔西江月〕為證:

\begin{myquote}
自幼乖滑伶俐,風流博流牢成。愛穿鴨綠出爐銀,雙陸象棋幫襯。琵琶笙ぬ簫管,彈丸走馬圓情。只有一件不堪聞:見了佳人是命。
\end{myquote}

月娘便道:「既是姐夫會看牌,何不進去咱同看一看?」經濟道:「娘和大姐看罷,兒子卻不當。」月娘道:「姐夫至親間,怕怎的?」一面進入房中。只見孟玉樓正在牀上鋪茜紅毡看牌。見經濟進來,抽身就要走。月娘道:「姐夫又不是别人,見個禮兒罷。」向經濟道:「這是你三娘哩。」那經濟慌忙躬身作揖,玉樓還了萬福。當下玉樓大姐三人同抹,經濟在旁邊觀看。抹了一囬,大姐輸了下來,經濟上來又抹。玉樓出了個天地分;經濟出了恨點不到頭;吴月娘出了個四紅沉八不就,雙三不搭兩么兒,和兒不出;左來右去,配不着色頭。只見潘金蓮掀開簾子走進來,銀絲䯼髻上戴着一頭鮮花兒,僊家體態玉貌,笑嘻嘻道:「我説是誰,原來是陳姐夫在這裏。」慌的陳經濟扭頸囬頭,猛然一見,不覺心蕩目搖,精魂已失。正是:五百年寃家今朝相遇,三十年恩愛一旦遭逢。月娘道:「此是五娘。姐夫也只見個常禮兒罷。」經濟忙向前深深作揖,金蓮一面還了萬福。月娘便道:「五姐你來看,小雛兒倒把老鴉子來贏了。」這金蓮近前,一手扶着牀護炕兒,一隻手拈着白紗團扇兒,在傍替月娘指點説道:「大姐姐,這牌不是這等出了。把雙三搭過來,卻不是天不同和牌?還贏了陳姐夫和三姐姐。」衆人正抹牌在熱鬧處,只見玳安抱進毡包來,説:「爹來家了。」月娘連忙攛掇小玉送陳姐夫打角門出去了。

西門慶下馬進門,先到前邊工上觀看了一遍,然後踅到潘金蓮房中來。金蓮慌忙接着,與他脱了衣裳,説道:「你今日送行去,來的早。」西門慶道:「提刑所賀千户新陞新平寨知寨,合衛所相知都郊外送他來,拿帖兒來知會我,不好不去的。」金蓮道:「你没酒,敎丫鬟看酒來你吃。」不一時,放了桌兒飲酒,菜蔬都擺在面前。飲酒中間,因説起後日花園捲棚上樑,約有許多親朋都要來遞菓盒酒、掛紅,少不得叫廚子置酒管待。說了一囬,天色已晚。春梅掌燈歸房,二人上牀宿歇。西門慶因起早送行,着了辛苦,吃了幾盃酒就醉了。倒下頭鼾睡如雷,齁齁不醒。那時正値七月二十頭天氣,夜裏有些餘熱,這潘金蓮怎生睡得着。忽聽碧紗帳内一派蚊雷,不免赤着身子起身來,執着燭滿帳照蚊。照一個,燒一個。回首見西門慶仰臥枕上,睡得正濃,搖之不醒。其腰間那話,带着托子,纍垂偉長。不覺淫心輙起,放下燭臺,用纖手捫弄。弄了一囬,蹲下身去,用口吮之。吮來吮去,西門慶醒了。罵道:「怪小淫婦兒!你達達睡睡,就摑混死了。」一面起來,坐在枕上,一發叫他在下儘着吮咂;又垂首玩之,以暢其美。正是:怪底佳人風性重,夜深偷弄紫鸞簫。有蚊子雙関〔踏莎行〕詞為證:

\begin{myquote}
我愛他身體輕盈,楚腰膩細。行行一派笙歌沸。黄昏人未掩朱扉,潛身撞入紗厨内。款傍香肌,輕憐玉體。嘴到處胭脂記。耳邊廂造就百般聲,夜深不肯敎人睡。
\end{myquote}

婦人於是頑了有一頓飯時,西門慶忽然想起一件事來,叫春梅篩酒過來,在牀前執壶而立。將燭移在牀背板上,敎婦人馬爬在他面前,那話隔山取火,插入牝中,令其自動,在上飲酒取其快楽。婦人駡道:「好個刁鑽的強盜!従幾時新興出來的例兒,怪剌剌敎丫頭看答着,甚麽張致!」西門慶道:「我對你説了罷,當初你瓶姨和我常如此幹,叫他家迎春在傍執壶斟酒,倒好耍子。」婦人道:「我不好罵出來的,甚麽瓶姨鳥姨,題那淫婦則甚?奴好心不得好報。那淫婦等不的,浪着嫁漢子去了。你前日吃了酒,你來家,一般的三個人在院子裏跳百索兒,只拿我煞氣,只踢我一個兒,倒惹的人和我拌了囬子嘴。想起來,奴是好欺負的!」西門慶問道:「你與誰拌嘴來?」婦人道:「那日你便進來了,上房的好不和我合氣。説我在他跟前頂嘴來,駡我不識高低的貨。我想起來為甚麽!養蝦蟆得水蠱兒病,如今倒敎人惱我!」西門慶道:「不是我也不惱。那日應二哥他們拉我到吴銀兒家,吃了酒出來,路上撞見馮媽媽子,如此這般告訴我,把我氣了個立睜。若嫁了别人,我倒罷了。那蔣太醫賊矮王八,那花大怎不咬下他下截來?他有甚麽起解?招他進去,與他本錢,敎他在我眼面前開舖子,大剌剌做買賣。」婦人道:「虧你有臉兒還説哩!奴當初怎麽説來?先下米的先吃飯。你不聽,只顧求他——問姐姐。常言:信人調,丢了瓢!你做差了,你抱怨那個?」西門慶被婦人這幾句話,冲得心頭一點火起,雲山半壁通紅,便道:「你由他,敎那不賢良的淫婦説去,到明日休想我這裏理他!」

看官聽説:自古讒言罔行,雖君臣、父子、夫婦、昆弟之間,猶不能免,況朋友乎?饒吴月娘恁般賢淑的婦人,居於正室,西門慶聽金蓮袵席睥睨之間言,卒致於反目,其他可不愼哉!自是以後,西門慶與月娘尚氣,彼此覿面,都不説話。月娘隨他往那房裏去也不管他;來遲去早,也不問他;或是他進房中取東取西,只教丫頭上前答應,也不理他。兩個都把心來冷淡了。正是:

\begin{myquote}
前車倒了千千輛,後車到了亦如然。

分明指與平川路,錯把忠言當惡言。
\end{myquote}

且説潘金蓮自西門慶與月娘尚氣之後,見漢子偏聽於己,自以為得志,每日抖搜着精神粧飾打扮,希寵巿愛。因為那日後邊會遇陳經濟一遍,見小夥兒生的乖猾伶俐,有心也要勾搭他。但只畏懼西門慶,不敢下手。只等的西門慶往那裏去不在家,便使了丫鬟叫進房中,與他茶水吃,常時兩個下棋做一處。一日,西門慶新蓋捲棚上樑,親友掛紅慶賀,遞菓盒的也有許多。落作人匠,都有犒勞賞賜。大廳上管待官客,吃到晌午時分人纔散了。西門慶看着收拾了家伙,歸後邊睡去了。陳經濟走來金蓮房中討茶吃。金蓮正在牀上彈弄琵琶,道:「前邊上樑,吃了恁半日酒,你就不曾吃了些甚麽,還來我屋裏要茶吃?」經濟道:「兒子不瞞你老人家説,従半亱起來,亂了這一五更,誰吃甚麽來?」婦人問道:「你爹在那裏?」經濟道:「爹後邊睡去了。」婦人道:「你既没吃甚麽,叫春梅揀粧裏拿我吃的那蒸酥菓餡餅兒來,與你姐夫吃。」這小夥兒就在他炕桌兒擺着四碟小菜,吃着點心。因見婦人彈琵琶,戲問道:「五娘,你彈的甚曲兒?怎不唱個兒我聽。」婦人笑道:「好陳姐夫,奴又不是你影射的,如何唱曲兒你聽?我等你爹起來,看我對你爹説不説。」那經濟笑嘻嘻慌忙跪下,央及道:「望乞五娘可憐見,兒子再不敢了。」那婦人笑起來了。自此這小夥兒和這婦人日近日親,或吃茶吃飯,穿房入屋,打牙犯嘴,挨肩擦膀,通不忌憚。月娘托以兒輩,放這樣不老實的女婿在家,自家的事卻看不見。正是:只曉採花釀成蜜,不知辛苦為誰甜!

\begin{myquote}
堪歎西門慮未通,惹將桃李笑春風。

滿牀錦被藏賊睡,三頓珍羞養大蟲!

愛物只圖夫婦好,貪財常把丈人坑。

還有一件堪誇事,穿房入屋弄乾坤。
\end{myquote}

畢竟未知後來何如,且聽下囬分解。

