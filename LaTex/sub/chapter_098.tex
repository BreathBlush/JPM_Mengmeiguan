\includepdf[pages={195,196},fitpaper=false]{tst.pdf}
\chapter*{第九十八囬 \\陳經濟臨清開大店 韓愛姐翠館遇情郎}
\addcontentsline{toc}{chapter}{第九十八囬 陳經濟臨清開大店 韓愛姐翠館遇情郎}
\markboth{{\titlename}卷之十}{第九十八囬 陳經濟臨清開大店 韓愛姐翠館遇情郎}


\begin{myquote}
心安茅屋穩,性定菜根香。

世味薄方好,人情淡最長。

因人成事業,避難遇豪強。

今日崢嶸貴,他年身必殃。
\end{myquote}

話説一日,周守備、濟南府知府張叔夜,領人馬征剿梁山泊,賊王宋江三十六人,萬餘草寇,都受了招安,地方平復。表奏,朝廷大喜,加陞張叔夜為都御史山東安撫大使;陞守備周秀為濟南兵馬制置,管理分巡河道,提察盗賊。部下従征有功人員,各陞一級。軍門帶得經濟名字,陞為參謀之職,月給米二石,冠帶榮身。守備至十月中旬,領了勅書,率領人馬來家,先使人來報與春梅家中知道。春梅滿心歡喜,使陳經濟與張勝李安出城迎接。家中廳上,排設酒筵,慶官賀喜。官員人等來拜賀送禮者不計其數。守備下馬,進入後堂。春梅孫二娘接着,參拜已畢,陳經濟換了衣巾,就穿大紅員領,頭戴冠帽,脚穿皂靴,束着角帶,和新婦葛氏兩口兒拜見。守備見好個女子,賞了一套衣服,十兩銀子打頭面,不在話下。

晚夕,春梅和守備在房中飲酒,未免敍些家常事務。春梅道:「為娶我兄弟媳婦,又費許多東西。」守備道:「阿呀!你止這個兄弟,投奔你來,無個妻室前程,不成個道理。就使費了幾兩銀子,不曾為了别人。」春梅道:「你今又替他掙了這個前程,足以榮身,夠了。」守備道:「朝廷旨意下來,不日我往濟南府到任。你在家看家,打點些本錢,教他搭個主管,做些大小買賣。三五日教他下去查算帳目一遭,賺得些利錢來,也夠他攪計。」春梅道:「你說的也是。」兩個晚夕,夫妻同歡,不可細述。在家只住了十個日子,到十一月初旬時分,守備收拾起身,帶領張勝李安前去濟南到任,留周仁周義看家。陳經濟送到城南永福寺方囘。

一日,春梅同經濟商議:「守備教你如此這般,河下尋些買賣,搭個主管,覓得些利息,也夠家中費用。」這經濟聽言,滿心懽喜。一日,正打街前所走,尋覓主管夥計。也是合當有事,不料撞遇舊時朋友陸二哥陸秉義,作揖説:「哥怎的一向不見?」這經濟道:「我因亡妻為事,被楊光彦那廝拐了我半船貨物,坑陷的我一貧如洗。我如今又好了,幸得我姐姐嫁在守備府中,又娶了親事,陞做參謀,冠帶榮身。如今要尋個夥計,做些買賣,一地裏沒尋處。」陸秉義道:「楊光彦那廝,拐了你貨物,如今搭了個姓謝的做夥計,在臨清馬頭上謝家大酒樓上,開了一座大酒店,又收錢放債,與四方趂熟窠子娼門人使,好不獲大利息!他每日穿好衣,喫好肉,騎着一疋驢兒,三五日下去走一遭,算帳收錢,把舊朋友都不理。他兄弟在家開賭場,鬬鷄養狗,人不敢惹他。」經濟道:「我去年曾見他一遍,他反面無情,打我一頓,被一朋友救了。我恨他入於骨髓!」因拉陸二郎入路旁一酒店内,兩個在樓上喫酒。兩人計議:「如何䖏置他,出我這口氣?」陸秉義道:「常言説得好:恨小非君子,無毒不丈夫!咱如今將理和他說,不見棺材不下淚,他必然不肯。小弟有一計策,哥也不消做别的買賣,只寫一張狀子,把他告到那裏,追出你貨物銀子來,就奪了這座酒店。再添上些本錢,和謝合夥,等我在馬頭上和謝三哥掌櫃發賣。哥哥,你三五日下去走一遭,查算帳目,管情見一月你穩拍拍的有百十兩銀子利息,強如做别的生意!」看官聽說:當時不因這陸秉義說出這樁事,有分教數個人死於非命。陳經濟一種死,死之太苦;一種亡,亡之太屈——死的不好,相似那五代的李存孝,《漢書》中彭越。正是:非干前定數,半點不由人。經濟聽了,忙與陸秉義作揖,便道:「賢弟,你說的正是了。我到家就對我姐夫和姐姐說。這買賣成了,就安賢弟同謝三郎做主管。」當下兩個喫了囬酒,同下樓來,還了酒錢,經濟吩咐:「陸二哥,兄弟,千萬謹言!有事我請你去。」陸二郎道:「我知道。」各散囘家。

這經濟就一五一十,對春梅說。春梅道:「爭奈他爺不在,如何理會?」有老家人周忠在旁,便道:「不打緊,等舅寫了一張狀子,該拐了多少銀子貨物,拏爺個拜帖兒,都封在裏面。等小的送與提刑所兩位官府案下,把這姓楊的拏去衙門中,一頓夾打追問,不怕那廝不拏出銀子來!」經濟大喜。一面寫就一紙狀子,拏守備拜帖,彌封停當,就使老家人周忠送到提刑院。兩位官府正升廳問事。門上人禀進說:「帥府周爺差人下書。」何千戶與張二官府喚周忠進見,問周爺上任之事,說了一遍。拆開封套觀看,見了拜帖、狀子,自恁要做分上,即便批行,差委緝捕番捉,往河下拏楊光彦去。囘了個拜帖,付與周忠:「到家多上覆你爺奶奶,待我這裏追出銀兩,伺候來領。」周忠拏囘帖到府中,囘覆了春梅說話:「即時准行,拏人去了。待追出銀子,使人領去。」經濟看見兩個摺帖上面寫着:「侍生何永壽張懋德頓首拜」,經濟心中大喜。

遲了不上兩日光景,提刑緝捕觀察番捉,往河下把楊光彦並兄弟楊二風都拏了,到於衙門中。兩位官府,據着陳經濟狀子審問,一頓夾打,監禁數日,追出三百五十兩銀子,一百桶生眼布,其餘酒店中家活,共算了五十兩。陳經濟狀上告着九百兩,還差三百五十兩銀子,把房兒賣了五十兩,家彦盡絶。

這經濟就把謝家大酒樓奪過來,和謝胖子合夥。春梅又打點出五百兩本錢,共凑了一千兩之數,委付陸秉義做主管。従新把酒樓粧修,油漆彩畫,闌干灼耀,棟宇光新,桌案鮮明,酒餚齊整。一日開張,鼓樂喧天,笙簫雜奏,招集往來客商,四方遊妓。陳經濟到那日,宰猪祭祀燒紙。常言:啟瓮三家醉,開樽十里香;神仙留玉珮,卿相解金貂。經濟上來大酒樓上,週圍都是推窻亮槅,綠油闌干。四望雲山疊疊,上下天水相連。正東看,隱隱青螺堆岱嶽;正西瞧,茫茫蒼霧鎖皇都;正北觀,層層甲第起朱樓;正南望,浩浩長淮如素練。樓上下有百十座閣兒,處處舞裙歌妓,層層急管繁絃。說不盡餚如山積,酒若流波。正是:得多少舞低楊柳樓心月,歌罷桃花扇底風。従正月半頭,這陳經濟在臨清馬頭上大酒樓開張,見一日也發賣三五十兩銀子。都是謝胖子和主管陸秉義眼同經手,在櫃上掌櫃。經濟三五日騎頭口,伴當小喜兒跟隨,往河下算帳一遭。若來,陸秉義和謝胖子兩個夥計,在樓上收拾一間乾淨閣兒,鋪陳床帳,安放桌椅,糊的雪洞般齊整,擺設酒席,呌四個好出色粉頭相陪。陳三兒那裏往來做量酒。

一日,三月佳節春光明媚,景物芬芳,翠依依槐柳盈堤,紅馥馥杏桃燦錦。陳經濟在樓上,搭伏定綠闌干,看那樓下景致,好生熱鬧。有詩為證:

\begin{myquote}
風拂煙籠錦旆揚,太平時節日初長。

能添壯士英雄膽,善解羈人愁悶腸。

三尺曉垂楊柳岸,一竿斜插杏花旁。

男兒未遂平生志,且樂高歌入醉鄉。
\end{myquote}

一日,經濟在樓窻後瞧看,正臨着河邊,泊着兩隻剝船。船上載着許多箱籠桌櫈家活,四五個人盡搬入樓下空屋裏來。船上有兩個婦人:一個中年婦人,長挑身材,紫膛色;一個年小婦人,搽脂抹粉,生的白淨標致,約有二十多歲,盡走入屋裏來。經濟問謝主管:「是甚麽人?不問一聲,擅自搬入我屋裏來!」謝主管道:「此是兩個東京來的婦人,投親不着,一時間無處尋房住,央此間隣居范老來說,暫住兩三日便去。正欲報知官人,不想官人來問。」這經濟正欲發怒,只見那年小婦人歛袵向前,望經濟深深的道了個萬福,告說:「官人息怒。非干主管之事,是奴家大膽,一時出於無奈,不及先來宅上稟報,望乞恕罪。容略住得三五日,拜納房金,就便搬去。」這經濟見小婦人會說話兒,只顧上上下下把眼看他。那婦人一雙星眼,斜盼經濟。兩情四目,不能定神。經濟口中不言,心内暗道:「倒像那裏會過,這般眼熟!」那長挑身材中年婦人也定睛看着經濟,說道:「官人,你莫非是西門老爺家陳姑夫麽?」這經濟喫了一驚,便道:「你怎的認得我?」那婦人道:「不瞞姑夫說,奴是舊夥計韓道國渾家,這個就是我女孩兒愛姐。」經濟道:「你兩口兒在東京,如何來在這裏?你老公在那裏?」那婦人道:「在船上看家活。」經濟急令量酒請來相見。

不一時,韓道國走來作揖,已是摻白鬚鬢。因說起:「朝中蔡太師童太尉李右相朱太尉高太尉李太監六人,都被太學國子生陳東上本參劾,後被科道交章彈奏倒了,聖旨下來,拏送三法司問罪,發煙瘴地面永遠充軍。太師兒子禮部尚書蔡攸䖏斬,家彦抄沒入官。我等三口兒各自逃生,投到清河縣我兄弟第二的那裏。第二的把房兒賣了,流落不知去向。三口兒僱船従河道中來。不想撞遇姑夫在此,三生有幸。」因問:「姑夫,今還在那邊西門老爺家裏?」經濟把頭一搖,把前項說了一遍,說:「我也不在他家了。我在姐夫守備周爺府中做了參謀官,冠帶榮身。近日合了兩個夥計,在此馬頭上開了個酒店,胡亂過日子便了。你們三口兒旣遇着我,也不消搬去,便在此間住也不妨,請自穩便。」婦人與韓道國一齊下禮。說罷,就搬運船上家活箱籠。經濟看得心痒,也使伴當小喜兒和陳三兒,也替他搬運了幾件家活。王六兒道:「不勞姑夫費心用力!」經濟道:「你我原是一家,何消計較。」彼此俱各歡喜。經濟見天色將晚,有申牌時分,要囘家,吩咐主管:「明早送些茶盒與他。」上馬,伴當跟隨來家。一夜心心念念,只是放韓愛姐不下。

過了一日,到第三日,早起身,打扮衣服齊整,伴當小喜跟隨,來河下大酒樓店中,看着做了囘買賣。韓道國那邊使的八老來請喫茶。經濟心下正要瞧去,恰八老來請,便起身進去。只見韓愛姐見了,笑容可掬,接將出來,道了萬福:「官人請裏面坐。」經濟到閣子内坐下,王六兒和韓道國都來陪坐。少頃茶罷,彼此叙些舊時已往的話。經濟不住把眼只睃那韓愛姐,愛姐涎瞪瞪秋波一雙眼,只看經濟,彼此都有意了。有詩為證:

\begin{myquote}
弓鞋窄窄剪春羅,香體酥胸玉一窝。

麗質不勝嬝娜態,一腔幽恨蹙秋波。
\end{myquote}

少頃,韓道國下樓去了。愛姐因問:「官人青春多少?」經濟道:「虚度二十六歲。敬問姐姐青春幾何?」愛姐笑道:「奴與官人一緣一會,也是二十六歲!舊日又是大老爹府上相會過面,如今又幸遇在一處,正是有緣千里來相會。」那王六兒見他兩個說得入港,看見關目,推個故事,也下樓去了,止有他兩人對坐。愛姐把些風月話兒挑勾經濟。經濟自幼幹慣的道兒,怎不省得?一逕起身出去。這韓愛姐従東京來,一路兒和他娘也做些道路,在蔡府中答應,與翟管家做妾,詩詞歌賦,諸子百家皆通,甚麽事兒不久慣!見經濟起身出去,無人處,走向前挨在他身邊坐下,作嬌作癡說道:「官人,你將頭上金簪子,借我看一看。」經濟正欲拔時,被愛姐一手按住經濟頭髻,一手拔下簪子來。便起身說:「我和你去樓上說句話兒!」一頭說,一頭走。經濟不免跟上樓來。正是:饒你奸似鬼,也喫洗脚水。經濟跟他上樓,便道:「姐姐,有甚話説?」愛姐道:「奴與你是宿世姻緣,你休要作假,願偕枕蓆之懽,共效于飛之樂!」經濟道:「只怕此間有人知覺,卻使不得。」那韓愛姐做出許多妖嬈來,摟經濟在懷。將尖尖玉手,扯下他褲子來。兩個情興如火,按納不住。愛姐不免解衣,仰臥在床上,交媾在一處。正是:色膽如天怕甚事,鴛幃雲雨百年情。

經濟問:「你叫幾姐?」那韓愛姐道:「奴是端午所生,就叫五姐,又名愛姐。」說畢話,霎時雲收雨散,偎倚共坐。韓愛姐便告經濟說:「自従三口兒東京來投親不着,盤纏缺欠,你有銀子,乞借應與我父親五兩,奴按利納還,不可推阻。」經濟應允說:「不打緊,姐姐開口,就兌五兩來。」愛姐見他依允,還了他金簪子。兩個又坐了半日。恐怕人談論,喫了一盃茶,愛姐留喫午飯,經濟道:「我那邊有事,不喫飯了。少間就送盤纏來與你。」愛姐道:「午後奴畧備一盃水酒,官人不要見卻,好歹來坐坐。」經濟在店中喫了午飯,又在街上閒散走了一囘,撞見昔日晏公廟師兄金宗明,作揖,把前事訴說了一遍。金宗明道:「不知賢弟在守備老爺府中認了親,在大酒樓開大店,有失拜望!明日就使徒弟送茶來。閒中請去廟中坐一坐。」說罷,宗明歸去了。

經濟走到店中,陸主管道:「裏邊住的老韓,請官人喫酒,沒處尋。」恰好八老又來請:「官人,就請二位主管相陪,再無他客。」經濟就同二主管走到裏邊房内,早已安排酒席齊整,無非魚肉菜菓之類。經濟上坐,韓道國主位,陸秉義謝胖子打横,王六兒與愛姐旁邊僉坐,八老往來篩酒下菜。喫過數盃,兩個主管會意,說道:「官人慢坐,小人櫃上看去。」起身去了。經濟平昔酒量不十分洪飲,又見主管去了,開懷與韓道國三口兒喫了數盃,便覺有些醉將上來。愛姐便問:「今日官人不囘家去罷了。」經濟道:「這早晚了,回去不得,明日起身去罷。」王六兒韓道國喫了一囘,下樓去了。經濟向袖中取出五兩銀子,遞與愛姐收了,到下邊交與王六兒。兩個交盃換盞,倚翠偎紅,喫至天晚。愛姐卸下濃粧,留經濟就在樓上閣兒裏歇了。當下枕畔山盟,衾中海誓,鶯聲燕語,曲盡綢繆,不能悉記。愛姐將來東京,在蔡太師府中曾扶持過翟管家老太太,也學會些彈唱,又能識字會寫,訴說一遍。經濟聽了,歡喜不勝,就同六姐一般,正可在心上,以此與他盤桓一夜,停眠整宿。免不的第二日起來得遲,約飯時纔起來。王六兒安排些鷄子肉圓子,做了個頭腦,與他扶頭。兩個喫了幾盃煖酒。少頃,主管來請經濟,那邊擺飯。經濟包巾梳洗穿衣,喫了飯,又來辭愛姐,要囘家去,那愛姐不捨,只顧抛淚。經濟道:「我到家三五日就來看你,你休煩惱。」說畢,伴當跟隨,騎馬往城中去了。一路上吩咐小喜兒:「到家休要說出韓家之事!」小喜兒道:「小的知道,不必吩咐。」經濟到府中,只推店中買賣忙,算了帳目,不覺天晚,歸來不得,歇了一夜。交割與春梅利息銀兩,現一遭也有三十兩銀子之數。囬到家中,又被葛翠屏聐聒:「官人怎的外邊歇了一夜?是必在柳陌花街行踏,把我丢在家中,獨自空房一個,就不思想來家!」一連留住陳經濟七八日,不放他往河下來。

這裏韓愛姐見他一去數日光景,不來店中,只使小喜兒來問主管討算利息,主管一一封了銀子去。韓道國免不得又教老婆王六兒,又招惹別的熟人兒,或是商客,來屋裏走動,喫茶喫酒。這韓道國當先嚐着這個甜頭,靠老婆衣飯肥家。况此時王六兒年約四十五六,年紀雖半,風韻猶存;恰好又得他女兒來接代,也不斷絶這樣行業,如今索性大做了:原來不當官身,衣飯别無生意,只靠老婆賺錢,謂之隱名娼妓,今時呼為私窠子是也。當時見經濟不來,量酒陳三兒替他勾了一個湖州販絲綿客人何官人來,請他女兒愛姐。那何官人年約五十餘歲,手中有千兩絲綿紬絹貨物,要請愛姐。愛姐一心想着經濟,推心中不快,三囘五次不肯下樓來。急的韓道國了不的。那何官人又見王六兒長挑身材,紫膛色瓜子面皮,描眉鋪鬢,大長水鬢,涎鄧鄧一雙星眼,眼光如醉,抹的鮮紅嘴唇,料此婦人一定好風情,就留下一兩銀子,在屋裏喫酒,和王六兒歇了一夜。韓道國便躲避在外間歇了。他女兒見做娘的留下客,只在樓上,不下樓來。自此以後,那何官人被王六兒搬弄得快活,兩個打得一似火炭般熱,沒三兩日不來與婦人過夜。韓道國也禁過他許多錢使。

這韓愛姐兒見經濟一去十數日不見來,心中思想,挨一日似三秋,盼一夜如半夏,未免害木邊之目,田下之心。使八老往城中守備府中探聽,看見小喜兒,悄悄問他:「官人如何不去?」小喜兒說:「官人這兩日有些身子不快,不曾出門。」囘來訴與愛姐。愛姐與王六兒商議,買了一副猪蹄、兩隻燒鴨、兩尾鮮魚、一盒酥餅,在樓上磨墨揮筆,拂開花箋,寫封柬帖,使八老送到城中與經濟去。當下把禮物裝在盒内,交八老挑着,叮嚀囑付:「你到城中,見了陳官人,須索見他親收,討囘帖來。」八老懷内揣着柬帖,挑着禮物,一路無詞,來到城内守備府前,坐在沿街石臺基上。只見伴當小喜兒出來,看見八老:「你又來做甚麽?」八老與他聲喏,拉在僻凈處說:「我特來見你官人,送禮來了,有話說。我只在此等你,你可通報官人知道。」小喜隨即轉身進去。不多時,只見經濟搖將出來。那時約五月,天氣暑熱,經濟穿着紗衣服,頭戴瓦楞帽,金簪子,脚上凉鞋凈襪。八老慌忙聲喏,說道:「官人,貴體好些?韓愛姐使我捎一柬帖,送禮來了。」經濟接了柬帖說:「五姐好麽?」八老道:「五姐見官人一向不去,心中也不快在那裏。多上覆官人,幾時下去走走。」經濟拆開柬帖,觀看上面寫着甚言詞:

\begin{myquote}[\markfont]
「賤妾韓愛姐歛袵拜,謹啟

情郎陳大官人台下:

自別尊顏,思慕之心,未嘗少怠,懸懸不忘於心。向蒙期約,妾倚門凝望,不見降臨蓬蓽。昨遣八老探問起居,不遇而囘。聽聞貴恙欠安,令妾空懷悵望,坐臥悶懨,不能頓生兩翼,而傍君之足下也。君在家自有嬌妻美愛,又豈肯動念於妾,猶吐去之菓核也。茲具腥味茶盒數事,少申問安誠意。幸希笑納。情照不宣。

外具錦綉鴛鴦香囊一個,青絲一縷,少表寸心。

\raggedleft{{\kaishu(下書)}仲夏念日賤妾愛姐再拜。」}
\end{myquote}

經濟看了柬帖並香囊,香囊裏面,安放青絲一縷,香囊是鴛鴦雙口做的,扣着「寄與情郎陳君膝下」八字。依先摺了,藏在袖中。府傍側首有個酒店,令小喜兒領八老同到店内喫鍾酒:「等我寫回帖與你。」吩咐小喜兒:「把禮物收進我房裏去。你娘若問,只說河下店主人謝家送的禮物。」小喜不敢怠慢,把四盒禮物收進去了。經濟走到書院房内,悄悄寫了囘柬,又包了五兩銀子,到酒店内問八老:「喫了酒不曾?」八老道:「多謝官人好酒。喫不得了,起身去罷。」經濟將銀子並囘柬付與八老說:「到家多多拜上五姐,這五兩白金與他盤纏。過三兩日,我自去看他。」八老收了銀柬下樓。經濟送出店門,八老一直去了。經濟走入房中,葛翠屏便問:「是誰家送的禮物?」經濟悉言:「店主人謝胖子,打聽我不快,送這禮物來問安。」翠屏亦信其實。兩口兒計較,教丫鬟金錢兒拏盤子,拏了一隻燒鴨,一尾鮮魚,半副蹄子,送到後邊與春梅喫,說是店主人家送的,也不查問。此事表過不題。

卻說八老到河下,天已晚了,入門將銀柬都付與愛姐收了。拆開囘柬,燈下觀看,上面寫道:

\begin{myquote}[\markfont]
「經濟頓首,字覆

愛卿韓五姐粧次:向蒙會問,又承厚款,亦且雲情雨意,袵席鍾愛,無時少怠。所云期望,正欲趨會,偶因賤軀不快,有失卿之盼望,又蒙遣人垂顧,兼惠可口佳餚,不勝感激。只在二三日間,容當面布。外具白金五兩,綾帕一方,少申遠芹之敬。伏乞心鑒,萬萬!

\raggedleft{{\kaishu(下書)}經濟再拜。」}
\end{myquote}

愛姐看了,見帕上寫着四句詩曰:

\begin{myquote}
「吳綾帕兒織迴紋,洒翰揮毫墨跡新。

寄與多情韓五姐,永諧鸞鳳百年情。」
\end{myquote}

看畢,愛姐把銀子付與王六兒,母子千歡萬喜等候經濟,不在話下。正是:得意友來情不厭,知心人至話相投。有詩為證:

\begin{myquote}
碧紗窻下啟箋封,一紙雲鴻香氣濃。

知你揮毫經玉手,相思都付不言中。
\end{myquote}

畢竟未知後來何如,且聽下囘分解。

