\includepdf[pages={123,124},fitpaper=false]{tst.pdf}
\chapter*{第六十二囬 \\潘道士解禳祭燈壇 西門慶大哭李瓶兒}
\addcontentsline{toc}{chapter}{第六十二囬 潘道士解禳祭燈壇 西門慶大哭李瓶兒}
\markboth{{\titlename}卷之七}{第六十二囬 潘道士解禳祭燈壇 西門慶大哭李瓶兒}


\begin{myquote}
行藏虚實自家知,祸福因由更問誰?

善惡到頭終有報,只爭來早與來遲!

閒中點檢平生事,靜裏思量日所為:

常把一心行正道,自然天理不相虧。
\end{myquote}

話説西門慶見李瓶兒服薬百般醫治無效,求神問卜發課皆有兇無吉,無法可䖏。初時李瓶兒還にな着梳頭洗臉,還自己下炕來坐淨桶;次後漸漸飲食減少,形容消瘦,下邊流之不止,那消幾時,把個花朵般人兒,瘦弱的不好看,也不起的炕了,只在裀褥上舖墊草紙。恐怕人進來嫌穢惡,敎丫頭燒下些香在房中。西門慶見他胳膊兒瘦的銀條兒相似,守着在房内哭泣,衙門中隔日去走一走。李瓶兒便道:「我的哥,你還往衙門中去,只怕悮了你公事。我不妨事,只喫下邊流的虧。若得止住不流了,再把口裏放開,喫下些飲食兒,就好了。你男子漢,常絆住你在房中,守着甚麽!」西門慶哭道:「我的姐姐,我見你不好,心中捨不的你!」李瓶兒道:「好儍子!又不死,死將來你攔的住那些!」又道:「我要對你説也沒與你説:我不知怎的,但沒人在房裏,心中只害怕,恰似影影綽綽有人在我跟前一般。夜裏要便夢見他,恰似好時的,拿刀弄杖,和我廝嚷。孩子也在他懷裏,我去奪,反被他推我一跤。説他那裏又買了房子,來纏了好幾遍,只呌我去。只不好對你説。」西門慶聽了,説道:「人死如燈滅。這幾年知道他往那裏去了!此是你病的久了,下邊流的你這神虚氣弱了。那裏有甚麽邪魔魍魎、家親外祟?我明日往吴道官廟裏討兩道符來,貼在這房門上,看有邪祟沒有!」説話中間,走到前邊,即差玳安騎頭口往玉皇廟討符去。

玳安走到路上,迎見應伯爵和謝希大,忙下頭口。因問:「你爹在家裏?」玳安道:「爹在家裏。」又問:「你往那裏去?」玳安道:「小的往玉皇廟討符去。」伯爵與謝希大到西門慶家,因説道:「謝子純聽見嫂子不好,唬了一跳,敬來問安。這兩日較好些?」西門慶告訴道:「身上瘦的通不像模樣了。丢的我上不上,下不下,卻怎生樣好!孩子死了,隨他罷了,成夜只是哭,生生憂慮出病兒來了。勸着又不依你,教我有甚法兒處!」伯爵道:「哥,你又使玳安往廟裏做甚麽去?」西門慶悉把李瓶兒房中無人害怕之事,告訴一遍:「只恐有邪祟,教小廝問吴道官那裏討兩道符來,貼在房中,鎮壓鎮壓。」謝希大道:「哥,此是嫂子神氣虚弱,那裏有甚麽邪祟魍魎來!」伯爵道:「哥若遣邪,也不難。門外五嶽觀潘道士,他受的是天心五雷法,極遣的好邪,有名喚做潘捉鬼,常將符水救人。哥,你差人請請他來,看看嫂子房裏有甚邪祟,他就知道。你就教他治病,他也治得。」西門慶道:「等討了吴道官符來,看在那裏住,沒奈何,你就領小廝騎了頭口請了他來。」伯爵道:「不打緊,等我去。天可憐見,嫂子好了,我就頭着地也走。」説了一囬話,伯爵和希大喫了茶,起身自勾當去了。

玳安兒討了符來,貼在房中。晚間,李瓶兒還害怕,對西門慶説:「死了的他剛纔和兩個人來㧱我。見你進來,躲出去了。」西門慶道:「你休信邪,不妨事。昨日應二哥説,此是你虚極了。他説門外五嶽觀有個潘道士,好符水治病,又遣的好邪。我明日早教應二哥去請他來看你,有甚邪祟,敎他遣遣。」李瓶兒道:「我的哥哥,你請他早早來。那廝他剛纔發恨而去,明日還來㧱我哩!你快些使人請去!」西門慶道:「你若害怕,我使小廝㧱轎子接了吴銀兒和你做兩日伴兒。」李瓶兒搖頭兒,説:「你不要呌他,只怕悮了他家裏勾當。」西門慶道:「呌老馮來伏侍你兩日兒如何?」李瓶兒點頭兒。這西門慶一面使來安往那邊房子裏呌馮媽媽,又不在,鎖了門出去了。與一丈青説下:「等他來,好歹敎他快來宅内,六娘呌他哩。」西門慶一面又差下玳安:「明日早起,你和應二爹往門外五嶽觀請潘道士去。」俱不在話下。

次日,只見觀音庵王姑子挎着一盒兒粳米、二十塊大乳餅、一小盒兒十香瓜茄來看。李瓶兒見他來,連忙敎迎春搊扶起來坐的。王姑子道了問訊,李瓶兒請他坐下,道:「王師父,你自印經時去了,影邊兒通不見你。我恁不好,你就不來看我看兒?」王姑子道:「我的奶奶,我通不知你不好。昨日他大娘使了大官兒到庵裏,我纔曉得的。又説印經來,你不知道,我和薛姑子老淫婦合了一場好氣!與你老人家印了一場經,只替他趕了網兒。背地裏和印經家打了一兩銀子夾帳,我通沒見一個錢兒!你老人家作福,這老淫婦到明日墮阿鼻地獄!為他氣的我不好了,把大娘的壽日都悮了,沒曾來。」李瓶兒道:「他各人作業,隨他罷,你休與他爭執了。」王姑子道:「誰和他爭執甚麽!」李瓶兒道:「大娘好不惱你哩,説你把他受生的經都悮了。」王姑子道:「我的菩薩,我雖不好,敢悮了他的經?在家整誦了一個月受生,昨日纔圓滿了。今日纔來,先到後邊見了他,把我這些屈氣告訴了他一遍。我説不知他六娘不好,沒甚麽,這盒粳米和些十香瓜茄,幾塊乳餅,與你老人家喫粥兒。大娘纔敎小玉姐領我來看你老人家。」小玉打開盒兒,與李瓶兒看了,説道:「多謝你費心。」王姑子道:「迎春姐,你把這乳餅就蒸兩塊兒來,我親看你娘喫些粥兒。」那迎春一面收下去了。李瓶兒吩咐迎春擺茶來與王師父喫。王姑子道:「我剛纔後邊大娘屋裏喫了茶。煎些粥米,我看着你喫些粥兒。」不一時,迎春安放桌兒,擺了四樣茶食,打發王姑子喫了。然後㧱上李瓶兒粥來,一碟十香甜醬瓜茄,一碟蒸的黃霜霜乳餅,兩盞粳米粥。一雙小牙筷迎春㧱着,奶子如意兒在旁㧱着甌兒,喂了半日,只呷了兩三口粥兒,咬了一些乳餅兒,就搖頭兒不喫了,敎:「㧱過去罷。」王姑子道:「人以水食為命。恁煎的好粥兒,你再喫些兒不是!」李瓶兒道:「也得我喫的下去是的。」迎春便把喫茶的桌兒掇過去。

王姑子揭開被,看李瓶兒身上ざ體,都瘦的沒了,唬了一跳,說道:「我的奶奶,我去時你好些了,如何又不好了,就瘦得恁樣的了!」如意兒道:「可知好了哩。娘原是氣惱上起的病,爹請了太醫來看,每日服薬,已是好到七八分了。只因八月内,哥兒着了驚唬不好,娘晝夜憂慼,那樣勞碌,連睡也不得睡。實指望哥兒好了,不想没了。成日着了那哭,又着了那暗氣暗惱在心裏,就是鐵石人也禁不的,怎的不把病又犯了!是人家有些氣惱兒,對人前分解分解也還好;娘又不出語,着緊問還不說哩!」王姑子道:「那討氣來?你爹又疼他,你大娘又敬他。左右是五六位娘,端的誰氣着他?」奶子道:「王爺,你不知道誰氣着他?」因使綉春外邊瞧瞧,「看関着門不曾。路上説話,不備草裏有人。俺娘都因為着了那邊五娘一口氣!他那邊貓撾了哥兒手,生生的唬出風來。爹來家那等問着,娘只是不説。落後大娘説了,纔把那貓來摔殺了。他還不承認,㧱俺們煞氣!八月裏哥兒死了,他每日那邊指桑樹罵槐樹,百般稱快。俺娘這屋裏分明聽見,有個不惱的?左右背地裏氣,只是抹眼淚!因此這樣暗氣暗惱,纔致了這一場病。天知道罷了!娘可是好性兒,好也在心裏,歹也在心裏。姊妹之間,自來沒有個面紅面赤。有件稱心的衣裳,不等的别人有了,他還不穿出來。這一家子,那個不叨貼娘些兒?可是説的,饒叨貼了娘的,還背地不道是。」王姑子道:「怎的不道是?」如意兒道:「像五娘那邊,潘姥姥來一遭,遇着爹在那邊歇,就過來這屋裏和娘做伴兒,臨去,娘與他鞋面、衣服、銀子,甚麽不與他?五娘還不道是!」李瓶兒聽見,便嗔如意兒:「你這老婆,平白只顧説他怎的?我已是死去的人了,隨他罷了!天不言而自高,地不言而自卑。」王姑子道:「我的佛爺,誰知道你老人家這等好心!天也有眼,望下看着哩。你老人家往後來還有好處!」李瓶兒道:「王師父,還有甚麽好䖏!一個孩兒也存不住,去了。我如今又不得命,身底下弄這等疾,就是做鬼,走一步也不得個伶俐!我心裏還要與王師父些銀子兒,望你到明日我死了,你替我在家請幾位師父,多誦些《血盆經》,懺我這罪業。還不知墮多少罪業哩!」王姑子道:「我的菩薩,你老人家忒多慮了!天可憐見,到明日假若好了是的。你好心人,龍天自有加護。」正説着,只見琴童兒進來對迎春説:「爹吩咐把房内收拾收拾,花大舅便進來看娘,在前邊坐着哩。」王姑子便起身説道:「我且往後邊走走去。」李瓶兒道:「王師父,你休要去了,與我做兩日伴兒,我還和你説話哩。」王姑子道:「我的奶奶,我不去。」

不一時,西門慶陪花大舅進來看問,見李瓶兒睡在炕上不言語,花子由道:「我不知道,昨日聽見這邊大官兒去説,纔曉的。明日你嫂子來看你。」那李瓶兒只説了一聲:「多有起動。」就把面朝裏去了。花子由坐了一囬,起身到前邊,向西門慶説道:「俺過世公公老爺,在廣南鎮守,带的那三七薬,曾喫了不曾?不拘婦女甚崩漏之疾,用酒調五分末兒,喫下去即止。大姐他手裏有收下此薬,何不服之?」西門慶道:「這薬也喫過了。昨日本府胡大尹來拜,我因説起此疾,他也得了個方兒,棕灰與白鷄冠花煎酒服之,只止了一日。到第二日,流的比常更多了。」花子由道:「這個就難為了。姐夫,你早替他看下副板兒,預備他罷。明日敎他嫂子來看他。」説畢起身,西門慶再三款留不住,作辭去了。

奶子與迎春正與李瓶兒墊草紙在身底下,只見馮媽媽來到,向前道了萬福。如意兒道:「馮媽媽貴人,怎的不來看看娘?昨日爹使來安兒呌你去來,説你鎖着門,往那裏去來?」馮婆子道:「説不得我這苦,成日往廟裏修法。早晨出去了,是也直到黑,不是也直到黑,來家偏有那些張和尚、李和尚、王和尚。」如意兒道:「你老人家怎的這些和尚?早是沒王師父在這裏!」那李瓶兒聽了,微笑了一笑兒,説道:「這媽媽子,單管只撒風!」和意兒道:「馮媽媽,呌着你還不來。娘這幾日粥兒也不喫,只是心内不耐煩。你剛纔來到,就引的娘笑了一笑兒。你老人家伏侍娘兩日,管情娘這病就好了。」馮媽媽道:「我是你娘退灾的博士!」又笑了一囬。因向被窝裏摸了摸他身上,説道:「我的娘,你好些兒也罷了!」又問:「坐榪子還下的來?」迎春道:「下的來倒好。前兩遭娘還にな,俺們搊扶着下來。這兩日通只在炕上舖墊草紙,一日換兩三遍。」如意兒道:「本等沒喫甚麽大食力,怎禁的這等流!」正説着,只見西門慶進來,看見馮媽媽,説道:「老馮,你也常來這邊瞧瞧,怎的去了就不來?」婆子道:「我的爺,我怎不來?這兩日醃菜的時候,掙兩個錢兒,醃些菜在屋裏,遇着人家領來的業障,好與他喫。不然,我那討閒錢買菜兒與他喫?」西門慶道:「你不對我説,昨日俺莊子上起菜,撥兩三畦與你也夠了。」婆子道:「又敢纏你老人家?」説畢,老馮過那邊屋裏去了。

西門慶便坐在炕沿上,迎春在旁薰爇芸香。西門慶便問:「你今日心裏覺怎樣?」又問迎春:「你娘早晨喫了些粥兒不曾?」迎春道:「喫的倒好。王師父送了乳餅,蒸來,娘只咬了一些兒,呷了不上兩口粥湯,就丢下了。」西門慶道:「剛纔應二哥小廝門外請那潘道士,又不在了。明日我敎來保騎頭口再請去。」李瓶兒道:「你上緊着人請去。那廝但合上眼,只在我跟前纏。」西門慶道:「此是你神弱了。只把心放正着,休要疑影他。管情請了他,替你把這那祟遣遣,再服他些薬兒,你就好了。」李瓶兒道:「我的哥哥,奴已是得了這個拙病,那裏好甚麽!若好,只除非再與你兩世為人是的。奴今日無人䖏,和你説些話兒:奴指望在你身邊團圓幾年,死了也是做夫妻一場!誰知到今二十七歲,先把寃家死了。奴又沒造化,這般不得命,抛閃了你去了。若得再和你相逢,只除非在鬼門関上罷了!」説着,一把拉着西門慶手,兩眼落淚,哽咽再哭不出聲來。那西門慶亦悲慟不勝,哭道:「我的姐姐,你有甚話,只顧説。」兩個正在屋裏哭,忽見琴童兒進來,説:「答應的禀爹:明日十五,衙門裏拜牌,畫公座,大發放,爹去不去?班頭好伺候。」西門慶道:「我明日不得去。㧱我帖兒,囬你夏老爹,自家拜了牌罷。」琴童應諾去了。李瓶兒道:「我的哥哥,你依我,還往衙門去,休要悮了,你公事要緊。我知道幾時死,還早哩。」西門慶道:「我不在家守你兩日兒,其心安忍!你把心來放開,不要只管多慮了。剛纔他花大舅和我説,教我早與你看下副壽木,冲你冲,管情你就好了。」李瓶兒點頭兒,便道:「也罷,你休要信着人,使那憨錢。將就使十來兩銀子,買副熟料材兒,把我埋在先頭大娘墳旁,只休把我燒化了,就是夫妻之情。早晚我就搶些漿水,也方便些。你偌多人口,往後還要過日子哩!」這西門慶不聽便罷,聽了如刀剜肝膽,劍挫身心相似,哭道:「我的姐姐,你説的是那裏話?我西門慶就窮死了,也不肯虧負了你!」正説着,只見月娘親自㧱着一小盒兒鮮蘋婆進來,説道:「李大姐,他大妗子那裏,送蘋婆兒來與你喫。」因令迎春:「你洗淨了,㧱刀兒切塊來你娘喫。」李瓶兒道:「又多謝他大妗子掛心!」不一時,迎春旋去皮兒,切了,用甌兒盛貯,西門慶與月娘在旁看着,拈喂了一塊與他,放在口内只嚼了些味兒,還吐出來了。月娘恐怕勞碌他,安頓他面朝裏,就睡了。

西門慶與月娘都出來外邊商議。月娘便道:「李大姐我看他有些沉重。你不早早與他看一副材板兒來預備着他,直到那臨時到節熱亂,又亂不出甚麽好板來,馬捉老鼠一般,不是那幹營生的道理。」西門慶道:「今日花大哥也是這般説。適纔我略與他提了提兒,他吩咐:『休要使多了錢,將就擡副熟板兒罷。你偌多人口,往後還要過日子!』倒把我傷心了這一會。我説一發請潘道士來看了再看板去罷。」月娘道:「你看沒分曉,一個人的形也脱了,関口都鎖住,勺水也不進來,還妄想指望好!咱一壁打鼓,一壁磨旗。幸的他若好了,把棺材就捨與人,也不值甚麽!」西門慶道:「旣是恁説……」同月娘到後邊,使小廝呌將賁四來,在廳上問他:「誰家有好材板?你和姐夫兩個㧱銀子看一副來。」賁四道:「大街上陳千户家,新到了幾副好板。」西門慶道:「旣有好板……」即令陳經濟:「你後邊問你娘要五錠大銀子來,你兩個看去。」那陳經濟少頃取了五錠元寳出來,同賁地傳去了。直到後晌纔來囬話。西門慶問:「怎的這咱纔來?」他二人囬説:「到陳千户家看了幾副板,都中等,又價錢不合。囬來到路上,撞見喬親家爹,説尚擧人家有一副好板。原是尚擧人父親在四川成都府做推官時帶來,預備他老夫人的。兩副桃花洞,他使了一副,只剩下這一副。墻磕底蓋堵頭俱全,共大小五塊,定要三百七十兩銀子,喬親家爹同俺們過去看了,板是無比的好板。喬親家與做擧人的講了半日,只退了五十兩銀子。不是明年上京會試用這幾兩銀子使,也還捨不得賣這副板。還看咱這裏要,別人家,定要三百五十兩。」西門慶道:「旣是你喬親家爹主張,兑三百二十兩擡了來罷,休要只顧搖鈴打鼓的了。」陳經濟道:「他那裏收了咱二百五十兩,還找與他七十兩銀子就是了。」一面問月娘又要出七十兩雪花銀子,二人去了。比及黄昏時分,只見許多閒漢,用大紅氈條裹着,擡板進門,放在前廳天井内。打開西門慶觀看,果然好板。隨即呌匠人來鋸開,裏面噴香,每塊五寸厚,二尺五寸寬,七尺五寸長,看了滿心歡喜。又旋尋了伯爵一道來看,向伯爵道:「這板也看得過了。」伯爵口不住只顧喝采,説道:「原説是姻緣板。大抵一物還有一主。嫂子嫁哥一場,今日情受這副材板夠了!」吩咐匠人:「你用心,只要做的好,你老爹賞你五兩銀子。」匠人道:「小人知道。」一面在前廳七手八脚,連夜趲造棺槨不題。伯爵囑來保:「明日早五更去請潘道士,他若來,就同他一答兒來,不可遲滯。」説畢,陪西門慶晚夕在前廳看着做材。到一更時分,纔家去了。西門慶道:「明日早些來,只怕潘道士來的早。」伯爵道:「我知道。」作辭出門去了。

卻説老馮與王姑子,晚夕都在李瓶兒屋裏相伴。只見西門慶前邊散了,進來看視,要在屋裏睡。李瓶兒不肯,説道:「沒的這屋裏齷齷齪齪的,他們都在這裏,不方便。你往别䖏睡去罷。」西門慶又見王姑子都在這裏,遂過那邊金蓮房中去了。李瓶兒教迎春把角門関了,上了栓。敎迎春點着燈,打開箱子,取出幾件衣服銀飾來,放在旁邊。先呌過王姑子來,與了他五兩一錠銀子、一疋紬子:「等我死後,你好歹請幾位師父,與我誦《血盆經懺》。」王姑子道:「我的奶奶,你忒多慮了。天可憐見,你只怕好了。」李瓶兒道:「你只收着,不要對大娘説我與你銀子,只説我與了你這疋紬子做經錢。」王姑子道:「我理會了。」於是把銀子和紬子接過來了。又喚過馮媽媽來,向枕頭邊也㧱過四兩銀子,一件白綾襖、黄綾裙,一根銀掠兒遞與他,説道:「老馮,你是個舊人,我従小兒你跟我到如今。我如今死了去,也沒甚麽,這一套衣服,並這件首飾兒,與你做一念兒。這銀子你收着,到明日做個棺材本兒。你放心,那房子等我對你爹説,你只顧住着,只當替他看房兒,他莫不就攆你不成!」馮媽媽一手接了銀子和衣服,倒身下拜,哭的説道:「老身沒造化了!有你老人家在一日,與老身做一日主兒;你老人家若有些好歹,那裏歸着!」李瓶兒又呌過奶子如意兒,與了他一襲紫紬子襖兒、藍紬裙,一件舊綾披襖兒,兩根金頭簪子,一件銀滿冠兒,説道:「也是你奶哥兒一場。哥兒死了,我原説的教你休撅上奶去,實指望我在一日,占用你一日。不想我又死去了!我還對你爹和你大娘説,到明日我死了,你大娘生了哥兒,也不打發你出去了,就敎接你的奶兒罷。這些衣物,與你做一念兒,你休要抱怨。」那奶子跪在地下,磕着頭,哭道:「小媳婦實指望伏侍娘到頭,娘自來沒曾大氣兒呵着小媳婦。還是小媳婦沒造化,哥兒死了,娘又這般病的不得命!好歹對大娘説,小媳婦男子漢又沒了,死活只在爹娘這裏答應了,出去投奔那裏?」説畢,接了衣服首飾,磕了頭起來,立在旁邊,只顧揩眼淚。李瓶兒一面呌過迎春綉春來,跪下,囑付道:「你兩個,也是你従小兒在我手裏答應一場。我今死去,也顧不得你們了。你們衣服都是有的,不消與你了。我每人與你這兩對金裹頭簪兒、兩枝金花兒,做一念兒。那大丫頭迎春,已是他爹收用過的,出不去了,我敎與你大娘房裏拘管着。這小丫頭綉春,我敎你大娘尋家兒人家,你出身去罷,省的觀眉説眼,在這屋裏敎人罵沒主子的奴才!我死了,就見出樣兒來了。你伏侍别人,還像在我手裏那等撒嬌撇癡,好也罷歹也罷了,誰人容的你?」那綉春跪在地下,哭道:「我娘,我就死也不出這個門!」李瓶兒道:「你看儍丫頭!我死了,你在這屋裏伏侍誰?」綉春道:「我守着娘的靈。」李瓶兒道:「就是我的靈,供養不久,也有個燒的日子。你少不的也還出去。」綉春道:「我和迎春都答應大娘。」李瓶兒道:「這個也罷了。」這綉春還不知甚麽,那迎春聽見李瓶兒囑付他,接了首飾,一面哭的言語説不出來。正是:流淚眼觀流淚眼,断腸人送断腸人。

當夜李瓶兒都把各人囑付了,到天明,西門慶走進房來。李瓶兒問:「買了我的棺材來了沒有?」西門慶道:「従昨日就擡了板來,在前邊做材哩,且沖你沖。你若好了,情願捨與人罷。」李瓶兒因問:「是多少銀子買的?休要使那枉錢,往後還過日子哩!」西門慶道:「沒多,只給了百十兩來銀子。」李瓶兒道:「也還多了,預備下與我放着。」那西門慶説了回出來,前邊看着做材去了。

只見吴月娘和李嬌兒先進房來,看見他十分沉重,便問道:「李大姐,你心裏卻怎樣的?」李瓶兒揝着月娘手,哭道:「大娘,我好不成了。」月娘亦哭道:「李大姐,你有甚麽話兒?二娘也在這裏,你和俺兩個説。」李瓶兒道:「奴有甚話説?奴與娘做姊妹這幾年,又沒曾虧了我。實承望和娘相守到白頭,不想我的命苦,先把個寃家沒了。如今不幸我又得了這個拙病死去了!我死之後,房裏這兩個丫頭無人收拘。那大丫頭已是他爹收用過的,教他往娘房裏伏侍娘。小丫頭,娘若要使喚,㽞下;不然,尋個單夫獨妻,與小人家做媳婦兒去罷,省的敎人罵沒主子的奴才!也是他伏侍奴一場。奴就死,口眼也閉。又奶子如意兒,再三不肯出去,大娘也看着奴分上,也是他奶孩兒一場,明日娘十月已滿,生下哥兒,就敎接他奶兒罷。」月娘道:「李大姐,你放寬心,都在俺兩個身上。説兇得吉,你若有些山高水低,迎春敎他伏侍我,綉春敎他伏侍二娘罷。如今二娘房裏丫頭不老實做活,早晚要打發出去,敎綉春伏侍他罷。奶子如意兒,旣是你説他沒投奔,咱家那裏占用不下他來?就是我有孩子沒孩子,到明日配上個小廝,與他做房家人媳婦也罷了。」李嬌兒在旁便道:「李大姐,你休只要顧慮,一切事都在俺兩個身上。綉春到明日過了你的事,我收拾房内伏侍我,等我擡擧他就是了。」李瓶兒一面敎奶子和兩個丫頭過來,與二人磕頭。那月娘由不得眼淚出。不一時,孟玉樓、潘金蓮、孫雪娥,都進來看他。李瓶兒都留了幾句姊妹仁義之言,不必細記。落後待的李嬌兒玉樓金蓮衆人都出去了,獨月娘在屋裏守着他。李瓶兒悄悄向月娘哭泣説道:「娘到明日生下哥兒,好生看養着,與他爹做個根蒂兒,休要似奴心粗,喫人暗算了!」月娘道:「姐姐,我知道。」看官聽説:只這一句話,就感觸月娘的心來。後次西門慶死了,金蓮就在家中住不牢者,就是想着李瓶兒臨終這句話。正是:惟有感恩並積恨,千年萬載不生塵。

正説話中間,只見琴童吩咐房中收拾焚下香,五嶽觀請了潘法官來了。月娘一面看着,敎丫頭收拾房中乾淨,伺候淨茶淨水,焚下百合眞香。月娘與衆婦女,都藏在那邊牀屋裏聽覷。不一時,只見西門慶領了那潘道士進來。怎生形相?但見:

\begin{myquote}
頭戴雲霞五嶽冠,身穿皂布短褐袍。腰繫雜色綵絲縧,背上横紋古銅劍。兩隻脚穿雙耳麻鞋,手執五明降鬼扇。八字眉,兩個杏子眼;四方口,一道落腮鬍。威儀凜凜,相貌堂堂。若非霞外雲遊客,定是蓬萊王府人。
\end{myquote}

只見進入角門,剛轉過影壁,恰走到李瓶兒房穿廊臺基下。那道士往後退訖兩步,似有呵叱之狀。爾語數四,方纔左右揭簾進入房中,向病榻而立。運雙睛,努力以慧通神目一視,仗劍手内,掐指步罡,念念有辭,早知其意。走出明間,朝外設下香案。西門慶焚了香。這潘道士焚符,喝道:「值日神將,不來等甚!」噀了一口法水去,見一陣狂風所過,一黄巾力士現於面前,但見:

\begin{myquote}
黄羅抹額,紫綉羅袍。獅蠻帶緊束狼腰,豹皮褌牢拴虎體。常遊雲路,每歷罡風。洞天福地片時過,嶽瀆酆都撚指到。業龍作孽,向海底以擒來;妖魅為殃,劈山穴而提出。玉皇殿上,稱為符使之名;北極車前,立有天丁之號。常在壇前護法,每來世上降魔。胸懸雷部赤銅牌,手執宣花金蘸斧。
\end{myquote}

那位神將,拱立階前。大言:「召吾神那廂使令?」潘道士便道:「西門氏門中,李氏陰人不安,投告於我案下。汝即與我拘當坊土地,本家六神,查考有何邪祟,即與我擒來,毋得遲滯!」言訖,其神不見。須臾,潘道士瞑目變神,端坐於位上,據案擊令牌,恰似問事之狀,久久乃止。出來,西門慶讓至前邊捲棚内,問其所以。潘道士便説:「此位娘子,惜乎為宿世寃愆所訴於陰曹,非邪祟也,不可擒之。」西門慶道:「法官,可解禳得麽?」潘道士道:「寃家債主,須得本人可捨則捨之,雖陰官亦不能強。」因見西門慶禮貌虔切,便問:「娘子年命若干?」西門慶道:「屬羊的,二十七歲。」潘道士道:「也罷,等我與他祭祭本命星壇,看他命燈何如。」西門慶問:「幾時祭?用何香紙祭物?」潘道士道:「就是今晚三更正子時,用白灰界畫,建立燈壇。以黄絹圍之,鎮以生辰壇斗,祭以五穀棗湯。不用酒脯,只用本命燈二十七盞,上浮以華蓋之儀,餘無他物。官人可齋戒青衣,在壇内俯伏行禮,貧道祭之。鷄犬皆関去,不可入來打攪。」這西門慶都一一備辦停當,就不敢進内。在書房中沐浴齋戒,換了淨衣。那日留應伯爵也不家去了,陪潘道士喫齋饌。

到三更天氣,建立燈壇完備。潘道士高坐在上,下面就是燈壇:按青龍、白虎、朱雀、玄武,上建三臺華蓋,周列十二宫辰,下首纔是本命燈,共合二十七盞。先宣念了投詞。西門慶穿青衣,俯伏階下。左右盡皆屏去,再無一人在左右。燈燭熒煌,一齊點將起來。那潘道士在法座上披下髮來,仗劍,口中念念有詞,望天罡,取眞炁,布步訣,躡瑤壇。正是:三信焚香三界合,一聲令下一聲雷。但見晴天星月朗燦,忽然一陣地黑天昏,捲棚四下皆垂着簾幕,須臾起一陣怪風所過,正是:

\begin{myquote}
非干虎嘯,豈是龍吟。彷彿入户穿簾,定是摧花落葉。推雲出岫,送雨歸川。雁迷失伴作哀鳴,鷗鷺驚羣尋樹杪。嫦娥急把蟾宫閉,列子空中呌救人。
\end{myquote}

大風所過三次,一陣冷氣來,把李瓶兒二十七盞本命燈盡皆刮滅,惟有一盞復明。那潘道士明明在法座上,見一個白衣人領着兩個青衣人従外進來。手裏持着一紙文書,呈在法案下。潘道士觀看,卻是地府勾批,上面有三顆印信。唬的慌忙下法座來,向前喚起西門慶來,如此這般説道:「官人,請起來罷。娘子已是獲罪於天,無所禱也。本命燈已滅,豈可復救乎?只在旦夕之間而已了。」那西門慶聽了,低首無語,滿眼落淚,哭泣哀告:「萬望法師搭救則個!」潘道士道:「定數難逃,難以搭救了!」就要告辭。西門慶再三款留:「等天明早行罷。」潘道士道:「出家人草行露宿,山棲廟止,自然之道。」西門慶不復強之,因令左右捧出布一疋,白金三兩,作經襯錢。潘道士道:「貧道奉行皇天至道,對天盟誓,不敢貪受世財,取罪不便。」推讓再四,只令小童收了布疋作道袍穿,就作辭而行。囑付西門慶:「今晚官人切記不可往病人房裏去,恐祸及汝身。慎之,慎之!」言畢,送出大門,拂袖而去。西門慶歸到捲棚内,看着收拾燈壇,見沒救星,心中甚慟,同伯爵坐的,不覺眼中淚出。伯爵道:「此乃各人稟的壽數。到此地位,強求不得,哥也少要煩惱。」因打四更時分,説道:「哥,你也辛苦了,安歇安歇罷。我且家去,明日再來。」西門慶道:「教小廝㧱燈籠送你去。」即令來安取了燈,送伯爵出去,関上門進來。

那西門慶獨自一個坐在書房内,掌着一枝蠟燭,心中哀慟,口裏只長吁氣。尋思道:「法官戒我休往房裏去,我怎生忍得!寜可我死了也罷,須得廝守着,和他説句話兒。」於是進入房中,見李瓶兒面朝裏睡。聽見西門慶進來,翻過身來,便道:「我的哥哥,你怎的就不進來了?」因問:「那道士點的燈怎麽説?」西門慶道:「你放心,燈上不妨事。」李瓶兒道:「我的哥哥,你還哄我哩。剛纔那廝領着兩個人,又來在我跟前鬧了一囬,説道:『你請法師來遣我,我已告准在陰司,决不容你!』發恨而去,明日便來㧱我也。」西門慶聽了,兩淚交流,放聲大哭道:「我的姐姐,你把心來放正着,休要理他。我實指望和你相伴幾日,誰知你又抛閃了我去了,寜敎我西門慶口眼閉了,倒也沒這等割肚牽腸!」那李瓶兒雙手摟抱着西門慶脖子,嗚嗚咽咽悲哭,半日哭不出聲,説道:「我的哥哥,奴承望和你並頭相守,誰知奴家今日死去也!趂奴不閉眼,我和你説幾句話兒。你家事大,孤身無靠,又沒幫手,凡事斟酌,休要那一冲性兒。大娘等,你也少要虧了他的。他身上不方便,早晚替你生下個根絆兒,庶不散了你家事。你又居着個官,今後也少要往那裏去喫酒,早些兒來家,你家事要緊。比不的有奴在,還早晚勸你。奴若死了,誰肯只顧的苦口説你?」西門慶聽了,如刀剜心肝相似,哭道:「我的姐姐,你所言我知道。你休掛慮我了。我西門慶那世裏絶緣短倖,今世裏與你夫妻不到頭。疼殺我也!天殺我也!」李瓶兒又説:「迎春綉春之事,奴已和他大娘説來,到明日我死,把迎春伏侍他大娘,那小丫頭,他二娘已承攬。他房内無人,便敎伏侍二娘罷。」西門慶道:「我的姐姐,你沒的説。你死了,誰人敢分散你丫頭?奶子也不打發他出去,都敎他守你的靈。」李瓶兒道:「甚麽靈!回個神主子,過五七兒燒了罷了。」西門慶道:「我的姐姐,你不要管他。有我西門慶在一日,供養你一日。」兩個説話之間,李瓶兒催促道:「你睡去罷,這早晚了!」西門慶道:「我不睡了,在這屋裏守你守兒。」李瓶兒道:「我死還早哩!這屋裏穢惡,薰的你慌。他們伏侍我不方便。」西門慶不得已,吩咐丫頭:「仔細看守你娘。」往後邊上房裏對月娘説,悉把祭燈不濟之事,告訴一遍:「剛纔我到他房中,我觀他説話兒還伶俐。天可憐,只怕還熬出來了也不見得!」月娘道:「眼眶兒也塌了,嘴唇兒也乾了,耳輪兒也焦了,還好甚麽?也只在早晚間了。他這個病,是恁伶俐,臨断氣還説話兒!」西門慶道:「他來了咱家這幾年,大大小小没曾惹了一個人,且是又好個性格兒,又不出語,你敎我捨得他那些兒!」題起來,又哭了。月娘亦止不住落淚。

不説西門慶與月娘説話。且説李瓶兒喚迎春奶子:「你扶我面朝裏略倒倒兒。」因問道:「天有多咱時分了?」奶子道:「鷄還未呌,有四更天了。」呌迎春替他鋪墊了身底下草紙,搊他朝裏,蓋被停當,睡了。衆人都熬了一夜沒曾睡,老馮與王姑子都已先睡了。那邊屋裏鎖着。迎春與綉春,在面前地坪上搭着鋪,那裏剛睡倒沒半個時辰,正在睡思昏沉之際,夢見李瓶兒下炕來,推了迎春一推,囑付:「你們看家,我去也。」忽然驚醒,見桌上燈尚未滅。向牀上視之,還面朝裏,摸了摸,口内已無氣矣:不知多咱時分,嗚呼哀哉,断氣身亡!可惜一個美色佳人,都化作一場春夢!正是:閻王敎你三更死,怎敢留人到五更。

迎春慌忙推醒衆人,點燈來照,果然見没了氣兒,身底下流血一窪。慌了手脚,走去後邊報知西門慶。西門慶聽見李瓶兒死了,和吴月娘兩步做一步奔到前邊,揭起被,但見面容不改,體尚微溫,脱然而逝,身上止着一件紅綾抹胸兒。這西門慶也不顧的甚麽身底下血漬,兩隻手抱着他香腮親着,口口聲聲只呌:「我的沒救星的姐姐,有仁義好性兒的姐姐!你怎的閃了我去了,寜可敎我西門慶死了罷。我也不久活於世了,平白活着做甚麽!」在房裏離地跳的有三尺高,大放聲號哭。吴月娘亦搵涙哭渧不止。落後李嬌兒、孟玉樓、潘金蓮、孫雪娥,合家大小丫鬟養娘,都擡起房子來也一般哀聲動地哭起來。月娘向李嬌兒孟玉樓道:「不知晚夕多咱死了,恰好衣服兒也不曾得穿一件在身上。」玉樓道:「娘,我摸他身上還溫溫兒的,也纔去了不多囬兒。咱不趂熱脚兒,不替他穿上衣裳,還等甚麽?」月娘因見西門慶磕伏在他身上,撾臉兒那等哭,只呌:「天殺了我西門慶了!姐姐,你在我家三年光景,一日好日子沒過,都是我坑陷了你了!」月娘聽了,心中就有些不耐煩了。説道:「你看韶刀,哭兩聲兒丟開手罷了!一個死人身上,也沒個忌諱,就臉撾着臉兒哭。倘忽口裏惡氣,撲着你怎的!他没過好日子,誰過好日子來?人死如燈滅。半晌時不借,留的住他倒好!各人壽數到了,誰人不打這條路兒來!」因令李嬌兒、孟玉樓:「你兩個㧱鑰匙,那邊屋裏尋他裝綁的衣服出來,咱眼看着與他穿上。」又呌:「六姐,咱兩個把這頭來替他整理整理。」西門慶又向月娘説:「多尋出兩套他心愛的好衣服,與他穿了去。」月娘吩咐李嬌兒玉樓:「你尋他新裁的大紅緞遍地錦襖兒,柳黄遍地金裙,並他今年喬親家去那套丁香色雲紬粧花衫、翠藍寬拖子裙,並新做的白綾襖、黄紬子裙出來罷。」當下迎春㧱着燈,孟玉樓㧱鑰匙,開了牀屋裏門,拔步牀上第二個描金箱子裏,都是新做的衣服。揭開箱蓋,玉樓李嬌兒尋了半日,尋出三套衣裳來。又尋出件綁身紫綾小襖兒,一件白紬子裙,一件大紅小衣兒,並白綾女襪兒,粧花膝褲腿兒。李嬌兒抱過這邊屋裏,與月娘瞧。月娘正與金蓮燈下替他整理頭髻,用四根金簪兒綰一方大鴉青手帕,旋勒停當。李嬌兒因問:「尋雙甚麽顏色鞋,與他穿了去?」潘金蓮道:「姐姐,他心裏只愛穿那雙大紅遍地金鸚鵡摘桃白綾高底鞋兒,只穿了没多兩遭兒。倒尋那雙鞋出來,與他穿了去罷。」吴月娘道:「不好。倒沒的穿上陰司裏好敎他跳火坑。你把前日門外往他嫂子家去,穿的那雙紫羅遍地金高底鞋,也是扣的鸚鵡摘桃鞋,尋出來與他裝綁了去罷。」這李嬌兒聽了,走來向他盛鞋的四個小描金箱兒,約百十雙鞋,翻遍了都沒有。迎春説:「俺娘穿了來,只放在這裏,怎的没有?」走來廚下問綉春。綉春道:「我看見娘包放在坐廚裏。」扯開坐廚子尋,還有一大包,都是新鞋。尋出來了,衆人七手八脚都裝綁停當。

西門慶率領衆小廝,在大廳上收捲書畫,圍上幃屏。把李瓶兒用板門擡出,停於正寢。下鋪錦褥,上覆紙被。安放几筵香案,點起一盞隨身燈來。專委兩個小廝在旁侍奉,一個打磬,一個燒紙。一面使玳安:「快請陰陽徐先生來看時批書。」月娘打點出裝綁衣服來,就把李瓶兒床房門鎖了,只留炕屋裏,交付與丫頭養娘。那馮媽媽見沒了主兒,哭的三個鼻頭,兩個眼淚。王姑子且口裏喃喃呐呐,替李瓶兒念《密多心經》、《薬師經》、《解寃經》、《楞嚴經》,並《大悲中道神咒》,請引路王菩薩與他接引冥途。西門慶在前廳,手拍着胸膛,由不的撫尸大慟,哭了又哭,把聲都呼啞了,口口聲聲只呌「我的好性兒有仁義的姐姐」不住。

比及亂着,鷄就呌了。玳安請了徐先生來,向西門慶施禮,説道:「老爹煩惱。奶奶没了,在於甚時候?」西門慶道:「因此時候不眞:睡下之時已打四更,房中人都困倦,睡熟了,不知多咱時分没了。」徐先生道:「此是第幾位奶奶?」西門慶道:「乃是第六的小妾。生了個拙病,淹淹纏纏,也這些時了。」徐先生道:「不打緊。」因令左右掌起燈,來廳上揭開紙被觀看,手掐丑更,説道:「正當五更二點徹,還屬丑時断氣。」西門慶即令取筆硯,請徐先生批書。這徐先生向燈下打開青囊,取出萬年曆通書來觀看,問了姓氏並生時八字,批將下來:「已故錦衣西門夫人李氏之丧,生於元祐辛未正月十五日午時,卒於政和丁酉九月十七日丑時。今日丙子,月令戊戌,犯重喪之日。煞高一丈,向西南方而去。遇太歲煞冲逥,斬之吉。避本家,忌哭聲,成服後無妨。入殮之時,忌龍虎鷄蛇四生人外,親人不避。」吳月娘使出玳安來,敎徐先生看看黑書上,往那方去了。這徐先生一面打開陰陽秘書觀看,説道:「今日丙子日,乃是己丑時死者。上應寳瓶宫,下臨齊地。前生曾在濱州王家作男子,打死懷胎母羊,今世為女人屬羊,稟性柔婉,自幼少陰謀之事。父母雙亾,六親無靠。先與人家作妾,受大娘子氣。及至有夫主,又不相投,犯三刑六害。中年雖招貴夫,常有疾病,比肩不和,生子夭亡。主生氣疾,肚腹流血而死。前九日魂去,托生河南汴梁開封府袁指揮家為女,艱難不能度日。後躭閣至二十歲,嫁一富家,老小不對。中年享福,壽至四十二歲,得氣而終。」看畢黑書,衆婦女聽了皆各嘆息。西門慶敎徐先生看破土安葬日期,徐先生請問:「老爹停放幾時?」西門慶哭道:「熱突突怎麽就打發出去的!須放過五七纔好。」徐先生道:「五七裏没有安葬日期。倒是四七裏,宜擇十月初八日丁酉午時破土,十二日辛丑巳時安葬。合家六位本命都不犯。」西門慶道:「也罷。到十月十二日發引,再沒挪移了。」徐先生當即寫殃榜,蓋伏死者身上,向西門慶道:「十九日辰時大殮,一應之物,老爹這裏備下。」

於是剛打發徐先生出了門,天已發曉。西門慶使琴童兒騎頭口往門外請花大舅,然後分班差家下人各親眷䖏報丧。又使人往衙門中給假,在家整理丧事。使玳安往獅子街取了二十桶瀼紗漂白,三十桶生眼布來,敎趙裁僱了許多裁縫,在西廂房先僱人造幃幕、帳子、桌圍,並入殮衣衾纏帶,各房裏女人衫裙。外邊小廝伴當,每人都是白唐巾,一件白直裰。又兑了一百兩銀子,敎賁四往門外店裏推了三十桶魁光麻布,二百疋黄絲孝絹。一面又敎搭綵匠在大天井内搭五間大棚。西門慶因想起李瓶兒動止行藏模檥兒來,心中忽然想起忘了與他傳神,呌過來保來問:「那裏有寫眞好畫師?尋一個傳神。我就把這件事忘了!」來保道:「舊時與咱家畫圍屏的韓先兒,他原是宣和殿上的畫士,革退來家。他傳的好神。」西門慶道:「他在那裏住?快與我請來。」這來保應諾去了。西門慶熬了一夜沒睡的人,前後又亂了一五更,心中又着了悲慟,神思恍亂,只是沒好氣,罵丫頭、踢小廝,守着李瓶兒屍首,由不的放聲哭呌。那玳安在傍亦哭的言不的語不的。

吴月娘正和李嬌兒、孟玉樓、潘金蓮,在帳子後,打夥兒分孝與各房裏丫頭並家人媳婦,看見西門慶只顧哭起來,把喉音也呌啞了,問他,與茶也不喫,只顧沒好氣。月娘便道:「你看恁勞叨!死也死了,你沒的哭的他活!哭兩聲丢開手罷了,只顧扯長絆兒哭起來了!三兩夜没睡,頭也沒梳,臉也還没洗,亂了恁五更,黄湯辣水還沒嚐着,就是鐵人也禁不的。把頭梳了出來喫些甚麽,還有個主張。好小身子,一時摔倒了却怎樣兒的?」玉樓道:「他原來還沒梳頭洗臉哩。」月娘道:「洗了臉倒好。我頭裏使小廝請他後邊洗臉,他把小廝踢進來,誰再問他來!」金蓮接過來道:「你還沒見,頭裏進他屋裏尋衣裳,敎我是不是倒好意説他,都像恁一個死了,你恁般起來,把骨禿肉兒也沒了。你在屋裏喫些甚麽兒,出去再亂也不遲。他倒把眼睜紅了的罵我:『狗攮的淫婦,管你甚麽事!』我如今鎮日不敎狗攮,却敎誰攮哩!恁不合理的行貨子,只説人和他合氣!」月娘道:「熱突突死了,怎麽不疼?你就疼也還放心裏。那裏就這般顯出來!人也死了,不管那有惡氣没惡氣,就口撾着口那等呌喚,不知甚麽張致!喫我説了兩句。他可可兒來三年沒過一日好日子?鎮日敎他挑水挨磨來?」孟玉樓道:「娘,不是這等説。李大姐倒也罷了,沒甚麽,倒喫了他爹恁三等九格的!」金蓮道:「他沒過好日子,那個偏受用着甚麽哩!都是一個跳板兒上人。」正説着,只見陳經濟手裏㧱着九疋水光絹:「爹説敎娘們剪各房裏手帕,剩下的與娘們做裙子。」月娘收了娟,便道:「姐夫,去請你爹進來扒口子飯,這咱七八待晌午,他茶水還沒嚐着哩!」經濟道:「我是不敢請他。頭裏小廝請他喫飯,差些沒一脚踢殺了。我又惹他做甚麽?」月娘道:「你不請他,等我另使人請他來喫飯。」良久,呌過玳安來,説道:「你爹還沒喫飯,哭這一日了。你㧱上飯去,趂溫先生在,陪他喫些兒。」玳安道:「請應二爹和謝爹去了,等他來時,娘這裏使人㧱飯上去,消不的他幾句言語兒,管情爹就喫了飯。」月娘道:「硶説嘴的囚根子!你是你爹肚裏蛔虫?俺們這幾個老婆,倒不如你了!你怎的就知道他兩個來纔喫飯?」玳安道:「娘們不知,爹的好朋友,大小酒席兒,那遭少了他兩個?爹三錢,他也是三錢,爹二星,他也是二星。爹隨問怎的着了惱,只他到,略説兩句話兒,爹就眉花眼笑的。」

説了一回,棋童兒請了應伯爵謝希大二人來到,進門撲倒靈前地下,哭了半日,只哭:「我的有仁義的嫂子!」被金蓮和玉樓罵道:「賊油嘴的囚根子,俺們都是沒仁義的!」二人哭畢,爬起來。西門慶與他回禮,兩個又哭了,説道:「哥煩惱,煩惱!」一面讓至廂房内,與溫秀才敍禮坐下。先是伯爵問道:「嫂子甚時候殁了?」西門慶道:「正丑時断氣。」伯爵道:「我到家已是四更多了。房下問我,我説:『看陰騭,嫂子這病已在七八了。』不想剛睡就做了一夢,夢見哥使大官兒來請我,説家裏喫慶官酒,敎我急急來到。見哥穿着一身大紅衣服,向袖中取出兩根玉簪兒與我瞧,説一根折了。教我瞧了半日,對哥説:『可惜了,這折了是玉的,完全的倒是硝子石。』哥説兩根都是玉的。俺兩個正説着,我就醒了,敎我説,此夢做的不好。房下見我只顧咂嘴,便問:『你和誰説話?』我道:『你不知,等我到天曉告訴你。』等到天明,只見大官兒到了,戴着白,教我只顧跌脚。果然哥有孝服!」西門慶道:「我前夜也做了恁個夢,和你這個一樣兒。夢見東京翟親家那裏寄送了六根簪子,内有一根ぼ折了。我説可惜兒的,教我夜裏告訴房下,不想前邊断了氣。好不睜眼的天,撇的我真好苦!寜可敎我西門慶死了,眼不見就罷了。到明日,一時半霎想起來,你敎我怎不心疼?平時我又沒曾虧欠了人,天何今日奪吾所愛之甚也!先是一個孩兒也沒了,今日他又長伸脚子去了,我還活在世上做甚麽!雖有錢過北斗,成何大用!」伯爵道:「哥,你這話就不是了。我這嫂子與你是那樣夫妻,熱突突死了,怎的不心疼?爭耐你偌大的家事,又居着前程,這一家大小泰山也似靠着你。你若有好歹,怎麽了得?就是這些嫂子都沒主兒。常言:一在三在,一亾三亾。哥你聰明,你伶俐,何消兄弟們説。就是嫂子他青春年少,你疼不過,越不過他的情,成服,令僧道念幾卷經,大發送葬埋在墳裏,哥的心也盡了,也是嫂子一場好事,再還要怎樣的?哥,你且把心放開。」當時被伯爵一席話,説的西門慶心地透徹,茅塞頓開,也不哭了。須臾,㧱上茶來喫了,便喚玳安:「後邊説去,看飯來,我和你應二爹、溫師父、謝爹喫。」伯爵道:「哥原來還未喫飯哩。」西門慶道:「自従你去了,亂了一夜,到如今誰嘗甚麽兒來!」伯爵道:「哥,你還不喫飯,這個就糊突了。常言道:寜可折本,休要饑損。《孝經》上不説的:『敎民無以死傷生,毀不滅性。』死的自死了,存者還要過日子。哥要做個張主!」正是:數語撥開君子路,片言題醒夢中人。

畢竟未知後來如何,且聽下囬分解。

