\includepdf[pages={81,82},fitpaper=false]{tst.pdf}
\chapter*{第四十一囬 \\西門慶與喬大戶結親 潘金蓮共李瓶兒鬬氣}
\addcontentsline{toc}{chapter}{第四十一囬 西門慶與喬大戶結親 潘金蓮共李瓶兒鬬氣}
\markboth{{\titlename}卷之五}{第四十一囬 西門慶與喬大戶結親 潘金蓮共李瓶兒鬬氣}


\begin{myquote}
富貴雙全世業隆,聯翩朱紫一門中。

官高位重如王導,家盛財豐比石崇。

畫燭錦帷消夜月,綺羅紅粉醉春風。

朝懽暮楽年年事,豈肯潛心任始終。
\end{myquote}

話説西門慶在家中,裁縫趲造衣服,那消兩日就完了。到十二日,喬家使人邀請。早晨,西門慶先送了禮去。那日月娘并衆姊妹、大妗子,六頂轎子一搭兒起身,留下孫雪娥看家。奶子如意兒抱着官哥,又令來興媳婦惠秀伏侍疊衣服,又是兩頂小轎。

西門慶在家,看着賁四叫了花兒匠來紮縛煙火,在大廳捲棚内掛燈。使小廝拏帖兒,往王皇親宅内定下戲子。俱不必細説。後响時分,走到金蓮房中,金蓮不在家。春梅在旁伏侍茶飯,放桌兒喫酒。西門慶因對春梅説:「十四日請衆官娘子,你們四個都打扮出去,與你娘跟着遞酒,也是好䖏。」春梅聽了,斜靠着桌兒説道:「你若叫,只叫他三個出去,我是不出去。」西門慶道:「你怎的不出去?」春梅道:「娘們都新裁了衣裳,陪侍衆官戶娘子,便好看。俺們一個一個,只像燒糊了卷子一般,平白出去惹人家笑話!」西門慶道:「你們都有各人的衣服首飾,珠翠花朵雲髻兒,穿戴出去。」春梅道:「頭上將就戴着罷了,身上有數那兩件舊片子,怎麽好穿出去見人的,倒沒的羞剌剌的!」西門慶笑道:「我曉的,你這小油嘴,見你娘們做了衣裳,都使性兒起來。不打緊,叫趙裁來,連大姐帶你四個,每人都替你裁三件。一套緞子衣裳,一件遍地錦比甲。」春梅道:「我不比與他。我還問你要件白綾襖兒,搭襯着大紅遍地錦比甲兒穿。」西門慶道:「你要,不打緊。少不的也與你大姐裁一件。」春梅道:「大姑娘有一件罷了,我卻沒有,他也説不的我。」西門慶於是拏鑰匙開樓門,揀了五套緞子衣服,兩套遍地金比甲兒,一疋白綾裁了兩件白綾對衿襖兒。惟大姐和春梅是大紅遍地錦比甲兒,迎春、玉簫、蘭香,都是藍綠顏色;衣服都是大紅緞子織金對衿襖,翠藍邊拖裙,共十七件。一面叫了趙裁來,都裁剪停當。又要一疋黄紗做裙腰,貼裏一色都是杭州絹兒。春梅方纔喜歡了,陪侍西門慶在屋裏喫了一日酒。按下家中不題。

且説吳月娘衆姊妹到了喬大戶家。原來喬大戶娘子,那日請了尚舉人娘子,幷左隣朱臺官娘子、崔親家母,幷兩個外甥侄女兒——段大姐及吳舜臣媳婦兒鄭三姐。叫了兩個妓女,席前彈唱。聽見月娘衆姊妹和吳大妗子到了,連忙出儀門首迎接,後廳敍禮,趕着月娘呼姑娘,李嬌兒衆人都排行叫二姑娘、三姑娘,稱着吳大妗子那邊稱呼之禮。也與尚舉人朱臺官娘子敍禮畢。段大姐鄭三姐向前拜見了,各依次坐下。丫鬟遞過了茶,喬大戸出來拜見,謝了禮。他娘子讓進衆人房中去寬衣服,就放桌兒擺茶。無非是蒸煠細巧茶食,菓餡點心、酥菓甜食,諸般菜蔬。擺設甚是齊整,請堂客坐下喫茶。奶子如意兒和惠秀在房中守着看官哥兒,另自管待。

須臾,喫了茶,到廳,屏開孔雀,褥隱芙蓉,正面設四張桌席。讓月娘坐了首位,其次就是尚舉人娘子、吳大妗子、朱臺官娘子、李嬌兒、孟玉樓、潘金蓮、李瓶兒、喬大戶娘子關席;坐位傍邊放一桌,是段大姐、鄭三姐,共十一位堂客。兩個妓女,在旁彈唱。上了湯飯,廚役上來獻了頭一道水晶鵝,月娘賞了二錢銀子。第二道是炖爛よ蹄兒,月娘又賞了一錢銀子。第三道獻燒鴨,月娘又賞了一錢銀子。喬大戶娘子下來遞酒,遞了月娘,過去,又遞尚舉人娘子。月娘就下來,往後房換衣服、匀臉去了。孟玉樓也跟下來。

到了喬大戶娘子臥房中,只見奶子如意兒看守着官哥兒,在炕上鋪着小褥子兒躺着。他家新生的長姐也在傍邊臥着。兩個你打我下兒,我打你下兒頑耍。把月娘玉樓見了喜歡的了不得,説道:「他兩個倒好像兩口兒!」只見吳大妗子進來,説道:「大妗子,你來瞧瞧,兩個倒像小兩口兒。」大妗子笑道:「正是!孩兒們在炕上張手兒蹬脚兒的,你打我,我打你,小姻緣一對兒耍子。」喬大戶娘子和衆堂客都進房來,吳妗子如此這般説,喬大戶娘子道:「列位親家聽着,小家兒人家,怎敢攀的我這大姑娘府上!」月娘道:「親家好説。我家嫂子是何人?鄭三姐是何人?我與你愛親做親,就是我家小兒,也玷辱不了你家小姐,如何卻説此話?」玉樓推着李瓶兒說道:「李大姐,你怎的說?」那李瓶兒只是笑。吳妗子道:「喬親家不依,我就惱了!」尚舉人娘子和朱臺官娘子皆説道:「難為吳親家厚情,喬親家你休謙辭了。」因問:「你家長姐,去年十一月生的?」月娘道:「我家小兒六月廿三日生的,原大五個月,正是兩口兒。」衆人於是不由分説,把喬大戶娘子和月娘李瓶兒拉到前廳,兩個就割了衫襟。兩個妓女彈唱着。旋對喬大戶説了,拏出菓盒、三段紅來遞酒。月娘一面吩咐玳安琴童,快往家中對西門慶説。旋擡了兩罈酒,三疋緞子紅,綠板兒絨金絲花,四個螺鈿大菓盒,兩家席前掛紅喫酒。一面堂中畫燭高檠,花燈燦爛,麝香靉靉,喜笑盈盈。席前兩個妓女,啓朱唇,露皓齒,輕撥玉阮,斜抱琵琶,唱一套〔鬬鵪鶉〕:

\begin{myquote}
「翡翠窻紗,鴛鴦碧瓦;孔雀銀屛,芙蓉綉榻;幕捲輕綃,香焚睡鴨。燈上下,簾上下,這的是南省尚書,東床駙馬。」

{\markfont〔紫花兒序〕}「帳前軍朱衣畫戟,門下士錦帶吳鉤,坐上客綉帽宫花。按教坊歌舞,依内苑奢華。板撥紅牙,一派簫韶准備下。立兩行美人如畫,粉面銀箏,玉手琵琶。」

{\markfont〔金蕉葉〕}「我則見銀燭明燒絳蠟,纖手高擎着玉斝。我見他舉止處堂堂俊雅,我去那燈影兒下,孜孜的覷着。」

{\markfont〔調笑令〕}「這生那裏我曾見他,莫不我眼睛花?呀!我這裏手抵着牙兒試記咱:不由我眼兒裏見了他心牽掛,莫不是五百年前歡喜寃家?是何處綠楊曾繫馬,莫不是夢兒中雲雨巫峽? 」

{\markfont〔小桃紅〕}「玉簫吹徹碧桃花,一刻千金價。燈影兒裏斜將眼梢兒抹,唬的我臉烘霞。酒盃中嫌殺春風凹,玉簫年當二八,未當招嫁,俺相公培養出牡丹芽。」

{\markfont〔鬼三台〕}「他說幾句淒涼話,我淚不住行兒般下,鎖不住心猿意馬。我是個嬌滴滴洛陽花,險些露出風流的話靶。這言詞道耍不是耍,這公事道假不是假。他那裏拔樹尋根,我這裏指鹿道馬!」

{\markfont〔秃廝兒〕}「我勸他似水底納瓜,他覷我似鏡裏觀花。更做道書生自來情性耍,調戯咱好人家嬌娃。」

{\markfont〔聖薬王〕}「你着我怎救他?難按納,公孫弘東閣鬧喧嘩:散了玳瑁筵,漾了這鸚鵡斝,踢番了銀燭絳籠紗,扯三尺劍離匣。」

{\markfont〔尾聲〕}「従來這秀才們色膽天來大,把俺這小膽文君唬殺。忒火性卓王孫,強風情漢司馬。」
\end{myquote}

當下衆堂客與吳月娘、喬大戶娘子、李瓶兒,三人都簪了花,掛了紅,遞了酒。各人都拜了。重新復安席,坐下飲酒。廚子上了一道菓餡壽字雪花糕,喜重重滿池嬌並頭蓮湯,割了一道燒花豬肉。月娘坐在上席,滿心歡喜,叫玳安過來,賞一疋大紅與廚役;兩個妓女每人都是一疋。俱磕頭謝了。喬大戶娘子還不放起身,還在後堂留坐,擺了許多勸碟,細菓攢盒。

約喫到一更時分,月娘等方纔拜辭回家,説道:「親家,明日好歹下降寒舍,那裏久坐坐。」喬大戶娘子道:「親家盛情!家老兒説來,只怕席間不好坐的,改日望親家去罷。」月娘道:「好親家,再沒人,親家只是見外!」因留了大妗子:「你今日不去,明日同喬親家一搭兒裏來罷。」大妗子道:「喬親家,別的日子你不去罷,到十五日,你正親家生日,你莫不也不去?」喬大戶娘子道:「親家十五日好的日子,我怎敢不去。」月娘道:「親家若不去,大妗子,我交付與你,只在你身上!」於是生死把大妗子留下了,然後作辭上轎。頭裏兩個排軍打着兩個大紅燈籠,後邊又是兩個小廝,打着兩個燈籠,喝的路走。吳月娘在頭裏,李嬌兒、孟玉樓、潘金蓮、李瓶兒,一字在中間,如意兒和惠秀煞後。奶子轎子裏用紅綾小被把官哥兒裹得嚴嚴的,恐怕冷,脚下還蹬着銅火爐兒。兩邊小廝圜随,到了家門首下轎。

西門慶正在上房喫酒。月娘等衆人進來,道了萬福,坐下。衆丫鬟都來磕了頭。月娘先把今日酒席上結親之話告訴了一遍。西門慶聽了,問道:「今日酒席上,有那幾位堂客?」月娘道:「有尚舉人娘子、朱序班娘子、崔親家母、兩個姪女。」西門慶説:「做親也罷了,只是有些不搬陪。」月娘道:「倒是俺嫂子見他家新養的長姐,和咱孩子在床炕上睡着,都蓋着那被窝兒,你打我一下兒,我打你一下兒,恰似小兩口兒一般。纔叫了俺們去,説將起來。酒席上,就不因不由做了這門親。我方纔使小廝來對你説,擡送了花紅菓盒去。」西門慶道:「既做親也罷了,只是有些不搬陪些。喬家雖如今有這個家事,他只是個縣中大戶,白衣人。你我如今現居着這官,又在衙門中管着事。到明日會親,酒席間他戴着小帽,與俺這官戶怎生相處?甚不雅相!就前日荆南岡央及營裏張親家,再三趕着和我做親,説他家小姐今纔五個月兒,也和咱家孩子同歲。我嫌他没娘母子,也是房裏生的,所以没曾應承他。不想倒與他家做了親。」潘金蓮在旁接過來道:「嫌人家是房裏養的,誰家是房外養的?就是今日喬家這孩子,也是房裏生的。正是險道神撞見那壽星老兒,你也休説我的長,我也休嫌你那短!」這西門慶聽了此言,心中大怒,罵道:「賊淫婦,還不過去!人這裏説話,也插嘴插舌的,有你什麽説處!」金蓮把臉羞的通紅了,抽身走出來,説道:「誰説這裏有我説處?可知我没説處哩!」

看官聽説:今日潘金蓮在酒席上,見月娘與喬大戶家做了親,李瓶兒都披紅簪花遞酒,心中甚是氣不憤。來家又被西門慶駡了這兩句,越發急了,走到月娘這邊屋裏哭去了。西門慶因問:「大妗子怎的不來?」月娘道:「喬親家母明日見有他衆官娘子,説不得來。我留下他在那裏,教明日同他一搭兒裏來。」西門慶道:「我説只這席間坐次上,也不好相處的。到明日怎麽廝會?」

説了囬話,只見孟玉樓也走過這邊屋裏來,見金蓮哭泣,説道:「你只顧惱怎的?隨他説了幾句罷了。」金蓮道:「早是你在旁邊聽着,我説他什麽歹話來?又是一説,他説別家是房裏養的,我説喬家是房外養的?也是房裏生的。那個紙包兒包着,瞞得過人?賊不逢好死的強人,就睜着眼駡起我來。罵的人那絶情絶義!我怎來的,没我説處?改變了心腸,教他明日現報了我的眼!我不好説的,喬小妗子出來,還有喬老頭子的些氣兒。你家的失迷了家鄉,還不知是誰家的種兒哩!人便圖往來,扳親家耍子兒,教他人拏我惹氣罵我,管我ず事!多大的孩子,又是和一個懷抱的尿泡種子平白子扳親家,有錢没處施展的。爭破臥單沒的蓋,狗咬尿胞空喜歡!如今做濕親家還好,到明日休要做了乾親家纔好。吹殺燈擠眼兒——後來的事看不見的勾當!做親時大家好,過後三年五載,妨了的纔一個兒!」玉樓道:「如今人也賊了,不幹這個營生。論起來也還早哩,纔養的孩子,割什麽衫襟?無過只是圖往來,扳陪着耍子兒罷了!」金蓮道:「你們便浪𢵞着圖扳親家耍子,平白教賊不合理的強人罵我!我養蝦蟆得水蠱兒病——着什麽來由來?」玉樓道:「誰敎你説話不着個頭頂兒就説出來。他不駡你罵狗?」金蓮道:「我不好説的,他不是房裏,是大老婆?就是喬家孩子,是房裏生的,還有喬老頭子的些氣兒。你家失迷家鄉,還不知是誰家的種兒哩!」玉樓聽了,一聲兒沒言語。坐了一囬,金蓮歸房去了。

李瓶兒見西門慶出來了,従新花枝招颺,與月娘磕頭,説道:「今日孩子的事,累姐姐費心!」那月娘笑嘻嘻,也倒身還下禮去,説道:「你喜呀。」李瓶兒道:「與姐姐同喜!」磕畢頭起來,與月娘李嬌兒坐着説話。只見孫雪娥大姐來與月娘磕頭,與李嬌兒李瓶兒道了萬福。小玉拏將茶來,正喫茶,只見李瓶兒房裏丫鬟綉春來請,説:「哥兒屋裏尋哩!爹使我請娘來了。」李瓶兒道:「奶子慌的三不知就抱的屋裏去了,一搭兒去也罷了。只怕孩子沒個燈兒。」月娘道:「頭裏進門,我教他抱的房裏去,恐怕晚了。」小玉道:「頭裏如意兒抱着他,來安兒打着燈籠送他來。」李瓶兒道:「這等也罷了。」於是作辭月娘,囬房中來。只見西門慶在屋裏,官哥兒在奶子懷裏睡着了,因説道:「你如何不對我説,就抱了他來?」如意兒道:「大娘見來安兒打着燈籠,就趁着燈兒來了。哥哥哭了一回,纔拍着他睡着了。」西門慶道:「他尋了這一囬,纔睡了。」李瓶兒説畢,望着他笑嘻嘻説道:「今日與孩子定了親,累你。我替你磕個頭兒。」於是插燭也似磕下去。喜歡的西門慶滿面堆笑,連忙拉起來,做一䖏坐的。一面令迎春擺上酒兒,兩個這屋裏喫酒。

且説潘金蓮到房中,使性子,沒好氣。明知西門慶在李瓶兒這邊,一徑因秋菊開的門遲了,進門就打兩個耳刮子,高聲駡道:「賊淫婦奴才,怎的叫了恁一日不開,你做什麽來?今兒我且不和你答話!」於是走到屋裏坐下。春梅走來磕頭、遞茶。婦人問他:「賊奴才他在屋裏做什麽來?」春梅道:「在院子裏坐着來。他聽了,我那等催他還不理。」婦人道:「我知道,他和我兩個毆氣。黨太尉喫匾食——他也學人照樣兒行事,欺負我!」待要打他,又恐西門慶在那屋裏聽見;不言語,心中又氣。一面卸了濃粧,春梅與他搭了鋪,上床就睡了。

到次日,西門慶衙門中去了。婦人把秋菊教他頂着大塊柱石,跪在院子裏。跪的他梳了頭,教春梅扯了他褲子,拏大板子要打他。那春梅道:「好乾淨的奴才,教我扯褲子,倒没的污濁了我的手!」走到前邊,旋叫了畫童兒小廝,扯去秋菊底衣。婦人打着他,罵道:「賊奴才淫婦,你従幾時就恁大來?別人興你,我卻不興你!姐姐,你知我見的,將就膿着些兒罷了,平白撑着頭兒逞什麽強!姐姐,你休要倚着,我到明日洗着兩個眼兒,看着你哩!」一面罵着又打,打了又罵,打的秋菊殺豬也似叫。李瓶兒那邊纔起來,正看着奶子官哥兒,打發睡着了,又唬醒了。明明白白聽見金蓮這邊打丫鬟,罵的言語兒妨頭,則一聲兒不言語,唬的只把官哥兒耳朵摀着。一面使綉春:「去對你五娘説,休打秋菊罷。哥兒纔喫了些奶睡着了。」金蓮聽了,越發打的秋菊狠了。駡道:「賊奴才!你身上打着一萬把刀子,這等叫饒!我是恁性兒,你越叫,我越打!莫不為你拉断了路行人?人家打丫頭,也來看着?你好姐姐,對漢子説,把我別變了罷!」李瓶兒這邊分明聽見指罵的是他,把兩隻手氣的冰冷,忍氣吞聲,敢怒而不敢言。早晨茶水也沒喫,摟着官哥兒在炕上就睡着了。

等到西門慶衙門中回家,入房來看官哥兒,見李瓶兒哭的眼紅紅的睡在炕上,問道:「你怎的這咱還不梳頭收拾?上房請你説話。你怎揉的眼恁紅紅的?」李瓶兒也不提金蓮那邊指罵之事,只説:「我心中不自在。」西門慶告説:「喬親家那裏,送你的生日禮來了:一疋尺頭、兩壜南酒、一盤壽桃、一盤壽麵、四樣嗄飯;又是哥兒送節的兩盤元宵、四盤蜜食、四盤細菓、兩掛珠子吊燈、兩座羊皮屏風燈、兩疋大紅官緞、一頂青緞㩟的金八吉祥帽兒、兩雙男鞋、六雙女鞋。咱家倒還没往他那裏去,他又早與咱孩兒送節來了。如今上房的請你計較去。只他那裏使了個孔嫂兒和喬通押了禮來。大妗子先來了,説明日喬親家母不得來,直到後日纔來。他家有一門子做皇親的喬五太太,聽見和咱們做親,好不喜歡,到十五日也要來走走。咱少不得補個帖兒請去。」李瓶兒聽了,方慢慢起來梳頭。走到後邊,拜了大妗子。孔嫂兒正在月娘房裏待茶,禮物都擺明間内,都看了。一面打發囬盒起身,與了孔嫂兒喬通每人兩方手帕、五錢銀子,寫了回帖。又差人補請帖,送與喬太太去了。正是:但將鐘鼓悦私愛,好把犬羊為國羞。有詩為證:

\begin{myquote}
西門濁富太驕矜,襁褓孩童結做親。

不獨資財如糞土,也應嗟歎後來人。
\end{myquote}

畢竟未知後來如何,且聽下囬分解。

