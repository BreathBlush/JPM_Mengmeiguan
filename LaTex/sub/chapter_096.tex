\includepdf[pages={191,192},fitpaper=false]{tst.pdf}
\chapter*{第九十六囬 \\春梅遊玩舊家池館 守備使張勝尋經濟}
\addcontentsline{toc}{chapter}{第九十六囬 春梅遊玩舊家池館 守備使張勝尋經濟}
\markboth{{\titlename}卷之十}{第九十六囬 春梅遊玩舊家池館 守備使張勝尋經濟}


\begin{myquote}
裏虚外實費張羅,待客酬人使用多。

馬死奴逃難宴集,臺傾樓倒罷笙歌。

租田税店歸舊主,玩好金珠托賣婆。

欲向富家權借用,當人開口奈羞何。
\end{myquote}

話説光陰迅速,日月如梭。又早到正月二十一日。春梅和周守備說了,備一張祭桌,四樣羹果,一罈南酒,差家人周仁,送與吳月娘。一者是西門慶三週年,二者是孝哥兒生日。月娘收了禮物,打發來人帕一方,銀三錢。這邊連忙就使玳安兒穿青衣,具請書兒請去。上寫着:

\begin{myquote}[\markfont]
「重承厚禮,感感。即刻舍具菲酌,奉酬

腆儀。仰希

高軒俯臨。不外,幸甚!

\raggedleft{{\kaishu(下書)}西門吴氏端肅拜請}

\raggedright{大德周老夫人粧次。」}
\end{myquote}

春梅看了,到日中纔來。戴着滿頭珠翠,金鳳頭面釵梳,胡珠環子;身穿大紅通袖四獸朝麒麟袍兒,翠藍十樣錦百花裙,玉玎璫禁步,束着金帶;脚下大紅繡花白綾高底鞋兒。坐着四人大轎,青緞銷金轎衣。軍牢執藤棍喝道,家人伴當跟隨,檯着衣匣;後邊兩頂家人媳婦小轎兒,緊緊跟着大轎。吴月娘這邊請了吴大妗子相陪,又叫了兩個唱的女兒彈唱。聽見春梅來到,月娘亦盛粧縞素打扮,頭上五梁冠兒,戴着稀稀幾件金翠首飾,耳邊二珠環子,金㩟領兒,上穿白綾襖,下邊翠藍緞子織金拖泥裙,脚下穿玉色緞高底鞋兒,與大妗子迎接至前廳。春梅大轎子擡至儀門首纔落下轎來,兩邊家人圍着,到於廳上叙禮,向月娘插燭也似拜下去。月娘連忙答禮相見,沒口説道:「向日有累姐姐費心,粗尺頭又不肯受。今又重承厚禮祭桌,感激不盡!」春梅道:「惶恐,家官府沒甚麽,這些薄禮,表意而已。一向要請姥姥過去,家官府不一時出巡,所以不曾請得。」月娘道:「姐姐,你是幾時好日子?我只到那日,買禮看姐姐去罷。」春梅道:「奴賤日是四月廿五日。」月娘道:「奴到那日一定去!」兩個叙畢禮。春梅務要把月娘讓起,受了兩禮。然後吴大妗子相見,亦還下禮去。春梅道:「你看大妗子,又沒正經!」一手扶起受禮。大妗子道:「姐姐,你今非昔比,折殺老身。」止受了半禮。一面讓上坐,月娘和大妗子主位相陪。然後家人媳婦丫鬟養娘都來參見。春梅見了奶子如意兒抱着孝哥兒,吴月娘道:「小大哥,還不來與姐姐磕個頭兒,謝謝姐姐,今日來與你做個生日!」那孝哥兒眞個爬下如意兒身來,與春梅唱喏。月娘道:「好小廝,不與姐姐磕頭,只唱喏?」那春梅連忙向袖中,掏出一方錦手帕,一付金八吉祥兒,教替他㩟帽兒上戴。月娘道:「又教姐姐費心!」又拜謝了。落後小玉奶子來見,磕頭。春梅與了小玉一對金頭簪子,與了奶子兩枝銀花兒。月娘道:「姐姐,你還不知,奶子與了來興兒做了媳婦兒了。來興兒那媳婦,害病沒了。」春梅道:「他一心要在咱家,倒也好。」一面丫鬟拿茶上來。喫了茶,月娘說:「請姐姐後邊明間内坐罷,這客位内冷。」

春梅來後邊,西門慶靈前又早點起燈燭,擺下桌面祭禮。春梅燒了紙,落了幾點眼淚。然後周圍設放圍屏,火爐内生起炭火,安放大八僊桌席,擺茶上來。無非是細巧蒸酥,異樣甜食,羙口菜蔬,希奇菓品,縷金碟,象牙筯,雪錠盤盞兒,絶品芽茶。月娘和大妗子陪着喫了茶,讓春梅進上房裏換衣裳。脫了上面袍兒,家人媳婦開衣匣取出衣服,更換了一套綠遍地錦粧花襖兒,紫丁香色遍地金裙。在月娘房中坐着,說了一囘。月娘因問道:「哥兒好麽,今日怎不帶他來這裏走走?」春梅道:「若不是,也帶他來與姥姥磕頭,他爺説天氣寒冷,怕風冒着他。他又不肯在房裏,只要那當直的抱出來廳上外邊走。這兩日不知怎的,只是哭。」月娘道:「你出來他也不尋你?」春梅道:「左右有兩個奶子,輪番看他也罷了。」月娘道:「他周爺也好大年紀,得你替他養下這點孩子,也夠了。也是你裙帶上的福。說他孫二娘還有位姐兒,幾歲兒了?」春梅道:「他二娘養的叫玉姐,今年交生四歲。俺這個叫金哥。」月娘道:「說他周爺身邊,還有兩位房裏姐兒?」春梅道:「是兩個學彈唱的丫頭子,都有十六七歲,成日淘氣在那裏。」月娘道:「他爺也常往他身邊去不去?」春梅道:「奶奶,他那裏得工夫在家?多在外,少在裏。如今四外,好不盜賊生發。朝廷勅書上,又教他兼管許多事情,鎮守地方,巡理河道,捉拿盜賊,操練人馬。常不時往外出巡幾遭,好不辛苦哩!」說畢,小玉拿茶來喫了。春梅向月娘說:「姥姥,你引我往俺娘那邊花園山子下走走。」月娘道:「我的姐姐,山子花園還是那咱的山子花園哩?自従你爹下世,沒人收拾他,如今丢搭的破零二落,石頭也倒了,樹木也死了,俺等閑也不去了。」春梅道:「不妨,奴就往俺娘那邊看看去。」這月娘強不過,只得教小玉拿花園門山子門鑰匙開了門,月娘大妗子陪春梅,衆人到裏面遊看了半日。但見:

\begin{myquote}
垣牆欹損,臺榭歪斜。兩邊畫壁長青苔,滿地花磚生碧草。山前怪石,遭塌毀不顯嵯峨;亭内涼牀,被滲漏已無框檔。石洞口蛛絲結網,魚池内蝦蟆成羣。狐狸常睡臥雲亭,黄鼠往來藏春閣。料想經年人不到,也知盡日有雲來。
\end{myquote}

春梅看了一囬,先走到李瓶兒那邊。見樓上丢着些折桌壞櫈破椅子,下邊房都空鎖着。地下草長的荒荒的。方來到他娘這邊,樓上還堆着生薬香料,下邊他娘房裏,止有兩座厨櫃,牀也沒了。因問小玉:「俺娘那張牀往那去了,怎的不見?」小玉道:「俺三娘嫁人,賠了俺三娘去了。」月娘走到跟前說:「因有你爹在日,將他帶來那張八步牀,陪了大姐在陳家。落後他起身,却把你娘這張牀賠了他嫁人去了。」春梅道:「我聽見大姐死了時,你老人家就把牀還擡的來家了。」月娘道:「那牀沒錢使,只賣了八兩銀子,打發縣中皂隸,都使了。」春梅聽言,點了點頭兒,那星眼中由不的酸酸的,口内不言,心下暗道:「想着俺娘,那咱爭強不伏弱的,問爹要買了這張牀。我實承望要囘了這張牀去,也做他老人家一念兒,不想又與了人去了。」由不的心下慘切。又問月娘:「俺六娘那張螺甸牀,怎的不見?」月娘道:「一言難盡。自従你爹下世,日逐只有出去的,沒有進來的。常言家無營活計,不怕斗量金。也是家中沒盤纏,擡出去交人賣了。」春梅問:「賣了多少銀子?」月娘道:「止賣了三十五兩銀子。」春梅道:「可惜了的!那張牀,當初我聽見爹說,值六十兩多銀子,只賣這些兒!早知你老人家打發,我倒與你老人家三四十兩銀子,我要了也罷。」月娘道:「好姐姐,諸般都有,——人沒有早知道的!」一面嘆息了半日。只見家人周仁走來接,說:「爹請奶奶早些家去,哥兒尋奶奶哭哩。」這春梅就抽身往後邊。月娘教小玉鎖了花園門,同來到後邊明間内,又早屏開孔雀,簾控鮫綃,擺下酒筳。兩個妓女,銀箏琵琶,在旁彈唱。吴月娘遞酒安席,不必細說。安春梅上坐,春梅不肯,務必拉大妗子同他一處坐的。月娘主位,筵前遞了酒,湯飯點心,割切上席。春梅教家人周仁,賞了厨子三錢銀子。說不盡盤堆異品酒泛金波。

當下傳盃換盞,喫至日色將落時分,只見宅内又差伴當拿燈籠來接。月娘那裏肯放,教兩個妓女,在跟前跪着彈唱勸酒,吩咐:「你把好曲兒,孝順你周奶奶一個兒。」一面叫小玉斟上大鍾,放在跟前,教春梅喫:「姐姐,你吩咐個心下愛的曲兒,敎他兩個唱與你聽下酒。」春梅道:「姥姥,奴喫不得的,怕孩兒家中尋找。」月娘道:「哥兒尋,左右有奶子看着。天色也還早哩,我曉得你好小量兒!」春梅因問那兩個妓女:「你叫甚名字?是誰家的?」兩個跪下說:「小的一個是韓金釧兒妹子韓玉釧兒,一個是鄭愛香兒姪女鄭嬌兒。」春梅道:「你們會唱〔懶畫眉〕不會?」玉釧兒道:「奶奶吩咐,小的兩個都會。」月娘道:「你兩個旣會唱,斟上酒你周奶奶喫,你們慢唱。」小玉在旁,連忙斟上酒。兩個妓女,一個彈箏,一個琵琶,唱道:

\begin{myquote}
「寃家為你幾時休?捱過春來又到秋,誰人知道我心頭。天,害的我伶仃瘦!聽的音書兩淚流。従前已往訴緣由,誰想你無情把我丢!」
\end{myquote}

那春梅喫過。月娘又令鄭嬌兒遞上一盃酒與春梅。春梅道:「你老人家也陪我一盃。」兩家於是都齊斟上,兩個妓女又唱道:

\begin{myquote}
「寃家為你减風流!鵲噪簷前不肯休,死聲活氣沒來由。天,倒惹的情迤逗,助的凄凉兩淚流。従他去後意無休,誰想你辜恩把我丢?」
\end{myquote}

春梅道:「姥姥,你也教大妗子喫盃兒。」月娘道:「大妗子喫不的,敎他拿小鍾兒陪你罷。」一面令小玉斟上大妗子一小鍾兒酒,兩個妓女又唱道:

\begin{myquote}
「寃家為你惹場憂!坐想行思日夜愁,香ざ憔瘦减溫柔。天,要見你不能够,悶的我傷心兩淚流!従前與你共綢繆,誰想你今番把我丢!」
\end{myquote}

當下春梅見小玉在跟前,也斟了一大鍾,教小玉喫。月娘道:「姐姐,他喫不的。」春梅道:「姥姥,他也喫兩三鍾兒。我那咱在家裏,沒和他喫?」於是斟上,教小玉也喫了一盃。妓女唱道:

\begin{myquote}
「寃家為你惹閑愁!病枕着牀無了休,滿懷憂悶鎖眉頭。天,忘了還依舊,助的我腮邊兩淚流。從前與你兩無休,誰想你經年把我丢!」
\end{myquote}

看官聽說:當時春梅為甚教妓女唱此詞?一向心中牽掛陳經濟在外,不得相會。情種心苗,故有所感,發於吟咏。又見他兩個唱的好,口兒甜,乖覺,奶奶長奶奶短侍奉,心中歡喜,呌家人周仁近前來拿出兩包兒賞賜來,每人二錢銀子。兩個妓女放下楽器,插燭也似磕頭,謝了賞賜。不一時,春梅起身,月娘款留不住,伴當打燈籠,拜辭出門,坐上大轎,家人媳婦都坐上小轎,前後打着四個燈籠,軍牢喝道而去。正是:時來頑鐵有生輝,運去黄金無艷色。有詩為證:

\begin{myquote}
點絳唇紅弄玉嬌,鳳凰飛下品鸞簫。

堂前高把湘簾捲,燕子還來續舊巢。
\end{myquote}

且說春梅自從來吴月娘家赴席之後,因思想陳經濟不知流落在何處,歸到府中,終日只是臥牀不起,心下沒好氣。守備察知其意,說道:「只怕思念你兄弟,不得其所。」一面叫將張勝李安來,吩咐道:「我一向委你尋你奶奶兄弟,如何不用心找尋?」二人告道:「小的一向找尋來,一地裏尋不着下落,已回了奶奶話了。」守備道:「限你二人五日,若找尋不着,討分曉!」這張勝李安領了鈞語下來,都帶了愁顔,沿街遶巷,各處留心找問不題。

話分兩頭,單表陳經濟自従守備府中打了出來,欲投晏公廟,聽見人說:「你師父任道士,因為你宿娼壞事,被人打了,拿在守備府去,查點房中箱籠,東西銀兩沒了,一口重氣,半夜就死了。你還敢進廟中去?衆徒弟就打死你!」這經濟害怕,就不敢進廟來。又沒臉見杏庵王老,白日裏到䖏打油飛,夜晚間還鑽入冷舖中存身。一日,也是合當有事,經濟正在街上站立,只見鐵指甲楊大郎頭戴新羅帽兒,身穿白綾襖子,玄色緞氅衣,沉香色襪口,光素琴鞋,騎着一疋驢兒,揀銀鞍轡,一個小廝跟隨,正打街心走過來。經濟認的是楊光彦,便向前一把手把嚼環拉住,說道:「楊大哥,一向不見!咱兩個同做朋友,往下江販布,船在清江浦泊着,我在嚴州府探親,喫人陷害,打了一場官司,你就不等我,把我半船貨物偸拐,走的不知去向。我好意往你家問,反喫你兄弟楊二風拿瓦楔礸破頭,赶着打上我家門來。今日弄的我一貧如洗。你是會搖擺受用!」那楊大郎見了經濟討喫,佯佯而笑,說:「如今晦氣,出門撞見瘟死鬼!量你這餓不死賊花子,那裏討半船貨,我拐了你的來了?你不撒手,須喫我一頓好馬鞭子!」那經濟便道:「我如今窮了,你有銀子與我些盤纏,不然咱到個去處!」楊大郎見他不放,跳下驢來,向他身上抽了幾鞭子,喝令小廝:「與我撏了這少死的花子去!」那小廝使力把經濟推了一跤。楊大郎又向前踢了幾脚,踢打的經濟怪叫。

須臾,圍了許多人。旁邊閃過一個人來,青高裝帽子,勒着手帕,倒披紫襖,白布て子,精着兩條脚,靸着蒲鞋;生的阿兜眼,掃帚眉,料綽口,三鬚鬍子,面上紫肉横生,手腕横觔競起;喫的楞楞睜睜,提着拳頭,向楊大郎說道:「你此位哥好不近理!他年少,這般貧寒,你只顧打他怎的?自古嗔拳不打笑面,他又不曾傷犯着你。你有錢,看平日相交,與他些;沒錢罷了,如何只顧打他?自古路見不平,也有向燈向火!」楊大郎說:「你不知,他賴我拐了他半船貨。量他恁窮嘴臉,有半船貨物?」那人道:「想必他當時也是根基人家娃娃,天生就這般窮來?閣下就到這般有錢?老兄,依我,你有銀子與他些盤纏罷!」那楊大郎見那人說了,袖内汗巾兒上拴着四五錢一塊銀子,解下來遞與經濟,與那人擧一擧手兒,上驢子揚長去了。

經濟地下爬起來,擡頭看那人時,不是別人,却是舊時同在冷舖内,和他一舖睡的土作頭兒飛天鬼侯林兒。近來領着五十多人,在城南水月寺曉月長老那裏做工,起蓋伽藍殿。因一隻手拉着經濟說道:「兄弟,剛纔若不是我拿幾句言語譏犯他,他肯拿出這五錢銀子與你?他賊,却知見範;他若不知範時,好不好喫我一頓好拳頭!你跟着我,咱往酒店内喫酒去。」來到一個食葷小酒店内,案頭上坐下,叫量酒拿四賣嗄飯、兩大壺酒來。不一時,量酒打抹條桌乾淨,擺下小菜嗄飯,四盤四碟,兩大坐壺時興橄欖酒,不用小盃,拿大磁甌子。因問經濟:「兄弟你喫麵喫飯?」量酒道:「麵是溫淘,飯是白米飯。」經濟道:「我喫麵。」須臾,掉上兩三碗濕麵上來,侯林兒只喫一碗,經濟喫了兩碗,然後喫酒。侯林兒向經濟說:「兄弟,你今日跟我往坊子裏睡一夜。明日我領你城南水月寺曉月長老那裏,修蓋伽藍殿並兩廊僧房。你哥率領着五十多人做工。你到那裏,不要你做重活,只擡幾筐土兒就是了。也算你一工,討四分銀子。我外邊賃着一間厦子,晚夕咱兩個就在那裏歇。做些飯打發咱的人喫,門你一把鎖鎖了,家都交與你,好不好?強如你在那冷舖中替花子搖鈴打梆子,這個還官樣些。」經濟道:「若是哥哥這般下顧兄弟,可知好哩!不知這工程做的長遠不長遠?」侯林兒道:「纔做了一個月。這工程做到十月裏,不知完不完。」兩個說話之間,你一鍾,我一盞,把兩大壺酒都喫了。量酒算帳,該一錢三分半銀子。經濟要會銀子,拿出銀子來秤。侯林兒推過一邊說:「儍兄弟,莫不教你出錢?哥有銀子在此!」一面扯出包兒來,秤了一錢五分銀子與掌櫃的,還找了一分半錢袖了。搭伏着經濟肩背,同到坊子裏,兩個在一處歇臥。二人都醉了。這侯林兒晚夕幹經濟後庭花,足幹了一夜,親哥親達達,親漢子親爺,口裏無般不叫將出來。

到天明同往城南水月寺。果然寺外侯林兒賃下半間廈子,裏面燒着炕柴竃,也買下許多碗盞家活。早晨上工,叫了名字。衆人看見經濟不上二十四五歲,白臉子,生的眉目清俊,就知是侯林兒兄弟,都亂調戯他。先問道:「那小夥子兒,你叫甚名字?」陳經濟道:「我叫陳經濟。」那人道:「陳經濟,可不由着你就擠了!」又一人說:「你恁年小小的,原幹的這營生,挨的這大扛頭子?」侯林兒喝開衆人,罵:「怪花子,你只顧奚落他怎的?」一面散了鍬鐝筐杠,派衆人擡土的擡土,和泥的和泥,打榪的打榪。原來曉月長老教一個葉頭陀做火頭,造飯與落作匠人喫。這葉頭陀年約五十歲,一個眼瞎。穿着皂直裰,精着脚,腰間束着爛絨縧,也不會看經,只會念佛。善會麻衣神相,衆人都叫他做葉道。一日,做了工下來,衆人都喫畢飯,閑坐的,站的,也有蹲着的。只見經濟走向前問葉頭陀討茶喫,這葉頭陀只顧上上下下看他。内有一人說:「葉道,這個小夥子兒是新來的。你相他一相。」又一人説:「你相他相,倒像個兄弟。」一人説:「倒像個二尾子。」葉頭陀教他近前,端詳了一囬,說道:「色怕嫩兮又怕嬌,聲嬌氣嫩不相饒。老年色嫩招辛苦,少年色嫩不堅牢。只喫了你面嫩的虧。一生多得陰人寵愛。八歲十八二十八,下至山根上至髮,有無活計兩頭消,三十印堂莫帶煞。眼光帶秀心中巧,不讀詩書也可人;做作百般人可愛,縱然弄假不成真。休怪我說,一生心伶機巧,常得陰人發跡。你今年多大年紀?」經濟道:「我二十四歲。」葉道道:「虧你前年怎麽打過來!喫了你印堂太窄,子丧妻亡;懸壁昏暗,人亡家破;唇不蓋齒,一生惹是招非;鼻若竈門,家私傾喪。那一年遭官司口舌,傾家丧業,見過不成?」經濟道:「都見過了。」葉頭陀道:「又一件,你這山根不宜斷絶。麻衣祖師說得兩句好:山根斷兮早虚化,祖業飄零定破家。早年父祖丢下家産,不拘多少,到你手裏都了當了。你上停短兮下停長,主多成多敗,錢財使盡又還來。總然你久後營得成家計,猶如烈日照冰霜。你走兩步我瞧。」那經濟真個走了兩步,葉頭陀道:「頭先過步,初主好而晚景貧窮;脚不點地,賣盡田園而走他鄉。一生不守祖業。你往後好,有三妻之命。尅過一個妻官不曾?」經濟道:「已尅過了。」葉頭陀道:「後來還有三妻之會。你面若桃花光焰,雖然子遲,但圖酒色懽娱。但恐羙中不羙,三十上小人有些不足,花柳中少要行走,還計較些。」一個人說:「葉道,你相差了!他還與人家做老婆,他那有三個妻來?」衆人正笑做一團,只聽得曉月長老打梆子,各人都拿鍬鐝筐杠,上工做活去了。如此者,經濟在水月寺也做了約一月光景。

一日,三月中旬天氣,經濟正與衆人擡出土來,在寺山門牆下,倚着牆根向日陽,蹲踞着捉身上虱蟣。只見一個人,頭戴萬字頭巾,腦後撲匾金環,身穿青窄衫,紫裹肚,腰繫纏帶,脚穿䩺靴,騎着一疋黄馬,手中提着一籃鮮花兒,見了經濟,猛然跳下馬來,向前深深的唱個喏,便叫:「陳舅,小人那裏沒處尋,你老人家原來在這裏!」倒唬了經濟一跳,連忙還禮不迭,問:「哥哥,你是那裏來的?」那人道:「小人是守備周爺府中親隨張勝,自従舅舅於府中官事出來,奶奶不好直到如今。老爺使小人那裏不曾找尋舅舅,不知在這裏!今早不是俺奶奶使小人往外庄上折取這幾朶芍薬花兒,打這裏所過,怎得看見你老人家在這裏?一來也是你老人家際遇,二者小人有緣。不消猶豫,就騎上馬,跟你老人家往府中去!」那衆做工的人看着,都面面相覷,不敢做聲。這陳經濟把鑰匙遞與侯林兒,騎上馬,張勝緊緊跟隨,逕往守備府中來。正是:良人得意正年少,今夜月明何處樓?有詩為證:

\begin{myquote}
白玉隱於頑石裏,黄金埋在汚泥中。

今朝貴人提拔起,如立天梯上九重。
\end{myquote}

畢竟未知後來如何,且聽下囬分解。

