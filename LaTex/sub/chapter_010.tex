\includepdf[pages={19,20},fitpaper=false]{tst.pdf}
\chapter*{第十囬 武二充配孟州道 妻妾宴賞芙蓉亭}
\addcontentsline{toc}{chapter}{第十囬 武二充配孟州道 妻妾宴賞芙蓉亭}
\markboth{第十囬 武二充配孟州道 妻妾宴賞芙蓉亭}{第十囬 武二充配孟州道 妻妾宴賞芙蓉亭}

朝看瑜珈經,暮誦消災呪。

種瓜須得瓜,種荳須得荳。

經咒本無心,寃結如何究?

地獄與天堂,作者還自受。

話說武二被地方保甲拿去縣裏見知縣去了。且表西門慶跳下樓窗,順着房山,趴伏在人家院裏藏了。原來是行醫的胡老人家。只見他家使的一個大胖丫頭走來毛厠裏凈手,蹶着大屁股,猛可見了一個漢子趴伏在院牆下,往前走不迭,大叫:「有賊了!」慌的胡老人急進來,看見認的是西門慶,便道:「大官人,且喜武二尋你不着,把那人打死了;地方拿去縣中見官去了,多是定死罪。大官人歸家去無事!」這西門慶拜謝了胡老人,搖擺着來家,一五一十對潘金蓮說。二人拍手喜笑,以為除了患害。婦人叫西門慶上下多使些錢:「務要結果了他,休要放他出來。」西門慶一面差心腹家人來旺兒,餽送了知縣一副金銀酒器、五十兩雪花銀;上下吏典也使了許多錢,只要休輕勘了武二。

知縣受了西門慶賄賂,到次日早衙陞廳,地方保甲押着武二,並酒保唱的一干證人,在廳前跪下。縣主一夜把臉翻了,便叫武二:「你這廝昨日虚告平人,我已再三寬你。如何不遵法度?今又平白打死了人,有何理說?」武二磕頭告道:「望相公與小人做主。小人本與西門慶執仇廝打,不料撞遇了此人在酒樓上,問道:『西門慶那裏去了?』他不說。小人一時怒起,悮打死了他。」知縣道:「這廝胡說!你豈不認得他是縣中皂隸?想必别有緣故,你不實說。」喝令左右:「與我加起刑來!人是苦蟲,不打不成。」兩邊閃出三四個皂隸役卒,抱許多刑具,把武松拖翻,雨點般篦板子打將下來。須臾,打了二十板,打得武二口口聲聲叫寃,說道:「小人平日也有與相公用力效勞之處,相公豈不憫念?相公休要苦刑小人。」知縣聽了此言,越發惱了:「你這廝親手打死了人,尚還口強抵賴那個!」喝令:「與我好生拶起來!」當下拶了武松一拶,敲了五十杖子。教取面長枷帶了,收在監内,一干人寄監在門房裏。内中縣丞佐貳官,也有和武二好的,念他是個義烈漢子,有心要周旋他,爭奈都受了西門慶賄賂,粘住了口,做不的張主。又見武松只是聲寃,延挨了幾日,只得朦朧取了供招。喚當該吏典並仵作、甲鄰人等,押到獅子街,檢驗李外傳身屍,填冩屍單格目:委的被武松尋問他索討,分錢不均,酒醉怒起,一時鬦毆,拳打脚踢,撞跌身死。左肋、面門、心坎、腎囊,俱有青赤傷痕不等。檢驗明白,囬到縣中。一日做了文書申詳,解送東平府來,詳允發落。

這東平府府尹姓陳,雙名文昭,乃河南人氏,極是個清廉的官。聽的報來,隨即陞廳。那官人,但見:

平生正直,禀性賢明。幼年向雪案攻書,長大在金鑾對策。常懷忠孝之心,每存仁慈之念。户口增,錢糧辦,黎民稱頌滿街衢;詞訟减,盜賊休,父老讚歌喧市井。攀轅截鐙,名標青史播千年;勒石鐫碑,聲振黄堂傳萬古。正直清廉民父母,賢良方正號青天。

這府尹陳文昭已知這事了。便教押過這一干人犯,就當廳先把清河縣申文看了,又把各人供狀招擬看過。端的上面怎生寫着?文曰:

「東平府清河縣為人命事,呈稱:犯人武松,年二十八歲,係陽谷縣人氏。因有膂力,本縣參做都頭。因公差囬還,祭奠亡兄,見嫂潘氏,守孝不滿,擅自嫁人。是日松在巷口打聽,不合在獅子街王鸞酒樓上,撞遇先不知名今知名李外傳,因酒醉索討前借錢三百文,外傳不與;又不合因而鬦毆,互相不服,揪打踢撞,傷重當時身死。比有娼婦牛氏包氏見證。致被地方保甲捉獲,委官前至屍所,拘集仵作、甲鄰人等,檢驗明白,取供、具結、填圖,解繳前來,覆審無異詞。擬武松合依鬦毆殺人,不問手足他物金刄,律絞。酒保王鸞,並牛氏包氏,俱供明無罪。今合行申到案發落,請允施行。

政和三年八月 日 

知縣李達天 縣丞樂和安 主簿華荷祿 典史夏恭基 司吏錢勞」

府尹看了一遍,將武松叫過面前跪下,問道:「你如何打死這李外傳?」那武松只是朝上磕頭,告道:「青天老爺,小的到案下,得見天日!容小的說,小的敢說。」府尹道:「你只顧說來。」武松道:「小的本為哥哥報仇,因尋西門慶廝打,悮打死此人。」把前情訴告了一遍,「委是小的負屈啣寃。西門慶錢大,禁他不得!小人死不足惜,但只是小人哥哥武大含寃地下,枉了性命!」府尹道:「你不消多言,我已盡知了。」因把司吏錢勞叫來,痛責二十板,說道:「你那知縣也不待做官,何故這等任情賣法?」於是將一干人衆,一一審錄過,用筆將武松供招都改了。因向佐貳官說道:「此人為兄報仇,悮打死這李外傳,也是個有義的烈漢,比故殺平人不同。」一面打開他長枷,換了一面輕罪枷枷了,下在牢裏。一干人等,都發囬本縣聽候。一面行文書,着落清河縣添提豪惡西門慶,並嫂潘氏、王婆、小廝鄆哥、仵作何九,一同従公根勘明白,奏請施行。武松在東平府監中,人都知道他是屈官司,因此押牢禁子都不要他一文錢,倒把酒食與他喫。

早有人把這件事報到清河縣,西門慶知道了,慌了手脚。陳文昭是個清廉官,不敢來打點他;只得走去央浼親家陳宅心腹,並使家人來保星夜來往東京,下書與楊提督。提督轉央内閣蔡太師,太師又恐怕傷了李知縣名節,連忙賚了一封緊要密書帖兒,特來東平府下書與陳文昭,免提西門慶潘氏。這陳文昭原係大理寺寺正,陞東平府府尹,又係蔡太師門生,又見楊提督乃是朝廷面前說得話的官,以此人情兩盡了,只把武松免死,問了個脊杖四十刺配二千里充軍。況武大已死,屍傷無存,事涉疑似,勿論。其餘一干人犯,釋放寜家。申詳過省院,文書到日,即便施行。陳文昭従牢中取出武松來,當堂讀了朝廷明降,開了長枷,免不得脊杖四十,取一具七斤半鐵葉團頭枷釘了,臉上刺了兩行金字,迭配孟州牢城。其餘發落已完,當堂府尹押行公文,差兩個防送公人,領了武松解赴孟州交割。

當日武松與兩個公人,出離東平府,來到本縣家中,將家活都變賣了,打發那兩個公人路上盤費。安撫左鄰姚二郎看管迎兒:「倘遇朝廷恩典,赦放還家,恩有重報,不敢有忘。」那街坊鄰舍、上户人家,見武二是個有義的漢子,不幸遭此刑,平昔與武二好的,都資助他銀兩,也有送酒食錢米的。武二到下處,問土兵要出行李包裹來,即日離了清河縣上路,迤ぜ往孟州大道而行,正遇着中秋天氣。此這一去,正是:若得苟全癡性命,也甘饑餓過平生。有詩為證:

府尹推詳禀至公,武松垂死又疏通。

今朝刺配牢城去,病草萋萋遇暖風。

這裏武二往孟州充配去了,不題。且說西門慶打聽他上路去了,一塊石頭方落地,心中如去了痞一般,十分自在。於是家中吩咐家人來旺來保來興兒,收拾打掃後花園芙蓉亭乾凈,鋪設圍屏,懸起錦障,安排酒席齊整,叫了一起樂人吹彈歌舞。請大娘子吴月娘、第二李嬌兒、第三孟玉樓、第四孫雪娥、第五潘金蓮,合家歡喜飲酒。家人媳婦、丫鬟使女,兩邊侍奉。怎見當日好筵席?但見:

香焚寳鼎,花插金瓶。器列象州之古玩,簾開合浦之明珠。水晶盤内,高堆火棗交梨;碧玉盃中,滿泛瓊漿玉液。烹龍肝,炮鳳腑,果然下筯了萬錢;黑熊掌,紫駝蹄,酒後獻來香滿座。更有那軟炊紅蓮香稻,細膾通印子魚。伊魴洛鯉,誠然貴似牛羊;龍眼荔枝,信是東南佳味。碾破鳳團,白玉甌中分碧浪;斟來瓊液,紫金壺内噴清香。畢竟壓賽孟嘗君,只此敢欺石崇富。

當下西門慶與吴月娘居上,其餘李嬌兒、孟玉樓、孫雪娥、潘金蓮,都兩傍列坐,傳盃弄盞,花簇錦攢飲酒。只見小廝玳安領下一個小廝、一個小女兒,纔頭髮齊眉兒,生得乖覺,拿着兩個盒兒,說道:「隔壁花太監家的,送花兒來與娘們戴。」走到西門慶月娘衆人跟前,都磕了頭,立在傍邊,說:「俺娘使我送這盒兒點心,並花兒與西門大娘戴。」揭開簾子看盒兒,一盒是朝廷上用的菓餡椒鹽金餅,一盒是新摘下來鮮玉簪花兒。月娘滿心歡喜,說道:「又叫你娘費心!」一面看菜兒,打發兩個喫了點心。月娘與了那小丫頭一方汗巾兒,與了小廝一百文錢,說道:「多上覆你娘,多謝了。」因問小丫頭兒:「你叫什麽名字?」他囬言道:「我叫綉春。小廝叫做天福兒。」打發去了,月娘便向西門慶道:「咱這裏間壁住的花家,這娘子兒倒且是好,常時使過小廝丫頭送東西與我,我並不曾囬些禮兒與他。」西門慶道:「花二哥他娶了這娘子兒,今不上二年光景。他自說娘子好個性兒。不然,房裏怎生得這兩個好丫頭?」月娘道:「前者六月間,他家老公公死了,出殯時,我在山頭會他一面。生得五短身材,團面皮,細彎彎兩道眉兒,且是白淨,好個溫克性兒!年紀還小哩,不上二十四五。」西門慶道:「你不知,他原是大名府梁中書妾,晚嫁花家子虚,帶了一分好錢來。」月娘道:「他送盒來親近你我,又是個緊鄰,咱休差了禮數,到明日也送些禮物囘答他。」

看官聽說:原來花子虚渾家,娘家姓李,因正月十五日所生,那日人家送了一對魚瓶兒來,就小字喚做瓶姐。先與大名府梁中書家為妾。梁中書乃東京蔡太師女婿。夫人性甚嫉妒,婢妾打死者,都埋在後花園中。這李氏只在外邊書房内住,有養娘扶侍。只因政和三年正月上元之夜,梁中書同夫人在翠雲樓上,李逵殺了全家老小,梁中書與夫人各自逃生。這李氏帶了一百顆西洋大珠,二兩重一對鴉青寳石,與養娘媽媽走上東京投親。那時花太監由御前班直陞廣南鎮守,因侄男花子虚没妻室,就使媒人說親,娶為正室。太監到廣南去,也帶他到廣南。住了半年有餘。不幸花太監有病,告老在家,因是清河縣人,在本縣住了。如今花太監死了,一分錢都在子虚手裏,每日同朋友在院中行走,與西門慶都是會中朋友。西門慶是個大哥;第二個姓應雙名伯爵,原是開紬絹舖的應員外兒子,没了本錢,跌落下來,專在本司三院,幫嫖貼食,會一脚好氣毬,雙陸棋子,件件皆通;第三個姓謝,名希大,字子純,亦是幫閑勤兒,會一手好琵琶,每日無營運,專在院中喫些風流茶飯;還有個祝日念、孫寡嘴、吴典恩、雲裏手、常時節、卜志道、白來搶,共十個朋友。卜志道故了,花子虚補了。每月會在一處,叫兩個唱的,花攢錦簇頑耍。衆人見花子虚乃是内臣家勤兒,手裏使錢撒漫,都亂撮合他在院中請婊子,整三五夜不歸家。正是:

紫陌春光好,紅樓醉管絃。

人生能有幾,不樂是徒然!

此事表過不題。且說當日西門慶率同妻妾,合家歡喜,在芙蓉亭上飲酒,至晚方散。歸到潘金蓮房中,已有半酣。乘着酒興,要和婦人雲雨。婦人連忙薰香打鋪,和他解衣上牀。西門慶且不與他雲雨,明知婦人第一好品簫,於是坐在青紗帳内,令婦人馬爬在身邊,雙手輕籠金釧,捧定那話,往口裏吞放。西門慶垂首玩其出入之妙,嗚咂良久,淫情倍增,因呼春梅進來遞茶。婦人恐怕丫頭看見,連忙放下帳子來。西門慶道:「怕怎麽的?」因說起:「隔壁花二哥房裏,倒有兩個好丫頭,今日送花來的是小丫頭;還有一個,也有春梅年紀,也是花二哥收用過了。但見他娘在門首站立,他跟出來,且是生得好模樣兒。誰知這花二哥年紀小小的,房裏恁般用人!」婦人聽了,瞅了他一眼,說道:「怪行貨,我不好罵你!你心裏要收這個丫頭,收他便了,如何遠打週折,指山說磨,拿人家來比奴?一則奴不是那樣人,他又不是我的丫頭。既然如此,明日我往後邊坐一囬,騰個空兒,你自在房中叫他來,收他便了。」說畢,當下與西門慶品簫過了,方纔抱頭交股而寢。正是自有内事迎郎意,殷勤快把紫簫吹。有〈西江月〉為證:

紗帳輕飄蘭麝,娥眉慣把簫吹。雪白玉體透房幃,禁不住魂飛魄蕩。玉腕款籠金釧,兩情如醉如癡。才郎情動囑奴知:慢慢多咂一會。

到次日,果然婦人往後邊孟玉樓房中坐了。西門慶叫春梅到房中,春點杏桃紅綻蕊,風欺楊柳綠翻腰,收用了這妮子。婦人自此一力擡擧他起來,不令他上鍋抹灶,只叫他在房中舖牀疊被、遞茶水。衣服首飾,揀心愛的與他,纏的兩隻脚小小的。原來春梅比秋菊不同,性聰慧,喜謔浪,善應對,生的有幾分顔色,西門慶甚是寵他。秋菊為人濁蠢,不任事體,婦人打的是他。正是:

燕雀池塘語話喧,皆因仁義說愚賢。

雖然異數同飛鳥,貴賤高低不一般。

畢竟未知後來何如,且聽下囬分解。

\part*{夢梅館本《金瓶梅詞話》卷之二}
\addcontentsline{toc}{part}{夢梅館本《金瓶梅詞話》卷之二}

