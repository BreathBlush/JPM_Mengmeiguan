\includepdf[pages={197,198},fitpaper=false]{tst.pdf}
\chapter*{第九十九囬 \\劉二醉罵王六兒 張勝忿殺陳經濟}
\addcontentsline{toc}{chapter}{第九十九囬 劉二醉罵王六兒 張勝忿殺陳經濟}
\markboth{{\titlename}卷之十}{第九十九囬 劉二醉罵王六兒 張勝忿殺陳經濟}


格言

\begin{myquote}
一切諸煩惱,皆従不忍生。

見機而耐性,妙悟生光明。

佛語戒無論,儒書貴莫爭。

好個快活路,只是少人行。
\end{myquote}

話説陳經濟過了兩日,到第三日,卻是五月二十五日他生日。春梅後廳整置酒餚,與他上壽,合家歡楽了一日。次日早晨,經濟說:「我一向不曾往河下去,今日沒事,去走一遭。一者和主管算帳,二來就避炎散暑,走走便囬。」春梅吩咐:「你去坐一乘轎子,少要勞碌。」敎兩個軍牢擡着轎子,小喜兒跟隨,逕往河下馬頭上謝家大酒樓店中來。

一路無詞,午後時分,早到河下大酒樓前,下了轎子,進入裏面。兩個主管齊來參見,說:「官府貴體好些?」那經濟一心只在韓愛姐身上,便道:「生受二位夥計掛心。」坐了一囬,便起身。吩咐主管:「查下帳目,等我來算。」就轉身到後邊。八老又早迎見,報與王六兒夫婦。韓愛姐正在樓上凭欄盼望,揮毫洒翰,作了幾首詩詞,以遣悶懷。忽報陳經濟來了,連忙輕移蓮步,欵蹙湘裙,走下樓來。母子面上堆下笑來迎接,說道:「官人,貴人難見面,那陣風兒吹你到俺這裏!」經濟與母子作了揖,同進入閣兒内坐定。少頃,王六兒點茶上來。喫畢茶,愛姐道:「請官人到樓上奴房内坐。」經濟上的樓來,兩個如魚得水,似漆投膠,無非説些深情密意的話兒。愛姐硯臺底下露出一幅花箋,經濟取來觀看。愛姐便說:「此是奴家這幾日盼你不來,閑中在樓上作得幾首詞,以消遣悶懷,恐汚官人貴目!」經濟念了一遍,上寫着:

\begin{myquote}
「倦倚繡牀愁懶動,閒垂繡帶鬢鬟低。

玉郎一去無消息,一日相思十二時。

\hspace*{2em}{\marktext{右春}}

危樓高處眺晴光,滿架薔薇靄異香。

十二欄杆閑凭遍,南薰一味透襟凉。

\hspace*{2em}{\marktext{右夏}}

帳冷芙蓉夢不成,知心人去轉傷情。

枕邊淚似堦前雨,隔着窗兒滴到明。

\hspace*{2em}{\marktext{右秋}}

羞對菱花試新粧,為郎瘦損减容光。

閉門不管閒風月,吩咐梅花自主張。

\hspace*{2em}{\marktext{右冬」}}
\end{myquote}

經濟看了,極口稱羡,喝采不已。不一時,王六兒安排酒餚上樓,撥過鏡架,就擺在梳粧桌上。兩個並坐,愛姐篩酒一盃,雙手遞與經濟,深深道個萬福,說:「官人一向不來,妾心無時不念。前八老來,又多謝盤纏,擧家感之不盡!」經濟接酒在手,還了喏說:「賤疾不安,有失期約,姐姐休怪!」酒盡,也篩一盃,敬奉愛姐喫過。兩人坐定,把酒來斟。王六兒韓道國上來也陪喫了幾盃,各取方便下樓去了,教他二人自在喫幾盃,叙些闊别話兒。良久,喫得酒濃時,情興如火,免不得再把舊情一叙。交歡之際,無限恩情。穿衣起來,洗手更酌,又飲數盃。醉眼朦朧,餘興未盡。這小郎君一向在家中不快,又心在愛姐,一向未與渾家行事。今日一旦見了情人,未肯一次即休。正是生死寃家,五百年前撞在一處,經濟魂靈,都被他引亂。少頃,情竇復起,又幹一度。自覺身體困倦,打熬不過,午飯也沒喫,倒在牀上就睡着了。

也是合當祸起,不想下邊販絲綿何官人來了,王六兒陪他在樓下喫酒。韓道國出去街上,買菜蔬餚品菓子來配酒。兩個在下邊行房。落後韓道國買將菓菜來,三人又喫了幾盃。約日西時分,只見洒家店坐地虎劉二,喫的酩酊大醉,袒開衣衫,露着一身紫肉,提着拳頭,走來酒樓下,大叫「採去何蠻子來」,要打。唬的兩個主管,見經濟在樓上睡,恐他聽見,慌忙走出櫃來,向前聲喏,說道:「劉二哥,何官人並不曾來。」這劉二那裏依聽,大拔步撞入後邊韓道國屋裏,一手把門簾扯下半邊來,見何官人正和王六兒並肩飲酒,心中大怒,罵那何官人:「賊狗男女,我㒲你娘!那裏沒尋你,卻在這裏!你在我店中占着兩個粉頭,幾遭歇錢不與,又塌下我兩個月房錢,卻來這裏養老婆!」那何官人忙出來道:「老二,你請回,我去也。」那劉二罵道:「去你這狗㒲的!」不防颼的一拳來,正打在何官人面門上,登時就青膅起來。那何官人起來奪門跑了。劉二將王六兒酒桌一脚登翻,家活都打了。王六兒便罵道:「是那裏少死的賊殺才,無事來老娘屋裏放屁?老娘不是耐驚耐怕兒的人!」被劉二向前一脚,跥了個仰八叉,罵道:「我㒲你淫婦娘!你是那裏來的無名少姓私窠子,不來老爺手裏報過,許你在這酒店内趁熟?還與我搬去!若搬遲,須喫我一頓好拳頭!」那王六兒道:「你是那裏來的光棍搗子?老娘就沒個親戚兒,許你便來欺負老娘?要老娘這命做甚麽?」一頭撞倒,哭起來。劉二罵道:「我把淫婦腸子也踢斷了,你還不知老爺是誰哩!」這裏喧亂,兩邊鄰舍並街上過往人,登時圍看的有許多。有知道的旁邊人說王六兒:「你新來,不知他是守備老爺府中管事張虞侯的小舅子,有名坐地虎劉二,在洒家店住,專一是打粉頭的班頭,降酒客的領袖。你讓他些兒罷,休要不知利害,這地方人誰敢惹他!」王六兒道:「還有大似他的,睬這殺材做甚麽!」陸秉義見劉二打得兇,和謝胖子做好做歹,把他勸的去了。

陳經濟正睡在牀上,聽見樓下嚷亂,便起來看時,天已日西時分,問:「那裏嚷亂?」那韓道國不知走的往那裏去了,只見王六兒披髮垢面上樓,如此這般告訴説:「那裏走來一個殺材搗子,諢名喚坐地虎劉二,在洒家店住,說是咱府裏管事張虞侯小舅子,因尋酒客,無事把我踢打罵了恁一頓去了!又把家活酒器,都打得粉碎。」一面放聲大哭起來。經濟叫上兩個主管問他,兩個都面面相覷不敢説。陸主管嘴快,説是:「府中張主管小舅子,來這裏尋何官人,説少他二個月房錢,又是歇錢,來討。見他在屋裏喫酒,不由分説,把簾子扯下半邊來,打了何官人一拳,唬的何官人跑了。又和老韓娘子兩個相罵,踢了一跤,哄的滿街人看。」這經濟恐怕天晚惹起事來,吩咐把衆人喝散,問:「劉二那廝如今在那裏?」主管道:「被小人勸他囘去了。」經濟聽了,記在心内,安撫王六兒母子:「放心,有我哩!不妨事,你母子只情住着。我家去自有處置。」主管算了利錢銀兩,遞與他,打發起身上轎,伴當跟隨,剛趕進城來,天已昏黑。心中甚惱,到家見了春梅,交了利息銀兩,歸入房中,一宿無話。到次日,心心念念要告春梅說,展轉尋思,「且住,等我慢慢尋張勝那廝幾件破綻,一發教我姐姐對老爺說了,斷送了他性命。叵耐這廝幾次在我身上欺心,敢說我是他尋得來,知我根本出身,量視我禁不得他!」正是:

\begin{myquote}
寃仇還報當如此,機會遭逢莫遠圖。

踏破鐵鞋無覓䖏,得來全不費工夫。
\end{myquote}

一日,經濟來到河下酒店内,見了愛姐母子說:「外日喫驚。」又問陸主管道:「劉二那廝不曾走動?」陸主管道:「自従那日去了,再不曾來。」又問韓愛姐:「那何官人也没來行走?」愛姐道:「也不曾來。」這經濟喫了飯,算畢帳目,不免又到愛姐樓上,兩個叙了回衷腸之話,幹訖一度出來。因閑中叫過量酒陳三兒近前:「如此這般,打聽府中張勝和劉二幾樁破綻。」這陳三兒千不合萬不合,說出張勝包占着府中出來的雪娥,在洒家店做婊子;劉二又怎的各處巢窝加三討利,擧放私債,「竊逞老爺行壞事。」這經濟一一聽記在心。又與了愛姐二三兩盤纏,和主管算了帳目,包了利息銀兩,作别騎頭口來家。

閑話休題,一向懷意在心。一者也是寃家相凑,二來合當祸這般起來。不料東京朝中徽宗天子,見大金人馬犯邊,搶至腹内地方,聲息十分緊急。天子慌了,與大臣計議,差官往北國講和,情願每年輸納歲幣金銀彩帛數百萬。一面傳位與太子登基,改宣和七年為靖康元年,宣帝號為欽宗。皇帝在位,徽宗自稱太上道君皇帝,退居龍德宫。朝中陞了李綱為兵部尚書,分部諸路人馬;種師道為大將,總督内外軍務。一日,降了一道勑書來濟南府,陞周守備為山東都統制,提調人馬一萬,往東昌府駐扎,會同巡撫都御史張叔夜,防守地方,阻當金兵。守備正在濟南府衙正坐,忽然左右來報:「有朝廷降勑來,請老爺接旨意!」這周守備不敢怠慢,香案迎接勅旨,跪聽宣讀。使命官開讀,其略曰:

\begin{myquote}[\markfont]
「奉天承運,皇帝制曰:朕聞文能安邦,武能定國;三皇憑禮楽而有封疆,五帝用征伐而定天下。事従順逆,人有賢愚。朕承祖宗不拔之洪基,

上皇付托之重位,創造萬事,惕然悚懼。自古舜征四兇,湯伐有苗,非用兵而不能尅,非威武而莫能安。兵乃邦家爪牙,武實封疆扞禦。茲者中原陸沉,犬羊犯順,遼寇擁兵西擾,金虜控騎南侵,生民塗炭,朕甚憫焉。山東濟南制置使周秀,老練之才,干城之將,屢建奇勳,忠勇茂著,用兵有略,出戰有方。今陞為山東都統制,兼四路防禦使。會同山東巡撫都御史張叔夜,提調所部人馬,前赴高陽關防守,聽大將種師道分布截殺。安幾危之社稷,驅猖獗之腥膻!嗚乎,任賢匡國,赴難勤王,乃臣子之忠誠;旌善賞功,激揚敵愾,實朝廷之大典。各殫厥忠,以副朕意。欽哉!故諭。

\raggedleft{{\kaishu(下書)}靖康元年秋九月 日諭。」}

\end{myquote}

周守備開讀已畢,打發使命官去了。一面叫過張勝李安兩個虞侯,近前吩咐:「先押兩車箱馱行李細軟器物家去。」原來在濟南做了一年官職,也賺得巨萬金銀,都裝在行李馱箱内,委托二人:「押到家中,交割明白,晝夜巡風仔細!我不日會同你巡撫張爺,調領四路兵馬,打清河縣起身。」二人當日領了鈞旨,打點車輛,起身先行。一路無詞。有日到於府中,交割明白。二人晝夜内外巡風,不在話下。

卻說陳經濟見張勝押車輛來家,守備陞了山東統制,不久將到,正欲把心腹中事,要告訴春梅,等守備來家,要發露張勝之事。不想一日,因渾家葛翠屏往娘家囬門住去了,他獨自個在西書房寢歇,春梅早晨驀進房中看他。見無丫鬟跟隨,兩個就解衣在房内雲雨做一處。不防張勝搖着鈴巡風過來,到書院角門外,聽見書房内彷彿有婦人笑語之聲,就把鈴聲按住,慢慢走來窗下竊聽。原來春梅在裏面與經濟交媾,聽見經濟告訴春梅說:「叵耐張勝那廝,好生欺壓於我,說我當初虧他尋得來,幾次在下人前敗壞我。昨日見我在河下開酒店來,一徑使小舅子坐地虎劉二打我酒店來,昨日把酒客都打散了。專一倚逞他在姐夫麾下,教他小舅子劉二在那裏開窠窝,放私債,把出去雪娥隱占在外姦宿,只瞞了姐姐一人眼目。我幾次含忍,不敢告姐姐說。趁姐夫來家,若不早說知,往後我定然不敢往河下做買賣去了。」春梅聽了,說道:「這廝恁般無禮!雪娥那賤人賣了,他如何又留住在外?」經濟道:「他非是欺壓我,就是欺壓姐姐一般!」春梅道:「等他爺來家,教他定結果了這廝。」

常言道:隔牆須有耳,窗外豈無人。兩個只管在内說,卻不知張勝窗外聽了個不亦楽乎。口中不言,心内暗道:「比是教他算計我們,我先算計了他罷!」一面撇下鈴,走到前邊班房内,取了把解腕鋼刀,說時遲,那時快,在石上磨了兩磨,走入書院中來。不想天假其便,還是春梅不該死於他手!忽被後邊小丫鬟蘭花兒慌慌走來叫春梅,報說:「小衙内金哥兒忽然風搐倒了,快請奶奶看去。」唬的春梅兩步做來一步走,奔入後房中看孩兒去了。剛進去了,那張勝提着刀子逕奔到書房内,不見春梅,只見經濟睡在被窝内。見他進來,呌道:「阿呀,你來做甚麽?」張勝怒道:「我來殺你!你如何對淫婦說,倒要害我?我尋得你來不是了,反恩將仇報?常言黑頭虫兒不可救,救之就要喫人肉。休走,喫我一刀子,明年今日,是你死忌!」那經濟光赤條身子,沒處躱,摟着被。乞他拉被過一邊,向他身就扎了一刀子來。扎着軟肋,鮮血就邈出來。這張勝見他掙扎,復又一刀去,攮着胸膛上,動彈不得了!一面採着頭髮,把頭割下來。正是:三寸氣在千般用,一日無常萬事休!可憐經濟青春不上三九,死於非命。

張勝提刀,繞屋裏牀背後尋春梅不見,大拔步逕望後廳走。走到儀門首,只見李安背着牌鈴,在那裏巡風。一見張勝兇神也似提着刀跑進來,便問:「那裏去?」張勝不答,只顧走。被李安攔住,張勝就向李安戳一刀來。李安冷笑,說道:「我叔叔是有名山東夜叉李貴,我的本事不用借。」早飛起右脚,只聽忒楞的一聲,把手中刀子踢落一邊。張勝急了,兩個就揪採在一䖏。被李安一個潑脚,跌翻在地,解下腰間纏帶,登時綁了。嚷的後廳春梅知道,說:「張勝持刀入内,小的拿住了。」那春梅方救得金哥甦省,聽言大驚失色,走到書院内,經濟已被殺死在房中,一地鮮血横流,不覺放聲大哭。一面使人報知他渾家葛翠屏,慌奔家來,看見經濟殺死,哭倒在地,不省人事,被春梅扶救甦省過來。拖過屍首,買棺材裝殯。把張勝墩鎖在監内,單等統制來家,處治這件事。

那消數日期程,軍情事務緊急,兵牌來催促,周統制調完各路兵馬,張巡撫又早先往東昌府,那裏等候取齊。統制到家,春梅把殺死經濟一節說了。李安將兇器放在面前,跪稟前事。統制大怒,坐在廳上,提出張勝,也不問長短,喝令軍牢五棍一換,打一百棍,登時打死。隨即馬上差旗牌快手,往河下捉拿坐地虎劉二,鎖解前來。孫雪娥見拿了劉二,恐怕拿他,走到房中,自縊身死。旗牌拿劉二到府中,統制也吩咐打一百棍,當日打死。哄動了清河縣,大鬧了臨清州。正是:平生作惡欺天,今日上蒼報應。有詩為證:

\begin{myquote}
為人切莫用欺心,擧頭三尺有神明。

若還作惡無報應,天下兇徒人食人。
\end{myquote}

當時統制打死二人,除了地方之害。吩咐李安將馬頭大酒店還歸本主,把本錢收算來家。吩咐春梅在家,與經濟做齋累七,打發城外永福寺擇吉日葬埋。留李安周義看家,把周忠周仁帶去軍門答應。春梅晚夕與孫二娘置酒送餞,不覺簇地兩行淚下說:「相公此去,未知幾時囘還。出戰之間,須要仔細。番兵猖獗,不可輕敵。」統制道:「你們自在家清心寡慾,好生看守孩兒,不必憂念。我旣受朝廷爵祿,盡忠報國。至於吉兇存亡,付之天也。」囑付畢,過了一宿。次日軍馬都在城外屯集,等候統制起程。果然人馬整齊,但見:

\begin{myquote}
繡旗飄號帶,畫鼓間銅鑼。三股叉,五股叉,燦燦秋霜;蘆葉鎗,點鋼鎗,紛紛瑞雪。蠻牌引路,強弓硬弩當先;火炮隨車,大斧長刀在後。鞍上將似南山猛虎,人人好鬦偏爭;坐下馬如北海蛟虬,騎騎能爭敢戰。端的刀鎗流水急,果然人馬撮風行。
\end{myquote}

當下一路無詞。有日哨馬來報說:「不可前進,馬哨達東昌府下。」統制差一面令字藍旗,把人馬屯城外,俄報進城。巡撫張叔夜聽見周統制人馬來到,與東昌府知府達天道出衙迎接,至公廳,叙禮坐下,商議軍情,打聽聲息緊慢。駐馬一夜,次日人馬早行,往關上防守去了。不在話下。

卻表韓愛姐母子在謝家樓店中,聽見經濟已死,愛姐晝夜只是哭泣,茶飯都不喫,一心只要往城内統制府中,見經濟屍首一見,死了也甘心。父母旁人百般勸解不従。韓道國無法可䖏,使八老往統制府中打聽,說經濟靈柩已出了殯,埋在城外永福寺内。這八老走來囘了話。愛姐一心只要到他墳上燒紙,哭一場,也是和他相交一場。做父母的只得依他。僱了一乘轎子,到永福寺中,問長老葬於何䖏。長老令沙彌引到寺後:「新墳堆便是。」這韓愛姐下了轎子,到墳前點着紙錢,道了萬福,叫聲:「親郎,我的哥哥!奴實指望和你同諧到老,誰想今日死了!」放聲大哭,哭的昏暈倒了,頭撞於地下,就死過去了。慌了韓道國和王六兒向前扶救,叫「姐姐」叫不應,越發慌了。不想那日,正是葬了三日,春梅與渾家葛翠屏坐着兩乘轎子,伴當跟隨,擡三牲祭物來與他煖墓燒紙。看見一個年小的婦人,穿着縞素,頭戴孝髻,哭倒在地;一個男子漢和一中年婦人摟抱他,扶起來又倒了,不省人事,喫了一驚。因問那男子漢:「是那裏的?」這韓道國夫婦向前施禮,把従前已往話告訴了一遍:「這個是我的女孩兒韓愛姐。」春梅一聞愛姐之名,就想起昔日曾在西門慶家中會過,又認得王六兒。韓道國悉把東京蔡府中出來一節,說了一遍:「女孩兒曾與陳官人有一面相交,不料死了,他只要來墳前見他一見,燒紙錢。不想到這裏又哭倒了。」當下兩個救了半日,這愛姐吐了口粘痰,方纔甦省,尚哽咽哭不出聲來。痛哭了一場,起來與春梅翠屏插燭也似磕了四個頭,說道:「奴與他雖是露水夫妻,他與奴説山盟言海誓,情深意厚。實指望和他同諧到老,誰知天不従人願,一旦他先死了,撇得奴四不着地。他在日曾與奴一方吳綾帕兒,上有四句情詩。知道宅中有姐姐,奴願做小。倘不信……」向袖中取出吳綾帕兒來,上面寫詩四句。春梅同葛翠屏看了,詩云:

\begin{myquote}
「吳綾帕兒織迴紋,洒翰揮毫墨跡新。

寄與多情韓五姐,永諧鸞鳳百年情。」
\end{myquote}

愛姐道:「奴也有個小小鴛鴦錦囊,與他佩帶在身邊。兩個都扣繡着並頭蓮,每朶蓮花瓣兒一個字兒:『寄與情郎陳君膝下。』」春梅便問翠屏:「怎的不見這個香囊?」翠屏道:「在他𧜽子上拴着不是,奴替他裝殮在棺槨内了。」

當下祭畢,讓他母子到寺中,擺茶飯與他喫了些飯食。做父母的見天色將晚,催促他起身。他只顧不思動身。一面跪着春梅葛翠屏哭說:「情願不歸父母,同姐姐守孝寡居,也是奴和他恩情一場,活是他妻小,死傍他魂靈。」那翠屏只顧不言語。春梅便說:「我的姐姐,只怕年小青春,守不住。只怕悞了你好時光!」愛姐便道:「奶奶説那裏話。奴旣為他,雖刳目斷鼻,也當守節,誓不再配他人!」囑付他父母:「你老公母囬去罷,我跟奶奶和姐姐府中去也!」那王六兒眼中垂淚;哭道:「我承望你養活俺兩口兒到老,纔従虎穴龍潭中奪得你來,今日倒閃賺了我!」那愛姐口裏只說:「我不去了。你就留下我,到家也尋了無常!」那韓道國因見女孩兒堅意不去,和王六兒大哭一場,洒淚而別,囬上臨清店中去了。這韓愛姐同春梅翠屛坐轎子往府裏來。那王六兒一路上悲悲切切,只是捨不的他女兒,哭了一場又一場。那韓道國又怕天色晚了,僱上兩疋頭口,望前趕路。正是:

\begin{myquote}
馬遲心急路途窮,身似浮萍類轉蓬。

只有都門樓上月,照人離恨各西東。
\end{myquote}

畢竟未知後來如何,且聽下囬分解。

