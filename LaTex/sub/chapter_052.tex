\includepdf[pages={103,104},fitpaper=false]{tst.pdf}
\chapter*{第五十二囬 \\應伯爵山洞戲春嬌 潘金蓮花園看蘑菇}
\addcontentsline{toc}{chapter}{第五十二囬 應伯爵山洞戲春嬌 潘金蓮花園看蘑菇}
\markboth{{\titlename}卷之六}{第五十二囬 應伯爵山洞戲春嬌 潘金蓮花園看蘑菇}


\begin{myquote}
海棠深院雨初收,苔徑無風蝶自由。

百結丁香誇美麗,三眠楊柳弄輕柔;

小桃酒膩紅尤淺,芳草寒餘綠漸稠。

寂寂珠簾歸燕子,子規啼處一春愁。
\end{myquote}

話説那日西門慶在夏提刑家吃酒,宋巡按送禮與他,心中十分歡喜;夏提刑亦敬重不同往日,攔門勸酒,吃至二更天氣纔放回家。潘金蓮又早向燈下除去冠兒,露着粉面油頭,教春梅床上設放衾枕,搽抹凉蓆乾淨,薰香澡牝,等候西門慶。進門接着,見他酒帶半酣,連忙替他脱了衣裳,春梅點茶來吃了,打發上床歇息。見婦人脱得光赤條身子,坐着床沿,低垂着頭,將那白生生腿兒横抱膝上纏脚,換了雙剛三寸,恰半扠,大紅平底睡鞋兒。西門慶一見,淫心輒興,麈柄挺然而起,因問婦人要淫器包兒。婦人連忙向褥子底下摸出來,遞與他。西門慶把兩個託子都帶上,一手摟過婦人在懷裏,因説:「你達今日要和你幹個後庭花兒,你肯不肯?」那婦人瞅了一眼,説道:「好個沒廉耻寃家!你成日和書童兒小廝幹的不値了,又纏起我來了。你和那奴才幹去不是!」西門慶笑道:「怪小油嘴兒,罷麽!你若依了我,又稀罕小廝做甚麽?你不知,你達心裏好的是這樁兒。管情放到裏頭去,我就過了。」婦人被他再三纏不過,説道:「奴只怕挨不的你這大行貨,你把頭子上圈去了一個,我和你耍一遭試試。」西門慶眞個除去硫黃圈,根下只束着銀託子,令婦人馬爬在床上,屁股高蹶,將唾津塗抹在龜頭上,往來濡硏頂入。龜頭昂健,半晌僅沒其稜,婦人在下,蹙眉隱忍,口中咬汗巾子難捱,叫道:「達達慢着些!這個比不的前頭,撑得裏頭熱炙火燎疼起來。」這西門慶叫道:「好心肝,你叫着達達,不防事。到明日,買一套好顔色粧花紗衣服與你穿。」婦人道:「那衣服倒也有在。我昨日見李桂姐穿的那玉色線掐羊皮金挑的油鵝黄銀條紗裙子倒好看,説是裏邊買的。他們都有,只我沒這條裙子。倒不知多少銀子,你倒買一條我穿罷了!」西門慶道:「不打緊,我到明日替你買。」一壁説着,在上頗作抽拽,只顧沒稜露腦,淺抽深送不已。婦人囘首流眸呌道:「好達達,這裏緊着人疼的了不的,如何只顧這般動作起來了?我央及你,好歹快些丢了罷!」這西門慶不聽,且扶其股,玩其出入之勢,一面口中呼道:「潘五兒,小淫婦兒,你好生浪浪的呌着達達,哄出你達達㞞兒來罷!」那婦人眞個在下星眼朦朧,鶯聲款掉,柳腰款擺,香肌半就,口中艷聲柔語,百般難述。良久,西門慶覺精來,兩手扳其股,極力而𢵞之,扣股之聲響之不絶。那婦人在下邊呻吟成一塊,不能禁止。臨過之時,西門慶把婦人屁股只一扳,麈柄盡沒至根,直抵於深異處,其美不可當,於是怡然感之,一泄如注。婦人承受其精。二體偎貼良久,拽出麈柄,但見猩紅染莖,蛙口流涎,婦人以帕抹之,方纔就寢。一宿晚景題過。

次日,西門慶早晨到衙門中囬來,有安主事黄主事那裏差人來下請書:二十二日,在磚廠劉太監庄上設席,請早去。西門慶打發人去了,従上房吃了粥,正出廳來。只見篦頭的小周兒趴倒地下磕頭,在傍伺候。西門慶道:「你來得正好,我正要尋你篦篦頭哩。」於是走到花園翡翠軒小捲棚内,西門慶坐在一張凉椅兒上,除了巾幘,打開頭髮。小周兒在後面桌上鋪下梳篦家活,與他篦頭櫛髮,觀其泥垢,辨其風霜,跪下討賞錢,説:「老爹今歲必有大遷轉,髮上氣色甚旺!」西門慶大喜,篦了頭,又敎他取耳,掐捏身上。他有滚身上一弄兒家活,到處都與西門慶滚捏過,又行導引之法,把西門慶弄的渾身通泰,賞了他五錢銀子,敎他吃了飯,伺候與哥兒剃頭。西門慶就在書房内,倒在大理石床上就睡着了。

那日楊姑娘起身,王姑子與薛姑子要家去。吳月娘將他原來的盒子,都裝了些蒸酥茶食,打發起身。兩個姑子,每人又是五錢銀子;兩個小姑子,與了他兩疋小布兒,管待出門。薛姑子又囑付月娘:「到壬子日,把那薬吃了,管情就有喜事。」月娘道:「薛爺,你這一去,八月裏到我生日好歹走走,我這裏盼你哩!」薛姑子合掌問訊,道:「打攪菩薩這裏!我到那日一定來。」於是作辭月娘,衆人都送到大門首。

月娘與大妗子囬後邊去了,只有孟玉樓、潘金蓮、李瓶兒、西門大姐、李桂姐——穿着白銀條紗對衿衫兒,鵝黃縷金挑線紗裙子,戴着銀絲䯼髻,翠水祥雲鈿兒,金累絲簪子,紫英石墜子,大紅鞋兒,——抱着官哥兒,來花園裏遊玩。李瓶兒道:「桂姐,你遞過來,等我抱罷。」桂姐道:「六娘,不妨事,我心裏要抱抱哥子。」孟玉樓道:「桂姐,你還沒到你爹新收拾書房兒瞧瞧!」來到花園内,金蓮見紫薇花開得爛熳,摘了兩朵與桂姐戴。於是順着松墻兒到翡翠軒,見裏邊擺設的床帳屏几,書畫琴棋,極其瀟灑。床上綃帳銀鉤,冰簟珊枕,西門慶正倒在床上,睡思正濃。傍邊流金小篆,焚着一縷龍涎。綠窻半掩,窻外芭蕉低映。那潘金蓮且在桌上掀弄他的香盒兒,玉樓和李瓶兒都坐在椅兒上。西門慶忽翻過身來,看見衆婦人都在屋裏,便道:「你們來做甚麽?」金蓮道:「桂姐要看看你的書房哩,俺們引他來瞧瞧。」那西門慶見他抱着官哥兒,又引鬬了一囬。忽見畫童來説:「應二爹來了。」衆婦人都亂走不迭,往李瓶兒那邊去了。

應伯爵走到松墻邊,看見桂姐抱着官哥兒,便道:「好呀,李桂姐在這裏!」故意問道:「你幾時來?」那桂姐走了走説道:「罷麽,怪花子,又不関你事,問怎的?」伯爵道:「好小淫婦兒,不關我事?也罷,你且與我個嘴罷。」於是摟過來就要親嘴。被桂姐用手只一推,駡道:「賊不得人意怪攮刀子!若不是怕唬了哥子,我這一扇把子打的你!」西門慶走出來,看見伯爵拉着桂姐,説道:「怪狗才,看唬了孩兒!」因教書童:「你抱哥兒,送與你六娘去。」那書童連忙接過來。奶子如意兒正在松墻拐角邊等候,接的去了。伯爵和桂姐兩個站着説話,問:「你的事怎樣的?」桂姐道:「多虧爹這裏可憐見,差保哥替我往東京説去了。」伯爵道:「好好!也罷了,如此你放心些。」説畢,桂姐就往後邊去了。伯爵道:「怪小淫婦兒,你過來,我還和你説話。」桂姐道:「我走走就來。」於是也往李瓶兒這邊來了。伯爵與西門慶纔唱喏,兩個在軒内坐的。西門慶道:「昨日我在夏龍溪家吃酒,大巡宋道長那裏差人送禮,送了一口鮮猪。我恐怕放不的,今早旋叫了廚子來卸開,用椒料連猪頭燒了。你休去了,如今請了謝子純來,咱們打雙陸,同享了罷。」一面使琴童兒:「快請你謝爹去,你説應二爹在這裏。」琴童兒應諾,一直去了。伯爵因問:「徐家銀子,討了來了?」西門慶道:「賊沒行止的狗骨秃!明日纔有,先與二百五十兩。你教他兩個後日來,少的我家裏凑與他罷。」伯爵道:「這等又好了。怕不的他今日買些鮮物兒來孝順你!」西門慶道:「倒不消教他費心。」説了一囬。西門慶問道:「老孫祝麻子兩個,都起身去了不曾?」伯爵道:「這咱哩!従李桂兒家拏出來,在縣裏監了一夜,第二日,三個一條鐵索都解上東京去了。到那裏,沒個清潔來家的。你只説成日圖飲酒块肉娼家串,好容易吃的菓子兒!似這等苦兒也是他受。路上這等大熱天,着鐵索扛着,又沒盤纏,有甚麽要緊!」西門慶笑道:「怪狗才,充軍擺站的不過?誰敎他成日跟着王家小廝只胡撞來?本亦他尋的苦兒他受!」伯爵道:「哥,你説的有理。蒼蝇不鑽沒縫的雞彈!他怎的不尋我和謝子純?清的只是清,渾的只是渾!」

正説着,謝希大到了。唱畢喏坐下,只顧搧扇子。西門慶問道:「你怎的走恁一臉汗?」希大道:「哥別題!大官兒去遲了一步兒,我不在家了。我剛出大門,可可他就到了。今日平白惹了一肚子氣!」伯爵問道:「你惹的又是甚麽氣?」希大道:「大清早晨,老孫媽媽子走到我那裏,説我弄了他去!因主何故?恁不合理的老淫婦!你家漢子成日摽着人在院裏頑,碗酒塊肉吃,大把家撾了銀子錢家去,你過陰去來?誰不知道?你討保頭錢,分與那個一分兒使也怎的!教我扛了兩句,走出來,不想哥這裏呼喚。」伯爵道:「我剛纔這裏和哥不説,新酒放在兩下裏,清自清渾自渾,當初咱們怎麽説來?我説跟着王家小廝,到明日必有一失。今日如何!撞到這網裏,怨悵不的人!」西門慶道:「王家那小廝,有甚大氣概?幾年兒了,腦子還未變全!養老婆,還不夠俺們那咱撒下的,羞死鬼罷了。」伯爵道:「他曾見過甚麽大頭面,怎比哥那咱的勾當!提起來,把他唬殺了罷了。」説畢,小廝拏茶上來吃了。西門慶道:「你兩個打雙陸。後邊做着過水麵,等我叫小廝拏麵來咱們吃。」不一時,琴童來放桌兒,畫童兒用方盒拏上四個靠山小碟兒,盛着四樣小菜兒:一碟十香瓜茄,一碟五方荳鼓,一碟醬油浸的鮮花椒,一碟糖蒜;三碟兒蒜汁,一大碗猪肉滷,一張銀湯匙,三雙牙筯。擺放停當,西門慶走來坐下。然後拏上三碗麵來,各人自取澆滷,傾上蒜醋。那應伯爵與謝希大拏起觔來,只三扒兩嚥,就是一碗;兩人登時狠了七碗。西門慶兩碗還吃不了,説道:「我的兒,你兩個吃這些!」伯爵道:「哥今日這麵是那位姐兒下的?又爽口,又好吃。」謝希大道:「本等滷打的停當。我只是剛纔家裏吃了飯來了,不然,我還禁一碗。」兩個吃的熱上來,把衣服脱了,搭在椅子上。見琴童兒收家活,便道:「大官兒,到後邊取些水來,俺們漱漱口。」謝希大道:「溫茶兒又好,熱的盪的死蒜臭。」少頃,畫童兒拏茶至。三人吃了茶,出來外邊松墻外,各花臺邊走了一遭。只見黃四家送了四盒子禮來,平安兒掇進來與西門慶瞧,一盒鮮烏菱,一盒鮮荸薺,四尾冰湃的大鰣魚,一盒枇杷果。伯爵看見,説道:「好東西兒!他不知那裏剜的送來,我且嚐個兒着。」一手撾了好幾個,遞了兩個與謝希大,説道:「還有活到老死還不知此物甚麽東西兒哩!」西門慶道:「怪狗才,還沒供養佛,就先撾了吃。」伯爵道:「甚麽沒供佛,我且入口無贓着。」西門慶吩咐:「交到後邊收了。問你二娘討三錢銀子賞他。」伯爵問:「是李錦送來?是黃寜兒?」平安道:「是黃寜兒。」伯爵道:「今日造化了這狗骨秃了,又賞他這三錢銀子。」這裏西門慶看着他兩個打雙陸不題。

且説桂姐和他乾娘、李嬌兒、孟玉樓、潘金蓮、李瓶兒、大姐,都在後邊上房明間内吃了飯,在穿廊下坐的。只見小周兒在影壁前探頭舒腦的。李瓶兒道:「小周兒,你來的好,且進來與小大官兒剃剃頭,把頭髮都長長了。」小周兒連忙向前都磕了頭,説:「剛纔老爹吩咐,教小的進來與哥兒剃頭。」月娘道:「六姐,你拏曆頭看看。好日子歹日子,就與孩子剃頭!」這金蓮便教小玉取了曆頭來,揭開看了一囬,説道:「今日是四月廿一日,是個庚戌日,金定婁金狗當直,宜祭祀、冠帶、出行、裁衣、沐浴、剃頭、修造、動土,宜用午時。好日期!」月娘道:「既是好日子,教丫頭熱水,你替孩兒洗頭。敎小周兒慢慢哄着他剃。」小玉在傍,替他用汗巾兒接着頭髮兒。那裏纔剃得幾刀兒下來,這官哥兒呱的聲怪哭起來。那小周連忙趕着他哭只顧剃。不想把孩子哭的那口氣憋下去,不言語了,臉便脹的紅了。李瓶兒也唬慌手脚,連忙説:「不剃罷,不剃罷!」那小周兒唬的收不迭家活,往外沒腳子跑。月娘道:「我説這孩子有些不長俊,護頭,自家替他剪剪罷。平白叫進來剃,剃的好麽?」天假其便,那孩子憋了半日氣,放出聲來了。李瓶兒一塊石頭方纔落地,只顧抱在懷裏,拍哄着他,説道:「好小周兒,恁大膽,平白進來把哥哥頭來剃了去了!剃的恁半落不合的,欺負我的哥哥!還不拏囬來,等我打與哥哥出氣!」於是抱到月娘跟前。月娘道:「不長俊的小花子兒,剃頭耍子,你便益了,這等哭!剩下這些,到明日做剪毛賊!」引鬬了一回,李瓶兒交與奶子。月娘吩咐:「且休與他奶吃,等他睡一囬兒與他吃。」奶子抱的他前邊去了。只見來安兒進來取小周兒的家活,説:「門首唬的小周兒臉焦黃的。」月娘問道:「他吃了飯不曾?」來安道:「他吃了飯,爹賞他五錢銀子。」月娘教來安:「你拏一甌子酒出去與他。唬着人家,好容易討這幾個錢!」小玉連忙篩了一盞,拏了一碟臘肉,教來安與他吃了,往家去了。

吳月娘因教金蓮:「你看看曆頭,幾時是壬子日?」金蓮看了,説道:「二十三是壬子日,交芒種五月節。」便道:「姐姐,你問他怎的?」月娘道:「我不怎的,問一聲兒。」李桂姐接過曆頭來看了,説道:「這二十四日苦惱,是俺娘的生日,我不得在家。」月娘道:「前月初十日,是你姐姐生日,過了。這二十四日,可可兒又是你媽的生日了!原來你院中人家,一日害兩樣病,做三個生日:日裏害思錢病,黑夜思漢子的病;早晨是媽的生日,晌午是姐姐生日,晚夕是自家生日。怎的都擠在一塊兒?趂着姐夫有錢,攛掇着都生日了罷!」桂姐只是笑,不做聲。只見西門慶使了畫童兒來請,桂姐方向月娘房中粧點匀了臉,往花園中來。

捲棚内又早放下八仙桌兒,前後放下簾櫳來。桌上擺設許多餚饌:兩大盤燒猪肉,兩盤燒鴨子,兩盤新蒸鮮鰣魚,四碟玫瑰點心,兩碟白燒笋鷄,兩碟燉爛鴿子雛兒。然後又是四碟臟子:血皮、豬肚、釀腸之類。衆人吃了一囬,桂姐在傍拏鍾兒遞酒。伯爵道:「你爹聽着説,不是我索落你,事情兒已是停當了。你爹又替你縣中説了,不尋你了。虧了誰?還虧了我再三央及你爹,他纔肯了。平白他肯替你説人情去了?隨你心愛的甚麽曲兒,你唱個兒我聽下酒,也是拏勤勞准折。」桂姐笑罵道:「怪硶花子,你虼蚤兒好大面皮兒!爹他肯信你説話?」伯爵道:「你這賊小淫婦兒,你經還沒唸,就先打和尚起來!要吃飯,休要惡了火頭。你敢笑和尚沒丈母?我就單丁擺佈不起你這小淫婦兒?你休笑話,我半邊俏,還動的!」被桂姐拏手中扇把子,儘力向他身上打了兩下。西門慶笑罵道:「你這狗才,到明日論個男盜女娼,還虧了原問處。」笑了一囬,桂姐慢慢纔拏起琵琶,横擔膝上,啟朱唇,露皓齒,唱了個〔伊州三台令〕:

\begin{myquote}
「思量你好辜恩,便忘了誓盟。遇花朝月夕良辰,好教我虚度了青春。悶懨懨把欄杆凭倚,凝望他怎生全無個音信?幾囬自忖,多應是我分薄緣輕。」

{\markfont〔黄鶯兒〕}「誰想有這一程,」{\markfont\small\color{mydarkgray}(伯爵道:「陽溝裏翻了舡,後十年也不知道。」)}「減香肌,憔瘦損;」{\markfont\small\color{mydarkgray}(伯爵道:「愛好貪他,閃在人水裏。」)}「鏡鸞塵鎖無心整,脂粉懶匀,花枝又懶簪;空教我黛眉蹙破春山恨。」{\markfont\small\color{mydarkgray}(伯爵道:「你記的説,接客千個,情在一人。無言對鏡長吁氣,半是思君半恨君。你兩個當初好,如今就為他躭些驚怕兒也罷,不抱怨了!」桂姐道:「汗邪了你,怎的胡説!」)}「最難禁,」{\markfont\small\color{mydarkgray}(伯爵道:「你難禁,別人卻怎樣禁的?」)}「樵樓上畫角,吹徹了断腸聲!」{\markfont\small\color{mydarkgray}(伯爵道:「腸子倒沒断。這一回,來提你的断了線,你兩個休提了。」被桂姐儘力打了一下,罵道:「賊們攮的,今日汗歪了你,只鬼混人的!」)}

{\markfont〔集賢賓〕}「幽窻靜悄月又明,恨獨倚幃屏。驀聽的孤鴻只在樓外鳴,把離愁又還題醒。更長漏永,早不覺燈昏香盡。眠未成,他那裏睡得安穩?」{\markfont\small\color{mydarkgray}(伯爵道:「儍小淫婦兒,他怎的睡不安穩?又沒拏了他去,落荅的在家裏睡覺兒哩。你便在人家躲着,逐日懷着羊皮兒,直等東京人來,一塊石頭方落地。」桂姐被他説急了,便道:「爹,你看應花子來!不知怎的,只發訕纏我!」伯爵道:「你這回纔認得爹了?」桂姐不理他,彈着琵琶又唱:)}

{\markfont〔雙聲疊韻〕}「思量起,思量起,怎不上心?」{\markfont\small\color{mydarkgray}(伯爵道:「揉着你癢癢處,不由你不上心。」)}「無人處,無人處,淚珠兒暗傾。」{\markfont\small\color{mydarkgray}(伯爵道:「一個人慣溺床。那一日,他娘死了,守孝,打鋪在靈前睡。晚了,不想又溺下了。人進來看見褥子濕,問:「怎的來?」那人沒的回答,只説:「你不知,我夜間眼淚打肚裏流出來了。」就和你一般,為他聲説不的,只好背地哭罷了。」桂姐道:「没羞的孩兒,你看見來?汗邪了你哩!」)}「我怨他,我怨他,説他不盡;」{\markfont\small\color{mydarkgray}(伯爵道:「我又一件説,你怎的不怨天,知道得了他多少錢兒?今日躲在人家,把買賣都悞了!説他不盡,是左門神,白臉子,極古來子,不知道甚麽兒的,好哄他。」)}「誰知道,這裏先走滚。」{\markfont\small\color{mydarkgray}(伯爵道:「可知拏着到手中,還飛了哩!」)}「只恨我,當初不合地認眞!」{\markfont\small\color{mydarkgray}(伯爵道:「儍小淫婦兒,如今年程,在這裏三歲小孩兒出來也哄不過,何況風月中子弟!你和他認眞?你且住了,等我唱個〔南枝兒〕你聽:「風月事,我説與你聽!如今年程,論不的假眞,個個人古怪精靈,個個人久慣牢成,倒將計活埋,他瞎缸暗頂。老虔婆只要圖財,小淫婦兒少不的拽着脖子往前掙!苦似投河,愁如覓井。幾時得把業罐子塡完,就變驢變馬也不幹這個營生!」當下把桂姐説的哭起來了。被西門慶向伯爵頭上打了一扇子,笑罵道:「你這謅断了腸子的狗才,生生兒吃你把人就嘔殺了!」因叫桂姐:「你唱,不要理他。」謝希大道:「應二哥,你好沒趣,今日左來右去,只欺負我這乾女兒!你再言語,口上生個大疔瘡!」那桂姐半日拏起琵琶又唱:)}

{\markfont〔簇御林〕}「人都道,他志誠,」{\markfont\small\color{mydarkgray}(伯爵纔待言語,被希大把口按了,説道:「桂姐,你唱,休理他!」李桂姐又唱道:)}「卻原來廝勾引。眼睜睜,心口不相應。」{\markfont\small\color{mydarkgray}(希大放了手,伯爵又説:「相應倒好了,弄不出此事來了。心口裏不相應,如今虎口裏倒相應——不多,也只兩三炷兒。」桂姐道:「白眉赤眼,你看見來?」伯爵道:「我沒看見,在楽星堂兒裏不是?」連西門慶衆人都笑起來了。)}「山誓海盟,説假道眞,險些兒不為他錯害了相思病!」{\markfont\small\color{mydarkgray}(伯爵道:「好保蟲兒,只有錯買了的,沒有錯賣了的。你院中人,肯把病兒錯害了?」)}「負心人,看伊家做作,如何敎你有前程?」{\markfont\small\color{mydarkgray}(伯爵道:「前程也不敢指望。他到明日,少不了他個招宣襲了罷!」)}

{\markfont〔琥珀貓兒〕}「日疎日遠,無計再相逢,枉了奴癡心寜耐等。」{\markfont\small\color{mydarkgray}(伯爵道:「等到幾日?到明日東京了畢事,再回爐也是不遲。」)}「想巫山雲雨夢難成。薄情,猛拚今生,和你鳳拆鸞分!」

{\markfont〔尾聲〕}「寃家下得忒薄倖,割捨的將人孤另。那日裏恩情翻成做畫餅!」
\end{myquote}

唱畢,謝希大道:「罷罷!叫畫童兒接過琵琶去,等我酧勞桂姐一盃酒兒!」伯爵道:「等我布菜兒。我本領兒不濟事,拏勤勞准折罷了。」桂姐道:「花子過去,誰理你!你大拳打了人,這囬拏手來摸挲。」當下希大一連遞了桂姐三盃酒。拉伯爵道:「咱們還有那兩盤雙陸,打了罷。」於是二人又打雙陸。西門慶遞了個眼色與桂姐,就往外走。伯爵道:「哥你往後邊去,捎些香茶兒出來。頭裏吃了些蒜,這回子倒反帳兒,惡泛泛起來了。」西門慶道:「我那裏得香茶兒來?」伯爵道:「哥,你還哄我哩。杭州劉學官送了你好少兒着?你獨吃也不好。」西門慶笑的後邊去了。那桂姐也走出來,在太湖石畔推掐花兒戴,也不見了。伯爵與希大一連打了三盤雙陸,等西門慶,白不見出來,問畫童兒:「你爹在後邊做甚麽哩?」畫童兒道:「爹在後邊,就出來了。」伯爵道:「就出來,卻往那去了?」因敎謝希大:「你這裏坐着,等我尋他尋去。」那謝希大且和書童兒兩個在書桌上下象棋。

原來西門慶只走到李瓶兒房裏,就出來了。在木香棚下看見李桂姐,就拉到藏春塢雪洞兒裏,把門兒掩着,兩個坐在矮床兒上説話。原來西門慶走到李瓶兒房裏,吃了薬出來。把桂姐摟在懷中,坐於腿上,一徑露出那話來與他瞧。把桂姐唬了一跳,便問:「怎的就這般大?」西門慶悉把吃胡僧薬,告訴了一遍。先教他低垂粉頸,款啓猩唇,品咂了一囬。然後輕輕搊起他剛半扠、恰三寸、如錐靶、賽藕芽、步香塵、舞翠盤、千人愛、萬人貪兩隻小小金蓮來,跨在兩邊胳膊,——穿着大紅素緞白綾高底鞋兒,粧花金欄膝褲腿兒用紗綠線帶紮着,——抱到一張椅兒上,兩個就幹起來。不想應伯爵到各亭兒上尋了一遭,尋不着,打滴翠巖小洞兒裏穿過去,到了木香棚,抹轉葡萄架,到松竹深處藏春塢邊,隱隱聽見有人笑聲,又不知在何處。這伯爵慢慢躡足潛蹤,掀開簾兒,見兩扇洞門兒虚掩,在外面只顧聽覷。聽見桂姐顫着聲兒,將身子只顧迎播着西門慶叫:「達達,快些了事罷,只怕有人來。」被伯爵猛然大叫一聲,推開門進來,看見西門慶把桂姐扛着腿子,在椅兒上正幹得好,説道:「快取水來,潑潑兩個攮心的,摟到一答裏了。」李桂姐道:「怪攮刀子,猛的進來,唬了我一跳!」伯爵道:「快些兒了事?好容易!也得値那些數兒是的。怕有人來看見,我就來了。且過來,等我抽個頭兒着!」西門慶便道:「怪狗才!快出去罷了,休鬼混我!只怕小廝來看見。」那應伯爵道:「小淫婦兒,你央及我央及兒;不然,我就吆喝起來,連後邊嫂子們都嚷的知道。你既認做乾女兒了,好意叫你躲住兩日兒,你又偷漢子!教你了不成?」桂姐道:「去罷,應怪花子。」伯爵道:「我去罷!我且親個嘴着。」於是按着桂姐,親訖一嘴,纔走出來。西門慶道:「怪狗才,還不帶上門哩!」伯爵一面走來,把門帶上,説道:「我兒,兩個儘着搗儘着搗。搗掉底子,不関我事。」纔走到那個松樹兒底下,又回來説道:「你頭裏許我的香茶,在那裏?」西門慶道:「怪狗才,等住囬我與你就是了,又來纏人!」那伯爵方纔一直笑的去了。桂姐道:「好個不淂人意的攮刀子的!」這西門慶和桂姐兩個在雪洞内,足幹夠約一個時辰,吃了一枚紅棗兒,纔得了事,雨散雲收。有詩為證:

\begin{myquote}
海棠枝上鶯梭急,綠竹陰中燕語頻: 

閒來付與丹青手,一段春嬌畫不成。
\end{myquote}

少頃,二人整衣出來。桂姐向他袖子内,掏出好些香茶來袖了。西門慶則使的滿身香汗,氣喘吁吁,走來馬纓花下溺尿。李桂姐腰裏摸出鏡子來,在月窻上擱着,整雲理鬢,往後邊去了。西門慶走到李瓶兒房裏,洗洗手出來。伯爵問他要香茶,西門慶道:「怪花子,你害了痞?如何只鬼混人!」每人掐了一撮與他。伯爵道:「只與我這兩個兒!由他由他,等我問李家小淫婦兒要。」正説着,只見李銘走來磕頭。伯爵道:「李日新,在那裏來?你沒曾打聽得他們的事怎麽樣兒了?」李銘道:「俺桂姐虧了爹這裏。這兩日縣裏也沒人來催,只等京中示下哩。」伯爵道:「齊家那小老婆子出來了?」李銘道:「齊香兒還在王皇親宅内躲着哩。桂姐在爹這裏好,誰人敢來尋?」伯爵道:「要不然也費手,虧我和你謝爹再三央勸你爹:『你不替他處處兒,教他那裏尋頭腦去?』」李銘道:「爹這裏不管,就了不成;俺三嬸老人家,風風勢勢的,幹出甚麽事!」伯爵道:「我記的這幾時是他生日,俺們會了你爹,與他做做生日。」李銘道:「爹們不消了。到明日,事情畢了,三嬸和桂姐愁不請爹們坐坐。」伯爵道:「到其間,俺們補生日就是了。」因叫他近前:「你且替我吃了這鍾酒着。我吃了這一日了,吃不的了。」那李銘接過銀把鍾來,跪着一飲而盡。謝希大敎琴童又斟了一鍾與他。伯爵道:「你敢沒吃飯?桌上還剩了一盤點心。」謝希大又拏兩盤燒猪頭肉和鴨子遞與他。李銘雙手接的下邊吃去了。伯爵用筯子又撥了半段鰣魚與他,説道:「我見你今年還沒食這個哩,且嚐新着。」西門慶道:「怪狗才,都拏與他吃罷了,又留下做甚麽?」伯爵道:「等住回吃的酒闌上來,餓了,我不會吃飯兒?你們那裏曉得,江南此魚,一年只過一遭兒!吃到牙縫兒裏,剔出來都是香的。好容易!公道説,就是朝廷還沒吃哩!不是哥這裏,誰家有?」正説着,只見畫童兒拏出四碟鮮物兒來:一碟烏菱,一碟荸薺,一碟雪藕,一碟枇杷。西門慶還沒曾放到口裏,被應伯爵連碟子都撾過去,倒的袖了。謝希大道:「你也留兩個兒我吃。」也將手撾一碟子烏菱來,只落下藕在桌子上。西門慶掐了一塊放在口内,別的與了李銘吃了。吩咐畫童後邊再取兩個枇杷來賞李銘。李銘接的袖了,「到家我與三媽吃!」李銘吃了點心上來,拏箏過來,纔彈唱了。伯爵道:「你唱個〔花薬欄〕俺們聽罷!」李銘調定箏弦,拏腔唱道:

\begin{myquote}
「新綠池邊,猛拍欄杆,心事向誰論?花也無言,蝶也無言,離恨滿懷縈牽。恨東君不解留去客,嘆舞紅飄絮,蝶粉輕霑。景依然,事依然,悄然不見郎面。」

{\markfont〔塞鴻秋〕}「俺相别時節正逢春,海棠花初綻蕊,微斥間現。不覺的榴花噴,紅蓮放,沉冰果,避暑搖紈扇。霎時間,菊花黄金風動,敗葉飄桐梧變。逡巡見臘梅開,冰花墜,暖閣内把香醪旋。四季景偏多,思想心中戀。不知俺那俏寃家,冷清清獨自個悶懨懨何處躭寂怨?」

{\markfont〔金殿喜重重〕}「嗟怨。自古風流悞少年,那堪暮春天!生怕到黃昏,愁怕到黃昏,獨自個悶不成歡。換寳香薰被誰共宿?嘆夜長枕冷衾寒。你孤眠,我孤眠,但只是魂夢裏相見。」

{\markfont〔貨郎兒〕}「有一日稱了俺平生心願,成合了夫妻謝天。今生一對兒好姻緣,冷清清躭寂寞,愁沉沉受熬煎。」

{\markfont〔醉太平煞尾〕}「只為俺多情的業寃,今日恨惹情牽。想當初,説山盟海誓在星前,擔閣了風流少年。有一日,朝雲暮雨成姻眷,畫堂歌舞排歡宴;有一日,羅幃錦帳永團圓,花燭洞房成連理,休忘了受過熬煎有萬千!」
\end{myquote}

當日三個吃至掌燈時候,還等着後邊拏出綠荳白米水飯來,吃了纔去。伯爵道:「哥,明日不得閒?」西門慶道:「我明日往磚廠劉太監莊子上,安主事黃主事兩個昨來請我吃酒,早去了。」伯爵道:「李三黄四那事,我後日會他來罷!」西門慶點頭兒,吩咐:「教他那日後晌來,休來早了。」二人也不等送就去了。西門慶敎書童看着收家活,就歸後邊孟玉樓房中歇去了,一宿無話。

到次日,西門慶早起,也沒往衙門中去,吃了粥,冠帶着,騎馬拏着金扇,僕從跟隨,出城南三十里,逕往劉太監莊上來赴席。那日書童與玳安兩個都跟去了,不在話下。潘金蓮趕西門慶不在家,與李瓶兒計較,將陳經濟輸的那三錢銀子,又教李瓶兒添出七錢來,叫來興兒買了一隻燒鴨,兩隻鷄,一錢銀子下飯,一罈金華酒,一瓶白酒,一錢銀子裹餡涼糕,教來興兒媳婦整理端正。金蓮對着月娘説:「大姐那日鬬牌,贏了陳姐夫三錢銀子。李大姐又添七錢,今治了東道兒,請姐姐在花園裏吃。」吳月娘就同孟玉樓、李嬌兒、孫雪娥、大姐、桂姐,先在捲棚内吃了一回。然後拏了酒菜兒,往山子上,一個最高的臥雲亭兒上,那裏下棋投壺耍子。孟玉樓便與李嬌兒、大姐、孫雪娥,都往玩花樓上去,凭欄杆望下看,那山子前面牡丹畦、芍薬圃、海棠軒、薔微架、木香棚、玫瑰樹,端的有四時不謝之花,八節長春之景。觀了一囬下來。小玉迎春卻在臥雲亭上,侍奉月娘,斟酒下菜。月娘猛然想起:「今日倒不請陳姐夫來坐坐!」大姐道:「爹又使他今日往門外徐家催銀子去了,也待好來也。」

不一時,陳經濟來到,穿着玄色練絨紗衣,脚下涼鞋淨襪,頭上纓子瓦楞帽兒,金簪子。向月娘衆人作了揖,就拉過大姐,一處坐下。向月娘説:「徐家銀子討了來了。共五封,二百五十兩,送到房裏,玉簫收了。」於是傳盃換盞,酒過數巡,各添春色。月娘與李嬌兒桂姐三個下棋;玉樓、李瓶兒、孫雪娥、大姐、經濟,便向各處遊玩觀花草。惟有金蓮在山子後那芭蕉叢深處,將手中白紗團扇兒且去撲蝴蝶為戲。不防經濟驀地走在背後,猛然叫道:「五娘,你不會撲蝴蝶,我等與你撲!這蝴蝶就和你老人家一般,有些毬子心腸,滚上滚下的走滚大。」那金蓮扭回粉頸,斜睨秋波,對着陳經濟笑罵道:「你這少死的賊短命!誰要你撲?待人來聽見,敢待死也!我曉得你也不怕死了,搗了幾鍾酒兒,在這裏來鬼混!」因問:「你買的汗巾兒怎了?」那經濟笑嘻嘻,向袖子中取出,一手遞與他,説道:「六娘的都在這裏了。」又道:「汗巾兒捎了來,你把甚來謝我?」於是把臉子挨向他身邊,被金蓮只一推。不想李瓶兒抱着官哥兒,并奶子如意兒跟着,従松墻那邊走來,見金蓮和經濟兩個在那裏嬉戲,撲蝴蝶,李瓶兒忙叫道:「你兩個撲個蝴蝶兒與官哥兒耍子!」慌的經濟趕眼不見,兩三步就鑽進去山子裏邊。那潘金蓮恐怕李瓶兒瞧見,故意問道:「陳姐夫與了汗巾不曾?」李瓶兒道:「他還沒與我哩。」金蓮道:「他剛纔袖着,對着大姐姐不好與咱的,悄悄遞與我了。」於是兩個坐在花臺石上,打開兩個分了。

金蓮見官哥兒脖子裏圍着條白挑線汗巾子,手裏把着個李子往口裏吮,問道:「是你的汗巾子?」李瓶兒道:「是剛才他大媽媽,見他口裏吮李子,流下水,替他圍上這汗巾子。」兩個只顧坐在芭蕉叢下,李瓶兒説道:「這答兒裏到且是蔭涼,咱在這裏坐一囬兒罷!」因使如意兒:「你去叫迎春,屋裏取孩子的小枕頭兒帶涼蓆兒,放他在這裏躺躺兒。就取骨牌來,我和五娘在這裏抹回牌兒,你就在屋裏看罷。」如意兒去了。不一時,迎春取了枕蓆幷骨牌來。李瓶兒鋪下蓆,把官哥兒放在小枕頭兒上躺着,教他頑耍,他便和金蓮抹牌。抹了一囬,教迎春往屋裏炖一壺好茶來。不想孟玉樓在臥雲亭欄杆上看見,點手兒叫李瓶兒説:「大姐姐呌你説句話兒來。」那李瓶兒撇下孩子,教金蓮看着:「我就來!」那金蓮記掛經濟在洞兒裏,那裏又去顧那孩子?趕空兒兩三步走入洞門首呌經濟説:「没人,你出來罷!」經濟就叫婦人進去瞧蘑菇:「裏面長出這些大頭蘑菇來了。」哄的婦人入到洞裏,就折跌腿跪着,要和婦人雲雨。兩個正摟着親嘴。也是天假其便,李瓶兒走到亭子上,吳月娘説:「孟三姐和桂姐投壺輸了,你來替他投兩壺兒。」李瓶兒道:「底下沒人看孩子哩!」玉樓道:「左右有六姐在那裏,怕怎的?」月娘道:「孟三姐,你去替他看看罷!」李瓶兒道:「三娘,累你,一發抱了他來罷。」叫小玉:「你去,就抱他的蓆和小枕頭兒來。」那小玉和玉樓走到芭蕉叢下,孩子便躺在蓆上,登手登脚的怪哭,並不知金蓮在那裏。只見傍邊大黑貓,見人來,一滚煙跑了。玉樓道:「他五娘那裏去了?耶嚛!耶嚛!把孩子丢在這裏,吃貓唬了他了!」那金蓮便従傍邊雪洞兒裏鑽出來,説道:「我在這裏淨了淨手,誰往那裏去來?那裏有貓來唬了他,白眉赤眼兒的!」那玉樓也更不往洞裏看,只顧抱了官哥兒拍哄着他,往臥雲亭兒上去了。小玉拏着枕蓆跟的去了。金蓮恐怕他學舌,隨屁股也跟了來。月娘問:「孩子怎的哭?」玉樓道:「我去時,不知是那裏一個大黑貓,蹲在孩子頭跟前。」月娘説:「乾淨唬着孩兒!」李瓶兒道:「他五娘看着他哩。」玉樓道:「六姐往洞兒裏淨手去來。」金蓮走上來説玉樓:「你怎的恁白眉赤眼兒的,我在,那裏討個貓來?他想必餓了,要奶吃哭,就賴起人了!」李瓶兒見迎春拏上茶來,就使他叫奶子來喂哥兒奶。那陳經濟見無人,従洞兒鑽出來,順着松墻兒,抹轉過捲棚,一直行前邊角門往外去了。正是雙手劈開生死路,一身跳出是非門。

月娘見孩子不吃奶,只是哭,吩咐李瓶兒:「你抱他到屋裏,好好打發他睡罷。」於是也不吃酒,衆人都散了。原來陳經濟也不曾與潘金蓮得手,做不成燕侶鶯儔,只得做了個蜂頭花嘴兒,事情不巧。歸到前邊廂房中,有些咄咄不楽。正是:無可奈何花落去,似曾相識燕歸來。有〔折桂令〕為證:

\begin{myquote}
我見他斜戴花枝,笑撚花枝。朱唇上不抹胭脂,似抹胭脂;逐日相逢,似有情兒,未見情兒。欲見許,何曾見許?似推辭未是推辭!約在何時?會在何時?不相逢,他又相思;旣相逢,我反相思。
\end{myquote}

畢竟未知後來何如?且聽下囬分解。

