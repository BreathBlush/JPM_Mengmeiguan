\chapter*{夢梅館校本《金瓶梅詞話》前言}
\addcontentsline{toc}{chapter}{夢梅館校本《金瓶梅詞話》前言}
\markboth{夢梅館校本《金瓶梅詞話》前言}{夢梅館校本《金瓶梅詞話》前言}


《金瓶梅詞話》是中國著名的古典長篇白話小説,也是最具爭論性的小説。自誕生以來,貶之者詆為「市諢之極穢者」,「當急投秦火」{\innerzhushi(明薛岡·天爵堂筆餘)};讚之者譽為「偉大的寫實小説」{\innerzhushi(鄭振鐸·談金瓶梅詞話)},「同時説部,無以上之」{\innerzhushi(魯迅·中國小説史略)}。其實,除去書中一些不雅的性描寫,《金瓶梅詞話》無疑是中國文學寶庫中之奇珍,與《水滸傳》、《紅樓夢》屬同一水平的作品。《金瓶梅》接枝自《水滸》,《紅樓夢》脱胎於《金瓶》;《水滸傳》寫江湖,《金瓶梅》寫市井,《紅樓夢》寫上層貴族,均曲盡其妙。《金瓶梅詞話》的文學地位,亦居於二者之間。

《金瓶梅》的作者,在文學史上至今仍是個謎。欣欣子序雖有「蘭陵笑笑生作金瓶梅傳」的説法,真實姓名與生平事迹均語焉不詳,且亦不見早期抄閱者著錄。後人指為王世貞、李開先、賈三近、屠隆等等,皆缺乏可靠證據。從本書的題旨、内容、取材、叙述結構和語言特徵看,應屬大衆消費性通俗文學,以平話為主體,穿插演唱流行曲。其作者是書會才人一類中下層知識分子,可能與流傳久遠的「羅公(貫中)書會」有某種關係。現存今本詞話,應為民間説書藝人的底本,有的學者甚至認為是「耳錄」{\innerzhushi(傅憎享·金瓶梅隱語揭秘)}。

《金瓶梅》借北宋年號名色,刻劃明代人情世態,開創中國古典長篇小説寫實的傳統。小説講述在明季商品經濟新潮中,一位破落户市棍,欺壓良善,結交勢要,官商通吃,飛黄驣達,享受榮華富貴,最后縱欲身亡的故事。書中的清河,當是運河沿岸一個城鎮。據書中稱淮安、清江浦為「淮上」,稱揚州為「下州」,稱淮河為「南河」,生活場景接近南清河的淮揚地區。《金瓶梅詞話》最初大概就由打談的在淮安、揚州、臨清、濟寧等繁榮、富庶的運河大碼頭上説唱,後來也傳至運河南端的蘇州和杭州{\innerzhushi(明張岱·陶庵夢憶·不繫園)}。聽衆多為客商、船夫、手藝人和市民。

《金瓶梅詞話》借樹開花,從《水滸》武松打虎故事直接切入,開頭有五、六回文字摭自《水滸》,其成書上限不能早於現存百回本《忠義水滸傳》的定型和刊行。書中寫官哥、李瓶兒、西門慶之喪,用的十二個日、月干支,均隆慶五年八月到六年二月的干支;第六十八回提到的「南河南徙」,始於萬暦五年閠八月,此可視為詞話成書的上限{\innerzhushi(梅節·金瓶梅成書的上限、金瓶梅成書於萬曆的新材料)}。大概公元十六世紀末葉、萬曆二十年或稍后,詞話一些不足本已在文人圈子中傳抄。當時如王宇泰、董其昌、袁宏道、王百榖、文在茲、丘志充、謝肇淛、袁中道、沈德符等均有過錄本。在輾轉傳抄過程中,開始形成兩種本子,一為原民間藝人的十卷本,書名《金瓶梅詞話》,有欣欣子序。一為經文士改編的二十卷本,書名《金瓶梅》,有東吴弄珠客序和廿公跋。十卷本《金瓶梅詞話》雖更接近藝人原本,它的刊行却在文人改編的二十卷本《金瓶梅》之後。現存之《新鎸綉像批評金瓶梅》,是這個二十卷本的第二代刻本{\innerzhushi(梅節·新刻金瓶梅詞話後出考)}。

二十卷本在萬曆末、天啓初刊行後風行一時,書林人士見有利可圖,乃梓行十卷本《金瓶梅詞話》。因所據底本也缺五十三至五十七回,乃採二十卷本由陋儒續撰的「這五回」頂補,并錄入二十卷本之弄珠客序、廿公跋以作招徕。十卷本《詞話》因底本訛誤太甚,可讀性差,梓行後并未引起注意,社會上流行的依然是二十卷本,包括後來有張竹坡評的第一奇書本。《詞話》在清初尚有人提及,以後即寂然無聞。一九三二年在山西發現的《新刻金瓶梅詞話》,已屬海内孤本。其後在日本倒發現兩部,一藏日光山輪王寺慈眼堂,一藏德山毛利栖息堂。另京都大學有殘本二十三回,完整者七回。日本兩部《詞話》除栖息堂本第五回末頁曾换版外,與中土本同為一刻。

東土本保持《新刻金瓶梅詞話》素潔的原貌;中土本則有後人的朱筆校改與批語,全書朱墨燦然。一九三三年,馬廉以「古佚小説刊行會」名義,醵資將中土本影印一百零四套{\innerzhushi(魯歌·簡説金瓶梅的幾種版本)}。因錢不够,未能兩色套印,只用單色,朱改變墨改,效果極差。以後中國大陸即據此影印本複印和排印,造成失真。一九六三年,日本大安株式會社據彼邦兩部補配,影印配本《金瓶梅詞話》,稱「日本大安本」。一九七八年,臺北聨經出版事業有限公司按照原藏北京圖書館、現存臺北故宫博物院中土本之原尺寸大小出版朱墨二色套印本。但聨經用作底本的不是原刻,而是傅斯年所藏的古佚小説本,複印經「整理後影印」{\innerzhushi(聨經本金瓶梅詞話·出版説明)},雖糾正古佚小説本據原刻本上朱批改正文的錯誤,墨改部分却未改正(可能没有發現),因此并不徹底。

《金瓶梅》過去被目為淫書,因為它自然主義的反映主人翁西門慶的淫行。從書中一些色情描寫的間歇地重覆出現看,顯然是編撰者為吸引下層聽衆所加添的有味作料。如所周知,《金瓶梅》誕生的時代,是淫風熾盛的明季。遠的不説,萬曆以前幾個皇帝,朱厚照(正德)是「屬皮匠的,縫著的就上」,最後死於豹房。朱厚熜(嘉靖)求丹方,講采補,服紅丸,淫幼女,陶仲文即以方術膺三孤、封恭誠伯。朱戴垕(隆慶)積年服春藥,弄至虛陽舉發,「晝夜不仆」,無法視朝。當時抗倭名將譚綸,一代名相张居正,均以服丹方御女致斃{\innerzhushi(明沈德符·萬曆野獲編·佞倖)}。上行下效,民間則流行淫器、淫藥,浮蕩子弟相率「養龜」。據明末清初佚名作者《如夢錄》記載,明季開封有七家性商店(淫店),都開在撫按諸署附近,專售「廣(景)東人事」、「房中技術」,能「助老扶幼」,「走馬、乌鬚」。在這樣淫靡的社會風氣下,像《金瓶梅詞話》這種大衆消費性通俗文學,為迎合聽衆口味,穿插一些葷笑話、性故事,實不足為怪。何况《金瓶梅》與其他專鋪叙床笫的淫書根本不同,它的一些性描寫只是作為書中人物感官生活的一部分,雖有誇張成分,仍属人欲的範圍。也正因為如此,通過《詞話》的暴露,人们可看到在過去那個時代的男人的性心理,看到女性的卑下和屈辱,也看到人性不那麽可愛的一面。如果把這些文字删去,作品將是残缺的,而且更會引起讀者的好奇,效果適得其反。我们提倡青年人可以先看如《三國演義》、《水滸傳》、《西游記》、《紅樓夢》這些古典文學作品,待心智更成熟,再看《金瓶梅》,也許較為合適。

筆者從八十年代中從事詞話的整理和校點,旨在為讀者提供一個可讀的、較少錯誤的、接近原著的本子。選擇以日本大安本为底本,覆以北京圖書館藏中土本,校以日本内閣文庫和北京大學之《新刻綉像批評金瓶梅》、在茲堂本和崇經堂本之《皋鶴堂批評第一奇書金瓶梅》、并先後参考鄭振鐸、施蟄存、劉本棟、增你智、戴鴻森、白維國卜鍵諸本,兼吸收姚靈犀、魏子雲、陳詔、李申、張惠英、張鴻魁、傅憎享、魯歌等專家研究成果,進行校訂。一九八七年完成第一次校訂,出版《全校本金瓶梅詞話》;一九九三年完成第二次校訂,出版《重校本金瓶梅詞話》一九九九年完成第三次校訂,出版影印《夢梅館定本金瓶梅詞話》手抄本。詳細情况可参閱拙作《金瓶梅詞話的版本與文本——金瓶梅詞話校讀記代序》,这裏不再重覆。

本書在校點过程中,曾得到许多學者和專家的幫助。王修齡、許桂林、馮统一諸先生曾参加全校本校點工作。王先生初校了三十五回至七十回,許先生協助校點了後二十回及書中有關星相部分;馮先生協助校點書中詞曲。陳詔、黄霖兩位先生為重校本作了簡明註釋;蔡敦勇先生對詞曲進行了覆校。陳少卿先生花了三年時間,工楷抄閱三校定本,協助出版《夢梅館定本金瓶梅詞話》手抄影印本。我在此謹向他們,以及對本書的校點出版給予支持和鼓勵的海内外師友,表示由衷的感謝。自己深知,由於寡學淺識,本書在校勘與整理方面都不免存在許多錯誤和不足之處,這只好留待後人來糾正及完善了。

\begin{quotation}
\begin{flushright}

\nopagebreak 前言脱稿於丙寅除夜,刊於全校本與\\重校本。
癸未暮春改定於青衣夢梅館

\end{flushright}
\end{quotation}

\setcounter{footnote}{0}

