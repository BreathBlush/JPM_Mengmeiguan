\includepdf[pages={175,176},fitpaper=false]{tst.pdf}
\chapter*{第八十八囬 \\潘金蓮託夢守禦府 吳月娘布施募緣僧}
\addcontentsline{toc}{chapter}{第八十八囬 潘金蓮託夢守禦府 吳月娘布施募緣僧}
\markboth{{\titlename}卷之九}{第八十八囬 潘金蓮託夢守禦府 吳月娘布施募緣僧}


\begin{myquote}
上臨之以天鑒,下察之以地祇;

明有王法相制,暗有鬼神相隨。

忠直可存於心,喜怒戒之在氣;

為不節而亡家,因不廉而失位。

勸君自警平生,可嘆可驚可畏!
\end{myquote}

話說武松殺了婦人王婆,劫去財物,逃上梁山為盗去了。卻表王潮兒去街上呌保甲,見武松家前後門都不開,又王婆家被劫去財物,房中衣服丢的地下横三豎四,就知是武松殺死二命,劫取財物而去。未免打開前後門,見血瀝瀝兩個死屍倒在地下,婦人心肝五臟,用刀插在後樓房簷下。迎兒倒扣在房中,問其故,只是哭泣。次日早衙,呈報到本縣,殺人兇刄都拿放在面前。本縣新任知縣也姓李,雙名昌期,乃河北眞定府棗強縣人氏,聽見殺人公事,即委差當該吏典,拘集兩隣保甲,並兩家苦主王潮迎兒,眼同擡出當街,如法檢驗。生前委被武松因忿带酒,殺潘氏王婆二命,疊成文案,就委地方保甲瘞埋看守。掛出榜文,四廂差人跟尋,訪拿正犯武松,有人首告者,官給賞銀五十兩。

守備府中張勝李安打着一百兩銀子到王婆家,看見王婆婦人俱已被武松殺死,縣中差人檢屍,捉拿兇犯。二人囘報到府中。春梅聽見婦人死了,整哭了兩三日,茶飯都不喫。慌了守備,使人門前呌了調百戲的貨郎兒進去,耍與他觀看,只是不喜歡。日逐使張勝李安打聽,拿住武松正犯,告報府中知道,不在話下。

按下一頭,卻表陳經濟前往東京取銀子,一心要贖金蓮,成其夫婦。不想走到半路,撞見家人陳定,従東京來告說家爺病重之事:「奶奶使我來請大叔往家去,囑託後事。」這經濟一聞其言,兩程做一程,路上趲行。有日到東京他姑夫張世廉家,——張世廉已死,止有姑娘見在。——他父親陳洪已是沒了三日光景,滿家帶孝。經濟參見他父親靈座,與他母親張氏並姑娘磕頭。張氏見他長成人,母子哭做一處,通同商議,張氏道:「如今一則以喜,一則以憂。」經濟便道:「如何是喜如何是憂?」張氏道:「喜者,如今且喜朝廷册立東宫,郊天大赦;憂則不想你爹爹得病,死在這裏。你姑夫又沒了,姑娘守寡,這裏住着不是常法,方使陳定呌將你來,和你打發你爹爹靈柩囘去,葬埋鄉井,也是好處。」這經濟聽了,心内暗道:「這一會發送,裝載靈柩家小粗重上車,少說也得許多日期躭閣,卻不悞了娶六姐?不如如此這般,先誆了兩車細軟箱籠家去,待娶了六姐,再來搬取靈柩不遲。」一面對張氏說道:「如今隨路盗賊,十分難走。假如靈柩家小箱籠一同起身,少說數輛車馱,未免起眼,倘遇小人囉唣怎了?寜可躭遲不躭錯。我先押兩車細軟箱籠家去,收拾房屋;母親随後和陳定家眷,跟父親靈柩,過年正月間起身囘家,寄在城外寺院,然後做齋念經,入坟安葬,也是不遲。」張氏終是婦人家,不合一時聽信經濟巧言念轉,先打點細軟箱籠,裝載兩大車,上插旗號,扮做香車,従臘月初一日東京起身。

不上數日,到了山東清河縣家門首。對他母舅張團練說:「父親已死,母親押靈車不久就到。我押了兩車行李,先來收拾,打掃房屋。」他母舅聽說,「旣然如此,我須搬囘家便了。」一面就令家人搬家活騰出房子來。這經濟見母舅搬去,滿心歡喜説:「且得寃家離眼前,落得我娶六姐來家,自在受用。我父親已死,我娘又疼我,先休了那個淫婦,然後一紙狀子,把俺丈母告到官,追要我寄放東西,誰敢道個不字?又挾制俺家充軍人數不成?」正是:人算如此如此,天理不然不然。

這經濟早攛掇他母舅出來,然後打了一佰兩銀子在腰裏,另外又袖着十兩謝王婆,來到紫石街王婆門首。可霎作怪,只見門前街旁埋着兩個尸首,兩桿鎗交叉上面挑着個燈籠,門首掛着一張手榜,上書:「本縣為人命事,兇犯武松,殺死潘氏王婆二命,有人捕獲首告官司者,官給賞銀五十兩。」這經濟仰頭還大看看,只見従窝舖中鑽出兩個人來,喝聲道:「甚麽人?看此榜文做甚?現今正身兇犯捉拿不着,你是何人?」大扠步便來捉獲。這經濟慌的奔走不迭。恰纔走到石橋下酒樓邊,只見一個人,頭戴萬字巾,身穿青衲襖,隨後赶到橋下,說道:「哥哥,你好大膽,平白在此看他怎的?」這經濟扭囘頭看時,卻是一個識熟朋友鐵指甲楊大郎。二人聲喏,楊大道:「哥哥,一向不見,那裏去來?」經濟便把東京父死往囘之事告説一遍:「卻纔這殺死婦人,是我丈人的小潘氏。不知他被人殺了,適纔見了榜文,方知其故。」楊大郎告道:「是他小叔武松,充配在外,遇赦囘還,不知因甚殺了婦人,連王婆子也不饒。他家還有個女孩兒,在我姑夫姚二郎家養活了三四年。昨日他叔叔殺了人,走的不知下落,我姑夫將此女縣中領出,嫁與人為妻小去了。現今這兩瘞屍首,日久只顧埋着,只是苦了地方保甲看守,更不知何年月日纔拿住兇犯武松。」説畢,楊大郎招了經濟上酒樓飲酒:「與哥哥拂塵。」

這經濟見婦人已死,心中轉痛不暇,那裏喫得下酒?約莫飲夠三盃,就起身下樓,作别來家。到晚夕,買了一陌錢紙,在紫石街離王婆門首遠遠的石橋邊,題着婦人:「潘六姐,我小兄弟陳經濟,今日替你燒陌錢紙。皆因我來遲了一步,悞了你性命!你活時為人,死後為神。早保佑捉獲住仇人武松,替你報仇雪恨!我在法場上看着剮他,方趁我平生之志!」說畢哭泣,燒化了錢紙。經濟囬家,關了門戶,走歸房中,恰纔睡着,似睡非睡,夢見金蓮身穿素服,一身帶血,向經濟哭道:「我的哥哥,我死的好苦也!實指望與你相處在一處,不期等你不來,被武松那廝害了性命。如今陰司不收,我白日遊遊蕩蕩,夜向各處尋討漿水。適間蒙你送了一陌錢紙與我,但只是仇人未獲,我的屍首埋在當街。你可念舊日之情,買具棺材盛了葬埋,免得日久暴露。」經濟哭道:「我的姐姐,我可知要葬埋你,但恐西門慶家中我丈母那無仁義的淫婦知道,他自恁賴我,倒趂了他機會。姐姐,你須往守備府中對春梅説知,敎他葬埋你身屍便了。」婦人道:「剛纔奴到守備府中,又被那門神戶尉攔攩不放,奴須慢慢再哀告他則個。」經濟哭着,還要拉着他説話,被他身上一陣血腥氣□□,撒手掙脱,卻是南柯一夢。枕上聽那更鼓時,正打三更二點,説道:「怪哉!我剛纔分明夢見六姐向我訴告衷腸,教我葬埋之意,又不知甚年月日拿住武松,是好傷感人也!」正是:夢中無限傷心事,獨坐空房哭到明。

不說經濟這裏也打聽武松不題。卻説縣中訪拿武松,約兩個月有餘,捕獲不着,已知逃遁梁山為盗。地方保甲隣佑,呈報到官,所瘗兩座屍首,相應責令家屬領埋。王婆屍首,便有他兒子王潮領的埋葬,止有婦人身屍無人來領。

卻説府中春梅,兩三日一遍,使張勝李安來縣中打聽,囬去只說:「兇犯還未拿住。屍首照舊埋瘗,地方看守,無人敢動。」直挨過年,正月初旬時節,忽一日晚間,春梅作一夢,恍恍惚惚,夢見金蓮雲髻蓬鬆,渾身是血,呌道:「龐大姐,我的好姐姐,奴死的好苦也!好容易來見你一面,又被門神把住嗔喝,不敢進來。今仇人武松已是逃走脱了,所瘗奴的屍首,在街暴露日久,風吹雨洒,鷄犬作踐,無人領埋。奴擧目無親,你若念舊日母子之情,買具棺木把奴埋在一個去處,奴死在陰司口眼皆閉!」説畢,大哭不止。春梅扯住他,還要再問他别的話,被他掙開,撒手驚覺,卻是南柯一夢。従睡夢中直哭醒來,心内猶疑不定。

次日,呌進張勝李安吩咐:「你二人去縣前打聽,那埋的婦人婆子屍首,還有無有?」張勝李安應諾去了。不多時,走來囬報:「正犯兇身已逃走脱了,所瘗殺死身屍,地方看守日久不便,相應責令各人家屬領埋。那婆子屍首,他兒子招領的去了;還有那婦人,無人來領,還埋在街心。」春梅道:「既然如此,我有樁事兒累你二人,替我幹得來,我還重賞你。」二人跪下道:「小夫人說那裏話!若肯在老爺前擡擧小人一二,自消受不了。雖赴湯跳火,敢說不去?」春梅走到房中,拿出十兩銀子、兩疋大布,委付二人:「這死的婦人,是我一個嫡親姐姐,嫁在西門慶家,今日出來,被人殺死。你二人休教你老爺知道,拿這銀子替我買一具棺材,把他裝殮了,擡出城外,擇方便地方,埋葬停當,我還重賞你!」二人道:「這個不打緊,小人就去。」李安説:「只怕縣中不敎你我領屍,怎了?須拿老爺個帖兒,下到縣官纔好。」張勝道:「只說小夫人是他妹子,嫁在府中,那縣官不敢不依,何消帖子!」於是領了銀子,來到班房内。張勝便向李安説:「想必這死的婦人,與小夫人曾在西門慶家做一處,相結的好,今日方這等為他費心。想着死了時,整哭了三四日,不喫飯,直敎老爺門前呌了調百戲貨郎兒,調與他觀看,還不喜歡。今日他無親人領去,小夫人豈肯不葬埋他?咱們若替他幹得此事停當,早晚他在老爺跟前,自方便你我,就是一點福星。現今老爺百依百隨,聽他說話,正經大奶奶二奶奶且打靠後。」

説畢,二人拿銀子到縣前,遞了領狀,就說他妹子在老爺府中,來領屍首。使了六兩銀子,合了一具棺木。把婦人屍首掘出,把心肝塡在肚内,頭用線縫上,用布裝殮停當,裝入材内。張勝說:「就埋在老爺香火院城南永福寺裏,那裏有空閒地。葬埋了,囬小夫人話去。」呌了兩名伴當,擡到永福寺,對長老説:「這是宅内小夫人親姐姐,要一塊地兒葬埋。」長老不敢怠慢,就在寺後揀一塊空心白楊樹下,那裏葬埋已畢。走來宅内囬春梅話説:「除買棺材裝殮,還剩四兩銀子。」交割明白。春梅吩咐:「多有起動你二人,將這四兩銀子,拿二兩與長老道堅,敎他早晚替他念些經懺,超度他生天。」又拿出一大瓶酒、一腿猪肉、一腿羊肉,「這二兩銀子,你每人將一兩家中盤纏。」二人跪下,那裏敢接,只說:「小夫人若肯在老爺面前擡擧,小人消受不了!這些小勞,豈敢接受銀兩?」春梅道:「我賞你,不收,我就惱了。」二人只得磕頭領了出來。兩個班房喫酒,甚是稱念小夫人好處。次日,張勝送銀子與長老念經,春梅又與五錢銀子,買紙與金蓮燒,俱不在話下。

卻説陳定従東京載靈柩、家眷,到清河縣城外,把靈柩寄在永福寺,待的念經發送,歸葬坟内。經濟在家聽見母親張氏家小車輛到了,父親靈柩寄停在城外永福寺,收卸行李已畢,與張氏磕了頭。張氏怪他:「就不去接我一接!」經濟只說:「心中不快,家裏無人看守。」張氏便問:「你舅舅怎的不見?」經濟道:「他見母親到了,連忙搬囘家去了。」張氏道:「且敎你舅舅住着,慌搬去怎的?」一面他母舅張團練來看他姐姐,姊娣抱頭而哭,置酒叙話,不必細説。

次日,他娘張氏,早使經濟拿五兩銀子,幾陌金銀錢紙,往門外與長老,替他父親念經。正騎頭口街上走,忽撞遇他兩個朋友,陸太郎、楊大郎,下頭口聲喏。二人問道:「哥哥往那裏去?」經濟悉言:「先父靈柩寄在門外寺裏,明日廿日是終七,家母使我送銀子與長老,做齋念經。」二人道:「兄弟不知老伯靈柩到了,有失弔問。」因問:「幾時發引安葬?」經濟道:「也只在一二日之間,念畢經,入坟安葬。」說罷,二人擧手作别。這經濟又呌住,因問楊大郎:「縣前我丈人的小,那潘氏屍首怎不見,被甚人領的去了?」楊大郎便道:「半月前,地方因捉不着武松,禀了本縣相公,令各家領去葬埋。王婆是他兒子領去,止有婦人屍首,丢了三四日,被守備府中買了一口棺木,差人擡出城外永福寺那裏葬去了。」經濟聽了,就知是春梅在府中收葬了他屍首,因問大郎:「城外有幾個永福寺?」大郎道:「本自南門外只一個永福寺,是周秀老爺香火院。那裏有幾個永福寺來?」經濟聽了暗喜:「就是這個永福寺!也是緣法凑巧,喜得六姐亦葬在此處。」一面作别二人,打頭口出城,逕到永福寺中。見了長老,且不說念經之事,就先問長老道堅:「此處有守備府中新迁葬的一個婦人在那裏?」長老道:「就在寺後白楊樹下,說是宅内小夫人的姐姐。」這陳經濟且不參見他父親靈柩,先拿錢紙祭物,到於金蓮墓上,與他祭了,燒化錢紙,哭道:「我的六姐,你兄弟陳經濟敬來與你燒一陌錢紙:你好處安身,苦處用錢。」祭畢,然後纔到方丈内,他父親靈柩跟前,燒紙祭祀。遞與長老經錢,敎他二十日請八衆禪僧,念断七經。長老接了經襯,備辦齋供。經濟來家,囘了張氏話。二十日都去寺中撚香,擇吉發引,把父親靈柩歸到祖塋。安葬已畢來家,母子過日,不題。

卻表吳月娘,一日二月初旬,天氣融和,孟玉樓、孫雪娥、西門大姐、小玉,出來大門首站立,觀看來往車馬,人煙熱鬧。忽見一簇男女,跟着個和尚,生的十分胖大。頭頂三尊銅佛,身上抅着數枝燈樹,杏黄袈娑風兜袖,赤脚行來泥没踝。自言說是五臺山戒壇上下來的行脚僧,雲遊到此,要化錢糧,蓋造佛殿。當時古人有幾句讚的這行脚僧好處:

\begin{myquote}
打坐參禪,講經説法。鋪眉苫眼,習成佛祖家風;賴敎求食,立起法門規矩。白日裏賣杖搖鈴,黑夜間舞鎗弄棒。有時門首磕光頭,餓了街前打響嘴。空色色空,誰見衆生離下土;去來來去,何曾接引到西方!
\end{myquote}

那和尚見月娘衆婦女在門首,向前道了個問訊,説道:「在家老菩薩施主,旣生在深宅大院,都是龍華一會上人。貧僧是五臺山下來的,結化善緣,蓋造十王功德三寳佛殿。仰賴十方施主菩薩,廣種福田,捨資財共成勝事,修來生功果。貧僧只是挑脚漢。」月娘聽了他這般言語,便喚小玉往房中取一頂僧帽、一雙僧鞋、一弔銅錢、一斗白米。原來月娘平昔好齋僧布施,常時閒中發心做下僧帽、僧鞋,預備布施。這小玉取出來,月娘吩咐:「你呌那師父近前來,布施與他!」這小玉故做嬌態,高聲呌道:「那秃驢的和尚還不過來,俺奶奶布施與你這許多東西,還不磕頭哩!」月娘便罵道:「怪墮業的小臭肉兒,一個僧家,是佛家弟子,你有要沒緊恁謗他怎的?不當家化化的!你這小淫婦兒,到明日不知墮多少罪業。」小玉笑道:「奶奶,這賊和尚我呌他,他怎的把那一雙賊眼眼上眼下打量我?」那和尚雙手接了鞋帽錢米,打問訊說道:「多謝施主老菩薩布施布施!」小玉道:「這秃廝好無禮,這些人站着,只打兩個問訊兒,就不與我打一個兒?」月娘道:「小肉兒,還恁説白道黑,他一個佛家之子,你也消受不的他這個問訊!」小玉道:「奶奶,他是佛爺兒子,誰是佛爺女兒?」月娘道:「像這比丘尼姑僧,是佛的女兒。」小玉道:「譬若說,像薛姑子王姑子、大師父,都是佛爺女兒。誰是佛爺女婿?」月娘忍不住笑,罵道:「這賊小淫婦兒,學的油嘴滑舌,見見就說下道兒去了!」小玉道:「奶奶只罵我,本等這秃和尚賊眉豎眼的只看我。」孟玉樓道:「他看你,想必認得你,要度脱你去。」小玉道:「他若度我,我就去。」說着,衆婦女笑了一囘。月娘喝道:「你這小淫婦兒,專一毀僧謗佛!」那和尚得了布施,頂着三尊佛,揚長去了。小玉道:「奶奶還嗔我罵他,你看這賊秃,臨去還看了我一眼,纔去了。」有詩單道月娘修善施僧好處:

\begin{myquote}
守寡看經歲月深,私邪空色久違心。

奴身好似天邊月,不許浮雲半點侵。
\end{myquote}

月娘衆人正在門首説話,忽見薛嫂兒提着花箱兒,従街上過來,見月娘衆人,道了萬福。月娘問:「你往那裏去來?怎的影跡兒不來我這裏走走?」薛嫂兒道:「不知我終日窮忙的是些甚麽!這兩日,大街上掌刑張二老爹家,與他兒子娶親,和北邊徐公公做親,娶了他侄兒,也是我和文嫂兒說的親事。昨日三日,擺大酒席。忙的連守備府裏咱家小大姐那裏呌,我也沒去,不知怎麽惱我哩!」月娘問道:「你如今往那裏去?」薛嫂道:「我有樁事,敬來和你老人家説來。」月娘道:「你有話進來說。」一面讓薛嫂兒到後邊上房裏坐下。喫了茶,薛嫂道:「你老人家還不知道,你陳親家従去年在東京得病沒了,親家母呌了姐夫去,搬取家小靈柩。従正月來家,已是念經發送墳上安葬畢。我只說你老人家這邊知道,怎不去燒張紙兒,探望探望?」月娘道:「你不來說,俺這裏怎得曉的?又無人打聽。倒只知道潘家的喫他小叔子殺了,和王婆子都埋在一處,卻不知如今怎樣了。」薛嫂兒道:「自古生有地兒死有處。五娘他老人家,不因那些事出去了,卻不好來?平日不守本分,幹出醜事來出去了!若在咱家裏,他小叔兒怎得殺了他?還是寃有頭債有主。倒還虧了咱家小大姐春梅,越不過娘兒們情腸,差人買了口棺材,領了他屍首葬埋了。不然,只顧暴露着,又拿不着小叔子,誰去管他?」孫雪娥在旁說:「春梅賣在守備府裏多少時兒,就這等大了?手裏拿出銀子替他買棺材埋葬,那守備也不嗔?當他甚麽人?」薛嫂道:「耶嚛,你還不知,守備好不喜他!每日只在他房裏歇臥,說一句依十句。一娶了他,生的好模樣兒,乖覺伶俐,就與他西廂房三間房住,撥了個使女伏侍他。老爺一連在他房裏歇了三夜,替他裁四季衣服。上頭三日,喫酒,賞了我一兩銀子,一疋緞子。他大奶奶五十歲,雙目不明,喫長齋,不管事。東廂孫二娘,生了小姐,雖故當家,撾着個孩子,如今大小庫房鑰匙倒都是他拿着,守備好不聽他說話哩!且説銀子,手裏拿不出來?」幾句説的月娘雪娥都不言了。

坐了一囘,薛嫂起身。月娘吩咐:「你明日來我這裏,備一張祭桌、一疋尺頭、一份冥紙,你來送大姐與他公公燒紙去。」薛嫂兒道:「你老人家不去?」月娘道:「你只説我心中不好,改日望親家去罷。」那薛嫂約定:「你敎大姐收拾下等着我,飯罷時候。」月娘道:「你如今到那裏去?守備府中不去也罷。」薛嫂道:「不去,就惹他怪死了。他使小伴當呌了我好幾遍了。」月娘道:「他呌你做甚麽?」薛嫂道:「奶奶你不知,他如今有了四五個月身孕了,老爺好不喜歡,呌了我去,一定賞我。」提着花箱作辭去了。雪娥便說:「老淫婦說的沒個行欵兒,他賣守備家多少時,就有了半肚孩子?那守備身邊少說也有幾房頭,莫不就興起他來,這等大時道!」月娘道:「他還有正經大奶奶,房裏還有一個生小姐的娘子兒哩!」雪娥道:「可又來!到底還是媒人嘴,一尺水十丈波的。」不因今日雪娥說話,正是従天降下鉤和線,就地引起是非來。有詩為證:

\begin{myquote}
曾記當年侍主傍,誰知今日變風光。

世間萬事皆前定,莫笑浮生空自忙。
\end{myquote}

畢竟未知後來如何,且聽下囘分解。

