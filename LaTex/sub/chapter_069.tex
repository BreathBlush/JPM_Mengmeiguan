\includepdf[pages={137,138},fitpaper=false]{tst.pdf}
\chapter*{第六十九囬 \\文嫂通情林太太 王三官中詐求奸}
\addcontentsline{toc}{chapter}{第六十九囬 文嫂通情林太太 王三官中詐求奸}
\markboth{{\titlename}卷之七}{第六十九囬 文嫂通情林太太 王三官中詐求奸}


\begin{myquote}
信手烹魚覓素音,神仙有路足登臨。

掃堦偶得任卿葉,彈月輕移司馬琴。

桑下肯期狄有意,懷中可犯柳無心。

黄昏悞入銷金帳,且犯羔兒獨自斟。
\end{myquote}

話説玳安同文嫂兒到家,平安説:「爹在對門房子裏。」進去禀報。西門慶正在書房中和溫秀才坐的,見玳安,隨即出來,小客位内坐下。玳安悉把尋文嫂兒一節説了:「小的呌了來,在外邊伺候着。」西門慶即令呌他進來。那文嫂悄悄掀開暖簾,進入裏面,向西門慶磕頭。西門慶道:「文嫂兒,許久不見你。」文嫂道:「小媳婦有。」西門慶道:「你如今搬在那裏住了?」文嫂道:「小媳婦因不幸為了場官司,把舊時那房兒棄了;如今搬在大南首王家巷住哩。」西門慶吩咐道:「起來説話。」那文嫂一面站立在傍邊,西門慶令左右都出去。那平安和畫童都躲在角門外伺候,只玳安兒影在簾兒外邊聽説話兒。西門慶因問:「你常在那幾家大人家走跳?」文嫂道:「就是大街王皇親家、守備府周爺家、喬皇親、張二老爹、夏老爹家,都相熟。」西門慶道:「你認的王招宣府裏不認的?」文嫂道:「是小媳婦定門主顧,太太和三娘常照顧小的花翠。」西門慶道:「你旣相熟,我有樁事兒央煩你,休要阻了我。」向袖中取出五兩一錠銀子與他,悄悄和他説:「如此這般,你卻怎的尋個路兒,把他太太吊在你那裏,我會他會兒。我還謝你!」那文嫂聽了,哈哈笑道:「是誰對爹説來?你老人家怎的曉得來?」西門慶道:「常言人的名兒,樹的影兒,我怎不得知道!」文嫂道:「若説起我這太太來,今年三十五歲,屬猪,端的上等婦人,百伶百俐,只好像三十歲的。他雖是幹這營生,好不幹的嚴密!就是往那裏去,坐大轎伴當跟着,喝着路走,逕路兒來,逕路兒去。三老爹在外為人做人,他怎在人家落脚?這個人説的訛了。倒只是他家裏深宅大院,一時三老爹不在,藏掖個兒去,人不知鬼不覺,倒還許說。若是小媳婦那裏,窄門窄戶,敢招惹這個事?説在頭上,就是爹賞的這銀子,小媳婦也不敢領去。寜可領了爹言語,對太太說就是了。」西門慶道:「你不收,還是推託,我就惱了。事成,我還另外賞幾個紬緞你穿。你不收,阻了我。」文嫂道:「愁你老人家沒也怎的!上人着眼覷,就是福星臨。」磕了個頭,把銀子接了,說道:「待小媳婦悄悄對太太說,來回你老人家。」西門慶道:「你當件事幹,我這裏等着。你來時只在這裏來就是了,我不使小廝去了。」文嫂道:「我知道。不在明日,只在後日,隨早隨晚,討了示下就來了。」一面走出來。玳安道:「文嫂,隨你罷了:我只要一兩銀子。也是我叫你一塲,你休要獨喫!」文嫂道:「猴孫兒,隔牆掠篩箕——還不知仰着合着哩!」於是出門,騎上驢子,他兒子籠着,一直去了。

西門慶和溫秀才坐了一囬。良久,夏提刑來,就到家待了茶,冠冕着,同往府裏羅同知名喚羅萬象那裏喫酒去了。直到掌燈以後纔來家。

且説文嫂兒㧱着西門慶與他五兩銀子,到家歡喜無盡,打發會茶人散了。至後晌時分,走到王宣府宅裏,見了林太太,道了萬福。林氏便道:「你怎的這兩日不來走走,看看我?」文嫂便把家中祈報會茶,趕臘月要往頂上進香一節,告訴林氏。林氏道:「你兒子去,你不去罷了。」文嫂兒道:「我如何得去?只教文ら兒帶進香去便了。」林氏道:「等臨期,我送些盤纏與你。」文嫂便道:「多謝太太布施。」説畢,林氏呌他近前烤火,丫鬟㧱茶來喫了。這文嫂一面喫了茶,問道:「三爹不在家了?」林氏道:「他有兩夜没回家,只在裏邊歇哩。逐日搭着這夥喬人,只眠花臥柳,把花枝般媳婦兒丢在房裏通不顧,如何是好!」文嫂又問:「三娘怎的不見?」林氏道:「他還在房裏未出來哩。」這文嫂見無人,便説道:「不打緊,太太寬心。小媳婦有個門路兒,管就打散了這干人,三爹收心,也再不進院去了。太太容小媳婦,便敢說;不容,定不敢說。」林氏道:「你説的話兒,那遭兒我不依你來?你有話只顧説不妨。」這文嫂方説道:「縣門前西門大老爹,如今現在提刑院做掌刑千戶,家中放官吏債,開四五處舖面:緞子舖、生薬舖、紬絹舖、絨線舖,外邊江湖又走標船,揚州興販鹽引,東平府上納香蠟;夥計主管約有數十。東京蔡太師是他乾爺,朱太尉是他衛主,翟管家是他親家。巡撫巡按都與他相交,知府知縣是不消說。家中田連阡陌,米爛陳倉;赤的是金,白的是銀,圓的是珠,光的是寳。身邊除了大娘子,——乃是清河左衛吳千戶之女,塡房與他為継室。——只成房頭,穿袍兒的也有五六個,以下歌兒舞女、得寵侍妾,不下數十。端的朝朝寒食,夜夜元宵。今老爹不上三十四五年紀,正是當年漢子,大身材,一表人物;也曾喫薬養龜,慣調風情;雙陸象棋,無所不通;蹴踘打毬,無所不曉;諸子百家,拆白道字,眼見就會。端的擊玉敲金,百伶百俐。聞知咱家乃世代簪纓人家,根基非淺,又三爹在武學肆業,也要來相交,只是不曾會過,不好來的。昨日聞知太太貴誕在邇,又四海納賢,也一心要來與太太拜壽。小媳婦便道,初會怎好驟然請見的?待小的達知老太太,討個示下,來請老爹相見。今老太太不但結識他來往相交,又央浼他把這干人断開,不使那行人打攪,這須玷辱不了咱家門戶。」看官聽説:水性下流,最是女婦人。當日林氏被文嫂這篇話,説的心中迷留摸亂,情竇已開。便笑向文嫂兒計較道:「人生面不熟,怎生好遽然相見的?」文嫂道:「不打緊。等我對老爹説,只説太太先央浼老爹,要在提刑院遞狀,告引誘三爹這起人,預先私請老爹來,私下先會一會。此計有何不可?」說得林氏心中大喜,約定後日晚夕等候。

這文嫂討了婦人示下歸家,到次日飯時前後,走來西門慶宅内。那日西門慶従衙門回來,家中無事,正在對門房子裏書院内坐的。忽有玳安來報:「文嫂來了。」西門慶聽了,即出小客位内坐,令左右放下簾兒。良久,文嫂進入裏面,磕了頭。玳安知局,就走出來了,敎二人自在説話。這文嫂便把怎的説念林氏,誇獎老爹人品家道,怎樣行時,結識官府,又怎的仗義疎財,風流博浪:「說得他千肯萬肯,約定明日晚間三爹不在家,家中設席等候。假以說人情為由,暗中相會。」西門慶聽了,滿心歡喜,又令玳安㧱了兩疋紬緞賞他。文嫂道:「爹明日要去,休要早了。直到掌燈以後,街上人靜了時,打他後門首扁食巷中——他後門傍有個住房的段媽媽,我在他家等着爹。只使大官兒彈門,我就出來引爹入港。休令左近人知道。」西門慶道:「我知道,你明日先去,不可離寸地,我也依期而至。」說畢,文嫂拜辭而去。又回林氏話去了。

西門慶那日歸李嬌兒房中宿歇,一宿無話。巴不到次日,培養着精神。午間,戴着白忠靖巾,便同應伯爵騎馬往謝希大家喫生日酒。叫了兩個唱的。西門慶喫了幾盃酒,約掌燈上來,就逃席走出來了。騎上馬,玳安琴童兩個小廝跟隨。那時約十九日,月色朦朧,帶着眼紗,由大街抹過,逕穿到扁食巷王招宣府後門來。那時纔上燈以後,街上人初靜之候。西門慶離他後門半舍遠把馬勒住,令玳安先彈段媽媽家門。原來這媽媽就住着王招宣府家後房,也是文嫂舉荐,早晚看守後門,開門閉戶,但有入港,在他家落脚做眼。文嫂在他屋裏聽見外邊彈門,連忙開了門。見西門慶來了,一面在後門裏等的西門慶下了馬,帶着眼紗兒引進來;吩咐琴童牽了馬,往對門人家西首房簷下那裏等候;玳安便在段媽媽屋裏存身。

這文嫂一面請西門慶入來,便把後門関了,上了栓。由夾道進内,轉過一層羣房,就是太太住的五間正房,傍邊一座便門閉着。這文嫂輕輕敲了門環兒,原來有個聽頭兒。少頃,見一丫鬟出來開了雙扉,文嫂導引西門慶到後堂,掀開簾櫳而入。只見裏面燈燭熒煌,正面供養着他祖爺太原節度邠陽郡王王景崇的影身圖,穿着大紅團龍蟒衣玉帶,虎皮校椅,坐着觀看兵書,有若関王之像,只是髯鬚短些;傍邊列着鎗刀弓矢。迎門硃紅匾上書「節義堂」三字。兩壁書畫丹青,琴書潇灑。左右泥金隸書一聯:「傳家節操同松竹,報國勳功並斗山。」西門慶正觀看之間,只聽得門簾上鈴兒響,文嫂従裏㧱出一盞茶來與西門慶喫。西門慶便道:「請老太太出來拜見。」文嫂道:「請老爹且喫過茶着;剛才稟過,太太知道了。」不想林氏悄悄従房門簾裏望外觀看,見西門慶身材凛凛,語話非俗,一表人物,軒昂出衆;頭戴白緞忠靖冠,貂鼠暖耳,身穿紫羊絨鶴氅,脚下粉底皂靴,上面綠剪絨獅坐馬,一溜五道金鈕子,就是個富而多詐奸邪輩,壓善欺良酒色徒。一見滿心歡喜,因悄悄呌過文嫂來,問:「他戴的孝是誰的?」文嫂道:「是他第六個娘子的孝。新近九月間没了不多些時。饒少數,家中如今還有一巴掌數兒。他老人家你看不出來,出籠兒的鵪鶉——也是個快鬬的。」這婆娘聽了,越發歡喜無盡。文嫂催逼他出去見他一見兒。婦人道:「我羞答答怎好出去?請他進來見罷。」文嫂一面走出來,向西門慶說:「太太請老爹房内拜見哩。」於是忙掀門簾,西門慶進入房中。但見簾幙垂紅,地平上毡毹匝地,麝蘭香靄,氣暖如春。綉榻則斗帳雲横,錦屏則軒轅月映。婦人頭上戴着金絲翠葉冠兒,身穿白綾寬袖襖兒,沉香色遍地金粧花緞子鶴氅,大紅宫錦寬襴裙子,老鴉白綾高底扣花鞋兒,就是個綺閣中好色的嬌娘,深閨内㒲𣭈的菩薩。有詩為證:

\begin{myquote}
面膩雲濃眉又彎,蓮步輕移實匪凡。

醉後情深歸帳内,始知太太不尋常!
\end{myquote}

這西門慶一見,躬身施禮,説道:「請太太轉上,學生拜見。」林氏道:「大人免禮罷。」西門慶不肯,就側身磕下頭去,拜兩拜。婦人亦叙禮相還。拜畢,西門慶正面椅子上坐了,林氏就在下邊梳背炕沿斜僉相陪坐的。文嫂又早把前邊儀門閉上了,再無一個僕人在後邊。三公子那邊角門也関了。一個小丫鬟名喚芙蓉,紅漆丹盤㧱茶上來。林氏陪西門慶喫了茶,丫鬟接下盞託去。文嫂就在傍開言説道:「太太久聞老爹在衙門中執掌刑名,敢使小媳婦請老爹來,央煩樁事兒,未知老爹可依允不依?」西門慶道:「不知老太太有甚事吩咐?」林氏道:「不瞞大人説,寒家雖世代做了這招宣,夫主去世年久,家中無甚積蓄。小兒年幼,優養未曾考襲。如今雖入武學肆業,年幼失學家中,有幾個奸詐不級的人,日逐引誘他在外嫖酒,把家事都失了。幾次欲待要往公門訴狀,爭奈妾身未曾出閨門,誠恐抛頭露面,有失先夫名節。今日敢請大人至寒家訴其衷曲,就如同遞狀一般;望乞大人千萬留情,把這干人怎生處断開了,使小兒改過自新,專習功名,以承先業,實出大人再造之恩,妾身感激不淺,自當重謝。」西門慶道:「老太太怎生這般説,乃言『謝』之一字?尊家乃世代簪纓,先朝將相,何等人家!令郎已入武學,正當努力功名,承其祖武。不意聽信遊食所哄,留連花酒,實出少年所為。太太旣吩咐,學生到衙門裏即時把這干人處分懲治,無損令郎分毫,亦可戒諭令郎,再不可蹈此故轍,庶可杜絶將來。」這婦人聽了,連忙起身向西門慶道個萬福説道:「容日妾身致謝大人。」西門慶道:「你我一家,何出此言!」説話之間,彼此言來語去,眉目顧盼留情。

不一時,文嫂放桌兒,擺上酒來。西門慶故意辭道:「學生初來進謁,倒不曾具禮來,如何反承老太太盛情留坐?」林氏道:「不知大人下降,沒作准備。寒天聊具一盃水酒,表意而已。」丫鬟篩上酒來,端的金壺斟美釀,玉盞泛羊羔。林氏起身捧酒,西門慶亦下席説道:「我當先奉老太太一盃。」文嫂兒在傍插口説道:「老爹你且不消遞太太酒,這十一月十五日是太太生日,那日送禮來與太太祝壽就是了。」西門慶道:「阿呀,早是你説!今日初九日,差六日,我在下一定來與太太登堂拜壽。」林氏笑道:「豈敢動勞大人厚意!」須臾,大盤大碗,就是十六碗熱騰騰美味佳餚,熬爛下飯,煎ぱ鷄魚,烹炮鵝鴨,細巧菜蔬,新奇菓品。傍邊絳燭高燒,下邊金爐添火。交盃換盞,行令猜枚,笑雨嘲雲,酒為色膽。看看飲至蓮漏已沉,窻月倒影之際,一雙竹葉穿心,兩個芳情已動。文嫂已過一邊,連次呼酒不至。西門慶見左右無人,漸漸促席而坐,言頗涉邪,把手揑腕之際,挨肩擦膀之間,初時戲摟粉項,婦人則笑而不言;次後款啟朱唇,西門慶則舌吐其口,嗚咂有聲,笑語密切。婦人於是自掩房門,解衣鬆珮,微開錦帳,輕展綉衾,鴛枕横床,鳳香薰被,相挨玉體,抱摟酥胸。原來西門慶知婦人好風月,家中帶了淫器包在身邊,又服了胡僧薬。婦人摸見他陽物甚大,西門慶亦摸其牝戶,彼此歡欣,情興如火。婦人在床傍伺候鮫綃軟帕,西門慶被底預備麈柄猙獰。當下展猿臂,不覺蝶浪蜂狂;蹺玉腿,那個羞雲怯雨。正是:縱横慣使風流陣,那管床頭墜玉釵。有詩為證:

\begin{myquote}
蘭房幾曲深悄悄,香騰寳鴨清煙裊;

夢囬夜月淡溶溶,展轉牙床春色少。

無心今遇少年郎,但知敲打須宫商;

殢情欲共嬌無力,須敎宋玉赴高唐。

打開重門無鎖鑰,露浸一枝紅芍薬。
\end{myquote}

這西門慶當下竭平生本事,將婦人儘力盤桓了一場。纏至更半天氣,方纔精泄。婦人則髮亂釵横,花憔柳困,鶯顫嚥喘,依稀耳中。兩個並頭交股,摟抱片時,比及起來穿衣之際,婦人下床,款剔銀燈,開了房門,照鏡整容,呼丫鬟捧水淨手,復飲香醪,再勸美酌。三盃之後,西門慶告辭起身,婦人挽留不已,叮嚀頻囑。西門慶躬身領諾,謝擾不盡,相别出門。婦人送到角門首回去了。文嫂先開後門,呼喚玳安、琴童牽馬過來,騎上回家。街上已喝號提鈴,更深夜靜,但見一天霜氣,萬籟無聲。西門慶回家,一宿無話。

到次日,西門慶到衙門中發放已畢,在後廳呌過該地方節級緝捕,吩咐如此如此,這般這般:「王招宣府裏三公子,看有甚麽人勾引他,院中在何人家行走,便與我查訪出名字來,報我知道。」因向夏提刑説:「王三公子甚不學好。昨日他母親再三央人來對我説,倒不関他這兒子事,只被這干光棍勾引他。今若不痛加懲治,將來引誘壞了人家子弟。」夏提刑道:「長官所見不錯,必須誡處他。」節級緝捕領了西門慶鈞語,到當日果然查訪出各人名姓來,打了事件,到後晌時分,來西門慶宅内呈遞揭帖。西門慶見上面有孫寡嘴、祝日念、張小閒、聶鉞兒、向三、于寬、白囬子,樂婦是李桂姐、秦玉芝兒。西門慶取過筆來,把李桂姐、秦玉芝兒,並老孫祝日念名字都抹了,吩咐:「只動這小張閒等五個光棍。即與我㧱了,明日早帶到衙門裏來。」衆公人應諾下去。至晚,打聽王三官衆人都在李桂姐家喫酒,踢行頭,都埋伏在後門首。深更時分,剛散出來,衆公人把小張閒、聶鉞、于寬、白囬子、向三五人都㧱了。孫寡嘴與祝日念爬李桂姐後房走了。王三官兒藏在李桂姐床底下,不敢出來。桂姐一家唬的捏兩把汗,更不知是那裏動人,白央人打聽實信。王三官躲了一夜,不敢出來。李家鴇子又恐怕東京做公的下來㧱人,到五更時分,攛掇李銘換了衣服,送王三官來家。節級緝捕把小張閒等㧱在聽事房,吊了一夜。到次日早晨,西門慶進衙門與夏提刑陞廳,兩邊刑杖羅列,帶人上去。每人一夾二十大棍,打得皮開肉綻,鮮血迸流,哭聲震天,哀號動地。西門慶囑付道:「我把你這起光棍!專一引誘人家子弟在院嫖風,不守本分。本當重䖏,今姑従輕責你這幾下兒。再若犯在我手裏,定然枷號在院門首示衆。」唱令左右:「扠下去!」衆人望外金命水命,走投無命。

兩位官府發放事畢,正在退廳喫茶。夏提刑因説起:「昨日京中舍親崔中書那裏書來,衛中投考察本上去了,還未下來哩;今日會了長官,咱倒好差人往懷慶府,同僚林蒼峯他那裏臨風近,打聽打聽消息去。」西門慶道:「長官所見甚明。」即喚走差的答應,上來跪下,吩咐:「與你五錢銀子盤纏,㧱俺兩個拜帖,即去南河,懷慶府提刑林千戶老爹那裏打聽京中考察本示下,看經歷司行下照會來不曾。務要打聽的實來回報。」那人領了銀子、拜帖,又到司房戴上范陽毡笠,結束行裝,討了疋馬,長行去了。兩位官府起身回家。

卻説小張閒等従提刑院打出來,走在路上,各人着恐,更不量今日受這場虧是那裏薬線,互相埋怨。小張閒道:「莫不還是東京六黄太尉那裏下來的消息?」白囬子道:「不是,若是那裏消息,怎肯輕饒素放?」常言說得好:乖不過唱的,賊不過銀匠,能不過架兒,聶鉞兒一口就説道:「你們都不知道,只我猜得着。此一定是西門官府和三官兒上氣,嗔請他婊子,故㧱俺們煞氣。正是:龍鬬虎傷,苦了小獐!」小張閒道:「列位倒罷了,只是苦了我在下了。孫寡嘴祝麻子都跟着,只把俺們頂缸了。」于寬道:「你怎的説渾話?他兩個是他的朋友,若㧱來跪在地下,他在上面坐着,怎生相處?」小張閒道:「怎的不㧱老婆?」聶鉞道:「兩個老婆都是他心上人。李家桂姐是他婊子,他肯㧱來?也休怪人,是俺們的晦氣,偏撞在這網裏!剛纔夏老爹怎生不言語,只是他説話?這個就見出情弊顯然來了。如今往李桂姐兒家尋王三官去,白為他打了這一屁股瘡來,打的腿爛爛的便罷了?問他要幾兩銀子盤纏也不喫家中老婆笑話。」

於是來來去去,轉彎抹角,逕入勾攔李桂姐家。見門関的鐵桶相似,就是樊噲也撞不開。呌了半日,丫頭隔門問是誰,小張閒道:「是俺們,尋三官兒説話。」丫頭囬説:「他従那日半夜就往家去了,不在這裏。無人在家中,不敢開門。」這衆人只得回來,到王招宣府宅内,逕入他客位裏坐下。王三官聽見衆人來尋他,唬得躲在房裏,不敢出來。半日使出小廝永定來説:「俺爹不在家了。」衆人道:「好自在性兒!不在家了?往那裏去了,呌不將來?」于寬道:「實和你説了罷,休推睡裏夢裏,剛纔提刑院打了俺們,押將出來,如今還要他正身見官去哩。」摟起腿來與永定瞧,敎他進裏面去說:「此事為你,打的俺們有甚要緊!」一個個都躺在板櫈上聲疼呌喊。那王三官兒越發不敢出來,只呌:「娘,怎麽樣兒,卻如何救我則個?」林氏道:「我女婦人家,如何尋人情去救得?」求了半日,見外邊衆人等的急了,要請老太太説話。那林氏又不出去,只隔着屏風説道:「你們略等他等,委的在莊上不在家了。我這裏使小廝呌他去。」小張閒道:「老太太快使人請他來。不然,這個癤子,也要出膿,只顧膿着不是事!俺們為他連累打了這一頓。剛纔老爹吩咐,押出俺們來要他。他若不出來,大家都不得清凈,就弄的不好了。」林氏聽言,連忙使小廝㧱出茶來,與衆人喫。

王三官唬的鬼也似,逼他娘尋人情。到至急之處,林氏方纔説道:「文嫂他只認的提刑西門官府家,昔年曾與他女兒説媒來,在他宅中走的熟。」王三官道:「就認的提刑也罷,快使小廝請他來。」林氏道:「他自従你前番説了他,使性兒一向不來走動,怎好又請他?他也不肯來。」王三官道:「好娘,如今事在至急,請他來,等我與他賠個禮兒便了。」林氏便使永定兒悄悄打後門出去,請了文嫂來。王三官再三央及他,一口一聲只呌:「文媽,你認的提刑西門大官府,好歹説個人情救我。」這文嫂故意做出許多喬張致來,説道:「舊時雖故與他宅内大姑娘説媒,這幾年誰往他門上走?大人家深宅大院,不去纏他。」王三官連忙跪下,説道:「文媽,你救我,自有重報,不敢有忘!那幾個人在前邊只要出官,我怎去得?」那文嫂只把眼看他娘。他娘道:「也罷,你替他説説罷了。」文嫂道:「我獨自個去不得。三叔,你衣巾着,等我領你親自到西門老爹宅上,你自拜見他,央浼他,等我在傍再說說,管情一天事就了了。」王三官道:「現今他衆人在前邊催逼甚急,只怕一時被他看見,怎了?」文嫂道:「有甚難處勾當?等我出去安撫他,再安排些酒肉點心茶水,哄他喫着。我悄悄領你従後門出去幹事回來,他會勝也不知道。」

這文嫂一面走出前廳,向衆人拜了兩拜,説道:「太太敎我出來,多上覆列位哥們,本等三叔往莊上去了,不在家。使人請去了,便來也;你們畧坐坐兒。喫打受駡,連累了列位。誰人不喫鹽米?等三叔來,敎他知遇你們。你們千差萬差,來人不差。恆屬大家只要圖了事,上司差派,不由自己。有了三叔出來,一天大事都了了。」當時衆人一齊道:「還是文媽見的多!你老人家早出來,就説句恁有南北的話兒,俺們也不恁急的了不的。執古法兒只囬不在家,莫不為俺們自做出來的事也罷;你倒帶累俺們喫官棒,上司要你,假推不在家。喫酒喫肉,敎人替你不成?文媽,你是曉道理的。你出來,俺們還透個路兒與你:破些東西兒,尋個分上兒説説,大家了事。你不出來見俺們,這事情也要銷繳。一個緝捕問刑衙門,平不答的就罷了?」文嫂兒道:「哥們説的是。你們略坐坐兒,我對太太説,安排些酒飯兒管待你們。你們來了這半日,也餓了。」衆都道:「還是我的文媽知人甘苦。不瞞文媽説,俺們従衙門裏打出來,黄湯兒也還没曾嚐着哩!」這文嫂走到後邊,一力攛掇打了二錢銀子酒,買了一錢銀子點心,猪羊牛肉,各切幾大盤,㧱將出去。一壁哄他衆人在前廳大酒大肉喫着。

這王三官儒巾青衣,寫了揭帖,文嫂領着,帶上眼紗,悄悄従後門出來,步行逕往西門慶家來。到了大門首,平安兒認的文嫂,説道:「爹纔在廳上,進去了。文媽有甚説話?」文嫂遞與他拜帖,説道:「哥哥,累你替他禀禀去。」連忙問王三官要了二錢銀子遞與他,那平安兒方進去替他禀知西門慶。西門慶見了手本拜帖上寫着:「眷晚生王寀頓首百拜。」一面先呌進文嫂,問了囬話。然後纔開大廳槅子門,使小廝請王三官進去大廳上。左右忙掀暖簾兒,西門慶頭戴忠靖冠,便衣出來迎接。見王三官衣巾進來,故意說道:「文嫂怎不早說?我褻衣在此!」便令左右:「取我衣服來。」慌的王三官向前攔住叫:「尊伯尊便!小姪敬來拜凟,豈敢動勞!」至廳内,王三官務請西門慶轉上行禮。西門慶笑道:「此是舍下。」再三不肯。西門慶居先拜下去,王三官說道:「小姪有罪在身,久仰,欠拜。」西門慶道:「彼此少禮。」王三官因請西門慶受禮,説道:「小姪人家,老伯當得受禮,以恕拜遲之罪。」務讓起來,受了兩禮,王三官然後挪座兒斜僉坐的。少頃,喫了茶,王三官見西門慶廳上錦屏羅列,四壁挂四軸金碧山水,座上鋪着緑錦緞镶嵌貂鼠椅座,地下氍毹匝地,正中間黄銅四方屏,水磨的耀目爭輝,上面牌扁下書「承恩」二字,係米元章妙筆。觀覽之餘,似有叩請疑難之貌,向西門慶說道:「小姪現有一事,不敢奉凟尊嚴。」因向袖中取出揭帖遞上,隨即離席跪下。被西門慶一手拉住,説道:「賢契有甚話,但説何害!」這王三官就說:「小姪不才,誠為得罪。望乞老伯念先父武弁一殿之臣,寬恕小姪無知之罪,完其廉耻,免令出官。則小姪垂死之日,實有再生之幸也!啣結圖報,惶恐惶恐!」西門慶展開揭帖,上面有小張閒等五人名字,說道:「這起光棍,我今日衙門裏已各重責發落,饒恕了他,怎的又央你去?」王三官道:「正是。如此這般,他說老伯衙門中責罰了他,押出他來,還要小姪見官。在小姪家百般稱罵喧嚷,索要銀兩,不得安生。無處控訴,前來老伯這裏請罪。」又把禮帖遞上。西門慶一見,便道:「豈有是理!」因說道:「這起光棍可惡!我倒饒了他,如何倒往那裏去攪擾!」把禮帖還與王三官收了,道:「賢契請囬,我也且不留你坐。如今即時就差人㧱這起光棍去,容日奉招。」王三官道:「豈敢!蒙老伯不棄,小姪容當踵門叩謝。」千恩萬謝出門。西門慶送至二門首説:「我褻服不好送的。」那王三官自出門,還帶上眼紗,小廝跟隨去了。文嫂還討了西門慶話。西門慶吩咐:「休要驚動他,我這裏差人㧱去。」

這文嫂同王三官暗暗到家。不想西門慶隨即差了一名節級,四個排軍,走到王招宣宅内。那起人正在那裏飲酒喧鬧,被公人進去,不由分説,都㧱了,帶上鐲子。唬得衆人面如土色,說道:「王三官幹得好事!把俺們穩在你家,倒把鋤頭,反弄俺們來了!」那個排軍節級駡道:「你這廝還胡説,當的甚麽?各人到老爹跟前哀告,討你那命正經!」小張閒道:「大爹教導的是。」不一時,都㧱到西門慶宅門首,門上排軍並平安都張着手兒要錢,纔去替他禀。衆人不免脱下褶兒,並㧱頭上簪圈下來,打發停當,方纔說進去。半日,西門慶出來坐廳,節級帶進去,跪在廳下。西門慶駡道:「我把你這起光棍!我倒將就了,如何指稱我這衙門往他家嚇詐去?實說,詐了多少錢?不説,令左右㧱拶子與我着實拶起來!」當下只說了聲,那左右排軍,登時取了五六把新拶子來伺候。小張閒等只顧在下叩頭哀告道:「小的並沒嚇詐分文財物。只說衙門中打出小的們來,對他説聲。他家㧱出些酒食來,管待小的,小的並沒需索他的。」西門慶道:「你也不該往他家去。你這起光棍,設騙良家子弟,白手要錢,深為可惡!旣不肯實供,都與我帶了衙門裏收監,明日嚴審取供,枷號示衆。」衆人一齊哀告,哭道:「天官爺,超生小的們罷!小的再不敢上他門纏擾了。休説枷號,這一送到監裏去,冬寒時月,小的們都是死數!」西門慶道:「我把你這光棍!我逭饒出你去,都要洗心改過,務安生理。不許你挨坊靠院引誘人家子弟,詐騙財物。再㧱到我衙門裏來,都活打死了!」喝令:「出去罷!」衆人得了個性命,往外飛跑。正是:敲碎玉籠飛彩鳳,頓開金鎖走蛟龍。

西門慶發了衆人去,回至後房。月娘問道:「這個是那個王三官兒?」西門慶道:「此是王招宣府中三公子。前日李桂兒為他那場事,就是他。今日賊小淫婦兒不改,又和他纏,每月三十兩銀子敎他包着,嗔道一向只哄着我。不想有個底脚裏人兒又告我説,教我昨日差幹事的㧱了這干人到衙門裏去,都夾打了。不想這干人又到他家裏嚷賴,指望要詐他幾兩銀子的情,只恐嚇衙門中要他。他従來沒曾見官,慌了,央文嫂兒㧱五十兩禮帖來,求我説人情。我剛纔把那起人又㧱了來,詐發了一頓,替他杜絶了,再不纏他去了。人家倒運,偏生出這樣不肖子弟出來。你家父祖何等根基,又做招宣,你又現入武學,放着那功名兒不幹,家中丢着花枝般媳婦兒——是東京六黄太尉姪女兒——不去理論,白日黑夜,只跟着這夥光棍在院裏嫖弄,把他娘子頭面都㧱出來使了。今年不上二十歲,年小小兒的,通不成器!」月娘道:「你不曾溺泡尿看看自家影兒。老鴉笑話猪兒黑,原來燈臺不照自。你自道成器的,你也喫這井裏水,無所不為,清潔了些甚麽兒?還要禁的人!」幾句説的西門慶不言語了。

正擺上飯來喫,小廝來安來報:「應二爹來了。」西門慶吩咐:「請書房裏坐,我就來。」王經連忙開了廳上書房門,伯爵進裏面暖爐炕傍椅上坐了。良久,西門慶出來。聲喏畢,就坐在炕上兩個説話。伯爵道:「哥,你前日在謝二哥那裏,怎的老早就起身?」西門慶道:「第二日我還要早起,衙門中連日有勾當,又考察在邇,差人東京打聽消息。我比你們閒人兒?」伯爵又問:「哥,連日衙門中有事沒有?」西門慶道:「事那日沒有?」伯爵又道:「王三官兒說,哥衙門中動人了,把小張閒他們五個,初八日晚夕在李桂姐屋裏都㧱的去了,只走了老孫祝麻子兩個,今早解到衙門裏,都打出來了,衆人都往招宣府纏王三官去了。怎的還瞞着我不説?」西門慶道:「儍狗才,誰對你說來?你敢錯聽了,敢不是我衙門裏,敢是周守備府裏!」伯爵道:「守備府中那裏管這閒事!」西門慶道:「只怕是都中提人。」伯爵道:「也不是。今早李銘對我說,那日把他一家子唬的魂也沒了。李桂兒至今唬的睡倒了,這兩日還沒曾起炕兒。頭裏生怕又是東京下來㧱人,今早打聽,方知是提刑院動人。」西門慶道:「我連日不進衙門,並没知道。李桂兒旣賭過誓不接他,隨他㧱去亂去,又害怕睡倒怎的!」伯爵見西門慶迸着臉兒待笑,説道:「哥,你是個人?連我也瞞着起來,不告我説。今日他告我説,我就知道哥的情:怎的祝麻子老孫走了,一個緝事衙門,有個走脱了人的?此是哥打着綿羊駒䮫戰,使李桂兒家中害怕,知道哥的手段。若都㧱到衙門去,彼此絶了情意,都沒趣了。事情許一不許二。如今就是老孫、祝麻子,見哥也有幾分慚愧。此是哥明修棧道,暗度陳倉的計策。休怪我説,哥這一着做的絶了。這一個叫做眞人不露相,露相不是眞人。若明使道兒,逞了臉,就不是乖人兒了。還是哥智謀大,見的多。」幾句説的西門慶撲喫的笑了,説道:「我有甚麽大智謀?」伯爵道:「我猜一定還有底脚裏人兒對哥説。怎得知道這等切,端的有鬼神不測之機!」西門慶道:「儍狗才,若要人不知,除非己莫為。」伯爵道:「哥衙門中如今不要王三官兒罷了?」西門慶道:「誰要他做甚麽!當初幹事的打上事件,我就把王三官、祝麻子、老孫,並李桂兒、秦玉芝名字都抹了。只來打㧱幾個光棍。」伯爵道:「他如今怎的還纏他?」西門慶道:「我實和你説罷。他指稱嚇詐他幾兩銀子,不想剛纔親上門來拜見,與我磕了頭,賠了不是。我還差人把那幾個光棍㧱了,要枷號,他衆人再三哀告,説再不敢上門纏他了。王三官一口一聲稱呼我是老伯,㧱了五十兩禮帖兒,我不受他的。他到明日還要請我家中知謝我去。」伯爵失驚道:「眞個他來和哥賠不是來了?」西門慶道:「我莫不哄你?」因喚王經:「㧱王三官拜帖兒與應二爹瞧!」那王經向房子裏取出拜帖,上面寫着:「晚生王寀頓首百拜。」伯爵見了,口中只是極口稱贊:「哥的所算,神妙不測!」西門慶吩咐伯爵:「你若看見他們,只說我不知道。」伯爵道:「我曉得。機不可泄,我怎肯和他説。」坐了一回,喫了茶,伯爵道:「哥,我去罷。只怕一時老孫和祝麻子摸將來,只說我沒到這裏。」西門慶道:「他就來,我也不出來見他,只答應不在家。」一面呌將門上人來,都吩咐了:「但是他二人,只答應不在。」西門慶従此不與李桂姐上門走動,家中擺酒,也不呌李銘唱曲,就疎淡了。正是:昨夜浣花溪上雨,綠楊芳草為何人?有詩為證:

\begin{myquote}
誰道天台訪玉眞,三山不見海沉沉。

侯門一入深如海,従此蕭郎是路人。
\end{myquote}

畢竟未知後來如何,且聽下回分解。

