\includepdf[pages={139,140},fitpaper=false]{tst.pdf}
\chapter*{第七十囬 \\西門慶工完陞級 羣僚庭參朱太尉}
\addcontentsline{toc}{chapter}{第七十囬 西門慶工完陞級 羣僚庭參朱太尉}
\markboth{{\titlename}卷之七}{第七十囬 西門慶工完陞級 羣僚庭參朱太尉}


\begin{myquote}
昨夜西風鼓角喧,曉來隆凍怯寒毡。

茫茫一片渾無地,浩浩四方俱是天!

綺壁凄凉宜未守,霸陵豪傑且停鞭。

陽春有脚恩如海,願借餘溫到客邊。
\end{myquote}

話説西門慶自此與李桂姐断絶,不題。卻説走差人到懷慶府林千户處打聽消息,林千户將陞官邸報封付與來人,又賞了五錢銀子,連夜來遞與提刑兩位官府。當廳夏提刑拆開,同西門慶先觀本衛行來考察官員照會。其略曰:

\begin{myquote}[\markfont]
「兵部一本:尊明旨,嚴考覈,以昭勸懲,以光聖治事。先該金吾衛提督官校太尉、太保,兼太子太保朱題前事,考察禁衛官員,除堂上官自陳外,其餘兩廂詔獄緝捕、捉察、譏察、觀察,典牧皇畿,内外提刑所指揮、千百户、鎭撫等官,各按册籍,祖職世襲、轉陞、功陞、蔭陞、納級等項,各挨次格,従公擧劾,甄別賢否,具題上請。當下該部詳議黜陟陞調降革等因。奉

聖旨:兵部知道,欽此欽遵。抄出到科,按行到部。看得太尉朱題前事,遵奉舊例,委的本官殫力致忠,公于考覈,委所同並内外屬官,各據册籍,博協輿論,甄別賢否,皆出聞見之實,而無偏執之私。足見本官仰扳天顔之咫尺,而存體國之忠謀也。分别等第,獎勵淑慝,井井有條,足以勵人心而孚公議,無容臣等再喙。但恩威賞罰,出自朝廷,合候命下之日,一體照例施行等因,庶考覈明而人心服,冒濫革而官箴肅矣。奉欽此,欽依擬行。

内開:

山東提刑所正千户夏廷齡,資望旣久,才練老成。昔視典牧而坊隅安靜,今理齊刑而綽有政聲。宜加獎勵,以冀甄陞,可備鹵簿之選者也。貼刑副千户西門慶,才幹有為,英偉素著。家稱殷實而在任不貪,國事克勤而臺工有績。翌神運而分毫不索,司法令而齊民畢仰。宜加轉正,以掌刑名者也。懷慶提刑千户所正千户林承勳,年清優學,占籍武科。継祖職抱負不凡,提刑獄幹濟有法,可加薦獎勵簡任者也。副千户謝恩,年齒旣殘,泰嚴亾度。昔在行伍,猶有可觀;今任理刑,罷軟尤甚。可宜罷黜革任者也。」
\end{myquote}

西門慶看了他轉正千户掌刑,心中大悦。夏提刑見他陞指揮管鹵簿,大半日無言,面容失色。於是又展開工部工完的本觀看,上面寫道:

\begin{myquote}[\markfont]
「工部一本:神運届京,天人胥慶。懇乞天恩,俯加渥典,以蘇民困,以廣聖澤事。奉

聖旨,這神運奉迎大内,奠安艮嶽,以承天眷,朕心嘉悦。你們旣效有勤勞,副朕事玄至意,所經過地方,委的小民困苦。着行撫按衙門,查勘明白,着行蠲免今歲田租之半。所毀壩閘,你部裏差官,會同巡按御史,即行修理。完日還差内侍孟昌齡前去致祭。蔡京、李邦彦、王煒、鄭居中、高俅,輔弼朕躬,直贊内庭,勳勞茂著:京加太師,邦彦加柱國太子太師,王煒太傅,鄭居中、高俅太保,各賞銀五十兩、四表裏。蔡京還蔭一子為殿中監。國師林靈素,明知朕兆,佐國宣化,遠致神運,北伐虜謀,實與天通,加封忠孝伯,食祿一千石,賜坐龍衣一襲,肩輿入内,賜號玉眞教主,加淵澄玄妙廣德眞人,金門羽客,通眞達靈玄妙先生。朱勔、黄經臣,督理神運忠勤可嘉。勔加太傅兼太子太傅,經臣加殿前都太尉,提督御前人船,各蔭一子為金吾衛正千户。内侍李彦、孟昌齡、賈祥、何沂、藍従熙,着直延福五位官近侍,各賜蟒衣玉帶,仍蔭弟姪一人為副千户,俱現任管事。禮部尚書張邦昌、左侍郎兼學士蔡攸、右侍郎白時中、兵部尚書余深、工部尚書林攄,俱加太子太保,各賞銀四十兩,彩緞二表裏。巡撫兩浙僉都御史張閣,陞工部右侍郎。巡撫山東都御史侯蒙,陞太常正卿。巡撫兩浙、山東監察御史尹大諒、宋喬年、都水司郎中安忱、伍訓,各陞俸一級,賞銀二十兩。祇迎神運千户魏承勳、徐相、楊廷珮、司鳳儀、趙友蘭、扶天澤、西門慶、田九皋等,各陞一級。内侍宋推等,營將王佑等,俱各賞銀十兩。所官薛顥忠等,各賞五兩。校尉昌玉等,絹二疋。該衙門知道。」
\end{myquote}

夏提刑與西門慶看畢,各散衙回家。後晌時分,有王三官差永定同文嫂㧱着請書盒兒來,内安泥金摺,十一日請西門慶往他府中赴席,少罄謝私之意。西門慶收下,不勝歡喜,以為其妻指日在於掌握。不期到初十日晚夕,東京本衛經歷司,差人行照會到:「曉諭各省提刑官員知悉,火速赴京,赶冬至令節見朝引奏謝恩,毋得違誤,取罪不便。」西門慶看了,到次日衙門中會了夏提刑,回手本打發來人回去,不在話下。各人到家,收拾行裝,備辦贄見禮物,不日約會起程。西門慶使玳安呌了文嫂兒,教他囬王三官,十一日不得來赴席,如此這般,「上京見朝謝恩去也。」王三官道:「旣是老伯有事,待容囬來,潔誠具請。」西門慶一面呌將賁四,吩咐教他跟了去,與他五兩銀子家中盤纏。留下春鴻看家,帶了玳安、王經,跟隨答應。又問周守備討了四名巡捕軍人,四匹小馬,打點馱裝、暖轎,排軍擡扛。夏提刑那邊夏壽跟隨。兩家有二十餘人跟従。十二日起身,離了清河縣,冬天易晚,晝夜趲行。到了懷西懷慶府,會林千户。千户已上東京去了。一路天寒坐轎,天暖乘馬,朝登紫陌紅塵,夜宿郵亭旅邸。正是:意急款搖青毡幙,心忙摔碎紫絲鞭。

評話捷説。到了東京,進得萬壽門來。依着西門慶就要分別,他主意要往相國寺下;夏提刑不肯,堅執要請往他令親崔中書家投下。西門慶不免先具拜帖拜見。正値崔中書在家,即出迎接,至廳敍禮相見,道及寒喧契闊之情,拂去塵土,坐下,茶湯已畢,拱手問西門慶尊號。西門慶道:「賤號四泉。」因問:「老先生尊號?」崔中書道:「學生性最愚朴,坐閑林下,賤名守愚,拙號遜齋。」因説道:「舍親龍溪,久稱盛德,全仗扶持,同心協恭,莫此為厚!」西門慶道:「不敢。在下常領敎誨,今又為堂尊,受益恆多,可幸可幸!」夏提刑道:「長官如何這等稱呼!雖有鎡基,不如待時。」崔中書道:「四泉説的也是,名分使然,不得不爾。」言畢,彼此笑了。不一時,收拾了行李,天晚了,崔中書吩咐童僕放桌擺飯,無非是菓酌餚饌之類,不必細說。當日二人在崔中書家宿歇不題。

到次日,各備禮物拜帖,家人跟隨,早往蔡太師府中叩見。那日太師在内閣還未出來,府前官吏人等如蜂屯蟻聚,通擠匝不開。西門慶與夏提刑與了門上官吏兩包銀子,㧱揭帖禀進去。翟管家見了,即出來相見,讓他到外邊私宅。先是夏提刑相見畢,然後西門慶敍禮,彼此道及往還酬答之意,各分賓主坐下。夏提刑先遞上禮帖:兩疋雲鶴金緞,兩疋色緞;翟管家的是十兩銀子。西門慶禮帖上是一疋大紅絨綵蟒,一疋玄色粧花斗牛補子員領,兩疋京緞;另外梯己送翟管家一疋黑綠雲絨,三十兩銀子。翟謙吩咐左右:「把老爺禮都交收進府中去,上簿籍。」他只受了西門慶那疋雲絨,將三十兩銀子連那夏提刑的十兩銀子都不受。説道:「豈有此理?若如此,不見至交親情!」一面令左右放桌兒擺飯,説道:「今日聖上奉艮嶽,新蓋上清寳籙宫奉安牌匾,該老爺主祭,直到午後纔散。到家同李爺又往鄭皇親家喫酒,只怕親家和龍溪等不的,悞了你們勾當。遇老爺閑,等我替二位禀,就是一般。」西門慶道:「蒙親家費心,若是這等又好了!」翟謙因問:「親家那裏住?」西門慶就把夏龍溪令親家下歇説了。不一時,安放桌席端正,就是大盤大碗,湯飯點心,一齊㧱上來,都是光祿烹炮羙味,極品無加。每人金爵飲酒三盃,就要告辭起身。翟謙於是款留,令左右再篩上一盃。西門慶因問:「親家,俺們幾時見朝?」翟謙道:「親家,你同不得夏大人。夏大人如今是京堂官,不在此例。你與本衛新陞的副千户何太監姪兒何永壽,他便貼刑,你便掌刑,與他作同僚了。他先謝了恩,只等着你見朝引奏畢,一同好領劄付。你凡事只會他去。」夏提刑聽了,一聲兒不言語。西門慶道:「請問親家,你曉的我還等冬至郊天畢囬來,見朝如何?」翟謙道:「親家你等不的。冬至聖上郊天回來,那日天下官員上表朝賀畢,還要排慶成宴,你們怎等的?不如你今日先鴻臚寺報了名,明日早朝謝了恩,直到那日堂上官引奏畢,領劄付起身就是了。」西門慶謝道:「蒙親家指教,何以克當!」臨起身,翟謙又拉西門慶到側凈處説話,甚是埋怨西門慶説:「親家,前日我的書去,那等寫了,大凡事要謹密,不可使同僚們知道。親家如何對夏大人説了,教他央了林眞人帖子來,立逼着朱太尉來對老爺説,要將他情願不官鹵簿,仍以指揮職銜在任所掌刑三年。兼况何太監又在内廷,轉央朝廷所寵安妃劉娘娘的分上,便也傳旨出來,親對太爺和朱太尉説了,要安他姪兒何永壽在山東理刑。兩下人情阻住了,教老爺好不作難。不是我再三在老爺跟前維持,回倒了林眞人,把親家不撑下去了?」慌的西門慶連忙打躬,説道:「多承親家盛情!我並不曾對一人説,此公何以知之?」翟謙道:「自古機事不密則害成,今後親家凡事謹愼些便了。」這西門慶千恩萬謝,與夏提刑作辭出門。

來到崔中書家,一面差賁四鴻臚寺報了名。次日見朝,青衣冠帶,同夏提刑進内,不想只在午門前謝了恩。出來,剛轉過西闕門來,只見一個青衣人走向前問道:「那位是山東提刑西門慶老爹?」賁四問道:「你是那裏的?」那人道:「我是内府匠作監何公公來請老爹説話。」言未畢,只見一個太監,身穿大紅蟒衣,頭戴三山帽,脚下粉底皂靴,従御街高聲呌道:「西門大人請了!」西門慶遂與夏大人分别,被這太監用手一把拉在傍邊一所直房内,都是明窗亮槅,裏靣籠的火暖烘烘的,桌上陳設的許多桌盒。一靣相見,作了揖,慌得西門慶倒身還禮不迭。這太監説道:「大人,你不認的我,在下是内府匠作太監何沂,現在延寜第四宫端妃馬娘娘位下近侍。昨日内工完了,蒙萬歲爺爺恩典,將姪男何永壽陞授金吾衛左所副千户,現在貴䖏提刑所理刑管事,與老大人作同僚。」西門慶道:「原來是何老太監!學生不知,恕罪恕罪!」一靣又作揖説道:「此禁地不敢行禮,容日到老太監外宅進拜。」於是叙禮畢,讓坐。家人捧茶,金漆朱紅盤托盞遞上茶去喫了。茶畢,就揭桌盒蓋兒。桌上許多湯飯餚品,㧱盞筯兒來安下。何太監道:「不消小盃了,我曉的大人朝下來,天氣寒冷,㧱個大盞來。沒甚麽餚,褻凟大人,且喫個頭腦兒罷。」西門慶道:「不敢叨擾!」何太監於是滿斟上一大盃,遞與西門慶。西門慶道:「承老太監所賜,學生領下。只是出去還要見官拜部,若喫得面紅,不成道理。」何太監道:「喫兩盞兒盪寒,何害?」因説道:「舍姪兒年幼,不知刑名。望乞大人看我面上,同僚之間,凡事教導他敎導。」西門慶道:「豈敢!老太監勿得太謙!令姪長官雖是年幼,居氣養體,自然福至心靈。」何太監道:「大人好説。常言:學到老,不會到老。天下事如牛毛,孔夫子也識得一腿。恐有不知到處,大人好歹説與他。」西門慶道:「學生謹領。」因問:「老太監外宅在何䖏?學生好去奉拜長官。」何太監道:「舍下在天漢橋東文華坊雙獅馬台就是。」亦問:「大人下處在那裏?我教做官的先去叩拜。」西門慶道:「學生暫借崔中書家下。」彼此問了住䖏,西門慶喫了一大盃就起身。何太監送出門,拱着手説道:「適間所言,大人凡事看顧看顧,他還等着你會同一答兒引奏,當堂上作主,進了禮,好領劄付。」西門慶道:「老太監不消吩咐,學生知道。」

於是出朝門,又到兵部。又遇見了夏提刑,同拜了部官來。比及到本衛參見朱太尉,遞履歷手本,繳劄付,又拜經歷司並本所官員,已是申刻時分。夏提刑改換指揮服色,另具手本,參見了朱太尉,免行跪禮,擇日南衙到任。剛出衙門,西門慶還等着,遂不敢與他同行,讓他先上馬。夏延齡那裏肯,定要同行。西門慶趕着他呼堂尊。夏指揮道:「四泉,你我同僚在先,為何如此稱呼?」西門慶道:「名分已定,自然之道,何故太謙?」因問:「堂尊高陞羙任,不還山東去了。寳眷幾時搬取?」夏延齡道:「欲待搬來,那邊房舍無人看守。如今且在舍親這邊權住,直待過年差人取家小罷了。日逐望長官早晚家中看顧一二!房子若有人要,就央長官替我打發,自當感謝。」西門慶道:「學生謹領。請問府上那房價値若干?」夏延齡道:「舍下此房,原是一千三百兩買的徐内相房子,後邊又蓋了一層,收拾使了二百兩。如今賣原價也罷了。」西門慶道:「堂尊說與我,有人問,我好回答,庶不悞了。」夏延齡道:「只是有累長官費心!」

二人歸到崔宅,王經向前禀説:「新陞何老爹來拜,下馬到廳,小的囘部中還未來家。何老爹説多拜上,還與夏老爹崔老爹都投下帖。午間差人送了兩疋金緞來。」宛紅帖兒㧱與西門慶看。上寫着:「謹具緞帕二端,奉引贄敬。寅侍敎生何永壽頓首拜。」西門慶看了,連忙差王經封了兩疋南京五彩獅補員領,寫了禮帖,喫了飯,連忙往何家囘拜去。到於廳上,何千户忙整衣迎接出來,穿着五彩粧花玄色雲絨獅補員領,烏紗皂履,腰繫玳瑁蒙金帶;年紀不上二十歲,生的面如傅粉,眉目清秀,唇若塗朱,趨下堦來,揖讓退遜,謙恭特甚。西門慶陞階,左右忙去掀簾。呼喚一聲,奔走後先應諾。二人到廳上叙禮,西門慶令玳安揭開緞盒,捧上贄見之禮,拜下去説道:「適承光顧,兼領厚儀,有失迎迓。今早又蒙老公公直房賜饌,感德不盡!」何千户忙頂頭還禮説:「小弟叨受微職,忝與長官同例,早晚得領敎益,實為三生有幸!適間進拜不遇,又承垂愛,蓬蓽光生!」令左右收下去。一面扯公座椅兒,都是麈皮坐褥,分賓主坐下。左右捧上茶來,何千户躬身捧茶,遞與西門慶。西門慶亦離席交換。喫茶之間彼此問號,西門慶道:「學生賤號四泉。」何千户道:「學生賤號天泉。」又問:「長官今日拜畢部堂了?」西門慶道:「従内裏蒙公公賜酒出來,拜畢部,又到本衙門見堂,繳了劄付,拜了所司,出來見長官尊帖,下顧失迎,不勝惶恐!」何千户道:「不知長官到,學生拜遲。」因問:「長官今日與夏公都見朝來?」西門慶道:「龍溪今已陞了指揮直駕,今日都見朝謝恩在一處。只到衙門見堂之時,他另具手本參見。」問畢,何千户道:「今日與長官計議了,咱們幾時與本主老爹見禮領劄付?」西門慶道:「依着舍親說,咱們先在衛主宅中進了禮,然後大朝引奏,還在本衙門到堂,同衆領劄付。」何千户道:「旣是長官如此說,咱們明日早備禮進了罷。」於是都會下各人禮數:何千户是兩疋蟒衣,一束玉帶;西門慶是一疋大紅麒麟金緞,一疋青絨蟒衣,一柄金鑲玉縧環;各金華酒四罈。明早在朱太尉宅前取齊。約會已定,茶湯兩換,西門慶告辭而囘,並不與夏延齡題此事。一宿晚景題過。

到次日早,到何千户家,何千户又是預備飯食,頭腦小席,大盤大碗,齊齊整整。連手下人飽餐一頓,然後同往太尉宅門前來。賁四同何家人,又早押着禮物,伺候已久。那時正値朱太尉新加太保,徽宗天子又差遣往南壇視牲未囬。各家餽送賀禮、伺候參見官吏人等,黑壓壓在門首等的鐵桶相似。何千户同西門慶下了馬,在左近一相識家坐的,差人打聽:「老爺道子響,就來通報。」

一等等到午後時分,忽見一人飛馬而來,傳報道:「老爺視牲囬來,進南薰門了,吩咐閒雜人打開!」不一時,騎報囬來傳:「老爺過天漢橋了!」頭一廚役跟隨茶盒攢盒到了。半日纔遠遠牌兒馬到了。衆官都頭带勇字鎖鐵盔,身穿摟漆紫花甲,青紵絲團花窄袖衲襖,紅綃裹肚,綠麂皮挑線海獸戰裙,脚下四縫着腿黑靴;弓彎雀畫,箭插雕翎,肩上横擔銷金令字藍旗。端的人如猛虎,馬賽飛龍。須臾一對藍旗過來,夾着一對對青衣節級上,一個個長長大大,搊搊搜搜,頭带黑青巾,身穿皂直裰,脚上乾黄皮底靴,腰間懸繫虎頭牌,騎在馬上,端的威風凜凜,相貌堂堂。須臾,三隊牌兒馬過畢,只聞一片喝聲傳來。那喝道者都是金吾衛士,直場排軍,身長七尺,腰闊三停,人人青巾桶帽,個個腿纏黑靴,左手執着藤棍,右手潑步撩衣,長聲道子一聲聲喝道而來,下路端的嚇魄消魂,陡然市衢澄靜。頭道過畢,又是二道摔手。摔手過後,兩邊雁翎排列二十名青衣緝捕,皆身ん長大,都是寬腰大肚之輩,金眼黄鬚之徒,個個貪殘類虎,人人那有慈悲。十對青衣後面,是八擡八簇肩輿明轎,轎上坐着朱太尉。頭戴烏紗,身穿猩紅斗牛絨袍,腰横四指荆山白玉玲瓏带,脚趿皂靴,腰懸太保牙牌,黄金魚鑰,頭带貂蟬,脚登虎皮踏,擡的那轎離地約有三尺高。前面一邊一個相抱角带,身穿青紵絲家人跟着。轎後又是一班兒六面牌兒馬,六面令字旗緊緊圍護,以聽號令。後約有數十人,都騎着寳鞍駿馬,玉勒金ね,都是官家親隨、掌案、書辦、書吏人等,都出於纨绔仕宦驕養,只知好色貪財,那曉王章國法。登時一隊隊都到宅門首,一字兒擺下。喝的人靜迴避,無一人聲嗽。那來見的官吏人等,黑壓壓一羣,跪在街前。良久,太尉轎到跟前,左右喝聲:「起來伺候!」那衆人一齊聲諾,誠然聲震雲霄。

只聽東邊鼕鼕鼓楽响動,原來本衙六員太尉堂官,見朱太尉新加光祿大夫、太保,又蔭一子為千戸,都各備大禮在此,治具酒筵,來此慶賀,故此有許多教坊伶官在此動楽。太尉纔下轎,樂就止了。各項官吏人等,預備進見。忽然一聲道子響,一青衣承差手㧱兩個紅拜帖,飛走而來,遞與門上人,説:「禮部張爺與學士蔡大爺來拜!」連忙稟報進去。須臾,轎在門首,尚書張邦昌與侍郎蔡攸,都是紅吉服孔雀補子,一個犀带,一個金带。進去拜畢,待茶畢,送出來。又是吏部尚書王祖道與左侍郎韓梠,右侍郎尹京,也來拜,朱太尉都待茶,送了。又是皇親嘉國公、樞密使鄭居中、駙馬掌宗人府王晋卿,都是紫花玉帶來拜,惟鄭居中坐轎,這兩個都騎馬。送出去,方是本衙堂上六員太尉到了,呵殿喧儀,行仗羅列。頭一位是提督管兩廂捉察使孫榮,第二位管譏察梁應龍,第三管内外觀察典牧畿童太尉姪兒童天胤,第四提督京城十三門巡察使黄經臣,第五管京營衛緝察皇城使竇監,第六督管京城内外巡捕使陳宗善。都穿大紅,頭帶貂蟬;惟孫榮是太子太保,玉帶,餘者都是金帶。下馬進去,各家都有金幣尺頭禮物。少頃,裏面楽聲響動,衆太尉插金花,㧱玉帶,與朱太尉把盞遞酒。堦下一派簫韶盈耳,兩行絲竹和鳴。端的食前方丈,花簇錦筵。怎見得太尉的富貴?但見:

\begin{myquote}
官居一品,位列三台。赫赫公堂,晝長鈴索靜;潭潭相府,漏定戟杖齊。林花散彩賽長春,簾影垂虹光不夜。芬芬馥馥,獺髓新調百和香;隱隱層層,龍紋大篆千金鼎。衾擁半牀翡翠,枕欹八寳珊瑚。時聞振珮玉叮咚,待看傳燈金錯落。虎符玉節,門庭甲仗生寒;象板銀箏,傀儡排場熱鬧。終朝謁見,無非公子王孫;逐歲追遊,盡是侯門戚里。雪兒歌發,驚聞麗曲三千;雲母屏開,忽見金釵十二。平鋪荷芰,遊魚沼内不驚人;高挂樊籠,嬌鳥簾前能對語。那裏解調和爕理,一味趨諂逢迎。端的笑談起干戈,吹嘘驚海嶽。假旨令八位大臣拱手,巧辭使九重天子點頭。督擇花石,江南淮北盡灾殃;進獻黄楊,國庫民財皆匱竭。當朝無不心寒,烈士為之屛息。正是:輦下權豪第一,人間富貴無雙。
\end{myquote}

須臾遞畢,安席坐下。一班兒五個俳優,朝上箏阮琵琶,方響箜篌,紅牙象板,唱了一套〔正宫·端正好〕,端的餘音繞梁,聲清韻羙。唱道:

\begin{myquote}
「享富貴,受皇恩;起寒賤,居高位。秉權衡威振京畿,惟君恃寵把君王媚,全不想存仁義。」

{\markfont〔滚綉球〕}「起官夫造水池,與兒孫買田基,苦求謀都只為一身之計。縱奸貪那裏管越瘦吳肥。趨附的身即榮;觸忤的命必危。妬賢才,喜親小輩,只想着復私仇公道全虧。你將九重天子深瞞眛,致令的四海生民總亂離,更不道天網恢恢!」

{\markfont〔倘秀才〕}「巧言詞,取君王一時笑喜,那裏肯效忠良使萬國雍熙。你只待顛倒豪傑把世迷。隔靴空揉癢,久症卻行醫,滅絶了天理!」

{\markfont〔滚綉球〕}「你有秦趙高指鹿心,屠岸賈縱犬機。待學漢王莽不臣之意,欺君的董卓燃臍。但行動絃管隨,出門時兵仗圍。入朝中百官悚畏,仗一人假虎張威。望塵有客趨奸黨,借劍無人斬佞賊,一任的你狂為!」

{\markfont〔尾聲〕}「金甌底下無名姓,青史編中有是非。你那知爕理陰陽調元氣,你止知盜賣江山結外夷!枉辱了玉带金魚挂蟒衣,受祿無功愧寢食。權方在手人皆懼,祸到臨頭悔後遲。南山竹罄難書罪,東海波乾臭未遺。萬古流傳,教人唾駡你!」
\end{myquote}

當時酒進三巡,歌吟一套,六員太尉起身,朱太尉親送出來。囬到廳,樂聲暫止,管家禀事,各處官員進見。朱太尉令左右擡公案,就在當廳一張虎皮校椅上坐下。吩咐出來,先令各勳戚、中貴、仕宦家人吏書人等送禮的進去。須臾打發出來,纔是本衛紀事,南北衙兩廂五所七司捉察、譏察、觀察、巡察、典牧、直駕、提牢、指揮、千百戸等官,各有首領,具手本呈遞。然後纔傳出來,呌兩淮、兩浙、山東、山西、関東、関西、河南、河北、福建、廣南、四川十三省提刑官,挨次進見。西門慶與何千户在第五起上,擡進禮物去,管家又早將何太監拜帖鋪在書案上,二人立在堦下,等上邊呌名字。這西門慶擡頭,見正面五間皆廠廳,歇山轉角,滴水重簷,珠簾高捲,週そ都是綠欄杆。上面朱紅牌扁,懸着徽宗皇帝御筆欽賜「執金吾堂」斗大小四個金字,乃是官家耳目牙爪所察緝訪密之所,常人到此者䖏斬。兩邊六間廂房,堦墀寬廣,院宇深沉。朱太尉身着大紅,在上面坐着。須臾,呌到跟前,二人應諾陞堦,到滴水簷前躬身參謁,四拜一跪,聽發放。朱太尉道:「那兩員千户,怎的又呌你家太監送禮來?」令左右收了,吩咐:「在地方謹愼做官,我這裏自有公道。伺候大朝引奏畢,來衙門中領劄赴任。」二人齊聲應諾。左右喝:「起去!」由左角門出來。

剛出大門來,尋見賁四等擡擔出來。正要走,忽聽一人㧱宛紅拜帖飛馬來報,説道:「王爺高爺來了。」西門慶與何千户閃在人家門裏觀看。須臾,軍牢喝道,人馬圍隨,塡街塞巷。只見總督京營八十萬禁軍隴西公王燁,同提督神策御林軍總兵官太尉高俅,俱大紅玉帶,坐轎而至。那各省參見官員,都一湧出來,又不得見了。西門慶與何千户,良久等了賁四盒擔出來,到於僻䖏,呼跟隨人拉過馬來,二人方纔騎上馬回寓。正是:不因奸佞居台鼎,那得中原血染衣!看官聽説:妾婦索家,小人亂國,自然之道。識者以為將來數賊必覆天下。果到宣和三年,徽欽北狩,高宗南遷,而天下為虜有,可深痛哉!史官意不盡,有詩為證:

\begin{myquote}
權姦誤國祸機深,開國承家戒小人。

六賊深誅何足道,奈何二聖遠蒙塵。
\end{myquote}

畢竟未知後來如何,且聽下回分解。

\part*{夢梅館校本《金瓶梅詞話》卷之八}
\addcontentsline{toc}{part}{夢梅館校本《金瓶梅詞話》卷之八}

