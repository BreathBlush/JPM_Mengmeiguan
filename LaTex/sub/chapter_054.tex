\includepdf[pages={107,108},fitpaper=false]{tst.pdf}
\chapter*{第五十四囬 \\應伯爵郊園會諸友 任醫官豪家看病症}
\addcontentsline{toc}{chapter}{第五十四囬 應伯爵郊園會諸友 任醫官豪家看病症}
\markboth{{\titlename}卷之六}{第五十四囬 應伯爵郊園會諸友 任醫官豪家看病症}


\begin{myquote}
來日陰晴未可商,常言極楽起憂惶。

浪遊年少躭紅陌,薄命嬌娥怨綠窻。

乍入杏村沽美酒,還従橘井問奇方。

人生多少悲歡事,幾度春風幾度霜。
\end{myquote}

話説西門慶在金蓮房裏起身,吩咐琴童玳安:「送猪蹄羊肉到應二爹家去。」兩個小廝正送去時,應伯爵正邀客回來,見了就進房,帶邀帶請的寫一張回字:「昨擾極。兹復承佳惠,謝謝!即刻屈吾兄過舍,同往郊外一楽。」寫完了,走出來,將交與玳安。玳安道:「别要寫字去了。爹差我們兩個在這裏伏侍,也不得去了。」應伯爵笑道:「怎好勞動你兩個親油嘴,折殺了你二爹哩!」就把回字來袖過了。玳安道:「二爹,今日在那笪兒吃酒?我們把桌子也擺擺麽,還是灰塵的哩!」伯爵道:「好人呀,正待要抹抹。先擺在家裏,吃了便飯,然後到郊園上去頑耍。」琴童道:「先在家裏吃飯,也倒有理,省得又到那裏吃飯,徑把攢盒酒小碟兒拿去罷。」伯爵道:「你兩個倒也聰明,正合二爹的粗主意。想是日夜被人鑽掘,掘開了聰明孔哩!」玳安道:「別要講閒話,就與你收拾起來。」伯爵道:「這叫做接連三個觀音堂——妙妙妙!」

兩個安童剛收拾得七八分,只見搖搖擺擺的走進門來,卻是白來創。見了伯爵,拱手,又見了琴童玳安道:「這兩個小親親,這等奉承你二爹?」伯爵道:「你莫待撚酸哩!」笑了一番。白來創道:「哥請那幾客?」伯爵道:「只是弟兄幾個坐坐,就當會茶,沒有別的新客。」白來創道:「這卻妙了!小弟極怕的,是外面沒相識的人同吃酒。今日我們弟兄輩小敍,倒也好吃酒頑耍。只是席上少不得唱的,和李銘吳惠兒彈唱彈唱,倒也好吃酒。」伯爵道:「不消吩咐,此人自然知趣。難道悶昏昏的吃了一場便罷了?你幾曾見我是恁的來?」白來創道:「停當停當!還是你老幫襯。只是停會兒,少罰我的酒。因前夜吃了火酒,吃得多了,嗓子兒怪疼的了不得,只吃些茶飯粉湯兒罷。」伯爵道:「酒病酒薬醫,就吃些何妨?我前日也有些嗓子痛,吃了幾盃酒,倒也就好了,你不如依我這方,絶妙。」白來創道:「哥,你只會醫嗓子,可會醫肚子麽?」伯爵道:「你想是没有用早飯?」白來創道:「也差不遠。」伯爵道:「怎麽處?」就跑的進去了,拿一碟子乾糕、一碟子檀香餅、一壺茶出來,與白來創吃。那白來創把檀香餅一個一口都吃盡了,讚道:「這餅卻好!」伯爵道:「糕亦頗通。」白來創就嗶嗶聲都吃了。只見琴童玳安收迭家活,一霎地明窻凈几。白來創道:「收拾恁的整齊了,只是弟兄們還未齊。早些來多頑頑也得,怎地只管縮在家裏,不知做甚的來?」

伯爵正望着外邊,只見常時節走進屋裏來。琴童正掇茶出來,常時節拱手畢,便瞧着琴童道:「是你在這裏?」琴童笑而不答。吃茶畢,三人剛立起散走,白來創看見廚上有一副棋枰,就對常時節道:「我與你下一盤棋。」常時節道:「我方走了熱剌剌的,正待打開衣帶搧搧扇子,又要下棋!也罷麽,待我胡亂下局罷。」就取下棋枰來下棋。伯爵道:「賭個東道兒罷?」白來創道:「今日擾兄了,不如着入己的,倒也徑捷些兒,省得虚脾胃,吃又吃不成。倒不如入己的有實惠!」伯爵道:「我做主人,不來。你們也着東道來凑凑麽?」笑了一番。常時節道:「如今説了,着甚麽東西,還是銀子?」白來創道:「我不帶得銀子,只有扇子在此,當得二三錢銀子起的,慢慢的贖了罷。」常時節道:「我是贏別人的絨繡汗巾在這裏,也値許多,就着了罷。」一齊交與伯爵,伯爵看看,一個是詩畫的白竹金扇,卻是舊做骨子;一個是簇新的繡汗巾。説道:「都値的,徑着了罷。」伯爵把兩件拿了,兩個就對局起來。琴童玳安見家主不在,不住的走在椅子後邊來看下棋。伯爵道:「小油嘴,有心央及你來,再與我泡一甌茶來。」琴童就對玳安暗暗裏做了一個鬼臉,走到後邊燒茶了。

卻説白來創與常時節棋子原差不多,常時節畧高些,白來創極會反悔。正着時,只見白來創一塊棋子漸漸的輸倒了。那常時節暗暗決他要悔,那白來創果然要拆幾着子。一手撇去常時節着的子,説道:「差了差了,不要這着。」常時節道:「哥子來,不好了!」伯爵奔出來道:「怎的鬧起來?」常時節道:「他下了棋,着了三四着後,又重待拆起來,不算帳。哥做個明府,那裏有這等率性的事?」白來創面色都紅了,太陽裏都是青觔綻起了,滿面涎唾的嚷道:「我也還不曾下,他又撲的一着了。我正待看個分明,他又把手來影來影去,混帳得人眼花撩亂了。那一着方纔着下,手也不曾放,又道我悔了。你断一断,怎的説我不是?」伯爵道:「這一着便將就着了,也還不叫悔,下次再莫待恁的了。」常時節道:「便罷,且容你悔了這着。後邊再不許你『白來創』我的子了。」白來創笑道:「你是『常時節』輸慣的,倒來説我!」正説話間,謝希大也到了。琴童掇茶吃了,就道:「你們自去完了棋,待我看着。」正看時,吳典恩也正走到屋裏來了。都敍過寒溫,就問:「可着甚的來?」伯爵把二物與衆人看,都道:「既是這般,須着完了。」白來創道:「九阿哥,完了罷,只管思量甚的?」常時節正在審局,吳典恩與謝希大旁賭。希大道:「九弟勝了。」吳典恩道:「他輸了,怎地倒説勝了?賭一盃酒。」常時節道:「看看區區叨勝了。」白來創臉都紅了,道:「難道這把扇子是送你的了?」常時節道:「也差不多。」於是塡完了官着,就數起來。白來創着了五塊棋頭,常時節只得兩塊。白來創又該找還常時節三個棋子,口裏道:「輸在這三着了。」連忙數自家棋子,輸了五個子。希大道:「可是我決着了。」指吳典恩道:「記你一盃酒,停會一准要吃還我。」吳典恩笑而不答。伯爵就把扇子并原繡汗巾送與常時節。常時節把汗巾原袖了,將扇子拽開賣弄,品評詩畫,衆人都笑了一番。

玳安外邊奔進來報,卻是吳銀兒與韓金釧兒兩個相牽相引,嬉笑進來了,深深的相見衆位。白來創意思還要下盤,卻被衆人笑止了。伯爵道:「罷罷,等大哥一來,用了飯,就到郊園上去。着到幾時,莫要着了!」於是琴童忙收棋子,都吃過茶。伯爵道:「大哥此時也該來了,莫待弄晏了,頑耍不來。」剛説時,西門慶來到,衣帽齊整,四個小廝跟隨。衆人都下席迎接,叙禮讓坐,兩個妓女都磕了頭。李銘吳惠都到來磕頭過了。伯爵就催琴童玳安拿上八個靠山小碟兒,盛着十香瓜茄、五方荳豉、醬油浸的花椒、釅醋滴的苔菜、一碟糖蒜、一碟糟筍乾、一碟辣菜、一碟醬的大通薑、一碟香菌,擺放停當。兩個小廝見西門慶坐地,加倍小心,比前越覺有些馬前健。伯爵見西門慶看他擺放家活,就道:「虧了他兩個,收拾了許多事,替了二爹許多力氣。」西門慶道:「恐怕也伏侍不來。」伯爵道:「忒會了些。」謝希大道:「自古道,強將手下無弱兵。畢竟經了他們,自然停當。」那兩個小廝擺完小菜,就拿上大壺酒來,不住的拿上廿碗下飯菜兒:蒜燒荔枝肉、蔥白椒料桂魚、煮的爛羊肉、燒魚、燒鷄、酥鴨、熟肚之類,説不得許多色樣。原來伯爵在各家吃轉來,都學了這些好烹庖了,所以色色俱精,無物不妙。衆人都拿起筯來,嗒嗒聲,都吃了幾大盃酒,就拿上飯來吃了。那韓金釧吃素,再不用葷,只吃小菜。伯爵道:「今日又不是初一月半,喬作衙甚的?當初有一個人,吃了一世素,死去見了閻羅王,説:『我吃了一世素,要討一個好人身。』閻王道:『那得知你吃不吃,且割開肚子驗一驗。』割開時,只見一肚子涎唾。原來平日見人吃葷,嚥在那裏的。」衆人笑得翻了。金釧道:「這樣搗鬼,是那裏來!可不怕地獄拔舌根麽?」伯爵道:「地獄裏只拔得小淫婦的舌根,道是他親嘴時會活動哩。」都笑一陣。伯爵道:「我們到郊外去一遊何如?」西門慶道:「極妙了!」衆人都說妙。伯爵就把兩個食盒,一罈酒,都央及玳安與各家人擡在河下。喚一隻小舡,一齊下了。又喚一隻空舡載人。衆人逐一上舡,就搖到南門外三十里有餘,徑到劉太監莊前。伯爵叫灣了舡,就上岸,扶了韓金釧吳銀兒兩個上岸。西門慶問道:「到那一家園上走走倒好?」應伯爵道:「就是劉太監園上也好。」西門慶道:「也罷,就是那笪也好。」衆人都到那裏,進入一處廳堂,又轉入曲廊深徑,茂林修竹,説不盡許多景致。但見:

\begin{myquote}
翠柏森森,修篁簌簌。芳草平舖青錦褥,垂楊細舞綠絲縧。曲砌重欄,萬種名花紛若綺;幽窻密牖,數聲嬌鳥弄如簧。眞同閬苑風光,不減清都景致。散淡高人,日涉之以成趣;往來遊女,每楽此而忘疲。果屬奇觀,非因過譽。
\end{myquote}

西門慶㩦了韓金釧吳銀兒手,走往各處,飽翫一番。到一木香棚下,蔭涼的緊,兩邊又有老大長的石櫈琴臺,恰好散坐的,衆人都坐了。伯爵就去教琴童,兩個舡上人,拿起酒盒菜蔬風爐器皿等上來,都放在綠蔭之下。先吃了茶,閒話起孫寡嘴祝麻子的事。常時節道:「不然,今日也在這裏。那裏説起!」西門慶道:「也是自作自受。」伯爵道:「我們坐了罷。」白來創道:「也用得着了。」於是就擺列坐了。西門慶首席坐下,兩個妓女就坐在西門慶身邊。李銘吳惠立在太湖石邊,輕撥琵琶,漫擎檀板,唱一隻曲,名曰〔水僊子〕:

\begin{myquote}
「據着俺老母情,他則待祅廟火,刮刮匝匝烈焰生;將水面上鴛鴦,忒楞楞騰生分開交頸;疎刺刺沙,鞲雕鞍撒了鎖鞓;廝琅琅湯,偸香䖏喝號提鈴;支楞楞箏,絃断了不續碧玉箏;咭叮叮噹,精甎上摔碎菱花鏡;撲通通鼕,井底墜銀瓶。」
\end{myquote}

唱畢,又移酒到水池邊,舖下毡單,都坐地了。傳盃弄盞,猜拳賽色,吃得恁地熱鬧。西門慶道:「董嬌兒那個小淫婦,怎地不來?」應伯爵道:「昨日我自去約他,他説要送一個漢子出門,約午前來的。想必此時曉得我們在這裏頑耍,他一定趕來也。」白來創道:「這都是二哥的過,怎的不約實了他來?」西門慶就向白來創耳邊説道:「我們與那花子賭了。只説過了日中董嬌兒不來,各罰主人三大碗。」白來創對應伯爵説了。伯爵道:「便罷。只是日中以前來了,要罰列位三大碗一個。」賭便一時賭了,董嬌兒那得見來?伯爵慌的只管笑。白來創與謝希大、西門慶、兩個妓女,這般這般,都定了計。西門慶假意淨手,起來吩咐玳安,教他假意嚷將進來,只説董姑娘在外來了,如此如此。玳安曉得了。停一會時,伯爵正在遲疑,只見玳安慌不迭的奔將來道:「董家姐姐來了!不知那裏尋的來。」那伯爵嚷道:「楽殺我老太婆也!我説就來的。快把酒來,各請三碗一個。」西門慶道:「若是我們贏了,要你吃你怎的就肯吃?」伯爵道:「我若輸了不肯吃,不是人了!」衆人道:「是便是了,你且去叫他進來,我們纔好吃。」伯爵道:「是了。好人口裏的言語呢!」一走出去,東西南北都看得眼花了,那得董嬌兒的魂靈?望空罵道:「賊淫婦,在二爺面上這般的拔短梯,喬作衙哩!」走進去,衆人都笑得了不的,擁住道:「如今日中過了,要吃還我們三碗一個。」伯爵道:「都是小油嘴哄我,你們倒做實了我的酒了,怎的擺佈?」西門慶不由分説,滿滿捧一碗酒,對伯爵道:「方纔説的,不吃不是人了。」伯爵接在手,謝希大接連又斟一碗來了,吃也吃不完,吳典恩又接手斟一大碗酒來了,慌得那伯爵了不的,嚷道:「不好了,嘔出來了。拿些小菜我過過便好。」白來創倒取甜東西去。伯爵道:「賊短命,不把酸的,倒把甜的來,混帳!」白來創笑道:「那一碗就是酸的來了。左右鹹酸苦辣,都待嚐到罷了。且沒慌着!」伯爵道:「精油嘴,硶誇口得好!」常時節又送一碗來了,伯爵只待奔開暫避,西門慶和兩個妓女擁住了,那裏得去?伯爵叫道:「董嬌兒,賊短命小淫婦,害得老子好苦也!」衆人都笑做一堆。那白來創又教玳安拿酒壺,滿滿斟着。玳安把酒壺嘴支入碗内一寸許多,骨都都只管篩,那裏肯住手。伯爵瞧着道:「癡客勸主人,也罷,那賊小淫婦慣打閛閛的。怎的把壺子都放在碗内了?看你一千年,我二爺也不攛掇你討老婆哩!」韓金釧吳銀兒各人斟了一碗送與應伯爵。伯爵道:「我跪了殺鷄罷!」韓金釧道:「都免禮,只請酒便了。」吳銀兒道:「怎的不向董家姐姐殺鷄,求他來了?」伯爵道:「休見笑了,也够吃了。」兩個一齊推酒到嘴邊,伯爵不好接一頭,兩手各接了一碗,就吃完了。連忙吃了些小菜,一時面都通紅了。叫道:「我被你們弄了。酒便慢慢吃還好,怎的灌得悶不轉的!」衆人只待斟酒。伯爵跪着西門慶道:「還求大哥説個方便,饒恕小人窮性命,還要留他陪客。若一醉了,便不知天好日暗,一些興子也沒有了。」西門慶道:「便罷,這兩碗一個,你且欠着,停斟了罷。」伯爵就起來謝道:「一發蠲免了罷,足見大恩!」西門慶道:「也罷,就恕了你。只是方纔説我們不吃不是個人。如今你漸有些没人氣了!」伯爵道:「我倒灌醉了。那淫婦不知那裏歪斯纏去了!」吳銀兒笑伯爵道:「咳,怎的大老官人在這裏做東道頑耍,董嬌姐也不來來?」伯爵假意道:「他是上檯盤的名妓,倒是難請的。」韓金釧兒道:「他是趕勢利去了。成甚的行貨,叫他是名妓!」伯爵道:「我曉得,你想必有些吃醋的宿帳哩!」西門慶認是蔡公子那夜的故事,把金釧一看,不在話下。

那時伯爵已是醉醺醺的。兩個妓女又不是耐靜的,只管調唇弄舌,一句來一句去歪斯纏,倒吃得冷淡了。白來創對金釧道:「你兩個唱個曲兒麽?」吳銀兒道:「也使得。」讓金釧先唱。常時節道:「我勝那白阿弟的扇子,倒是板骨的,倒也好打板。」金釧道:「借來打一打板。」接去看看道:「我倒少這把打板的扇子。不如作我贏的棋子,送與我罷。」西門慶道:「這倒好。」常時節吃衆人攛掇不過,只得送與他了。金釧道:「吳銀姐在這裏,我怎的好獨要?我與你猜色,那個色大的拿了罷。」常時節道:「這卻有理。」就猜一色,是吳銀兒贏了。金釧就遞與銀兒了。常時節假冠冕道:「這怎麽處?我還有一條汗巾,送與金釧姐,補了扇罷。」遂送過去。金釧接了道:「這卻撒漫了。」西門慶道:「我可惜不曾帶得好川扇兒來,也賣富賣富。」常時節道:「這是打我一下了。」那謝希大驀地嚷起來道:「我幾乎忘了!又是説起扇子來!」敎玳安斟了一大盃酒,送與吳典恩道:「請完了旁賭的酒。」吳典恩道:「這罷了。停了幾時纔想出來,他們的東西都花費了,那在一盃酒?」被謝希大逼勒不過,只得呷完了。那時金釧就唱一曲,名喚〔荼䕷香〕:

\begin{myquote}
「記得初相守,偶爾間因循成就,美滿效綢繆。花朝月亱同宴賞,佳節須酧,到今日一旦休。常言道好事天慳,美姻緣他娘間阻,生拆散鸞交鳳友。

坐想行思,傷懷感舊。辜負了星前月下深深咒。願不損,愁不煞,神天還佑。他有日不測相逢,話別離情取一塲消瘦。」
\end{myquote}

唱畢,吳銀兒接唱一曲,名〔青杏兒〕:

\begin{myquote}
「風雨替花愁,風雨過花也應休。勸君莫惜花前醉,今朝花謝,明朝花謝,白了人頭。

乘興兩三甌。揀溪山好䖏追遊。但教有酒身無事,有花也好,無花也好,選甚春秋?」
\end{myquote}

唱畢,李銘吳惠排立,謝希大道:「還有這些伎藝不曾做哩。」只見彈的彈,吹的吹,琵琶簫管,又唱一隻〔小梁州〕:

\begin{myquote}
「門外紅塵滚滚飛。飛不到魚鳥清溪,緑陰高柳聽黃鸝,幽棲意,料俗客幾人知。山林本是終焉計,用之行舍之藏兮。悼後世,追前輩:五月五日,歌楚些,弔湘纍。」
\end{myquote}

唱畢,酒興將闌。那白來創尋見園廳上架着一面小小花框羯鼓,被他馱在湖山石後,又折一枝花來,要催花擊鼓。西門慶呌李銘吳惠擊鼓。一個眼色,他兩個就曉得了,従石孔内瞧着,到會吃的面前,鼓就住了。白來創道:「畢竟賊油嘴有些作獘!我自去打鼓。」也弄西門慶吃了幾盃。正吃得熱鬧,只見書童搶進來,到西門慶身邊,附耳低言道:「六娘身上不好的緊,快請爹回來。馬也備在門外接了。」西門慶聽得,連忙走起告辭。那時酒都有了,衆人都起身。伯爵道:「哥,今日不曾奉酒,怎的好去?是這些耳報法,極不好。」便待㽞住。西門慶以實情告訴他,就謝了上馬來。伯爵又留衆人。一個韓金釧霎眼挫不見了,伯爵躡足潛蹤尋去,只見在湖山石下撒尿,露出一條紅線,抛卻萬顆明珠。伯爵在隔籬笆眼,把草戲他的牝口。韓金釧撒也撒不完,吃了一驚,就立起,褌腰都濕了。罵道:「硶短命,恁尖酸的沒槽道!」面都紅了,帶笑帶罵出來。伯爵與衆人説知,又笑了一番。西門慶原留琴童與伯爵收拾家活。琴童收拾風爐食具下舡,都進城了。衆人謝了伯爵,各散去訖。伯爵打發兩隻舡錢,琴童送進家活,伯爵就打發琴童吃酒。都不在話下。

卻説西門慶來家,兩步做一步走,一直走進六娘房裏。迎春道:「俺娘了不得病,爹快看看他。」走到床邊,只見李瓶兒咿嚶的呌疼,卻是胃脘作疼。西門慶聽他呌得苦楚,連忙道:「快去請任醫官來看你。」就叫迎春:「喚書童寫帖,去請任太醫。」迎春出去説了,書童隨寫侍生帖去請任太醫了。西門慶擁了李瓶兒坐在床上,李瓶兒道:「恁的酒氣!」西門慶道:「是胃虚了,便厭着酒氣。」又對迎春道:「可曾吃些粥湯?」迎春囬道:「今早至今,一粒米也沒有用,只吃了兩三甌湯兒。心口肚腹兩腰子,都疼得異樣的。」西門慶攢着眉,皺着眼,嘆了幾口氣。又問如意兒:「官哥身子好了麽?」如意兒道:「昨夜還有頭熱,還要哭哩!」西門慶道:「恁的悔氣!娘兒兩個都病了,怎的好?留得娘的精神,還好去支持孩子哩!」李瓶兒又呌疼起來了。西門慶道:「且耐心着,太醫也就來了。待他看過脉,吃兩鍾薬,就好了的。」迎春打掃房裏,抹淨桌椅,燒香點茶。又支持奶子,引鬬得官哥睡着。此時有更次了,外邊狗叫得不迭,卻是琴童歸來。不一時,書童掌了燈照着,任太醫四角方巾,大袖衣服,騎馬來了。進門坐在軒下。書童走進來説:「請了來了,坐在軒下了。」西門慶道:「好了,快拿茶出去。」玳安即便掇茶,跟西門慶出去迎接任太醫。太醫道:「不知尊府那一位看脉?失候了,負罪實多!」西門慶道:「昏夜勞動,心切不安。萬惟垂諒!」太醫着地打躬道:「不敢!」吃了一鍾燻荳子撒的茶,就問:「看那一位尊恙?」西門慶道:「是第六個小妾。」又換一鍾鹹櫻桃的茶,說了幾句閒話。玳安接鍾,西門慶道:「裏面可曾收拾?你進去話聲,掌燈出來照進去。」玳安進到房裏去話了一聲,就掌燈出來囬報。

西門慶就起身打躬,邀太醫進房。太醫遇着一個門口,或是階頭上,或是轉彎去處,就打一個半喏的躬,渾身恭敬,滿口寒温。走進房裏,只見沉煙繞金鼎,蘭火爇銀缸。錦帳重圍,玉鉤齊下。眞是繁華深處,果然別一洞天。西門慶看了太醫的椅子,太醫道:「不消了。」也答看了西門慶椅子,就坐下了。迎春便把繡褥來襯起李瓶兒的手,又把錦帕來擁了玉臂,又把自己袖口籠着他纖指,従帳底下露出一段粉白的臂來,與太醫看脉。太醫澄心定氣,候得脉來,卻是胃虚氣弱,血少肝經旺,心境不清,火在三焦,須要降火滋榮。就依書據理,與西門慶説了。西門慶道:「先生,果然如見,實是這樣的。這個小妾,性子極忍耐得。」太醫道:「正為這個緣故,所以他肝經原旺,人卻不知他。如今木尅了土,胃氣自弱了。氣那裏得滿?血那裏得生?水不能載火,火都生上截來,胸膈作飽作疼,肚子也時常作疼。血虚了,兩腰子渾身骨節裏頭,通作酸痛,飲食也吃不下了。可是這等的?」迎春道:「正是這樣的。」西門慶道:「眞正任僊人了!貴道裏望聞問切,如先生這樣明白脉理,不消問的,只管説出來了,也是小妾有幸!」太醫深打躬道:「晚生曉得甚的?只是猜多了。」西門慶道:「太謙遜了些。」又問:「如今小妾該用什麽薬?」太醫道:「只是降火滋榮,火降了,這胸膈自然寬泰;血足了,腰脅自然不作疼了。不要認是外感,一些也不是的,都是不足之症。」又問道:「經事來得匀麽?」迎春道:「便是不得准。」太醫道:「幾時便來一次?」迎春道:「自従養了官哥,還不見十分來。」太醫道:「元氣原弱,産後失調,遂致血虚了——不是壅積了,要用疏通薬。要逐漸吃些丸薬,養他轉來才好。不然,就要做癆病了。」西門慶道:「便是,極看得明白。如今先求煎劑,救得目前痛苦。還要求些丸薬。」太醫道:「當淂。晚生返舍,即便送來。沒事的。只要知此症乃不足之症:其胸膈作痛,乃火痛,非外感也;其腰脅怪疼,乃血虚,非血滯也。吃了薬去,自然逐一好起來,不須焦躁得。」西門慶謝不絕口。剛起身出房,官哥又醒覺了,哭起來。太醫道:「這位公子好聲音。」西門慶道:「便是也會生病,不好得緊。連累小妾日夜不得安枕。」一路送出來了。

卻説書童對琴童道:「我方纔去請他,他已早睡了。敲得半日門,纔有人出來。那老子一路揉眼出來,上了馬,還打盹不住,我只愁突了下來。」琴童道:「你是苦差使。我今日遊翫得了不的,又吃了一肚子酒。」正在閒話,玳安掌燈,跟西門慶送出太醫來。到軒下,太醫只管走。西門慶道:「請寬坐,再奉一茶,還要便飯點心。」太醫搖頭道:「多謝盛情,不敢領了。」一直走到出來。西門慶送上馬,就差書童掌燈送去。別了太醫,飛的進去。教玳安拿一兩銀子,趕上隨去討薬。直到任太醫家,太醫下了馬,對他兩個道:「阿叔們,且坐着吃茶,我去拿薬出來。」玳安拿禮盒送與太醫道:「薬金請收了。」太醫道:「我們是相知朋友,不敢受你老爺的禮。」書童道:「定求收了,纔好領薬。不然,我們薬也不好拿去。恐怕回家去,一定又要送來,空走腳步。不如作速收了,候的薬去便好。」玳安道:「無錢課不靈,定求收了。」太醫只得收了。見薬金盛了,就進去簇起煎劑,連瓶内丸子薬,也倒了淺半瓶。兩個小廝吃茶畢,裏面打發回帖出來與玳安書童,徑閉了門。

兩個小廝囬家。西門慶見了薬袋厚大的,説道:「怎地許多!」拆開看時,卻是丸薬也在裏面了。笑道:「有錢能使鬼推磨。方纔他説先送煎薬,如今都送了來!也好,也好。」看薬袋上是冩着:「降火滋榮湯。水二鍾,姜不用,煎至捌分,食遠服,渣再煎。忌食麩麪油膩炙煿等物。」又打上「世醫任氏薬室」的印記。又一封筒,大紅票簽,寫着「加味地黄丸」。西門慶把薬交迎春,先吩咐煎一帖起來。李瓶兒又吃了些湯。迎春把薬熬了,西門慶自家看薬瀘清了渣出來。捧到李瓶兒床前,道:「六娘,薬在此了。」李瓶兒翻身轉來,不勝嬌顫。西門慶一手拿薬,一手扶着他頭頸,李瓶兒吃了呌苦,迎春就拿滚水來過了口。西門慶吃了粥,洗了足,就伴李瓶兒睡了。迎春又燒些熱湯護着,也連衣服假睡了。説也奇怪,吃了這薬,就有睡了。西門慶也熟睡去了。官哥只管要哭起來,如意兒恐怕哭醒了李瓶兒,把奶子來敎他吃,後邊也寂寂的睡了。

到次早,西門慶將起身,問李瓶兒:「昨夜覺好些兒麽?」李瓶兒道:「可霎作怪!吃了薬,不知怎地睡的熟了。今早心腹裏都覺不十分怪疼了。學了昨的下半晚,眞要痛死人也!」西門慶笑道:「謝天謝天!如今再煎他二鍾吃了,就全好了。」迎春就煎起第二鍾來,吃了。西門慶一個驚魂,落向爪哇國去了。怎見得?有詩為證:

\begin{myquote}
西施時把翠蛾顰,幸有僊丹妙入神;

信是薬醫不死病,果然佛度有緣人。
\end{myquote}

畢竟未知如何,且聽下囬分解。

