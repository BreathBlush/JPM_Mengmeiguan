\includepdf[pages={171,172},fitpaper=false]{tst.pdf}
\chapter*{第八十六囬 \\雪娥唆打陳經濟 王婆售利嫁金蓮}
\addcontentsline{toc}{chapter}{第八十六囬 雪娥唆打陳經濟 王婆售利嫁金蓮}
\markboth{{\titlename}卷之九}{第八十六囬 雪娥唆打陳經濟 王婆售利嫁金蓮}


\begin{myquote}
人生雖未有十全,處事規模要放寬。

好歹但看君子語,是非休聽小人言。

但看世俗如幻戯,也畏人心似隔山。

寄與知音女娘道,莫將苦處認為甜。
\end{myquote}

話説潘金蓮自従春梅出去,房中納悶,不題。單表陳經濟,次日早飯時出去,假作討帳,騎頭口到於薛嫂兒家。薛嫂兒正在屋裏,一面讓進來坐。經濟拴了頭口,進房坐下,點茶喫了。春梅在裏間屋裏不出來。薛嫂故意問:「姐夫來有何話說?」經濟道:「我往前街討帳,徑到這裏。昨晚小大姐出來了,在你這裏?」薛嫂道:「是在我這裏,還未上主兒哩。」經濟道:「在這裏,我要見他,和他說句話兒。」薛嫂故作喬張致說:「好姐夫,昨日你家丈母好不吩咐我,因為你們通同作弊,弄出醜事來,纔被他打發出門,教我防範你們,休要與他會面說話。你還不趁早去哩,只怕他一時使將小廝來看見,到家學了,又是一場兒,倒沒的弄的我也上不的門!」那經濟便笑嘻嘻袖中拿出一兩銀子來:「權作一茶,你且收了,改日還謝你。」那薛嫂見錢眼開,說道:「好姐夫,自恁沒錢使,將來謝我?只是我去年臘月,你舖子當了人家兩付扣花枕頂,將有一年來,本利該八錢銀子,你討與我罷。」經濟道:「這個不打緊,明日就尋與你。」這薛嫂兒一面請經濟裏間房裏去與春梅廝見,一面呌他媳婦金大姐定菜兒:「我去買茶食點心。」又打了一壺酒,並肉鮓之類,敎他二人喫。

這春梅看見經濟,說道:「姐夫,你好人兒,就是個弄人的劊子手!把俺娘兒兩個,弄的上不上下不下,出醜惹人嫌到這步田地!」經濟道:「我的姐姐,你旣出了他家門,我在他家也不久了。妻兒趙迎春——各自尋投奔。你教薛媽替你尋個好人家去罷,我醃韮已是入不的畦了。我往東京俺父親那裏去計較了囘來,把他家女兒休了,只要我家寄放的箱子。」說畢,不一時,薛嫂買將茶食酒菜來,放炕桌兒擺了,兩個做一處飲酒叙話。薛嫂也陪他喫了兩盞。一遞一句,說了囘月娘心狠:「宅裏恁個出色姐兒出來,通不與一件兒衣服簪環!就是往人家上主兒去,裝門面也不好看。還要舊時原價,就是清水,這碗裏傾倒那碗内,也抛撒些兒。原來這等夾腦風!臨時出門,倒虧了小玉丫頭做了個分上,敎他娘拿了兩件衣服與他。不是,往人家相去,拿甚麽做上蓋?」比及喫得酒濃時,薛嫂教他媳婦金大姐,抱孩子躲去人家坐的,教他兩個在裏間自在坐個房兒。正是:

\begin{myquote}
雲淡淡天邊鸞鳳,水沉沉波底鴛鴦。

寫成今世不休書,結下來生歡喜帶。
\end{myquote}

兩個幹訖一度,作别之時,難割難捨。薛嫂恐怕月娘使人來瞧,連忙攛掇經濟出港,騎上頭口來家。

遲不上兩日,經濟又捎了兩方銷金汗巾,兩雙膝褲與春梅;又尋枕頂出來與薛嫂兒。拿銀子打酒,在薛嫂兒房内正和春梅喫酒。不想月娘使了來安小廝來催薛嫂兒:「怎的還不上主兒?」看見頭口栓在門首,來安兒到家學了舌,說:「姐夫也在那裏來!」這月娘聽了,心中大怒,使人一替兩替,叫了薛嫂兒去,儘力數說了一頓:「你領了奴才去,今日推明日,明日推後日,只顧不上緊替我打發,好窝藏着養漢,掙錢兒與你家使!若是你不打發,把丫頭還與我領了來,我另敎馮媽媽子賣,你再休上我門來!」這薛嫂兒聽了,到底還是媒人的嘴,恨不的生出七八個口來,說道:「天麽,天麽!你老人家怪我差了,我赶着增福神着棍打?你老人家照顧我,怎不打發?昨日也領着走了兩三個主兒,都出不上你老人家要十六兩原價,俺媒人家,那裏有這些銀子賠上!」月娘又道:「小廝說,陳家種子今日在你家和丫頭喫酒來!」薛嫂慌道:「耶嚛耶嚛!又是一場兒!還是去年臘月,當了人家兩付枕頂在咱家獅子街舖内,銀子收了,今日姐夫送枕頂與我,我讓他喫茶,他不喫,忙忙就上頭口來了。幾時進屋裏喫酒來?原來咱家這大官兒,恁快搗謊駕舌!」月娘喫他一篇說的不言語了,說道:「我只怕一時被那種子設念隨邪,差了念頭。」薛嫂道:「我是三歲小孩兒,豈可恁些事兒不知道?你那等吩咐了我,我長喫好,短喫好?他在那裏,也没得久停久坐,與了我枕頂,茶也沒喫就來了,幾曾見咱家小大姐面兒來?萬物也要個眞實,你老人家就上落我起來!旣是如此,如今守備周爺府中,要他圖生長,只出十二兩銀子,看他若添到十三兩上,我兑了銀子來罷。說起來,守備老爺前者在咱家酒席上也曾見過小大姐來。因他會這幾套唱,好模樣兒,纔出這幾兩銀子。又不是女兒,其餘别人出不上。」

這薛嫂當下和月娘砸死了價錢,次日早,把春梅收拾打扮,粧點起來,戴着圍髮雲髻兒,滿頭珠翠,穿上紅緞襖兒,下着藍緞裙子,脚上雙彎尖趫趫,一頂轎子送到守備府中。周守備見了春梅,生的比舊時越發標緻,模樣兒又紅又白,身段兒不短不長,一對小脚兒,滿心歡喜,就兑出五十兩一錠元寳來。這薛嫂兒拿來家,鑿下十三兩銀子,往西門慶家交與月娘,另外又拿出一兩來說:「是周爺賞我的喜錢。你老人家這邊不與我些兒?」那吴月娘免不過,只得又稱出五錢銀子與他。恰好他還禁了三十七兩五錢銀子!十個九個媒人,都是如此赚錢養家。

却表陳經濟見賣了春梅,又不得往金蓮那邊去;見月娘凡事不理他,門户都嚴緊,到晚夕親自出來打燈籠前後照看了,方纔關後邊儀門,夜裏上鎖方纔睡去,因此弄不得手脚。十分急了,先和西門大姐嚷了兩場,淫婦前淫婦後罵大姐:「我在你家做女婿,不道的雌飯喫喫傷了!你家都收了我許多金銀箱籠,你是我老婆,不顧贍我,反說我雌你家飯喫!我白喫你家飯來?」罵的大姐只是哭涕。

十一月廿七日,孟玉樓生日。玉樓安排了幾碟酒菜點心,好意教春鴻拿出前邊舖子,教經濟陪傅夥計喫。月娘便攔説:「他不是材料,休要理他。要與傅夥計,只與傅夥計自家喫就是了,不消呌他。」玉樓不肯。春鴻拿出來,擺在水櫃上。一大壺酒都喫了,不够,又使來安兒後邊要去。傅夥計便說:「姐夫,不消要酒去了,這酒夠了。我也不喫了。」經濟不肯,定敎來安要去。等了半晌,來安兒出來,囘說沒了酒了。這陳經濟也有半酣,酒兒在肚内,又使他要去,那來安不動。經濟又另拿錢打了酒來,喫着罵來安兒:「賊小奴才兒,你别要慌!你主子不待見我,連你這奴才們也欺負我起來了,使你使兒不動。我與你家做女婿,不道的酒肉喫傷了!有爹在,怎麽行來?今日爹沒了,就改變了心腸,把我來不理,都亂來擠撮我。我大丈母聽信奴才言語,反防範我起來。凡事托奴才,不托我。由他,我好耐驚耐怕兒!」傅夥計勸道:「好姐夫,快休舒言。不敬奉姐夫,再敬奉誰?想必後邊忙,怎不與姐夫喫?你罵他不打緊,牆有縫,壁有耳,恰似你醉了一般。」經濟道:「老夥計,你不知道,我酒在肚裏,事在心頭。俺丈母聽信小人言語,架我一篇是非,就算我㒲了人,人沒㒲了我?好不好,我把這一屋子裏老婆都刮剌了,到官也只是後丈母通奸,論個不應罪名。如今我先把你家女兒休了,然後一紙狀子告到官!再不,東京萬壽門進一本:你家現收着我家許多金銀箱籠,都是楊戩應沒官贜物!好不好,把你這幾間業房子都抄沒了,老婆便當官變賣。我不圖打魚,只圖混水耍子。會事的,把俺女婿須收籠着,照舊看待,還是大鳥便益!」傅夥計見他話頭兒來的不好,說道:「姐夫,你原來醉了。王十九,自喫酒,且把散話擱起。」這經濟睜眼瞅着傅夥計便罵:「賊老狗,怎的說我散話擱起,我醉了,喫了你家酒了?我不才,是他家女婿嬌客,你無過只是他家行財,你也擠撮我起來?我敎你這老狗别要慌,你這幾年赚的俺丈人錢夠了,飯也喫飽了,心裏要打夥兒把我疾發了去,要獨權兒做買賣,好禁錢養家。我明日本狀也帶你一筆,教你打官司!」那傅夥計最是個小膽兒的人,見頭勢不好,穿上衣裳,悄悄往家一溜煙走了。小廝收了家活,後邊去了。經濟倒在炕上睡下,一宿晚景題過。

次日,傅夥計早晨進後邊,見月娘把前事具訴一遍,哭哭啼啼,要告辭家去,交割帳目,不做買賣了。月娘便勸道:「夥計,你只安心做買賣,休要理那潑材料,如臭屎一般丢着他!當初你家為官事,投到俺家來權住着,有甚金銀財寳?也只是大姐幾件粧奩,隨身箱籠。你家老子便躲上東京去了,教俺家那一個不恐怕小人不足,晝夜躭憂的那心!你來時纔十六七歲,黄毛團兒也一般,也虧在丈人家養活了這幾年,調理的諸般買賣兒都會。今日翅膀毛兒乾了,反恩將仇報,一掃箒掃的光光的。小孩兒家說話欺心,恁没天理,到明日只天照看他!夥計,你自安心做你買賣,休理他便了,他自然也羞。」一面把傅夥計安撫住了,不題。

一日,也是合當有事。印子鋪擠着一屋裏人,贖討東西。只見奶子如意兒,抱着孝哥兒,送了一壺茶來與傅夥計喫,放在桌上。孝哥兒在奶子懷裏,哇哇的只管哭。這陳經濟對着那些人,作耍當真說道:「我的哥哥,乖乖兒,你休哭了!」向衆人説:「這孩子倒像我養的,依我說話。敎他休哭,他就不哭了!」那些人就獃了。如意兒說:「姐夫,你說的好妙話兒,越發叫起兒來了,看我進房裏說不說!」這陳經濟趕上踢了奶子兩脚,戲駡道:「怪賊邋遢,你説不是?我且踢個響屁股兒着。」那奶子抱孩子走到後邊,如此這般向月娘哭說:「經濟對衆人,將哥兒這般言話發出來!」這月娘不聽便罷,聽了此言,正在鏡臺邊梳着頭,半日說不出話來,往前一撞,就昏倒在地,不省人事。但見:

\begin{myquote}
荆山玉損,可惜西門慶正室夫妻;寳鑑花殘,枉費九十日東君匹配。花容淹淡,猶如西園芍薬倚朱欄;檀口無言,一似南海觀音來入定。小園昨日春風急,吹折江梅就地横。
\end{myquote}

慌了小玉,呌將家中大小,扶起月娘來炕上坐的。孫雪娥跳上炕,獗救了半日,舀姜湯灌下去,半日甦醒過來。月娘氣堵心胸,只是哽咽,哭不出聲來。奶子如意兒對孟玉樓孫雪娥將經濟對衆人將哥兒戲言之事,說了一遍:「我好意說他,又趕着我踢了兩脚,把我也氣的發昏在這裏!」

雪娥扶着月娘,待的衆人散去,悄悄在房中對月娘說:「娘也不消生氣,氣的你有些好歹,越發不好了。這小廝因賣了春梅,不得與潘家那淫婦弄手脚,纔發出話來。如今一不做,二不休,大姐已是嫁出女,如同賣出田一般,咱顧不的他這許多。常言養蝦蟆得水蠱兒病,只顧教這小廝在家裏做甚麽?明日哄賺進後邊,老實打與他一頓,即時趕離門,敎他家去。然後呌將王媽媽子,來是是非人,去是是非者,把那淫婦敎他領了去,變賣嫁人,如同狗屎臭尿,掠將出去,一天事都没了。平空留着他在屋裏做甚麽?到明日,沒的把咱們也扯下水去了!」月娘道:「你説的也是。」當下計議已定了。

到次日飯時已後,月娘埋伏下丫鬟媳婦七八個人,各拿短棍棒槌,使小廝來安兒誆進陳經濟來後邊,只推說話。把儀門關了,教他當面跪着,問他:「你知罪麽?」那陳經濟也不跪,還似每常臉兒高揚。月娘便道,有長詞為證:

\begin{myquote}
起初時,月娘不犯觸,龐兒變了。次則陳經濟耐搶白,臉兒揚着:「不消你枉話兒絮叨叨,須和你討個分曉。」月娘道:「此是你丈人深宅院,又不是麗春院鶯燕巢,你如何把他婦女廝調?他是你丈人愛妾,寡居守孝。你因何把他戯嘲?也有那沒廉耻斜皮,把你刮剌上了。自古母狗不掉尾,公狗不跳槽。都是些污家門罪犯難饒!」陳經濟道:「閃出夥縛鍾馗女妖,你做成這慣打姦夫的圈套,我臀尖難禁這頓拷。梅香休鬧,大娘休焦,險些不大棍無情打折我腰!」月娘道:「賊材料,你還敢嘴兒挑!常言冰厚三尺不是一日惱,最恨無端難恕饒。虧你呵,再躺着筒兒蒲棒剪稻。你再敢不敢?我把你這短命王鸞兒割了,敎你直孤到老!」
\end{myquote}

當下月娘率領雪娥,並來興兒媳婦、來昭妻一丈青、中秋兒、小玉、綉春,衆婦人七手八脚,按下地下,拿棒槌短棍,打了一頓。西門大姐走過一邊,也不來救。打的這小夥兒急了,把褲子脱了,露出那直豎一條棍來。唬的衆婦女看見,都丢下棍棒亂跑了。月娘又是那惱,又是那笑,口裏駡道:「好個没根基的王八羔子!」經濟口中不言,心中暗道:「若不是我這個好法兒,怎得脱身!」於是爬起來,一手兜着褲子,往前走了。月娘隨令小廝跟隨,敎他算帳,交與傅夥計。經濟自然也存立不住,一面收拾衣服鋪蓋,也不作辭,使性兒一直出離西門慶家,逕往他母舅張團練住的他舊房子内住去了。正是:自古感恩並積恨,萬年千載不成塵。

潘金蓮在房中,聽見打了經濟,趕離出門去了,越發憂上加憂,悶上添悶。

一日,月娘聽信雪娥之言,使玳安去呌王婆子來。那王婆,自従他兒子王潮兒跟淮上客人,拐了起車的一佰兩銀子來家,得其發跡,也不賣茶了,買了兩個驢兒,安了盤磨,一張羅櫃,開起磨房來。聽見西門慶宅裏呌他,連忙穿衣就走。到路上問玳安說:「我的哥哥,幾時没見你,又早籠起頭去了。有了媳婦兒不曾?」玳安道:「還不曾有哩。」王婆子道:「你爹沒了,你家誰人請我?做甚麽?莫不是你五娘養了兒子了,請我去抱腰?」玳安道:「俺五娘倒沒養兒子,倒養了女婿。俺大娘請你老人家,領他出來嫁人。」王婆子道:「天麽天麽,你看麽!我説這淫婦,死了你爹,怎守得住。只當狗改不了喫屎,就弄硶兒來了。就是你家大姐那女婿子?他姓甚麽?」玳安道:「他姓陳,名喚陳經濟。」王婆子道:「想着去年,我為何老九的事去央煩你爹。到宅内,你爹不在,賊淫婦他就没留我房裏坐坐兒,折針也迸不出個來!只呌丫頭倒了一鍾清茶,我喫了出來了。我只道千年萬歲在他家,如何今日也還出來?好個狼家子淫婦!休說我是你個媒主,替你作成了恁好人家,就是世人進去,也不該那等大意。」玳安道:「為他和俺姐夫在家裏毆作嚷亂,昨日差些兒没把俺大娘氣殺了哩!俺姐夫已是打發出去了,只有他老人家,如今敎你領他去哩。」王婆子道:「他原是轎兒來,少不得還呌頂轎子。他也有個箱籠來,這裏少不的也與他個箱子兒。」玳安道:「這個少不的,俺大娘他有個處。」

兩個説話中間,到於西門慶門首。進入月娘房裏,道了萬福,坐下,丫鬟拿茶喫了。月娘便道:「老王,無事不請你來。」悉把潘金蓮如此這般,上項說了一遍:「今來是是非人,去是是非者,一客不煩二主,還起動你領他出去,或聘嫁,或打發,敎他喫自在飯去罷。我男子漢已是没了,招攬不過這些人來。說不的當初死鬼為他丢了許多錢底那話了,就打他恁個銀人兒也有。如今隨你聘嫁多少兒,交得來,我替他爹念個經兒,也是一場勾當。」王婆道:「你老人家是稀罕這錢的?只要把祸害離了門,就是了。我知道,我也不肯差了。」又道:「今日好日,就出去罷。又一件,他當初有個箱籠兒,有頂轎兒來,也少不的與他頂轎兒坐了去。」月娘道:「箱子與他一個,轎子不容他坐。」小玉道:「俺奶奶氣頭上便是這等說,到臨岐少不的僱頂轎兒。不然,街坊人家看着,抛頭露面的,不乞人笑話!」月娘不言語了。一面使丫鬟綉春前邊呌金蓮來。

這金蓮一見王婆子在房裏,就睜了,向前道了萬福,坐下。王婆子開言便道:「你快收拾了,剛纔大娘說,敎我今日領你出去哩。」金蓮道:「我漢子死了多少時兒,我為下甚麽非,作下甚麽歹來?如何憑空打發我出去?」月娘道:「你休稀裏打哄,做啞裝聾!自古蛇鑽窟嚨蛇知道,各人幹的事兒各人心裏明。金蓮,你休獃裏撒奸,兩頭白面,說長並道短,我手裏使不的你巧語花言,幫閑鑽懶!自古没個不散的筵席,出頭椽兒先朽爛。人的名兒,樹的影的,蒼蝇不鑽沒縫兒彈。你休把養漢當飯,我如今要打發你上陽關!」金蓮道:「你打人休打臉,罵人休揭短!常言一鷄死了一鷄鳴。誰打籮,誰喫飯,誰人常把鐵箍子戴,那個長將席篾兒支着眼?為人還有相逢處,樹葉兒落還到根邊。你休要把人赤手空拳往外攆,是非莫聽小人言!」正是:女人不穿嫁時衣,男兒不喫分時飯,自有徒勞話歲寒。

當下金蓮與月娘亂了一囘,月娘到他房中,打點與了他兩個箱子,一張抽替桌兒,四套衣服,幾件釵梳簪環,一牀被褥,其餘他穿的鞋脚,都填在箱内。把秋菊呌得後邊來,一把鎖把他房門鎖了。金蓮穿上衣服,拜辭月娘,在西門慶靈前大哭了一場。又走到孟玉樓房中,也是姊妹相處了一場,一旦分離,兩個落了一囘眼淚。玉樓悄瞞着月娘,與了他一對金碗簪子,一套翠藍緞襖紅裙子,說道:「六姐,奴與你離多會少了!你看個好人家,往前進了罷。自古道:千裏長蓬,也沒個不散的筵席。你若有了人家,使人來對奴說聲,奴往那裏去,順便到你那裏看你去,也是姊妹情腸。」於是洒淚而别。臨出門,小玉送金蓮,悄悄與了金蓮兩根金頭簪兒。金蓮道:「我的姐姐,你倒有一點人心兒在我上!」轎子在大門首,王婆又早僱人把箱籠桌子,擡的先去了。獨有玉樓小玉送金蓮到門首,坐上轎子纔囘。正是:世上萬般哀苦事,除非死別共生離。

却說金蓮到王婆家,王婆安插他在裏間,晚夕同他一處睡。他兒子王潮兒,也長成一條大漢,籠起頭去了,還未有妻室,外間支着牀子睡。這潘金蓮次日依舊打扮喬眉喬眼,在簾下看人。無事坐在炕上,不是描眉畫眼,就是彈弄琵琶。王婆不在,就和王潮兒鬦葉兒下棋。那王婆自去掃麪喂養驢子,不去管他。朝來暮去,又把王潮兒刮剌上了。晚間等的王婆子睡着了,婦人推下炕溺尿,走出外間牀子上,和王潮兒兩個幹。搖的牀子一片響聲,被王婆子醒來聽見,問:「那裏響?」王潮兒道:「是櫃底下貓捕的老鼠響。」王婆子睡夢中,喃喃呐呐,口裏說道:「只因有這些麩麪在屋裏,引的這扎心的半夜三更耗爆人,不得睡。」良久,又聽見動彈,搖的牀子格支支響。王婆又問:「那裏響?」王潮道:「是貓咬老鼠,鑽在炕洞底下嚼的響。」婆子側耳,果然聽見貓在炕洞裏狼虎,方纔不言語了。婦人和小廝幹完事,依舊悄悄上炕睡去了。有幾句雙關,說得這老鼠好:

\begin{myquote}
你身驅兒小膽兒大,嘴兒尖忒潑皮。見了人藏藏躲躲,耳邊廂呌呌唧唧,攪混人半夜三更不睡。不行正人倫,徧好鑽穴隙。更有一樁兒不老實,到底改不了偸饞抹嘴。
\end{myquote}

有日,陳經濟打聽得金蓮出來,還在王婆子家聘嫁,帶着銀錢,走到王婆子家來。婆子正在門前掃驢子撒下的糞。這經濟向前,深深地唱個喏。婆子問道:「哥哥你做甚麽?」經濟道:「請借裏邊説話。」王婆便讓進裏面,經濟揭起眼紗,便道:「動問西門大官人宅内有一位娘子潘六姐,在此出嫁?」王婆便道:「你是他甚麽人?」那經濟嘻嘻笑道:「不瞞你老人家說,我是他兄弟,他是我姐姐。」那王婆子眼上眼下打量他一囬說:「他有甚兄弟,我不知道?你休哄我,你莫不是他家女婿姓陳的,來此處撞蠓子?我老娘手裏放不過!」經濟笑向腰裏解下兩弔銅錢來,放在面前,說:「這兩弔錢,權作王奶奶一茶之費。敎我且見一面,改日還重謝你老人家!」

婆子見錢,越發喬張致起來,便道:「休說謝的話!他家大娘子吩咐將來,不敎閑雜人來看他。咱放倒身說話:你既要見這雌兒一面,與我五兩銀子;見兩面,與我十兩。你若娶他,便與我一百兩銀子,我的十兩媒人錢在外。我不管閑帳。你如今兩串錢兒,打水不渾的做甚麽?」經濟見這虔婆口硬不收錢,又向頭上拔下一對金頭銀脚簪子,重五錢,殺鷄扯腿跪在地下,說道:「王奶奶,你且收了,容日再補一兩銀子來與你,不敢差了。且容我見他一面,説些話兒則個。」那婆子於是收了他簪子和錢,吩咐:「你進去見他,說了話就與我出來。不許你涎眉睜目,只顧坐着。所許那一兩頭銀子,明日就送來與我。」於是掀簾,放經濟進裏間。

婦人正坐在炕邊納鞋,看見經濟,放下鞋扇,會在一處,埋怨經濟:「你好人兒!弄的我前不着村,後不着店,有上梢,沒下梢,出醜惹人嫌,你就影兒不見,不來看我看兒了!我娘兒們好好兒的,拆散開你東我西,皆因是為誰來?」說着,扯住經濟,只顧哭泣。王婆又嗔哭,恐怕有人聽見。經濟道:「我的姐姐,我為你剮皮割肉,你為我受氣耽羞,怎不來看你?昨日到薛嫂兒家,已知春梅賣在守備府裏去了,又打聽你出離了他家門,在王奶奶這邊聘嫁。今日特來見你一面,和你計議。咱兩個恩情難捨,拆散不開,如之奈何?我如今要把他家女兒休了,問他要我家先前寄放金銀箱籠。他若不與我,我東京萬壽門一本一狀進下來,那時他雙手奉與我還是遲了。我暗地裏假名托姓,一頂轎子娶你到家去,咱兩個永遠團圓,做上個夫妻,有何不可!」婦人道:「現今王乾娘要一百兩銀子,你有這些銀子與他?」經濟道:「如何要這許多?」婆子說道:「你家大丈母說,當初你家爹為他,打個銀人兒也還多,定要一百兩銀子,少一絲毫也成不的。」經濟道:「實不瞞你老人家說,我與六姐打得熱了,拆散不開。看你老人家下顧,退下一半兒來,五六十兩銀子也罷。我往張舅那裏典上兩三間房子,娶了六姐家去,也是春風一度。你老人家少賺些兒罷!」婆子道:「休說五十兩銀子,八十兩也輪不到你手裏了。昨日湖州販紬絹何官人,出到七十兩。大街坊張二官府,如今現在提刑院掌刑,使了兩個節級來,出到八十兩上,拿着兩封銀子來兑,還成不的,都囬去了。你這小孩兒家,空口來説空話,倒還敢奚落老娘,老娘不道的喫傷了哩!」當下一陣風走出街上,大吆喝說:「誰家女婿要娶丈母?還來老娘屋裏放屁!」

這經濟慌了,一手扯進婆子來,雙膝跪下,央及:「王奶奶噤聲,我依了奶奶價值一百兩銀子罷。爭奈我父親在東京,我明日起身,往東京取銀子去。」婦人道:「你旣為我一塲,休與乾娘爭執,上緊取去。只恐來遲了,别人娶了奴去了,就不是你的人了。」經濟道:「我僱上頭口,連夜兼程,多則半月,少則十日就來了。」婆子道:「常言先下米,先食飯。我的十兩銀子在外,休要少了,我得說明白着。」經濟道:「這個不必說,恩有重報,不敢有忘。」說畢,經濟作辭出門。到家收拾行李,次日早僱頭口,上東京取銀子去。此這去,正是:青龍與白虎同行,吉兇事全然未保。

畢竟未知後來如何,且聽下囘分解。

