\includepdf[pages={159,160},fitpaper=false]{tst.pdf}
\chapter*{第八十囬 \\陳經濟竊玉偸香 李嬌兒盜財歸院}
\addcontentsline{toc}{chapter}{第八十囬 陳經濟竊玉偸香 李嬌兒盜財歸院}
\markboth{{\titlename}卷之八}{第八十囬 陳經濟竊玉偸香 李嬌兒盜財歸院}


詩曰:

\begin{myquote}
寺廢僧居少,橋塌客過稀;

家貧奴婢懶,官滿吏民欺;

水淺魚難住,林疎鳥不棲:

世情看冷煖,人面逐高低。
\end{myquote}

此八句詩,單說着這世態炎凉,人心冷煖,可嘆之甚也!西門慶死了,首七光景,報恩寺朗僧官十六衆僧人做水陸,有喬大户家上祭。玉皇廟吴道官受齋在家攢念二七經,不題。

卻說那日,這應伯爵約會了齋祀中幾位朋友,頭一個是應伯爵,第二個謝希大,第三個花子由,第四個祝日念,第五個孫天化,第六個常時節,第七個白來創,七人坐在一處。伯爵先開口說道:「大官人没了,今二七光景。你我相交一場,當時也曾喫過他的,也曾用過他的,也曾使過他的,也曾借過他的,也曾嚼過他的。今日他沒了,莫非推不知道?洒土也瞇瞇後人眼睛兒,不然,他就到五閻王跟前,也不饒你我了。你我如今這等計較,每人各出一錢銀子,七人共凑上七錢。使一錢六分,連花兒買上一張桌面,五碗湯飯,五碟菓子;使了一錢,一付三牲;使了一錢五分,一瓶酒;使了五分,一盤冥紙香燭;使了二錢,買一個軸子,再求水先生作一篇祭文,使一錢二分銀子雇人擡了去。大官人靈前,衆人祭奠了,咱還便益:又討了他値七分銀一條孝絹,拿到家做裙腰子;他莫不白放咱們出來?咱還喫他一陣;到明日出殯,山頭饒飽餐一頓,每人還得他半張靠山桌面,來家與老婆孩子喫,省兩三日買燒餅錢。這個好不好?」衆人都道:「哥說的是!」當下每人凑出銀子來,交與伯爵,整理備祭物停當。買了軸子,央門外人水秀才做了祭文。這水秀才平昔知道應伯爵這起人,與西門慶乃小人之朋,於是包含着裏面,作就一篇祭文。祭軸停當,把祭祀擡到西門慶靈前擺下。陳經濟穿孝,在旁還禮。伯爵為首,各人上了香。人人都粗俗,那裏曉的其中滋味!澆了奠酒,只顧把祝文來宣念。其文略曰:

\begin{myquote}[\markfont]
「維重和元年,歲戊戍,二月戊子朔,越初三日庚寅,侍生應伯爵、謝希大、花子由、祝日念、孫天化、常時節、白來創,謹以清酌庶羞之奠,致祭于

故錦衣西門大官人之靈曰:維靈生前梗直,秉性堅剛。軟的不怕,硬的不降。常濟人以點水,恒助人以精光。囊篋頗厚,氣概軒昂。逢薬而擧,遇陰伏降。錦襠隊中居住,團腰庫裏收藏。有八角而不用撓摑,逄虱蟣而騷癢難當。受恩小子,常在胯下隨幫。也曾在章臺而宿柳,也曾在謝館而猖狂。正宜撑頭豁腦,久戰熬場;胡何一疾,不起之殃!現今你便長伸着脚子去了,丢下小子軰如班鳩跌彈,倚靠何方?難上他煙花之寨,難靠他八字紅牆;再不得同席而偎軟玉,再不得並馬而傍溫香。撇的人垂頭跌脚,閃得人囊溫郎當!今特奠茲白濁,次獻寸觴。靈其不昧,來格來歆,尚享!」
\end{myquote}

衆人祭畢,陳經濟下來還禮,請去捲棚内,三湯五割,管待出門。

那日院中李家虔婆,聽見西門慶死了,鋪謀定計,備了一張祭桌,使了李桂卿李桂姐坐轎子來上紙弔問。月娘不出來,都是李嬌兒孟玉樓在上房管待。李家桂卿桂姐悄悄對李嬌兒說:「俺媽說,人已是死了,你我院中人,守不的這樣貞節。自古千里長棚,沒個不散的筵席。呌你手裏有東西,悄悄教李銘捎了家去防後。你還恁儍!常言道:揚州雖好,不是久戀之家。不拘多少時,也少不的離他家門。」那李嬌兒聽記在心。

不想那日韓道國妻王六兒亦備了張祭桌,喬素打扮,坐轎子來與西門慶燒紙。在靈前擺下祭祀,只顧站着。站了半日,白没個人兒出來陪待。原來西門慶死了,首七時分,就把王經打發家去不用了。小廝們見王六兒來,都不敢進去說。那來安兒不知就裏,到月娘房裏向月娘説:「韓大嬸來與爹上紙,在前邊站了一日了。大舅使我來對娘説。」這吴月娘心中還氣忿不過,便喝罵道:「怪賊奴才!不與我走,還來甚麽韓大嬸𣭈大嬸!賊狗攮的養漢的淫婦,把人家弄的家敗人亡,父南子北,夫逃妻散的,還來上甚麽𣭈紙!」一頓駡的來安兒摸門不着。來到靈前,吴大舅問道:「對後邊說了不曾?」來安兒把嘴谷都着,不言語。問了半日,才說:「娘捎出四馬兒來了!」這吴大舅連忙進去,對月娘說:「姐姐,你怎麽這等的!快休要舒口。自古人惡禮不惡。他男子漢領着咱偌多的本錢,你如何這等待人?好名兒難得,快休如此!你就不出去,教二姐姐三姐姐好好待他也是一般。做甚麽恁樣的,教人說你不是?」那月娘見他哥這等說,纔不言語了。良久,孟玉樓出去還了禮,陪他在靈前坐的。只喫一鍾茶,婦人也有些省䐩,就坐不住,隨即告辭起身去了。正是:誰人汲得西江水,難洗今朝一面羞!

那李桂卿桂姐吴銀兒都在上房坐着,見月娘罵韓道國老婆淫婦長淫婦短,砍一枝,損百林,兩個就有些坐不住,未到日落,就要家去。月娘再三留他姐兒兩個:「晚夕夥計們伴宿,你們看了提偶的,明日去罷。」留了半日,只桂姐銀姐不去了,只打發他姐姐桂卿家去了。到了晚夕,僧人散了,果然有許多街坊夥計主管、喬大户,吴大舅、吴二舅、沈姨夫、花子由、應伯爵、謝希大、常時節,也有二十餘人,叫了一起偶戲,在大捲棚内擺設酒席伴宿。提演的是「孫榮孫華殺狗勸夫」戯文。堂客都在靈旁廳内,圍着幃屏,放下簾來,擺放桌席朝外觀看。李銘吴惠在這裏答應,晚夕也不家去了。不一時,衆人都到齊了。祭祀已畢,捲棚内點起燭來,安席坐下。打動鼓楽,戯文上開,直搬演到三更天氣,戯文方了。

原來陳經濟自從西門慶死後,無一日不和潘金蓮兩個嘲戲。或在靈前溜眼,帳子後調笑。至是趕人散一亂,衆堂客都往後邊去了,小廝們都收家活,這金蓮趕眼錯,捏了經濟一把,說道:「我兒,你娘今日可成就了你罷!趂大姐在後邊,咱們就往你屋裏去罷。」經濟聽了,巴不的一聲,先往屋裏開門去了。婦人黑影裏抽身鑽入他房内,更不答話,解開裙子,仰臥在炕上,雙鳧飛肩,教陳經濟奸耍。正是:色膽如天怕甚事,鴛幃雲雨百年情。眞個是:

\begin{myquote}
二載相逢,一朝配偶;數年姻眷,一旦和諧。一個柳腰欵擺,一個玉莖忙舒。耳邊訴雨意雲情,枕上說山盟海誓。鶯恣蝶採,旖旎搏弄百千般;狂雨羞雲,嬌媚施逞千萬態。一個低聲不住叫親親,一個摟抱未免呼達達。正是:得多少柳色乍翻新樣綠,花容不减舊時紅!
\end{myquote}

霎時雲雨了畢,婦人恐怕人來,連忙出房,往後邊去了。到次日,這小夥兒嚐着這個甜頭兒,早晨走到金蓮房來。金蓮還在被窝裏未起來,従窗眼裏張看,見婦人被擁紅雲,粉腮印玉,說道:「好個管庫房的,這咱還不起來!今日喬親家爹來上祭,大娘吩咐教把昨日擺的李三黄四家那祭桌,收進來罷。你快些起來,且拿鑰匙出來與我。」婦人連忙教春梅拿鑰匙與經濟。經濟先教春梅樓上開門去了,婦人便従窗眼裏遞出舌頭,兩個咂了一囬。正是:得多少脂香滿口涎空嚥,甜唾融心溢肺肝。有詞為證:

\begin{myquote}
恨杜鵑聲透珠簾,心似針簽,情似膠粘。我則見笑臉腮窝,愁生粉黛,瘦顯春纖。寳髻亂、雲鬆翠鈿,睡顔酡、玉減紅添。檀口曾粘,甜唾曾沾,到如今唇上猶香,想起來口内猶甜。
\end{myquote}

良久,春梅樓上開了門,經濟往前邊看搬祭祀去了。不一時,喬大户家祭來擺下。喬大户娘子並喬大户許多親眷,靈前祭畢,吴大舅二舅甘夥計陪侍,請至捲棚管待。李銘吴惠彈唱。那日鄭愛月兒家也來上紙弔孝。月娘俱令玉樓打發了孝絹,後邊與堂客一䖏坐的。鄭愛月兒看見吴銀姐李桂姐都在這裏,便嗔他兩個不對他說:「我若知道爹没了,有個不來的?你們好人兒,就不會我會兒去!」又見月娘生了孩兒,說道:「娘一喜一憂。惜乎只是爹去世太早了些兒!你老人家有了主兒,也不愁。」月娘俱打發了孝裙束腰,留坐至晚方散。

到二月初三日,西門慶二七,玉皇廟吳道官十六個道衆,在家念經做法事。那日衙門中何千户作創,約會了劉薛二内相、周守禦、荆統制、張團練、雲指揮等數員武官,合着上了一壇祭。月娘這裏請了喬大户吴大舅應伯爵來陪侍。李銘吴惠兩個小優兒彈唱,捲棚管待去了。俱不必細說。到晚夕念經送亡,月娘吩咐把李瓶兒靈牀,連影擡出去,一把火焚之,將箱籠都搬到上房内堆放。奶子如意兒並迎春收在後邊答應,把綉春與了李嬌兒房内使喚。將李瓶兒那邊房門一把鎖鎖了。可憐正是畫棟雕梁猶未乾,堂前不見癡心客。有詩為證:

\begin{myquote}
襄王臺下水悠悠,一種相思兩地愁。

月色不知人事改,夜深還到粉牆頭!
\end{myquote}

那時李銘日日假意孝堂助忙,暗暗教李嬌兒偸轉東西,與他掖送到家,又來答應。常兩三夜不往家去,只瞞過月娘一人眼目。吴二舅又和李嬌兒舊有首尾,誰敢道個不字。初九日念了三七經,月娘出了暗房。四七就沒曾念經。十六日,陳經濟破了土囘來,二十日早發引。也有許多冥器紙劄,送殯之人終不似李瓶兒那時稠密。臨棺材出門,陳經濟摔盆扶柩。也請了報恩寺朗僧官起棺,坐在轎上,捧的高高的,念了幾句偈文,說西門慶一生始末,道得好;

\begin{myquote}
「恭維

故錦衣武略將軍西門大官人之靈:伏以人生在世,如電光易滅,石火難留。落花無返樹之期,逝水絶歸源之路。你畫堂繡閣,命盡有若風燈;極品高官,祿絶猶如作夢。黄金白玉,空為祸患之資;紅粉輕裘,總是塵勞之費。妻孥無百載之歡,黑暗有千重之苦。一朝枕上,命掩黄泉,空榜揚虚假之名,黄土埋不堅之骨。田園百頃,其終被兒女爭奪;綾錦千箱,死後無寸絲之分。風火散時無老少,溪山磨盡幾英雄。苦苦苦,氣化清風形歸土。三寸氣斷去弗迴,改頭換面無遍數。

{\markfont{詩曰:}}

人生最苦是無常,個個臨終手脚忙。

地水火風相逼迫,精神魂魄各飛揚。

生前不解尋活路,死後知他去那廂?

一切萬般將不去,赤條條的見閻王。
\end{myquote}

朗僧官念畢偈文,陳經濟摔破紙盆,棺材起身,合家大小孝眷,放聲號哭動天。吳月娘坐魂轎,後面衆堂客上轎,都尾隨材走,逕出南門外五里原祖塋安厝。陳經濟備了一疋尺頭,請雲指揮點了神主;陰陽徐先生下了葬,衆孝眷掩土畢。山頭祭桌,可憐通不上幾家:只是吴大舅喬大户何千户沈姨夫韓姨夫與衆夥計五六處而已。吳道官還留下十二衆道童囘靈,安於上房明間正寢。大小安靈、陰陽灑掃已畢,打發衆親戚出門。吴月娘等,不免伴夫靈守孝。一日,煖了墓囘來,答應班上排軍節級,各都告辭囘衙門去了。西門慶五七,月娘請了薛姑子、王姑子、大師父、十二衆尼僧,在家誦經禮懺,超度夫主生天。吴大妗子並吴舜臣媳婦,都在家中相伴。

原來出殯之時,李桂卿桂姐在山頭,悄悄對李嬌兒如此這般:「媽說你,摸量你手中沒甚細軟東西,不消只顧在他家了。你又沒兒女,守甚麽?呌你一場嚷亂登開了罷。昨日應二哥來說,如今大街坊張二官府,要破五百兩金銀娶你做二房娘子,當家理紀。你那裏便圖出身,你在這裏守到老死也不怎麽。你我院中人家,棄舊迎新為本,趨炎附勢為強,不可錯過了時光!」這李嬌兒聽記在心,過了西門慶五七之後,因風吹火,用力不多,——不想潘金蓮對孫雪娥說:「出殯那日,在坟上看見李嬌兒與吴二舅在花園小房内兩個說話來;春梅孝堂中又親眼看見李嬌兒帳子後,遞了一包東西與李銘㩙在腰裏,轉了家去。」嚷的月娘知道,把吴二舅罵了一頓,趕去舖子裏做買賣,再不許進後邊來。吩咐門上平安,不許李銘來往。這花娘惱羞變成怒,正尋不着這個由頭兒哩!一日,因月娘在上房和大妗子喫茶,請孟玉樓不請他,就惱了,與月娘兩個大嚷大鬧,拍着西門慶靈牀子哭哭啼啼,呌呌嚎嚎,到半夜三更,在房中要行上弔。丫鬟來報與月娘。月娘慌了,與大妗子計議,請將李家虔婆來,要打發他歸院。虔婆生怕留下他衣服頭面,說了幾句言語:「我家人在你這裏做小伏低,頂缸受氣,好容易就開交了罷?須得幾十兩遮羞錢!」吴大舅居着官,又不敢張主。相講了半日,教月娘把他房中衣服首飾箱籠牀帳家活盡與他,打發出門。只不與他元宵綉春兩個丫鬟去。李嬌兒一心要這兩個丫頭,月娘生死不與他,說道:「你倒好,買良為娼!」一句慌了鴇子,就不敢開言,變做笑吟吟臉兒,拜辭了月娘,李嬌兒坐轎子擡的往家去了。

看官聽說:院中唱的,以賣俏為活計,將脂粉作生涯。早晨張風流,晚夕李浪子。前門進老子,後門接兒子。棄舊迎新,見錢眼開,自然之理!未到家中,撾打揪撏,燃香燒剪,走死哭嫁;娶到家,改志従良,饒君千般貼戀,萬種牢籠,還鎖不住他心猿意馬,不是活時偸食抹嘴,就是死後嚷鬧離門。不拘幾時,還喫舊鍋粥去了!正是:蛇入筒中曲性在,鳥出籠輕便飛騰。有詩為證:

\begin{myquote}
堪嘆煙花不久長,洞房夜夜換新郎。

兩隻玉腕千人枕,一點朱唇萬客嘗。

造就百般嬌艷態,生成一片假心腸。

饒君縱有牢籠計,難保臨時思故鄉。
\end{myquote}

月娘於是打發李嬌兒出門,大哭了一場,衆人都在旁勸解。潘金蓮道:「姐姐,罷,休煩惱了!常言道:娶淫婦,養海青,食水不到想海東!這個都是他當初幹的營生,今日教大姐姐這等惹氣!」

家中正亂着,忽有平安兒來報:「巡鹽蔡老爹來了,在廳上坐着哩。我說家老爹没了。他問沒了幾時了,我囘正月二十一日病故,到今過了五七。他問有靈没靈?我囬有靈,在後邊供養着哩。他要來靈前拜拜,我來對娘說。」月娘吩咐:「教你姐夫出去見他。」不一時,陳經濟穿上孝衣,出去拜見了蔡御史。良久,後邊收拾停當,請蔡御史進來,西門慶靈前參拜了。月娘穿着一身重孝,出來囬禮。再不交一言,就讓月娘:「夫人請囬房。」因問經濟說道:「我昔時曾在府相擾,今差滿囬京去,敬來拜謝拜謝,不期作了故人!」便問:「甚麽病來?」陳經濟道:「是個痰火之疾。」蔡御史道:「可傷,可傷!」即喚家人上來,取出兩疋杭州絹,一雙羢襪,四尾白鮝,四罐蜜餞,說道:「這些微禮,權作奠儀罷!」又拿出五十兩一封銀子來:「這個是我向日曾貸過老先生些厚惠,今積了些俸資奉償,以全始終之交。」吩咐:「大官,交進房去。」經濟道:「老爹忒多計較了!」月娘道:「請老爹前廳坐。」蔡御史道:「也不消坐了。拿茶來我喫一鍾就是了。」左右須臾拿茶上來,蔡御史喫了,揚長起身上轎去了。月娘得了這五十兩銀子,心中又是那歡喜,又是那慘切!想有他在時,似這樣官員來到,肯空放去了?又不知喫酒到多早晚!今日他伸着脚子,空有家私,眼看着就無人陪侍。正是:人得交游是風月,天開圖畫即江山。有詩為證:

\begin{myquote}
靜掩重門春日長,為誰展轉怨流光。

更憐無似秋波眼,默地懷人淚兩行。
\end{myquote}

話說李嬌兒到家,應伯爵打聽得知,報與張二官兒,就拿着五兩銀子,來請他歇了一夜。原來張二官小西門慶一歲,屬兔的,三十二歲了。李嬌兒三十四歲,虔婆瞞了六歲,只説二十八歲,教伯爵也瞞着。使了三百兩銀子,娶到家中,做了二房娘子。祝日念、孫寡嘴,依舊領着王三官兒還來李家行走,與桂姐打熱,不在話下。伯爵李三黄四借了徐内相五千兩銀子,張二官出了五千兩,做了東平府古器這批錢糧,逐日寳鞍大馬,在院中搖擺。張二官見西門慶死了,又打點了千兩金銀,上東京尋了樞密院鄭皇親人情,對堂上朱太尉說,要討提刑所西門慶這個缺,家中收拾買花園蓋房子。應伯爵無日不在他那邊趨奉,把西門慶家中大小之事,盡告訴與他,說:「他家中還有第五個娘子潘金蓮,排行六姐,生的極標致,上畫兒般人材!詩詞歌賦,諸子百家,拆白道字,雙陸象棋,無不通曉;又會識字,一筆好寫。彈一手好琵琶。今年不上三十歲,比唱的還喬!」說的這張二官心中火動,巴不得就要了他。便問道:「莫非是當初的賣炊餅武大郎的妻子麽?」伯爵道:「就是他。被他占來家中,今也有五六年光景。不知他嫁人不嫁。」張二官道:「累你打聽着,待有嫁人的聲口,你來對我說,等我娶了罷。」伯爵道:「我酩子裏有個人在他家做家人,名來爵兒。等我對他說,若有出嫁聲口,就來報你知道。難得你若娶過他這個人來家,也強如娶個唱的!當時有西門慶在,為娶他也費了許多心。大抵物各有主,也說不的,只好有福的匹配。你如今有了這般勢耀,不得此女貌同享榮華,枉自有許多富貴!我只叫來爵兒密密打聽,但有嫁人的風縫兒,憑我甜言羙語,打動春心;你卻用幾百兩銀子,娶到家中,儘你受用便了。」

看官聽說:但凡世上幫閒子弟,極是勢利小人。見他家豪富,希圖衣食,便竭力奉承,稱功誦德;或肯撒漫使用,說是疏財仗義,慷慨丈夫。脅肩諂笑,獻子出妻,無所不至。一見那門庭冷落,便唇譏腹誹說他外務,不肯成家立業;祖宗不幸,有此敗兒!就是平日深恩,視如陌路。當初西門慶待應伯爵如膠似漆,賽過同胞弟兄,那一日不喫他的,穿他的,受用他的?身死未幾,骨肉尚熱,便做出許多不義之事!正是:畫虎畫皮難畫骨,知人知面不知心!有詩為證:

\begin{myquote}
昔年意氣似金蘭,百計趨承不等閒。

今日西門身死後,紛紛謀妾伴人眠。
\end{myquote}

畢竟未知後來如何,且聽下囬分解。

\part*{夢梅館校本《金瓶梅詞話》卷之九}
\addcontentsline{toc}{part}{夢梅館校本《金瓶梅詞話》卷之九}

