\includepdf[pages={87,88},fitpaper=false]{tst.pdf}
\chapter*{第四十四囬 \\吳月娘留宿李桂姐 西門慶醉拶夏花兒}
\addcontentsline{toc}{chapter}{第四十四囬 吳月娘留宿李桂姐 西門慶醉拶夏花兒}
\markboth{第四十四囬 吳月娘留宿李桂姐 西門慶醉拶夏花兒}{第四十四囬 吳月娘留宿李桂姐 西門慶醉拶夏花兒}
\thispagestyle{empty}

\begin{myquote}
窮途日日困泥沙,上苑年年好物華:

荆棘不當車馬道,管絃長奏綺羅家;

王孫草上悠揚蝶,少女風前爛漫花。

懶出任従遊子笑,入門還是舊生涯。
\end{myquote}

話説經濟同傅夥計衆人前邊喫酒,吳大妗子轎子來了,收拾要家去。月娘款留再三,説道:「嫂子再住一夜兒,明日去罷。」吴大妗子道:「我連在喬親家那裏就是三四日了。家裏沒人,你哥衙裏又有事,不得在家,我家去罷。明日請姑娘衆位好歹往我那裏大節坐坐,晚夕走百病兒來家。」月娘道:「俺們明日只是晚上些去罷了。」吴大妗子道:「姑娘早些坐轎子去,晚夕同走了來家就是了。」説畢,裝了兩個盒子:一盒子元宵,一盒子饅頭,叫來安兒送大妗子到家。

李桂姐等四個都磕了頭,拜辭月娘,也要家去。月娘道:「你們慌怎的,也就要去?還等你爹來家着你去。他去吩咐我留下你們,只怕他還有話和你們説,我是不敢放你去。」桂姐道:「爹去喫酒,到多早晚來家!俺們原等的他?娘先教我和吴銀姐先去罷,他兩個今日纔來,俺們住了兩日,媽在家裏不知怎麽盼望。」月娘道:「可可的就是你媽盼望,這一夜兒等不的?」李桂姐道:「娘且是説的好。我家裏沒人,俺姐姐又被人包住了。寜可拿楽器來唱個與娘聽,娘放了奴去罷!」正説着,只見陳經濟走進來交剩下的賞賜與吴月娘,説道:「喬家并各家貼轎賞一錢,共使了十包,重三兩。還剩下十包在此。」月娘收了。桂姐便道:「我央及姑夫,你看外邊俺們的轎子來了不曾?」經濟道:「只有他兩個的轎子。你和銀姐的轎子沒來。従頭裏不知誰回了去了。」桂姐道:「姑夫,你眞個囬了?你哄我哩!」那陳經濟道:「你不信,瞧去不是!我哄你?」剛言未罷,只見琴童抱進毡包來説:「爹家來了。」月娘道:「早是你們不去了,這不你爹來了?」

不一時,西門慶進來,戴着冠帽,已帶七八分酒了,走入房中,正面坐下。玉簫便遞茶。董嬌兒、韓玉釧兒,二人向前磕頭。西門慶便問月娘道:「人都散了,你怎的不教他唱?」月娘道:「他們這裏求着我要家去,且說更已深了。」西門慶向桂姐説:「你和銀兒一發過了節兒去。且打發他兩個去罷。」月娘道:「如何?我説你們不信,恰像我哄你一般。」那桂姐把臉兒苦低着,不言語。西門慶問玳安:「他兩個轎子在這裏不曾?」玳安道:「只有董嬌兒韓玉釧兒兩頂轎子伺候着哩。」西門慶道:「我也不喫酒了。你們拿楽器來唱〈十段錦兒〉我聽,打發他兩個先去罷。」當下四個唱的,李桂姐彈琵琶,吴銀兒彈箏,韓玉釧兒撥阮,董嬌兒打着緊急鼓子,一遞一個唱〈十段錦·二十八半截兒〉。吳月娘、李嬌兒、孟玉樓、潘金蓮、李瓶兒,都在屋裏坐的聽唱。

先是桂姐唱:〈山坡羊〉
\begin{myquote}
「俏寃家,生的出類拔萃。翠衾寒,孤燈獨自。自别後,朝思暮想;想寃家,何時得遇?遇見寃家好同往,好同往!」
\end{myquote}

該吴銀兒唱:

\begin{myquote}
{\markfont〈金字經〉}「惜花人何處,落紅春又殘。倚遍危樓十二欄,十二欄。」
\end{myquote}

韓玉釧唱:

\begin{myquote}
{\markfont〈駐雲飛〉}「悶倚欄杆,燕子鶯兒怕待看。色戒誰曾犯?鬼病誰經慣?」
\end{myquote}

董嬌兒唱:

\begin{myquote}
{\markfont〈江兒水〉}「呀!減盡了花容月貌,重門常是掩。正東風料峭,細雨漣瀸,落紅千萬點。」
\end{myquote}

桂姐唱:

\begin{myquote}
{\markfont〈畫眉序〉}「自會俏寃家,銀箏塵鎖怕湯抹。雖然是人離咫尺,如隔天涯。記得百種恩情,那裏討半星兒狂詐。」
\end{myquote}

吴銀兒唱:

\begin{myquote}
{\markfont〈紅綉鞋〉}「水面上鴛鴦一對,順河岸步步相隨。怎見個打漁船驚拆在雨下裏飛。」
\end{myquote}

韓玉釧唱:

\begin{myquote}
{\markfont〈耍孩兒〉}「自從他去添憔瘦,不似今番病久。才郎一去正逢春,急囬頭鴈過了中秋。」
\end{myquote}

董嬌兒唱:

\begin{myquote}
{\markfont〈傍粧臺〉}「到如今,瑤琴絃断少知音,花好時誰共賞?」
\end{myquote}

桂姐唱:

\begin{myquote}
{\markfont〈鎖南枝〉}「紗窗外,月兒斜,久想我人兒常常不捨。你為我力盡心謁,我為你珠淚偸揩。」
\end{myquote}

吴銀兒唱:

\begin{myquote}
{\markfont〈桂枝香〉}「楊花心性,隨風不定。他原來假意兒虚名,倒使我眞心陪奉。」
\end{myquote}

韓玉釧唱:

\begin{myquote}
{\markfont〈山坡羊〉}「惜玉憐香,我和他在芙蓉帳底。抵面,共你把衷腸來細講;講離情,如何把奴抛棄。氣的我,似醉如癡來呵;何必,你變心另叙上知己;幾時,得重整佳期?佳期,實相逢如同夢裏!」
\end{myquote}

董嬌兒唱:

\begin{myquote}
{\markfont〈金字經〉}「彈,淚痕羅帕斑;江南岸,夕陽山外山。」
\end{myquote}

李桂姐唱:

\begin{myquote}
{\markfont〈駐雲飛〉}「嗏!書寄兩三番,得見艱難。再倩霜毫,寫下喬公案,滿紙春心墨未乾。」
\end{myquote}

吳銀兒唱:

\begin{myquote}
{\markfont〈江兒水〉}「香串懶重添,針兒怕待拈。瘦體嵓嵓,鬼病懨懨。俺將這舊恩情重檢點,愁壓損兩眉翠尖。空惹的張郎憎厭,這些時對鶯花不捲簾。」
\end{myquote}

韓玉釧唱:

\begin{myquote}
{\markfont〈畫眉序〉}「想在枕上溫存的話,不由人肉顫身麻。」
\end{myquote}

董嬌兒唱:

\begin{myquote}
{\markfont〈紅綉鞋〉}「一個兒投東去,一個兒向西飛;撇的俺一個兒南來,一個兒北去。」
\end{myquote}

李桂姐唱:

\begin{myquote}
{\markfont〈耍孩兒〉}「你那裏偎紅倚翠銷金帳,我這裏獨守香閨淚暗流。從記得説來咒:負心的隨燈兒滅!海神廟放着根由。」
\end{myquote}

吳銀兒唱:

\begin{myquote}
{\markfont〈傍粧臺〉}「羙酒兒誰共斟?意散了如萍兒,難見面似參辰。従別後歲月深,畫劃兒畫損了掠兒金。」
\end{myquote}

韓玉釧唱:

\begin{myquote}
{\markfont〈鎖南枝〉}「兩下裏心腸牽掛,誰知道風掃雲開,今宵復顯出團圓月。重令情郎把香羅再解。訴説情誰負誰心,湏共你説個明白。」
\end{myquote}

董嬌兒唱:

\begin{myquote}
{\markfont〈桂枝香〉}「怎忘了舊時山盟為證,坑人性命。有情人,従此分離了去,何時再得成?」
\end{myquote}

李桂姐唱:

\begin{myquote}
{\markfont〈尾聲〉}「半叉綉羅鞋,眼兒見了心兒愛。可喜才,捨着搶白,忙把這俏身挨。」
\end{myquote}

唱畢,西門慶與了韓玉釧董嬌兒兩個唱錢,拜辭出門;留李桂姐吴銀兒兩個:「這裏歇罷!」忽聽前邊玳安兒和琴童兒兩個嚷亂,簇擁定李嬌兒房裏夏花兒進來禀西門慶,説道:「小的剛送兩個唱的出去,打燈籠往馬房裏拌草,牽馬上槽。只見二娘房裏夏花兒躲在馬槽底下,唬了小的一跳。不知甚麽緣故?小的們問着他,又不説。」西門慶聽見,便道:「那奴才在那裏?與我拿來。」就走出外邊明間穿廊下椅子上坐着,一邊打着,兩個簇把那丫頭兒揪着跪下。西門慶問他:「往前邊做甚麽去?」那丫頭不言語。李嬌兒在傍邊説道:「我又不使你,平平白白往馬坊裏做甚麽去?」見他慌做一團,西門慶只説丫頭要走之情,即令小廝:「與我與他搜身上。」他又不容搜。於是琴童把他一拉,倒在地,只聽滑浪一聲,沉甸甸従腰裏掉下一件東西來。西門慶問:「是甚麽?」玳安遞上去。可霎作怪,卻是一錠金子。西門慶燈下看了道:「是頭裏不見了的那錠金子。尋不見,原來是你這奴才偸了!」他説:「是拾的。」西門慶問:「是那裏拾的?」他又不言語。西門慶於是心中大怒,令琴童往前邊去取拶子來。須臾,把丫頭拶起來,拶的殺猪也似叫。拶了半日,又敲二十敲。月娘見他有酒了,又不敢勸。那丫頭挨忍不過,方説:「我在六娘房裏地下拾的。」西門慶方命放了拶子。又吩咐與李嬌兒領到屋裏去:「明日叫媒人,即時與我拉出去賣了!這個奴才,還留着做甚麽?」那李嬌兒沒的話兒説,便道:「恁賊奴才,誰叫你往前頭去來?養在家裏,也問我聲兒,三不知就出去了。你就拾了他屋裏金子,也對我説一聲兒!」那夏花兒只是哭。李嬌兒道:「拶死你這奴才纔好哩,你還哭!」西門慶道:「罷!」把金子交與月娘收了,就往前邊李瓶兒房裏去了。那小廝都出去了。

月娘令小玉關上儀門,因叫過玉簫來,問他:「頭裏這丫頭也往前邊去來麽?」小玉道:「二娘三娘陪大妗子娘兒兩個往六娘那邊去,他也跟了去來。誰知他三不知就偸了他這錠金子在手裏。頭裏聽見娘説爹使小廝買狼筋去了,唬的他了不的,在廚房問我:『狼筋是甚麽?』教俺們衆人笑道:『狼筋敢是狼身上的筋,若是那個偸了東西不拿出來,把狼筋抽將起來,就纏在那人身上,抽攢的手脚兒都在一處。』他聽見想必慌了。到晚夕趕唱的出去,就要走的情。見大門首有人,纔藏入馬坊裏,鑽在槽底下躲着。不想被小廝又看見了,採出來。」月娘道:「那裏看人去?恁小丫頭,原來這等賊頭鼠腦的!倒就不是個咍咳的。」

且説李嬌兒領夏花兒到房裏,李桂姐晚間甚是説夏花兒:「你原來是個儍孩子,你恁十五六歲,也知道些人事兒,還這等懵懂?要着俺裏邊,纔使不的。這裏沒人,你就拾了些東西,來屋裏悄悄交與你娘。似這等拖出來,他在傍邊也好救你。你怎的不望他題一字兒?剛纔這等拶打着好麽?乾淨儍丫頭!常言道:穿青衣,抱黑柱。你不是他這屋裏人?就不管他?剛纔這等掠掣着你,你娘臉上有光沒光?」又説他姑娘:「你也忒不長俊。要着是我,怎教他把我房裏丫頭對衆拶恁一頓拶子?有不是,拉到房裏來,等我打。前邊幾個房裏丫頭怎的不拶,只拶你房裏丫頭?你是好欺負的,就鼻子口裏沒些氣兒?等不到明日,眞個教他拉出這丫頭去罷,你也就沒句話兒説?你不説,等我説,休教他領出去,教別人好笑話。你看看孟家的和潘家的,兩家一似狐狸一般,你原鬦的過他了?」因叫了夏花兒過來,問他:「你出去不出去?」那丫頭道:「我不出去。」桂姐道:「你不出去,今後要貼你娘的心,凡事要你和他一心一計。不拘拿了甚麽,交付與他,教似元宵一般擡擧你。」那夏花兒説:「姐吩咐,我知道了。」按下這裏教唆夏花兒不題。

且説西門慶走到前邊李瓶兒房裏,只見李瓶兒和吳銀兒炕上做一處坐的,心中就要脫衣去睡。李瓶兒道:「銀姐在這裏,沒地方兒安插,你且過一家兒罷!」西門慶道:「怎的沒地方兒?你娘兒兩個在兩邊,等我在當中睡就是。」李瓶兒便瞅了他眼兒道:「你就説下道兒去了。」西門慶道:「我如今在那裏睡?」李瓶兒道:「你過六姐那邊去睡一夜罷!」西門慶坐了一回,起身走了,説道:「也罷,也罷!省的我打攪你娘兒們,我過那邊屋裏睡去罷。」於是一直走過金蓮這邊來。金蓮聽見西門慶進房來,天上落下來一般。向前與他接衣解帶,鋪陳牀鋪乾淨,展放鮫綃,款設珊枕,喫了茶,兩個上牀歇宿不題。

李瓶兒這裏打發西門慶出來,和吴銀兒兩個燈下放炕桌兒,撥下黑白棋子,對坐下象棋兒。吩咐迎春:「定兩盞茶兒,拿個菓盒兒,把這甜金華酒兒篩一壺兒來,我和銀姐喫。」因問:「銀姐你喫飯?教他盛飯來你喫。」吴銀兒道:「娘,我且不餓,休叫姐盛來。」李瓶兒道:「也罷!銀姐不喫飯,你拿個盒蓋兒,我揀粧裏有菓餡餅兒拾四個兒來,與銀姐喫罷。」湏臾,迎春拿了四碟小菜:一碟糟蹄子筋、一碟鹹鷄、一碟れ雞疍、一碟炒的荳芽菜拌海蜇;一個菓盒,都是細巧菓仁兒;一盒菓餡餅兒;準備在傍邊。少頃,與吳銀兒下了三盤棋子。篩上酒來,拿銀鍾兒兩個共飲。吴銀兒叫迎春:「姐,你遞過琵琶來,我唱個曲兒與娘聽。」李瓶兒道:「銀姐,不唱罷,小大官兒睡着了。他爹那邊又聽着,教他説。咱擲骰子耍耍罷。」於是教迎春遞過色盆來。兩個擲骰兒賭酒為楽。擲了一囬,吴銀兒因叫迎春:「姐,你那邊屋裏請過奶媽兒來,教他喫鍾酒兒。」迎春道:「他摟着哥兒在那邊炕上睡哩!」李瓶兒道:「教他摟着孩子睡罷。拿一甌子酒,送與他喫就是了。你不知,俺這小大官好不伶俐,人只離開來,他就醒了。有一日兒,在我這邊炕上睡,他爹這裏敢動一動兒,就睜開眼醒了,恰似知道的一般。敎奶子抱了去那邊屋裏,只是哭,只要我摟着他。」吴銀兒笑道:「娘有了哥兒,和爹自在覺兒也不得睡一個兒。爹幾日來這屋裏走一遭兒?」李瓶兒道:「他也不論,遇着一遭也不可定,兩遭也不可定,常進屋裏看他。為這孩子,來看他不打緊,教人把肚子也氣破了。將他爹和這孩子,背地咒的白湛湛的。我是不消説的,只與人家墊舌根!誰和他有甚麽大閒事,寜可他不來我這裏還好。第二日教人眉兒眼兒的只説俺們什麽把攔着漢子。為甚麽剛纔到這屋裏,我就攛掇他出去?銀姐,你不知,俺這家人多舌頭多!自今日為不見了這錠金子,早是你看着,就有人氣不憤,在後邊調白你大娘,説拿金子進我這屋裏來了,怎的不見了。落後不想是你二娘屋裏丫頭偸了,纔顯出個青紅皂白來。不然,綁着鬼只是俺這屋裏丫頭和奶子。老馮媽媽急的那哭,只要尋死,説道:『若沒有這金子,我也不家去。』落後見有了金子,那咱纔肯去,還打了燈家去了。」吳銀兒道:「娘,也罷!你看爹的面上,你守着哥兒,慢慢過到那裏是那裏。論起後邊大娘,沒甚言語,也罷了。倒只是别人見娘生了哥兒,未免都有些兒氣。爹他老人家有些張主就好。」李瓶兒道:「若不是你爹和你大娘看覷,這孩子也活不到如今!」説話之間,你一鍾,我一盞,不覺坐到三更天氣,方纔宿歇。正是:得意客來情不厭,知心人到話相投。有詩為證:

\begin{myquote}
畫樓明月轉窗寮,相伴嬋娟宿一宵。

玉骨冰肌誰不愛,一枝梅影夜迢迢。
\end{myquote}

畢竟未知後來何如,且聽下回分解。

