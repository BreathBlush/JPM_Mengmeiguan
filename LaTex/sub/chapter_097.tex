\includepdf[pages={193,194},fitpaper=false]{tst.pdf}
\chapter*{第九十七囬 \\經濟守禦府用事 薛嫂賣花說姻親}
\addcontentsline{toc}{chapter}{第九十七囬 經濟守禦府用事 薛嫂賣花說姻親}
\markboth{{\titlename}卷之十}{第九十七囬 經濟守禦府用事 薛嫂賣花說姻親}


\begin{myquote}
在世為人保七旬,何勞日夜弄精神。

世事到頭終有盡,浮華過眼恐非眞。

貧窮富貴天之命,得失榮枯隙裏塵。

不如且放開懷楽,莫待無常鬼使侵。
\end{myquote}

話説陳經濟到於守備府中,下了馬,張勝先進去禀報春梅。春梅吩咐,教他在外邊班直房内,用香湯澡盆,沐浴了身體乾淨。後邊使養娘包出一套新衣服靴帽來,與他更換了。張勝把他身上脱下來舊藍縷衣服,捲做一團,擱在班直房内梁上吊着,然後禀了春梅。那時守備還未退廳,春梅請經濟到後堂,盛粧打扮,出來相見。這經濟進門,就望春梅拜了四雙八拜:「請姐姐受禮!」那春梅受了半禮,對面坐下,叙說寒溫離别之情,彼此皆眼中垂淚。春梅恐怕守備退廳進來,見無人在跟前,使眼色與經濟,悄悄説:「等住囬他若問你,只說是姑表兄弟,我大你一歲,二十五歲了,四月廿五日午時生的。」經濟道:「我知道了。」不一時,丫鬟拿上茶來。兩人喫了茶,春梅便問:「你一向怎麽出了家,做了道士?守備不知是我的親,錯打了你,悔的了不的。若不是,那時就留下你,爭奈有雪娥那賤人在我這裏,不好又安插你的,所以放你去了。落後打發了那賤人,纔使張勝到處尋你不着。誰知打我這府中出去,你在城外做工,流落至於此地位!」經濟道:「不瞞姐姐説,一言難盡。自従與你相别,要娶六姐。我父親死在東京,來遲了,不曾娶成,被武松殺了。聞得你好心,葬埋了他永福寺,我也到那裏燒紙來。在家又把俺娘沒了,剛打發丧事出去,被人坑陷了資本。來家又是大姐死了,被俺丈母那淫婦告了我一狀,牀帳粧奩,都搬的去了,打了一場官司,將房兒賣了,弄的我一貧如洗。多虧了俺爹朋友王杏庵賙濟,把我纔送到臨清晏公廟那裏出家。不料又被光棍打了,拴到咱府中,打了十棍。出去,投親不理,投友不顧,因此在寺内傭工。多虧姐姐掛心,使張管家尋將我來見姐姐一面,恩有重報,不敢有忘!」說到傷心處,兩個都哭了。

正説話中間,只見守備退廳,進入後邊來。左右掀開簾子,守備進來,這陳經濟向前倒身下拜,慌的守備答禮相還說:「向日不知是賢弟,被下人隱瞞,悞有衝撞,賢弟休怪。」經濟道:「不才有玷,一向缺禮,有失親近,望乞恕罪。」又磕下頭去。守備一手拉起,讓他上坐。那經濟乖覺,那裏肯,務要拉下椅兒,旁邊坐了。守備關席,春梅陪他對坐下。須臾,換茶上來喫畢,守備便問:「賢弟貴庚?一向怎的不見?如何出家?」經濟便告說:「小弟虚度二十四歲。俺姐姐長我一歲,是四月二十五日午時生。向因父母雙亡,家業凋丧,妻又没了,出家在晏公廟。不知家姐嫁在府中,有失探望。」守備道:「自従賢弟那日去後,你令姐晝夜憂心,常時啾啾唧唧不安,直到如今。一向使人找尋賢弟不着,不期今日相會,實乃三生有緣!」一面吩咐左右放桌兒,安排酒上來。須臾,擺設許多盃盤,鷄蹄鵝鴨,烹炮蒸煠,湯飯點心,堆滿桌上。銀壺玉盞,酒泛金波。守備相陪叙話,喫至晚來,掌上燈燭方罷。守備吩咐家人周仁,打掃西書院乾淨,那裏書房牀帳都有。春梅拿出兩牀鋪蓋衾枕與他安歇,又撥一個小廝喜兒答應他。又包出兩套紬絹衣服來,與他更換。每日飯食,春梅請進後邊喫。正是:一朝時運至,半點不由人。

光陰迅速,日月如梭,但見:

\begin{myquote}
行見梅花臘底,忽逢元旦新正;

不覺艷杏盈枝,又早新荷貼水。
\end{myquote}

經濟在守備府裏,住了一個月有餘。一日,四月二十五日,春梅的生日,吳月娘那邊買了禮來,一盤壽桃,一盤壽麵,兩隻湯鵝,四隻鮮鷄,兩盤菓品,一罈南酒。玳安穿青衣,拿帖兒送來。守備正在廳上坐的,門上人禀報進去,擡進禮來。玳安遞上帖兒,趴在地下磕頭。守備看了禮帖兒,說道:「多承你奶奶費心,又送禮來。」一面吩咐家人:「收進禮去,討茶來與大官兒喫。把禮帖教小伴當送與你舅收了。封了一方手帕三錢銀子與大官兒,擡盒人錢一百文。拿回帖兒,多上覆。」說畢,守備穿了衣服,就起身出去拜人去了。玳安只顧在廳前伺候,討囘帖兒。只見一個年小的,戴着瓦楞帽兒,穿着青紗道袍,凉鞋淨襪,従角門裏走出來,手中拿着帖兒賞錢,遞與小伴當,一直往後邊去了。「可霎作怪,模樣倒好像陳姐夫一般,他如何卻在這裏?」只見小伴當遞與玳安手帕銀錢,打發出門。到於家中,囬月娘話。見囘帖上寫着「周門龐氏歛袵拜」,月娘便問:「你沒見你姐?」玳安道:「姐姐倒沒見,倒見姐夫來。」月娘笑道:「怪囚,你家倒有恁大姐夫!守備好大年紀,你也叫他姐夫?」玳安道:「不是守備,是咱家的陳姐夫!我初進去,周爺正在廳上。我遞上帖兒,與他磕了頭,他說:『又生受你奶奶送重禮來。』吩咐伴當拿茶與我喫,『把帖兒拿與你舅收了,討一方手帕三錢銀子與大官兒,擡盒人是一百文錢。』說畢,周爺穿衣服,出來上馬,拜人去了。半日,只見他打角門裏出來,遞與伴當囘帖賞賜,他就進後邊去了,我就押着盒擔出來。不是他卻是誰?」月娘道:「怪小囚兒,休胡說白道的!那羔子知道流落在那裏討喫,不是凍死,就是餓死。他平白在那府裏做甚麽?守備認的他甚麽毛片兒,肯招攬下他如何?」玳安道:「奶奶敢和我兩個賭?我看得千眞萬眞,就燒的成灰骨兒,我也認的!」月娘問:「他穿着甚麽?」玳安告訴:「他戴着新瓦楞帽兒,金簪子,身穿着青紗道袍,涼鞋淨襪,喫的好了!」月娘道:「我不信,不信!」這裏說話不題。

卻說陳經濟進入後邊,春梅還在房中鏡臺前搽臉,描畫雙蛾。經濟拿吳月娘禮帖兒與他看,因問:「他家如何送禮來與你,是那裏緣故?」這春梅便把従前已往,清明郊外永福寺撞遇月娘相見的話訴說一遍:後來怎生平安兒偸了解當鋪頭靣,吳巡檢怎生夾打平安兒,追問月娘奸情之事,薛嫂又怎生說人情,守備替他䖏斷了事,「落後他家買禮來相謝,正月裏我往他家與孝哥兒做生日,勾搭連環到如今;他許下我生日,買禮來看我」一節。經濟聽了,把眼瞅了春梅一眼,說:「姐姐,你好沒志氣!想着這賊淫婦,那咱把咱姐兒們生生的拆散開了,又把六姐命丧了,永世千年門裏門外不相逢纔好,反替他説人情兒?那怕那吳典恩追拷着平安小廝,供出奸情來,隨他那淫婦一條䋲子拴去出醜見官,管咱們大腿事!他沒和玳安小廝有姦,怎的把丫頭小玉配與他?有我早在這裏,我斷不教你替他說人情。他是你我仇人,又和他上門往來做甚麽?六月連陰,想他好晴天兒!」幾句話說得春梅閉口無言。春梅道:「過往勾當也罷了。還是我心好,不念舊仇。」經濟道:「如今人好心不得好報哩!」春梅道:「他旣送了禮,莫不白受他的?還等着我這裏人請他去哩。」經濟道:「今後不消理那淫婦了,又請他怎的?」春梅道:「不請他又不好意思的,丢個帖與他,來不來隨他就是了。他若來時,你在那邊書院内,休出去見他。往後咱不招惹他就是了。」經濟惱的一聲兒不言語,走到前邊,寫了帖子。春梅使家人周義去請吳月娘。

月娘打扮出門,教奶子如意兒抱着孝哥兒,坐着一頂小轎,玳安跟隨,來到府中。春梅孫二娘都打扮出來迎接,至後廳相見,叙禮坐下。如意兒抱着孝哥兒,相見磕頭畢。經濟躱在那邊書院内,不走出來。由着春梅孫二娘在後廳擺茶安席遞酒,叫了兩個妓女韓玉釧鄭嬌兒彈唱,俱不必細説。玳安在前邊廂房内管待,只見一個小伴當,打後邊拿出一盤湯飯點心下飯,往西角門書院中走。玳安便問他:「拿與誰喫?」小伴當道:「是與舅喫的。」玳安道:「你舅姓甚麽?」小伴當道:「姓陳。」這玳安賊,悄悄後邊跟着他到西書院,小伴當便掀簾子進去。玳安慢慢打紗窻外往裏張看,卻不是陳姐夫!正在書房牀上歪着,見拿進湯飯點心來,連忙起來。放桌兒正喫,這玳安悄悄走出外邊來,依舊坐在廂房内。直待天晚,家中燈籠來接,吳月娘轎子起身。到家一五一十告訴月娘說:「果然陳姐夫在他家居住。」自従春梅這邊被經濟把攔,兩家都不相往還。正是:誰知豎子多間阻,一念翻成怨恨媒。

自此經濟在府中與春梅暗地勾搭,人都不知。或守備不在,春梅就和經濟在房中喫飯喫酒,閒時下棋調笑,無所不至。守備在家,便使丫頭小廝拿飯往書院與他喫。或白日裏,春梅也常往書院内和他坐,半日方歸後邊來。彼此情熱,俱不必細說。

一日,守備領人馬出巡,正値五月端午佳節,春梅在西書院花亭上置了一桌酒席,和孫二娘陳經濟喫雄黄酒,解粽懽娛。丫鬟侍妾,都兩邊侍奉。當日怎見的蕤賓好景?但見:

\begin{myquote}
盆栽綠柳,瓶插紅榴。水晶簾捲蝦鬚,雲母屏開孔雀。菖蒲切玉,佳人笑捧紫霞觴;角黍堆金,侍妾高擎碧玉盞。食烹異品,菓獻時新。靈符艾虎簪頭,五色絨䋲繫臂。家家慶賞午節,處處懽飲香醪。遨遊身外醉乾坤,消遣壺中閒日月。得多少珮環聲碎金蓮小,紈扇輕搖玉笋柔。
\end{myquote}

春梅令海棠月桂兩個侍妾在席前彈唱。當下直喫到炎光西墜、微雨生凉的時分,春梅拿起大金荷花盃來相勸。酒過數巡,孫二娘不勝酒力,起身先往後邊房中看去了。獨落下春梅和經濟在花亭上喫酒,猜枚行令,你一盃,我一盃。不一時,丫鬟掌上紗燈上來,養娘金匱玉堂,打發金哥兒睡去了。經濟輸了,便走去書房内,躱酒不出來。這春梅先使海棠來請,見經濟不去,又使月桂來,吩咐:「他不來,你好歹與我拉將來!拉不將來,囘來把你這賤人打十個嘴巴。」這月桂走至西書房中,推開門見經濟歪在牀上,推打鼾睡不動。月桂說:「奶奶教我來請你老人家,請不去,要打我哩!」那經濟口裏喃喃呐呐說:「打你不干我事。我醉了,喫不的了!」被月桂用手拉將起來,推着他:「我好歹拉你去,拉不將你去,也不算好漢!」推拉的經濟急了,黑影子裏,佯装着醉,作耍當眞,摟了月桂在懷裏,就親個嘴。那月桂一發上頭上腦說:「人好意呌你,你做大不正,倒做這個營生!」經濟道:「我的兒,你若肯了,那個好意做大不成?」又按着親了個嘴,方走到花亭上。月桂道:「奶奶要打我,還是我把舅拉將來了。」春梅令海棠斟上大鍾,兩個下盤棋,賭酒為楽。當下你一盤,我一盤,熬的丫鬟都打睡去了。春梅又使月桂海棠後邊取茶去。兩個在花亭上,解珮露相如之玉,朱唇點漢署之香。正是:得多少花陰曲檻燈斜照,旁有墜釵雙鳳翹!有詩為證:

\begin{myquote}
花亭歡洽鬢雲斜,粉汗凝香沁絳紗。

深院日長人不到,試看黄鳥啄名花。
\end{myquote}

當下兩個正幹得好,忽然丫鬟海棠送茶來:「請奶奶後邊去,金哥睡醒了,哭着尋奶奶哩。」春梅陪經濟又喫了兩鍾酒,用茶漱了口,然後抽身往後邊來。丫鬟收拾了家活,喜兒扶經濟歸書房寢歇。不在話下。

一日,朝廷敕旨下來,命守備領本部人馬,會同濟州府知府張叔夜,征剿梁山泊賊王宋江,早晚起身。守備對春梅説:「你在家看好哥兒,叫媒人替你兄弟尋上一門親事。我帶他個名字在軍門,若早僥倖得功,朝廷恩典,陞他一官半職,於你面上也有光輝。」這春梅應諾了。遲了兩三日,守備打點行裝,整率人馬,留下張勝李安看家,止帶家人周仁跟了去,不題。

一日,春梅呌將薛嫂兒來,如此這般和他說:「他爹臨去吩咐,替我兄弟尋門親事。你替我尋個門當戶對好女兒,不拘十六七歲的也罷。只要好模樣,脚手兒聰明伶俐些的。他性兒也有些刁厥些兒。」薛嫂兒道:「我不知道他也怎的,何消你老人家吩咐?想着大姐那等的還嫌哩!」春梅道:「若是尋的不好,看我打你耳刮子不打。我要趕着他呌小妗子兒哩,休要當耍子兒!」說畢,春梅令丫鬟擺茶與他喫。只見陳經濟進來喫飯,薛嫂向他道了萬福說:「姑夫,你老人家一向不見,在那裏來?且喜呀,剛纔奶奶吩咐,教我替你老人家尋個好娘子,你怎麽謝我?」那陳經濟把臉兒蛙着不言語。薛嫂道:「老花子,怎的不言語?」春梅道:「你休叫他姑夫,那個已是揭過去的帳了,你只叫他陳舅就是了。」薛嫂道:「這該打我這片子狗嘴,只要叫錯了。往後趕着你只叫舅爺罷!」那陳經濟忍不住撲喫的笑了,說道:「這個纔可到我心上。」那薛嫂撒風撒癡,趕着打了他一下,說道:「你看老花子說的好話兒!我又不是你影射的,怎麽可在你心上?」連春梅也笑了。

不一時,月桂安排茶食與薛嫂喫了,提着花箱兒出來,說道:「我替你老人家用心踏看,有人家相應好女子兒,就來説。」春梅道:「財禮羹果,花紅酒禮,頭面衣服,不少他的。只要好人家好女孩兒,方可進入我門來。」薛嫂道:「我曉得,管情應的你老人家心便了。」良久,經濟喫了飯,往前邊去了。薛嫂兒還坐着,問春梅:「他老人家幾時來的?」春梅便把出家做道士一節說了:「我尋得他來,做我個親人兒。」薛嫂道:「好好,你老人家有後眼。」又道:「前日你老人家好的日子,說那頭他大娘來做生日來?」春梅道:「先送禮來,然後纔使人送帖兒請他。坐了一日去了。」薛嫂道:「我那日在一個人家鋪牀,整亂了一日,心内要來,急的我了不的。」又問:「他陳舅也見他那頭大娘來?」春梅道:「他肯下氣見他?為請他,好不和我亂成一塊。嗔我替他家說人情,說我沒志氣:『那怕吳典恩打着小廝,攀扯他出官纔好。管你腿事,你替他尋分上!想着他昔日好情兒?』」薛嫂道:「他老人家也説的是。及到其間,你做人不計舊仇。」春梅道:「咱旣受了他禮,不請他來坐坐兒又使不的。寜可教他不仁,休要咱不義。」薛嫂道:「怪不的你老人家有恁大福,你的心忒好了!」當下薛嫂兒說了半日話,提着花箱兒拜辭出門。

過了兩日,先來説城裏朱千戶家小姐,今年十五歲,也好陪嫁,只是沒了娘的兒了,春梅嫌小,不要。又說應伯爵第二個女兒,年二十二歲,春梅又嫌應伯爵死了,在大爺手内聘嫁,沒甚陪送,也不成,都回出婚帖兒來。又遲了幾日,薛嫂兒送花兒來,袖中取出個婚帖兒,大紅緞子上寫着開緞舖葛員外家大女兒,「年二十歲,屬鷄的,十一月十五日子時生,小字翠屏,生的上畫兒般模樣兒,五短身材,瓜子面皮,溫柔典雅,聰明伶俐。針指女工,自不必説;父母俱在,有萬貫錢財,在大街上開緞子舖,走蘇杭南京,無比好人家,都是南京牀帳箱籠。」春梅道:「旣是好,成了這家子的罷。」就教薛嫂兒先通信去,那薛嫂兒連忙說去了。正是:欲向繡房求艷質,須憑紅葉是良媒。有詩為證:

\begin{myquote}
天僊機上繫香羅,千里姻緣竟足多。

天上牛郎配織女,人間才子伴嬌娥。
\end{myquote}

這裏薛嫂通了信來,葛員外家知是守備府裏,情願做親,又使一個張媒人同說媒。春梅這裏備了兩擡茶葉饊餅羹果,教孫二娘坐轎子往葛員外家插定,女兒帶戒指兒。囬來,對春梅說:「果然好個女子,生的一表人材,如花似朶,人家又相當。」春梅這裏擇定吉日,納綵行禮,十六盤羹果茶餅,兩盤上頭麵,二盤珠翠,四擡酒,兩牽羊,一頂䯼髻,全付金銀頭面簪環之類,兩件羅緞袍兒,四季衣服,其餘綿花布絹,二十兩禮銀,不必細說。陰陽生擇在六月初八日,准娶過門。春梅先問薛嫂兒:「他家那裏有陪牀使女沒有?」薛嫂兒道:「牀帳粧奩,描金箱厨都有,只沒有使女陪牀。」春梅道:「咱這裏買一個十三四歲丫頭子,與他房裏使喚,掇桶子倒水,方便些。」薛嫂道:「有兩個人家賣的丫頭子,我明日帶一個來。」到次日,果然領了一個丫頭,說是商人黄四家兒子房裏使的丫頭:「今年纔十三歲。黄四因用下官錢糧,和李三家,還有咱家出去的保官兒,都為錢糧,拿在監裏追贓。監了一年多,家産盡絶,房兒也賣了。李三先死,拿兒子李活監着。咱家保官兒那兒子僧寳兒,如今流落在外,與人家跟馬哩。」春梅道:「是來保?」薛嫂道:「他如今不叫來保,改了名字,呌湯保了。」春梅道:「這丫頭是黄四家丫頭,要多少銀子?」薛嫂道:「只要四兩半銀子,緊等着要交贓去。」春梅道:「甚麽四兩半,與他三兩五錢銀子留下罷。」一面就交了三兩五錢雪花官銀與他,寫了文書,改了名字,喚做金錢兒。

話休饒舌。又早到六月初八。春梅打扮珠翠鳳冠,穿通袖大紅袍兒,束金鑲碧玉帶,坐四人大轎,鼓楽燈籠,娶葛家女子,奠雁過門。陳經濟騎大白馬,揀銀鞍轡,青衣軍牢喝道,頭戴儒巾,穿着青緞圓領,脚下粉底皂靴,頭上簪着兩枝金花。正是:久旱逢甘雨,他鄉遇故知,洞房花燭夜,金榜掛名時。一番拆洗一番新!到守備府中,新人轎子落下,戴着大紅銷金蓋袱,添粧含飯,抱着寳瓶,進入大門。陰陽生引入畫堂,先參拜家堂,然後歸到洞房。春梅安他兩口兒坐帳,然後出來。陰陽生撒帳畢,打發喜錢出門,鼓手都散了。經濟與這葛翠屏小姐坐了囬帳,騎馬打燈籠,往岳丈家謝親,喫的大醉而歸。晚夕,女貌郎才,未免燕爾新婚,交姤雲雨。正是:得多少春點杏桃紅綻蕊,風欺楊柳綠翻ん。有詩為證:

\begin{myquote}
近覩多情風月標,敎人無福也難消。

風吹列子歸何處,夜夜嬋娟在柳梢。
\end{myquote}

當夜經濟與這葛翠屏小姐,倒且是合得着,兩個被底鴛鴦,帳中鸞鳳,如魚似水,合巹歡娛。三日完飯,春梅在府廳後堂張筵掛綵,鼓楽笙歌,請親眷喫會親酒,俱不必細說。每日春梅喫飯,必請他兩口兒同在房中一處喫,彼此以姑妗稱之,同起同坐。丫頭養娘,家人媳婦,誰敢道個不字?原來春梅收拾西廂房三間,與他做房,裏面鋪着牀帳,翻的雪洞般齊整,垂着簾幃。外邊西書院是他書房,裏面亦有牀榻、几席、古書,并守備往來書柬拜帖,并各䖏遞來手本揭帖,都打他手裏過,或登記簿籍,或御使印信。筆硯文房都有,架閣上堆滿書集。春梅不時常出來書院中,和他閒坐說話,兩個暗地交情,非止一日。正是:

\begin{myquote}
朝陪金谷宴,暮伴綺樓娃;

休道歡娛處,流光逐落霞。
\end{myquote}

畢竟未知後來何如,且聽下囘分解。

