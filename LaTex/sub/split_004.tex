\chapter*{金瓶梅序}
\addcontentsline{toc}{chapter}{東吴弄珠客《金瓶梅序》}
\markboth{金瓶梅序}{金瓶梅序}
\thispagestyle{empty}

《金瓶梅》,穢書也。袁石公亟稱之,亦自寄其牢騷耳,非有取於《金瓶梅》也。然作者亦自有意,蓋為世戒,非為世勸也。如諸婦多矣,而獨以潘金蓮李瓶兒春梅命名者,亦楚《檮杌》之意也。蓋金蓮以姦死,瓶兒以孽死,春梅以淫死,較諸婦為更慘耳。借西門慶以描畫世之大凈,應伯爵以描畫世之小醜,諸淫婦以描畫世之醜婆凈婆,令人讀之汗下。蓋為世戒,非為世勸也。余嘗曰:讀《金瓶梅》而生憐憫心者,菩薩也;生畏懼心者,君子也;生歡喜心者,小人也;生效灋心者,乃禽獸耳。余友人褚孝秀,偕一少年同赴歌舞之筵,衍至〈霸王夜宴〉,少年垂涎曰:「男兒何可不如此!」孝秀曰:「也只為這烏江設此一着耳。」同座聞之,歎為有道之言。若有人識得此意,方許他讀《金瓶梅》也。不然,石公幾為導淫宣慾之尤矣!奉勸世人,勿為西門慶之後車可也。\\*

\begin{quotation}\begin{flushright}萬曆丁巳季冬東吴弄珠客漫書於金閶道中。\end{flushright}\end{quotation}

