\includepdf[pages={119,120},fitpaper=false]{tst.pdf}
\chapter*{第六十囬 \\李瓶兒因暗氣惹病 西門慶立緞舖開張}
\addcontentsline{toc}{chapter}{第六十囬 李瓶兒因暗氣惹病 西門慶立緞舖開張}
\markboth{{\titlename}卷之六}{第六十囬 李瓶兒因暗氣惹病 西門慶立緞舖開張}


\begin{myquote}
赤䋲緣盡再難期,造化無端敢恨誰!

殘淚驚秋和葉落,断魂隨月到窻遲。

金風拂面思兒䖏,玉燭成灰墮淚時。

任是肝腸如鐵石,不生悲也自生悲。
\end{myquote}

話説當日孫雪娥吳銀兒兩個,在旁邊勸解了李瓶兒一囬云云,到後邊去了。那潘金蓮見孩子沒了,李瓶兒死了生兒,每日抖擻精神,百般的稱快。指着丫頭罵道:「賊淫婦,我只説你日頭常晌午,卻怎的今日也有錯了的時節!你斑鳩跌了彈——也嘴答谷了;春櫈折了靠背兒——沒的倚了;王婆子賣了磨——推不的了;老鴇子死了粉頭——沒指望了。卻怎的也和我一般?」李瓶兒這邊屋裏分明聽見,不敢聲言,背地裏只是掉淚。着了這暗氣暗惱,又加之煩惱憂戚,漸漸心神恍亂,夢魂顛倒,且每日茶飯都減少了。自従墳上葬埋了官哥兒囬來,第二日吳銀兒就家去了。老馮領了十三歲丫頭來,賣與孫雪娥房中使喚,要了五兩銀子,改名翠兒,不在話下。這李瓶兒一者思念孩兒,二者着了重氣,把舊時病症又發起來,照舊下邊經水淋漓不止。西門慶請任醫官來看一遍,討將薬來,喫下去如水澆石一般,越喫薬越旺。那消半月之間,漸漸容顔頓减,肌膚消瘦,而精彩丰標無復昔時之態矣。正是:肌骨大都無一把,如何禁架許多愁!

一日,九月初旬,天氣凄凉,金風淅淅。李瓶兒夜間獨宿在房中。銀牀枕冷,紗窻月浸。不覺思想孩兒,欷歔長歎,似睡不睡,恍恍然恰似有人彈的窻櫺響。李瓶兒呼喚丫鬟,都睡熟了不答。乃自下牀來,倒靸弓鞋,翻披繡襖,開了房門,出戶視之。彷彿見花子虚抱着官哥兒呌他:新尋了房兒,同去居住。這李瓶兒還捨不的西門慶,不肯去,雙手就去抱那孩兒,被花子虚只一推,跌倒在地。撒手驚覺,卻是南柯一夢。嚇了一身冷汗,嗚嗚咽咽,直哭到天明。正是:有情豈不愛,着相自家迷。有詩為證:

\begin{myquote}
纖纖新月照銀屛,人在幽閨欲断魂。

益悔風流多不足,須知恩愛是愁根!
\end{myquote}

那時來保南京貨船又到了,使了後生王顯上來取單税銀兩。西門慶這裏寫書,差榮海㧱一百兩銀子,又具羊酒金緞禮物謝錢主事,就説:「此船貨過税,還望青目一二。」家中收拾舖面完備,又擇九月初四日開張。就是那日卸貨,連行李共裝二十大車。那日親朋遞果盒掛紅者約有三十多人。喬大戶呌了十二名吹打的樂工,雜耍撮弄;西門慶這裏,李銘吳惠鄭春三個小優兒彈唱。甘夥計與韓夥計都在櫃上發賣,一個看銀子,一個講説價錢。崔本專管收生活,不拘經紀、買主進來,讓進去,每人飲酒二盃。西門慶穿大紅,冠帶着。燒罷紙,各親友都遞菓盒,把盞畢,後邊廳上安放十五張桌席;五菓五菜,三湯五割,重新遞酒上坐,鼓樂喧天。那日夏提刑家差人送禮花紅來。西門慶回了禮物,打發去了。在座者有喬大戶、吳大舅、吳二舅、花大舅、沈姨夫、韓姨夫、吳道官、倪秀才、溫葵軒、應伯爵、謝希大、常時節,還有李智、黄四、傅自新等衆夥計主管,並街坊隣舍,都坐滿了席面。三個小優兒在席前唱了一套〔南呂·紅衲襖〕:「混元初生太極」云云。須臾,酒過五巡,食割三道,下邊樂工吹打彈唱,雜耍百戲過去,席上觥籌交錯。當日應伯爵謝希大飛起大鍾來,盃來盞去,飲至日落時分。把衆人打發散了,西門慶只留下吳大舅、沈姨夫、倪秀才、溫葵軒、應伯爵、謝希大,従新擺上桌席,留後坐。那日新開張,夥計攢帳,就賣了五百餘兩銀子。西門慶滿心歡喜,晚夕收了舖面,把甘夥計、韓夥計、傅夥計、崔本、賁四,連陳經濟都邀來到席上飲酒。吹打良久,把吹打樂工打發去了,止留下三個小優兒在席前唱。

那應伯爵坐了一日,喫的已醉上來。出來前邊解手,呌過李銘,問李銘:「那個紮包髻兒的清俊小優兒,是誰家的?」李銘道:「二爹不知道?」因掩口説道:「他是鄭奉的兄弟鄭春。前日爹在裏邊他家喫酒,請了他姐姐愛月兒了。」伯爵道:「眞個?怪道前日上紙送殯都有他!」於是歸到酒席上,向西門慶道:「哥,你又恭喜!又招了小舅子了。」西門慶笑道:「怪狗才,休要胡説。」一面呌過王經來:「斟與你應二爹一大盃酒。」伯爵向吳大舅説道:「老舅,你怎麽説?這鍾罰的我沒名。」西門慶道:「我罰你這狗才一個出位妄言!」那伯爵低頭想了想兒,呵呵笑了,道:「不打緊處,等我喫,我喫!死不了人。」又道:「我從來喫不得啞酒,你呌鄭春上來唱個兒我聽,我纔罷了。」當下三個小優,一齊上來彈唱。伯爵令李銘吳惠下去:「不要你兩個。我只要鄭春單彈着箏兒,只唱個小小曲兒我下酒罷。」謝希大呌道:「鄭春,你過來,依着你應二爹唱。」西門慶道:「和花子講過:有一個曲兒喫一鍾酒。」於是玳安旋取了兩個大銀鍾,放在應二面前。那鄭春款按銀箏,低低唱〔清江引〕道:

\begin{myquote}
「一個姐兒十六七,見一對蝴蝶戲。香肩靠粉墻,春笋彈珠淚。喚梅香,趕他去別䖏飛。」
\end{myquote}

鄭春唱了個:「請酒!」伯爵剛纔飲訖,那玳安在旁連忙又斟上一盃酒。鄭春又唱道:

\begin{myquote}
「轉過雕欄正見他,斜倚定荼䕷架。佯羞整鳳釵,不説昨宵話。笑吟吟,掐將花片兒打。」
\end{myquote}

伯爵喫過,連忙推與謝希大,説道:「罷,我是成不的,成不的!這兩大鍾,把我就打發的了。」謝希大道:「儍化子,你喫不的,推於我來,我是你家有ず的蠻子?」伯爵道:「儍花子,我明日就做了堂上官兒,少不的是你替。」西門慶道:「你這狗才,到明日只好做個韶武。」伯爵笑道:「儍孩兒,我做了韶武,把堂上讓與你就是了。」西門慶笑令玳安兒:「㧱磕瓜來打這賊花子。」那謝希大悄悄向他頭上打了一個響瓜兒,説道:「你這花子,溫老先生在這裏,你口裏只恁胡説。」伯爵道:「溫老先兒他斯文人,不管這閒事。」溫秀才道:「二公與我這東君老先生原來這等厚。酒席中間,誠然不如此也不楽。悦在心,楽主發散在外,自不覺手之舞之,足之蹈之如此。」

座上沈姨夫向西門慶説:「姨夫,不是這等。請大舅上席還行個令兒,或擲骰,或猜枚,或看牌,不拘詩詞歌賦,頂眞續麻,急口令,説不過來喫酒。這個庶幾均匀,彼此不亂。」西門慶道:「姨夫説的是。」先斟了一盃,與吳大舅起令。吳大舅㧱起骰盆兒來,説道:「列位,我行一令,説差了,罰酒一盃。先用一骰,後用兩骰,遇點飲酒:

\begin{myquote}
一,百萬軍中捲白旗;二,天下豪傑少人知;

三,秦王斬了余元帥;四,駡得將軍無馬騎;

五,唬得吾今無口應;六,衮衮街頭脱去衣;

七,皂人頭上無白髮;八,分屍不得帶刀歸;

九,一丸好薬無人點;十,千載終湏一撇離。」
\end{myquote}

吳大舅擲畢,遇有兩點,飲過酒。該沈姨夫起令,説道:「用一骰六擲,遇點飲酒。」説道:

\begin{myquote}
「天象六色地像雙,人數推來中二紅,

三見巫山梅五出,算來能有幾人通?」
\end{myquote}

當下只遇了個四紅,飲過一盃,過盆與溫秀才。秀才道:「我學生奉令了。遇點要一花名,名下接《四書》一句頂眞:

\begin{myquote}
一擲一點紅,紅梅花對白梅花;

二擲並頭蓮,蓮漪戲彩鴛;

三擲三春柳,柳下不整冠;

四擲狀元紅,紅紫不以為褻服;

五擲臘梅花,花迎劔珮星初落;

六擲滿天星,星辰之遠也。」
\end{myquote}

溫秀才只遇了一鍾酒,該應伯爵行令。伯爵道:「我在下一個字也不識,行個急口令兒罷:

\begin{myquote}
一個急急腳腳的老小,左手㧱着一個黄荳巴斗,右手㧱着一條綿花叉口,望前只管跑走。撞着一個黄白花狗,咬着那綿花叉口。那急急脚脚的老小,放下那左手提的那黄荳巴斗,走向前去打黄白花狗。不知手鬬過那狗,狗鬬過那手?」
\end{myquote}

西門慶笑罵道:「你這賊謅斷了腸子的天殺的,誰家一個手去鬬狗來!一口不被那狗咬了?」伯爵道:「誰呌他不㧱個棍兒來?我如今抄化子不見了拐棒兒,受狗的氣了!」謝希大道:「大官人,你看花子倒了架,説他是花子。」西門慶道:「該罰他一鍾,不成個令。謝子純,你行罷。」謝希大道:「我這令兒比他更妙。説不過來,罰一鍾:

\begin{myquote}
墻上一片破瓦,墻下一疋騾馬。落下破瓦,打着騾馬。不知是那破瓦打傷騾馬,不知是那騾馬踏碎了破瓦?」
\end{myquote}

伯爵道:「你笑話我的令不好,你這破瓦倒好?你家娘子兒劉大姐就是個騾馬,我就是個破瓦。俺兩個破磨對瘸騾。」謝希大道:「你家那杜蠻婆老淫婦,撒把黑荳只好喂猪拱,狗也不要他!」兩個人鬬了回嘴,每人罰了一鍾。該傅自新行令。傅自新道:「小人行個江湖令,遇點飲酒,先一後二:

\begin{myquote}
一舟二櫓,三人搖出四川河;五音六律,七人齊唱八僊歌。九十春光齊賞翫,十一十二慶元和。」
\end{myquote}

擲畢,皆不遇。吳大舅道:「總不如傅黟計這個令兒行得切實些。」伯爵道:「太平鍾也該他喫一盃兒。」於是親下席來,斟了一盃與傅自新喫。如今該韓夥計。韓道國道:「老爹在上,小人怎敢占先?」西門慶道:「你們行過,等我行罷。」於是韓道國道:「頭一句要天上飛禽,第二句要菓名,第三句要骨牌名,第四句要一官名,俱要貫串,遇點照席飲酒。」説:

\begin{myquote}
「天上飛來一僊鶴,落在園中喫鮮桃,

卻被孤紅㧱住了,將去獻與一提學。

天上飛來一鷂鷹,落在園中喫朱櫻,

卻被二姑㧱住了,將去獻與一公卿。

天上飛來一老鸛,落在園中喫菱芡,

卻被三綱㧱住了,將去獻與一通判。

天上飛來一斑鳩,落在園中喫石榴,

卻被四紅㧱住了,將來獻與一戶侯。

天上飛來一錦鷄,落在園中喫苦株,

卻被五嶽㧱住了,將來獻與一尚書。

天上飛來一淘鵝,落在園中喫蘋婆,

卻被緑暗㧱住了,將來獻與一照磨。」
\end{myquote}

擲畢,該西門慶擲。西門慶道:「我只擲四擲,遇點飲酒:

\begin{myquote}
六口載成一點霞,不論春色見梅花,

摟抱紅娘親個嘴,抛閃鶯鶯獨自嗟。」
\end{myquote}

擲到遇紅一句,果然擲出個四來。應伯爵看見,説道:「哥,今年上冬,管情高轉加官,主有慶事。」於是斟了一大盃酒與西門慶,一面喚李銘等三個上來彈唱。頑耍至更闌方散。西門慶打發小優兒出門,看着收了家伙。派定韓道國、甘夥計、崔本、來保,四人輪流上宿,吩咐仔細門戶,就過那邊去了。一宿晚景不題。

卻説次日,應伯爵領了李智黄四來交銀子,説:「此遭只関了一千四百五六十兩銀子,不夠還人,只挪了這三百五十兩銀子與老爹。等下遭銀子関出來再找完,不敢遲了。」伯爵在旁,又替他説了兩句羙言。西門慶把銀子教陳經濟來㧱天平兑收明白,打發去了。銀子還擺在桌上。西門慶因問伯爵道:「常二哥説,他房子尋下了,前後四間,只要三十五兩銀子就賣了。他來對我説,正値小兒病重了,我心裏正亂着哩,打發他去了。不知他對你説來不曾?」伯爵道:「他對我説來。我説你去的不是了,他乃郎不好,他自亂亂的,有甚麽心緒和你説話?你且休囬那房主兒,等我見哥替你提就是了。」西門慶聽了,便道:「也罷,你喫了飯,㧱一封五十兩銀子,今日是個好日子,替他把房子成了來罷。剩下的,敎常二哥門面開個小本舖兒,月間賺的幾錢銀子兒,夠他兩口兒盤攪過來就是了。」伯爵道:「此是哥下顧他了。」不一時,放桌兒,擺上飯來。西門慶陪他喫了飯,道:「我不留你。你拿了這銀子去,替他幹幹這勾當去罷。」伯爵道:「你這裏還敎個大官,和我兩個㧱這銀子去。」西門慶道:「沒的扯淡,你袖了去就是了。」伯爵道:「不是這等説。今日我還有小事去。實和哥説,家表弟杜三哥生日,早晨我送了些禮兒去,他使小廝來,請我後晌坐坐,我不得來回你。敎個大官兒跟了去,成了房子,我敎大官兒好來回你。」説罷,西門慶道:「若是恁説,教王經跟了你去罷。」一面呌了王經,跟伯爵去了。

到了常時節家,常時節正在家。見伯爵至,讓進裏面坐。伯爵㧱出銀子來與常時節看,説:「大官人如此如此,敎我同你今日成房子去。我又不得閒,杜三哥請我喫酒。我如今了畢你的事,我方纔得去。所以呌大官兒跟了我來,成了房子,我不囬他爹話去,教他囬囬便了。」常時節連忙呌渾家快看茶來,説道:「哥的盛情,誰肯!」一面喫畢茶,呌了房中人來,同到新市街,兑與賣主銀子,寫立房契。伯爵吩咐與王經,歸家囬西門慶話。剩的銀,教與常時節收了。他便與常時節作别,往杜家喫酒去了。西門慶看了文契,還使王經:「送與你常二叔收了。」不在話下。正是:

\begin{myquote}
求人須求大丈夫,濟人須濟急時無。

一切萬般皆下品,誰知陰德是良圖。
\end{myquote}

正是:三光有影遺誰翳?萬事無根只自生。

畢竟未知後來何如,且聽下回分解。

\part*{夢梅館校本《金瓶梅詞話》卷之七}
\addcontentsline{toc}{part}{夢梅館校本《金瓶梅詞話》卷之七}

