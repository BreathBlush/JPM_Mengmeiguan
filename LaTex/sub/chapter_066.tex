\includepdf[pages={131,132},fitpaper=false]{tst.pdf}
\chapter*{第六十六囬 \\翟管家寄書致賻 黄眞人煉度薦亡}
\addcontentsline{toc}{chapter}{第六十六囬 翟管家寄書致賻 黄眞人煉度薦亡}
\markboth{{\titlename}卷之七}{第六十六囬 翟管家寄書致賻 黄眞人煉度薦亡}


\begin{myquote}
八面明窻次第開,佇看環珮下瑤臺。

閨門春色連新柳,嶺角寒香帶早梅。

影動花梢明月上,風敲竹徑故人來。

佳人留下鴛鴦錦,都付東君仔細裁。
\end{myquote}

話説西門慶那日陪吳大舅應伯爵等飲酒中間,因問韓道國:「客夥中標船幾時起身?咱好收拾打包。」韓道國道:「昨日有人來會,也只在二十四日開船。」西門慶道:「過了二十念經,打包便了。」伯爵問:「這遭起身那兩位去?」西門慶道:「三個人都去。明年先打發崔大哥押一船杭州貨來,他與來保還徃松江下,各處置買些布貨來發賣。家中緞貨紬絹都還有哩。」伯爵道:「哥主張極妙,常言道:要的般般有,纔是買賣。」説畢,已至起更時分。吳大舅起身説:「姐夫,你連日辛苦。俺們酒已夠了,告囬,你可歇息歇息。」西門慶不肯,還要留住,令小優兒奉酒唱曲,每人喫三鍾,纔放出門。西門慶賞了小優四人六錢銀子,再三不敢接,說:「宋爺出票,呌小的們來,官身如何敢受老爺重賞?」西門慶道:「雖是官差,此是我賞你,怕怎的!」四人方磕頭領去,不在話下。西門慶便歸後邊歇去了。

次日早起,徃衙門中去。早有玉皇廟吳道官差了一個徒弟,兩名鋪排來,在大廳上鋪設壇場。上安三清四御,中安太乙救苦天尊,兩邊東嶽、酆都,下列十王九幽,冥曹幽壤;監壇神虎二大元帥,桓劉吳魯四大天君,太陰神后,七眞玉女,側懸冥司提魂攝魄一十七員神將。内外壇場,鋪設的齊齊整整;香花燈燭,擺列的燦燦輝輝。爐中都焚百合名香,周圍高懸弔挂。經筵羅列,幕幃銷金;法鼓高架,彩雲旋繞。西門慶來家看見,心中大喜,打發徒弟,鋪排齋食喫了,囬廟中去了。隨即令溫秀才寫帖兒,請喬大戶、吳大舅、吳二舅、花大舅、沈姨夫、孟二舅、應伯爵、謝希大、常時節、吳舜臣,許多親眷並堂客,明日念經。家中廚役落作治辦齋供,不題。

次日五更,道衆皆挨門進城,到於西門慶家,叫開門,進入經壇内,明起燈燭,沐手焚香,打動響樂,諷誦諸經,敷演生神玉章。鋪排大門首挂起長旛,懸弔榜文,兩邊黄紙門對一聯,大書:「東極垂慈,僊識乘晨而超登紫府;南丹赦罪,凈魄受煉而逕上朱陵。」榜上寫着:

\begin{myquote}[\markfont]
「大宋國山東東平府清河縣某坊居住,奉

道追修孝夫信官西門慶,合家孝眷人等,即日皈誠,上干慈造。意者伏為室人李氏之靈,存日陽年二十七歲,元命辛未相正月十五日午時受生,大限於政和七年九月十七日丑時分身故。伏以伉儷情深,嘆鳳鸞之先別;閨門月冷,嗟琴瑟以斷鳴。徒追悼以何堪,憶音容而緬想。光陰易逝,五七俄臨。欲拔幽魂,敬陳丹悃。謹以今月二十日仗延官道,爰就孝居,建盟眞煉度齋壇,庸頒玉簡;演九轉生神寳範,奏啓琅函。迓獅馭以垂光,金燈破暗;降龍章而滅罪,鐵柱停酸。爰至深宵,度綵橋而鳴玉珮;頻餐沆瀣,登碧落而謁金眞。伏願:玉陛垂慈,青宫降鑒,廣覃惻隱之仁,大賜提撕之力,亾魂早超逍遙之境,滯爽咸登極楽之天。存殁眷屬,均沐休祥;宗親人等,同登道岸。凡預薦修,悉希元化,故榜。 

政和年月日榜。

《上清大洞經籙》九天金闕大夫、神霄玉府上筆判雷霆諸司府院事、清微弘道體玄養素崇敎高士、領太乙宫提點、皇壇知磬兼管天下道教事、高功黄元白奉行。」
\end{myquote}

大廳經壇,懸挂齋題二十字,大書:

\begin{myquote}[\markfont]
「青玄救苦頒符告簡五七轉經水火煉度薦揚齋壇。」
\end{myquote}

即日黄眞人穿大紅,坐牙轎,繫金帶,左右圍隨,儀従喧呵,日高方到。吳道官率衆接至壇所,行畢禮,然後西門慶着素衣絰巾拜見,遞茶畢。洞案傍邊,安設經筵法席,大紅銷金桌幃,粧花椅褥,二道童侍立左右。黄眞人儀偉容貌,戴王冠,韜以烏紗,穿大紅斗牛衣服,靸烏履。發文書之時,西門慶備金緞一疋僉字。登壇之時,換了九陽雷巾,大紅金雲白鶴法氅,與袖飛鬣,脚下白綾軟襪,朱紅登雲朝舄。朝外建天地亭,張兩把金傘蓋。金童揚煙,玉女散花,執幢捧節。監壇神將,三界符使,四直功曹,城隍社令,土地祇迎,無不畢陳。高功香案上列五色天皇號令,召雷皂纛,天蓬玉尺,七星寳劍,淨水法盂。先是表白宣畢齋意,齋官沐手上香拜懺,二人飄手爐向外三信禮召請。然後高功擊令焚香,蕩穢淨壇,飛符召將,関發一應文書符命,啟奏三天,告盟十地。三獻禮畢,打動音楽,化財行香。西門慶與陳經濟執手爐跟隨,排軍喝路,前後四把銷金傘,三對纓絡挑搭。孝眷列於大門首,孤魂棚建於街上。湯飯淨供,委付四名排軍看守。行香回來,安請監齋壇已畢,在捲棚擺齋。那日各親友街隣夥計,送茶者絡繹不絶。西門慶悉令玳安王經收記,打發回盒人銀錢。

早晨開啟,請三寳證盟,頒告符簡,破獄召亡。又動音楽,徃李瓶兒靈前攝召,引魂朝參玉陛,傍設几筵,聞經悟道。高功搭高座,演《九天生神經》,焚燒太乙東嶽酆都十王冠帔雲馭。午朝,高功冠裳,步罡踏斗,拜進朱表,逕達東極青宫,遣差神將,飛下羅酆。原來黄眞人年約三旬,儀表非常,粧束起來,午朝拜表,儼然就是個活神僊。端的生成甚模樣?但見:

\begin{myquote}
星冠攢玉葉,鶴氅縷金霞。神清似長江皓月,貌古如太華喬松。踏罡朱履步丹霄,步虚琅函浮瑞氣。長髯廣頰,修行到無漏之天;皓齒明眸,佩籙掌五雷之令。三島十洲存性到,洞天福地出神遊。高餐沆瀣,靜裏朝元。三更步月鸞聲遠,萬里乘雲鶴背高。就是都僊太史臨凡世,廣惠眞人降下方。
\end{myquote}

拜了表文,吳道官當壇頒生天寳籙,神虎玉劄。行畢午香,回來捲棚内擺齋。黄眞人前大桌面定勝,吳道官等稍加差小,其餘散衆俱平頭桌席。黄眞人、吳道官,皆襯緞尺頭,四位披花,四疋絲紬;散衆各布一疋。桌面俱令人擡送廟中,散衆各有手下徒弟收入箱中,不必細説。喫畢午齋,謝了西門慶,都徃花園各亭臺洞内遊玩散食去了。一面收下家伙,従新桌上擺下齋饌上來,請吳大舅等衆親朋夥計來喫。

正喫之間,忽報東京翟爺那裏差人來下書。西門慶即出到廳上,請來人進入。只見是府前承差幹辦,青衣窄袴,萬字頭巾,乾黄靴,全付弓箭,向前施禮。西門慶答還下禮。那人向身邊取出書來遞上,書内封折賻儀銀十兩。問來人上姓,那人道:「小人姓王名玉,蒙翟爺差遣,送此書來。不知老爹這邊有喪事,安老爹書到京纔知道。」西門慶問道:「你安老爹書幾時到來?」那人說:「安老爹書十月纔到京。因催皇木一年已滿,陞都水司郎中。如今又奉勅修理河道,直到工完囬京。」西門慶問了一遍,即令來保廂房中管待齋飯,吩咐明日來討囬書。那人問:「韓老爹在那裏住?宅内捎信在此。小的見了,還要趕徃東平府下書去。」西門慶即喚出韓道國來見那人。陪喫齋食畢,同徃家中去了。西門慶拆看書中之意,於是乘着喜歡,將書㧱到捲棚内教溫秀才看,說:「你照此修一封回書答他,就捎寄十方縐紗汗巾,十方綾汗巾,十副揀金挑牙,十個烏金酒盃,作囬奉之禮。他明日就來取回書。」溫秀才接過書來觀看,其書曰:

\begin{myquote}[\markfont]
\hspace*{4em}「寓京都眷生翟謙頓首,書奉

即擢大錦堂西門四泉親家大人門下:自京邸執手話别之後,未得從容相叙,心甚歉然。其領敎之意,生已與家

老爺前悉陳之矣。邇者因安鳳山書到,方知老親家有鼓盆之嘆,但不能一弔為恨,奈何奈何!伏望以禮節哀可也。外具賻儀,少表微忱,希莞納。又久仰貴任榮修德政,擧民有五袴之歌,境内有三留之譽。今歲考績,必有甄陞。昨日神運都功兩次工上,生已對

老爺說了,安上親家名字。工完題奏,必有恩典,親家必有掌刑之喜。夏大人年終類本,必轉京堂,指揮列銜矣。謹此預報,伏惟高照,不宣。{\kaishu(附云)}此書可自省覽,不可使聞之於渠。謹密!謹密!{\kaishu(又云)}楊老爺前月二十九日卒於獄。

\raggedleft{{\kaishu(下書)}冬上澣具。」}
\end{myquote}

卻説溫秀才看畢,纔待袖,早被應伯爵取過來,觀看了一遍,還付與溫秀才收了,說道:「老先生把囬書千萬加意做好些,翟公府中人才極多,休要敎他笑話。」溫秀才道:「貂不足,狗尾續。學生匪才,焉能在班門中弄大斧,不過乎塞責而已。」西門慶道:「老先生他自有個主意,你這狗才曉的甚麽!」須臾,喫罷午齋,西門慶吩咐來興兒打發齋饌,送各親眷街隣家;又使玳安回院中李桂姐、吳銀兒、鄭愛月兒、韓金釧兒、洪四兒、齊香兒,六家香儀人情禮去,每家還答一疋大布、一兩銀子;後晌就呌李銘、吳惠、鄭奉,三個小優兒來伺候。

良久,道衆陞壇,發擂,上朝,拜懺,觀燈,解壇,送聖。天色漸晚,比及設了醮,就有起更天氣。門外花大舅被西門慶留下,已不去了。喬大戶、沈姨夫、孟二舅,告辭先回家。止有吳大舅、二舅、應伯爵、謝希大、溫秀才、常時節,並衆夥計在此,晚夕觀看水火煉度。就在大廳棚内搭高座、扎綵橋、安設水池火沼,放擺斛食。李瓶兒靈位另有几筵幃幕,供獻齊整,傍邊一首魂旛;一首紅旛,一首黄旛,上書「制魔保擧」、「受煉南宫」,先是,道衆音楽兩邊列坐,持節捧盂劍四個道童,侍立法座兩邊。黄眞人頭戴黄金降魔冠,身披絳綃雲霞衣,登高座,口中念念有詞。音楽止,二人執手爐宣偈云:

\begin{myquote}
「太乙慈尊降駕臨,夜壑幽関次第開。

童子雙雙前引導,死魂受煉步雲階。」
\end{myquote}

黄眞人薰沐焚香,念曰:

\begin{myquote}[\markfont]
「伏以玄皇闡敎,廣開度於冥途;正一垂科,俾煉形而昇擧。恩霑幽爽,澤被饑虚。謹運眞香,志誠上請:東極宫中大慈仁者,尋聲赴感太乙救苦天尊,青玄九陽上帝,十方救苦諸大眞人,天僊地僊,三界官屬,五嶽十王,水府羅酆聖衆,仗此眞香,來臨法會。伏望獅座浮空,龍旂耀日,空青枝洒,頻除熱惱;甘露普滋,廣濟孤虚。今則暫供几筵,告頒符命:九幽滅罪,罷對停毆。切以人處塵凡,日縈俗務。不知有死,惟欲貪生。鮮能種於善根,多隨入於惡趣。昏迷弗省,恣慾貪嗔。將謂自己長存,豈信無常易到。一朝傾逝,萬事皆空。業障纏身,冥司受苦。今奉道伏為亾過室人李氏靈魂,一棄塵緣,久淪長夜。若非薦拔於愆辜,必致難逃於苦報。恭惟天尊,號隆億劫,氣應九陽。秉好生之仁,救尋聲之苦。洒甘露而普滋羣類,放瑞光而遍燭昏衢。命三官寬考較之條,詔十殿擱推硏之筆。開囚釋禁,宥過解寃。各隨符使,盡出幽関。咸令登火池之沼,悉蕩滌黄華之形。凡得更生,俱歸道岸。」
\end{myquote}

高功念《五廚經》、《變食神咒》,散法食:

\begin{myquote}[\markfont]
「聞天浮九炁,九炁出乎太空之先;地凝九幽,九幽欝於重陰之壘。九炁列正,萬物並受生成,所以為天地之根。各受生於胞胎,賴三光而育養。人之有死壞者,皆所以不能受其形,保其神,貴其炁,固其根,離其本眞耳。若得還生,湏得濯形於太陰,煉質於太陽,復受九炁,凝合三元,結成胞胎乃可成形。匪仗太上之金科,玄元之秘旨,豈可開度幽魂,全形復體,駕景朝元?兹焚《制魔保擧靈寳煉形眞符》,謹當宣奏:

太微迴黃旗,無英命靈旛,攝召長夜府,開度受生魂。」
\end{myquote}

道衆先將魂旛安於水池内,焚結靈符,換紅旛。次於火沼内,焚欝儀符,換黃旛。高功念:「天一生水,地二生火;水火交煉,乃成眞形。」煉度畢,請神主冠帔,步金橋,朝參玉陛,皈依三寳。朝玉清,衆擧〔五供養〕:

\begin{myquote}
「道中尊,玉清主!溟滓無光包九炁,萬象森羅一黍珠。死魂受煉,受煉超僊界。」
\end{myquote}

朝上清〔五供養〕:

\begin{myquote}
「經中尊,上清主!赤明開圖推運極,元綱流演洞渺溟。死魂受煉,受煉超僊界。」
\end{myquote}

朝太清〔五供養〕:

\begin{myquote}
「師中尊,太清主!道包天地玄元始,歷劫度開出迷魂。死魂受煉,受煉超僊界。」
\end{myquote}

高功曰:「既受三皈,當宣九戒:

\begin{myquote}
第一戒者,敬讓,孝養父母。

第二戒者,克勤,忠於君王。

第三戒者,不殺,慈救衆生。

第四戒者,不淫,正身處物。

第五戒者,不盜,推義損己。

第六戒者,不嗔,兇怒凌人。

第七戒者,不詐,諂賊害善。

第八戒者,不驕,傲忽至眞。

第九戒者,不二,奉戒專一。

汝當諦聽,戒之戒之!」
\end{myquote}

九戒畢。道衆擧音楽,宣念符命,幷十類孤魂〔挂金索〕:

\begin{myquote}
「大慈仁者,救苦青玄帝,獅座浮空,妙化成神力。清淨斛食,示現焦面鬼。法界孤魂,來受甘露味! 

北戰南征,貫甲披袍士。捨死忘生,報效於國家。砲響一聲,身臥沙場裏。陣忘孤魂,來受甘露味!

好兒好女,與人為奴婢。暮打朝喝,衣不遮身體。逐趕出門,僵臥長街内。饑死孤魂,來受甘露味!

坐賈行商,僧道雲遊士。動歲經年,在外尋衣食。病疾臨身,旅店無依倚。客死孤魂,來受甘露味!

鬬惡爭強,枷鎖囹圄閉。斬絞凌遲,身丧長街裏。律有明條,犯了王法罪。刑死孤魂,來受甘露味!

宿世寃仇,今世來相會。暗計陰謀,毒薬攛腸胃。九竅生煙,丧了身和體。薬死孤魂,來受甘露味!

乳哺三年,父母恩難極。十月懷胎,坐草臨盆際。性命懸絲,子母歸陰世。産死孤魂,來受甘露味!

急難顛危,受忍難迴避。私債官錢,逐日來催逼。自刎懸梁,断了三寸氣。屈死孤魂,來受甘露味!

久病淹纏,氣蠱癱癆類。疥癬痍瘡,遍體膿腥氣。菽水無親,醫薬無調治。病死孤魂,來受甘露味!

巨浪風濤,洪水滔天至。纜断舟沉,身丧長江裏。回首家鄉,無人捎書寄。溺死孤魂,來受甘露味!

回祿風煙,一時難迴避。猛火無情,燒燬身和體。爛額焦頭,死作煙薰鬼。焚死孤魂,來受甘露味!

附木精邪,無主魍魎輩。鱗介飛潛,莫不回生意。太上慈悲,廣垂方便澤。十類孤魂,來受甘露味!」
\end{myquote}

煉度已畢,黃眞人下高座,道衆音楽送至門外,化財焚燒箱庫。囬來,齋功圓滿。道衆都換了冠服,鋪排收捲道像。西門慶又早大廳上畫燭齊明,酒筵羅列。三個小優彈唱,衆親友都在堂前。西門慶先與黄眞人把盞,左右捧着一疋天青雲鶴金緞,一疋色緞,十兩白銀,叩首下拜道:「亡室今日已賴我師經功救拔,得遂超生,均感不淺!微禮聊表寸心。」黄眞人道:「小道謬忝冠裳,濫膺玄敎,有何德以達人天?皆賴大人一誠感格,而尊夫人已駕景朝元矣。此禮若受,實為赧顏!」西門慶道:「此禮甚薄,有褻眞人,伏乞笑納。」黄眞人方令小童收了。西門慶遞了眞人酒,又與吳道官把盞,乃一疋金緞,伍兩白銀,又是十兩經資。吳道官只受了經資,餘者不肯受,說:「小道素蒙厚愛,自恁效勞,誦經追拔夫人徃生僊界,以盡其心。受此經資,尚為不可,又豈當此盛禮乎?」西門慶道:「師父差矣。眞人掌壇,其一應文檢法事,皆乃師父費心。此禮當與師父酬勞,何為不可?」吳道官不得已方領下,再三致謝。

西門慶與道衆遞酒已畢,然後吳大舅應伯爵等上來,與西門慶散福遞酒。吳大舅把盞,伯爵執壺,謝希大捧菜,一齊跪下,伯爵道:「兄為嫂子今日做此好事,請得眞人在此,又是吳師父費心,方纔化財,見嫂子頭戴鳳冠,身穿素衣,手執羽扇,騎着白鶴,望空騰雲而去。此賴眞人追薦之力,哥的虔心,嫂子的造化,連我好不快活!」於是滿斟一盃,送與西門慶。西門慶道:「多蒙列位連日勞神,言謝不盡,何敢當此盛意?」説畢,一飲而盡。伯爵又斟一盞,説:「哥喫酒,喫個雙盃,不要喫單盃。」希大慌忙遞一筯菜來喫了。西門慶回敬衆人畢,安席坐下。小優彈唱起來,廚役上來割道。當夜在席前猜拳行令,品竹彈絲,直喫到二更時分,西門慶已帶半酣,衆人方作辭起身而去。西門慶進來,賞小優兒三錢銀子,徃後邊去了。正是:人生有酒須當醉,一滴何曾到九泉!有詩為證:

\begin{myquote}
百年方誓日,一夕竟為雲。

飛鳳金鈿落,翔鸞寳鏡分。

超生空自喜,長恨不勝情。

盃物頻頻飲,愁懷且暫清。
\end{myquote}

畢竟不知後項如何,且聽下回分解。

