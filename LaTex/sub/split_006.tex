\chapter*{校者弁言}
\addcontentsline{toc}{chapter}{校者弁言}
\markboth{校者弁言}{校者弁言}

《金瓶梅詞話》是中國著名長篇白話小說,也是一部最具爭論性作品。自誕生以來,貶之者詆為「市諢之極穢者,當急投秦火。」讚之者譽為「偉大的冩實小說,同時說部,無以上之。」其實除去書中一些不雅的性事描冩,《金瓶梅詞話》無疑是中國文學寳庫中之奇珍,與《水滸傳》《紅樓夢》屬同一水平作品。而其反映社會生活之廣闊,刻劃人性之深刻,運用語言之鮮活,恐猶在二書之上。今本詞話原為说書藝人底本,上板前又未加認真校訂,訛誤太甚,可讀性差,入清以後即湮没無聞。社會上流行的是經删節改編的崇禎本、竹坡本。及至本世紀三十年代,《金瓶梅詞話》原刻在山西發現,世人始目睹此書之真面目。筆者従八十年代中從事詞話的整理校點,旨在為讀者提供一個可讀的、較少錯誤的、接近原著的本子。選擇以日本大安株式會社影印彼邦配本《金瓶梅詞話》為主校本,臺灣聯經出版事業有限公司硃墨二色套印原北京圖書館藏《金瓶梅詞話》為副校本,校以北京大學本和日本内閣文庫本之《新刻繡像批評金瓶梅》、在玆堂本和崇經堂本之臯鶴堂批評第一奇書金瓶梅,並先後參攷鄭振鐸、施蟄存、增爾智、劉本棟、戴鴻森、白維國、卜健諸本,兼吸收姚靈犀、魏子雲、李申、張慧英、傅憎享、張鴻魁、魯歌、馬征等專家研究成果,進行校訂。一九八八年出版《全校本金瓶梅詞話》,一九九三年完成第二次校點,出版《重校本金瓶梅詞話》,近日完成最後一次校定,合共校正詞話原本訛錯衍奪七千多處,雖不敢自詡完善,然已恢復詞話原來流暢活脱的話本風貌,大大提高其可讀性。山西青年書灋家陈少卿先生花三年時間,業餘抄閱三校定本,都八十萬字,共二十册。字體隽秀,楮墨精好,是可寳也。爰綴數言,弁其首,以述經過如此。時維戊寅菊月重陽後一日於香港青衣島夢梅館。

\begin{quotation}\begin{flushright}梅節\end{flushright}\end{quotation}

