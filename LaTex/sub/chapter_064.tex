\includepdf[pages={127,128},fitpaper=false]{tst.pdf}
\chapter*{第六十四囬 \\玉簫跪央潘金蓮 合衛官祭富室娘}
\addcontentsline{toc}{chapter}{第六十四囬 玉簫跪央潘金蓮 合衛官祭富室娘}
\markboth{{\titlename}卷之七}{第六十四囬 玉簫跪央潘金蓮 合衛官祭富室娘}


\begin{myquote}
着人情思覺初闌,手把鮫綃仔細看。

到老春蠶絲乃盡,成灰蠟燭淚初乾。

鸞交鳳友驚風散,軟玉嬌香異世問。

兩字風流誇未了,鷄鳴殘月五更寒。
\end{myquote}

話説衆人散了,已有鷄唱時分。西門慶歇息去了。玳安㧱了一大壺酒,幾碟下飯,在前邊舖子裏還和傅夥計陳經濟同喫。傅夥計老頭子,熬到這咱,已是不楽坐,搭下鋪,倒在炕上就睡了,因向玳安道:「你自和平安兩個喫罷,陳姐夫想是也不來了。」這玳安櫃上點着夜燭,呌進平安來,兩個把那壺酒你一鍾,我一盞,都喫了。把家伙收過一邊,平安便去門房裏去睡了。玳安一面関上舖子門,上炕和傅夥計兩個通廝脚兒睡下。傅夥計閒中因話提起,向玳安説道:「你六娘沒了,這等樣棺槨祭祀,念經發送,也夠他了。」玳安道:「一來他是福好,只是不長壽。俺爹饒使了這些錢,還使不着俺爹的哩。俺六娘嫁俺爹,瞞不過你老人家,該帶了多少帶頭來。别人不知道,我知道。把銀子休説,光金珠玩好,玉帶縧環䯼髻,値錢寳石,還不知有多少。為甚俺爹心裏疼?不是疼人,是疼錢。是便説起俺這過世的六娘,性格兒這一家子都不如他,又有謙讓,又和氣,見了人只是一面兒笑。俺們下人,自來也不曾呵俺們一呵,並沒失口駡俺們一句「奴才」,要的誓也沒賭一個。使俺們買東西,只拈塊兒。俺們但説:『娘,㧱等子你稱稱,俺們好使。』他便笑道:『㧱去罷,稱甚麽。你不圖落圖甚麽來?只要替我買値着。』這一家子,那個不借他銀使?只有借出來,沒有個還進去的。還也罷,不還也罷。俺大娘和俺三娘使錢也好,只是五娘和二娘慳吝些。他當家,俺們就遭瘟來,會把腿磨細了!會勝買東西,也不與你個足數。綁着鬼,一錢銀子㧱出來只稱九分半,着緊只九分,俺們莫不賠出來!」傅夥計道:「就是你大娘還好些。」玳安道:「雖故俺大娘好,毛司火性兒。一回家好,娘兒們親親噠噠説話兒,你只休惱狠着他,不論誰,他也罵你幾句兒。總不如六娘,萬人無怨。又常在爹跟前替俺們説方便兒。随問天來大事,受不的人央。俺們央他央兒對爹説,無有個不依。只是五娘快戳無路兒,行動就説:『你看我對你爹説。』把這『打』只題在口裏。如今春梅姐又是個合氣星,天生的都出在他一屋裏!」傅夥計道:「你五娘來這裏也好幾年了。」玳安道:「你老人家是知道,他想的起那咱來哩!他一個親娘也不認的,來一遭,要便搶的哭了家去。如今六娘死了,這前邊又是他的世界。那個管打掃花園,又説地不乾淨,一清早晨喫他罵的狗血噴了頭。」兩個説了一回,那傅夥計在枕上齁齁就睡着了。玳安亦有酒了,合上眼,不知天高地下,直至紅日三竿,都還未起來。

原來西門慶每常在前邊靈前睡,早晨玉蕭出來收疊牀鋪,西門慶便往後邊梳頭去。書童蓬着頭,要便和他兩個在前邊打牙犯嘴,互相嘲鬦,半日才進後邊去。不想今日西門慶歸後邊上房歇去,這玉簫趕人沒起來,暗暗走出來,與書童遞了眼色,兩個走在花園書房裏幹營生去了。不料潘金蓮起的早,驀地走到廳上,只見靈前燈兒也沒了,大棚裏丢的桌椅横三豎四,没一個人兒。只見畫童兒正在那裏掃地。金蓮道:「賊囚根,乾淨只你在這裏掃地,都往那裏去了?」畫童道:「他們都還沒起來哩。」金蓮道:「你且丢下苕帚,到前邊對你姐夫説,有白絹㧱一疋來,你潘姥姥還少一條孝裙子。再㧱一副頭鬚繫腰來與他,他今日家去。」畫童道:「怕不俺姐夫還睡哩,等我問他去。」良久回來道:「姐夫説不是他的首尾,書童哥與崔大哥管孝帳,娘問書童哥要就是了。」金蓮道:「知道那奴才往那去了?你去尋他來。」畫童向廂房裏瞧了瞧,説道:「纔在這裏來,敢往花園書房裏梳頭去了。」金蓮道:「你自在這裏掃完了地,等我自家問這囚根子要去。」於是輕移蓮步,款蹙湘裙,走到花園書房内。偶然聽見裏面有人笑聲,推開門,只見他和玉簫在牀上正幹得好哩。便罵道:「好囚根子,你兩個在此幹得好事!」唬得兩個做手腳不迭,齊跪在地下哀告。金蓮道:「賊囚根子,你且㧱一疋孝絹,一疋布來,打發你潘姥姥家去。」那書童連忙拿來遞上。金蓮逕歸房來。那玉簫跟到房中,打旋磨兒跪在地下,央及:「五娘,千萬休對爹説。」金蓮便問:「賊狗肉,你和我實説,這奴才従前已往偸了幾遭?一字兒休瞞我便罷。」那玉簫便把和他偸的緣由説了一遍。金蓮道:「旣要我饒恕你,你要依我三件事。」玉簫道:「娘饒了我,隨問幾件事我也依娘。」金蓮道:「一件,你娘房裏但凡大小事兒,就來告我説。你不説,我打聽出,定不饒你。第二件,我但問你要甚麽,你就捎出來與我。第三件,你娘向來沒有身孕,如今他怎生便有了?」玉簫道:「不瞞五娘説,俺娘如此這般,喫了薛姑子的衣胞符薬,便有了。」這潘金蓮一一聽記在心,纔不對西門慶說了。

那書童見潘金蓮冷笑領進玉簫去了,知此事有幾分不諧。向書房廚櫃内收拾了許多手帕汗巾、挑牙、簪紐,並收的人情,他自己也攢夠十來兩銀子,又到前邊櫃上,誆了傅夥計二十兩——只説要買孝絹,逕出城外,僱了長行頭口,到馬頭上,搭在鄉里船上,往蘇州原籍家去了。正是:撞碎玉籠飛彩鳳,頓開金鎖走蛟龍。

不想那日李桂姐、吳銀兒、鄭愛月,都家去了;薛内相劉内相早晨差了人擡三牲桌面來祭奠燒紙,又每人送了一兩銀子伴宿分資,呌了兩個唱道情的來,白日裏要和西門慶坐坐:緊等着要打發他孝絹。尋書童兒要鑰匙,一地裏尋不着。傅夥計道:「他早晨問我櫃上要了二十兩銀子買孝絹去了。口稱爹吩咐他孝絹不夠,敢是向門外買去哩!」西門慶道:「我並沒吩咐他,如何問你要銀子?」一面使人往門外絹舖找尋他,那裏得來?月娘便向西門慶說:「我猜這奴才有些蹺蹊,不知弄下甚麽硶兒,拐了幾兩銀子走了。你那書房子裏開了門,還大瞧瞧,沒脚蟹的營生,只怕還㧱甚麽去了。」西門慶走到兩個書房裏都瞧了,見庫房裏鑰匙掛在墻上,大櫥櫃裏不見了許多汗巾手帕並書禮銀子、挑牙紐扣之類。西門慶心中大怒,呌將該地方的管役來,吩咐:「各䖏三瓦兩巷,與我訪緝!」那裏得來?正是:不獨懷家歸興急,五湖煙水正茫茫。

那時薛内相従晌午時就坐轎來了,西門慶請下吳大舅應伯爵溫秀才相陪。先到靈前上香,打了個問訊,然後與西門慶敍禮,説道:「可傷,可傷!如夫人是甚麽病兒殁了?」西門慶道:「不幸患崩漏之疾,看治不好,殁了。又多謝老公公費心!」薛内相道:「沒多兒,將就表意罷了。」因看見掛着影,説道:「好個標緻娘子,正好青春享福,只是去世太早些!」溫秀才在傍道:「物之不齊,物之情也。窮通壽夭,自有個定數,雖聖人亦不能強。」薛内相扭囬頭來,見溫秀才衣巾穿着素服,說道:「此位老先兒是那學裏的?」溫秀才躬身道:「學生不才,備名府庠。」薛内相道:「我瞧瞧娘子的棺木兒。」西門慶即令左右把兩邊帳子撩起,薛内相進去,觀看了一遍,極口稱贊道:「好副板兒,請問多少價買的?」西門慶道:「也是舍親的一付板,學生回了他的來了。」應伯爵道:「請老公公試估估,那裏地道?甚麽名色?」薛内相仔細看了説:「此板不是建昌,是副鎭遠。」伯爵道:「就是鎭遠,也値不多。」薛内相道:「最高者必定是楊宣楡。」伯爵道:「楊宣楡單薄短小,怎麽看的過。此板還在楊宣楡之上,名喚做桃花洞,在於湖廣武陵川中。昔日唐漁父入此洞中,曾見秦時毛女在此避兵,是個人跡罕到之處。此板七尺多長,四寸厚,二尺五寬,還看一半親家分上,要了三百二十兩銀子哩。公公,你不曾看見,解開噴鼻香的,裏外俱有花色。」薛内相道:「是娘子這等大福,纔享用了這板。俺們内官家,到明日死了,還沒有這等發送哩!」吳大舅道:「老公公好說。與朝廷有分的人,享大爵祿。俺們外官焉能趕的上?老公公日近清光,代萬歲傳宣金口,現今童老爺加封王爵,子孫皆服蟒腰玉,何所不至哉!」薛内相便道:「此位會説話的兄,請問上姓。」西門慶道:「此是妻兄吳大哥,現居本衛千户之職。」薛内相道:「就是此位娘子的令兄麽?」西門慶道:「不是,乃賤荆之兄。」薛内相復於吳大舅聲諾,説道:「吳大人,失瞻!」

看了一囬,西門慶讓至捲棚内,正面安放一把校椅,薛内相坐下,打茶的㧱上茶來喫了。薛内相道:「劉公公怎的這咱還不到?叫我答應的迎迎去。」青衣人跪下禀道:「公公起身時,差小的邀劉公公去。劉公公轎已伺候下了,便來也。」薛内相又問道:「那兩個唱道情的來了不曾?」西門慶道:「早上就來了。叫上來。」不一時,走來面前磕頭。薛内相道:「你們喫了飯不曾?」那人道:「小的們得了飯了。」薛内相道:「旣喫了飯,你們今日用心答應,我重賞你。」西門慶道:「老公公,學生這裏還預備着一起戲子,唱與老公公聽。」薛内相問:「是那裏戲子?」西門慶道:「是一班海鹽戲子。」薛内相道:「那蠻聲哈剌,誰曉的他唱的是甚麽!那酸子們在寒窗之下,三年受苦,九載遨遊,背着個琴劍書箱來京應擧,恁得了個官,又無妻小在身邊,便希罕他這樣人。似我一個光身漢,老内相,要他做甚麽?」溫秀才在傍笑説道:「老公公説話太不近情了。居之齊則齊聲,居之楚則楚聲。老公公䖏於高堂廣廈,豈無一動其心哉?」這薛内相便拍手笑將起來道:「我就忘了溫先兒在這裏,你們外官原來只護外官!」溫秀才道:「雖是士大夫,也只是秀才做的。老公公砍一枝,損百林,兔死狐悲,物傷其䫫。」薛内相道:「不然。一方之地,有賢有愚。」

正説着,忽左右來報:「劉公公下轎了。」吳大舅等出去迎接進來,向靈前作了揖。叙禮已畢,薛内相道:「劉公公,你怎的這咱纔來?」劉内相道:「北邊徐同家來拜望,陪他坐了一囬,打發去了。」一面分席坐下,左右遞上茶去。因問答應的:「祭奠桌面兒都擺上了?」下邊人説:「都排停當了。」劉内相道:「咱們去燒了紙罷。」西門慶道:「老公公不消多禮,頭裏已是見過禮了。」劉内相道:「此來為何?還當親祭祭。」當下左右接過香來,兩個内相上了香,遞了三鍾酒,拜下去。西門慶道:「老公公請起。」於是拜了兩拜起來。西門慶還了禮,復至捲棚内坐下。然後收拾安席,遞酒上坐。兩位内相分左右坐了,吳大舅溫秀才應伯爵從次,西門慶下邊相陪。子弟鼓板響動,遞上関目揭帖。兩位内相看了一囬,揀了一段《劉智遠紅袍記》。唱了還未幾摺,心下不耐煩,一面呌上唱道情去,「唱個道情兒耍耍到好。」於是打起漁鼓,兩個並肩朝上高聲唱了一套「韓文公雪擁藍関」故事,下去。只見廚役上來磕頭,兩位内相都有賞賜。西門慶預備酒肉,賞賜跟隨人等,不用細説。

薛内相便與劉内相兩個席上説説話兒道:「劉哥,你不知道,昨日這八月初十日,下大雨如注,雷電把内裏凝神殿上鴟尾裘碎了,唬死了許多宫人。朝廷大懼,命各官修省,逐日在上清宫宣精靈疏建醮,禁屠十日,法司停刑,百官不許奏事。昨日大金遣使臣進表,要割内地三鎭。依着蔡京老賊,就要許他。掣童掌事的兵馬,敎都卸史譚稹黄安十大使節制,三邊兵馬又不肯,還交多官計議。昨日立冬,萬歲出來祭太廟,太常寺一員博士,名喚方軫,早晨値着打掃,看見太廟磚縫出血,殿東北上地陷了一角,寫表奏知萬歲。科道官上本,極言童掌事大了,宦官不可封王。如今馬上差官,㧱金牌去取童掌事囬京。」劉内相道:「你我如今出來在外做土官,那朝裏事也不干咱們。俗語道,咱過了一日是一日,便塌了天,還有四個大漢。到明日,大宋江山管情被這些酸子弄壞了。王十九,咱們只喫酒!」因呌唱道情的上來,吩咐:「你唱個『李白好貪盃』的故事。」那人立在席前,打動漁鼓,又唱了一囬。直喫至日暮時分,吩咐下人看轎起身。西門慶款留不住,送出大門,喝道而去。

囬來,吩咐點起燭來,把桌席休動,教廚役上來攢整停當,留下吳大舅應伯爵溫秀才坐的。又使小廝請傅夥計、甘夥計、韓道國、賁地傳、崔本和陳經濟復坐,叫上子弟來,吩咐:「還找着昨日《玉環記》上來。」因向伯爵道:「内相家不曉的南戲滋味,早知他不聽,我今日不留他。」伯爵道:「哥,倒辜負你的意思。内臣斜局的營生,他只喜《藍関記》,搗喇小子山歌野調,那裏曉的大関目,悲歡離合?」於是下邊打動鼓板,將昨日《玉環記》做不完的摺數,一一緊做慢唱,都搬演出來。西門慶令小廝席上頻斟羙酒。伯爵與西門慶同桌而坐,便問:「他姐兒三個還沒家去,怎的不叫出來遞盃酒兒?」西門慶道:「你還想那一夢兒,他們去的不耐煩了。」伯爵道:「他們在這裏住了有兩三日?」西門慶道:「吳銀兒住的久了。」當日衆人坐到三更時分,搬戲已完,方起身各散。西門慶邀下吳大舅,明日早些來陪上祭官員。與了戲子四兩銀子,打發出門。

到次日,周守備、荆都監、張團鍊、夏提刑,合衛許多官員,都合了分資,辦了一副猪羊喫桌祭奠,有禮生讀祝。西門慶預備酒席,李銘等三個小優兒伺候答應。到晌午,只聽鼓響,祭禮到了。吳大舅、應伯爵、溫秀才,在門首迎接。只見後擁前呼,衆官員下馬,在前廳換衣服。良久,把祭品擺下。衆官齊到靈前,西門慶與陳經濟伺候還禮。禮生喝禮,三獻畢,跪在傍邊讀祝:

\begin{myquote}
「維政和七年,歲次丁酉,九月庚申朔,越二十五日甲申,寅侍生周秀、荆忠、夏延齡、張関、文臣、范勳、吳鎧、徐鳳翔、潘磯等,謹以剛鬣、柔毛、庶羞之儀,致奠於

故錦衣西門孺人李氏之靈曰:維靈秀毓閨閫,善淑女紅。金玉其德,蘭蕙其姿。相内政而有道,主中饋而無闕。重積學而和睦内眷,尊所天而擧案齊眉。人願耆艾,天晞絶奇。正宜同諧鸞琴,何乃嗇後而促其期。噫,修短有數也,天厭善類。珠沉璧碎,雲慘風悲。扣玄扃而莫啟,歎薤露而易晞!秀等忝居僚儕,情重交誼。崇餚於俎,酌酒於巵。庶乎來享,鑒此哀辭,嗚呼尚饗!」
\end{myquote}

祭畢,西門慶下來謝禮已畢。吳大舅等讓衆官至捲棚内,寬去素服,侍茶。小優彈唱起來,安席上坐。手下跟隨之人,自有管待。三道五割,酒餚比前兩日更豐盛齊整。廚役上來照席,還磕了頭。西門慶與吳大舅、應伯爵、溫秀才,下席相陪,觥籌交錯,殷勤勸酒。李銘等三個小優兒,銀箏象板,朝上彈唱。外邊自有夥計主管,將跟隨祭來各項人役盒擔錢,都照例打發銀子停當。衆官坐到後晌時分,就要起身。西門慶不肯,與吳大舅伯爵等㧱大盃款留。教李銘等彈楽器,唱小曲兒,歡飲直到日暮時分方散。西門慶還要留吳大舅衆人坐,吳大舅道:「各人連日打攪,姐夫也辛苦了。各自歇息去罷。」當時告辭回家。正是:

\begin{myquote}
天上碧桃和露種,日邊紅杏倚雲栽。

家中巨富人趨附,手内多時莫論財。
\end{myquote}

畢竟不知後來如何,且聽下回分解。

