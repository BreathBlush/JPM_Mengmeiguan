\includepdf[pages={21,22},fitpaper=false]{tst.pdf}
\chapter*{第十一囬 \\潘金蓮激打孫雪娥 西門慶梳籠李桂姐}
\addcontentsline{toc}{chapter}{第十一囬 潘金蓮激打孫雪娥 西門慶梳籠李桂姐}
\markboth{\titlename}{第十一囬 潘金蓮激打孫雪娥 西門慶梳籠李桂姐}


\begin{myquote}
婦人嫉妒非常,浪子落魄無賴。

一聽巧語花言,不顧新歡舊愛!

出逢紅袖相牵,又把風情别賣。

果然寒食元宵,誰不幫興幫敗。
\end{myquote}

話説潘金蓮在家,恃寵生驕,顛寒作熱,鎮日夜不得個寜靜。性極多疑,專一聽籬察壁,尋些頭惱廝鬧。那個春梅,又不是十分耐煩的。一日,金蓮為些零碎事情,不凑巧駡了春梅幾句。春梅沒處出氣,走往後邊廚房去搥枱拍盤,悶狠狠的模樣。那孫雪娥看不過,假意戲他道:「怪行貨子!想漢子便別處去想,怎的在這裏硬氣?」春梅正在悶時,聽了幾句,不一時暴跳起來:「那個歪斯纏説我哄漢子!」雪娥見他性不順,只做不開口。春梅便使性做幾步走到前邊來,如此如此,這般這般,一五一十,又添些話頭道:「他還説娘教爹收了我,和娘捎一幫兒哄漢子。」挑撥與金蓮知道。金蓮滿肚子不快活。只因送吴月娘出去送殯,起身早些,也有些身子倦,睡了一覺,走到亭子上。只見孟玉樓搖颭的走來,笑嘻嘻道:「姐姐如何悶悶的不言語?」金蓮道:「不要説起,今早倦到了不得。三姐,你在那裏去來?」玉樓道:「纔到後面廚房裏走了一下。」金蓮道:「他與你説些什麽來?」玉樓道:「姐姐沒言語。」金蓮雖故口裏不説着,終久懷記在心,與雪娥結仇,不在話下。

兩個做了一囬針指,只見春梅抱着湯瓶,秋菊㧱了兩盞茶來。吃畢茶,兩個放桌兒,擺下棋子盤兒下棋。正下在熱鬧䖏,忽見看園門小廝琴童走來報道:「爹來了。」慌的兩個婦人收棋子不迭。西門慶恰進門檻,看見二人家常都戴着銀絲䯼髻,露着四鬢,耳邊青寳石墜子,白紗衫兒,銀紅比甲,挑線裙子,雙彎尖趫紅鴛瘦小,一個個粉粧玉琢,不覺滿面堆笑,戲道:「好似一對兒粉頭,也值百十兩銀子!」潘金蓮説道:「俺們纔不是粉頭,你家正有粉頭在後邊哩。」那玉樓抽身就往後走,被西門慶一手扯住,説道:「你往那裏去?我來了,你脱身去了!實説,我不在家,你兩個在這裏做甚麽?」金蓮道:「俺倆個悶的慌,在這裏下了兩盤棋,早是沒做賊。誰知道你就來了。」一面替他接了衣服,説道:「你今日送殯來家早。」西門慶道:「今日齋堂裏,都是内相同官,一來天氣暄熱,我不耐煩,先來家。」玉樓問道:「他大娘怎的還不來家?」西門慶道:「他的轎子也待進城,我使囬兩個小廝接去了。」一面脫了衣服坐下。因問:「你兩個下棋賭些什麽?」金蓮道:「俺兩個自恁下一盤耍子,平白賭什麽?」西門慶道:「等我和你們下一盤,那個輸了,拿出一兩銀子做東道。」金蓮道:「俺們並沒銀子。」西門慶道:「你沒銀子,拿簪子問我手裏當,也是一般。」於是擺下棋子,三人下了一盤,潘金蓮輸了。西門慶纔數子兒,被婦人把棋子撲撒亂了,一直走到瑞香花下,倚着湖山,推掐花兒。西門慶尋到那裏,説道:「好小油嘴兒,你輸了棋子,却躲在這裏。」那婦人見西門慶來,昵笑不止,説道:「怪行貨子,孟三兒輸了,你不敢禁他,卻來纏我。」將手中花撮成瓣兒,灑西門慶一身。被西門慶走向前雙関抱住,按在湖山畔,就口吐丁香,舌融甜唾,戲謔做一䖏。不防玉樓走到跟前,叫道:「六姐,他大娘來家了,咱後邊去來!」這婦人方纔撇了西門慶,説道:「哥兒,我囬來和你答話。」同玉樓到後邊,與月娘道了萬福。月娘問:「你們笑甚麽?」玉樓道:「六姐今日和他爹下棋,輸了一兩銀子,到明日整治東道,請姐姐耍子。」月娘笑了。金蓮當下只在月娘面前只打了個照面兒,就走來前邊陪伴西門慶。吩咐春梅房中薰下香,預偹澡盆浴湯,準備晚間兩個效魚水之歡。

看官聽説:家中雖是吴月娘大娘子在正房居住,常有疾病,不管家事;只是人情來往,出門走動。出入銀錢,都在唱的李嬌兒手裏。孫雪娥單管率領家人媳婦在廚中上灶,打發各房飲食。譬如西門慶在那房裏宿歇,或吃酒吃飯,造甚湯水,俱經雪娥手中整理。那房裏丫頭,自往廚下拿去,此事表過不說。當晚西門慶在金蓮房中吃了囬酒,洗畢澡,兩人歇了。

次日,也是合當有事。西門慶許了金蓮要往廟上替他買珠子,要穿箍兒戴。早起來,等着要吃荷花餅、銀絲鮓湯。纔起身,使春梅往廚下説去。那春梅只顧不動身。金蓮道:「你休使他。有人説我縱容他,敎你收了,捎成一幫兒哄漢子。百般指猪罵狗,欺負俺娘兒們。你又使他後邊做甚麽去?」西門慶便問:「是誰説此話欺負他?你對我説。」婦人道:「説怎的,盆罐都有耳朵。你只不叫他後邊去,另使秋菊去便了。」這西門慶遂叫過秋菊,吩咐他往廚下對雪娥説去。約有兩頓飯時,婦人已是把桌兒放了,白不見拿來。急的西門慶只是暴跳。

婦人見秋菊不來,使春梅:「你去後邊瞧瞧,那奴才只顧生根長苗不見來。」春梅有幾分不順,使性子走到廚下,只見秋菊正在那裏等着哩,便罵道:「賊淫婦,娘要卸你那腿哩!説你怎的就不去了哩。爹緊等着,吃了餅要往廟上去。急的爹在前邊暴跳,叫我採了你去哩!」這孫雪娥不聽便罷,聽了心中大怒,駡道:「怪小淫婦兒,馬囬子拜節——來到的就是!鍋兒是鐵打的,也等慢慢兒的熱來。預備下熬的粥兒又不吃,忽剌八新娘興出來要烙餅,做湯。那個是肚裏蛔虫?」春梅不忿他罵,説道:「沒的扯じ淡!主子不使了來問你,那個好來問你要?有沒,俺們到前邊只説的一聲兒。有那些聲氣的!」一隻手擰着秋菊的耳朶,一直往前邊來。雪娥道:「主子奴才,常遠似這等硬氣,有時道着!」春梅道:「中有時道使時道!沒的把俺娘兒兩個別變了罷?」於是氣狠狠走來。婦人見他臉氣的黄黄,拉着秋菊進門,便問:「怎的來了?」春梅道:「你問他,我去時還在廚房裏雌着,等他慢條廝禮兒纔和麫兒。我自不是,説了一句:『爹在前邊等着,娘説你怎的就不去了;使我來叫你來了。』倒被小院兒裏的千奴才萬奴才罵了我恁一頓,說爹『馬回子拜節——來到的就事』,只像那個調唆了爹一般。『預備下粥兒不吃,平白新生發起要餅和湯』!只顧在廚房裏駡人,不肯做哩。」婦人在旁便道:「我説別要使他去,人自恁和他合氣,説俺娘兒兩個だ攔你在這屋裏;只當吃人罵將來。」這西門慶聽了,心中大怒,走到後邊廚房裏,不由分説,向雪娥踢了幾脚,駡道:「賊歪剌骨,我使他來要餅,你如何罵他?你罵他奴才,你如何不溺泡尿把你自家照照!」那雪娥被西門慶踢駡了一頓,敢怒而不敢言。西門慶剛走出廚房門外,雪娥對着大家人來昭妻一丈青説道:「你看我今日晦氣!早是你在旁聽着,我又沒曾説什麽。他走將來,兇神也一般,大吆小喝,把丫頭採的去了,反對主子面前輕事重報,惹的走來平白地把恁一場兒。我洗着眼兒看着,主子奴才長遠恁硬氣着,只休要錯了脚兒!」不想被西門慶聽見了,復囬來又打了幾拳,罵道:「賊奴才,淫婦!你還説不欺負他?親耳朵聽見你還罵他!」打的雪娥疼痛難忍。西門慶便往前邊去了,那雪娥氣的在廚房裏兩淚悲啼,放聲大哭。

吴月娘正在上房,纔起來梳頭,因問小玉:「廚房裏亂的些什麽?」小玉囬道:「爹要餅吃了往廟上去,説姑娘罵五娘房裏春梅來,被爹聽見了,在廚房裏踢了姑娘幾脚,哭起來。」月娘道:「也沒見,他要餅吃,連忙做了與他去就罷了,平白又罵他房裏丫頭怎的?」於是使小玉走到廚房,攛掇雪娥和家人媳婦,連忙趲造湯水。打發西門慶吃了,騎馬,小廝跟隨,往廟上去不題。

這雪娥氣憤不過,走到月娘房裏,正告訴月娘此事。不防金蓮驀然走來,立於窗下潛聽。見雪娥在屋裏對月娘李嬌兒説他怎的だ攔漢子,背地無所不為:「娘,你不知淫婦,説起來比養漢老婆還浪,一夜沒漢子也成不的。背地幹的那繭兒,人幹不出,他幹出來!當初在家,把親漢子用毒薬擺死了,跟了來;如今把俺們也吃他活埋了,弄的漢子烏眼鷄一般,見了俺們便不待見!」月娘道:「也沒見你,他前邊使了丫頭要餅,你好好打發與他去便了,平白又駡他怎的?」雪娥道:「我罵他秃也瞎也來?那頃這丫頭在娘房裏,着緊不聽手,俺沒曾在灶上把刀背打他,娘尚且不言語。可可今日輪他手裏,便驕貴的這等的了!」正説着,只見小玉走到説:「五娘在外邊。」少頃,金蓮進房,望着雪娥説道:「比是我當初擺死親夫,你就不消叫漢子娶我來家,省得我だ攔着他,撑了你的窝兒。論起春梅,又不是我房裏丫頭,你氣不憤,還教他伏侍大娘就是了,省的你和他合氣,把我扯在裏頭。那個好意死了漢子嫁人?如今也不難的勾當,等他來家,與我一紙休書,我去就是了。」月娘道:「我也不曉的你們底事。你們大家省言一句兒便了。」孫雪娥道:「娘,你看他嘴似淮洪也一般,隨問誰也拌不過他。纔在漢子跟前戳舌兒,轉過眼就不認了。依你説起來,除了娘,把俺們都攆了,只留着你罷。」那吴月娘坐着,由着他那兩個你一句我一句,只不言語。後來見罵起來,雪娥道:「你罵我奴才,你便是眞奴才!」拉些兒不曾打起來。月娘看不上,使小玉把雪娥拉往後邊去。

這潘金蓮一直歸到前邊,卸了濃粧,洗了脂粉,烏雲散亂,花容不整,哭得兩眼如桃,躺在床上。到日西時分,西門慶廟上來,袖着四兩珠子,進入房中。一見便問:「怎的來?」婦人放聲號哭起來,問西門慶要休書,如此這般,告訴一遍:「我當初又不曾圖你錢財,自恁跟了你來,如何今日教人這等欺負!千也説我擺殺漢子,萬也説我擺殺漢子。拾了本有,掉了本無,沒丫頭便罷了,如何要人房裏丫頭伏侍,吃人指駡?我一個還多着影兒哩!」這西門慶不聽便罷,聽了此言,三尸神暴跳,五陵氣衝天。一陣風走到後邊,採過雪娥頭髮來,儘力㧱短棍打了幾下。多虧吴月娘向前拉住了手,説道:「沒的大家省事些兒罷了,好教你主子惹氣!」西門慶便道:「好賊歪剌骨,我親自聽見你在廚房裏罵,你還攪纏別人?我不把你下截打下來,也不算!」看官聽説:不爭今日打了孫雪娥,管敎潘金蓮従前作過事,沒興一齊來。有詩為證:
\begin{myquote}
金蓮侍寵仗夫君,到使孫娥忌怨深。

自古感恩並積恨,千年萬載不生塵。
\end{myquote}

當下西門慶打了雪娥,走到前邊,窝盤住了金蓮,袖中取出今日廟上買的四兩珠子,遞與他穿箍兒戴。婦人見漢子與他做主兒,出了氣,如何不喜?由是要一奉十,寵愛愈深。一日,在園中置了一席,請吴月娘孟玉樓,連西門慶四人共飲酒。

話休饒舌。那西門慶立了一夥,結識了十個人做朋友,每月會茶飲酒。頭一個名喚應伯爵,是個破落户出身,一份兒家財都嫖沒了,專一跟着富家子弟幫嫖貼食,在院中頑耍,諢名叫做應花子;第二個姓謝名希大,乃清河衛千户官兒應襲子孫,自幼兒沒了父母,遊手好閑,善能踢的好氣毬,又且賭博,把前程丟了,如今做幫閑的;第三名喚吳典恩,乃本縣陰陽生,因事革退,專一在縣前與官吏保債,以此與西門慶來往;第四名孫天化,綽號孫寡嘴,年紀五十餘歲,專在院中闖寡門,與小娘傳書寄柬,勾引子弟,討風流錢過日子;第五是雲參將兄弟,名喚雲離守;第六是花太監侄兒花子虚;第七姓祝,名喚祝日念;第八姓常,名常時節;第九個姓白,名喚白來創;連西門慶共十個。衆人見西門慶有些錢鈔,讓西門慶做了大哥,每月輪流會茶擺酒。一日,輪該花子虚家擺酒會茶,就在西門慶緊隔壁。内官家擺酒,都是大盤大碗,甚是豐盛。衆人都到齊了,那日西門慶有事,約午後不見到來,都㽞席面。少頃,西門慶來到,衣帽整齊,四個小廝跟隨,衆人都下席迎接,敍禮讓坐。東家安席,西門慶居首席。一個粉頭,兩個妓女,琵琶箏ぬ,在席前彈唱。端的説不盡梨園嬌豔,色藝雙全。但見:
\begin{myquote}
羅衣疊雪,寳髻堆雲。櫻桃口,杏臉桃腮;楊柳腰,蘭心蕙性。歌喉宛囀,聲如枝上流鶯;舞態蹁躚,影似花間鳳轉。腔依古調,音出天然。舞回明月墜秦樓,歌遏行雲遮楚館。高低緊慢,按宫商吐玉噴珠;輕重疾徐,依格調鏗金戛玉。箏排鴈柱聲聲慢,板排紅牙字字新。
\end{myquote}

少頃,酒過三巡,歌吟兩套,三個唱的放下楽器,向前花枝搖颭,繡帶飄飄磕頭。西門慶呼答應小廝玳安,書袋内取三封賞賜,每人二錢,拜謝了下去。因問東家花子虚:「這位姐兒上姓?端的會唱。」東家未及答,在席應伯爵插口道:「大官人多忘事,就不認的了。這ち箏的,是花二哥令翠,勾欄後巷吴銀兒;那撥阮的,是朱毛頭的女兒朱愛愛;這彈琵琶的,是二條巷李三媽的女兒,李桂卿的妹子,小名叫做桂姐。你家中现放着他親姑娘,大官人如何推不認的?」西門慶笑道:「六年不見,就出落得成了人兒了。」落後酒闌,上席來遞酒。這桂姐殷勤勸酒,情話盤桓。西門慶因問:「你三媽你姐姐桂卿在家做甚麽?怎的不來我家走走,看看你姑娘?」桂姐道:「俺媽従去歲不好了一場,至今腿脚半邊通動不的只扶着人走。俺姐姐桂卿,被淮上一個客人包了半年,常時接到店裏住,兩三日不放來家,家中好不無人。只靠着我逐日出來供唱,答應這幾個相熟的老爹,好不辛苦。也要往宅裏看看姑娘,白不得個閒。爹許久怎的也不在裏邊走走?放姑娘家去看看俺媽?」這西門慶見他一團和氣,説話兒乖覺伶變,就有幾分㽞戀之意,説道:「我今日約兩位好朋友送你家去,你意下如何?」桂姐道:「爹休哄我,你肯貴人脚兒踏俺賤地?」西門慶道:「我不哄你。」到是袖中取出汗巾,連挑牙與香茶盒兒,遞與桂姐收了。桂姐道:「多咱去?如今使保兒先家去説一聲,作個預備。」西門慶道:「直待人散,一同起身。」少頃,遞畢酒,約掌燈人散時分,西門慶約下應伯爵、謝希大,也不到家,騎馬同送桂姐,逕進勾欄往李家去。正是:錦繡窝中,入手不如撒手美;紅綿套裏,鑚頭容易出頭難。有詞為證:

\begin{myquote}
陷人坑,土窖般暗開掘;迷魂洞,囚牢般巧砌疊;檢屍塲,屠舖般明排列:衠一味死溫存活打劫。招牌兒大字書者:買俏金哥哥休撦,纏頭錦婆婆自接,賣花錢姐姐不賒!
\end{myquote}

西門慶等送桂姐轎子到門首,李桂卿迎門接入堂中。見畢禮數,請老媽出來拜見。不一時,虔婆扶拐而出,半邊胳膊通動彈不得。見了西門慶道個萬福,説道:「天麽天麽!姐夫貴人,那陣風兒刮你到於此處?」西門慶笑道:「一向窮冗,没曾來得,老媽休怪,休怪!」虔婆便問:「這二位老爹貴姓?」西門慶道:「是我兩個好友:應二哥、謝子純。今日在花家會茶,遇見桂姐,因此同送囬來。快看酒來!俺們樂飲三盃。」虔婆讓三位上首坐了,一面點了茶,一面下去打抹春檯,收拾酒菜。少頃,保兒上來放桌兒,掌上燈燭,酒餚羅列。桂姐従新房中打扮出來,旁邊陪坐。真個是風月窝,鶯花寨,免不得姊妹兩個在旁金樽滿泛,玉阮同調,歌唱遞酒。有詩為證:

\begin{myquote}
琉璃鍾,琥珀濃,小槽酒滴珍珠紅。烹龍炮鳳玉脂粒,羅帷繡幕圍香風。吹龍笛擊鼉鼓;皓齒歌,細腰舞。况是青春莫虚度。銀缸掩映嬌娥語:酒不到劉伶墳上土。
\end{myquote}

當下桂卿姐兒兩個唱了一套,席上觥籌交錯飲酒。西門慶向桂卿說道:「今日二位在此,久聞桂姐善舞能歌唱南曲,何不請歌一詞,以奉勸二位一盃兒酒,意下如何?」那應伯爵道:「我等不當起動,洗耳愿聽佳音。」那桂姐坐着只是笑,半日不動身。原來西門慶有心要梳籠桂姐,故發此言,先索落他唱。却被院中婆娘見經識經,看破了八九分。李桂卿在旁就先開口說道:「我家桂姐,從小兒養得嬌,自來生得腼腆,不肯對人胡亂便唱。」於是西門慶便叫玳安小廝,書袋内取出五兩一錠銀子來,放在桌上,便說道:「這些不當甚麽,權與桂姐為脂粉之需,改日另送幾套織金衣服。」那桂姐連忙起身相謝了。方纔一面令丫鬟收下了,一面放下一張小桌兒,請桂姐下席來唱。當下桂姐不慌不忙,輕拂羅袖,擺動湘裙,袖口邊搭剌着一方銀紅撮穗的落花流水汗巾兒,歌唱一隻〈駐雲飛〉:

\begin{myquote}
「擧止從容,壓盡勾欄占上風。行動香風送,頻使人欽重。嗏!玉玷污泥中,豈凡庸?一曲清商,滿座皆驚動。何似襄王一夢中,何似襄王一夢中!」
\end{myquote}

唱畢,把個西門慶喜歡的沒入脚䖏。吩咐玳安囬馬家去,晚夕就在李桂卿房裏歇了一宿。緊着西門慶要梳籠這女子,又被應伯爵謝希大兩個在跟前一力攛掇,就上了道兒。次日,使小廝往家去拿五十兩銀子,緞舖内討四套衣裳,要梳籠桂姐。那李嬌兒聽見要梳籠他家中侄女兒,如何不喜?連忙拿了一錠大元寳,付與玳安,拿到院中打頭面、做衣服、定桌席。吹彈歌舞,花攢錦簇,做三日,飲喜酒。應伯爵謝希大又約會了孫寡嘴、祝日念、常時節,每人出五分銀子人情作賀,都來囋他,鋪的蓋的,俱是西門慶出。每日大酒大肉,在院中頑耍,不在話下。

\begin{myquote}
舞裙歌板逐時新,散盡黄金只此身!

寄語富兒休暴殄,儉如良薬可醫貧。
\end{myquote}

畢竟未知後來如何,且聽下囬分解。

