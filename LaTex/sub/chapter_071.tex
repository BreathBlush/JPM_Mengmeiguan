\includepdf[pages={141,142},fitpaper=false]{tst.pdf}
\chapter*{第七十一囬 \\李瓶兒何千戶家託夢 提刑官引奏朝儀}
\addcontentsline{toc}{chapter}{第七十一囬 李瓶兒何千戶家託夢 提刑官引奏朝儀}
\markboth{{\titlename}卷之八}{第七十一囬 李瓶兒何千戶家託夢 提刑官引奏朝儀}


\begin{myquote}
暫時罷鼓膝間琴,閒把遺篇閱古今。

常嘆賢君務勤儉,深悲庸主事荒淫;

致平端自親賢哲,稔亂無非近侫臣。

説破興亡多少事,高山流水有知音。
\end{myquote}

話説西門慶同何千戶囬來,走到大街,何千戶先差人去囬何太監話去了。一面邀請西門慶到家一飯。西門慶再三固辭。何千戶令手下把馬嚼拉住,説道:「學生還有一事與長官商議。」於是並馬相行,到宅前下馬。賁四同擡盒逕往崔中書家去了。原來何千戶盛陳酒筵,在家等候。進入廳上,但見屏開孔雀,褥隱芙蓉,獸炭焚燒,金爐香靄。正中獨獨設一席,下邊一席相陪,傍邊東首又設一席,皆盤堆異菓,花插金瓶,桌椅鮮明,幃屏齊整。西門慶問道:「長官今日筵何客?」何千戶道:「家公公今日下班,敢與長官敍一中飯。」西門慶道:「長官這等費心盛設待學生,就不是同僚之情!」何千戶笑道:「倒是家公公主意,治此粗酌,屈尊請教。」一面看茶,喫了,西門慶請老公公拜見。何千戶道:「家公公便出來。」

不一時,何太監従後邊出來,穿着綠絨蟒衣,冠帽皂靴,寳石縧環。西門慶展拜四拜,請公公受禮。何太監不肯,説道:「使不的。」西門慶道:「學生與天泉同寅晚輩,老公公齒德俱尊,又係中貴,自然該受禮。」講了半日,何太監受了半禮。讓西門慶上坐,他主席相陪,何千戶傍坐。西門慶道:「老公公,這個断然使不的,同僚之間,豈可傍坐?老公公叔姪便罷了,學生使不的。」何太監大喜道:「大人甚是知禮。罷罷,我閣老位兒傍坐罷,教做官的陪大人主席就是了。」西門慶道:「這等學生坐的也安。」於是各敍禮坐下。何太監道:「小的兒們,再燒的炭來,今日天氣寒冷些。」須臾,左右火池火叉,㧱上一包暖閣水磨細炭,向中間四方黄銅火盆内只一倒,廳前放下油紙暖簾來,日光掩映,十分明亮。何老太監道:「大人請寬了盛服罷。」西門慶道:「學生裏邊沒穿甚麽衣服,使小价下處取來。」何太監道:「不消取去。」令左右:「接了衣服,㧱我穿的飛魚綠絨氅衣來,與大人披上。」西門慶笑道:「老公公職事之服,學生何以穿得?」何太監道:「大人只顧穿,怕怎的?昨日萬歲賜了我蟒衣。我也不穿他了,就送了大人遮衣服兒罷。」不一時,左右取上來。西門慶揑了帶,令玳安接去員領,披上氅衣,作揖謝了。又請何千戶也寬去上蓋,陪坐。又㧱上一道茶來喫了,何太監道:「呌小廝們來。」原來家中教了十二名吹打的小廝,兩個師範領着上來磕頭。何太監吩咐擡出銅鑼銅鼓,放在廳前,一面吹打動起樂來。端的聲震雲霄,韻驚魚鳥。然後左右伺候酒筵,上坐。何太監親自把盞,西門慶慌道:「老公公請尊便。有長官代勞,只安放鍾筯兒,就是一般。」何太監道:「我與大人遞一鍾兒。我家做官的,初入蘆葦,不知深淺,望乞大人凡事扶持一二,就是情了。」西門慶道:「老公公説那裏話!常言同僚三世親。學生亦託賴老公公餘光,豈不同力相助。」何太監道:「好説好説!共同王事,彼此扶持。」西門慶也沒等他遞酒,只接了盃兒,領到席上,隨即囬奉一盃,安在何千戶並何太監席上,彼此告揖過,坐下。吹打畢,三個小廝連師範,在筵前銀箏象板,三絃琵琶,唱了一套〔正宫·端正好〕:

\begin{myquote}
「水晶宫,鮫綃帳;光射水晶宫,冷透鮫綃帳。夜深沉,睡不穩龍牀;離金門,私出天街上,正風雪空中降。」 

{\markfont〔滚綉毬〕}「似紛紛蝶翅飛,如漫漫柳絮狂。舞冰花,旋風兒飄蕩,踐瓊瑤,脚步兒匆忙。將白欄兩袖遮,把烏紗小帽蕩。猛囬頭把鳳樓凝望,全不見碧琉璃瓦甇鴛鴦。一霎時九重宫闕如銀砌,半合兒萬里乾坤似玉粧。恰便是粉塡滿封疆。」

{\markfont〔倘秀才〕}「我只見鐵桶般重門閉上,我將這銅獸面雙環扣響。敲門的我是萬歲山前趙大郎:堂中無客伴,燈下看文章,特來聽講。」

{\markfont〔獃骨朶〕}「衝寒風冒凍雪來相望。有些個機密事緊待要商量。忙怎麽了事公人,免禮咱招賢宰相。這的是調鼎鼐三公府,那裏也剃頭髮唐三藏。我向這坐席間聽講書,你休來我耳邊廂呌點湯!」

{\markfont〔倘秀才〕}「朕不學漢高皇身居未央,朕不學唐天子停眠在晉陽。常則是翠被寒生金鳳凰。有心思傅説,無夢到高唐。這的是為君的勾當!」

{\markfont〔滚綉毬〕}「雖然與四海為一人,必索要正三綱謹五常。朕幼年間廣學鎗棒,恨則恨未曾到孔子門墻。《尚書》是幾篇?《毛詩》共幾章?講《禮記》始知謙讓,論《春秋》可鑑興亡。朕待學禹湯文武宗堯舜,卿可及房杜蕭曹立漢唐?則要你爕理陰陽。」

{\markfont〔倘秀才〕}「卿道是用《論語》治朝廷有方,卻原來這半部運山河在掌!聖道如天不可量,談經臨絳帳,索強如開宴出紅粧。聽説罷神清氣爽。」

{\markfont〔滚綉毬〕}「銀臺上畫燭明,金爐内寳篆香。不當煩教老兄自斟佳釀,又何須嫂嫂親捧着霞觴。卿道是糟糠妻不下堂,朕須想貧賤交不可忘。常言道表壯不如裏壯,妻若賢夫免災殃。朕得卿如太甲逢伊尹,卿得嫂嫂恰便似梁鴻配孟光,則願你福壽綿長。」

{\markfont〔倘秀才〕}「但歇息呵論前王後王,恰合眼慮興邦丧邦。因此上曉夜無眠想萬方。須不是歡娛嫌夜短,早難道寂寞恨更長,憂愁事幾樁!」

{\markfont〔滚綉毬〕}「憂則憂當軍的身無挂體衣,憂則憂走站的家無隔宿糧;憂則憂甘貧的晝眠深巷,憂則憂讀書的夜寐寒窻;憂則憂駕車的恁時分萬里行商,憂則憂行船的一江風浪;憂則憂嚎寒妻怨夫,憂則憂啼饑子呼娘;憂則憂是布衣賢士無活計,憂則憂鐵甲將軍守戰場:題將來感嘆悲傷!」

{\markfont〔倘秀才〕}「憂的是百姓苦,向御榻心勞意攘。憂的是天下小,教寡人眠思夢想。太原府劉崇拒北方。我只待暫離丹鳳闕,親擁碧油幢,先取那河東的上黨。」

{\markfont〔滚綉毬〕}「卿道是錢王共李王,劉鋹與孟昶。他們都無仁政着萬民失望,行霸道百姓遭殃。差何人收西川?命誰人定兩廣?取吳越必湏名將,下江南宜用忠良。要定奪展江山白玉擎天柱,索問您拯宇宙黄金駕海梁,卿仔細参詳。」

{\markfont〔脱布衫〕}「取金陵飛渡長江,到錢塘平定他邦。西川路休辭棧惡,南蠻地莫愁煙瘴。」

{\markfont〔醉太平〕}「陣衝開虎狼,身冒着風霜,用六韜三畧定邊疆,把元戎印掌。則要你人披鐵甲添雄壯,馬搖玉勒難遮當,鞭敲金ね響叮噹,早班師汴梁。」 

{\markfont〔二煞〕}「有那等順天心達天理去邪歸正皆疎放,有那等霸王業抗王師耀武揚威盡滅亾。休擄掠民財,休傷殘民命,休淫汚民妻,休燒毀民房。恤軍馬施仁立法,實錢糧定賞行罰,保城池討逆招安,沿路上安民挂榜,従賑濟任開倉。」

{\markfont〔尾聲〕}「朕專待正衣冠尊相貌就凌煙圖畫你那功臣像,卿莫負勒金石銘鐘鼎向青史標題姓字香。能用兵善為將,有心機有膽量。仰瞻天文算星象,俯察山川辨形狀。決戰先將九地量,晝戰須將旗幟張,夜戰須將火鼓揚;步戰屯雲護軍帳,水戰隨風使帆槳。奇正相生兵最強,仁智兼行勇怎當。耳聽將軍定這廂,坐擬元戎取那廂,飛奏邊庭進表章,齊賀昇平囬帝鄉。比及你列土分茅拜卿相,先將你各部下的軍卒重重的賞!」
\end{myquote}

唱了一套下去。酒過數巡,食割兩道,看看天晚,秉上燈來。西門慶喚玳安㧱賞賜與廚役並吹打各色人役,就要起身囬去,説:「學生不當,厚擾一日了,就此告囬。」那公公那裏肯放,説道:「我今日正是下班,要與大人請教。有甚大酒席!只是清坐而已,教大人受饑。」西門慶道:「承老公公賜這等太美饌,如何反言受饑!學生囬去歇息歇息,明早還與天泉參謁參謁兵科,好領劄付挂號。」何太監道:「旣是如此,大人何必又囬下處,就在我這裏歇了罷!明早好與我家做官的幹事。敢問如今下處在那裏?」西門慶道:「學生就暫借敝同僚夏龍溪令親崔中書宅中權寓,行李都在那邊。」何太監道:「這等也不難。大人何不令人把行李搬過來我家住兩日何如?我這後園兒裏有幾間小房兒,甚是僻淨,就早晚和做官的理會些公事兒,也方便些兒,強如在別人家。這個就是一家!」西門慶道:「在這裏也罷了,只是使夏公見怪,好像學生疎他一般。」何太監道:「沒的説。如今時年,早晨不做官,晚夕不唱喏,衙門是恁偶戲衙門。雖故當初與他同僚,今日前官已去,後官接管承行,與他就無干,他若這等説?他就是個不知道理的人了。今日我定然要和大人坐一夜,不放大人去。」喚左右:「下邊房裏快放桌兒,管待你西門老爹大官兒飯酒。我家差幾個人,跟他即時把行李都搬了來。吩咐打發後花園西院乾淨,預備舖陳,炕中籠下炭火。」堂上一呼,堦下百諾,答應下去了。西門慶道:「老公公盛情,只是學生得罪夏公了。」何太監道:「沒的扯淡了!他旣出了衙門,不在其位,不謀其政。他管他那裏鑾駕庫的事,管不的咱提刑所的事了,難怪於你。」不由分説,就打發玳安並馬上人喫了酒飯,差了幾名軍牢,各㧱䋲扛,逕往崔中書家搬取行李去了。

何太監道:「又一件相煩大人,我家做官的若是到任所,還望大人那裏替他看所宅舍兒,然後好搬取家小。今先教他同大人去,待尋下宅子,然後打發家小起身。也不多,連幾房家人,也有二三十口。」西門慶道:「天泉去了,老公公這宅子誰人看守?」何太監道:「我兩個名下官兒,第二個姪兒何永福,現在莊子上,呌他來住了罷。」西門慶道:「老公公吩咐,要看多少銀子宅舍?」何太監道:「也得千金出外銀子的房兒纔夠住。」西門慶道:「敝同僚夏龍溪,他京任不去了,他一所房子倒要打發,老公公何不要了與天泉住?一舉兩得其便,甚好!門面七間,到底五層;儀門進去大廳,兩邊廂房鹿角頂,後邊住房、花亭。周圍群房也有許多,街道又寬闊,正好天泉住。」何太監道:「他要許多價値兒?」西門慶道:「他對我説來,原是一千三百兩,又後邊添蓋了一層平房,收拾了一處花亭。老公公若要,隨公公與他多少罷了。」何太監道:「我乃託大人,隨大人主張就是了。趂今日我在家,差個人和他説去,討他那原文書我瞧瞧。難得尋下這房舍兒,我家做官的去到那裏,就有個歸着了。」不一時,只見玳安同衆人搬了行李來囬話。西門慶問:「賁四王經來了不曾?」玳安道:「王經同押了衣箱行李先來了,還有轎子,又呌賁四在那裏看守着。」西門慶因附耳低言,如此如此,這般這般,吩咐:「㧱我帖兒,上覆夏老爹,借過那裏房子的原契來,與何公公瞧瞧。就同賁四一答兒來。」這玳安應的去了。不一時,賁四青衣小帽,同玳安前來,㧱文書囬西門慶説:「夏老爹多上覆,旣是何公公要,怎好説價錢?原文書都㧱的來了。又收拾添蓋,使費了許多。隨爹主張了罷。」西門慶把原契遞與何太監,親看了一遍,見上面寫着一千二百兩,説道:「這房兒想必也住了幾年,裏面未免有些糟爛。也别要説收拾,大人面上,我家做官的旣治産業,還與他原價。」那賁四連忙跪下説:「何爺説的是。自古使的憨錢,治的莊田!千年房舍換百主,一番拆洗一番新。」把這何太監聽了,喜歡的了不的,便道:「你是那裏的?此人倒會説話兒。常言成大事者不惜小費!其實説的是。他呌甚麽名字?」西門慶道:「此是舍下夥計,名喚賁四。」何太監道:「也罷,沒個中人,你就做個中人兒,替我討了文契來。今日是個上官好日期,就把銀子兑與他罷。」西門慶道:「如今晚了,待的明日也罷了。」何太監道:「到五更,我早進去,明日大朝。今日不如先交與他銀子,就了事而已。」西門慶問道:「明日甚時駕出?」何太監道:「子時駕出到壇,三更鼓祭了,寅正一刻就囬到宫裏,擺了膳,就出來設朝陞大殿,又朝賀天下,諸司都上表拜冬。次日,文武百官喫慶成宴。你們是外任官,大朝引奏過,就沒你們事了。」

説畢,何太監吩咐何千戶進後邊,連忙打點出二十四錠大元寳來,用食盒擡着,差了兩個家人,同賁四玳安押送到崔中書家交割。夏公見擡了銀子來,滿心歡喜,隨即親手寫了文契,付與賁四等。㧱來遞與何太監,不勝歡喜,賞了賁四十兩銀子,玳安王經每人三兩。西門慶道:「小孩子家,不當與他。」何太監道:「胡亂與他買嘴兒喫。」三人磕了頭謝了。何太監吩咐管待酒飯,又向西門慶唱了兩個喏:「全仗大人餘光。」西門慶道:「豈有此理,還是看老公公金面。」何太監道:「還望大人對他説說,早把房兒騰出來,這裏好打發家小起身。」西門慶道:「學生一定與他説,教他早騰。何長官這一去,且在衙門公廨中權住幾日。待他家小搬取入京,收拾了,這裏長官家小起身不遲。」何太監道:「收拾直待過年罷了,先打發家小去纔好,十分在衙門中也不方便。」説話之間,已有二更天氣,西門慶説道:「老公公請安置罷,學生亦不勝酒力了。」何太監方作辭,歸後邊暖房内寢歇去了。何千戶教家楽彈唱,還與西門慶投壺,喫了一囬,方纔起身。歸至後園,正北三間書院,四面都是粉牆,臺榭湖山,盆景花木。房内絳燭高燒,疊席牀張錦幔,倭金屏護琴書,几席清幽,翠簾低掛,鋪陳整齊;爐上茶煮寳瓶,篆内香焚麝餅。何千戶又陪西門慶叙話良久,小童看茶喫了,方道安置,起身歸後邊去了。

西門慶向了囬火,方纔摘去冠帽,解衣就寢。王經玳安打發脱了靴襪,伸下被褥,合了燈燭,自往下邊暖炕歇去了。這西門慶有酒的人,睡在枕畔,見都是綾錦被褥,貂鼠綉帳,火箱泥金暖閣牀。在被窝裏,見滿窻月色,翻來覆去睡不着。良久,只聞夜漏沉沉,花陰寂寂,寒風吹得那窻紙有聲。況離家已久,欲待要呼王經進來陪他睡,忽然聽得窻外有婦人語聲甚低。即披衣下牀,靸着鞋襪,悄悄啟戶視之。只見李瓶兒霧鬢雲鬟,淡粧麗雅,素白舊衫籠雪體,淡黄軟襪襯弓鞋。輕移蓮步,立於月下。西門慶一見,挽之入室,相抱而哭,説道:「冤家,你如何在這裏?」李瓶兒道:「奴尋訪至此。對你説,我已尋了房兒了,今特來見你一面,早晚便搬去也。」西門慶忙問道:「你房兒在於何處?」李瓶兒道:「咫尺不遠。出此大街迤東,造釜巷中間便是。」言訖,西門慶共他相偎相抱,上牀雲雨,不勝羙快之極。已而整衣扶髻,徘徊不捨。李瓶兒叮嚀囑咐西門慶:「我的哥哥,切記休貪夜飲,早早囬家。那廝不時伺害於你,千萬勿忘奴言,是必記於心者!」言訖,執手而行,挽西門慶相送到家。走出大街,見月色如晝,果然往東轉過牌坊,到一小巷,旋踵見一座雙扇白板門,指道:「此奴之家也。」言畢,頓袖而入。西門慶急向前拉之,恍然驚覺,乃是南柯一夢。但見月影横窻,花枝倒影而矣。西門慶向褥底摸了摸,見精流滿席,餘香在被,殘唾猶甜。追悼莫及,悲不自勝。正是:世間好物不堅牢,彩雲易散琉璃脆。有詩為證:

\begin{myquote}
玉宇微茫霜滿襟,疎窻淡月夢魂驚。

凄凉睡到無聊處,恨殺寒鷄不肯鳴。
\end{myquote}

西門慶翻來覆去盼鷄叫,巴不得天亮。比及天亮,又睡着了。次日清晨,何千戶家童僕起來,伺候㧱洗面湯、手巾,王經玳安打發西門慶梳洗畢,何千戶又早出來,陪侍喫了姜茶,放桌兒請喫粥。西門慶問:「老公公怎的不見?」何千戶道:「家公公従五更鼓進内去了。」須臾,㧱上粥,圍着火盆,四碟齊整小菜,四大碗熬爛下飯。喫了粥,又㧱上一盞肉圓子餛飩鷄疍頭腦湯,金匙銀鑲雕漆茶鍾。一面喫着,吩咐出來伺候備馬。何千戶與西門慶冠冕,僕従跟隨,早進内參見兵科。出來,何千戶便分路來家,西門慶又到相國寺拜智雲長老。長老又留擺齋,西門慶只喫了一個點心,餘者收下來與手下人喫了。玳安毡包内㧱着金緞,従東街穿過來,要往崔中書家拜夏龍溪去。因従造釜巷所過,中間果見有雙扇白板門,與夢中所見一般。悄悄使玳安問隔壁賣豆腐老嫗:「此家姓甚名誰?」老嫗答道:「乃袁指揮家也。」西門慶於是不勝嘆異。到了崔中書家,夏公纔待出門拜人去。見西門慶到,令左右把馬牽過,迎西門慶至廳上,拜揖叙禮。西門慶令玳安㧱上賀禮:青織金綾紵一端,色緞一端。夏公道:「學生還不曾拜賀長官,到承長官先施!昨者小房又煩費心,感謝不盡。」西門慶道:「何太監央學生看房,一則我因堂尊吩咐,就說此房來。何公倒好,就估着要,學生無不作成。討了房契去看了,一口就還了原價,是内臣性兒,立馬蓋橋,就成了。還是堂尊大福!」説畢,呵呵笑了。夏公道:「何天泉我也還未囬拜他。」因問:「他此去,與長官同行罷了。」西門慶道:「他已會定同學生一路去,家小還且待後。昨日他老公公多致意,煩堂尊早些把房兒騰出來,搬取家眷。他如今且權在衙門裏住幾日罷了。」夏公道:「學生也不肯久稽。待這裏尋了房兒,就使人搬取家小,也只待出月罷了。」説畢,西門慶起身,又留了個拜帖與崔中書。夏公便道:「要留長官坐坐,爭奈在於客中,彼此情諒!」送出上馬,歸至何千戶家。何千戶又早伺候午飯等候。西門慶悉把拜夏公之事,說了一遍:「騰房已在出月,搬取家小。」何千戶大喜,謝道:「足見長官盛情。」

喫畢飯,二人正在廳上着棋,忽左右來報:「府裏翟爹那裏差人送下程來了,找尋到崔老爹那裏,崔老爹使他來這裏來了。」於是㧱帖來,宛紅帖兒上寫着:「謹具金緞一端,雲紵一端,鮮猪一口,北羊一腔,内酒二罈,點心二盒。眷生翟謙頓首拜。」西門慶見來人,説道:「又蒙翟大爹費心。」一面收了禮物,寫囬帖,賞來人二兩銀子,擡盒人五錢,説道:「客中不便,有褻管家。」那人連忙接了,説道:「小的不敢領。」西門慶道:「將就買盃酒喫便了。」那人方纔磕頭收了。王經在傍插口悄悄説:「小的姐姐説,教我府裏去看看愛姐,有物事捎與他。」西門慶問:「甚物事?」王經道:「是家中做的兩雙鞋脚子。」西門慶道:「單單兒怎好㧱去?」吩咐玳安:「我皮箱内有捎带的玫瑰花餅,取兩罐兒,用小描金盒兒盛着。」就把囬帖付與王經,穿上青衣,教他同跟了往府裏看愛姐不題。這西門慶寫了帖兒,送了一腔羊、一罈酒,謝了崔中書;把那一口猪、一罈酒、兩盒點心,擡到後邊:「孝順老公公,在此多有打擾!」慌的何千戶就來拜謝,説道:「長官,你我一家,如何這等計較!」

且說王經到府内,請出韓愛姐,外廳拜見了,打扮如瓊林玉樹一般,比在家出落自是不同,長大了好些。管待了酒飯,因見王經身上穿的單薄,與了一件天青紵絲貂鼠氅衣兒,又與了五兩銀子。㧱來囬覆西門慶話,西門慶大喜。正與何千戶下棋,忽聞綽道之聲,門上人來報:「夏老爹來拜,㧱了兩個拜帖兒。」忙的兩個整衣冠,迎接到廳敍禮。何千戶又謝昨日房子之事。夏提刑具了兩分緞帕酒禮,奉賀二公。西門慶與何千戶再三致謝,令左右收了。夏公又賞了賁四玳安王經十兩銀子。一面分賓主坐下,茶罷,共敍寒溫。夏公道:「請老公公拜見。」何千戶道:「家公公進内去了。」夏公又留下了一個雙紅拜帖兒,説道:「多頂上老公公,拜遲,恕罪!」言畢,告辭起身去了。何千戶隨即也具一分賀禮,一疋金緞,差人送去,不在言表。到晚夕,何千戶又在花園暖閣中擺酒,與西門慶共酌夜飲,家楽歌唱,到二更方寢。西門慶因其夜裏夢遺之事,晚夕令王經㧱舖蓋來書房地平上睡。半夜呌上牀,脱的精赤條摟在被窝内,兩個口吐丁香,舌融甜唾。正是:不能得與鶯鶯會,且把紅娘去解饞。一晚題過。

到次日起五更,與何千戶一行人跟隨進朝。先到待漏院候時,等的開了東華門進入。但見:

\begin{myquote}
星斗依稀禁漏殘,禁中環珮響珊珊。

花迎劍戟星初落,柳拂旌旗露未乾。

瑞靄光中瞻萬歲,祥煙影裏擁千官。

欲知今日天顔喜,遙覩蓬萊紫氣蟠。
\end{myquote}

少頃,只聽九重門啟,鳴噦噦之鸞聲;閶闔天開,覩巍巍之龍衮。當重熙累洽之日,致履端嘉慶之時。當時天子祀畢南郊囬來,文武百官,聚集於宫省等候設朝。須臾鐘響罷,天子駕出宮,陞崇政大殿,受百官朝賀。須臾,香毬撥轉,簾捲扇開。怎見的當日朝儀整肅?但見:

\begin{myquote}
皇風清穆,溫溫靄靄氣氤氳;麗日當空,郁郁蒸蒸雲靉靆。微微隱隱,龍樓鳳閣散滿天香霧;霏霏拂拂,珠宫寳殿映萬縷朝霞。大慶殿,崇慶殿,文德殿,集賢殿,燦燦爛爛,金碧交輝;乾明宫,坤寜宫,昭陽宫,合壁宫,清寜宫,光光彩彩,丹青炳煥。蒼蒼凉凉,日映着玉砌雕欄;裊裊嬰嬰,霧鎖着金椽畫棟。紫扉黄閣,寳鼎内縹縹緲緲沉檀齊爇;丹堦彤墀,玉砌臺明明朗朗畫燭高焚。龍龍鼕鼕,振天鼓擂疊三通;鑑鑑鍧鍧,長楽鐘撞一百八下。枝枝楂楂,叉刀手互相磕撞;搖搖曳曳,龍虎旂來往盤旋。錦衣花帽,擎着的是圓蓋傘、方蓋傘,上上下下開展;玉節龍蟠,駕着的是金輅輦、玉輅輦,左左右右相陳。又見那立金瓜、臥金瓜,三三兩兩;雙龍扇、單龍扇,疊疊重重。羣羣隊隊,金鞍馬、玉轡馬,性貌馴習;雙雙對對,寳匣象、駕轅象,猛力猙獰。鎭殿將軍,一個個長長大大賽天神,甲披金葉;侍朝勳衛,一人人齊齊整整如地煞,刀繫綉春。嚴嚴肅肅,殿門内擺列着糾儀御史,人人豸冠森聳,秉簡當胸;端端正正,姜擦邊立站定衆官員,個個錦衣炳煥,候宣聽旨。金殿上,參參差差齊開寳扇;畫棟前,輕輕款款高捲珠廉。文樓上,嘐嘐噦噦報時鷄人三唱;玉堦前,剌剌刮刮肅靜鞭響三聲。齊齊整整,侍螭頭列簪纓有五等之爵;巍巍蕩蕩,坐龍牀倚綉褥瞻萬乘之尊:遠遠望見頭戴十二旒平頂冠,身穿赭黄衮龍袍,腰繫藍田玉帶,脚靸烏油舃履,手執金鑲白玉圭,背靠九雷龍鳳扆。正是:

晴日明開青鎖闥,天風吹下御爐香。

千條瑞靄浮金闕,一朶紅雲捧玉皇。
\end{myquote}

這帝皇果生得堯眉舜目,禹背湯肩。若説這個官家,才俊過人,口賡詩韻,目數羣羊;善寫墨君竹,能揮薛稷書;通三敎之書,曉九流之典。朝歡暮樂,依稀似劔閣孟蜀王;愛色貪盃,彷彿如金陵陳後主。従十八歲登基即位,二十五年倒改了五遭年號;先改建中靖國,後改崇寜,改大觀,改政和,改重和,改宣和。

當下駕坐寳位,靜鞭響罷,文武百官,九卿四相,秉簡當胸,向丹墀五拜三叩頭禮,進上表章。已而有殿頭官身穿紫窄衫,腰繫金鑲帶,步着金堦,口傳聖勅道:「朕今即位二十禩於兹矣,艮嶽告成,上天降瑞。今値履端之慶,與卿等共之!」言未畢,班首中閃過一員大臣來,朝靴踏地響,袍袖列風生,官不知多大,玉帶顯功名。視之,乃左丞相、崇政殿大學士兼吏部尚書、太師、魯國公蔡京也。幞頭象簡,俯伏金堦叩首,口稱:「萬歲,萬歲,萬萬歲!臣等誠惶誠恐,稽首頓首:恭惟皇上御極二十禩以來,海宇清寜,天下豐稔。上天降鑒,禎祥疊見。日重輪,星重輝,海重瀾,聖上握乾符,永享萬年之正統;天保定,地保寜,人保安,皇圖膺寳曆,益增永壽之無疆。三邊永息於兵戈,萬國來朝於天闕。銀嶽排空,玉京挺秀。寳籙膺頒於昊闕,絳霄深聳於乾宮。臣等何幸,欣逢盛世,交際明良,永效華封之祝,常霑日月之光。不勝瞻天仰聖、激切屏營之至。謹獻頌以聞。」良久,聖旨下來:「賢卿獻頌,益見忠誠,朕心嘉悦。詔改明年為重和元年,正月元旦,受定命寳,肆赦、覃賞有差。」蔡太師承旨下來,殿頭官口傳聖旨:「有事出班早奏,無事捲廉退朝。」言未畢,見一人出離班部,倒笏躬身,緋袍象簡,玉帶金魚,跪在金堦,口稱:「光祿大夫、掌金吾衛事、太尉、太保兼太子太保臣朱勔,引奏天下提刑官員事,後面跪的兩准、兩浙、山東、山西、河東、河北、関東、関西、福建、廣南、四川等䖏刑獄千戶章隆等二十六員,例該考察,已更陞補,繳換劄付,合當引奏,未敢擅便,請旨定奪。」聖旨傳下來:「照例給領。」朱太尉承旨下來,天子袍袖一展,群臣皆散,駕即囬宫。

百官皆従端禮門兩分而出。那十二象不待牽而先走。鎭將長隨,紛紛而散,只聽甲響;叉刀力士、團子紅軍,盡盡而出,惟見戈明。朝門外,車馬縱横,侍仗羅列。人喧呼,海沸波翻;馬嘶喊,山崩地裂。衆提刑官皆出朝上馬,都來本衙門伺候,鐵桶相似。良久,只見承局㧱了印牌來傳道:「老爺不進衙門了,轎兒已在西華門裏安放。如今要往蔡爺李爺宅内拜冬去了。」以此衆官都散了。

西門慶與何千戶囬到家中,又過了一夕。到次日,衙門中領了劄付,向兵科中掛了號,又拜辭了翟管家,打點馱裝,收拾行李,與何千戶一同起身。何太監晚夕置酒餞行,囑付何千戶:「凡事請敎西門大人,休要自專,差了禮數。」従十一月十一日東京起身,兩家也有二十人跟隨,竟往山東大道而來。已是數九嚴寒之際,點水滴凍之時,一路上見了些荒郊野路,枯木寒鴉,疎林淡日影斜暉,暮雪凍雲迷晚渡,一山未盡一山來,後村已過前村望。比及剛過黄河,到水関八角鎭,驟然撞遇天起一陣大風。但見:

\begin{myquote}
非干虎嘯,豈是龍吟。卒律律寒飈撲面,急颼颼冷氣侵人。旣不能卸柳□□,暗藏着水妖山怪。初時節無蹤無影,次後來捲霧收雲。驚得那綠楊堤鷗鳥雙飛,紅蓼岸鴛鴦並起。則見那入紗窻,撲銀燈,穿畫閣,透羅裳,亂舞飄。吹花擺柳昏慘慘,走石揚砂白茫茫。刮得那大樹連聲吼,驚得那孤鴈落深濠。須臾砂石打地,塵土遮天。砂石打地,猶如滿天驟雨即時來;塵土遮天,好似百萬貔貅捲土至。赶趨得村落漁翁罷釣,捲鉤綸疾走囬家;山中樵子魂驚,掖斧斤急忙歸舍。唬得那山中虎豹縮着頭,隱着足,潛藏深壑。刮得那海底蛟拳着爪,蟠着尾,難顯猙獰。刮多時,只見那房上瓦飛似燕;吹良久,□□□山中走石如飛。瓦飛似燕,打得客旅迷蹤失道;石走如飛,唬得那商船緊纜收帆。大樹連根拔起,小樹有條無梢。這風大不大,眞個是吹折地獄門前樹,刮起酆都頂上塵。嫦娥急把蟾宫閉,列子空中呌救人。險些兒玉皇住不的崑崙頂,只刮的大地乾坤上下搖!
\end{myquote}

西門慶與何千戶坐着兩頂毡幃暖轎,被風刮得寸步難行。又見天色漸晚,恐深林中撞出小人來,對西門慶説:「投奔前村安歇一夜,明日風住再行。」找尋了半日,遠遠望見路傍一座古刹,數株疏柳,半堵横牆。但見:

\begin{myquote}
石砌碑横蔓草遮,迴廊古殿半欹斜。

夜深宿客無燈火,月落安禪更可嗟!
\end{myquote}

西門慶與何千戶入寺内投宿,見題着「黄龍寺」。見方丈内幾個僧人在那裏坐禪,又無燈火,房舍都毀壞,半用籬遮。長老出來問訊,旋吹火煮茶,伐草根喂馬。煮出茶來,西門慶行囊中帶得乾鷄臘肉、菓餅棋子之䫫,晚夕與何千戶胡亂食得一頓。長老爨一鍋豆粥喫了,過得一宿。次日風止,天氣始晴,與了老和尚一兩銀子相謝,作辭起身,往山東來。正是:

\begin{myquote}
王事驅馳豈憚勞,関山迢遞赴京朝。

夜投古寺無煙火,解使行人心内焦。
\end{myquote}

畢竟未知後來如何,且聽下回分解。

