\includepdf[pages={29,30},fitpaper=false]{tst.pdf}
\chapter*{第十五囬 \\佳人笑賞玩燈樓 狎客幫嫖麗春院}
\addcontentsline{toc}{chapter}{第十五囬 佳人笑賞玩燈樓 狎客幫嫖麗春院}
\markboth{{\titlename}卷之二}{第十五囬 佳人笑賞玩燈樓 狎客幫嫖麗春院}


\begin{myquote}
日墜西山月出東,百年光景似飄蓬。

點頭纔羡朱顔子,轉眼翻為白髮翁。

易老韶華休浪度,掀天富貴等雲空。

不如且討紅裙趣,依翠偎紅院宇中。
\end{myquote}

話説光陰迅速,又早到正月十五日。西門慶這裏,先一日差小廝玳安,送了四盤羹菜、兩盤壽桃、一壜酒、一盤壽麵、一套織金重絹衣服,寫吴月娘名字:「西門吴氏歛袵拜」,送與李瓶兒做生日。李瓶兒纔起來梳粧,叫了玳安兒到臥房裏,説道:「前日打攪你大娘那裏,今日又教你大娘費心送禮來。」玳安道:「娘多上覆,爹也上覆二娘,不多些微禮,與二娘賞人。」李瓶兒一面吩咐迎春,外邊明間内放小桌兒,擺了四盒茶食管待玳安。臨出門,與二錢銀子、八寳兒一方閃色手帕:「到家多上覆你列位娘,我這裏使老馮拿帖兒請去,好歹明日都光降走走。」玳安磕頭出門,兩個抬盒子的,與一百文錢。李瓶兒這裏隨即使老馮兒用請書盒兒,拿着五個柬帖兒,十五日請月娘與李嬌兒、孟玉樓、潘金蓮、孫雪娥。又捎了一個帖,暗暗請西門慶,那日晚夕赴席。

月娘到次日,留下孫雪娥看家,同李嬌兒、孟玉樓、潘金蓮,四頂轎子出門。都穿着粧花錦綉衣服,來興、來安、玳安、畫童,四個小廝跟隨,竟到獅子街燈市李瓶兒新買的房子:門面四間,到底三層,臨街是樓。儀門進去,兩邊廂房,三間客坐,一間稍間;過道穿進去第三層,三間臥房,一間廚房;後邊落地緊靠着喬皇親花園。李瓶兒知月娘衆人來看燈,臨街樓上設放圍屏桌席,懸掛許多花燈。先迎接到客位内見畢禮數,次讓入後邊明間内待茶,房裏換衣裳,擺茶,俱不必細説。到午間,李瓶兒客位内設四張桌席,叫了兩個唱的董嬌兒、韓金釧兒,彈唱飲酒。及酒過五巡,食割三道,前邊樓上設着細巧添換酒席,又請月娘衆人登樓,看燈頑耍。樓簷前掛着湘簾,懸着彩燈。吴月娘穿着大紅粧花通袖襖兒,嬌綠緞裙,貂鼠皮襖;李嬌兒、孟玉樓、潘金蓮,都是白綾襖兒、藍緞裙;李嬌兒是沉香色遍地金比甲,孟玉樓是綠遍地金比甲,潘金蓮是大紅遍地金比甲;頭上珠翠堆盈,鳳釵半卸,鬢後挑着许多各色燈籠兒。俱搭伏定樓窗,往下觀看。見那燈市中人煙凑集,十分熱鬧。當街搭數十座燈架,四下圍列些諸門買賣。玩燈男女,花紅柳綠,車馬轟雷,鰲山聳漢。怎見好燈市?但見:

\begin{myquote}
山石穿雙龍戲水,雲霞映獨鶴朝天。金蓮燈、玉樓燈,見一片珠璣;荷花燈、芙蓉燈,散千圍錦綉。綉毬燈,皎皎潔潔;雪花燈,拂拂紛紛。秀才燈,揖讓進止,存孔孟之遺風;媳婦燈,容德溫柔,效孟姜之節操。和尚燈,月明與柳翠相連;通判燈,鍾馗共小妹並坐。師婆燈,揮羽扇,假降邪神;劉海燈,背金蟾,戯吞至寳。駱駝燈、青獅燈,馱無價之奇珎,咆咆哮哮;猿猴燈、白象燈,進連城之秘寳,頑頑耍耍。七手八脚,螃蠏燈倒戲清波;巨口大髯,鮎魚燈平吞綠藻。銀荷鬦彩,雪柳爭輝。雙雙隨綉带香毬,縷縷拂華旛翠幰。魚龍沙戯,七眞五老獻丹書;吊掛流蘇,九夷八蠻來進寳。村裏社鼓,隊隊共喧闐;百戲貨郎,樁樁齊鬦巧。轉燈兒一來一往,吊燈兒或仰或垂。琉璃瓶現美女奇花,雲母障呈瀛州閬苑。往東看:雕漆牀、螺鈿牀,金碧交輝;向西瞧:羊皮燈、掠彩燈,錦綉奪眼。北一带,都是古董玩器;南壁廂,盡皆書畫瓶爐。王孫爭看,小欄下蹴鞠齊雲;仕女相携,高樓上妖嬈衒色。卦肆雲集,相幕星羅:講新春造化如何,定一世榮枯有准。又有那站高坡打談的,詞曲楊恭;到看這げ响鈸遊脚僧,演説三藏。賣元宵的,高堆菓餡;粘梅花的,齊插枯枝。剪春娥,鬢邊斜插鬧東風;禱涼釵,頭上飛金光耀日。圍屏畫石崇之錦帳,珠簾繪梅月之雙清。雖然覽不盡鰲山景,也應豐登快活年!
\end{myquote}

吴月娘看了一囬,見樓下人亂,和李嬌兒各歸席上吃酒去了。惟有潘金蓮孟玉樓同兩個唱的,只顧搭伏着樓窗子,引下人觀看。那潘金蓮一徑把白綾襖袖子摟着,顯他遍地金掏袖兒,露出那十指春葱來,带着六個金馬鐙戒指兒,探着半截身子,口中嗑瓜子兒,把嗑了的瓜子皮兒都吐下來,落在人身上,和玉樓兩個嘻笑不止。一囬指道:「大姐姐,你來看!那家房簷底下,掛了兩盞玉綉毬燈,一來一往,滾上滾下,且是倒好看!」一囬又道:「二姐姐你來看!這對門架子上挑着一盞大魚燈,下面還有許多小魚鱉蝦蟹兒跟着他,倒好耍子!」一囬又叫孟玉樓:「三姐姐你看!這門首裏,這個婆兒燈,那老兒燈!」正看着,忽然被一陣風來,把個婆子兒燈下半截刮了一個大窟礲。婦人看見,笑個不了。引惹的那樓下看燈的人挨肩擦背,仰望上瞧,通擠匝不開,都壓ばば兒,須臾,哄圍了一圈人。内中有幾個浮浪子弟,直指着談論。一個説道:「一定是那公侯府第裏出來的宅眷。」一個又猜:「是貴戚皇孫家豔妾來此看燈。不然,如何内家粧束?」那一個説道:「莫不是院中小娘兒,是那大人家叫來這裏看燈彈唱?」又一個走過來,便道:「只我認的,你們都猜不着。你把他當唱的,把後面那兩個放到那裏?我告説吧:這兩個婦人也不是小可人家的,他是閻羅大王的妻,五道將軍的妾,是咱縣門前開生薬舖、放官吏債西門大官人的婦女!你惹他怎的?想必跟他大娘子來這裏看燈。這個穿綠遍地金比甲的,我不認的。那穿大紅遍地金比甲兒,上带着個翠面花兒的,倒好似賣炊餅武大郎的娘子。大郎因為在王婆茶坊内捉姦,被大官人踢中了死了,把他娶在家裏做了妾。後次他小叔武松東京囬來告狀,誤打死了皂隸李外傳,被大官人墊發充軍去了。如今一二年不見,出落的這等標致了。」正説着,只見一個多口過來説道:「你們沒要緊,指説他怎的?咱們散開罷。」

樓上吴月娘見樓下人圍的多了,叫了金蓮玉樓歸席坐下,聽着兩個粉頭彈唱燈詞飲酒。坐了一囬,月娘要起身,説道:「酒夠了。我和他二娘先行一步,㽞下他姊妹兩個,再坐一囬兒,以盡二娘之情。今日他爹不在家,家裏無人,光丢着些丫頭們,我不放心。」這李瓶兒那裏肯放,説道:「好大娘,奴沒敬心也怎的?今日大娘菜也没好生揀一筯兒。大節間,燈兒也沒點,飯兒也沒上,就要家去?就是西門爹不在家中,還有他姑娘們哩,怕怎的?待月色上來的時候,奴送四位娘去。」月娘道:「二娘,不是這等説。我又不大十分用酒,㽞下他姊妹兩個,就同我在這裏一般。」李瓶兒道:「大娘不用,二娘也不吃一鍾,也沒這個道理。想奴前日在大娘府上,那等鍾鍾不辭,衆位娘竟不肯饒我。今日來到奴這湫涸之處,雖無甚物供獻,也盡奴一點窮心。」於是拿大銀鍾遞與李嬌兒,説道:「二娘好歹吃一盃兒。大娘奴曉的吃不的了,不敢奉大盃,只奉小盃兒哩。」于是滿斟遞與。月娘因説李嬌兒:「二娘,你用過此盃罷!」兩個唱的,月娘每人與了他二錢銀子。待的李嬌兒吃過酒,月娘起身,囑咐玉樓金蓮:「我兩個先起身。我囬去便使小廝拿燈籠來接你們,也就來罷。家裏沒人。」玉樓應諾。李瓶兒送月娘李嬌兒到門首上轎去了。歸到樓上,陪玉樓金蓮飲酒。看看天晚,玉兔東生,樓上點起燈來。兩個唱的彈唱飲酒,不在話下。

却説西門慶那日同應伯爵謝希大兩個,家中吃了飯,同往燈市裏遊玩。到了獅子街東口,西門慶因為月娘衆人今日都在李瓶兒家樓上吃酒,恐怕他兩個看見,就不往西街去看大燈,只到賣紗燈的跟前,就囬了。不想轉過彎來,撞遇孫寡嘴、祝日念,唱喏、説道:「連日不會哥,心中渴望。」見了應伯爵謝希大,罵道:「你兩個天殺的好人兒!你來和哥遊玩,就不説叫俺一聲兒?」西門慶道:「祝兄弟,你錯怪了他兩個,剛纔也是路上相遇。」祝日念道:「如今看了燈,往那裏去?」西門慶道:「同衆位兄弟到大酒樓上吃三盃兒。不是也請衆兄弟家去,房下們今日都往人家吃酒去了。」祝日念道:「比是哥請俺們到酒樓上,咱何不往裏邊,望望李桂姐去?只當大節間往他家拜年去,混他混。前日俺兩個在他家,望着俺們好不哭哩。説他従臘月裏不好到如今,大官人通影邊兒不進裏面看他看兒。俺們便囬說:『只怕哥事忙』,替哥摭過了。哥今日倒閑,俺們情願相伴哥進去走走。」西門慶因記掛着晚夕李瓶兒還席,推辭道:「今日我還有小事,不得去,明日罷。」怎禁這夥人死拖活拽,于是同進去院中。正是:

\begin{myquote}
柳底花陰壓路塵,一回遊賞一囬新。

不知買盡長安笑,活得蒼生幾户貧?
\end{myquote}

西門慶同衆人到了李家,桂卿正打扮着在門首站立,一面迎接入中堂相見了,都道了萬福。祝日念高叫道:「快請三媽出來!還虧俺衆人,今日請的大官人來了。」少頃,老虔婆扶拐而出,向西門慶見畢禮數,説道:「老身又不曾怠慢了姐夫,如何一向不進來看看姐姐兒?想必別處另敍了新婊子來。」祝日念走來插口道:「你老人家會猜算。俺大官人近日相與了個絶色的婊子,每日只在那裏行走,不想你家桂姐兒。剛纔不是俺二人在燈市裏撞見拉他來,他還不來哩。老媽不信,問孫天化就是了。」因指着應伯爵謝希大説道:「這兩個天殺的,和他都是一路神祇。」老虔婆聽了,呷呷笑道:「好應二哥!俺家没惱着你,如何不在姐夫面前美言一句兒?雖故姐夫裏邊頭緒兒多,常言道:好子弟不嫖一個粉頭,好粉頭不接一個孤老。天下錢眼兒都一樣。不是老身誇口説,我家桂姐也不醜,姐夫自有眼界,也不消人説。」孫寡嘴道:「我是老實説,哥如今新敍的這個婊子不是裏面的,是外面的婊子,還把裏邊人㒲巴。」敎那西門慶聽了,趕着孫寡嘴只顧打,説道:「老媽,你休聽這天災人祸老油嘴,弄殺人的!」孫寡嘴和衆人笑成一塊。

西門慶向袖中掏出三兩銀子來,遞與桂卿:「大節間,我請衆朋友。」桂卿道:「哄我!」不肯接,遞與老媽,老媽説道:「怎麽兒姐夫就笑話我家,大節下拿不出酒菜兒管待列位老爹?又教姐夫壞鈔,拿出銀子,顯的俺們院裏人家只是愛錢了。」應伯爵走過來説道:「老媽,你依我收了,只當正月裏頭二主子搶快。快安排酒來俺們吃。」那虔婆説道:「這個理上却使不得。」一壁推辭,一壁把銀子接的袖了,深深道了個萬福,説道:「謝姐夫的布施。」應伯爵道:「老媽,你且住,我説個笑話兒你聽聽:一個子弟在院裏嫖小娘兒,那一日作耍,裝做貧子進去。老媽見他衣服藍褸,不理他。坐了半日,茶也不拿出來。子弟説:『媽,我肚飢,有飯尋些來我吃。』老媽道:『米囤也晒了,那討飯來?』子弟又道:『既沒飯,有水拿些來我洗洗臉罷。』老媽道:『少挑水錢,連日沒送水來。』這子弟向袖中取出十兩一錠銀子放在桌子上,教買米雇水去。慌的老媽沒口子道:『姐夫吃了臉洗飯?洗了飯吃臉?』」把衆人都笑了。虔婆道:「你還是這等快取笑。可可兒的來,自古有恁説,沒這事。」應伯爵道:「你拿耳朵,我對你説。大官人新近請了花二哥婊子——後巷兒吴銀兒了,不要你家桂姐了。今日不是我們纏了他來,他還往你家來哩!」虔婆笑道:「我不信,俺桂姐,今日不是強口,比吴銀兒好多着哩!我家與姐夫,是快刀兒割不断的親戚。姐夫是何等人兒?他眼裏見的多,着緊䖏金子也估出個成色來。」説畢,客位内放四把校椅,應伯爵謝希大祝日念孫天化四人上坐,西門慶對席。老媽下去收拾酒菜去了。

半日,李桂姐出來。家常挽着一窝絲杭州攢,金纍絲釵,翠梅花鈿兒,珠子箍兒,金燈籠墜子。上穿白綾對衿襖兒,粧花眉子,綠遍地金掏袖;下著紅羅裙子。打扮的粉粧玉琢。望下不當不正道了萬福,與桂卿一邊一個,打横坐下。少頃,頂老彩漆方盤拿七盞來,雪錠般盞兒、銀杏葉茶匙、玫瑰潑滷瓜仁泡茶,甚是馨香美味。桂卿桂姐每人遞了一盞。陪着吃畢茶,接下茶托去。保兒上來打抹春臺。纔待收拾擺放案酒,忽見簾子外探頭舒腦,有幾個穿藍褸衣者,謂之架兒,進來跪下,手裏拿三四升瓜子兒:「大節間孝順大老爹!」西門慶只認頭一個叫于春兒,問:「你們那幾位在這裏?」于春道:「還有段綿紗青聶鉞在外邊伺候。」段綿紗進來,看見應伯爵在裏,説道:「應爹也在這裏!」連忙磕了頭。西門慶起來,吩咐收了他瓜子兒,打開銀子包兒,捏一兩一塊銀子掠在地下。于春兒接了,和衆人趴在地下,磕了個頭,説道:「謝爹賞賜!」往外飛跑。有〔朝天子〕單道這架兒行藏為證:

\begin{myquote}
這家子打和,那家子撮合,他的本分少虚頭大。一些兒不巧人騰挪,遶院裏都踅過。席面上幫閑,把牙兒閑磕。攘一回纔散火,賺錢又不多。歪斯纏怎麽?他在虎口裏求津唾。
\end{myquote}

西門慶打發架兒出門,安排酒上來吃酒。桂姐滿泛金盃,雙垂紅袖。餚烹異品,菓獻時新,倚翠偎紅,花濃酒豔。酒過兩巡,桂卿外與桂姐一個琵琶一個箏,兩個彈着,唱了一套「霽景融和」。正唱在熱鬧䖏,見三個穿青衣、黄板辮者——謂之圓社——手裏捧着一個盒兒,盛着一隻燒鵝,提着兩瓶老酒,「大節間來孝順大官人貴人!」向前打個半跪。西門慶平昔認的,一個喚白秃子,一個是小張閑,那一個是羅囬子。因説道:「你們且外邊候候兒,待俺們吃過酒,踢三跑。」于是向桌上拾了四盤下飯、一大壶酒、一碟點心,打發衆圓社吃了,整理氣毬齊備。西門慶出來外面院子裏,先踢了一跑。次教桂姐上來,與兩個圓社踢。一個揸頭,一個對障。抅踢拐打之間,無不假喝彩奉承,就有些不到䖏,都快取過去了。反來向西門慶面前討賞錢,説:「桂姐的行頭,比舊時越發踢熟了。撇來的丢拐,敎小人每凑手脚不迭。再過一二年,這邊院中,似桂姐的這行頭,就數一數二的蓋了羣,絶了倫,強如二條巷董家女兒數十倍。」當下桂姐踢了兩跑下來,使的塵生眉畔,汗濕腮邊,氣喘吁吁,腰肢困乏。袖中取出春扇兒搖涼,與西門慶携手並觀,看桂卿與謝希大張小閑踢行頭。白秃子羅囬子在傍虚撮腳兒等漏,往來拾毬。亦有〔朝天子〕一詞,單道這踢圓社的始末為證:

\begin{myquote}
在家中也閑,到處刮涎。生理全不幹,氣毬兒不離在身邊。每日街頭站,窮的又不趨,富貴他偏羡。従早晨直到晚,不得甚飽餐。賺不得大錢,他老婆常被人包占。
\end{myquote}

西門慶正看着衆人在院内打雙陸、踢氣毬,飲酒,只見玳安騎馬來接,悄悄附耳低言説道:「大娘二娘家去了。花二娘敎小的請爹早些過去哩。」這西門慶聽了,暗暗叫玳安把馬吊在後邊門首等着。於是酒也不吃,拉桂姐房中,只坐了沒多一囬兒,就出來推淨手,於後門上馬,一溜煙走了。應伯爵使保兒去拉扯,西門慶只説:「我家裏有事。」那裏肯回來。教玳安拿了一兩五錢銀子,打發三個圓社。李家恐怕他又往後巷吴銀兒家,使丫鬟直跟至院門首方囬。應伯爵等衆人還吃到二更鼓纔散。正是:唾罵由他唾罵,歡娱我且歡娱。

畢竟未知後來何如,且聽下囬分解。

