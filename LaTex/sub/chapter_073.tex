\includepdf[pages={145,146},fitpaper=false]{tst.pdf}
\chapter*{第七十三囬 \\潘金蓮不憤憶吹簫 郁大姐夜唱鬧五更}
\addcontentsline{toc}{chapter}{第七十三囬 潘金蓮不憤憶吹簫 郁大姐夜唱鬧五更}
\markboth{第七十三囬 潘金蓮不憤憶吹簫 郁大姐夜唱鬧五更}{第七十三囬 潘金蓮不憤憶吹簫 郁大姐夜唱鬧五更}
\thispagestyle{empty}

\begin{myquote}
巧厭多勞拙厭閒,善嫌懦弱惡嫌頑;

富遭嫉妬貧遭辱,勤又貪圖儉又慳;

觸目不分皆笑拙,見機而作又疑奸。

思量那件合人意,為人難做做人難!
\end{myquote}

話説應伯爵囬家去了。西門慶正在花園藏春塢坐着,看泥水匠打地爐炕:墻外燒火,裏邊地暖如春,安放花草,庶不至煤煙薰觸。忽見平安㧱進帖來,禀說:「帥府周爺那裏差人送分資來了。」盒内封着五封分資:周守備、荆都監、張團練、劉薛二内相,「每人五星,粗帕二方,奉引賀敬。」西門慶令左右收入後邊,㧱囘帖打發來人去了。

且說那日楊姑娘與吴大妗子、潘姥姥,坐轎子先來了,然後薛姑子、大師父、王姑子,並兩個小姑子妙趣、妙鳳,並郁大姐,都買了盒兒來與玉樓做生日。吴月娘在上房擺茶,衆姊妹都在一䖏陪侍。須臾喫了茶,各人都取便去了。潘金蓮想着要與西門慶做白綾帶兒,三不知走到房裏,㧱過針綫匣,揀一條白綾兒,用扣針兒親手䋺龍帶兒,用纖手向减粧磁盒兒内傾了些顫聲嬌薬末兒,裝在裏面,周圍又用倒口針兒撩縫兒,甚是細法,預備晚夕要與西門慶雲雨之歡。不想薛姑子驀地進房來,送那安胎氣的衣胞符薬。這婦人連忙收過一邊,陪他坐的。這薛姑子見左右無人,悄悄遞與他,向他説:「都整理完備了。你揀個壬子日空心服,到晚夕與官人在一處,管情一度就成胎氣。你看後邊大菩薩,也是貧僧替他安的胎,今也有了半肚子了。我還説個法兒與你,縫做個錦香囊,我贖道硃砂雄黄符兒,安放在裏面,帶在身邊,管情就是男胎,好不准驗。」這婦人聽了滿心歡喜,一面接了符薬,藏放在箱中。㧱過曆日來看,二十九日是壬子日。於是就稱了三錢銀子送與他説:「這個不當什麽,㧱到家買根菜兒喫。等坐胎之時,你明日捎了硃砂符兒來着,我尋疋絹與你做鍾袖。」薛姑子道:「菩薩,快休計較!我不像王和尚那樣利心重。前者因過世那位菩薩念經,他說我攙了他的主顧,好不和我兩個嚷鬧,到處㧱言語丧我。我的爺,隨他墮業,我不與他爭執。我只替人家行好,救人苦難!」婦人道:「薛爺,你只行好事,各人心地不同。我這裏勾當,你也休和他說。」薛姑子道:「法不傳六耳,我肯和他説?去年為後邊大菩薩喜事,他還説我背地得了多少錢,擗了一半與他纔罷了。一個僧家,戒行也不知,利心又重,得了十方施主錢糧,不修功果。到明日死後,披毛戴角還不起!」説了囘話,婦人教春梅:「看茶與薛爺喫。」那姑子喫了茶,又同他到李瓶兒那邊參了參靈,方歸後邊來。

約後晌時分,月娘放兩個桌兒,炕屋裏請諸堂客並三個姑子坐的。明間内錦帳圍屏,放八僊桌,鋪着火盆,擺的案酒整齊。晚夕,孟玉樓與西門慶遞酒。西門慶穿着何太監與他那五彩飛魚氅衣,白綾襖子,同月娘居上,其餘四位都兩邊列坐。不一時,堂中畫燭高燒,壺内羊羔滿泛。邵謙、韓佐,兩個優兒,銀箏象板,月面琵琶,席前彈唱「紛紛瑞靄飄,朶朶祥雲墜」。玉樓打扮粉粧玉琢,蓮臉生春,與西門慶遞酒,花枝招颭,綉帶飄飄,磕了四個頭,然後方與月娘衆姊妹俱見了禮,安席坐下。只見陳經濟向前,大姐執壺,先遞了西門慶、月娘,後與玉樓上壽。行畢禮,傍邊坐下。廚下壽麵點心添換,一齊㧱上來。只見來安㧱進盒兒來說:「應寳送人情來了。」西門慶敎月娘收了,敎來安:「送應二娘帖兒去,請你應二爹和大舅來坐坐。我曉的他娘子兒明日也是不來,請二哥來坐坐罷。改日回人情與他就是了。」來安㧱帖兒同應寳去了。西門慶坐在上面,不覺想起去年玉樓上壽,還有李大姐。今日姊妹五個只少了他,由不得心中痛,眼中落淚。不一時,李銘下邊喫過湯飯上來了,斟上酒;兩個小優兒也來了,月娘吩咐:「你會唱『比翼成連理』不會?」韓佐道:「小的有。」纔待㧱起楽器來彈唱,被西門慶叫近前來,吩咐:「你唱一套『憶吹簫』我聽罷。」兩個小優連忙改調唱〈集賢賓〉:

\begin{myquote}
「憶吹簫玉人何處也,今夜病較添些。白露冷秋蓮香謝,粉牆低皓月光斜。止不過暫時間鏡破釵分,倒勝似數十年信断音絶。對西風倚樓空自嗟。望不断嶺樹重疊,怕的是流光去馬,雁陣擺蛇。」

{\markfont〈逍遙楽〉}「歡娱前夜,喜報燈花,香生帶結。剛得個和協,誰承望又早離别。常記得相靠相偎笑語喋。畫堂中那日驕奢:受用些樽中綠蟻,扇底紅牙,枕上蝴蝶。」

{\markfont〈醋葫蘆〉}「我和他初相逢臉带羞,乍交歡心尚怯。半裝醉、半裝醒、半裝獃。兩情濃到今難棄捨。錦帳裏鴛衾纔方溫熱,把一枝鳳凰簪兒掂做了三兩截。」
\end{myquote}

又:

\begin{myquote}
「我為他挑着燈將好句兒裁,背着人將心事說。直等到碧梧窗外影兒斜,惜花心怕將春漏泄。步蒼苔脚尖輕躡,露珠兒常污了踏青靴。」
\end{myquote}

又:

\begin{myquote}
「我為他親朋上將謊話兒丢,他為我母親行將喬樣兒摭。我為他在家中費盡了巧喉舌,他為我褪湘裙杜鵑花上血。」
\end{myquote}

原來潘金蓮見唱此詞,盡知西門慶念思李瓶兒之意。唱到此句,在席上故意把手放在臉兒上,這點兒那點兒羞他,說道:「孩兒,那裏猪八戒走到冷舖中坐着,你怎的醜的沒對兒!一個後婚老婆,又不是女兒,那裏討杜鵑花上血來?好個没羞的行貨子!」西門慶道:「怪奴才,我自知道,你那裏曉的什麽?」那個小優唱道:

又:

\begin{myquote}
「我為他耳輪兒常熱,他為我面皮紅羞把扇兒遮。」

{\markfont〈梧葉兒〉}「一個是相府内懷春女,一個是君門前彈劍客,半路裏忽逢着。剛幾個千金夜,忽剌八抛去也,我怎肯恁隨邪,又去把牆花亂折?」

{\markfont〈後庭花〉}「夢了些虚飄飄枕上蝶,聽了些咭叮噹簷前鐵。剛合上溫郎鏡,却又早攔囘桌氏車。我這裏痛傷嗟,鴛帳冷香消蘭麝。困將來剛睡些,望陽臺道路賒。那憂愁怎打疊,這相思索害也。看銀河直又斜,對孤燈明又滅。」

{\markfont〈青哥兒〉}「呀!風亂掃堦前堦前黄葉,雲半遮柳梢柳稍殘月。這離情更比前比前春較陡些。害的來乜斜,瘦的來唓嗻。待桑田重變海枯竭,還不了風流業。」

{\markfont〈浪裏來煞〉}「這愁呵剛不在眼角踅,又來到眉上惹。恨不的倩三尸肺腑細鐫碣。有一日綉幃中玉肌重廝貼,我將他指尖兒輕捏,直説到樓頭北斗柄兒斜。」
\end{myquote}

唱畢,那潘金蓮不憤他唱這套,兩個在席上只顧拌嘴起來。月娘就有些看不上,便道:「六姐,你也耐煩,兩個只顧且強什麽?楊姑奶奶和他大妗子丢的在屋裏冷清清的,没個人兒陪他。你們着兩個進去陪他坐坐兒,我就來。」當下金蓮和李嬌兒往房裏陪楊姑娘潘姥姥大妗子坐去了。

不一時,只見來安向前說:「應二娘帖兒送到了。二爹來了,大舅便來。」西門慶道:「你對過請溫師父來坐坐。」因對月娘說:「你吩咐厨下㧱菜出來,我前邊陪他坐坐。」又呌李銘:「你往前邊唱來罷。」李銘即跟着西門慶出來,西廂房内陪伯爵坐的,又謝他人情:「明日請令正好歹來看看。」伯爵道:「他怕不得來,家下沒人。」良久,溫秀才到,作揖坐下。伯爵擧手道:「早晨多有累老先生兒。」溫秀才道:「豈敢。」吴大舅也到了,相見讓位畢,一面琴童兒秉燭來,四人圍暖爐坐定。來安㧱着春盛案酒,擺在桌上。伯爵燈下看見西門慶白綾襖子上,罩着青緞五彩飛魚蟒衣,張爪舞牙,頭角崢嶸,揚鬚鼓鬣,金碧掩映,蟠在身上,唬了一跳,問:「哥,這衣服是那裏的?」西門慶便立起身來,笑道:「你們瞧瞧,猜是那裏的?」伯爵道:「俺們如何猜得着?」西門慶道:「此是東京何太監送我的。我在他家喫酒,因害冷,他㧱出這件衣服與我披。這是飛魚,朝廷另賜了他蟒龍玉帶,他不穿這件,就相送了。此是一個大分上。」伯爵方極口誇獎:「這花衣服,少說也值幾個錢兒。此是哥的先兆,到明日高轉,做到都督上,愁没玉帶蟒衣?何況飛魚,穿過界兒去了!」說着,琴童安放鍾筯,湯飯、點心、酒上來了。李銘在面前彈唱。伯爵道:「也該進去與三嫂遞盃酒兒纔好,如何就喫酒?」西門慶道:「我兒,你有孝順之心,往後邊與三嫂磕個頭兒就是了,說他怎的!」伯爵道:「不打緊,等我磕頭去。着緊磕不成頭,炕沿兒上見個意思兒出來就是了。」被西門慶向他頭上儘力打了一下,罵道:「你這狗才,單管恁沒大小!」伯爵道:「孩兒們若肯了,那個好意做大?」兩個又犯了囘嘴。

不一時,㧱將壽麵來。西門慶讓吴大舅溫秀才伯爵喫。西門慶因在後邊喫了,遞與李銘喫了。那李銘喫了,又上來彈唱。伯爵教吴大舅吩咐曲兒敎他唱。大舅道:「不要索落他,隨他揀熟的唱去。」西門慶道:「大舅好聽〈瓦盆兒〉這一套兒。」一面令琴童斟上酒,李銘於是箏排雁柱,款定冰弦,唱了一套「敎人對景無言,終日減芳容。」下邊去了。只見來安上來禀説:「厨子家去,請問爹,明日呌幾名答應?」西門慶吩咐:「六名厨役,二名茶酒。明日具酒筵共五桌,俱要齊備。」來安應諾去了。吴大舅便問:「姐夫,明日請甚麽人?」西門慶悉把安郎中作東請蔡九知府說了。吴大舅道:「明日大巡在姐夫這裏喫酒,又好了。」西門慶道:「怎的說?」吴大舅道:「還是我修倉的事,就在大巡手裏題本。望姐夫明日說說,教他青目青目。到年終他考滿之時,圖他保擧一二,就是姐夫情分。」西門慶道:「這不打緊,大舅明日寫個履歷揭帖來,等我會便和他説。」這大舅連忙下來打恭。伯爵道:「老舅,你老人家放心。你是個幫根主子,不替你老人家説,再替誰説?管情消不得吹灰之力,一箭就上垜。」前邊喫酒,到二更時分散了。西門慶打發了李銘等出門,就吩咐:「明日俱早來伺候。」李銘等去了,小廝收進家活。上房内擠着一屋裏人,聽見前邊散了,都往那房裏去了。

卻説金蓮只説往他屋裏去,慌的往外走不迭。不想西門慶進儀門來了,他便藏在影壁邊,黑影兒裏看着西門慶進入上房,悄悄走來窗下聽覷。只見玉簫站在堂屋門首,說道:「五娘怎的不進去?爹進來屋裏來,和三娘都坐着不是。」又問:「姥姥怎的不見?」金蓮道:「老行貨子,他害身上疼,往房裏睡去了。」良久,只聽月娘便問:「你今日怎的叫恁兩個新小王八子?唱又不會唱,只一味會『三弄梅花』。」玉樓道:「只你臨了敎他唱『鴛鴦浦蓮開』,他纔依了你唱這套。好個猾小王八子,又不知呌什麽名字,一日在這裏只是頑。」西門慶道:「他兩個一個呌韓佐,一個呌邵謙。」月娘道:「誰曉的他呌什麽謙兒、李兒!」不防金蓮慢慢躡足潛踪,掀開簾兒進去,立在煖炕兒背後,便道:「你問他,正經姐姐吩咐的曲兒不敎他唱,平白胡枝扯葉的,敎他唱什麽『憶吹簫』,『李吹簫』,支使的個小王八子亂騰騰的,不知依那個的是。」這玉樓扭囘一瞟,看見是金蓮,便道:「是這一個六丫頭,你在那裏來?猛可說出句話,倒唬我一跳。單愛行鬼路兒!你從多咱跕在我背後?怎的没看見你進來脚步兒響?」小玉道:「五娘在三娘背後好小一囘兒。」金蓮點着頭兒向西門慶道:「哥兒,你膿着些兒罷了!你的小見識兒,只說人不知道。他是甚『相府中懷春女?』他和我都是一般後婚老婆!什麽他為你『褪湘裙杜鵑花上血』,三個官唱兩個喏,誰見來?孫小官兒問朱吉,别的都罷了,這個我不敢許!可是你對人說的,自従他死了,好應心的菜也没一碟子兒。沒了王屠,連毛喫猪,空有這些老婆,睜着你日逐只お屎哩?現有大姐姐在上,——俺們便不是上數的,可不着你那心的了!——一個大姐姐恁當家理紀,也扶持不過你來?可可兒只是他好來?他死,你怎的不拉掣住他?當初没他來時,你也過來,如今就是諸般兒稱不上你的心了!題起他來,就疼的你這心裏格地地的,㧱别人當他,借汁兒下麵,也喜歡的你了不的!只他那屋裏水好喫麽?」月娘道:「好六姐,常言不說的:好人不長壽,祸害一千年。自古鏇的不圓砍的圓。你我本等是瞞貨,應不上他的心,隨他說去罷了!」金蓮道:「不是咱不說他,他說出來的話灰人的心,只說人憤不過他。」那西門慶只是笑,罵道:「怪小淫婦兒,胡說了你!我在那裏說過這個話來?」金蓮道:「還是請黄内官那日,你沒對着應二和溫蠻子說:従他死了,好菜也拿沒出一碟子來。怪不的你老婆都死絶了!就是當初有他在,也不怎麽的。到明日,再扶一個起來和他做對兒麽?賊沒廉耻撒根基的貨!」說的西門慶急了,跳起來,趕着㧱靴脚踢他。那婦人奪門一溜煙跑了。

這西門慶趕出去不見他,只見春梅站在上房門首,就一手搭伏着春梅肩背,往前邊來。月娘見他醉了,巴不的打發他前邊去睡,要聽三個姑子晚夕宣卷,於是教小玉打個燈籠,送他前邊去。金蓮和玉簫站在穿廊下黑影中,西門慶沒看見他。玉簫向金蓮道:「我猜爹管情向娘屋裏去了。」金蓮道:「他醉了快發訕,由他先睡,等我慢慢進去。」這玉簫便道:「娘你等等,我取些菓子兒捎與姥姥喫去。」於是走到牀房内,袖出兩個柑子,兩個蘋婆,一包蜜餞,三個石榴與婦人。婦人接的袖了,一直走到他前邊。只見小玉送了西門慶囬來,說道:「五娘端的在那邊?爹好不尋五娘。」這金蓮到房門首,不進去,悄悄向窗眼裏望裏張覷,覷見西門慶坐在牀上,正摟着春梅做一處頑耍。恐怕攪擾他,連忙走到那邊屋裏,將菓子交付與了秋菊,因問:「姥姥睡没有?」秋菊道:「睡了一大囬了。」囑付他:「菓子好生收在揀粧内。」原復往後邊來。只見月娘、李嬌兒、孟玉樓、西門大姐、大妗子、楊姑娘,幷三個姑子,帶兩個小姑子妙趣妙鳳,坐了一屋裏人。姑子便盤膝坐在月娘炕上,薛姑子在當中,放着一張炕桌兒,炷了香,衆人都圍着他,聽他說佛法。

只見金蓮笑掀簾子進來。月娘道:「你惹下祸來,他往屋裏尋你去了。你不打發他睡,如何又來了?我還愁他到屋裏打你?」金蓮笑道:「你問他敢打我不敢?」月娘道:「他不打你嫌腥,我見你頭裏話出來的忒緊了,常言漢子臉上有狗毛,老婆臉上有鳳毛。他有酒的人,我怕一時激犯他起來,激的惱了,不打你打狗不成?俺們倒替你捏兩把汗,原來你倒這等潑皮!」金蓮道:「他就惱,我也不怕他。看不上那三等兒九格的!正經姐姐吩咐的曲兒不敎唱,且東溝犁西溝耙,支使的個小王八子亂烘烘的,不知依那個的是。就是今日孟三姐的好日子,不該唱『憶吹簫』這套離別之詞。人也不知死那裏去了,偏有那些佯慈悲假孝順,我和剌不上!」大妗子道:「你姐兒們亂了這一囘,我還不知因爲什麽來。姑夫好好的進來坐着,怎的又出去了?」月娘道:「大妗子,你還不知道。那一個因想起李大姐來,說年時孟三姐生日還有他,今年就没他了。落了幾點眼淚,教小優兒唱了一套『憶吹簫,玉人兒何處也』。這一個就不憤他唱這詞,剛纔搶白了爹幾句。搶白的那個急了,趕着踢打;這賊,就走了。」楊姑娘道:「我的姐姐,你隨官人吩咐敎他唱罷了,又搶白他怎的?想必每常見姐姐們都全全兒的,今日只不見了李家姐姐,漢子的心怎麽不慘切個兒?」玉樓道:「好奶奶,這半日你還敎他唱!誰嗔他唱?俺這六姐姐,平昔曉的曲子裏滋味。那個誇死了的李大姐,比古人那個尚的不如他,又怎的兩個交的情厚,又怎麽說山盟海誓,你為我,我為你,無比賽的好!這個牢成的又不久慣,只顧㧱言語白他,和他整廝亂了這半日。」楊姑娘道:「我的姐姐,原來這等聰明!」月娘道:「他什麽曲兒不知道!但題起頭兒,就知尾兒。像我,若叫唱老婆和小優兒來,俺們只曉的唱出來就罷了。偏他又說那一段兒唱的不是了,那一句兒唱的差了,又那一節兒稍了。但是他爹說出來個曲兒,就和爹熱亂,兩個白搽白折的,必須搽惱了纔罷。俺們便不去管他。」孟玉樓在傍戲道:「姑奶奶,你不知,我三四胎兒只存了這個丫頭子。這丫頭子這般精靈兒古怪的,如今他大了,成了人兒,就不依我管教了。」金蓮便向他打了一下,笑道:「你又做我的娘起來了!我好又來打上輩。」玉樓道:「你看恁慣的少條兒失敎的,又來打上輩!」楊姑娘道:「姐姐,你今後讓他官人一句兒罷。常言一夜夫妻百夜恩。相隨百步也有個徘徊之意。一個熱突突人兒,指頭兒似的少了一個,如何不想不疼不題念的!」金蓮道:「怎的不想,也有個常時兒!一般都是你的老婆,做什麽擡一個滅一個?俺們都是劉湛兒鬼兒,不出材的!大姐在後邊,他也不知道。你還沒見哩,每日他従那裏喫了酒來,就先到他房裏,望着他影,深深唱喏,口裏恰似嚼蛆一般,供着個羹飯兒,對着擧筯兒只像活的一般兒讓他,不知什麽張致!又嗔俺們不替他戴孝,俺們便不說。他又不是婆婆,胡亂帶過断七罷了,只顧帶幾時?又與俺們亂了幾場。」楊姑娘道:「姐姐們見一半不見一半兒罷!」大妗子道:「好快,断七過了這一向,又早百日來。」楊姑娘問:「幾時是百日?」月娘道:「早哩,臘月二十六日。」王姑子道:「少不的念個經兒?」月娘道:「挨年近節,忙忙的,且念什麽經?他爹只怕過年念罷了。」

說着,只見小玉㧱上一道土荳泡茶來,每人一盞。須臾喫畢,月娘洗手,向爐中炷了香,聽薛姑子講說佛法。先念偈曰:

\begin{myquote}
「禪宗法敎豈非凡,佛祖流傳在世間。

落葉風飄着地易,等閒復上故枝難!
\end{myquote}

此四句詩,單説着這為僧的,戒行最難。言人生就如同鐵樹花開一般,落得容易,全枝復節甚難;墮業容易,成佛作祖難。卻說當初治平年間,浙江寜海軍錢塘門外南山淨慈孝光古刹,有兩個得道的真僧,一個喚作五戒禪師,一個喚作明悟禪師。如何謂之五戒?第一不殺生命,第二不偸財物,第三不染淫聲美色,第四不飲酒茹葷,第五不妄言綺語。如何謂之明悟?言其明心見性,覺悟我真。這五戒禪師,在家年方三十一歲,身不滿五尺,形容古怪;自幼明悟,眇其一目,俗姓金,禪宗佛教,如法了得。他與明悟是師兄師弟。一日,同來寺中,訪大行禪師。禪師觀五戒佛法曉得,留在寺中做個首座。不數年,大行圓覺,衆僧立他做了長老,每日打坐參禪。那第二個明悟,年二十九歲,生得頭圓耳大,面闊口方,身體長大,貌類羅漢,俗姓王。兩個如同一母所生,但遇說法,同升法座。

忽一日,冬盡春初時節,天道嚴寒陰雲作雪,下了兩日,雪霽天晴。這五戒禪師早晨坐在禪椅上,耳邊連連只聞得小兒啼哭,便叫一個身邊知心腹的清一道人:『你往山門前看有甚事,來報我知道。』這道人開了山門,見松樹下雪地上一塊破蓆,放着一個小孩兒。『這是什麽人家丢在此䖏?』向前看,是五六個月的女孩兒,破衣包裹,懷内片紙,寫着他生時八字。清一道:『救人一命,勝造七級浮屠。』連忙到方丈禀知長老,長老道:『善哉!難得你善心。即抱囘房中,好生喂養,救他性命,這是好事。』到了周歲,長老起了個名字,喚做『紅蓮』。日往月來,養在寺中,無人知覺。一向長老也忘了。不覺紅蓮長成十六歲。清一道人每日出鎖入鎖,如親生女一般。女子衣服鞋襪如沙彌打扮,且是生得清俊。無事在房做針線,只指望招尋個女婿,養老送終。一日,六月熱天,這五戒禪師忽想十數年前之事,逕來千佛閣後清一道人房中來。清一道:『長老希行,來此何幹?』五戒因問:『紅蓮女子在於何處?』清一不敢隱諱,請長老進房。長老一見,就差了念頭,邪心輒起,吩咐清一:『你今早送他到我房中,不可有悮。你若依我,後日擡擧你,切不可泄漏與人。』清一不敢不依,暗思今夜必壞了這女身。長老見他應得不爽利,喚入方丈,與了他十兩白金及度牒。清一只得收了銀子,至晚送紅蓮到方丈。長老遂破了他身,每日藏鎖他在牀後紙帳房内,把些飯食與他喫。

卻説他師弟明悟禪師在禪牀上入定囬來,已知五戒差了念頭,犯了色戒,淫姤了紅蓮女子,把多年德行一旦抛棄了。『我去勸醒他,再不可如此!』次日,寺門前荷蓮花開,明悟令行者採一朶白蓮花來,插在膽瓶内,令請五戒來賞蓮花,吟詩談笑。不一時,五戒至,兩個禪師坐下。明悟道:『師兄,我今日見此花甚盛,竟請吾兄賞玩,吟詩一首。』行者㧱茶喫了,預備文房四寶。五戒道:『將那荷根為題。』明悟道:『便將蓮花為題。』五戒捻起筆來,寫詩四句:

\begin{myquote}
『一枝菡萏瓣兒張,相伴蜀葵花正芳。

紅榴似火開如錦,不如翠蓋芰荷香。』
\end{myquote}

明悟道:『師兄有詩,小弟豈得無詩?』於是拈筆寫四句:

\begin{myquote}
『春來桃杏柳舒張,千花萬蕊鬦芬芳。

夏賞芰荷如燦錦,紅蓮爭似白蓮香!』
\end{myquote}

寫畢,呵呵大笑。五戒聽了此言,心中一悟,面有愧色。轉身辭囘方丈,命行者快燒湯。洗浴罷,換了一身新衣,取紙筆忙寫八句頌曰:

\begin{myquote}
『吾年四十七,萬法本歸一;

只為念頭差,今朝去得急。 

傳語悟和尚,何勞苦相逼!

幻身如閃電,依舊蒼天碧。』
\end{myquote}

寫畢,放在佛前,歸到禪牀上就坐化了。行者忙去報與明悟。明悟聽得大驚,走來佛前看見〈辭世頌〉,遂說:『你好却好了,只可惜差了這一着。你如今雖得個男身,長成不信佛法僧三寳,必然滅佛謗僧,後世墮落苦輪,不得歸依正道,深可痛哉!你道你去得,我趕你不着?』當下歸房,令行者燒湯洗浴,坐在禪牀上:『吾今趕五戒和尚去也,汝可將兩個龕子盛了,放三日,一時焚化。』說畢,亦圓寂坐化。衆僧皆驚,有如此異事?傳得四方知道:本寺連日坐化了兩僧。燒香禮拜,佈施者人山人海,擡去寺前焚化。這清一道人遂將紅蓮改嫁平人養老。不日後,五戒托生在西川眉州,與蘇老泉居士做兒子,名喚蘇軾,字子瞻,號東坡。明悟托生與本州姓謝名原字道清為子,名為端卿,後出家爲僧,取名佛印。他兩個還在一處作對,相交契厚。正是:

\begin{myquote}
自到川中數十年,曾在毘盧頂上眠。

參透趙州関捩子,好姻緣做惡姻緣。

桃紅柳緑還依舊,石邊流水響潺潺。

今朝指引菩提路,再休錯意戀紅蓮。」
\end{myquote}

薛姑子說罷,只見玉樓房中蘭香,㧱了兩方盒細巧素菜菓碟,茶食點心,收了香爐,擺在桌上,又是一壺茶,與衆人陪三個師父喫了。然後又㧱葷下飯來,打開一罈麻姑酒,衆人圍爐喫酒。月娘便與大妗子擲骰兒搶紅;金蓮便與李嬌兒猜枚。玉簫便傍邊斟酒,又替金蓮打桌底下轉子兒。須臾,把李嬌兒贏了數盃。玉樓道:「等我和你猜,你只顧贏他罷。」這玉樓却要金蓮露出手來,不許他褪在袖口邊,玉簫不許他近前。當下一連反贏了金蓮幾鍾酒,又敎郁大姐彈唱。月娘道:「你唱個〈鬧五更〉俺們聽。」郁大姐便調絃高聲唱〈玉交枝〉道:

\begin{myquote}
「彤雲密布,剪鵝毛雪花辭舞,朔風凛冽穿窗户。你心毒,奴更受苦。爹娘罵得奴心忒狠毒,你説來的話全不顧。把更兒従頭細數。」

{\markfont〈金字經〉}「夜迢迢孤另另,冷清清更靜初。不寄平安一紙書。腮邊流淚珠,不把佳期顧。一更裏無限的苦。」

{\markfont〈玉交枝〉}「一更纔至,冷清清撇奴在帳裏。翻來復去如何睡?二更裏淚珠垂。」
\end{myquote}

又:

\begin{myquote}
「二更難過,討一覺頻頻的睡着。今宵今宵夢兒裏來托,我思他他思我。去時節海棠花兒開了半朶,到如今樹葉兒皆零落。枉敎奴癡心等着。」

{\markfont〈金字經〉}「我癡心終日家等待你,何日是可?合少離多咱命薄、命薄,孤另另怎生奈何,好着敎難存坐,三更裏睡夢兒多。」

{\markfont〈玉交枝〉}「三更月上好難挨,今宵夜長。燒殘蠟燭銀臺上,淚珠流三兩行。紅綾的被兒閒了半牀。新挑的手帕兒在誰行放?瘦損了腰肢,腰肢沈郎。」

{\markfont〈金字經〉}「沈郎的腰肢瘦,每日家愁断了腸。盼望情人淚兩行、兩行,對菱花懶梳粧。瘦損了嬌模樣,四更裏偏夜長。」

{\markfont〈玉交枝〉}「四更如晝,枕邊想不覺的淚流:靈神廟裏曾發咒,剪青絲兩下裏收。説來的話兒不應口,到如今閃的我似章臺柳、章臺柳,教奴癡心等守。」

{\markfont〈金字經〉}「我癡心終日家等待你,何日是休?望盼情人空倚樓、倚樓,想情人一筆勾,不由把眉雙皺。五更裏淚珠流。」 

{\markfont〈玉交枝〉}「五更鷄唱,看看兒天色漸曉。放聲、欲待放聲又恐怕傍人笑,一會家心内焦。燒香告禱神前筊,負心的自有天知道,枉敎奴癡心等着。」

{\markfont〈金字經〉}「我癡心終日家等待你,何日是了?簷外叮噹鐵馬兒敲、鐵馬兒敲,攪的奴睡不着。一壁廂寒鴉呌,凄凄凉凉直到曉。」

{\markfont〈玉交枝〉}「曉來梳洗傍粧臺,懶上畫眉。房簷上喜鵲兒喳喳的,小梅香來報喜。報道是有情郎真個歸,奴好同入羅幃裏,向前來奴家問你!」 

{\markfont〈後庭花〉}「我問你個負心賊你盡知:一去了半年來怎生無個信息?我道你應擧求官去,誰想你戀煙花家貪酒盃。我為你受孤悽,你那裏偎紅倚翠!我為你病懨懨減了飲食,瘦伶仃消了玉體。挨清晨怕晚夕,一更裏聽天邊孤雁飛,二更裏想情人魂夢裏,五更裏醒來時不見你。」

{\markfont〈柳葉兒〉}「呀!空閒了鴛鴦錦被,寂寞了燕約鶯期。海神廟現放着傍州例,不由我心中氣。你盡知,負心的自有個天知。」

{\markfont〈尾聲〉}「流蘇錦帳同歡會,錦被裏鴛鴦成對,永遠團圓直到底。」
\end{myquote}

當下金蓮與玉樓猜枚,被玉樓贏了一二十鍾酒,坐不住,往前邊去了。到前邊呌了半日,角門纔開。只見秋菊揉眼,婦人駡道:「賊奴才,你睡來?」秋菊道:「我沒睡。」婦人道:「見睡起來,你哄我?你倒自在,就不説往後來接我接兒去。」因問:「你爹睡來?」秋菊道:「爹睡了這一日了。」婦人走到炕房裏,摟起裙子來就坐在炕上烤火。婦人要茶喫,秋菊連忙傾了一盞茶來。婦人道:「賊奴才,好乾淨手兒,你倒茶我喫!我不喫這陳茶,熬的怪泛湯氣。你叫春梅來,敎他另㧱小銚兒炖些好甜水茶兒,多着些茶葉,炖的苦艷艷我喫。」秋菊道:「他在那邊牀屋裏睡哩,等我呌他起來。」婦人道:「你休呌他,且敎他睡罷。」這秋菊不依,走到那邊屋裏,見春梅歪在西門慶脚頭睡得正好。被他搖推醒了,道:「娘來了,要喫茶,你還不起來哩。」這春梅噦他一口,駡道:「見鬼的奴才,娘來了罷了,平白唬人剌剌的!」一面起來,慢條斯禮撒腰拉袴,走來見婦人,只顧倚着炕兒揉眼。婦人反罵秋菊:「恁奴才,你睡的甜甜兒的,把你呌醒了。」因敎他:「你頭上汗巾子跳上去了,還不往下扯扯哩。」又問:「你耳朵上墜子,怎的只帶着一隻,一隻往那裏去了?」這春梅摸了摸,果然只有一隻金玲瓏墜子。便點燈往那邊牀上尋去,尋不見;良久,不想落在牀脚踏板上,拾起來。婦人問:「在那裏來?」春梅道:「都是他失驚打怪呌我起來,乞帳鈎子抓下來了,纔在踏板上拾起來。」婦人道:「我那等說着,他還只當呌起你來。」春梅道:「他說娘要喫茶來。」婦人道:「我要喫口茶兒,嫌他那手不乾淨。」這春梅連忙舀了一小銚子水,坐在火上,使他撾了些炭在火内,須臾就是茶湯。滌盞兒乾淨,濃濃的點上去遞與婦人。婦人問春梅:「你爹睡下多大囘了?」春梅道:「我打發睡了這一日了。問娘來,我說娘在後邊還未來哩。」

這婦人喫了茶,因問春梅:「我頭裏袖了幾個菓子和蜜餞,是玉簫與你姥姥喫的,交付這奴才接進來,你收了?」春梅道:「我沒見他,知道放在那裏!」這婦人一面呌秋菊問他:「菓子在那裏?」秋菊道:「有,我放在揀粧内哩。」走去取來。婦人數了一數,只是少了一個柑子。問他:「那裏去了?」秋菊道:「娘遞與㧱進來,就放在揀粧内。那個害饞癆爛了口喫他不成?」婦人道:「賊奴才,還漒嘴!你不偸,往那去了?我親手數了交與你的。賊奴才,你看着手拈搭的,零零落落只剩下這些兒,乾淨喫了一半,原來只孝順了你!」教春梅:「你與我把那奴才一邊臉上打與他十個嘴巴。」春梅道:「那臢臉彈子,倒没的齷齪了我這手!」婦人道:「你與我拉過他來。」春梅用雙手推顙到婦人跟前。婦人用手擰着他腮頰,罵道:「賊奴才,這個柑子是你偸喫了不是?你即實實説了,我就不打你。不然取馬鞭子來,我這一旋剝,就打個不數!我難道醉了?你偸喫了,一徑裏ろ混我!」因問春梅:「我醉不醉?」那春梅道:「娘清省白淨,那討酒來!娘信他,不是他喫了?娘不信,掏他袖子,怕不的還有柑子皮兒在袖子裏不定的。」婦人於是扯過他袖子來,用手掏他袖子。秋菊慌用手撇着,不敎掏。春梅一面拉起手來,果然掏出些柑子皮兒來。被婦人儘力臉上擰了兩把,打了兩個嘴巴,便罵道:「賊奴才痞,不長俊奴才!你諸般兒不會,像這說舌偸嘴喫偏會!剛纔掏出皮來,喫了,真贜實犯㧱住,你還賴那個?我如今要打你,——你爹睡在這裏,我茶前酒後:我且不打你,到明日清淨白省,和你算帳!」春梅道:「娘到明日,休要與他輕輕湯湯的。好生旋剝了,教一個人把他實辣辣打與他幾十板子,敎他忍疼,他也懼怕些。甚麽鬦猴兒似湯那幾棍兒,他纔不放心上!」那秋菊被婦人擰的臉脹腫的,谷都着嘴,往厨下去了。婦人把那一個柑子平擘兩半,又㧱了個蘋婆、石榴,遞與春梅,說道:「這個與你喫。把那個留與姥姥喫。」這春梅也不瞧,接過來似有如無掠在抽屜内。婦人把蜜餞也要分開,春梅道:「娘不要分,我懶待喫這甜行貨子,留與姥姥喫罷。」以此婦人不分,都留下了不題。

婦人走到桶子上小解了,教春梅掇進坐桶來,澡了牝。又問春梅:「這咱天有多少時分?」春梅道:「月兒大倒西,也有三更天氣。」婦人摘了頭面,走來那邊牀房裏,見桌上銀燈已殘,従新剔了剔,向牀上看,西門慶正打鼾睡。於是解鬆羅帶,卸褪湘裙,坐換睡鞋,脫了褌褲,上牀鑽在被窝裏與西門慶並枕而臥。睡下不多時,向他腰間摸他那話,弄了一囘,白不起。原來西門慶與春梅纔行房不久,那話綿軟,急切捏弄不起來。這婦人酒在腹中,慾情如火,蹲身在被底,把那話用口吮咂,挑弄蛙口,吞裹龜頭,只顧往來不絶。西門慶猛然醒了,見他在被窝裏,便道:「怪小淫婦兒,如何這咱纔來?」婦人道:「俺們在後邊喫酒,孟三兒又安排了兩大方盒酒菜兒。郁大姐唱着,俺們陪大妗子、楊姑娘,猜枚擲骰兒,又頑了這一日,被我把李嬌兒先贏醉了,落後孟三兒和我兩個五子三猜,俺兩個倒輸了好幾鍾酒。你倒是便益,睡起一覺兒來好熬我,你看我依你不依!」西門慶道:「你整治那帶子了?」婦人道:「在褥子底下不是?」一面探手取出來與西門慶看了,扎在麈柄根下,繫在腰間,拴的緊緊的。又問:「你喫了不曾?」西門慶道:「我喫了。」須臾,那話乞婦人一壁廂弄起來,只見奢稜跳腦,挺身直舒,比尋常更舒——七寸有餘。婦人趴在身上,龜頭昂大,兩手げ着牝户往裏放,須臾突入牝中。婦人兩手摟定西門慶脖項,令西門慶亦扳抱其腰,在上只顧揉搓,那話漸没至根。婦人呌西門慶:「達達,你取我的と腰子,墊在你腰底下。」這西門慶便向牀頭取過他大紅綾抹胸兒,四摺疊起,墊着腰。這婦人在他身上馬伏着,那消幾揉,那話盡入。婦人道:「達達,你把手摸摸,都全放進去了,撑的裏頭滿滿兒的,你自在不自在?都揉進去。」西門慶用手摸摸,見盡沒至根,間不容髮,止剩二卵在外,心中覺翕翕然暢美不可言。婦人道:「好急的慌,只是觸冷,咱不得㧱燈兒照着幹。趕不上夏天好,這冬月間,只是冷的慌。」因問西門慶說道:「這帶子比那銀托子,試好不好?強如格的陰門生疼的。這個顯的該多大,又長出許多來,你不信摸摸我小肚子,七八頂到奴心。」又道:「你摟着我,等我今日一發在你身上睡一覺。」西門慶道:「我的兒,你睡,達達摟着。」那婦人把舌頭放在他口裏含着,一面朦朧星眼,欵抱香肩。睡不多時,怎禁那慾火燒身,芳心撩亂,於是兩手按着他肩膊,一擧一坐,抽徹至首,復送至根,呌:「親心肝,罷了,六兒的死了。」往來抽提,又三百囬,比及精洩,婦人口中只呌:「我的親達達,把腰扱緊着。」一面把奶頭敎西門慶咂,不覺一陣昏迷,淫水溢下。停不多回,婦人兩個抱摟在一䖏,婦人心頭小鹿突突的跳,登時四肢困軟,香雲撩亂,於是拽出來,猶剛勁如故。婦人用帕搽之,便道:「我的達達,你不過卻怎麽的?」西門慶道:「等睡起一覺來再耍罷。」婦人道:「我也挨不的,身子已軟癱熱化的。」當下雲收雨散,兩個並肩交股,枕籍於牀上,不覺東方之旣白。正是:等閒試把銀釭照,一對天生連理人。

畢竟未知後來何如,且聽下囘分解。

