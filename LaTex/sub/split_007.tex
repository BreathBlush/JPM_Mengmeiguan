\chapter*{抄餘綴語}
\addcontentsline{toc}{chapter}{抄餘綴語}
\markboth{抄餘綴語}{抄餘綴語}

余於庚午至甲戌間,小楷手抄紅樓夢與水滸傳,欲再用五載抄閱另兩部名著,而意取金瓶梅及西遊記也。彼四書者,就語言文學而言,確是中國古典文學之四大名著。金瓶梅一書,述日常琐事,繪聲繪色,鋪陳周密,語言明快,雅俗兼俱,而蓄意蓋深,實開小説之新面。因覔求善本,請教於大方,承香港梅節先生慷慨提供聚十數年研究结晶之夢梅館定本金瓶梅詞話,並允諾將由夢梅館影印出版。遂於乙亥閏八月開始抄寫,至戊寅三月谷雨日竣工,又用半年時間反復校對,始得完善。縱觀洋洋八十萬言之抄本,其中字跡未免參差不齊,免強與梅先生之校本相得益彰。適今付梓之時,窃以能為金瓶梅之傳播和研究出些微力而欣然自得云爾。

\begin{quotation}\begin{flushright}時戊寅八月中龝日金城陳少卿漫識於鴻蒙居\end{flushright}\end{quotation}


\begin{quotation}\small\color{gray}

\begin{flushright}
(庚子歲戊寅月湘人陳善軍製書於

米國山景城{\quad}{\LaTeX}製版)
\end{flushright}
\end{quotation}

