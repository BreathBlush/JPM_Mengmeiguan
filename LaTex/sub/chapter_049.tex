\includepdf[pages={97,98},fitpaper=false]{tst.pdf}
\chapter*{第四十九囬 \\西門慶迎請宋巡按 永福寺餞行遇胡僧}
\addcontentsline{toc}{chapter}{第四十九囬 西門慶迎請宋巡按 永福寺餞行遇胡僧}
\markboth{{\titlename}卷之五}{第四十九囬 西門慶迎請宋巡按 永福寺餞行遇胡僧}


\begin{myquote}
寬性寬懷過幾年,人死人生在眼前。

隨高隨下隨緣過,或長或短莫埋怨;

自有自無休歎息,家貧家富總由天。

平生衣祿隨緣度,一日清閒一日僊。
\end{myquote}

話説夏壽到家囬覆了話,夏提刑隨即就來拜謝西門慶,説道:「長官活命之恩!不是託賴長官餘光,這等大力量,如何了得?」西門慶笑道:「長官放心,料着你我沒曾過為,隨他説去便了。老爺那裏自有個明見。」一面在廳上放桌兒留飯,談笑至晚,方纔作辭囬家。到次日,依舊入衙門裏理事,不在話下。

卻表巡按曾公,見本上去不行,就知道二官打點了,心中忿怒。因蔡太師所陳七事,内多乖方舛訛,皆損下益上之事,即赴京見朝覆命,上了一道表章,極言天下之財,貴於通流,取民膏以聚京師,恐非太平之治。民間結糶俵糴之法不可行,當十大錢不可用,鹽鈔法不可屢更:「臣聞民力殫矣,誰與守邦?」蔡京大怒,奏上徽宗天子,説他大肆猖言,阻撓國事。即時將曾公付吏部考察,黜為陝西慶州知州。陝西巡按御史宋聖寵,是學士蔡攸之婦兄也。太師陰令聖寵劾其私事,逮其家人,煆煉成獄,將孝序除名,竄于嶺表,以報其仇。此係後事,表過不題。

再説西門慶在家,一面使韓道國與喬大戶外甥崔本,拏倉鈔早徃高陽關戶部韓爺那裏趕着掛號。㽞下來保,家中定下菓品,預備大桌面酒席,打聽祭御史舡到。

一日,來保打聽得他與巡按宋御史舡一同京中起身,都行至東昌府地方,使人先來家通報。這裏西門慶就會夏提刑起身。知府州縣及各衛有司官員,又早預備祇應人馬,鐵桶相似。來保従東昌府舡上就先見了蔡御史,送了下程。然後西門慶與夏提刑出郊五十里迎接。到新河口,地名百家村,先到蔡御史舡上拜見了,備言邀請宋公之事。蔡御史道:「我知道,一定同他到府。」那時東平胡知府及合屬州縣方面有司、軍衛官員、吏典生員、僧道陰陽,都具連名手本,伺候迎接。帥府周守備、荆都監、張團練,都領人馬披執跟隨,清蹕傳道,鷄犬皆隱跡。鼓吹進東平府察院,各䖏官員都見畢,呈遞了文書。安歇一夜。到次日,只見門吏來報:「巡鹽蔡爺來拜。」宋御史急令撤去公案,連忙整冠出迎。兩個叙畢禮數,分賓主坐下。少頃,獻茶已畢。宋御史便問:「年兄事期,幾時方行?」蔡御史道:「學生還待一二日。」因告説:「清河縣有一相識西門千兵,乃本處巨族。為人清愼,富而好禮。亦是蔡老先生門下,與學生有一面之交。蒙他遠接,學生正要到他府上拜他拜。」宋御史問道:「是那個西門千兵?」蔡御史道:「他如今現是本䖏提刑千戶,昨日已參見過年兄了。」宋御史令左右取遞的手本來,看見西門慶與夏提刑名字,説道:「此莫非與翟雲峯有親者?」蔡御史道:「就是他。如今現在外面伺候,要央學生奉陪年兄到他家一飯。未審年兄尊意若何?」宋御史道:「學生初到此處,不好去得。」蔡御史道:「年兄怕怎的?既是雲峯分上,你我走走何害?」於是吩咐看轎,就一同起行。一面傳將出來。西門慶知了此消息,與來保賁四騎快馬先奔來家,預備酒席。門首搭照山綵棚,兩院楽人奏樂,叫海鹽戲並雜耍承應。

原來宋御史將各項伺候人馬,都令散了,只用幾隊藍旗清道,官吏跟隨,與蔡御史坐兩頂大轎,打着雙簷傘,同徃西門慶家來。當時哄動了東平府,擡起了清河縣,都説:「巡按老爺也認的西門大官人,來他家喫酒來了。」慌的周守備荆都監張團練各領本哨人馬,把住左右街口伺候。西門慶青衣冠帶,遠遠迎接,兩邊鼓楽吹打。到大門首,下了轎,進去。宋御史與蔡御史都穿着大紅獬豸繡服,舃紗皂履,鶴頂紅帶,従人執着兩把大扇。只見五間廳上,湘簾高捲,錦屏羅列。正面擺兩張喫看桌席,高頂方糖,定勝簇盤,十分齊整。二官揖讓進廳,與西門慶敍禮。蔡御史令家人具贄見之禮,兩端湖紬,一部文集,四袋芽茶,一面端溪硯。宋御史只投了個宛紅單拜帖,上書:「侍生宋喬年拜。」向西門慶道:「久聞芳譽,學生初臨此地,尚未盡情,不當取擾。若不是蔡年兄見邀,同來進拜,何以幸接尊顔!」慌的西門慶倒身下拜,説道:「僕乃一介武官,屬於按臨之下。今日幸蒙清顧,蓬蓽生光。」於是鞠恭展拜,禮容甚謙。宋御史亦答禮相還,敍了禮數。當下蔡御史讓宋御史居左,他自在右。西門慶垂首相陪。茶湯獻罷,堦下蕭韶盈耳,鼓楽喧闐,動起楽來。西門慶遞酒安席已畢,下邊呈獻割道。説不盡餚列珍羞,湯陳桃浪,酒泛金波。端的歌舞聲容,食前方丈。西門慶知道手下跟従人多,堦下兩位轎上跟従人,每位五十瓶酒,五百點心,一百斤熟肉,都領下去。家人吏書門子人等,另在廂房中管待,不必用説。

當日西門慶這席酒,也費夠千兩金銀。那宋御史又係江西南昌人,為人浮躁。只坐了沒多大囬,聽了一摺戲文就起來。慌的西門慶再三固留。蔡御史在傍便説:「年兄無事,再稍坐一時,何遽回之太速耶?」宋御史道:「年兄還坐坐,學生還欲到察院中䖏分些公事。」西門慶早令手下把兩張桌席,連金銀器,已都裝在食盒内,共有二十擡,叫下人夫伺候。宋御史的一張大桌席,兩壜酒,兩牽羊,兩對金絲花,兩疋緞紅,一副金臺盤,兩把銀執壺,十個銀酒盃,兩個銀折盃,一雙牙筯。蔡御史的也是一般的。都遞上揭帖。宋御史再三辭道:「這個我學生怎麽敢領?」因看着蔡御史。蔡御史道:「年兄貴治所臨,自然之道。我學生豈敢當之?」西門慶道:「些湏微儀,不過乎侑觴而已,何為見外?」比及二官推讓之次,而桌席已擡送出門矣。宋御史不得已,方令左右收了揭帖,向西門慶致謝,説道:「今日初來識荆,既擾盛席,又承厚貺,何以克當?餘容圖報,不忘也!」因向蔡御史道:「年兄還坐坐,學生告别。」於是作辭起身。西門慶還要遠送,宋御史不肯,急令請回,擧手上轎而去。

西門慶囬來,陪侍蔡御史,解去冠帶,請去捲棚内後坐。因吩咐把楽人都打發散去,只㽞下戲子。西門慶令左右重新安放桌席,擺設珍羞菓品上來,二人飲酒。蔡御史道:「今日陪我這宋年兄坐便僭了。又叨管待盛席酒器,何以克當!」西門慶笑道:「微物惶恐,表意而已。」因問道:「宋公祖尊號?」蔡御史道:「號松原。松樹之松,原泉之原。」又説起:「頭裏他再三不來。被我學生因稱道四泉盛德,與老先生那邊相熟,他纔來了。他也知府上與雲峯有親。」西門慶道:「想必翟親家有一言於彼。我觀宋公,為人有些蹺蹊。」蔡御史道:「他雖故是江西人,倒也沒甚蹺蹊處。只是今日初會,怎不做些模樣?」説畢笑了。西門慶便道:「今日晚了,老先生不囬舡上去罷了。」蔡御史道:「我明早就要開舡長行。」。西門慶道:「請不棄在舍留宿一宵,明日學生長亭送餞。」蔡御史道:「過蒙愛厚。」因吩咐手下人:「都囬門外去罷,明日來接。」衆人都應諾去了,只留下兩個家人伺候。

西門慶見手下人都去了,走下席來,叫玳安兒附耳低言,如此這般吩咐:「即去院中,坐名叫了董嬌兒韓金釧兒兩個,打後門裏用轎子擡了來,休教一人知道。」那玳安一面應諾去了。西門慶復上席陪蔡御史喫酒。海鹽子弟在傍歌唱。西門慶因問:「老先生到家多少時就來了?令堂老夫人起居康健麽?」蔡御史道:「老母倒也安。學生在家,不覺荏苒半載。囬來見朝,不想被曹禾論劾,將學生敝同年一十四人之在史館者,一時皆黜授外職。學生便選在西臺,新點兩淮巡鹽。宋年兄便在貴䖏巡按,他也是蔡老先生門下。」西門慶問道:「如今安老先生在那裏?」蔡御史道:「安鳳山他已陞了工部主事,徃荆州催趲皇木去了,也待好來也。」説畢,西門慶叫海鹽子弟上來遞酒。蔡御史吩咐:「你唱個〔漁家傲〕我聽。」子弟排手在傍唱道:

\begin{myquote}
「别後杳無書,不疼不痛病難除。恨凄凄,旅館有誰相知,魚沉不見鴈傳書。三山美人知何處?眠思夢想,此情為誰?懨懨憔瘦,一似風中柳絮。知他幾時再得重相會!」

{\markfont〔皂羅袍〕}「滿目黄花初綻,怪淵明怎不囬還。敎人盼得眼睛穿,寃家怎不行方便。従伊别後,相思病纏;昏昏如醉,汪汪淚漣。知他幾時再得重相見!」

{\markfont〔前腔〕}「我愛他桃花為面,笋生成十指纖纖。我愛他春山淡淡柳拖煙,我愛他清俊一雙秋波眼。烏鴉堆鬢,青絲翠綰。玳鉤月釣,丹霞襯臉。敎人想得肝腸断。」

{\markfont〔前腔〕}「戍鼓鼕鼕初轉,聽樓頭畫角聲殘。搥牀搗枕數千番,長吁短歎千千遍。精神撩亂,語言倒顛;忘餐廢寢,和衣淚漣:終朝懞憧昏沉倦。」

{\markfont〔前腔〕}「我為你終朝思念,在那裏耍笑貪歡?忽然想起意懸懸,一番提起一番怨。恩深如海,情重似山;佳期非偶,離別最難。常言道藕断絲不断!」
\end{myquote}

正唱著,只見玳安走來請西門慶下邊説話。玳安道:「叫了董嬌兒、韓金釧兒,打後門來了,在娘房裏坐着哩。」西門慶道:「你吩咐把轎子擡過一邊纔好。」玳安道:「擡過一邊了。」這西門慶走至上房,兩個唱的向前磕了頭。西門慶道:「今日請你兩個來,晚夕在山子下服侍你蔡老爹。他如今現任巡鹽御史,你不可怠慢了他。用心扶持他,我另酬答你兩個。」那韓金釧兒笑道:「爹不消吩咐,俺們知道。」西門慶因戲道:「他南人的營生,好的是南風。你們休要扭手扭脚的。」董嬌兒道:「娘在這裏聽着,爹你老人家羊角蔥靠南牆——越發老辣已定了。王府門首磕了頭——俺們不喫這井裏水了?」這西門慶笑的徃前邊來。走到儀門首,只見來保和陳經濟拏着揭帖走來,與西門慶看。説道:「剛纔喬親家爹説,趂着蔡老爹這回閒,爹倒把這件事對蔡老爹説了罷,只怕明日起身忙了。教姐夫寫了俺兩個名字在此。」西門慶道:「你跟了來。」那來保跟到捲棚槅子外邊跪着。西門慶飲酒中間,因提起:「有一事在此,不敢干凟。」蔡御史道:「四泉有甚事,只顧吩咐,學生無不領命。」西門慶道:「去歲因舍親那邊在邊上納過些糧草,坐派了有些鹽引,正派在貴治揚州支鹽。只是望乞到那裏青目青目,早些支放,就是愛厚。」因把揭帖遞上去。蔡御史看了,上面寫着:「商人來保崔本舊派淮鹽三萬引,乞到日早掣。」蔡御史看了笑道:「這個甚麽打緊?」一面把來保叫至近前跪下,吩咐:「與你蔡爺磕頭。」蔡御史道:「我到揚州,你等逕來察院見我。我比別的商人早掣取你鹽一個月。」西門慶道:「老先生下顧,早放十日就夠了。」蔡御史把原帖就袖在袖内,一面書童傍邊斟上酒。子弟又唱〔下山虎〕:

\begin{myquote}
「中秋將至,漸覺心酸。只見穿窻月,不見故人還。聽叮噹砧聲滿耳,嘹嚦嚦北雁南還,怎不教人心中慘然?料想相思,断送少年。黄昏後,更漏殘,把銀燈剔盡方眠。」

{\markfont〔前腔〕}「當初擕手,月下並肩。説下山盟海誓,對天禱告:若有個負義忘恩,早歸九泉。一向如何音信遠,空教我卜金錢,廢寢忘餐,有誰見憐?黄昏後,更漏殘,把銀燈剔盡方眠。」

{\markfont〔尾聲〕}「蒼天若肯行方便,早遣情人到枕邊,免使書生獨自眠!」
\end{myquote}

唱畢,當下掌燈時分,蔡御史便説:「深擾一日,酒告止了罷。」因起身出席,左右便欲掌燈。西門慶道:「且休掌燭。請老先生後邊更衣。」於是従花園裏遊翫了一囬,讓至翡翠軒。那裏又早湘簾低簇,銀燭熒煌,設下酒席完備。海鹽戯子西門慶已命手下管待酒飯,與了二兩賞錢,打發去了。書童把捲棚内家活收了,關上角門。只見兩個唱的,盛粧打扮,立於堦下,向前花枝招颭磕頭。但見:

\begin{myquote}
綽約容顔金縷衣,香塵不動下階墀;

時來水濺羅裙濕,好似巫山行雨歸。
\end{myquote}

蔡御史看見,欲進不能,欲退不可,便説道:「四泉,你如何這等愛厚,恐使不得!」西門慶笑道:「與昔日東山之遊又何別乎?」蔡御史道:「恐我不如安石之才,而君有王右軍之高致矣。」於是月下與二妓擕手,不啻恍若劉阮之入天台。因進入軒内,見文物依然。因索紙筆,要留題。西門慶即令書童,連忙將端溪硯硏的墨濃,拂下錦箋。這蔡御史終是狀元之才,撚筆在手,文不加點,字走龍蛇,燈下一揮而就,作詩一首。詩曰:

\begin{myquote}
「不到君家半載餘,軒中文物尚依然:

雨過書童開薬圃,風囬僊子步花臺。

飲將醉䖏鍾何急,詩到成時漏更催。

此去又添新悵望,不知何日是重來?」
\end{myquote}

寫畢,交書童黏於壁上,以為後日之遺焉。因問二妓:「你等叫甚名字?」一個道:「小的姓董,名喚嬌兒,他叫韓金釧兒。」蔡御史又道:「你二人有號沒有?」董嬌兒道:「小的無名娼妓,那討號來!」蔡御史道:「你等休要太謙。」問至再三,韓金釧兒方説:「小的號玉卿。」董嬌兒道:「小的賤號薇僊。」蔡御史一聞「薇僊」二字,心中甚喜,遂留意在懷。令書童取棋桌來,擺下棋子。蔡御史與董嬌兒兩個着棋,西門慶陪侍。韓金釧兒把金樽在旁邊遞酒。書童拍手歌唱〔玉芙蓉〕。唱道:

\begin{myquote}
「東風柳絮飄,玉砌蘭芽小,這春光艷冶巧鬬難描。墻頭紅粉佳人笑,蹴罷鞦韆香汗消。尋芳興,不辭路遙。我只見酒旗搖曳杏花梢。」
\end{myquote}

唱畢,蔡御史贏了董嬌兒一盤棋。董嬌兒喫過,囬奉蔡御史。韓金釧兒這裏遞與西門慶,陪飲一盃。書童又唱道:

\begin{myquote}
{\markfont〔前腔〕}
「風吹蕉尾翻,雨灑荷珠亂。見佳人,盤鬢如蟬。湘紈半掩芙蓉面,綵袖輕飄賽小蠻。秋波臉,兩情牽好難。引的人意遲寂寞淚闌干。」
\end{myquote}

飲了酒,兩人又下。董嬌兒贏了,連忙遞酒一盃與蔡御史。西門慶在傍又陪飲一盃。書童又唱:

\begin{myquote}
{\markfont〔前腔〕}
「黄花遍地開,百草皆凋敗,小蛩吟喞喞空堦。牛郎夜夜依然在,織女緣何不見來?懨懨害,糊突夢怎猜?我為他涙滴濕表記鳳頭鞋。」
\end{myquote}

唱畢,蔡御史道:「四泉,夜深了,不勝酒力了。」於是走出外邊來,站立在於花下。那時正是四月半頭時分,月色纔上。西門慶道:「老先生,天色還早哩。還有韓金釧未曾賞他一盃酒。」蔡御史道:「正是,你喚他來,我就此花下立飲一盃。」於是韓金釧兒拏大金桃盃滿斟一盃,用纖手捧遞上去,董嬌兒在傍捧菓。書童拍手又唱第四個:

\begin{myquote}
{\markfont〔前腔〕}
「梨花散亂飛,不見遊蜂翅。小窻前鵲踏枯枝。愁聞冒雪尋梅至,忽聽銅壺更漏遲。傷心事,把離情自思。我為他寫情書閣不住筆尖兒。」
\end{myquote}

蔡御史喫過,斟上一盃賞與韓金釧兒,因告辭道:「四泉,今日酒太多了,令盛价收過去罷。」於是與西門慶握手相語,説道:「賢公盛情盛德,此心懸懸。若非斯文骨肉,何以至此?向日所貸,學生耿耿在心,在京已與雲峯表過。倘我後日有一步寸進,断不敢有辜盛德!」西門慶道:「老先生何出此言,倒不消介意!」那韓金釧兒見他一手拉着董嬌兒,知局就徃後邊去了。到了上房裏,月娘便問:「你怎的不陪他睡,來了?」韓金釧笑道:「他㽞下董姐兒了。我不來,只在那裏做甚麽?」良久,西門慶亦告了安置進來。叫了來興兒,吩咐;「明日早五更,打發食盒酒米,點心嘎飯。呌了廚役,跟了徃門外永福寺去,那裏與你蔡老爹送行。兩個小優兒答應,休要誤了。」來興兒道:「家裏二娘上壽,沒人看來。」西門慶道:「留下棋童兒買東西,叫廚子後邊大竃上做罷。」

不一時,書童玳安收下家活來。又討了一壺好茶,徃花園裏去,與蔡老爹漱口。翡翠軒書房牀上,舖陳衾枕,俱各完備。蔡御史見董嬌兒手中拏着一把湘妃竹泥金面扇兒,上面水墨畫着一種湘蘭,平溪流水。董嬌兒道:「敢煩老爹賞我一首詩在上面。」蔡御史道:「無可為題,就指着你這薇僊號。」於是燈下乘興,撚起筆來,寫了四句在上:

\begin{myquote}
「小院閒庭寂不譁,一池月上浸窻紗。

邂逅相逢天未晚,紫薇郎對紫薇花。」
\end{myquote}

寫畢,那董嬌兒連忙拜謝了,兩個收拾上牀就寢。書童玳安與他家人在明間裏睡,一宿晚景不題。

次日早晨,蔡御史與了董嬌兒一兩銀子,用紅紙大包封着。到於後邊,拏與西門慶瞧。西門慶笑説道:「文職的營生,他那裏有大錢與你?這個就是上上籤了。」因教月娘每人又與了他五錢,早従後門打發他去了。書童舀洗面水,打發他梳洗穿衣。西門慶出來,在廳上陪他喫了粥。手下又早伺候轎馬來接,與西門慶作辭,謝了又謝。西門慶又道:「學生日昨所言之事,老先生到彼處,學生這裏書去,千萬留神一二,足叨不淺。」蔡御史道:「休説賢公華札下臨,只盛价有片紙到,學生無不奉行。」説畢,二人同上馬,左右跟隨。出城外,到於永福寺,借長老方丈擺酒餞行。來興兒與廚役早已安排桌席停當。李銘吳惠兩個小優彈唱。數盃之後,坐不移時,蔡御史起身。夫馬坐轎,在於山門外伺候。臨行,西門慶説起苗青之事:「乃學生相知,因詿誤在舊大巡曾公案下,行牌徃揚州案候捉他。此事情已問結了。倘見宋公,望乞借重一言,彼此感激。」蔡御史道:「這個不妨。我見宋年兄説,設使就提來,放了他去就是了。」西門慶又作揖謝了。——看官聽説:後來宋御史徃濟南去,河道中又與蔡御史會在那舡上,公人揚州提了苗青來,蔡御史説道:「此係曾公手裏案外的,你管他怎的?」遂放囬去了。倒下詳去東平府,還只把兩個舡家決不待時,安童便放了。正是:人事如此如此,天理未然未然。有詩單表人情之有虧欠處。詩曰:

\begin{myquote}
公道人情兩是非,人情公道最難為。

若依公道人情失,順了人情公道虧。
\end{myquote}

胡知府已受了西門慶夏提刑囑託,無不做分上。要説此係後事。當日西門慶要送至舡上,蔡御史不肯,説道:「賢公不消遠送,只此告別。」西門慶道:「萬惟保重,容差小价問安。」説畢,蔡御史上轎而去。

西門慶囬到方丈坐下,長老走來遞茶,頭戴僧伽帽,身披袈裟,小沙彌拏着茶託,遞茶罷,合掌道了問訊。西門慶答禮相還,見他鬚眉皎白,便問:「長老多大年紀?」長老道:「小僧七十有五。」西門慶道:「倒還這等康健!」因問:「法號稱呼甚麽?」長老道:「小僧法名道堅。」「有幾位徒弟?」長老道:「止有兩個小徒。本寺也有三十餘僧行。」西門慶道:「你這寺院倒也寬大,只是欠修整。」長老道:「不瞞老爹説,這座寺原是周秀老爹蓋造,常住裏沒錢糧修理,丢得壞了。」西門慶道:「原來就是你守備府周爺的香火院!我見他家莊子不遠,不打緊處,你禀了你周爺,寫個緣簿,一般別䖏也再化着,來我那裏,我也資助你些布施。」道堅連忙合掌問訊謝了。西門慶吩咐玳安兒,書袋内取一兩銀子謝長老:「今日打攪長老這裏!」道堅道:「小僧不知老爺來,不曾預備齋供。」西門慶道:「我要徃後邊更更衣去。」道堅連忙叫小沙彌開便門。

西門慶更了衣,因見方丈後面五間大禪堂,有許多雲遊和尚,在那裏敲着木魚念經。西門慶不因不由,信步走入裏面觀看。見一個和尚,形骨古怪,相貌搊搜:生的豹頭凹眼,色若紫肝;戴了鷄蠟箍兒,穿一領肉紅直裰;頦下髭鬚亂拃,頭上有一溜光簷。就是個形容古怪眞羅漢,未除火性獨眼龍。在禪牀上旋定過去了,垂着頭,把脖子縮到腔子裏,鼻口中流下玉筯來。西門慶口中不言,心内暗道:「此僧必然是個有手段的高僧;不然,如何有此異相?等我叫醒他,問他個端的。」於是揚聲叫那位僧人:「你是那裏人氏,何處高僧,雲遊到此?」叫了頭一聲,不答應;第二聲,也不言語;第三聲,只見這個僧人在禪牀上把身子打了個挺,伸了伸腰,睜開一隻眼,跳將起來,向西門慶點了點頭兒,粗聲應道:「你問我怎的?貧僧行不更名,坐不改姓,乃西域天竺國密松林齊腰峯寒庭寺下來的胡僧,雲遊至此,施薬濟人。官人,你叫我有甚話説?」西門慶道:「你旣是施薬濟人,我問你求些滋補的薬兒,你有也沒有?」胡僧道:「我有!我有!」又道:「我如今請你到家,你去不去?」胡僧道:「我去!我去!」西門慶道:「你説去,即此就行。」那胡僧直竪起身來,向牀頭取過他的鐵柱杖來拄着,背上他的皮褡褳,褡褳内盛着兩個薬葫蘆兒,下的禪堂,就徃外走。西門慶吩咐玳安:「叫了兩個驢子,同師父先徃家去等着,我就來。」胡僧道:「官人不消如此。你騎馬只顧先行,貧僧也不騎頭口,管情比你先到。」西門慶道:「一定是個有手段的高僧,不然如何開這等朗言?」恐怕他走了,吩咐玳安好歹跟着他同行。於是作辭長老上馬,僕従跟隨,逕直進城來家。

那日四月十七日,不想是王六兒生日,家中又是李嬌兒上壽,有堂客喫酒。後晌時分,只見王六兒家沒人使,使了他兄弟王經來請西門慶,吩咐他宅門首只尋玳安兒説話。不見玳安在門首,只顧立着。立了約一個時辰,正値月娘與李嬌兒送院裏李媽媽出來上轎,看見一個十五六歲扎包髻兒小廝,問:「是那裏的?」那小廝三不知走到跟前,與月娘磕了個頭,説道:「我是韓家,尋安哥説話。」月娘問:「那安哥?」平安在傍邊,恐怕他知道是王六兒那裏來的,恐怕他説岔了話,向前把他拉過一邊,對月娘説:「他是韓夥計家使了來尋玳安兒,問韓夥計幾時來。」以此哄過,月娘不言語,囬後邊去了。

不一時,玳安與胡僧先到門首,走的兩腿皆酸,渾身是汗,抱怨的了不的。那胡僧體貌従容,氣也不喘。平安把王六兒那邊使了王經來請爹,尋他説話一節,對玳安兒説了:「不想大娘正送院裏李奶奶出來,門首上轎,看見他冒冒勢勢走到跟前,與大娘磕頭。大娘問他,説『我是韓家的』,早是我在傍邊,拉過一邊。落後大娘問我,我説是韓夥計家的,使他來問他韓夥計幾時來,大娘纔不言語了。早是沒曾禡覺出來。等住囬大娘若問你,也是這般説。」那玳安走的睜睜的,只顧搧扇子:「今日造化低也怎的,平白爹叫我跟了這賊禿囚來。好近道兒,従門外寺裏直走到家,路上通沒歇脚兒,走的我上氣兒接不着下氣兒!爹教僱驢子與他騎,他又不騎。他便走着沒事沒事的,難為我這兩條腿了!把鞋底子也磨透了,脚也踏破了,攘氣的營生!」平安道:「爹請他來家做甚麽?」玳安道:「誰知道?他説問他討甚麽薬哩!」

正説著,只聞喝道之聲。西門慶到家,看見胡僧在門首,説道:「吾師眞乃人中神也,果然先到!」一面讓至裏面大廳上坐。西門慶叫書童接了衣裳,換了小帽,陪他坐的。那胡僧睜眼觀見廳堂高遠,院宇深沉,門上掛的是龜背紋、蝦鬚織抹綠珠簾,地下舖獅子滚綉毬絨毛線毯,正當中放一張蜻蜓腿螳螂肚肥皂色起楞的桌子,桌子上安着縧環樣須彌座大理石屏風,週圍擺的都是泥鰍頭楠木靶腫觔的校椅,兩壁掛的畫,都是紫竹桿兒綾邊瑪瑙軸頭。正是:鼉皮畫鼓振庭堂,烏木春擡盛酒器。胡僧看畢,西門慶問道:「吾師用酒不用?」胡僧道:「貧僧酒肉齊行。」西門慶一面吩咐小廝:「後邊不消看素饌,拏酒飯來。」

那時正是李嬌兒生日,廚下餚饌下飯都有。安放桌兒,只顧拏上來。先綽邊兒放了四碟菓子,四碟小菜,又是四碟案酒:一碟頭魚,一碟糟鴨,一碟烏皮鷄,一碟舞鱸公。又拏上四樣下飯來:一碟羊角蔥𤆑炒的核桃肉,一碟細切的ほぽ樣子肉,一碟肥肥的羊貫腸,一碟光溜溜的滑鰍。次又拏了一道湯飯出來,一個碗内兩個肉圓子,夾着一條花筯滚子肉,名喚「一龍戯二珠湯」;一大盤裂破頭高裝肉包子。西門慶讓胡僧喫了,教琴童拏過團靶鉤頭鷄脖壺來,打開腰州精製的紅泥頭,一股一股邈出滋陰摔白酒來,傾在那倒垂蓮蓬高脚鍾内,遞與胡僧。那胡僧接放口内,一吸而飲之。隨即又是兩樣添換上來:一碟寸扎的騎馬腸兒,一碟子醃臘鵝脖子。又是兩樣艷物與胡僧下酒:一碟子癩葡萄,一碟流心紅李子。落後又是一大碗鱔魚麵,與菜卷兒一齊拏上來,與胡僧打散。登時把胡僧喫的楞子眼兒,便道:「貧僧酒醉飯飽,足可以夠了。」西門慶叫左右拏過酒桌去,因問他求房術的薬兒。胡僧道:「我有一枝薬,乃老君煉就,王母傳方。非人不度,非人不傳,專度有緣。旣是官人厚待於我,我與你幾丸罷。」於是向褡褳内取出葫蘆兒,傾出百十丸。吩咐:「每次只一粒,不可多了。用燒酒送下。」又揭開那一個葫蘆兒,捏取了二錢一塊粉紅膏兒,吩咐:「每次只許用二釐,不可多用。若是脹的慌,用手捏着,兩邊腿上只顧摔打百十下,方得通。你可撙節用之,不可輕泄於人!」西門慶雙手接了,説道:「我且問你,這薬有何功効?」胡僧説:

\begin{myquote}[\markfont]
「形如雞卵,色似鵝黄。三次老君炮煉,王母親手傳方。外視輕如糞土,内覷貴乎玕琅。比金金豈換?比玉玉何償?任你腰金衣紫,任你大廈高堂。任你輕裘肥馬,任你才俊棟梁。此薬用託掌内,飄然身入洞房:洞中春不老,物外景長芳。玉山無頽敗,丹田夜有光。一戰精神爽,再戰氣血剛。不拘嬌艷寵,十二羙紅妝。交接従吾好,徹夜硬如鎗。服久寬脾胃,滋腎又扶陽。百日鬚髮黑,千朝體自強。固齒能明目,陽生姤始藏。恐君如不信,拌飯與貓嚐。三日淫無度,四日熱難當,白貓變為黑,尿糞俱停亡。夏月當風臥,冬天水裏藏。若還不解泄,毛脱盡精光。每服一釐半,陽興愈健強。一夜歇十女,其精永不傷。老婦顰眉蹙,淫娼不可當。有時心倦怠,收兵罷戰場。冷水吞一口,陽囬精不傷。快羙終宵楽,春色滿蘭房。贈與知音客,永作保身方!」
\end{myquote}

西門慶聽了,要問他求方兒,説道:「請醫須請良,傳薬湏傳方。吾師不傳於我方兒,倘或我久後用沒了,那裏尋師父去?隨師父要多少東西,我與師父。」因令玳安:「後邊快取二十兩白金來!」遞與胡僧,要問他求這一枝薬方。那胡僧笑道:「貧僧乃出家之人,雲遊四方,要這資財何用?官人趂早收回去!」一面就要起身。西門慶見他不肯傳方,便道:「師父,你不受資財。我有一疋四丈長大布,與師父做件衣服罷。」即令左右取來,雙手遞與胡僧。胡僧方纔打問訊謝了。臨出門,又吩咐:「不可多用。戒之,戒之!」言畢,背上褡褳,拄定拐杖,出門揚長而去。正是:拄杖挑擎雙日月,芒鞋踏遍九軍州。有詩為證:

\begin{myquote}
彌勒和尚到神州,布袋横拖拄杖頭。

饒你化身千百億,一身還有一身愁。
\end{myquote}

畢竟未知後來何如,且聽下囬分解。

