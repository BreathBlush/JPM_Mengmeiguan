\includepdf[pages={89,90},fitpaper=false]{tst.pdf}
\chapter*{第四十五囬 \\桂姐央留夏花兒 月娘含怒駡玳安}
\addcontentsline{toc}{chapter}{第四十五囬 桂姐央留夏花兒 月娘含怒駡玳安}
\markboth{{\titlename}卷之五}{第四十五囬 桂姐央留夏花兒 月娘含怒駡玳安}


\begin{myquote}
佳名號作百花王,幼出冰肌異衆芳:

映日妖嬈呈素豔,隨風冷淡散清香;

玉容每妒啼粧女,雪臉渾如傅粉郎。

檀板金樽歌勝賞,何誇魏紫與姚黄。
\end{myquote}

話説西門慶因放假,沒往衙門裏去。早晨起來,前廳看着差玳安送兩張桌面與喬家去:一張與喬五太太,一張與喬大户娘子,俱有高頂方糖、肘件樹菓之類。喬五太太賞了玳安兩方手帕、三錢銀子;喬大户娘子是一疋青絹,俱不必細説。

原來應伯爵自從與西門慶作别,趕到黄四家,黄四又早夥中封下十兩銀子謝他:「大官人吩咐教俺過節去,口氣兒只是搗那五百兩銀子文書的情。你我錢糧拿甚麽支持?」應伯爵道:「你如今還得多少纔夠?」黄四道:「李三哥他不知道,又要靠着問那内臣借,一般也是五分行利,不如這裏。借着衙門中勢力兒,就是上下使用也省些。如今我看,再得出五十個銀子來,把一千兩合用,就是每月也好認利錢。」應伯爵聽了,低了低頭兒,説道:「不打緊。假若我替你説成了,你夥計衆人怎生謝我?」黄四道:「我對李三説夥中再送五兩銀子與你。」伯爵道:「休説五兩的話。要我手段,五兩銀子要不了你的。我只消一言,替你們巧一巧兒,就在裏頭了。今日俺房下往他家喫酒,我且不去。明日他請俺們晚夕賞燈,你兩個明日絶早買四樣好下飯,再着上一罈金華酒;不要呌唱的,他家裏有李桂姐吳銀兒還沒去哩!你院裏叫上六名吹打的,等我領着送了去。他就要請你兩個坐。我在傍邊,那消一言半句,管情就替你説成了。找出五百兩銀子來,共搗一千兩文書。一個月滿破認他五十兩銀子,那裏不出了,只當你包了一個月老婆了。常言道,秀才無假漆無眞,進錢糧之時,香裏頭多上些木頭,蠟裏頭多攙些㧮油,那裏查帳去!不圖打魚,只圖混水,借着他這名聲兒纔好行事。」於是計議已定。

到時,李三黄四果然買了酒禮,伯爵領着兩個小廝,擡着送到西門慶家來。西門慶正在前廳打發桌面,只見伯爵來到,作了揖,道及:「昨日房下在這裏打攪,回家晚了。」西門慶道:「我昨日周南軒那裏喫酒,回家也有一更天氣,也不曾見的新親,説老早就去了。今早衙門中放假,也沒去。看着打發了兩張桌面,與喬親家那裏去。」説畢,坐下了。伯爵就喚李錦:「你把禮擡進來。」不一時,兩個擡進儀門裏放下。伯爵道:「李三哥黄四哥再三對我説,受你大恩,節間沒甚麽,買了些微禮來孝順你賞人。」只見兩個小廝向前趴在地下磕頭。西門慶道:「你們又送這禮來做甚麽?我也不好受的,還教他擡囬去。」伯爵道:「哥,你不受他的,這一擡出去,就醜死了!他還要叫唱的來伏侍,是我阻住他了,只叫了六名吹打的,在外邊伺候。」西門慶即令:「與我叫進來。」不一時,把六名楽工叫至當面跪下。西門慶向伯爵道:「他既是叫將來了,莫不又打發他?不如請他兩個來坐坐罷。」伯爵得不的一聲兒,即叫過李錦來吩咐:「到家對你爹説,老爹收了禮了。這裏不着人請去了,叫你爹同黄四爹早來這裏坐坐。」那李錦應諾下去。須臾,收進禮去。西門慶令玳安封二錢銀子賞他。磕頭去了。六名吹打的下邊伺候。

少頃,棋童兒拿茶上來,西門慶陪伯爵喫了茶,説道:「有了飯,請問爹那裏喫?」西門慶讓伯爵西廂房裏坐,因問伯爵:「你今日沒會謝子純?」伯爵道:「我早晨起來時,李三就到我那裏,看着打發了禮來,誰得閑去會他?」西門慶即使棋童兒:「快請你謝爹去。」不一時,書童兒放桌兒擺飯,畫童兒用罩漆方盒兒拿了四碟小菜兒,都是裏外花精緻靠山碟兒:一碟羙甘甘十香瓜茄、一碟甜孜孜五方豆豉、一碟香噴噴的橘醬、一碟紅馥馥的糟笋;四大碗下飯:一碗火燎羊頭、一碗滷燉的炙鴨、一碗黄芽菜並𤆑的餛飩鷄疍湯、一碗山薬燴的紅肉圓子;上下安放了兩雙金筯牙兒。伯爵面前是一盞上新白米飯兒,西門慶面前是一甌兒香噴噴軟稻粳米粥兒。兩個同喫了飯,收了家伙去,揩抹的桌兒乾淨。西門慶與伯爵兩個坐着,賭酒兒打雙陸。伯爵趁謝希大未來,乘先問下西門慶,説道:「哥明日找與李智黄四多少銀子?」西門慶道:「把舊文書收了,另搗五百兩銀子文書就是了。」伯爵道:「這等也罷了。哥,你總不如再找上一千兩,到明日也好認利錢。我又一句話,那金子你用不着,還算一百五十兩與他,再找不多兒了。」西門慶聽罷,道:「你也説的是。我明日再找三百五十兩與他罷,改一千兩銀子文書就是了。省的金子放在家也只是閒着。」

兩個正打雙陸,忽見玳安兒走來説道:「賁四拿了一座大螺鈿大理石屏風,兩架銅鑼銅鼓連鐺兒,説是向皇親家的,要當三十兩銀子。爹當與他不當他?」西門慶道:「你教賁四拿進來我瞧。」不一時,賁四同兩個人擡進去,放在廳堂上。西門慶與伯爵撇下雙陸,走出來觀看,原來是三尺闊,五尺高,可桌放的螺鈿描金大理石屏風,端的是一樣黑白分明。伯爵近觀了一囬,悄與西門慶道:「哥,你仔細瞧,恰像好似蹲着個鎭宅獅子一般。兩架銅鑼銅鼓,都是彩畫金粧,雕刻雲頭,十分齊整。」在傍一力攛掇,説道:「哥,該當下他的。休説兩架銅鼓,只一架屏風,五十兩銀子還沒處尋去。」西門慶道:「不知他明日贖不贖?」伯爵道:「沒的説,贖甚麽?下坡車兒營生,及到三年過來,七八本利相等。」西門慶道:「也罷!教你姐夫前邊鋪子裏兑三十兩與他罷。」剛打發去了,西門慶把屏風拂抹乾淨,安在大廳正面,左右看視,金碧彩霞交輝。因問:「吹打楽工喫了飯不曾?」琴童道:「在下邊打發喫飯哩。」西門慶道:「叫他喫了飯來,吹打一回我聽。」於是廳内擡出大鼓來,穿廊下邊一架,安放銅鑼銅鼓,吹打起來,端的聲震雲宵,韻驚魚鳥。

正吹打着,只見棋童兒請了謝希大到了,進來與二人唱了喏。西門慶道:「謝子純,你過來,估估這座屏風兒値多少價?」謝希大近前觀看了半日,口裏只顧誇獎不已,説道:「哥,你這屏風,買的巧也得一百兩銀子,與他少了他不肯。」伯爵道:「你看,連這外邊兩架銅鑼銅鼓帶鐺鐺兒,通共與了三十兩銀子。」那謝希大拍着手兒叫道:「我的南無耶,那裏尋本兒利兒!休説屏風,三十兩銀子還攪給不起這兩架銅鑼銅鼓來。你看這兩座架,做的這工夫,硃紅彩漆,都照依官司裏的樣範,少説也有四十斤響銅,該値多少銀子?怪不的一物一主,那裏有哥這等大福,偏有這樣巧價兒來尋你的!」説了一囬,西門慶請入書房裏坐的。不一時,李智黄四也到了。西門慶説道:「你兩個如何又費心送禮來?我又不好受你的。」那李智黄四慌的下了禮,説道:「小人惶恐,微物胡亂與爹賞人罷了。蒙老爹呼喚,不敢不來。」於是搬過坐兒來,打横坐了。須臾,小廝畫童兒拿了五盞茶上來,衆人喫了,收下盞托去。少頃,玳安走上來請問:「爹,在那裏放桌兒?」西門慶令:「擡進桌兒就在這裏坐罷。」於是玳安與書童兩個,一肩搭擡進一張八僊瑪瑙籠漆桌兒進來,騎着火盆安放在地平上。伯爵希大居上,西門慶主位,李智黄四兩邊打横坐了。須臾拿上春檠按酒,大盤大碗湯飯點心,無非鵝鴨鷄蹄各樣下飯之類。酒泛羊羔,湯浮桃浪。楽工都在窗外吹打。西門慶叫了吳銀兒席上遞酒。這裏前邊飲酒不題。

卻説李桂姐家保兒,吳銀兒家丫頭蠟梅,都叫了轎子來接他姐姐家去。那桂姐聽保兒來,慌的走到門外,和保兒兩個悄悄説了半日話。囬到上房,告辭要囬家去。月娘再三留他:「俺們如今便都往吳大妗子家去,連你們也帶了去。你一發晚了従他那裏起身,也不用轎子,伴俺們走百病兒,就往家去便了。」桂姐道:「娘不知,我家裏無人,俺姐姐又不在家,有我五姨媽那裏又請了許多人來做盒子會,俺媽不知怎麽盼我,昨日等了我一日。他不急時,不使將保兒來接我。若是閒常日子,隨娘留我幾日,我也住了。」月娘見他不肯,一面教玉簫將他那原來的盒子,裝了一盒元宵,一盒白糖薄脆,交與保兒掇着;又與桂姐一兩銀子,打發他早去。

這桂姐先辭月娘衆人,然後他姑娘送他到前邊,教畫童替他抱了毡包,竟來書房門首,教玳安請出西門慶來説話。這玳安慢慢掀簾子,進入書房,向西門慶請道:「桂姐家去,請爹説話。」應伯爵道:「李桂兒這小淫婦兒原來還沒去哩。」西門慶道:「他今日纔家去。」一面走出前邊來,看見李桂姐穿着紫丁香色潞州紬粧花眉子對衿襖兒,白展光五色線挑的寬襴裙子,用青點翠的白綾汗巾兒搭着頭。向前花枝招颭,綉帶飄飄,磕了四個頭,就道:「打攪爹娘這裏。」西門慶道:「你明日家去罷!」桂姐道:「家裏無人,媽使保兒拿轎子來接了。」又道:「我還有一件事對爹説。俺姑娘房裏那孩子,休要領出去罷!俺姑娘昨日晚夕,又打了他幾下。説起來還小哩,恁甚麽不知道。喫我説了他幾句,従今改了,他也再不敢了。不爭打發他出去,大節間俺姑娘房中沒個人使,你心裏不急麽?自古木杓火杖兒短,強如手撥剌。爹好歹看我分上,留下這丫頭罷。」西門慶道:「既是你恁説,留下這奴才罷。」一面吩咐玳安:「你去後邊對你大娘説,休要叫媒人去了。」玳安見畫童兒抱着桂姐毡包,説道:「拿桂姨毡包等我抱着。敎畫童兒後邊説去罷。」那畫童應諾,一直往後邊去了。桂姐與西門慶説畢話,去窗子前揚聲叫道:「應花子,我不拜你了!你娘家去。」伯爵道:「拉回賊小淫婦兒來,休放他去了。叫他唱一套兒,且與我聽聽着。」桂姐道:「等你娘閒了,唱與你罷。」伯爵道:「只你兩個説梯己話兒,就不教我知道了?由他乾乾淨淨恁大白日就家去了,便益了賊小淫婦兒了。投到黑,還接好幾個漢子。」桂姐道:「汗邪了你這花子。」一面笑着出去。玳安跟着,打發他上轎去了。

西門慶與桂姐説了話,後邊更衣去了。應伯爵向謝希大説:「李家桂兒這小淫婦兒就是個眞脱牢的強盜,越發賊的疼人子!恁個大節,他肯只顧在人家住着?鴇子來叫他,又不知家裏有甚麽人兒等着他哩!」謝希大道:「你好猜?」悄悄向伯爵耳邊,如此如此,這般這般,説未數句,伯爵道:「悄悄裏説,這哥還不知道哩!」不一時,西門慶走的脚步兒響進來,兩個就不言語了。這應伯爵就把吳銀兒摟在懷裏,和他一遞一口兒喫酒,説:「還是我這乾女兒又溫柔又軟款,強如李家狗不要的小淫婦兒一百倍了!」吳銀兒笑道:「二爹好罵!説一個就一個,百個就百個。一般一方之地,也有賢有愚,可可兒一個就比一個來?俺桂姐沒惱着你老人家!」西門慶道:「你聽賊狗才,單管只六説白道的!」伯爵道:「你休管他家,等我守着我這乾女兒過日子。乾女兒過來,拿琵琶且先唱個兒我聽。」這吳銀兒不忙不慌,輕舒玉指,款跨鮫綃,把琵琶横於膝上,低低唱了一回〔柳搖金〕:

\begin{myquote}
「心中牽掛,飯不飯茶不茶,難割拾我俏冤家。凄凉,因爲我心上放不下,更不知你在誰家!要離别,與我兩句伶俐話。抛閃殺奴家,閃賺殺奴家,你休要把奴來干罷!」
\end{myquote}

伯爵喫過酒,又遞謝希大。吳銀兒又唱道:

\begin{myquote}
「常懷憂悶,何時得趁我心,牽掛着我有情人。姊妹們拘管的緊,老尊堂不放鬆,顯的我言而無信。不愛你寳和金,只愛你,只愛你生的龐兒俊。我和你做夫妻,死了甘心。教奴和你往來相趁。」
\end{myquote}

這裏和吳銀兒前邊遞酒彈唱不題。且説畫童兒走到後邊,月娘正和孟玉樓、李瓶兒、大姐、雪娥,並大師父,都在上房裏坐的。只見畫童兒進來,月娘纔待使他叫老馮來領夏花兒出去,畫童便道:「爹使小的對大娘説,教且不要領他出去罷了。」月娘道:「你爹教賣他,怎的又不賣他了?你實説,是誰對你爹説,教休要領他出去。」畫童兒道:「剛纔小的抱着桂姨毡包,桂姨臨去對爹説,央及留下了:『且將就使着罷,休領出去了。』爹使玳安進來對娘説。玳安不進來,在爹跟前使小的進來了;奪過毡包送桂姨去了。」這月娘聽了,就有幾分惱在心中。罵玳安道:「恁賊兩頭戳舌獻勤欺主的奴才!嗔道他頭裏使他叫媒人,他就説道:『爹教領出去。』原來都是他弄鬼!如今又幹辦着送他去了。住囬等他進後來,我和他答話。」

正説着,只見吳銀兒前邊唱了進來。月娘對他説:「你家蠟梅接你來了。李家桂兒家去了,你莫不也往家去了罷?」吳銀兒道:「娘旣留我,我又家去,顯的不識敬重了!」因問蠟梅:「你來做甚麽?」蠟梅道:「媽使我來瞧瞧你。」吳銀兒問道:「家裏沒甚勾當?」蠟梅道:「沒甚事。」吳銀兒道:「旣沒事,你來接我怎的?你家去罷。娘留下我,晚夕還同衆娘們往妗奶奶家走百病兒去。我那裏囬來纔往家去哩。」説畢,蠟梅就要走。月娘道:「你叫他囬來,打發他喫些甚麽兒。」吳銀兒道:「你大奶奶賞你東西喫哩!等着就把衣裳包子帶了家去。對媽媽説,休教轎子來,晚夕我走了家去。」因問:「吳惠他怎的不來?」蠟梅道:「他在家裏害眼哩。」月娘吩咐玉簫領蠟梅到後邊,拿下兩碗肉,一盤子饅頭,一甌子酒,打發他喫。又拿他原來的盒子,裝了一盒元宵,一盒細茶食,囬與他拿去。

原來吳銀兒的衣裳包兒,放在李瓶兒房裏。李瓶兒連忙又早尋下一套上色織金緞子衣服,兩方銷金汗巾兒,一兩銀子,安放在他毡包内與他。那吳銀兒喜孜孜辭道:「娘,我不要這衣服罷。」又笑嘻嘻道:「實和娘説,我沒個白襖兒穿。娘收了這緞子衣服,不拘娘的甚麽舊白綾襖兒,與我一件兒穿罷。」李瓶兒道:「我的白襖子都寬大,你怎好穿?」於是叫迎春拿鑰匙上大廚櫃裏,拿一疋整白綾來與銀姐:「對你媽説,教裁縫替你裁兩件好襖兒。」因問:「你要花的要素的?」吳銀兒道:「娘,我要素的罷,圖襯着比甲兒好穿。」笑嘻嘻向迎春説道:「又起動叫姐往樓上走一遭,明日我沒甚麽孝順,只是唱曲兒與姐姐聽罷了。」須臾,迎春從樓上取了一疋松江闊機尖素白綾,下號兒寫着重三十八兩,遞與吳銀兒。銀兒連忙花枝招颭,綉帶飄飄,插燭也似與李瓶兒磕了四個頭,起來,又深深拜了迎春幾拜。李瓶兒道:「銀姐,你把這緞子衣服還包了去,早晚做酒衣兒穿。」吳銀兒道:「娘賞了白綾做襖兒,又包了這衣服去?」於是又磕頭謝了。不一時,蠟梅喫了東西,交與盒子、毡包,都拿回家去了。月娘便説:「銀姐,你這等我纔喜歡。你休學李桂兒那等喬張致,昨日和今早,只像臥不住虎子一般,留不住的只要家去。可可兒家裏就忙的恁樣兒?連唱也不用心唱了!見他家人來接,飯也不喫就去了,就不待見了。銀姐,你快休學他!」吳銀兒道:「好娘,這裏一個爹娘宅裏是那裏去䖏?就有虚篢,放着別處使,敢在這裏使!桂姐年幼,他不知事,俺娘休要惱他。」

正説着,只見吳大妗子家使了小廝來定兒來請,説道:「俺娘上覆三姑娘,好歹同衆位娘並桂姐銀姐請早些過去罷;又請雪姑娘也走走。」月娘道:「你到家對你娘説,俺們如今便收拾去。二娘害腿疼不去,他在家看家哩。你姑夫今日前邊有人喫酒,家裏沒人,後邊姐也不去。李桂姐家去了,連大姐銀姐和俺們六位去。你家少費心整治甚麽,俺們坐一囬,晚上就來。」因問來定兒:「你家叫了誰在那裏唱?」來定兒道:「是郁大姐。」說畢,來定兒先去了。月娘一面同玉樓金蓮李瓶兒大姐并吳銀兒,對西門慶説了,吩咐奶子在家看哥兒,都穿戴收拾定當,共六頂轎子起身。派定玳安兒棋童兒來安兒三個小廝,四名排軍跟轎,往吳大妗子家來。正是:

\begin{myquote}
萬井風光春落落,千門燈火夜漫漫;

此生此夜不長見,明月明年何處看?
\end{myquote}

畢竟未知後來何如,且聽下囬分解。

