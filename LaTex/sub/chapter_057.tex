\includepdf[pages={113,114},fitpaper=false]{tst.pdf}
\chapter*{第五十七囬 \\道長老募修永福寺 薛姑子勸捨陀羅經}
\addcontentsline{toc}{chapter}{第五十七囬 道長老募修永福寺 薛姑子勸捨陀羅經}
\markboth{{\titlename}卷之六}{第五十七囬 道長老募修永福寺 薛姑子勸捨陀羅經}


\begin{myquote}
本性圓明道自通,翻身跳出網羅中。

修成禪那非容易,煉就無生豈俗同。

清濁幾番隨運轉,闢開數刼任西東。

逍遙萬億年無計,一點神光永注空。
\end{myquote}

話説那山東東平府地方,向來有個永福禪寺,起建自梁武帝普通二年,開山是那萬迴老祖。怎麽叫做萬迴老祖?因那老師父七八歲的時節,有個哥兒從軍邊上,音信不通,不知生死。因此上那老娘兒思想那大的孩兒,掉不下的心腸,時常在家啼哭,忽一日,那孩子問着母親説道:「娘,這等清平世界,孩兒們又沒的打攪你,頓頓兒小米飯兒,咱家也儘挨的過。怎地的,你時時掉下淚來。娘,你説與咱,咱也好分憂哩。」那老娘兒就説:「小孩子,你還不知道老人家的苦哩!自從你老頭兒去世,你大哥兒到邊上去做了長官,四五年他信兒也不捎一個來家,不知他死生存亡,教我老人家怎生丢的下?」説了又哭起來。那孩子説:「早是這等,有何難哉!娘,如今哥在那裏?咱做弟郎的早晚間走去,找着哥兒,討個信來回覆你老人家,卻不是好?」那婆婆一頭哭,一頭笑起來,説道:「怪獃子!説起你哥在甚地,若是那一百二百里程途,便可去的。直在那遼東地面,去此一萬餘里,就是那好漢子,也走得了不的,直要四五個月纔到哩。笑你孩兒家怎麽去的?」那孩子就説:「嗄!若是果在遼東,也終不在個天上,我去去,尋哥兒就囬也。」只見把靸鞋兒繫好了,把直裰兒整一整,望着婆兒拜個揖,一溜煙去了。那婆婆呌之不應,追之不及,愈添愁悶。也有鄰舍街坊婆兒婦女,挨肩擦背,拿湯送水,説長道短,前來解勸。也有説的是的,説道:「孩兒小,怎去的遠?早晚間卻回也。」因此婆婆也收着兩眶眼淚,悶悶的坐地。

看看紅日西沉,東隣西舍,一個個燒湯煮飯,一個個上榻關門。那婆婆探頭探腦,那兩隻眼珠兒一直向外,恨不的趕將上去。只見遠遠的,望見那黑魆魆影兒裏有一個小的兒來也。那婆婆就説:「靠天靠地,靠着日月三光,若得俺小的子兒來也,也不虧了俺修齋吃素的念頭!」只見那萬迴老祖一忽地跪到跟前説:「娘,你還未睡炕哩,咱已到遼東找着哥兒,討的平安家信來也。」婆婆笑道:「孩兒,你不去的正好,免教你老人家掛心。只是不要調着謊哄着老娘。那裏有一萬里路程朝暮往還的?」孩兒道:「娘,你不信不信麽?」一直裏卸下衣包,取出平安家信,果然是那哥兒手筆。又取出一件汗衫帶囘漿洗的,也是那個婆婆親手縫紉的,毫釐不差。因此哄動了街坊,呌做「萬回」。日後捨俗出家,就呌做萬回長老。果然是道德高妙,神通廣大。曾在那後趙皇帝石虎跟前,吞下兩升鐵針兒;又在那梁武皇殿下,在頭頂上取出舍利三顆。因此勑建那永福禪寺,做那萬回老祖的香火院,正不知費了多少錢糧。正是:神僧出世神通大,聖主尊隆聖澤深。

不想那歲月如梭,時移事改。只見那萬迴老祖歸天圓寂,那些得皮得肉的上人們,一個個都化去了。只見有幾個憊賴的和尚,撇賴了百丈清規,養婆兒,吃燒酒,咋事兒不弄出來?打哄哄,燒苦蔥,咱勾當兒不做?卻被那些潑皮賴虎,常常作酒撈錢抵當。不過一會兒,把袈裟也當了,鐘兒磬兒都典了,殿上椽兒賣了,沒人要的燒了,磚兒瓦兒換酒吃了。弄得那雨淋風刮,佛像兒倒了;荒荒凉凉,燒香的也不來了。主顧門徒、做道場的、荐亡的,都是關大王賣荳腐——鬼兒也沒的上門了。一片鐘鼓道場,忽變做荒煙衰草!驀地裏三四十年,那一個扶衰起廢?

原來那寺裏有個道長老,原是西印度國出身,因慕中國清華,發心要到上方行脚。打従那流沙河、星宿海、㴶兒水地方,走了八九個年頭,才到中華區䖏。迤邐來到山東地方,卓錫在這個破寺院裏面。面壁九年,不言不語。眞個是:佛法原無文字障,工夫好向定中尋。忽一日,發個念頭,説道:「呀!這寺院兒坍塌的這模樣了。你看這些蠢頭村腦的禿驢,止會吃酒噇飯。把這古佛道塲,弄得赤白白地,豈不可惜!那一個尋得一磚半瓦,重整家風?常記的古人説得好:人傑地靈。事到今日,咱不做主,那個做主?咱不出頭,那個出頭?况且前日山東有個西門大官人,官居錦衣之職。他家私巨萬,富比王侯,家中那一件没有?前日餞送蔡御史,曾在咱這裏擺設酒席。他因見咱這裏寺宇傾頽,就有個舍錢布施、鼎建重新的意思。咱那時口雖不言,心窝裏已存下幾分了。今日呵,若得那個檀越為主作倡,管情早晚間把咱好事成就也!咱湏辦自家去走一遭。」當時間喚起法子徒孫,打起鐘,敲起鼓,擧集大衆,上堂宣揚此意。那長老怎生打扮?只見:

\begin{myquote}
身上禪衣猩血染,雙環掛耳是黄金;

手中錫杖光如鏡,百八胡珠耀日明。

開覺明路現金繩,提起凡夫夢亦醒;

龐眉紺髮銅鈴眼,道是西天老聖僧。
\end{myquote}

那長老宣揚已畢,就教行者拿過文房四寳,磨起龍香劑,飽揝鼠鬚筆,展開烏絲欄,寫着一篇疏文。先叙那始末根由,後勸人捨財作福。寫的行行端正,字字清新。好長老,眞個是古佛菩薩現身,従此辭了大衆,着上了禪鞋,戴上個斗篷笠子,一壁廂直奔到西門慶家府裏來。

且説西門慶辭別了應伯爵,轉到後廳,直到捲棚下卸了衣服。走到吳月娘房内,把那應伯爵荐水秀才的事體,説了一番。就説道:「咱前日東京去的時節,多虧那些親朋齊來與咱把盞。如今少不的也要整辦些兒小酒回答他。倒今日空閒,沒件事體,就把這事兒完了也罷。」當下就叫了玳安拿了籃兒,到十市街坊買下些時鮮菓品,猪羊魚肉,腌臘鷄鵝嗄飯之類。吩咐了當,就吩咐小廝分頭去請各位。一面拉着月娘一同走到李瓶兒房裏來看官哥。李瓶兒笑嘻嘻的接住了月娘西門慶。西門慶道:「娘兒來看孩子哩。」李瓶兒就叫奶子抱出官哥。只見眉目稀疎,就如粉塊裝成一般,笑欣欣直攢到月娘懷裏來,月娘把手接着,抱起道:「我的兒,恁地乖覺。長大來定是聰明伶俐的。」又向那孩子説:「兒長大起來,怎地奉養老娘哩?」那李瓶兒就説:「娘説那裏話?假饒兒子長成,討的一官半職,也先向上頭封贈起。娘,那鳳冠霞帔穩穩兒先到娘哩,好生奉養老人家!」西門慶接口便説:「兒,你長大來,還掙個文官。不要學你家老子,做個西班出身。雖有興頭,卻沒十分尊重。」正説着,不想那潘金蓮正在外邊聽見,不覺的怒従心上起,就駡道:「没廉耻弄虚脾的臭娼根,偏你會養兒子哩!也不曾經過三個黄梅四個夏至,又不曾長成十五六歲,出幼過關上學堂讀書,還是水的泡,與閻羅王合養在這裏的,怎見的就做官?就封贈那老夫人?我那怪賊囚根子,沒廉耻的貨,怎地就見的要他做個文官,不要像你?」正在嘮嘮叨叨,喃喃噥噥,一頭駡,一頭着惱的時節,只見那玳安走將進來,叫聲五娘,説道:「爹在那裏?」潘金蓮便駡:「怪尖嘴的賊囚根子,那個曉的你什麽爹在那裏?爹怎的到我這屋裏來?他自有五花官誥的太奶奶、老封婆,八珍五鼎奉養他的在那裏,那裏問着我討?」那玳安就曉的不是路了,説:「是了。」望六娘房裏便走。走到房門前,打個咳嗽,朝着西門慶道:「應二爹在廳上。」西門慶道:「應二爹纔送的他去,又做甚?」玳安道:「爹自家出去便知。」

西門慶只得撇了月娘李瓶兒,仍到那捲棚下面,穿了衣服,走到外邊迎接伯爵。正要動問間,只見那募緣的道長老已到西門慶門首了。高聲呌:「阿彌陀佛!這是西門老爹門首麽?那個掌事的管家與吾傳報一聲,説道扶桂子、保蘭孫,求福有福,求壽有壽,東京募緣的長老求見。」原來西門慶平日原是一個撒漫好使錢的漢子,又是新得官哥,心下十分歡喜,也要幹些好事,保佑孩兒。小廝也通曉得,並不嗔道作難,一壁廂進報西門慶。西門慶就説:「且教他進來看。」只見管家的三步挪來兩步走,就如見子活佛的一般,慌忙請了長老。那長老進到花廳裏面,打了個問訊,説道:「貧僧出身西印度國,行脚到東京汴梁,卓錫在永福禪寺,面壁九年,頗傳心印。止為那殿宇傾頽,琳宫倒塌。貧僧想的起來,為佛弟子,自然應的為佛出力,總不然儹到那個身上去,因此上,貧僧發了這個念頭。前日老檀越餞行各位老爹時悲憐本寺廢壞,也有個良心美腹,要和本寺作主。那時諸佛菩薩已作證盟。貧僧記的佛經上説的好:『如有世間善男子、善女人,以金錢喜捨,莊嚴佛像者,主得桂子蘭孫,端嚴美貌,日後早登科甲,蔭子封妻之報。』故此特叩高門,不拘五百一千,要求老檀那開疏發心,成就善果。」就把錦帊展開,取出那募緣疏簿,雙手遞上。不想那一席話兒,早已把西門慶的心兒打動了。不覺的歡天喜地接了疏簿,就叫小廝看茶。揭開疏簿,只見寫道:

\begin{myquote}
「伏以白馬駝經開象教,竺騰衍法啟宗門。大地衆生,無不皈依佛祖;三千世界,盡皆蘭若裝嚴。看此瓦礫傾頽,成甚名山勝境?若不慈悲喜捨,何稱佛子賢人?今有永福禪寺,古佛道場,焚修福地,啟建自梁武皇帝,開山是萬迴祖師。規制恢弘,彷彿那給孤園萬金鋪地;雕鏤精製,依稀似祇洹舍白玉為堦。高閣摩空,旃檀氣直接九霄雲表;層基亙地,大雄殿可容千衆禪僧。兩翼嵬峨,盡是琳宫紺宇;廊房潔淨,果然精勝洞天。那時鐘鼓宣揚,盡道是寰中佛國;只這淄流濟楚,卻也像塵界人天。那知歲久年深,一瞬時移事異。莽和尚縱酒撒潑,首壞清規;獃道人懶惰貪眠,不行打掃。漸成寂寞,断絶門徒。以致凄涼,罕稀瞻仰。兼以烏鼠穿蝕,那堪風雨漂搖?棟宇摧頹,一而二,二而三,支持靡計;墻垣坍塌,日復日,年復年,振起無人。朱紅櫺槅,拾來煨酒煨茶;合抱梁檻,拿去換鹽換米。風吹羅漢金消盡,雨打彌陀化作塵。吁嗟乎,金碧焜炫,一旦為灌莽榛荆。雖然有成有敗,終湏否極泰來。幸而有道長老之虔誠,不忍見梵王宫之廢敗,發大弘願,遍叩檀那。伏願咸起慈悲,盡興惻隱。梁柱椽楹,不拘大小,喜捨到高題姓字;銀錢布幣,豈論豐贏,投櫃日疏簿標名。仰仗着佛祖威靈,福祿壽永永百年千載;倚靠他伽藍明鏡,父子孫個個厚祿高官。瓜瓞綿綿,森挺三槐五桂;門庭奕奕,煌煌金阜錢山。凡所營求,吉祥如意。疏文到日,各破慳心,謹疏。」
\end{myquote}

看畢,西門慶就把册葉兒收好,粧入那錦套裏頭,把插銷兒銷着,錦帶兒拴着,恭恭敬敬放在桌兒上面,叉手而言,對長老説:「實不相瞞,在下雖不成個人家,也有幾萬産業。忝居武職,交遊世輩儘有。不想偌大年紀,未曾生下兒子。房下們也有五六房,只是放心不下,有意做些善果。去年第六房賤累生下孩子。咱萬事已是足了。偶因餞送俺友,得到上方,因見廟宇傾頹,有個捨財助建的念頭。蒙老師下顧,西門慶那敢推辭?」拿着兔毫妙筆,正在躊躇之際,那應伯爵就説:「哥,你既有這片好心為姪兒發願,何不一力獨成,也是小可的事體!」西門慶拿着筆,哈哈地笑道:「力薄,力薄!」伯爵又道:「極少也助一千。」西門慶又哈哈地笑道:「力薄,力薄!」那長老就開口説道:「老檀越在上,不是貧僧多口,只是我們佛家的行徑,都要隨緣喜捨,終不強人所難。隨分但憑老爹發心便是!此外親友,更求檀越吹噓吹噓。」西門慶又説道:「還是老師體諒,少也不成,就寫上五百兩。」閣了兔毫筆。那長老打個問訊謝了。西門慶又說:「我這裏内官太監,府縣倉巡,一個個都與我相好的。我明日就拿疏簿去要他們寫。寫的來,就不拘三百二百、一百五十,管教與老師成就這件好事。」當日留了長老素齋,相送出門。正是:慈悲作善豪家事,保福消災父母心。又有一首詞,單道那些施主的事體:

\begin{myquote}
佛法無多止在心,種瓜種菓是根因。

珠和玉珀寳和珍,誰人拿淂見閻君?

積善之人貧也好,豪家積業枉抛銀。

若使年齡財可買,董卓還應活到今!
\end{myquote}

卻說西門慶送了長老,轉到廳上,與應伯爵坐地,道:「二哥,我正要差人請你,你來的正好。我前日因往東京,多虧衆親友們與咱把個盞兒。今日吩咐小的買辦,你家大嫂安排小酒與衆人囬答,要哥在此相陪。不想遇着這個長老,鬼混了一會兒。」那伯爵就說道:「好個長老,想是果然有德行的。他説話中間,連咱也心動起來,做了施主。」西門慶説道:「二哥,你又幾曾做施主來的?疏簿又是幾時寫的?」應伯爵笑道:「咦!難道我出口的不是施主不成?哥,你也不曾見佛經過來?佛經上第一重的是心施,第二法施,第三纔是財施。難道我従傍攛掇的,不當個心施的不成?」西門慶又笑道:「二哥,只怕你有口無心哩!」兩人拍手大笑。應伯爵就説:「小弟在此等待客來。哥有正事,自與嫂子商議去來。」

只見西門慶別了伯爵,轉到内院裏頭。只見那潘金蓮哰哰唔唔,没瞅沒睬,不覺的睡魔纏擾,打了幾個噴㖒,走到房中,倒在象牙牀上,一忽地睡去了。那李瓶兒又為孩子啼哭,自與那奶子丫鬟在房中坐地,看官哥喜笑。只有那吳月娘與孫雪娥,兩個伴當在那裏整辦嗄飯。西門慶走到面前坐地,就把那道長老募緣與那自己開疏的事,備細對月娘説了一番。又把那應伯爵耍笑打覷的說話也説了一番。歡天喜地,大家嘻笑了一會。只見那吳月娘,畢竟是個正經的人,不慌不忙,不思不想,説下幾句話兒,倒是西門慶頂門上針。正是:妻賢每致鷄鳴警,款語常聞薬石言。畢竟那説話怎麽講?月娘説道:「哥,你天大的造化,生下孩兒!你又發起善念,廣結良緣,豈不是俺一家兒的福分?只是那善念頭怕他不多,那惡念頭怕他不盡。哥,你日後那沒來由沒正經、養婆兒沒搭煞、貪財好色的事體少幹幾樁兒也好。攢下些陰功,與那小的子也好。」西門慶笑道:「你的醋話兒又來了。卻不道天地尚有陰陽,男女自然配合。今生偷情的、苟合的,都是前生分定,姻緣簿上注名,今生了還。難道是生剌剌胡搊、亂扯歪斯纏做的?咱聞那佛祖西天,也只不過要黄金鋪地;陰司十殿,也要些楮鏹營求。咱只消儘這家私廣為善事,就使強姦了嫦娥,和姦了織女,拐了許飛瓊,盜了西王母的女兒,也不減我潑天富貴!」月娘笑道:「笑哥狗吃熱屎——原道是個香甜的!生血掉在牙兒内——怎生改得?」

正説笑間,只見那王姑子同了薛姑子提一個盒子,直闖進來,飛也似朝月娘道個萬福,又向西門慶拜了拜説:「老爹,你倒在家裏!我自前日別了,因為有些小事,不得空,不曾來看得你老人家,心子裏丢不下,今日同這薛姑子來看你!」原來這薛姑子,不是従幼出家的。少年間曾嫁丈夫,在廣成寺前居住,賣蒸餅兒生理。不料生意淺薄,那薛姑子就有些不尷不尬,專一與那些寺裏的和尚行童調嘴弄舌,眉來眼去,説長説短。弄的那些和尚們的懷中個個是硬幫幫的。乘那丈夫出去了,茶前酒後,早與那和尚們刮上了四五六個。也常有那火燒、波波、饅頭、栗子,拿來進奉他,又有那付應錢與他買花,開地獄的布送與他做裹脚。他丈夫那裏曉得?以後丈夫得病死了,他因佛門情熟,這等就做了個姑子,專一在些士夫人家往來,包攬經懺。又有那些不長進要偷漢子的婦人,叫他牽引和尚進門,他就做個馬泊六兒,多得錢鈔。聞的那西門慶家裏豪富,見他侍妾多人,思想拐些用度,因此頻頻往來。那西門慶也不曉的,三姑六婆,人家最忌出入。正是:

\begin{myquote}
當年行徑是窠兒,和尚闍黎舖。中間打扮念彌陀,開口兒就説西方路。尺布裹頭顱,身穿直裰,繫個黄縧,早晚捱門傍戶。騙金銀猶自可,心窝裏畢竟胡塗。算來不是好姑姑,幾個清名被點汚。
\end{myquote}

又有一隻歌兒道得好:

\begin{myquote}
尼姑生來頭皮光,拖子和尚夜夜忙。三個光頭好像師父師兄并師弟,只是鐃鈸緣何在裏牀?
\end{myquote}

那薛姑子坐下,就把那個小盒兒揭開,説道:「咱們沒有什麽孝順,拿得施主人家幾個供佛的菓子兒,權當獻新。」月娘道:「要來竟自來便了,何苦要你費心。」只見那潘金蓮睡覺,聽得外邊有人說話,又認是前番光景,便走向前來聽看。那李瓶兒在房中弄孩子,因曉得王姑子在此,也要與他商議保佑官哥,一同到月娘房中,大家道個萬福,各各坐地。西門慶因見李瓶兒不曾曉的,又把那道長老募緣,與那自家開疏捨財,替官哥求福的事情,重新又説一番。不想道惱了潘金蓮,抽身竟走,喃喃噥噥,一溜煙竟自去了。只見那薛姑子站將起來,合掌着手,叫聲:「佛阿!老爹,你這等樣好心作福,怕不的壽年千歲,五男二女,七子團圓。只是我還有一件,説與你老人家,這個因果費不甚多,更自獲福無量。咦!老檀越,你若幹了這件功德,就是那老瞿曇雪山修道,迦葉尊散髮鋪地,二祖可投崖飼虎,給孤老滿地黄金,也比不的你功德哩!」西門慶笑道:「姑姑且坐下,細説甚麽功果,我便依你。」那薛姑子就説:「我們佛祖㽞下一卷《陀羅經》,專一勸人法西方淨土的。佛説那三禪天、四禪天、忉利天、兜率天、大羅天、不周天,急切不能即到。唯有西方極楽世界,這是阿彌陀佛出身所在,沒有那春夏秋冬,也沒有那風寒暑熱,常常如三春時候融和天氣,也沒有夫婦男女。其人生在七寳池中,金蓮臺上……」西門慶道:「那一朶蓮花有幾多大?生在上邊,一陣風擺,怕不骨碌碌掉在池裏麽?」薛姑子道:「老爹,你還不曉的。我依那經上説,佛家以五百里為一由旬,那一朶蓮花好生利害,大的緊,大的緊,大的五百由旬。寳衣隨願至,玉食自天來;又有那些好鳥和鳴,如笙簧一般,委的好個境界!因為那肉眼凡夫不知去向,不生尊信,故此佛祖演説此經,勸人專心念佛,竟往西方見了阿彌陀佛。自此一世二世,以至百千萬世,永永不落輪迴。那佛祖説的好:如有人持頌此經,或將此經印刷抄寫,轉勸一人,至千萬人持誦,獲福無量!况且此經裏面,又有獲諸童子經咒。凡有人家生育男女,必要従此發心,方得易長易養,災去福來。如今這付經板見在,只沒人印刷施行。老爹,你只消破些工料,印上幾千卷,裝釘完成,普施十方,那個功德,眞個大的緊!」西門慶道:「也不難。只不知這一卷經,要多少紙札?多少裝釘工夫?多少印刷?有個細數,纔好動彈。」薛姑子又道:「老爹,你一發獃了,説那裏話去,細細算將起來?止消先付九兩銀子,交付那經坊裏,要他印造幾千幾萬卷。裝釘完滿,以後一攪果算還他工食紙札錢兒就是了,卻怎地要細細算將出來!」

正説的熱鬧,只見那陳經濟要與西門慶説話,跟尋了好一回不見,問那玳安,説在月娘房裏。走到捲棚底下,剛剛凑巧,遇着了那潘金蓮,凭闌獨惱。猛然擡起頭來,見了經濟,就是個貓兒見了魚鮮飯,一心心要啖他下去了。不覺的把一天愁悶,都改做春風和氣。兩個乘着没有人來,執手相偎,做剝嘴咂舌頭。兩下肉麻,好生兒頑了一囘兒。因恐怕西門慶出來撞見,連那算帳的事情也不吆呼,兩雙眼又像老鼠兒見了貓來,左顧右盼提防着,又没個方便,一溜煙自出去了。

且說西門慶聽罷了薛姑子的話頭,不覺心上打動了一片善念。就叫玳安取出拜匣,把汗巾上的小匙鑰兒開了,取出一封銀子,准准三十兩足色松紋,便交付薛姑子與那王姑子:「即便同去,隨分那裏經坊,與我印下五千卷經。待完了我就算帳,找他。」

正話間,只見那書童忙忙的來報道:「請的各位客人都到了。」少不的是吳大舅、花大舅、謝希大、常時節,這一班,都各齊齊整整一齊到。西門慶忙的不迭,即便整衣出外迎接,升堂,就叫小廝擺下桌兒,放下小菜兒。請吳大舅上坐了,衆人一行兒分班列次,各敍長幼,各各坐地。那些醃臘煎熬、大魚大肉、燒鷄燒鴨、時鮮菓品,一齊兒都捧將出來。西門慶又叫道:「開那麻姑酒兒盪來。」只見酒逢知己,形迹都忘。猜枚的、打鼓的、催花的、三拳兩謊的,歌的歌,唱的唱。談風月,盡道是杜工部、賀黃門乘春賞翫;掉文袋,也曉的蘇玉局、黄魯直赤壁清遊。投壺的定要那正雙飛、拗雙飛、八僊過海;擲色的又要那正馬軍、拗馬軍、鰍入菱窠。輸酒的要喝個無滴,不怕你玉山頹倒;贏色的又要去掛紅,誰讓你倒着接䍦。頑不盡少年場光景,說不了醉鄉裏日月。正是:

\begin{myquote}
秋月春花隨處有,賞心楽事此時同。

百年若不千場醉,碌碌營營總是空。
\end{myquote}

畢竟未知後來何如,且聽下回分解。

