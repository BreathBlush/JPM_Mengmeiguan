\includepdf[pages={43,44},fitpaper=false]{tst.pdf}
\chapter*{第二十二囬 \\西門慶私淫來旺婦 春梅正色罵李銘}
\addcontentsline{toc}{chapter}{第二十二囬 西門慶私淫來旺婦 春梅正色罵李銘}
\markboth{{\titlename}卷之三}{第二十二囬 西門慶私淫來旺婦 春梅正色罵李銘}


\begin{myquote}
巧厭多勞拙厭閒,善嫌懦弱惡嫌頑;

富遭嫉妒貧遭辱,勤怕貪圖儉怕慳。

觸事不分皆笑拙,見機而作又疑奸:

思量那件合人意,為人難做做人難!
\end{myquote}

話説次日有吴大妗子、楊姑娘、潘姥姥衆堂客,都來與孟玉樓做生日。月娘在後廳與衆客飲酒,倒也罷了,其中惹出一件事來。

那來旺兒因他媳婦自家癆病死了,月娘新近與他娶了一房媳婦,娘家姓宋,乃是賣棺材宋仁的女兒。當先賣在蔡通判家房裏使喚,後因壞了事出來,嫁與廚役蔣聰為妻小。這蔣聰常在西門慶家做活答應,來旺兒早晚到蔣聰家叫蔣聰去,看見這個老婆,兩個吃酒刮言,就把這個老婆刮上了。一日,不想這蔣聰因和一般廚役分財不均,酒醉廝打,動起刀杖來,把蔣聰戳死在地,那人便越墙逃走了。老婆央來旺兒對西門慶説了,替他拿帖兒縣裏和縣丞説,差人捉住正犯,問成死罪,抵了蔣聰命。後來,來旺兒哄月娘,只説是小人家媳婦兒,會做針指。月娘使了五兩銀子,兩套衣服,四疋青紅布,幷簪環之類,娶與他為妻。月娘因他叫金蓮,不好稱呼,遂改名蕙蓮。這個老婆屬馬的,小金蓮兩歲,今年二十四歲了。生的黄白淨面,身子兒不肥不瘦,模樣兒不短不長,比金蓮脚還小些兒。性明敏,善機變,會粧飾,龍江虎浪,就是嘲漢子的班頭,壞家風的領袖。若説他底本事,他也曾:

\begin{myquote}
斜倚門兒立,人來側目隨。托腮并咬指,無故整衣裳。坐立隨搖腿,無人曲唱低。開窗推户牖,停針不語時。未言先欲笑,必定與人私。
\end{myquote}

初來時,同衆家人媳婦上竃,還沒甚麽粧飾,猶不作在意裏。後過了一個月有餘,看了玉樓金蓮衆人打扮,他把䯼髻墊的高高的,梳的虚籠籠的頭髮,把水鬢描的長長的,在上邊遞茶遞水,被西門慶睃在眼裏。一日設了條計策,敎來旺兒押了五百兩銀子,往杭州替蔡太師製造慶賀生辰錦綉蟒衣,并家中穿的四季衣服,往囬也有半年期程。約従十一月半頭,搭在旱路車上,起身去了。西門慶安心早晚要調戲他這老婆,不期到此正值孟玉樓生日,月娘和衆堂客在後廳吃酒。西門慶那日在家,沒往那去,月娘吩咐玉簫:「房中另放桌兒,打發酒菜湯飯點心你爹吃。」西門慶因打簾内看見惠蓮身上穿着紅紬對衿襖、紫絹裙子,在席上斟酒,故意問玉簫:「那個穿紅襖的是誰?」玉簫囬道:「是新娶的來旺兒的媳婦子惠蓮。」西門慶道:「這媳婦子怎的紅襖配着紫裙子?怪模怪樣。到明日對你娘説,另與他一條别的顔色裙子,配着穿。」玉簫道:「這紫裙子還是問我借的裙子。」説了就罷了。

湏臾,過了玉樓生日。一日,月娘往對門喬大户家吃生日酒去了。約後晌時分,西門慶従外來家,已有酒了;走到儀門首,這惠蓮正往外走,兩個撞個滿懷。西門慶便一手摟過脖子來,就親了個嘴,口中喃喃呐呐説道:「我的兒,你若依了我,頭面衣服隨你揀着用!」那老婆一聲兒沒言語,推開西門慶手,一直往前走了。西門慶歸到上房,叫玉簫送了一疋藍緞子到他屋裏,如此這般對他説:「爹昨日見你酒席上斟酒,穿着紅襖,配着紫裙子,怪模怪樣的不好看。我説這紫裙子還是問我借的,爹纔開廚櫃拿了這疋緞子,使我送與你,敎你做裙子穿。」這惠蓮開看,却是一疋翠藍四季團花兼喜相逢緞子。説道:「我做出來,娘若見了問怎了?」玉簫道:「爹到明日還對娘説,你放心。爹説來,你若依了這件事,隨你要甚麽,爹與你買。今日趕娘不在家,要和你會會兒,你心下何如?」那老婆聽了,微笑而不言。因問:「爹多咱時分來?我好在屋裏伺候。」玉簫道:「爹説小廝們看着,不好進你這屋裏來的。敎你悄悄往山子底下洞兒裏,那裏無人,堪可一會兒。」老婆道:「只怕五娘六娘知道了,不好意思的。」玉簫道:「三娘和五娘都在六娘屋裏下棋,你去不妨事。」當下約會已定,玉簫走來回西門慶説話。兩個都往山子底下成事,玉簫在門首與他觀風。

却不想金蓮玉樓都在李瓶兒房裏下棋,只見小鸞來請玉樓説:「爹來家了。」三人就散了,玉樓囬後邊去了。金蓮走到房中匀了臉,亦往後邊來。走入儀門,只見小玉立在上房門首。金蓮問:「你爹在屋裏?」小玉搖手兒,往前指。這金蓮就知其意,走到前邊山子角門首,只見玉簫攔着門。金蓮只猜玉簫和西門慶在此私狎,便頂進去。玉簫慌了,説道:「五娘休進去,爹在裏面有勾當哩!」金蓮駡道:「怪狗肉,我又怕你爹了?」不由分説,進入花園裏來,各處尋了一遍。走到藏春塢山子洞兒裏,只見他兩個人在裏面纔了事。老婆聽見有人來,連忙繫上裙子往外走,看見金蓮,把臉通紅了。金蓮問道:「賊臭肉,你在這裏做甚麽?」老婆道:「我來叫畫童兒來。」説着,一溜煙走了。金蓮進來,看見西門慶在裏邊繫褲子,駡道:「賊沒廉耻的貨,你和奴才淫婦大白日裏在這裏端的幹好勾當兒!剛纔我打與那淫婦兩個耳刮子纔好,不想他往外走了。原來你就是畫童兒,他來尋你!你與我實説,和這淫婦偷了幾遭?若不實説。等住回大姐姐來家,看我説不説!我若不把奴才淫婦臉打的脹猪,也不筭。俺們閒的聲喚在這裏,你也來插上一把子,老娘眼裏却放不過!」西門慶笑道:「怪小淫婦兒,悄悄兒罷,休要嚷的人知道。我實對你説,如此這般,連今日纔一遭。」金蓮道:「一遭二遭,我不信。你既要這奴才淫婦,兩個瞞神唬鬼弄剌子兒,我打聽出來休怪了,我却和你們答話!」那西門慶笑的出去了。金蓮到後邊,聽見衆丫頭們説:「爹來家,使玉簫手巾裹着一疋藍緞子,往前邊去,不知與誰。」金蓮就知是與來旺兒媳婦子的,對玉樓亦不提起此事。

這老婆每日在那邊,或替他造湯飯,或替他做針指鞋脚,或跟着李瓶兒下棋,常賊乖趨附金蓮。被西門慶撞在一處,無人,敎他兩個苟合,圖漢子喜歡。惠蓮自従和西門慶私通之後,背地不算與他衣服、汗巾、首飾、香茶之類,只銀子成兩家帶在身邊,在門首買花翠胭粉,漸漸顯露,打扮的比往日不同。西門慶又對月娘説他做的好湯水,不敎他上大竃,只敎他和玉簫兩個,在月娘房裏後邊小竃上,專炖茶水,整理菜蔬,打發月娘房裏吃飯,與月娘做針指,不必細説。看官聽説:凡家主,切不可與奴僕并家人之婦苟且私狎,久後必紊亂上下,竊弄奸欺,敗壞風俗,殆不可制!有詩為證:

\begin{myquote}
西門貪色失尊卑,羣妾爭妍竟莫疑。

何事月娘欺不在,暗通僕婦亂倫彝!
\end{myquote}

一日,臘月初八日,西門慶早起,約下應伯爵,與大街坊尚推官家送殯。敎小廝馬也備下兩疋,等伯爵白不見到。一囬,李銘來了,教春梅等四人彈唱。西門慶正在大廳上圍爐坐的,教春梅、玉簫、蘭香、迎春,一般兒四個都打扮出來,看着李銘指撥,教演他彈唱。女婿陳經濟,在傍陪着説話。正唱〔三弄梅花〕還未了,只見伯爵來,應寳跟着,夾着毡包進門。那春梅等四個就要往後走,被西門慶喝住,説道:「左右是你應二爹,都來見見罷,躲怎的?」與伯爵兩個相見作揖,纔待坐下,西門慶令四個過來:「與應二爹磕頭。」那春梅等朝上磕頭下去,慌的伯爵還喏不迭,誇道:「誰似哥好有福,出落的恁四個好姐姐,水葱兒的一般,一個賽一個。却怎生好?你應二爹今日素手,促忙促急,沒曾帶的甚麽在身邊,改日送脂粉錢來罷。」少頃,春梅等四人見了禮進去了。陳經濟向前作揖,一同坐下。西門慶道:「你如何今日這咱纔來?」應伯爵道:「不好告訴你的。大小女病了一向,近日纔敎好些;房下記掛着,今日接了他家來散心住兩日。亂着,旋叫應寳叫了轎子,買了些東西在家,我纔來了。遲了一步兒!」西門慶道:「敎我只顧等着你。咱吃了粥,好去了。」隨即一面吩咐小廝,後邊看粥來吃。只見李銘見伯爵,打個半跪。伯爵道:「李日新,一向不見你。」李銘道:「小的有。連日小的在北邊徐公公那裏答應,這兩日來爹宅裏伺候。」説着,兩個小廝放桌兒,拿粥來吃。就是四個鹹食,十樣小菜兒,四碗炖爛下飯:一碗蹄子,一碗鴿子雛兒,一碗春不老蒸乳餅,一碗餛飩鷄兒。銀鑲甌兒粳米投着各樣榛松栗子果仁、玫瑰白糖粥兒。西門慶陪應伯爵陳經濟吃了,就拿小銀鍾篩金華酒,每人吃了三盃。壺裏還剩下上半壺酒,吩咐小廝畫童兒:「連桌兒擡下去,廂房内與李銘吃。」就穿衣服起身,同應伯爵並馬而行,與尚推官送殯去了。只落下李銘在西廂房,吃畢酒飯。

那月娘房裏玉簫和蘭香衆人打發西門慶出了門,在廂房内亂廝打鬧,頑成一塊。一囬,都往對過東廂房西門大姐房裏鬼混去了,止落下春梅一個,和李銘在這邊教演琵琶。李銘也有酒了。春梅袖口子寬,把手兜住了。李銘把他手拿起,畧按重了些。被春梅怪叫起來,駡道:「好賊王八!你怎的捻我的手,調戲我?賊少死的王八,你還不知道我是誰哩!一日好酒好肉,越發養活的那王八靈聖兒出來了,平白捻我的手來了。賊王八,你錯下這個鍬撅了,你問聲兒去,在我手裏你來弄鬼!爹來家,等我説了,把你這賊王八一條棍攆的離門離户!沒你這王八,學不成唱了?愁本司三院尋不出王八來?撅臭了你這王八了!」被他千王八萬王八,駡的李銘拿着衣服往外,金命水命,走投無命。正是:兩手劈開生死路,翻身跳出是非門。

李銘唬的往外走了,春梅氣狠狠直罵進後邊來。金蓮正和孟玉樓李瓶兒并宋惠蓮在房裏下棋,只聽見春梅従外罵將來,金蓮便問道:「賊小肉兒,你駡誰哩,誰惹你來?」氣的春梅道:「情知是誰,叵耐李銘那王八!爹臨去,好意吩咐小廝,㽞下一桌菜並粳米粥兒與他吃。也有玉簫他們,你推我,我打你,頑成一塊,對着王八雌牙露嘴的,狂的有些摺兒也怎的。頑了一囬,都往大姐那邊廂房裏去了。王八見無人,儘力向我手上捻了一下。吃的醉醉的,看着我嗤嗤待笑。我饒了他!那王八見我吆喝罵起來,他就即夾着衣裳往外走了。剛纔打與賊王八兩個耳刮子纔好!賊王八,你也看個人兒行事,我不是那不三不四的邪皮行貨,教你這王八在我手裏弄鬼。我把王八臉打緑了!」金蓮道:「怪小肉兒,學不學沒要緊,把臉兒氣的黄黄的。等爹來家説了,把賊王八攆了去就是了。那裏緊等着供唱赚錢哩也怎的,教王八調戲我這丫頭!我知道賊王八業罐子滿了。」春梅道:「他就倒運,着量二娘的兄弟,那怕他二娘莫不挾仇打我五棍兒也怎的?」宋惠蓮道:「論起來,你是楽工,在人家教唱,也不該調戲良人家女子!照顧你一個錢,也是養身父母;休説一日三茶六飯兒扶持着。」金蓮道:「扶持着,臨了還要錢兒去了。按月兒,一個月與他五兩銀子。賊王八他錯上了墳。你問聲家裏這些小廝們,那個敢望着他雌牙笑一笑兒,吊個嘴兒,遇喜歡,駡兩句;若不喜歡,拉到他主子跟前就是打,着緊把他爹扛的眼直直的。看不出他來,賊王八造化低。你惹他生姜,你還没曾經着他辣手!」因向春梅道:「沒見你,你爹去了,你進來便罷了,平白只顧和他在那廂房裏做甚麽?却敎那王八調戲你!」春梅道:「都是玉簫和他們,只顧頑笑成一塊,不肯進來。」玉樓道:「他三個如今還在那屋裏?」春梅道:「都往對過大姐房裏去了。」玉樓道:「等我瞧瞧去。」那玉樓起身去了。良久,李瓶兒亦回房,使綉春叫迎春去。

至晚,西門慶來家,金蓮一五一十,告訴西門慶。西門慶吩咐來興兒,今後休放進李銘來走動;自此送断了路兒,不敢上門。這李銘正是:従前作過事,沒興一齊來。有詩為證:

\begin{myquote}
習教歌妓逞家豪,每日閑庭弄錦槽。

不意李銘遭譴斥,春梅聲價競天高。
\end{myquote}

畢竟未知後來何如,且聽下囬分解。

