\includepdf[pages={91,92},fitpaper=false]{tst.pdf}
\chapter*{第四十六囬 \\元夜遊行遇雪雨 妻妾笑卜龜兒卦}
\addcontentsline{toc}{chapter}{第四十六囬 元夜遊行遇雪雨 妻妾笑卜龜兒卦}
\markboth{{\titlename}卷之五}{第四十六囬 元夜遊行遇雪雨 妻妾笑卜龜兒卦}


\begin{myquote}
帝里元宵,風光好,勝僊島蓬萊。玉塵飛動,車喝綉轂,月照樓臺。

三宫此夕歡諧,金蓮萬盞,撒向天街。迓鼓通宵,華燈競起,五夜齊開。
\end{myquote}

此隻詞兒,是前人所作,單題這元宵景致,人物繁華。且説西門慶那日打發吳月娘衆人,徃吳大妗子家喫酒去了。李智黄四約坐到黄昏時分,就告辭去了。伯爵趕送出去,如此這般告訴:「我已替你二公説了,准在明日,還找五百兩銀子。」那李智黄四向伯爵打了恭又打恭。伯爵復到廂房中,和謝希大還陪西門慶飲酒。

只見李銘掀簾子進來。伯爵看見,便道:「李日新來了。」李銘趴在地下磕頭。西門慶問道:「吳惠怎的不來?」李銘道:「吳惠今日東平府官身也沒去,在家裏害眼。小的叫了王柱來了。」便叫王柱:「進來,與爹磕頭。」那王柱掀簾進入房裏,朝上磕了頭,與李銘站立在旁。伯爵道:「你家桂姐剛纔家去了,你不知道?」李銘道:「小的官身到家洗了洗臉就來了,並不知道。」伯爵向西門慶説:「他兩個怕不的還沒喫飯哩,哥吩咐拏飯與他兩個喫。」書童在旁説:「二爹,叫他等一等,一發和吹打的一答裏喫罷,敢也拏飯去了。」伯爵令書童取過一個託盤來,桌上掉了兩碟下飯,一盤燒羊肉,遞與李銘:「等拏了飯,你們拏兩碗,在這明間喫罷。」説書童兒:「我那儍侄子,常言道:方以類聚,物以羣分。你不知他這行人,故雖是當院出身小優兒,比楽工不同,一概看待也罷了,顯的説你我不幫襯了。」被西門慶向伯爵頭上打了一下,笑罵道:「怪不的你這狗才,行記中人只護行記中人,又知這當差的苦甘!」伯爵道:「儍孩兒,你知道甚麽?你空做子弟一場,連『惜玉憐香』四個字,你還不曉的怎生説!粉頭小優兒如同鮮花兒,你惜憐他,越發有精神。你但折挫他,敢就〔八聲甘州〕『懨懨瘦損』,難以存活!」西門慶笑道:「還是我的兒曉的道理。」那李銘王柱須臾喫了飯。應伯爵叫過來吩咐:「你兩個會唱『雪月風花共裁剪』不會?」李銘道:「此是黄鍾,小的們記的。」於是拏過箏來,王柱彈琵琶,李銘ち箏,頓開喉音唱〔黄鍾·醉花陰〕:

\begin{myquote}
「雪月風花共裁剪,雲雨夢香嬌玉軟。花正好,月初圓,雪壓風顛,人比天涯遠。這些時欲寄断腸篇,爭奈我無邊岸的相思好着我難運轉。」

{\markfont〔喜遷鶯〕}「指滄溟為硯,管城毫健筆如椽。松煙,將泰山作墨研,把萬里青天為錦箋,都做了草聖傳。一會家書,書不盡心事;一會家訴,訴不盡熬煎。」

{\markfont〔出隊子〕}「憶當時初見,見俺風流小業冤,兩心中便結下死生緣。一載間渾如膠漆堅,誰承望半路翻騰,倒做了離恨天。」

{\markfont〔出隊子〕}「二三朝不見,渾如隔了十數年。無一頓茶飯不掛牽,無一刻光陰不唱念,無一個更兒,將他來不夢見。」

{\markfont〔四門子〕}「無一個來人行,將他來不問遍;害的人有似風顛,相識們見了重還勸。不由我記掛在心間,思量的跟前活現,作念的口中黏涎。襟領前,袖兒邊,淚痕湮遍。想従前我和他語在先,那時節嬌小當年。論聰明貫世何曾見?他敢眞誠處有萬千。」

{\markfont〔刮地風〕}「憶咱家為他情無倦,涙江河成眷戀。俺也曾坐並着膝,語並着肩。俺也曾芰荷香效他交頸鴛。俺也曾把手兒行,共枕眠。天也,是我緣薄分淺!」

{\markfont〔水僊子〕}「非干是我自專,只覓的鸞膠續断絃。憶枕上盟言,念神前發願,心堅石也穿。暗暗的禱告青天:若咱家負他前世緣,俏寃家不趁今生願,俺那世裏再團圓。」

{\markfont〔尾聲〕}「囑咐你衷腸莫更變,要相逢則除是動載經年。則你那身去遠莫教心去遠!」
\end{myquote}

説話唱完了,看看晚來。正是:金烏漸漸落西山,玉兔看看上畫闌。佳人款款來傳報,報道月移花影上紗窻。西門慶命收了家伙,使人請傅夥計、韓道國、雲主管、賁四、陳經濟,大門首用一架圍屏圍,安放兩張桌席,懸掛兩盞羊角燈,擺設酒筵,堆集許多春檠菓盒,各樣餚饌。西門慶與伯爵希大都一答上面坐了,夥計主管兩邊打横。大門首兩邊,一邊十二盞金蓮燈。還有一座小煙火,西門慶吩咐等堂客來家時放。先是六個楽工擡銅鑼銅鼓,在大門首吹打,動起楽來。打一囬銅鑼銅鼓,又清吹細楽上來。李銘王柱兩個小優兒,箏琵琶上來彈唱燈詞〔畫眉序〕:「花月滿春城」云云。那街上來徃圍看的人,莫敢仰視。西門慶带忠靖冠,絲絨鶴氅,白綾襖子。玳安與平安兩個,一遞一桶放花兒。兩名排軍,各執攔杆,攔擋閒人,不許向前擁擠。不一時碧天雲靜,一輪皓月東升之時,街上遊人十分熱鬧。但見:

\begin{myquote}
戶戶鳴鑼擊鼓,家家品竹彈絲;遊人隊隊踏歌聲,士女翩翩垂舞袖。鰲山結綵,巍峩百尺矗晴空;鳳禁縟香,縹緲千層籠綺陌。閒庭内外,溶溶寳月光輝;畫閣高低,燦燦花燈照耀。三市六街人鬧熱,鳳城佳節賞元宵。
\end{myquote}

且説後邊春梅迎春玉簫蘭香小玉衆人,見月娘不在,聽見大門首吹打銅鼓彈唱,又放煙火,都打扮着走來,在圍屏背後扒着望外瞧。書童兒和畫童兒兩個在圍屏背後火盆上篩酒。原來玉簫和書童舊有私情,兩個常時戯狎。兩個因按在一處奪瓜子兒嗑,不妨火盆上坐着一錫瓶酒,推倒了,那火烘烘望上騰起來,漰了一地灰。起先那玉簫還只顧嘻笑。被西門慶聽見,使下玳安兒來問:「是誰笑?怎的這等灰起?」那日春梅穿着新白綾襖子,大紅遍地金比甲,正坐在一張椅兒上,看見他兩個推倒了酒,一徑揚聲罵玉簫:「好個怪浪的淫婦!見了漢子就邪的不知怎麽樣兒的了!只當兩個把酒推倒了纔罷了,都還嘻嘻哈哈,不知笑的是甚麽。把火也漰死了,平白落了人恁一頭灰!」那玉簫見他罵起來,唬的不敢言語,徃後走了。慌的書童兒走上去,囬説:「小的火盆上篩酒來,扒倒了錫瓶裏酒了。」那西門慶聽了,更不問其長短,就罷了。

先是那日賁四娘子打聽月娘不在,平昔知道春梅玉簫迎春蘭香四個是西門慶貼身答應,得寵的姐兒,大節下安排下許多菜蔬菓品,使了他女孩兒長兒來,要請他四個去他家裏散心坐坐。衆人領了來見李嬌兒。李嬌兒説:「我燈草拐扙不定,你還請問你爹去!」問雪娥,雪娥一發不敢承攬。看看挨到掌燈已後,賁四娘子又使了長兒來邀。四人蘭香推玉簫,玉簫推迎春,迎春推春梅,要會齊了徃見李嬌兒,轉央和西門慶説,放他去。那春梅坐着紋絲兒也不動,反駡玉簫等:「都是那沒見世面的行貨子!縱沒見酒席,也聞些氣兒來!我就去不成也不到央及他家去。一個個鬼攛揝的也似,不知忙的是甚麽,你教我有半個眼兒看的上!」那迎春玉簫蘭香都穿上衣裳,打扮的齊齊整整出來,又不敢去。這春梅又只顧坐着不動身。書童見賁四嫂又使了長兒來邀,説道:「我破着爹罵兩句也罷,等我上去替姐們禀禀去!」一直走到西門慶身邊,掩口附耳説道:「賁四嫂家大節間要請姐們坐坐。姐教我來禀問爹,去不去?」西門慶聽了,吩咐:「教你姐們收拾去,早些來,家裏沒人。」這書童連忙走下來,説道:「還虧我,到上頭一言就准了。教姐們快收拾去,早些來。」那春梅慢慢纔徃房裏勻施脂粉去了。不一時,四個都一答兒裏出門,書童扯圍屏掩過半邊來,遮着過去。到了賁四家,賁四娘子見了,如同天上落下來的一般,迎進裏間。屋裏頂槅上點着綉毬紗燈,一張桌兒上整齊菜餚,春盛堆滿滿的。趕着春梅叫大姑,迎春叫二姑,玉簫是三姑,蘭香是四姑,都見過禮。又請過韓囬子娘子來相陪,教下人家另是一分菜蔬。當下春梅迎春上坐,玉簫蘭香對席,賁四嫂與韓囬子娘子打横,長兒徃來盪酒拏菜。按下這裏不題。

西門慶因叫過楽工來吩咐:「你們吹了一套『東風料峭』〔好事近〕與我聽。」正値後邊拏上玫瑰元宵來,銀杏匙,衆人拏起來同喫。端的香甜羙味,入口而化,甚應佳節。李銘王柱席前又拏楽器,接着彈唱此詞,端的聲韻悠揚,疾徐合節。道:

\begin{myquote}
「東野翠煙消,喜遇芳天晴曉。惜花心性,春來又起得偏早。教人探取,問東君肯與我春多少?見丫鬟笑語回言道:昨夜海棠開了! 」

{\markfont〔千秋歲〕}「杏花稍間着梨花雪,一點點梅荳青小。流水橋邊,流水橋邊,只聽的賣花人聲聲頻叫。鞦韆外,行人道。我只聽的粉牆内佳人歡笑。笑道春光好!我把這花籃兒旋簇,食壘高挑。」

{\markfont〔越恁好〕}「鬧花深處,滴溜溜的酒旗招。牡丹亭左側,尋女伴鬬百草。翠巍巍的柳條,忒楞楞的曉鶯飛過樹梢;撲簌簌落紅,舞翩翩粉蝶兒飛過畫橋。一年景,四季中,惟有春光好。向花前暢飲,月下歡笑。」

{\markfont〔紅綉鞋〕}「聽一派鳳管鸞簫,見一簇翠圍珠繞。捧玉樽,醉頻倒,歌金縷,舞六幺。任明月上花梢,月上花梢。」

{\markfont〔尾聲〕}「醉敎酩酊眠芳草,高把銀燭花下燒。韶光易老,休把春光虚度了!」
\end{myquote}

這裏彈唱飲酒不題。且説玳安與陳經濟袖着許多花炮,又叫兩個排軍拏着兩個燈籠,竟徃吳大妗子家接月娘。衆人正在明間和吳大妗吳二妗子吳舜臣媳婦兒正飲酒,郁大姐在傍彈唱着。見了陳經濟來,教二舅和姐夫房裏坐:「你大舅今日不在家,衙裏看着造册哩。」一面放桌兒,拏春盛點心酒菜上來陪經濟。玳安走到上邊,對月娘説:「爹使小的來接娘們來了。請娘早些家去。恐晚夕人亂,和姐夫一答兒來了。」月娘因着頭裏惱他,就一聲兒沒言語答他。吳大妗子便叫來定兒:「拏些甚麽兒與玳安兒喫。」來定兒道:「酒肉湯飯,都前頭擺下,和他一處兒喫罷。」吳月娘道:「忙怎的?那裏纔來乍到就與他喫罷。敎他前邊站着,我們就起身。」吳大妗子道:「三姑娘,慌怎的!上門兒怪人家?比來衆姑娘們在俺這裏,大節下姊妹間衆位開懷大坐坐兒。左右家裏有他二娘和他姐在家裏,怕怎的!老早就要家去?是別人家,又是一説。」因叫郁大姐:「你唱個好曲兒伏侍他衆位娘,謝你。」孟玉樓道:「他六娘好不惱他哩!不與他做生日。」郁大姐連忙下席來與李瓶兒磕了四個頭,説道:「自從與五娘做了生日,家去就不好起來。昨日妗奶奶這裏接我去,敎我纔收拾なふ了來。若好時,怎的不與你老人家磕頭!」金蓮道:「郁大姐,你六娘不自在哩!你唱個好的與他聽,他就不惱你了。」那李瓶兒在旁只是笑,不做聲。郁大姐道:「不打緊,拏琵琶過來,等我唱。」大妗子叫吳舜臣媳婦鄭三姐:「你把你二位姑娘和衆位娘的酒兒斟上。這一日還沒上過鍾酒兒。」那郁大姐接琵琶在手,唱〔一江風〕道:

\begin{myquote}
「子時那,這凄涼如何過?羅幃錦帳和衣臥。歹哥哥,你許下我子丑時來,不覺寅時錯!癡心腸等他待如何?抛閃了我。願神靈降與他災和祸。

卯時明,亂挽起乌雲髻,羞對菱花鏡。想多情,穿不的錦綉衣裳,戴不起翡翠珍珠,解不開心頭悶。辰時已過了,巳時不見影。奴家為你憂成病。

午時牌,這相思眞個害,害的我魂不在。想多才,你記的月下星前,誓海盟山,誰把你輕看待?他若是未時來,也把奴愁懷解,申時買個猪頭兒賽。

酉時下,不由人心牽掛,誰説幾句知心話。謊冤家,你在謝館秦樓倚翠偎紅,色膽天來大。戌時點上燈,早晚不見他,亥時去卜個龜兒卦。」
\end{myquote}

正唱着,月娘便道:「怎的這一回子恁涼凄凄的起來?」來安在旁説道:「外邊天寒下雪哩。」孟玉樓道:「姐姐,你身上穿的不單薄?我倒帶了個綿披襖子來了,咱這一囬夜深不冷麽?」月娘道:「既是下雪,叫個小廝,家裏取皮襖來咱們穿。」那來安連忙走下來,對玳安説:「娘吩咐教人家去取娘們皮襖哩。」那玳安便叫琴童兒:「你取去罷,等我在這裏伺候。」那琴童也不問,一直家去了。少頃,月娘想起金蓮的皮襖,因問來安兒:「誰取皮襖去了?」來安道:「琴童取去了。」月娘道:「也不問我就去了。」玉樓道:「剛纔短了一句話。就教他拏俺們的皮襖,他五娘沒皮襖,只取姐姐的來罷。」月娘道:「怎的家中沒有?還有當的人家一件皮襖,取來與六姐穿就是了。」月娘便問:「玳安那奴才怎的不去,卻使這奴才去了?你叫他來。」一面把玳安叫到跟前,喫月娘儘力罵了幾句好的:「好奴才!使你怎的不動?又遣將兒,使了那個奴才去了,也不問我聲兒,三不知就去了。但坐壇遣將兒,怪不的,你做了大官兒,恐怕打動你展翅兒來,就只遣他去!」玳安道:「娘錯怪了小的,頭裏娘吩咐若是敎小的去,小的敢不去?來安下來,只説教一個家裏去。」月娘道:「那來安小奴才敢吩咐你?俺們恁大老婆,還不敢使你哩!如今慣的你這奴才們想有些摺兒也怎的!一來主子烟薰的佛像掛在牆上,有恁施主有恁和尚。你説你恁行動兩頭戳舌,獻勤出尖兒,外合裏應,奸懶貪饞,奸消流水,背地瞞官作弊,幹的那繭兒我不知道?頭裏你家主子沒使你送李桂兒家去,你怎的送他?人拏着毡包,你還劈手奪過去了。留丫頭不留丫頭不在你,使你進來説,你怎的不進來?你便就恁送他,裏頭圖嘴喫去了,卻使別人進來。須知我若罵,只罵那個人了,你還説你不久慣牢成?」玳安道:「這個也沒人,就是畫童兒過的舌。爹見他抱着毡包,敎我:『你送送你桂姨去罷。』使了他進來對娘説,留丫頭不留丫頭不在於小的,小的管他怎的?」月娘大怒,罵道:「賊奴才,還要説嘴哩!我可不這裏閒着,和你犯牙兒哩!你這奴才胳膊倒拗過腿了?我使着不動,耍嘴兒!我就不信,到明日不對他説,把這欺心奴才,打與他個爛羊頭也不算!」吳大妗子道:「玳安兒,還不快替你娘們取皮襖去!他惱了。」又道:「姐姐,你吩咐他拏那裏皮襖與五娘穿?」潘金蓮接過來説道:「姐姐,不要取去,我不穿皮襖。教他家裏捎了我的披襖子來我穿罷。人家當的,知道好也夕也?黄狗皮也似的,穿在身上敎人笑話,也不氣長,久後還贖的去了。」月娘道:「這皮襖纔不是當的,倒是商人李智少十六兩銀子準折的皮襖。當的王招宣府裏那件皮襖,與李嬌兒穿了。」因吩咐玳安:「皮襖在大橱裏,教玉簫尋與你,就把大姐的披襖也带了來。」

那玳安把嘴谷都走出來。陳經濟問道:「你徃那去?」玳安道:「精是攘氣的營生,一遍生活兩遍做!這早晚又徃家裏跑一遭。」徑走到家。西門慶還在大門首喫酒,傅夥計雲主管都去了,還有應伯爵謝希大韓道國賁四衆人喫酒未去。便問玳安:「你娘們來了?」玳安道:「沒來。使小的取皮襖來了。」說畢,便徃後走。

先是琴童到家,上房裏尋玉簫要皮襖。小玉坐在炕上,正沒好氣,説道:「四個淫婦今日都在賁四老婆家喫酒哩,我不知道皮襖放在那裏,徃他家問他要去。」這琴童一直走到賁四家,且不叫,在窻外悄悄覷聽。只見賁四嫂説道:「大姑和二姑,怎的這半日酒也不上,菜兒也不揀一筯兒?嫌俺小家兒人家整治的不好喫也怎的?」春梅道:「四嫂,俺們酒夠了。」賁四嫂道:「耶嚛!沒的説。怎的這等上門兒怪人家?」又叫韓回子老婆:「你便是我的切隣,就如副東一樣,三姑四姑跟前酒,你也替我勸勸兒,怎的單板着像客一般?」叫長姐:「篩酒來,斟與三姑喫。你四姑鍾兒斟淺些兒罷。」蘭香道:「我自來喫不的。」賁四嫂道:「你姐兒們今日受餓,沒甚麽可口的菜兒管待,休要笑話。今日要叫個先生來唱,與姑娘們下酒,又恐怕爹那裏聽着。淺房淺屋,説不的俺小家兒人家的苦。」説着,琴童兒敲了敲門,衆人都不言語了。半日,只聽長兒問:「是誰?」琴童道:「是我,尋姐説話。」一面開了門,那琴童入來。玉簫便問:「娘來了?」那琴童看着待笑,半日不言語。玉簫道:「怪雌牙兒,因問着你!看雌的那牙,問着不言語。」琴童道:「娘們還在妗子家喫酒哩。見天陰下雪,使我來家取皮襖來,都教包了去哩。」玉簫道:「皮襖在外描金箱子裏不是?叫小玉拏與你。」琴童道:「小玉説教我來問你要。」玉簫道:「你信那小淫婦兒,他不知道也怎的!」春梅道:「你們有皮襖的,都打發與他。俺娘沒皮襖,只我不動身。」蘭香對琴童:「你三娘皮襖問小鸞要。」迎春便向腰裏拏鑰匙與琴童兒:「教綉春開裏間門拏與你。」

那琴童兒走到後邊,上房小玉和玉樓房中小鸞都包了皮襖交與他。正拏着徃外走,遇見玳安,問道:「你來家做甚麽?」玳安道:「你還説哩,為你來了,平白教大娘罵了我一頓好的。又使我來取五娘的皮襖來。」琴童道:「我如今取六娘的皮襖去也。」玳安道:「你取了還在這裏等着我,一答兒裏去。你先去了不打緊,又惹的大娘駡我!」説畢,玳安來到上房,小玉正在炕上籠着爐臺烤火,口中嗑瓜子兒。見了玳安,問道:「原來你也來了?」玳安道:「你又説哩,受了一肚子氣在這裏。」於是把月娘駡他一節,前後訴説一遍:「着琴童取皮襖,嗔我不來,説我遣將兒。因為五娘沒皮襖,又教我來取,説大橱裏有李三準折的一領皮襖,敎拏與他去哩!」小玉道:「玉簫拏了裏間門上鑰匙。都在賁四家喫酒哩,教他來㧱!」玳安道:「琴童徃六娘房裏去取皮襖便來也,敎他叫去。我且歇歇腿兒,烤烤火兒着。」那小玉便讓炕頭兒與他,並肩相挨着向火。小玉道:「壺裏有酒,篩盞子你喫?」玳安道:「可知好哩,若你下顧!」

小玉下來,把壺坐在火上,抽開抽屜,拏了一碟子臘鵝肉,篩酒與他。無人處,兩個就摟着咂舌親嘴。正喫着酒,只見琴童兒進來。玳安讓他喫了一盞子,便使他:「叫玉簫姐來,拏皮襖與五娘穿。」那琴童把毡包放下,走到賁四家呌玉簫。玉簫駡道:「賊囚根子,又來做甚麽?」又不來,遞與鑰匙教小玉開門。那小玉開了裏間房門,取了一把鑰匙,通了半日,白通不開鎖。又問那玉簫,道:「不是那個鑰匙,娘橱裏鑰匙在牀褥子底下哩。」小玉又罵道:「那淫婦釘子釘在人家不來,兩頭來回只敎使我。」甫能開了,橱裏又沒皮襖。琴童兒又徃賁四家問去。來回走的抱怨了:「就死也死三日三夜,以省合氣!又撞着恁瘟死鬼小奶奶兒門,把人魂也走出了。」向玳安道:「你説此回去,又惹的娘罵。不説屋裏鎖,只怪俺們!」走去又對玉簫説:「裏間娘橱裏尋,沒有皮襖。」玉簫想了想,笑道:「我也忘記,在外間大橱裏。」到後邊,又被小玉罵道:「淫婦喫那野漢子搗昏了,皮襖在這裏,卻到處尋。」一面取出來,將皮襖包了,連大姐披襖,都交付與玳安琴童兩個。拏到吳大妗子家,吳月娘又罵道:「賊奴才,你説囬了都不來罷了!」那玳安又不敢言語。琴童道:「娘的皮襖都有了,等着姐又尋這件青鑲皮襖。」於是打開取出來。吳大妗子燈下觀看,説道:「也好一件皮襖,五娘你怎的説他不好?説是黄狗皮,那裏有恁黄狗皮!與我一件穿也罷了。」月娘道:「新新的皮襖兒,只是面前歇胸舊了些兒。到明日従新換兩個遍地金歇胸,穿着就好了。」孟玉樓拏過來,與金蓮戲道:「我兒,你過來,你穿上這黄狗皮,娘與你試試看好不好?」金蓮道:「有本事到明日問漢子要一件穿,也不枉的。平白拾了人家舊皮襖來,披在身上做甚麽?」玉樓戲道:「好個不認業的,人家有這一件皮襖,穿在身念佛。」於是替他穿上,見寬寬大大,潘金蓮纔不言語。

當下吳月娘是貂鼠皮襖,孟玉樓與李瓶兒俱是貂鼠皮襖,都穿在身上,拜辭吳大妗子二妗子起身。月娘與了郁大姐一包二錢銀子。吳銀兒道:「我這裏就辭了妗子列位娘,磕了頭罷。」當下吳大妗子與了一對銀花兒,月娘與李瓶兒每人袖中掏出一兩銀子與他,磕頭謝了。吳大妗子同二妗子鄭三姐都還要送月娘衆人,因見天氣落雪,月娘阻回去了。琴童道:「頭裏下的還是雪,這囬霑在身都是水珠兒,只怕濕了娘們的衣服。問妗子這裏討把傘打了家去。」吳二妗子連忙取了傘來,琴童兒打着。頭裏兩個排軍打着燈籠,一簇男女跟了,走幾條小巷,到大街上。陳經濟路上放了許多花炮,因叫:「銀姐,你家不遠了,俺們送你到家。」月娘便問:「他家在那裏?」經濟道:「這條衚衕内,一直進去,中間一座大門樓,就是他家。」那吳銀兒道:「我這裏就辭了娘們家去。」月娘道:「地下濕,銀姐家去了罷,頭裏已是見過禮了。我還着小廝送你到家。」因叫過玳安:「你送送銀姐家去。」經濟道:「娘,我與玳安兩個去罷。」月娘道:「也罷,姐夫你與他兩個同送他送。」那經濟得不的一聲,同玳安一路送去了。

吳月娘衆人便回家來。潘金蓮路上説:「大姐姐,你原説咱們送他家去,怎的又不去了?」月娘笑道:「你也只是個小孩兒,哄你説着耍子兒,你就信了。麗春院裏,那處是那裏,你我送去!」潘金蓮道:「像人家漢子,在院裏嫖院來,家裏老婆沒曾徃那裏尋去?尋出沒曾打成一鍋粥?」月娘道:「你見來?待他爹到明日徃院裏去,尋他尋試試;倒沒的教人家漢子當粉頭拉了去,看你那兩個眼兒哩!」説着,看看走到東街口上,將近喬大戶門首。只見喬大戶娘子和他外甥媳婦段大姐,在門首站立,遠遠的見月娘這邊一簇男女過來,拉請月娘進去。月娘再三説道:「多謝親家盛情,天晚了,不進去罷!」那喬大戶娘子那裏肯放,説道:「好親家,你怎的上門兒怪人家?」強把月娘衆人拉進去了。客位内掛着燈,擺設酒菓,有兩個女兒彈唱,飲酒不題。

卻説西門慶在家門首,與伯爵衆人飲酒,酒已將闌。先是伯爵與希大二人整喫了一日,頂顙喫不下去。見西門慶在椅子上打盹,趕眼錯把菓碟兒帶減碟倒在袖子裏,都收拾了個淨光,和韓道國就走了。只落下賁四,又不敢徃屋裏去,直陪着。西門慶打發了楽工酒來喫了,各都與了賞錢,打發出門。看着收了家伙,滅息了燈燭,歸後邊去了。只見平安走來賁四家叫道:「姐們還不起身?爹進去了。」那春梅聽見,和迎春玉簫等慌的徃回跑,不顧辭了賁四嫂辭的,一溜煙跑了。只落下蘭香在後邊了,別了鞋趕不上,駡道:「你們都搶棺材奔命哩!把人的鞋都別了,白穿不上。」到後邊,打聽西門慶在李嬌兒房裏,都來磕頭。大師父見西門慶進入李嬌兒房中,都躱到上房,和小玉在一處。玉簫進來道了萬福。那小玉還説玉簫:「娘那裏使了小廝來要皮襖,你就不來管管兒?教我來拏,我又不知那根鑰匙開橱門,甫能開了又沒有,落後卻在外邊大橱櫃裏尋出來。你放在裏頭,又搗昏了你不知道?姐姐們都喫夠來了罷,也不曾見長出塊兒來。」那玉簫倒喫搶的臉飛紅,便道:「怪小淫婦兒,如何狗撾了臉似的,人家不請你,怎的和俺們使性兒?」小玉道:「我稀罕那淫婦請!」大師父在傍勸道説:「姐姐們義讓一句兒罷,你爹在屋裏聽着。只怕你娘們來家,炖下些茶兒伺候着。」正説着,只見琴童抱進毡包來。玉簫便問:「娘來了?」琴童道:「娘們來了,又被喬親家娘在門首讓進去喫酒哩!也將好起身。」兩個纔不言語了。

不一時,月娘等従喬大戶娘子家出來。到家門首,賁四娘子走出來廝見。陳經濟和賁四一面取出一架小煙火來,在門首又看放了一回煙火,方纔進來。衆人與李嬌兒大師父道了萬福。雪娥走來,向月娘跟前磕了頭,與玉樓等三人見了禮。月娘因問:「他爹在那裏?」李嬌兒道:「剛纔在我那屋裏,我打發他睡了。」月娘一聲兒沒言語。只見春梅迎春玉簫蘭香進來磕頭。李嬌兒便説:「今日前邊賁四嫂請了四個出去,坐了囬兒就來了。」月娘聽了,半日没言語,駡道:「恁成精狗肉們,平白去做甚麽!誰教他去來?」李嬌兒道:「問過他爹纔去來。」月娘道:「問他好有張主的貨!你家初一十五開的廟門早了,都放出些小鬼來了!」大師父道:「我的奶奶,恁四個上畫兒的姐姐,還説是小鬼?」月娘道:「上畫兒只畫的半邊兒!平白放出做甚麽,與人家喂眼兒?」孟玉樓見月娘説話來的不好,就先走了。落後金蓮見玉樓起身,和李瓶兒大姐也走了。止落下大師父和月娘同在一處睡了。那雪霰直下到四更方止。正是:香消燭冷樓臺夜,挑菜燒燈掃雪天。一宿晚景題過。

到次日,西門慶徃衙門中去了。月娘約飯時前後,與孟玉樓李瓶兒三個,同送大師父家去。因在大門裏首站立,看見一個鄉裏卜龜兒卦兒的老婆子,穿着水合襖、藍布裙子,勒黑包頭,背着搭褳,正従街上走來。月娘使小廝叫進來,在二門裏鋪下卦帖,安下靈龜,説道:「你卜卜俺們。」那老婆趴在地下磕了四個頭:「請問奶奶多大年紀?」月娘道:「你卜個屬龍兒的女命。」那老婆道:「若是大龍兒四十二歲,小龍兒三十歲。」月娘道:「是三十歲了,八月十五日子時生。」那老婆把靈龜一擲,轉了一遭兒,住了。揭起頭一張卦帖兒,上面畫着一個官人,和一位娘子在上面坐;其餘都是侍従人,也有坐的,也有立的,守着一庫金銀財寳。老婆道:「這位當家的奶奶是戊辰生。戊辰己巳大林木,為人一生有仁義。性格寬洪,心慈好善,看經佈施,廣行方便。一生操持把家做活,替人頂缸受氣,還不道是。喜怒有常,主下人不足。正是喜楽起來笑嘻嘻,惱將起來鬧哄哄。别人睡到日頭半天還未起,你人早在堂前禁轉梅香洗銚鐺。雖是一時風火性,轉眼卻無心,就和人説也有笑也有。只是這疾厄宫上着刑星,常霑些啾唧。喫了你這心好,濟過來了。徃後有七十歲活哩。」孟玉樓道:「你看這位奶奶,命中有子沒有?」婆子道:「休怪婆子説。兒女宫上有些貴,徃後只好招個出家的兒子送老罷了;不論隨你多少,也存不的。」玉樓向李瓶兒笑道:「就是你家吳應元,現做道士寄名哩。」月娘指着玉樓:「你也叫他卜卜。」玉樓道:「你卜個三十四歲的女命,十一月二十七日寅時生。」那婆子従新撇了卦帖,把靈龜一卜,轉到命宫上住了。揭起第二張卦帖來,上面畫着一個女人,配着三個男人,頭一個小帽商旅打扮,第二個穿紅官人,第三個是個秀才。也守着一庫金銀,有左右侍従人伏侍。婆子道:「這位奶奶是甲子年生。甲子乙丑海中金,命犯三刑六害,夫主尅過方可。」玉樓道:「已尅過了。」婆子道:「你為人溫柔和氣,好個性兒。你惱那個人也不知,喜歡那個人也不知,顯不出來。一生上人見喜下欽敬,為夫主寵愛。只一件,你饒與人為了羙,多不得人心。命中一生替人頂缸受氣,小人駁雜,饒喫了還不道你是。你心地好,囗了去了;雖有小人,也拱不動你。」玉樓笑道:「剛纔為小廝討銀子,和爹亂了這囬子。亂將出來,是我喫了?確是頂缸受氣。」月娘道:「你看這位奶奶,徃後有子沒有?」婆子道:「濟得好,見個女兒罷了,子上不敢許。若説壽,倒儘有。」月娘道:「你卜卜這位奶奶。李大姐,你與他八字兒。」李瓶兒笑道:「我是屬羊的。」婆子道:「若屬小羊的,今年廿七歲,辛未年生的。生幾月?」李瓶兒道:「正月十五日午時。」那婆子卜轉龜兒,到命宫上矻磴住了。揭起卦帖來,上面畫着一個娘子,三個官人。頭個官人穿紅,第二個官人穿綠,第三個穿青。懷着個孩兒,守着一庫金銀財寳,傍邊立着個青臉撩牙紅髮的鬼。婆子道:「這位奶奶,庚午辛未路傍土,一生榮華富貴,喫也有,穿也有。所招的夫主都是貴人。為人心地有仁義,金銀財帛不計較。人喫了賺了他的,他喜歡;不喫他不賺他倒惱。只是喫了比肩不和的虧,凡事恩將仇報。正是:比肩刑害亂擾擾,轉眼無情就放刁。寜逢虎生三张嘴,休遇人前兩面刀。奶奶你休怪我説,你儘好疋紅羅,只可惜尺頭短了些,氣惱上要忍耐些,就是子上也難為。」李瓶兒道:「今已是寄名,做了道士。」婆子道:「旣出了家,無妨了。又一件,你老人家今年計都星照命,主有血光之災。仔細七八月,不見哭聲纔好。」説畢,李瓶兒袖中掏出五分一塊銀子,月娘和玉樓每人與錢五十文。

剛打發卜龜卦婆子去了,只見潘金蓮和大姐従後邊出來,笑道:「我説後邊不見,原來你們都徃前頭來了。」月娘道:「俺們剛纔送大師父出來,卜了這回龜兒卦。你早來一步,也教他與你卜卜兒也罷了。」金蓮搖頭兒道:「我是不卜他。常言:算的着命,算不着好。想着前日道士打看,説我短命哩、怎的哩?説的人心裏影影的。隨他,明日街死街埋,路死路埋,倒在洋溝裏就是棺材。」説畢,和月娘同歸後邊去了。正是:萬事不由人計較,一生都是命安排。有詩為證:

\begin{myquote}
甘羅發早子牙遲,彭祖顔回壽不齊;

范丹家貧石崇富,算來各是只爭時。
\end{myquote}

畢竟未知後來何如,且聽下囬分解。

