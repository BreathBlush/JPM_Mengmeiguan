\includepdf[pages={17,18},fitpaper=false]{tst.pdf}
\chapter*{第九囬 \\西門慶計娶潘金蓮 武都頭誤打李外傳}
\addcontentsline{toc}{chapter}{第九囬 西門慶計娶潘金蓮 武都頭誤打李外傳}
\markboth{{\titlename}卷之一}{第九囬 西門慶計娶潘金蓮 武都頭誤打李外傳}


\begin{myquote}
色膽如天不自由,情深意密兩綢繆。

只思當日同歡愛,豈想蕭牆有後憂。

只貪快樂恣悠遊,英雄壯士報寃仇。

天公自有安排䖏,勝負輸贏卒未休。
\end{myquote}

話說西門慶與潘金蓮燒了武大靈,換了一身艷色衣服,晚夕安排了一席酒,請王婆來作辭,就把迎兒交付與王婆養活。吩咐等武二囬來,只說大娘子度日不過,他娘教他前去,嫁了外京客人去了。婦人箱籠,早先一日都打發過西門慶家去,剩下些破桌、壞凳、舊衣裳,都與了王婆。西門慶又將一兩銀子相謝。到次日,一頂轎子,四個燈籠,王婆送親,玳安跟轎,把婦人擡到家中來。那條街上,遠近人家,無有一人不知此事,都懼怕西門慶是個刁徒潑皮,有錢有勢,誰敢來多管,地街上編了四句口號,說得極好:
\begin{myquote}
「堪笑西門不識羞,先奸後娶醜名留。

轎内坐着浪淫婦,後邊跟着老牽頭。」
\end{myquote}

西門慶娶婦人到家,收拾花園内樓下三間與他做房。一個獨獨小院,角門進去,設放花草盆景。白日間人跡罕到極是一個幽僻去處。一邊是外房,一邊是臥房。西門慶旋用十六兩銀子,買了一張黑漆歡門描金牀,大紅羅圈金帳幔,寳象花揀粧,桌椅錦杌,擺設齊整。大娘子吴月娘房裏使着兩個丫頭,一名春梅,一名玉簫。西門慶把春梅叫到金蓮房内,令他伏侍金蓮,趕着呌娘。却用五兩銀子,另買一個小丫頭,名喚小玉,伏侍月娘。又替金蓮六兩銀子買了一個上灶丫頭,名喚秋菊。排行金蓮做第五房。先頭陳家娘子陪牀的,名喚孫雪娥,約二十年紀,生的五短身材,有姿色。西門慶與他帶了䯼髻,排行第四;以此把金蓮做個第五房。此事表過不題。

這婦人一娶過門來,西門慶家中大小都不歡喜。看官聽說:世上婦人,眼裏火的極多,隨你甚賢慧婦人,男子漢娶小,說不嗔,及到其間,見漢子往他房裏同牀共枕歡樂去了,雖故性兒好煞,也有幾分臉酸心窄。正是:可惜團圝今夜月,清光咫尺别人圓。

西門慶當下就在婦人房中宿歇,如魚似水,羙愛無加。到第二日,婦人梳粧打扮,穿一套豔色衣服,春梅捧茶,走來後邊大娘子吴月娘房裏,拜見大小,遞見面鞋脚。月娘在坐上仔細定睛觀看,這婦人年紀不上二十五六,生的這樣標致。但見:
\begin{myquote}
眉似初春柳葉,常含着雨恨雲愁;臉如三月桃花,暗帶着風情月意。纖腰嬝娜,拘束的燕嬾鶯慵;檀口輕盈,勾引得蜂狂蝶亂。玉貌妖嬈花解語,芳容窈窕玉生香。
\end{myquote}

吴月娘從頭看到脚,風流往下跑;從脚看到頭,風流往上流。論風流,如水晶盤内走明珠;語態度,似紅杏枝頭籠曉日。看了一囬,口中不言,心内暗道:「小廝們家來,只說武大怎樣一個老婆,不曾看見;今日見了,果然生的標緻,怪不的俺那強人愛他。」金蓮先與月娘磕了頭,遞了鞋脚;月娘受了他四禮。次後李嬌兒、孟玉樓、孫雪娥,都拜見,平叙了姊妹之禮,立在傍邊。月娘教丫頭拿個坐兒教他坐。吩咐丫頭媳婦趕着他叫五娘。這婦人坐在傍邊,不轉睛把眼兒只看吴月娘:約三九年紀,——因是八月十五日生的,故小字叫做月娘。——生的面若銀盆,眼如杏子,擧止溫柔,持重寡言。第二個李嬌兒,乃院中唱的,生的肌膚豐肥,身體沉重,人前多咳嗽,上牀懶追陪;雖數名妓者之稱,而風月多不及金蓮也。第三個就是新娶的孟玉樓,約三十年紀,生得貌若梨花,腰如楊柳;長挑身材,瓜子臉兒,稀稀多幾點微麻,自是天然俏麗。惟裙下雙彎,與金蓮無大小之分。第四個孫雪娥,乃房裏出身,五短身材,輕盈體態;能造五鮮湯水,善舞翠盤之妙。這婦人一抹兒都看到在心裏。過三日之後,每日清晨起來,就來房裏與月娘做針指、做鞋脚。凡事不拿強拿,不動強動。指着丫頭趕着月娘一口一聲只叫大娘。快把小意兒貼戀幾次,把月娘喜歡的沒入脚䖏,稱呼他做六姐。衣服首飾揀心愛的與他,喫飯喫茶和他同桌兒一䖏喫。因此,李嬌兒等衆人見月娘錯敬他,各人都不做喜歡,說:「俺們是舊人,倒不理論!他來了多少時,便這等慣了他?大姐姐好沒分曉。」正是:
\begin{myquote}
前車倒了千千輛,後車到了亦如然。

分明指與平川路,错把忠言當惡言。
\end{myquote}

且説西門慶娶潘金蓮來家,住着深宅大院,衣服頭面又相趁,二人女貌郎才,正在妙年之際;凡事如膠似漆,百依百隨,淫慾之事,無日無之。按下這裏不題。

單表武松,八月初旬到了清河縣,且去縣裏交納了囬書。知縣看了大喜,已知金銀寳物交得明白,賞了武松十兩銀子,酒食管待他,不必細說。武松囬到下處,房裏換了衣服鞋脚,帶上一頂新頭巾,鎖了房門,一逕投紫石街來。兩邊衆鄰舍看見武松囬來,都喫一驚,揑兩把汗,說道:「這番蕭牆祸起了!這個太歲歸來,怎肯干休?必然弄出事來!」武松走到哥哥門前,揭起簾子,探身入來,看見迎兒小女在樓穿廊下攆線。說道:「我莫不眼花了?」叫聲嫂嫂,也不應;叫聲哥哥,也不應。道:「我莫不耳聾了!如何不見我哥嫂聲音?」向前便問迎兒小女。那迎兒小女見他叔叔來,唬的不敢言語。武松道:「你爹娘往那裏去了?」迎兒只是哭,不做聲。

正問着,隔壁王婆聽得是武二歸來,生怕决撒了,只得走過幫着迎兒支吾。武二見王婆過來,唱了個喏,問道:「我哥哥往那裏去了!嫂嫂也怎的不見?」那婆子道:「二哥請坐,聽我告訴:你哥哥自從你去了,到四月間得個拙病,死了。」武二道:「我哥哥四月幾時死了?得什麽病?吃誰的藥來?」王婆道:「你哥哥四月二十頭,猛可地害急心疼起來;病了八九日,求神問卜,什麽藥吃不到?醫治不好,死了。」武二道:「我的哥哥從來不曾有這病,如何心疼便死了?」王婆道:「都頭,却怎的這般說!天有不測風雲,人有旦夕祸福。今早脱下鞋和襪,未審明朝穿不穿。誰人保得常沒事!」武二道:「我哥哥如今埋在那裏?」王婆道:「你哥哥一倒了頭,家中一文錢也沒有,大娘子又是沒脚蟹,那裏去尋墳地放着?虧他左近一個財主,前與大郎有一面之交,捨助一具棺木,沒奈何,放了三日,擡出去一把火燒了。」武二道:「今嫂嫂往那裏去了?」婆子道:「他少女嫩婦的,又沒的養贍過日子。胡亂守了百日孝,他娘勸導,前月他嫁了外京人去了。丟下這個業障丫頭子,教我替他養活,專等你囬來交付與你,也了我一場事。」

武二聽言,沉吟了半晌,便撇下了王婆出門去,逕投縣前下䖏去。開了門,去門房裏換了一身素凈衣服。便教土兵街上打了一條麻縧,買了一雙綿鞋,一頂孝帽,帶在頭上。又買了些菓品、點心、香燭、冥紙、金銀錠之類,歸到哥哥家,従新安設武大郎靈位,安排羹飯。就在桌子上點起燈燭,鋪設酒肴,掛起經幡紙繒。那消兩個時辰,安排得端正。約一更已後,武二拈了香,撲翻身便拜道:「哥哥陰魂不遠!你在世時,為人軟弱;今日死後,不見分明。你若是負屈啣冤,被人害了,托夢與我,兄弟替你報寃雪恨!」把酒一面澆奠了,燒化冥紙,武二便放聲大哭。倒還是一路上來的人,哭的那兩家鄰舍,無不悽惶。武二哭罷,將這羹飯酒肴,和土兵迎兒喫了。討兩條蓆子,教土兵房中傍邊睡,武二把迎兒房中睡;他便把條蓆子,就武大靈桌子前睡。約莫將半夜時分,武二翻來覆去那裏睡得着?口裏只是長吁氣。那土兵齁齁的,恰似死人一般挺在那裏。武二爬將起來看時,那靈桌子上,琉璃燈半明半滅。武二坐在蓆子上自言自語,口裏說道:「我哥哥生時懦弱,死後卻無分明。」說猶未了,只見那靈桌子下,捲起一陣冷風來。但見:
\begin{myquote}
無形無影,非霧非煙。盤旋似怪風侵骨冷,凛冽如殺氣透ざ寒。昏昏暗暗,靈前燈火失光明;慘慘幽幽,壁上紙錢飛散亂。隱隱遮藏食毒鬼,紛紛飄逐影魂幡。
\end{myquote}

那陣冷風,逼得武二毛髮皆竪起來。定睛看時,見一個人従靈桌底下鑽將出來,叫聲:「兄弟,我死得好苦也!」武二看不仔細,卻待向前再問時,只見冷氣散了,不見了人。武二一跤跌翻在蓆子上坐的,尋思道:「怪哉!是夢?非夢?剛纔我哥哥正要報我知道,又被我的神氣衝散了他的魂。想來他這一死,必然不明。」聽那更鼓,正打三更三點;囬頭看那土兵,正睡得好。於是咄咄不樂,「等到天明,卻再理會。」胡亂盹了一囬,看看五更雞叫,東方將明,土兵起來燒湯。武二洗漱了,喚起迎兒看家,帶領土兵出了門,在街上訪問街坊鄰舍:「我哥哥怎的死了?嫂嫂嫁得何人去了?」那街坊鄰舍,明知此事,都懼怕西門慶,誰肯來管?只說:「都頭不消訪問,王婆在緊隔壁住,只問王婆就知了。」有那多口的說:「賣梨的鄆哥兒與仵作何九二人,最知詳細。」

這武二竟走來街坊前去尋鄆哥,只見那小猴子手裏拿着個柳籠簸羅兒,正糴米囬來。武二便叫:「鄆哥兄弟!」唱喏。那小廝見是武二叫他,便道:「武都頭,你來遲了一步兒,須動不得手!只是一件,我的老爹六十歲,沒人養贍,我却難伴你們打官司耍子。」武二道:「好兄弟,跟我來。」引他到一個飯店樓上,武二叫過賣:「造兩份飯來。」武二對鄆哥道:「兄弟,你雖年幼,倒有養家孝順之心。我沒甚麽……」向身邊摸出五兩碎銀子,遞與鄆哥道:「你且拿去,與老爹做盤費,我自有用你處。待事務畢了,我再與你十來兩銀子做本錢。你可備細說與我,哥哥和甚人合氣?被甚人謀害了?家中嫂嫂被那一個娶去?你一一說來,休要隱匿!」這鄆哥一手接過銀子,自心裏想道:「這五兩銀子,老爹也夠盤費得三五個月,便陪他打官司也不妨。」一面說道:「武二哥,你聽我說。只怕說與你——休氣苦!」於是把賣梨兒尋西門慶,後被王婆怎地打他,不放進去,又怎的幫扶武大捉姦,西門慶怎的踢中了武大,心疼了幾日,不知怎的死了,従頭至尾,诉說了一遍。武二聽了,便道:「你這話是實麽?」又問道:「我的嫂子嫁與甚麽人去了?」鄆哥道:「你嫂子乞西門慶擡到家,待搗掉底子兒,自還問他實也是虚!」武二道:「你休說謊。」鄆哥道:「我便官府面前,也只是這般說!」武二道:「兄弟,既然如此,討飯來喫。」須臾,大盤大碗喫了飯。武二還了飯錢,兩個下樓來。吩咐鄆哥:「你囬家把盤費交與你老爹,明日早來縣前與我證一證。」又問:「何九在那裏居住?」鄆哥道:「你這時候尋何九?你未曾來時,三日前走的不知往那裏去了。」這武二放了鄆哥家去。

到第二日,武二早起,先在陳先生家寫了狀子,走到縣門前,只見鄆哥在此伺候,一直帶到廳上跪下,聲寃起來。知縣看見,認的是武松,便問:「你告什麽?因何聲寃?」武二告道:「小人哥哥武大,被豪惡西門慶與嫂潘氏通奸,踢中心窝,王婆主謀,陷害性命。何九朦朧入殮,燒毀屍傷,現今西門慶霸占嫂在家為妾。現有這個小廝鄆哥是證見,望相公作主則個!」因遞上狀子。知縣接着,便問:「何九怎的不見?」武二道:「何九知情在逃,不知去向。」知縣於是摘問了鄆哥口詞,當下退廳,與佐貳官吏通同商議。原來知縣、縣丞、主簿、吏典,上下都是與西門慶有首尾的,因此官吏通同計較,這件事難以問理。知縣出來,便叫武松道:「你也是個本縣中都頭,不省得灋度?自古捉姦見雙、捉賊見贜、殺人見傷。你那哥哥屍首又没了,又不曾捉得他姦。如今只憑這小廝口内言語,便問他殺人的公事,莫非公道忒偏向麽?你不可造次,須要自己尋思!當行即行,當止即止。」武二道:「告稟相公,這都是實情,不是小人捏造出來的。」知縣道:「你且起來,待我従長計議。可行時便與你拿人。」武二方纔起來,走出外邊,把鄆哥留在裏面,不放囬家。

早有人把這件事報與西門慶得知,說武二囬來,帶領鄆哥告狀一節。西門慶慌了,即使心腹家人來保來旺,身邊袖着銀兩,打點官吏,都買囑了。到次日早晨,武二在廳上,正告禀知縣催逼拿人。誰想這官人貪圖賄賂,囬下狀子來,說道:「武二,你休聽外人挑撥,和西門慶做對頭。這件事欠明白,難以問理。聖人云:經目之事,猶恐未真;背後之言,豈能全信?你不可一時造次。」當該吏典在旁便道:「都頭,你在衙門裏也曉得法律,但凡人命之事,須要屍傷病物踪五件事俱完,方可推問。你那哥哥屍首又没了,怎生問理?」武二道:「既然相公不准所告,且却再理會。」收了狀子下廳來。來到下處,放了鄆哥歸家,不覺仰天長歎一聲,咬牙切齒,口中駡淫婦不絶。

這漢子怎消洋這一口氣?一直奔到西門慶生薬店前,要尋西門慶廝打。正見他開舖子的傅夥計在木櫃裏面,見武二狠狠的走來聲喏,問道:「大官人在宅上麽?」傅夥計認的是武二,便道:「不在家了。都頭有甚話說?」武二道:「且請借一步說話。」傅夥計不敢不出來,被武二引到僻靜巷口說話。武二翻過臉來,用手撮住他衣領,睜圓怪眼,說道:「你要死,却是要活?」傅夥計道:「都頭在上,小人又不曾觸犯了都頭,都頭何故發怒?」武二道:「你若要死,便不要說;若要活時,你對我實說。西門慶那廝,如今在那裏?我個嫂子被他娶了多少日子?一一説來,我便罷休!」那傅夥計是個小膽之人,見武二發作,慌了手脚,說道:「都頭息怒。小人在他家,每月二兩銀子,僱着小人只開舖子,並不知他閑帳。大官人本不在家,剛纔和一相知,往獅子街大酒樓上喫酒去了,小人並不敢說謊。」武二聽了此言,方纔放了手,大扠步雲飛奔到獅子街來,唬的傅夥計半日移脚不動。那武二逕奔到獅子街橋下酒樓前。

且說西門慶正和縣中一個皂隸李外傳,——專一在縣在府綽攬些公事,往來聽聲氣兒賺錢使。若有兩家告狀的,他便賣串兒;或是官吏打點,他便兩下裏打背公。因此縣中起了他個渾名,叫做「裏外賺」。那日見知縣囬出武松狀子,討得這個消息,要來囬報西門慶知道:武二告狀不行。一面西門慶讓他在酒樓上飲酒,把五兩銀子送他。正喫酒在熱鬧䖏,忽然把眼向樓窗下看,只見武松兇神般從橋下直奔酒樓前來,已知此人來意不善,推更衣從樓後窗只一跳,順着房山跳下人家後院内去了。那武二奔到酒樓前,便問酒保:「西門慶在此麽?」那酒保道:「西門大官人和一相識,在樓上喫酒哩。」武二撥步撩衣,飛搶上樓去。只見一個人坐在正面,兩個唱的粉頭坐在兩邊。認的是本縣皂隸李外傳,就知來報信的,心中甚怒,向前便問:「西門慶那裏去了?」那李外傳見是武二,唬的慌了,半日說不出來。被武二一脚把桌子踢倒了,碟兒盞兒都打的粉碎;兩個唱的,也唬得走不動。武二劈面向李外傳打一拳來。李外傳叫聲「阿呀」時,便跳起來立在凳子上,向樓後窗尋出路。被武二雙手提住,隔着樓前窗,倒撞落在當街心裏來,跌得個發昏。下邊酒保見武二行惡,都驚得獃了,誰敢向前?街上兩邊人都住了脚,睜大眼。武二又氣不捨,奔下樓;見那人已跌得半死,直挺挺在地,只把眼動。於是兜襠又是兩脚,嗚呼哀哉,斷氣身亡。衆人道:「都頭,此人不是西門慶,錯打了他。」武二道:「我問他,如何不說?我所以打他。原來不經打,就死了。」那地方保甲,見人死了,又不敢向前捉武二,只得慢慢挨近上來收籠他,那裏肯放鬆。連酒保王鸞,並兩個粉頭包氏牛氏都拴了,竟投縣衙裏來見知縣。此時哄動了獅子街,鬧了清河縣;街上看的人不計其數。都說:西門慶不當死,不知走的那裏去了,却拿這個人來頂缸。正是:張公喫酒李公醉,桑樹上喫刀柳樹上暴。誰人受用,誰人喫官司,有這等事!有詩為證:
\begin{myquote}
英雄雪恨被刑纏,天公何事黑漫漫。

九泉乾死食毒客,深閨笑殺一金蓮。
\end{myquote}

畢竟未知後來如何,且聽下囬分解。

