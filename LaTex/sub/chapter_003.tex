\includepdf[pages={5,6},fitpaper=false]{tst.pdf}
\chapter*{第三囬 \\王婆定十件挨光計 西門慶茶房戲金蓮}
\addcontentsline{toc}{chapter}{第三囬 王婆定十件挨光計 西門慶茶房戲金蓮}
\markboth{{\titlename}卷之一}{第三囬 王婆定十件挨光計 西門慶茶房戲金蓮}


\begin{myquote}
色不迷人人自迷,迷他端的受他虧:

精神耗散容顔淺,骨髓焦枯氣力微。

犯着姦情家易散,染成色病薬難醫。

古来饱煖生閑事,祸到頭來總不知。
\end{myquote}

話説西門慶央王婆,一心要會那雌兒一面,便道:「乾娘,你端的與我說這件事成,我便送十兩銀子與你。」王婆道:「大官人,你聽我說,但凡『挨光』的兩個字最難。——怎的是『挨光』?似如今俗呼『偸情』就是了。——要五件事俱全,方纔行的。第一,要潘安的貌;第二,要驢大行貨;第三,要鄧通般有錢;第四,要粧小伏低,就要綿裏針一般軟款忍耐;第五、要閑工夫。此五件喚做『潘驢鄧小閑』,都全了,此事便獲得着。」西門慶道:「實不瞞你說,這五件事我都有。第一件,我的貌雖比不得潘安,也充得過。第二件,我小時在三街兩巷遊串,也曾養得好大龜。第三,我家裏也有幾貫錢財,雖不及鄧通,也頗得過日子。第四,我最忍耐,他便就打我四百頓,休想我囬他一拳。第五,我最有閑工夫。不然,如何來得恁勤?乾娘,你自作成我,完備了時,我自重重謝你!」西門慶當日意已在言表。王婆道:「大官人,你說五件事都全,我知道還有一件事打攪,也都是成不得!」西門慶道:「且說甚麽一件事打攪?」王婆道:「大官人,休怪老身直言,但凡挨光,最難十分。肯使錢到九分九厘,也有難成處。我知你從來慳吝,不肯胡亂便使錢。只這件打攪。」西門慶道:「這個容易,我只聽你言語便了。」王婆道:「若大官人肯使錢時,老身有一條妙計,須教大官人和這雌兒會一面。只不知大官人肯依我麽?」西門慶道:「不揀怎的,我都依你。端的有甚妙計?」王婆笑道:「今日晚了,且囬去,過半年三個月來商量。」西門慶央及道:「乾娘,你休撒科,自作成我則個!恩有重報。」王婆笑哈哈道:「大官人卻又慌了。老身這條計,雖然入不得武成王廟,端的強似孫武子敎女兵,十捉八九着。大官人,今日實對你說了罷,這個雌兒來歷,雖然微末出身,卻倒百伶百俐,會一手好彈唱。針指女工,百家詞曲,雙陸象棋,無般不知。小名叫做金蓮,娘家姓潘。原是南關外潘裁的女兒,賣在張大戶家學彈唱。後因大戶年老,打發出來,不要武大一文錢,白白與了他為妻。占用這幾年,武大為人軟弱,每日早出晚歸,只做買賣。這雌兒等閑不出來。老身無事,常過去與他閑坐,他有事亦來請我理會。他也叫我做乾娘。武大這兩日出門早。大官人如幹此事,便買一疋藍紬、一疋白紬、一疋白絹,再用十兩好綿,都把來與老身。老身卻走過去,問他借曆日——央及人揀個好日期,叫個裁縫來做送終衣服。他若見我這般來說,揀了日期不肯與我來做時,此事便休了;他若歡天喜地,說『我替你做』,不要我叫裁縫,這光便有一分了。我便請得他來做,就替我裁,這便二分了。他若來做時,午間我卻安排些酒食點心,請他喫。他若說不便當,定要將去家中做,此事便休了;他不言語喫了時,這光便有三分了。這一日你也莫來。直到第三日晌午前後,你整整齊齊打扮了來,以咳嗽為號。你在門前叫道:『怎的連日不見王乾娘?我來買盞茶喫。』我便出來請你入房裏坐,喫茶。他若見你,便起身來走了歸去,——難道我扯住他不成?此事便休了;他若見你入來,不動身時,這光便有四分了。坐下時,我便對雌兒說道:『這個便是與我衣料施主的官人,虧殺他!』我便誇大官人許多好䖏,你便賣弄他針指,若是他不來兜攬答應時,此事便休了;他若口裏答應,與你說話時,這光便有五分。我卻說道:『難為這位娘子,與我作成出手做。虧殺你兩施主,一個出錢,一個出力。不是老身路歧相央,難得這位娘子在這裏,官人做個主人,替娘子澆澆手。』你便取銀子出來,央我買。若是他便走時,——不成我扯住他?此事便休了;若是不動身時,事務易成,這光便有六分了。我卻拿銀子,臨出門時,對他說:『有勞娘子相待官人坐一坐。』他若起身走了家去,——我難道阻擋他?此事便休了;若是他不起身,又好了,這光便有七分了。待我買得東西,擺在桌子上,便說:『娘子,且收拾過生活去,且喫一盃兒酒,難得這官人壞錢。』他不肯和你同桌喫,走了囬去,此事便休了;若是只口裏說要去,卻不動身,此事又好了,這光便有八分了。待他喫得酒濃時,正說得入港,我便推道没了酒,再教你買。你便拿銀子,又央我買酒去,並菓子來配酒。我把門拽上,關你和他兩個在屋裏。若焦躁跑了歸去時,此事便休了;他若由我拽上門,不焦躁時,這光便有九分,只欠一分了便完就。這一分倒難。大官人,你在房裏,便着幾句甜話兒說入去,卻不可躁爆便去動手動脚,打攪了事。那時我不管你。你先把袖子向桌子上拂落一雙筯下去,只推拾筯,將手去他脚上揑一揑。他若鬧將起來,我自來搭救。此事便休了,再也難成。若是他不做聲時,此事十分光了,他必然有意。這十分光做完備,你怎的謝我?」西門慶聽了大喜道:「雖然上不得凌煙閣,乾娘,你這條計端的絶品好妙計!」王婆道:「卻不要忘了許我那十兩銀子!」西門慶道:「便得一片橘皮喫,切莫忘了洞庭湖。這條計,乾娘幾時可行?」王婆道:「亦只今晚來有囬報。我如今趂武大未歸,過去問他借曆日,細細說念他。你快使人送將紬絹綿子來,休要遲了。」西門慶道:「乾娘若完成得這件事,如何敢失信!」於是作别了王婆,離了茶肆,就去街上買了紬絹三疋,並十兩清水好綿。家裏叫了個貼身答應的小廝,名喚玳安,用包袱包了,一直送入王婆家來。王婆歡喜收下,打發小廝囘去。正是:巫山雲雨幾時就?空使襄王築楚臺。有詩為證:

\begin{myquote}
兩意想投似蜜甜,王婆撮合更稀奇。

安排十件挨光計,管取交歡不負期。
\end{myquote}

當下王婆收了紬絹綿子,開了後門,走過武大家來。那婦人接着,請去樓上坐的。王婆道:「娘子怎的這兩日不過貧家喫茶?」那婦人道:「便是我這幾日身子不快,懶去走動。」王婆道:「娘子家裏有曆日,借與老身看一看,要個裁衣的日子。」婦人道:「乾娘裁甚衣服?」王婆道:「便是因老身十病九痛,怕一時有些山高水低,我兒子又不在家。」婦人道:「大哥怎的一向不見?」王婆道:「那廝跟了個客人在外邊,不見個音信囬來,老身日逐躭心不下。」婦人道:「大哥今年多少青春?」王婆道:「那廝十七歲了。」婦人道:「怎的不與他尋個親事?與乾娘也替得手。」王婆道:「因是這等說。家中沒人,待老身東擯西補的來。早晚也替他尋下個兒。等那廝來,卻再理會。現如今老身白日黑夜,只發喘咳嗽,身子打碎般睡不倒的只害疼,一時先要預備下送終衣服。難得一個財主官人,常在貧家喫茶。但凡他宅裏看病、買使女、說親,見老身這般本分,大小事兒無不照顧老身。又布施了老身一套送終衣料,紬絹表裏俱全,又有若干好綿,放在家裏一年有餘,不能夠閑做得。今年覺得好生不濟,不想又撞着閏月,趂着兩日倒閑,要做,又被那裁縫勒掯,只推生活忙不肯來做。老身說不得這苦也!」那婦人聽了,笑道:「只怕奴家做得不中意,若是不嫌時,奴這幾日倒閑,出手與乾娘做如何?」那婆子聽了,堆下笑來,說道:「若得娘子貴手做時,老身便死也得好䖏去。久聞娘子好針指,只是不敢來相央。」那婦人道:「這個何妨?旣是許了乾娘,務要與乾娘做了。將曆日去敎人揀個黄道好日,奴便動手。」王婆道:「娘子,休推老身不知!你詩詞百家曲兒内字樣,你不知會了多少,如何叫人看曆日?」婦人微笑道:「奴家自幼失學。」婆子道:「好說,好說!」便取曆日遞與婦人。婦人接在手内,看了一囬道:「明日是破日,後日也不好。直到外後日,方是裁衣日期。」王婆一把手取過歷頭來,掛在牆上,便道:「若得娘子肯與老身做時,就是一點福星,何用選日!老身也曾央人看來,說明日是個破日。老身只道裁衣日可用破日,不忌他!」那婦人道:「歸壽衣服,正用破日便好。」王婆道:「旣是娘子肯作成,老身膽大,只是明日起動娘子到寒家則個。」那婦人道:「不必,將過來做不得?」王婆道:「便是老身也要看娘子做生活,又怕門首没人。」婦人道:「旣是這等說,奴明日飯後過來。」那婆子千恩萬謝,下樓去了,當晚囬覆了西門慶話,約定後日准來。當夜無話。

次日清晨,王婆收拾房内乾凈,預備下針線,安排了茶水,在家等候。且說武大喫了早飯,挑着擔兒自出去了。那婦人把簾兒掛了,吩咐迎兒看家,従後門走過王婆家來。那婆子歡喜無限,接入房裏坐下,便濃濃點一盞胡桃松子泡茶,與婦人喫了。抹得桌子乾凈,便取出那紬絹三疋來。婦人量了長短,裁得完備,縫將起來。婆子看了,口裏不住聲假喝采道:「好手段!老身也活了六七十歲,眼裏眞個不曾見這個好針線!」那婦人縫到日中,王婆安排些酒食請他,又下了一筯麵與那婦人喫。再縫一歇,將次晚來,便收拾了生活自歸家去。恰好武大挑擔兒進門,婦人拽開門,下了簾子。武大入屋裏,看見老婆面色微紅,問道:「你那裏來?」婦人應道:「便是間壁乾娘,央我做送終衣服。日中安排了些酒食點心,請我喫。」武大道:「你也不要喫他的纔得,我們也有央及他䖏。他便央你做得衣裳,你便自歸來喫些點心,不値得甚麽,便攪擾他?你明日再去做時,带些錢在身邊,也買些酒食與他囬禮。常言道:遠親不如近鄰,休要失了人情!他若不肯敎你還禮時,你便拿了生活來家,做還與他便了。」有詩為證:

\begin{myquote}
阿母牢籠設計深,大郎愚鹵不知音。

帶錢買酒酬奸詐,卻把婆娘白送人。
\end{myquote}

婦人聽了武大言語,當晚無話。次日飯後,武大挑擔兒出去了,王婆便踅過來相請。婦人去到他家房裏,取出生活來,一面縫起。王婆忙點茶來,與他喫了茶。看看縫到日中,那婦人向袖中取出三百文錢來,向王婆說道:「乾娘,奴和你買盞酒喫。」王婆道:「阿呀,那裏有這個道理。老身央及娘子在這裏做生活,如何教娘子倒出錢?婆子的酒食不到的喫傷了哩!」那婦人道:「卻是拙夫吩咐奴來。若是乾娘見外時,只是將了家去,做還乾娘便了。」那婆子聽了道:「大郎直恁地曉事!旣然娘子這般說時,老身且收下。」這婆子生怕打攪了事,自又添錢去買好酒好食、希奇果子來,殷勤相待。看官聽說:但凡世上婦人,由你十分精細,被小意兒過縱,十個九個着了道兒。這婆子安排了酒食點心,請那婦人喫了。再縫了一歇,看看晚來,千恩萬謝歸去了。

話休絮煩。第三日早飯後,王婆只張武大出去了,便走過來後門首,叫道:「娘子,老身大膽!」那婦人從樓上應道:「奴卻待來也。」兩個廝見了,來到王婆房裏坐下,取過生活來縫。那婆子隨即點盞茶來,兩個喫了。婦人看看縫到晌午前後。

卻說西門慶巴不到此日,打選衣帽,齊齊整整,身邊帶着三五兩銀子,手裏拿着洒金川扇兒,搖搖擺擺,逕往紫石街來。到王婆茶坊門首,便咳嗽道:「王乾娘,連日如何不見?」那婆子瞧科,便應道:「兀的誰叫老娘?」西門慶道:「是我。」那婆子趕出來看了,笑道:「我只道是誰,原來是大官人!你來得正好,且請入屋裏去看一看。」把西門慶袖子只一拖,拖進房裏來。看那婦人道:「這個便是與老身衣料施主官人。」西門慶睜眼看着那婦人:雲鬟疊翠,粉面生春;上穿白夏布衫兒,桃紅裙子,藍比甲;正在房裏做衣服。見西門慶過來,便把頭低了。這西門慶連忙向前屈身唱喏。那婦人隨即放下生活,還了萬福。王婆便道:「難得官人與老身緞疋紬絹,放在家一年有餘,不曾做得;虧殺鄰家這位娘子出手與老身做成全了。眞個是布機也似針線,縫的又好又密,眞個難得!大官人,你過來且看一看。」西門慶把起衣服來看了,一面喝采,口裏道:「這位娘子傳得這等好針指,神僊一般的手段!」那婦人笑道:「官人休笑話。」西門慶故問王婆道:「乾娘,不敢動問,這娘子是誰家宅上的娘子?」王婆道:「大官人,你猜。」西門慶道:「小人如何猜得着?」王婆哈哈笑道:「大官人你請坐,我對你說了罷。」那西門慶與婦人對面坐下。那婆子道:「好教大官人得知了罷,大官人你那日屋簷下頭過,打得正好!」西門慶道:「就是那日在門首叉竿打了我網巾的?倒不知是誰宅上娘子。」婦人笑道:「那日奴誤衝撞官人,休怪。」一面立起身來,道了個萬福。那西門慶慌的還禮不迭,因說道:「小人不敢。」王婆道:「就是這位,卻是間壁武大郎的娘子。」西門慶道:「原來就是武大郎的娘子!小人自認的大郎,是個養家經纪人。且是街上做買賣,大大小小不曾惡了一個,又會賺錢,又且好性格,眞個難得這等人!」王婆道:「可知哩。娘子自従嫁了這大郎,但有事,百依百隨。且是合得着!」這婦人道:「拙夫是無用之人,官人休要笑話。」西門慶道:「娘子差矣!古人道:柔軟是立身之本,剛強是惹祸之胎。似娘子的夫主所為良善時,萬丈水無涓滴漏,一生只是志誠為,倒不好?」

王婆一面打着攛鼓兒說,西門慶獎了一囬。王婆因望婦人說道:「娘子,你認得這位官人麽?」婦人道:「不認得。」婆子道:「這位官人,便是本縣裏一個財主,知縣相公也和他來往,呌做西門大官人。家有萬萬貫錢財,在縣門前開生藥舖,家中錢過北斗,米爛陳倉。黄的是金,白的是銀,圓的是珠,光的是寳。也有犀牛頭上角,大象口中牙。又放官吏債,結識人。他家大娘子,也是我說的媒,是吳千戶家小姐,生得百伶百俐。」因問:「大官人,怎的連日不過貧家喫茶?」西門慶道:「便是連日家中小女有人家定了,不得閑來。」婆子道:「大姐有誰家定了?怎的不請老身去說媒?」西門慶道:「被東京八十萬禁軍楊提督親家陳宅合成帖兒。他兒子陳經濟,纔十七歲,還上學堂。不是也請乾娘説媒,他那邊有了個文嫂兒來討帖兒,俺這裏又使常在家中走的賣翠花的薛嫂兒同做保山,說此親事。乾娘若肯去,到明日下小茶,我使人來請你。」婆子哈哈笑道:「老身哄大官人耍子。俺這媒人們都是狗娘養下來的。他們說親時又没我,做成的熟飯兒怎肯搭上老身一份?常言道:當行厭當行。到明日娶過了門時,老身胡亂三朝五日拿上些人情去走走,討得一張半張桌面,倒是正經。怎的好和人鬦氣?」兩個一遞一句,說了一囬。婆子只顧誇獎,西門慶口裏假嘈,那婦人便低了頭縫針線。有詩為證:

\begin{myquote}
水性從來是女流,背夫常與外人偸。

金蓮心愛西門慶,淫蕩春心不自由。
\end{myquote}

西門慶見金蓮十分情意欣喜,恨不得就要成雙。王婆便去點兩盞茶來,遞一盞與西門慶,一盞與婦人,說道:「娘子,相待官人喫些茶。」喫畢,便覺有些眉目送情。王婆看着西門慶,把手在臉上摸一摸,西門慶已知有五分光了。自古風流茶說合,酒是色媒人。王婆便道:「大官人不來,老身也不敢去宅上相請。一者緣灋撞遇,二者來得正好。常言道:一客不煩二主。大官人便是出錢的,這位娘子便是出力的,虧殺你這兩位施主!不是老身路歧相煩,難得這位娘子在這裏,官人好與老身做個主人,拿出些銀子,買些酒食來,與娘子澆澆手,如何?」西門慶道:「小人也見不到這裏!有銀子在此。」便向茄袋裏取出來,約有一兩一塊,遞與王婆子,交備辦酒食。那婦人便道:「不消生受官人。」口裏說着,卻不動身。王婆將銀子,臨出門便道:「有勞娘子相陪大官人坐一坐,我去就來。」那婦人道:「乾娘,免了罷。」卻亦不動身。也是姻緣,都有意了。

王婆便出門去了,丟下西門慶和那婦人在屋裏。這西門慶一雙眼,不轉睛只看着那婦人。那婆娘也把眼來偸睃西門慶,見了他這表人物,心中到有五七分意了。又低着頭,只做生活。不多時,王婆買了現成肥鵝、燒鴨、熟肉、鮮鮓、細巧果子歸來,盡把盤碟盛了,擺在房裏桌子上。看那婦人道:「娘子且收拾過生活,喫一盃兒酒。」那婦人道:「你自陪大官人喫,奴卻不當。」那婆子道:「正是專與娘子澆手,如何卻說這話?」一面將盤饌都擺在面前,三人坐定,把酒來斟。這西門慶拿起酒盞來遞與婦人,說道:「請不棄,滿飲此盃。」婦人謝道:「多承官人厚意,奴家量淺,喫不得。」王婆道:「老身知得娘子洪飲,且請開懷喫兩盞兒。」有詩為證:

\begin{myquote}
從來男女不同筵,賣俏迎奸最可憐。

不獨文君奔司馬,西門今亦遇金蓮。
\end{myquote}

那婦人一面接酒在手,向二人各道了萬福。西門慶拿起筯來,說道:「乾娘,替我勸娘子些菜兒。」那婆子揀好的遞將過來,與婦人喫。一連斟了三巡酒,那婆子便去盪酒來。西門慶道:「小人不敢動問,娘子青春多少?」婦人應道:「奴家虚度二十五歲,屬龍的,正月初九日丑時生。」西門慶道:「娘子倒與家下賤累同庚,也是庚辰,屬龍的,只是娘子月份大七個月,他是八月十五日子時。」婦人道:「將天比地,折殺奴家。」王婆便插口道:「好個精細的娘子,百伶百俐!又不枉做得一手好針線,諸子百家,雙陸象棋,拆白道字皆通,一筆好寫。」西門慶道:「卻是那裏去討!武大郎好有福,招得這位娘子在屋裏。」王婆道:「不是老身說是非,大官人宅上枉有許多,那裏討得一個似娘子的!」西門慶道:「便是這等,一言難盡!只是小人命薄,不曾招得一個好的在家裏。」王婆道:「大官人先頭娘子須也好。」西門慶道:「休說我先妻,若在他在時,卻不恁的家無主,屋倒豎。如今身邊枉自有三五七口人喫飯,都不管事。」那婦人便問:「大官人恁的時沒了大娘子,得幾年了?」西門慶道:「說不得。小人先妻陳氏,雖是微末出身,卻倒百伶百俐,是件都替的小人。如今不幸他没了,已過三年來也。繼娶這個賤累,又常有疾病,不管事。家裏的勾當都七顛八倒。為何小人只是走了出來?在家裏時,便要嘔氣。」婆子道:「大官人休怪我直言!你先頭娘子並如今娘子,也没武大娘子這手針線,這一表人物。」西門慶道:「便是先妻,也没武大娘子這一般兒風流。」那婆子笑道:「官人,你養的外宅,東街上住的,如何不請老身去喫茶?」西門慶道:「便是唱慢曲兒的張惜春!我見他是歧路人,不喜歡。」婆子又道:「官人,你和勾欄中李嬌兒卻長久。」西門慶道:「這個人現今已娶在家裏。若得他會當家時,自册正了他。」王婆道:「與卓二姐卻相交得好?」西門慶道:「卓丢兒我也娶在家,做了第三房。近來得了個細疾,白不得好。」婆子道:「若有似武大娘子這般中官人意的,來宅上說,不妨事麽?」西門慶道:「我的爹娘俱已没了,我自主張,誰敢說個不字?」王婆道:「我自說耍,急切便那裏有這般中官人意的。」西門慶道:「做甚麽便没?只恨我夫妻緣分上薄,自不撞着哩。」

西門慶和婆子一遞一句,說了一囬。王婆道:「正好喫酒,卻又没了。官人休怪老身差撥,再買一瓶兒酒來喫,如何?」西門慶便把茄袋内還有三四兩散銀子都與王婆,說道:「乾娘,你拿了去。要喫時,只顧取來。多的乾娘便就收了。」那婆子謝了官人,起身睃那粉頭時,三鍾酒下肚,烘動春心,又自兩個言來語去,都有意了,——只低了頭,不起身。正是:滿前野意無人識,幾朵碧桃春自開。有詩為證:

\begin{myquote}
眼意眉情卒未休,姻緣相凑遇風流。

王婆貪賄無他技,一味花言巧舌頭。
\end{myquote}

畢竟未知後來如何,且聽下囬分解。

