\includepdf[pages={151,152},fitpaper=false]{tst.pdf}
\chapter*{第七十六囬 孟玉樓解愠吳月娘 西門慶斥逐溫葵軒}
\addcontentsline{toc}{chapter}{第七十六囬 孟玉樓解愠吳月娘 西門慶斥逐溫葵軒}
\markboth{第七十六囬 孟玉樓解愠吳月娘 西門慶斥逐溫葵軒}{第七十六囬 孟玉樓解愠吳月娘 西門慶斥逐溫葵軒}

動靜謀為要三思,莫將煩惱自招之。

人生世上風波險,一日風波十二時。

話説西門慶見月娘半日不出去,又親自進來催促了一遍。見月娘穿衣裳,方纔請進任醫官,到上房明間内坐下。見正面洒金軟壁,兩邊安放春櫈,地平上鋪着毡毯,安放火盆。少頃,月娘従房内出來,五短身材,團面皮兒,黄白淨兒,模樣兒不肥不瘦,身體兒不短不長;兩兩春山月鈎,一雙鳳眼纖長;春笋露甄妃之玉,朱唇點漢署之香,望上道個萬福。慌的任醫官躲在傍邊,屈身還禮。月娘就在對面一椅坐下。琴童安放桌兒綿裀,月娘向袖口邊伸玉腕,露青葱,教任醫官診脉。良久診完,月娘又道個萬福,抽身囘房去了。房中小廝㧱出茶來,喫畢茶,任醫官説道:「老夫人原來禀的氣血弱,尺脉來的又浮澀,雖有胎氣,有些榮衛失調,易生嗔怒,又動了肝火。如今頭目不清,中脘有些阻滯,作惡煩悶;四肢之内,血少而氣多。」月娘使出琴童來說:「娘如今只是有些頭疼心脹,胳膊發麻,肚腹往下墜,背疼腰酸,喫飲食無味。」任醫官道:「我已知道,說得明白了。」西門慶道:「不瞞后溪説,房下如今現懷臨月身孕,因着氣惱,不能運轉,滞在胸膈間。望乞老先生留神加減一二,足見厚情。」任醫官道:「豈勞吩咐,學生無不用心!此去就奉過薬來,清胎、理氣、和中、養榮蠲痛之劑,老夫人服過,要戒氣惱,就厚味也少喫。」西門慶道:「望乞老先生把他這胎氣好生安一安。」任醫官道:「一定安胎理氣,養其榮衛。不勞多囑,學生自有斟酌。」西門慶復説:「學生第三房下有些肚冷,望乞有暖宮丸薬見賜來。」任醫官道:「學生謹領,就封過來。」說畢起身。走到前廳院内,見許多教坊楽工伺候,因問:「老翁今日府上有甚事?」西門慶悉言:「巡按宋公連兩司官员,請巡撫侯石泉老先生,在舍下擺酒。」這任醫官聽了,越發心中駭然尊敬西門慶,在門前揖讓上馬,禮法比尋常不同,倍加敬重。西門慶送他囬來,隨即封了一兩銀子,兩方手帕,即使琴童㧱盒兒騎馬討藥去。

李嬌兒孟玉樓衆人都在月娘屋裏裝定菓盒,搽抹銀器,便説:「大娘,你頭裏還要不出去,怎麽知道你心中如此這般病。」月娘道:「甚麽好成樣的老婆,由他,死便死了罷。可是他説的:不知那淫婦他怎麽的行動管着俺們,你是我婆婆?無故只是大小之分罷了,我還大他八個月哩!漢子疼我,你只好看我一眼兒哩!他不討了他口裏話,他怎麽和我大嚷大鬧?若不是你們攛掇我出去,我後十年也不出去。隨他死教他死去!常言道:一鷄死,一雞鳴,新來鷄兒打鳴忒好聽。我死了,把他立起來,也不亂,也不嚷,纔拔了蘿蔔地皮寬!」玉樓道:「大娘,耶嚛耶嚛!那裏有此話?俺們就代他賭個大誓。這六姐,不是我說他,有的不知好歹,行事兒有些勉強,恰似咬羣出尖兒的一般,一個大有口没心的行貨子。大娘你若惱他,可是錯惱了。」月娘道:「他是比你没心?他一團兒心哩。他怎的會悄悄聽人兒,行動拿話兒譏諷着人說話?」玉樓道:「娘,你是個當家人,惡水缸兒,不恁大量些罷了,卻怎樣兒的?常言:一個君子,待了十個小人。你手放高些,他敢過去了;你若與他一般見識起來,他敢過不去!」月娘道:「只有了漢子與他做主兒,看把那大老婆且打靠後!」玉樓道:「哄那個哩!如今像大娘心裏恁不好,他爹敢往那屋裏去麽?」月娘道:「他怎的不去?可是他説的,他屋裏拿猪毛䋲子套他,不去?一個漢子的心,如同没籠頭的馬一般,他要喜歡那一個,只喜歡那個。誰敢攔他?攔他,又説是浪了!」玉樓道:「罷麽,大娘!你已是説過,通把氣兒納納兒。等我教他來與娘磕頭,賠個不是。趁着他大妗子在這裏,你們兩個笑開了罷。你不然教他爹兩下裏不作難?就行走也不方便。但要往他屋裏去,又怕你惱;若不去,他又不敢出來。今日前邊恁擺酒,俺們都在這定菓盒,忙的了不得,落得他在屋裏這會躲猾兒悄靜兒,俺們也饒不過他。大妗子,我說的是不是?」大妗子道:「姑娘,也罷,他三娘也説的是。不爭你兩個話差,只顧不見面,教他姑夫也難,兩下裏都不好行走的。」那月娘通一聲也不言語。

這孟玉樓抽身就往前走。月娘道:「孟三娘,不要叫他去,隨他來不來罷。」玉樓道:「他不敢不來。若不來,我可拿猪毛䋲子套了他來。」一直走到金蓮房中,見他頭也不梳,把臉黄着,坐在炕上。玉樓說:「六姐,你怎的裝憨兒?把頭梳起來。今日前邊擺酒,後邊恁忙亂,你也進去走走兒,怎的只顧使性兒起來?剛纔如此這般,俺們對大娘說了,勸了他這一囘。你去到後邊,把惡氣兒揣在懷裏,將出好氣兒來,看怎的與他下個禮,賠個不是兒罷!你我既在簷底下,怎敢不低頭?常言:甜言羙語三冬暖,惡語傷人六月寒。你兩個已是見過話,只顧使性兒到幾時?人受一口氣,佛受一爐香。你去與他賠個不是兒,天大事都了了。不然,你不教他爹兩下裏為難。待要往你這邊來,他又惱。」金蓮道:「耶嚛,耶嚛!我㧱甚麽比他?可是他說的,他是真材實料正經夫妻,你我都是趁來的露水兒,能有多大湯水兒?比他的脚指頭兒也比不的!」玉樓道:「你由他説不是!我昨日不說的,一棒打三四個人。那就我嫁了你的漢子,也不是趁將來的,當初也有個三媒六證,只恁就跟了往你家來來?砍一枝,損百株;兔死狐悲,物傷其類。就是六姐惱了你,還有沒惱你的。有勢休要使盡,有話休要説盡。凡事看上顧下,留些兒防後纔好!不管蝗虫螞蚱,一例都説着,——對着他三位師父、郁大姐。人人有面,樹樹有皮,俺們臉上就沒些血兒!一切來往都罷了,你不去卻怎樣兒的?少不的逐日唇不離腮,還在一䖏兒!你快些把頭梳了,咱兩個一答兒後邊去。」那潘金蓮見他這般說,尋思了半日,忍氣吞聲,鏡臺前拿過抿鏡,只抿了頭,戴上䯼髻,穿上衣裳,同玉樓逕到後邊上房内。

玉樓掀開簾兒先進去,説道:「大娘,我怎的走了去,就牽了他來?他不敢不來。」便道:「我兒,還不過來與你娘磕頭?」在傍邊便道:「親家,孩兒年幼,不識好歹,冲撞親家。高擡貴手,將就他罷,饒過這一遭兒。到明日再無禮,犯到親家手裏,隨親家打,我老身却不敢說了!」那潘金蓮插燭也似與月娘磕了四個頭,跳起來趕着玉樓打,道:「汗邪了你這麻淫婦,你又做我娘來了!」連衆人都笑了,那月娘忍不住也笑了。玉樓道:「賊奴才,你見你主子與了你好臉兒,就抖毛兒打起老娘來了!」大妗子道:「這個你姊妹們笑開,恁歡喜歡喜却不好?就是俺這姑娘一時間一言半語聐聒的你們,大家廝擡廝敬,儘讓一句兒就罷了。常言:牡丹花兒雖好,還要緑葉兒扶持。」月娘道:「他不言語,那個好説他?」金蓮道:「娘是個天,俺們是個地。娘容了俺們,俺們骨秃扠着心裏!」玉樓打了他肩背一下,說道:「我的兒,你這囬兒也打你一面口袋了。」便道:「休要説嘴,俺們做了這一日活,也該你來助助忙兒。」這金蓮便洗手剔甲,在炕上與玉樓裝定菓盒,不在話下。

那孫雪娥單管率領家人媳婦,竈上整理菜蔬。厨役又在前邊大厨房内,烹炮蒸煮,燒錦纏羊,割獻花猪。琴童討將藥來,西門慶看了薬帖,吩咐把丸藥送到玉樓房中,煎薬與月娘。月娘便問玉樓:「你也討薬來?」玉樓道:「還是前日那根兒,下首裏只是有些怪疼。我敎他爹對任醫官説,捎帶兩服丸子薬來我喫。」月娘道:「你還是前日空心掉了冷氣了,那裏管下寒的事?」

按下後邊,却說前廳。宋御史先到了,看了桌席。西門慶陪他在捲棚内坐。宋御史又深謝其爐鼎之事:「學生還當奉價。」西門慶道:「早是我正要奉送公祖,猶恐見却,豈敢云價?」宋御史道:「這等何以克當?」一面又作揖致謝。茶罷,因説起地方民情風俗一節,西門慶大略可否而答之。次問其有司官員,西門慶道:「卑職只知其本府胡正尹民望素着,李知縣吏事克勤,其餘不知其詳,不敢妄說。」宋御史問道:「守禦周秀,曾與執事相交,為人却也好不好?」西門慶道:「周總兵雖歷練老成,還不如濟州荆都監,青年武擧出身,才勇兼備。公祖倒看他看。」宋御史道:「莫不是都監荆忠?執事何以相熟?」西門慶道:「他與我有一面之交,昨日遞了個手本與我,也要乞望公祖情盼一二。」宋御史道:「我也久聞他是個好將官。」又問其次者,西門慶道:「卑職還有妻兄吴鎧,現任本衛右所正千户之職。昨日委管修義倉,例該陞擢指揮,亦望公祖提拔,實卑職之沾恩惠也!」宋御史道:「既是令親,到明日類本之時,不但俾他加陞本等職級,我還保擧他現任管事。」這西門慶連忙作揖謝了。因把荆都監並吴大舅履歷手本遞上。宋御史看了,即令書辦吏典收執,吩咐:「到明日類本之時,呈行我看。」那吏典收下去了。西門慶又令左右悄悄遞上三兩銀子與他,那書吏如同印板刻在心上,不在話下。

正說話間,前廳鼓樂響。左右來報,兩個老爹都到了。慌的西門慶即出迎接,到廳上敍禮。這宋御史慢慢纔走出花園角門。衆官見畢禮數,觀看筵席,正中擺設大插桌一張,五老定勝方糖,高頂簇盤,大飯、五牲、菓品,甚是齊整,周圍桌席甚豐盛,心中大悦。都望西門慶謝道:「生受!容當奉補。」宋御史道:「分資誠為不足。四泉看我的分上罷了,諸公也不消補奉。」西門慶道:「豈有此理。」一面各分次序坐下。左右拿上茶來,衆官都説:「侯老先生那裏已各人差官邀去了。還在都府衙未起身哩!」兩邊俳長楽工,鼓楽笙笛簫管方響,在二門裏伺候的鐵桶相似。

看看等到午後時分,只見一疋報馬來到,說:「侯爺來了!」這裏兩邊鼓楽一齊響起,衆官都出大門前迎接,宋御史在二門裏相候。不一時,藍騎馬過盡,侯巡撫穿大紅孔雀,戴貂鼠暖耳,渾金帶,坐四人大轎,直至門首下轎。衆官迎接進來。宋御史亦換了大紅金雲白豸員領,犀角带,相讓而入。到於大廳上,叙畢禮數。各官廷參畢,然後與西門慶拜見。宋御史道:「此是主人西門千兵,現在此間理刑,亦是蔡老先生門下。」這侯巡撫即令左右官吏拿雙紅「友生侯蒙」單拜帖遞與西門慶。西門慶雙手接了,吩咐家人捧上去。一面參拜畢,寬衣上坐。衆官兩傍僉坐,宋御史居主位。捧畢茶,堦下動起楽來。宋御史把盞遞酒,簪花,捧上尺頭,隨即擡下桌席來,裝在盒内,差官吏送到公廳去了。然後上坐,獻湯飯,厨役上來割獻花猪,俱不必細説。先是教坊間弔上隊舞囘數,都是官司新錦繡衣裝,撮弄百戲,十分齊整。然後纔是海鹽子弟上來磕頭,呈上關目揭帖,侯公吩咐搬演〈裴晉公還帶記〉,唱了一摺下來,又割錦纏羊。端的花簇錦攢,吹彈歌舞,簫韶盈耳,金貂滿座。有詩為證:

華堂非霧亦非煙,歌遏行雲酒滿筵。

不但紅娥垂玉珮,果然綠鬢插金蟬。

侯巡撫只坐到日西時分,酒過數巡,歌唱兩摺下來,令左右拿下來五兩銀子,分賞厨役、茶酒、楽工、脚下人等,就穿衣起身。衆官俱送出大門,看着上轎而去。囬來,宋御史與衆官辭謝西門慶,亦告辭而歸。

西門慶送了囬來,打發楽工散了。因見天色尚早,吩咐把桌席休動,教厨役上來攢整菜蔬肴饌,一面使小廝請吴大舅來,並溫秀才、應伯爵、傅夥計、甘夥計、賁地傳、陳經濟來坐,聽唱。拿下兩桌酒饌肴品,打發海鹽子弟喫了,等的人來,敎他唱〈四節記〉:「冬景韓熙載夜宴陶學士」,擡出梅花來放在兩邊桌上,賞梅飲酒。原來那日賁四來興兒管厨,陳經濟管酒,傅夥計甘夥計看管家伙,聽見西門慶請,都來傍邊坐的。不一時,溫秀才過來,作揖坐下。吴大舅吴二舅應伯爵都來了。應伯爵與西門慶聲喏:「前日空過幾位嫂子,又多謝重禮!」西門慶笑罵道:「賊天殺的狗才!你打窗户眼兒内偸瞧的你娘們好!」伯爵道:「你休聽人胡説,豈有此理?我想來也没人……」指王經道:「就是你這賊狗骨禿兒,乾淨來家就學舌!我到明日把你這小狗骨禿兒肉也咬了!」説畢,喫了茶。

吴大舅要到後邊,西門慶陪下來,向吴大舅如此這般說:「我今對宋大巡替大舅說了説那個,他看了揭帖,交付書辦收了。我又與了書辦三兩銀子,連荆大人的都放在一處。他親口說下,到明日類本之時,自有意思。」吴大舅聽見,滿心歡喜,連忙與西門慶唱喏:「多累姐夫費心!」西門慶道:「我就說是我妻兄。他說旣是令親,我一定見過分上。」於是同到房中見了月娘。月娘與他哥道萬福。大舅向大妗子說道:「你往家去罷了!家没人,如何只顧不囬去了?」大妗子道:「三姑娘留下,教我過了初三日,初四日家去哩。」吴大舅道:「旣是姑娘留你,到初四日去便了。」説畢,月娘留他坐,不坐。來到前邊,安排上酒來飲酒。當下吴大舅、二舅、應伯爵、溫秀才上坐,西門慶主位,傅夥計、甘夥計、賁地傳、陳經濟,兩邊打横,共五張桌兒。下邊戲子鑼鼓響動,搬演「韓熙載夜宴,郵亭佳遇」。

正在熱鬧處,忽見玳安來説:「喬親家爹那裏使了喬通在下邊,請爹說話。」這西門慶隨即下席,到東角門首見喬通。喬通道:「爹説昨日空過親家,爹使我送那援例銀子來,一封三十兩,另外又拿着五兩與吏房使用。」西門慶道:「我明日早封過與胡大尹,他就與了劄付來。又與吏房銀子做甚麽?你還拿囬去。」一面吩咐玳安,教厨下拿了酒飯點心,在書房内管待喬通,打發去了。

話休饒舌,當日唱了「郵亭」兩摺,約有一更時分,西門慶前邊人散了,收了家伙,進入月娘房來。月娘正與大妗子在炕上坐的,大妗子見西門慶進來,連忙往那邊屋裏去了。西門慶因向月娘說:「我今日替你哥如此這般對宋巡按說,他許下嘉他,除加陞一級,還教他現任管事,就是指揮僉事。我剛纔已對你哥說了,他好不喜歡。只在年終就題本,有旨意下來。」月娘便道:「没的說,他一個窮衛家官兒,那裏有二三百兩銀子使?」西門慶道:「誰問他要一白文錢兒?我就對宋御史説,是我妻兄。他親口旣許下,無有個不做分上的。」月娘道:「隨你與他幹,我不管你。」西門慶便問玉簫:「替你娘煎了薬?拿來我瞧,打發你娘喫了罷。」月娘道:「你去,休管他。等我臨睡自家喫。」那西門慶纔待往外走,被月娘又呌囬來,問道:「你往那去?是往前頭去,趁早兒不要去。他頭裏與我賠了不是了,只少你與他賠不是去哩!」西門慶道:「我不往他屋裏去。」月娘道:「你不往那屋裏去,往誰屋裏去?那前頭媳婦子跟前也省可去。惹的他昨日對着大妗子好不拿話兒咂我,説我縱容着你要他,圖你喜歡哩!你又恁沒廉耻的!」西門慶道:「你理那小淫婦兒怎的!」月娘道:「你只依我,今日偏不要往前邊去,也不要你在我這屋裏。你往下邊李嬌姐房裏睡去。隨你明日去不去,我就不管你了。」這西門慶見恁説,無法可䖏,只得往李嬌兒房裏歇了一夜。

到次日,臘月初一日,早往衙門中去,同何千户發牌陞廳畫卯,發放公文,一早晨纔來家。又打點禮物猪酒,並三十兩銀子,差玳安往東平府送胡府尹去。胡府尹收下禮物,即時封過劄付來。西門慶在家請了陰陽徐先生,廳上擺設猪羊酒菓,燒紙還願心畢,打發徐先生去了。因見玳安到了,看了囬帖,已封過劄付來,上面用着許多印信,填寫「喬洪本府義官」名目。一面使玳安送兩盒胙肉與喬大户家,就請喬大户來喫酒,與他劄付瞧。又分送與吴大舅、溫秀才、應伯爵、謝希大、傅夥計、甘夥計、韓道國、賁地傳、崔本,每人都是一盒,俱不在話下。一面又發帖兒,初三日請周守禦、荆都監、張團練、劉薛二内相、何千户、范千户、吴大舅、喬大户、王三官兒,共十位客,呌一起雜耍樂工,四個唱的。

那日孟玉樓在月娘房内攢了帳,遞與西門慶,就交代與金蓮管理使用銀錢,他不管了。因問月娘道:「大娘,你昨日喫了薬兒,可好些?」月娘道:「怪不的人説怪浪肉!平白敎人家漢子捏了捏手,今日好了,頭也不疼,心口也不發脹了。」玉樓笑道:「大娘,你原來只少他一捏兒!」連大妗子也笑了。西門慶㧱了攢的帳來,又問月娘。月娘道:「該那個管,你交與那個就是了,來問我怎的?誰肯,讓的誰。」這西門慶方纔兑了三十兩銀子,三十吊錢,交與金蓮管理,不在話下。

良久,喬大户到了,西門慶陪他廳上坐的,如此這般,拿胡府尹劄付與他看。看見上寫「義官喬洪」名字,「援例上納白米三十石,以濟邊儲」。滿心歡喜,連忙向西門慶打恭致謝:「多累親家費心,容當叩謝。」呌喬通好生送到家去,因説:「明日若親家見招,在下有此冠帶,就敢來陪坐也不妨。」西門慶道:「初三日親家好歹早些下降。」一面喫畢茶,吩咐琴童:「西廂房書房裏放桌兒,親家請那裏坐,還暖些。」到書房,地爐内籠着火。西門慶與喬大户對面坐下,因告訴說:「昨日巡按兩司請侯老之事,侯老甚喜。明日起身,少不的俺同僚們都送郊外方囘。」纔抹桌兒收拾放菜兒,只見應伯爵到了。斂了幾分人情,呌應寳用盒兒拿來,交與西門慶說:「此列位奉賀哥的分資。」西門慶打開觀看,裏面頭一位就是吴道官,其次應伯爵、謝希大、祝日念、孫寡嘴、常時節、白來創、李智、黄四、杜三哥,共十分人情。西門慶道:「我的這邊,還有舍親吴二舅、沈姨夫、門外任醫官、花大哥,並三個夥計、溫葵軒,也有二十多人,就在初四日請罷。」一面令左右收進人情後邊去,使琴童兒:「拿馬請你吴大舅來,陪你喬親家爹坐。」因問:「溫師父在家不在?」來安兒道:「溫師父不在家,従早晨望朋友去了。」不一時,吴大舅來到,連陳經濟五人共坐,把酒來斟。桌上擺列許多熱下飯、湯碗,無非是猪蹄羊頭,燒爛煎煿,鷄魚鵝鴨,添換之類。飲酒中間,西門慶因向吴大舅説喬親家恭喜的事:「今日已領下義官劄付來了。容日我這裏備禮寫文軸,咱們従府中迎賀迎賀。」喬大户道:「惶恐!甚大職役,敢起動列位親家費心?」忽有本縣衙差人送曆日來了,共二百五十本。西門慶拿囬帖、賞賜,打發來人去了。應伯爵道:「新曆日俺們不曾見哩。」西門慶把五十本拆開,與喬大户吴大舅伯爵三人分了。伯爵看了看:開年改了重和元年,該閏正月。

不説當日席間猜枚行令飲酒。至晚,喬大户先告家去。西門慶陪吴大舅坐到起更時分方散,吩咐伴當:「早伺候備馬,邀你何老爹到我這裏,起身同往郊外送侯爺。留下四名排軍,與來安春鴻兩個,跟大娘轎往夏家去。」説畢,就歸金蓮房中來。

那婦人未及他進房,就先摘了冠兒,亂挽烏雲,花容不整,朱粉懶施,渾衣兒歪在牀上。房内燈兒也不點,靜悄悄的。西門慶進來,便呌春梅,不應。只見婦人睡在牀内,呌着,只不做聲。西門慶便坐在牀上,問道:「怪油嘴,你怎的恁個腔兒?」也不答應。被西門慶用手拉起他來,說道:「你如何悻悻的?」那婦人便做出許多喬張致來,把臉扭着,止不住紛紛的香腮上滚下淚來。那西門慶就是鐵石人,也把心來軟了,連忙一隻手摟着他脖子說:「怪油嘴,好好兒的,平白你兩個合甚麽氣?」問他一聲兒,那婦人半日方囬言說道:「誰和他合氣來?他平白尋起個不是,對着人罵我是攔漢精趂漢精,趁了你來了!他是真材實料正經夫妻!誰敎你又來我這屋裏做甚麽?你守着他去就是了,省的我把攔着你。說你來家,只在我這屋裏纏!早是肉身聽着,你這幾夜只在我這屋裏睡來?白眉赤眼兒,你嚼舌根!一件皮襖,也説我不問他,擅自就問漢子討了。我是你的奴才丫頭?莫不往你屋裏與你磕頭去?為這小肉兒罵了那賊瞎淫婦,也説不管。偏有那些聲氣的!你是個男子漢,若是有張主的,一拳拄定,那裏有這些閒言悵語?怪不的俺們自輕自賤。常言道:賤裏買來賤裏賣,容易得來容易捨。趁將你家來,與你家做小老婆,不氣長!自古人善得人欺,馬善得人騎,便是如此。你看昨日,生怕氣了他,在屋裏守着的是誰?請太醫的是誰?在跟前攛撥侍奉的是誰?苦惱俺們這陰山背後,就死在這屋裏也沒個人兒來瞅問!這個就見出那人的心來了!還教含着那眼淚兒,走到後邊,與他賠個不是!」説着,那桃花臉上止不住又滚下珍珠兒,倒在西門慶懷裏嗚嗚咽咽,哭的摔鼻涕,彈眼淚。西門慶一面摟抱着,勸道:「罷麽,我的兒!我連日心中有事,你兩家各省這一句兒就罷了。你教我說誰的是?昨日要來看你,他說我來與你賠不是,不放我來。我往李嬌兒睡了一夜。雖然我和人睡,一片心只想着你!」婦人道:「罷麽,我也見出你那心來了。一味在我面上虚情假意,到了還疼你那正經夫妻。他如今現替你懷着孩子,俺們一根草兒,拿甚麽比他!」被西門慶摟過脖子來,親了個嘴道:「怪油嘴,休要胡説!」

只見秋菊拿進茶來,西門慶便道:「賊奴才,好乾淨兒!如何教他拿茶?」因問:「春梅怎的不見?」婦人道:「你還問春梅哩,他餓的只有一口遊氣兒,那屋裏躺着不是?帶今日,三四日没喫點湯水兒了,一心只要尋死在那裏。說他大娘對着人駡了他奴才,氣生氣死,整哭了三四日了。」這西門慶聽了,說道:「真個?」婦人道:「莫不我哄你不成?你瞧去不是!」

這西門慶慌過這邊屋裏,只見春梅容粧不整,雲髻斜歪,睡在炕上。西門慶呌道:「怪小油嘴,你怎的不起?」呌着他,只不做聲,推睡。被西門慶雙關抱將起來。那春梅従酩子裏伸腰,一個鯉魚打挺,險些兒没把西門慶掃了一跤,早是抱的牢,有護炕倚住不倒。春梅道:「達達,放開了手。你又來理論俺們這奴才做甚麽?也沾辱了你這兩隻手!」西門慶道:「小油嘴兒,你大娘説了你兩句兒罷了,只顧使起性兒來了。説你這兩日没喫飯?」春梅道:「喫飯不喫飯,你管他怎的?左右是奴才貨兒,死便隨他死了罷!我做奴才,一來也沒幹壞了甚麽事,並沒敎主子罵我一句兒,攩我一下兒。做甚麽為這㒲遍街搗遍巷的賊瞎淫婦,教大娘這等罵我,嗔俺娘不管我!莫不為瞎淫婦扯倒打我五板兒?等到明日韓道國老婆不來便罷,若來,你看我指與他,一頓好的不罵!原來送了這瞎淫婦來,就是個祸根!」西門慶道:「就是送了他來,也是好意,誰曉的為他合起氣來了。」春梅道:「他若肯放和氣些,我好意罵他?他小量人家!」西門慶道:「我來這裏,你還不倒鍾茶兒我喫?那奴才手不乾淨,我不喫他倒的茶。」春梅道:「死了王屠,連毛喫猪。我如今走也走不動在這裏,還敎我倒甚麽茶!」西門慶道:「怪小油嘴兒,誰敎你不喫些甚麽兒!」因說道:「咱們往那邊屋裏去,我也還没喫飯哩。教秋菊後邊取菜兒,篩酒,烤菓餡餅兒,炊鮓湯,咱們喫。」於是不由分説,拉着春梅手,到婦人房内,吩咐秋菊:「拿盒子後邊取喫飯的菜兒去。」不一時,拿了一方盒菜蔬:一碗燒猪頭,一碗炖爛牛肉,一碗熬鷄,一碗煎煿鮮魚,和白米飯四碗;喫酒的菜蔬:海蜇、荳芽菜、魚鮓、蝦米之類。西門慶吩咐春梅,把肉鮓打上幾個鷄旦,加上酸笋、韮菜,和上一大碗香噴噴餛飩湯來,放下桌兒擺下,一面盛飯來,又烤了一盒菓餡餅兒。西門慶和金蓮並肩而坐,春梅在傍邊隨着同喫。三個你一盃,我一盃,喫了一更方散。

就睡到次日,西門慶早起,約會何千户來到,喫了頭腦酒起身,同往郊外送侯巡撫去了。吳月娘這裏先送了禮去,然後打扮,坐大轎,排軍唱道,來安春鴻跟隨,往夏指揮家來喫酒,看他娘子兒,不在話下。

玳安王經在家,只見午後時分,有縣前賣茶的王媽媽,領着何九,來大門首尋問玳安:「老爹在家不在家?」玳安道:「王奶奶,何老人家,稀行!今日那陣風兒吹你老人家來這裏走走?」王婆子道:「没勾當怎好來踅門踅户?今日不因老九因為他兄弟的事,敢來央煩老爹,老身還不來哩。」玳安道:「老爹今日與侯爺送行去了。俺大娘也不在家。你老人家站站,等我進去對五娘說聲。」進入不多時,出來說道:「俺五娘請你老人家進去哩。」王婆道:「我敢進去?你引我引兒,只怕有狗。」那玳安引他進入花園金蓮房門首,掀開簾子,王婆進去。見婦人家常戴着臥兔兒,穿着一身錦緞衣裳,搽抹的如粉粧玉琢,正在房中炕上,脚登着爐臺兒,坐的嗑瓜子兒。房中帳懸錦繡,牀設縷金,玩器爭輝,箱奩耀日。進去不免下禮,慌的婦人答禮,説道:「老王,免了罷。」那婆子見畢禮,坐在炕邊頭。婦人便問:「怎的一向不見你?」王婆子道:「老身可心中想着娘子,只是不敢來親近。」問:「添了哥哥不曾?」婦人道:「有倒好了。小産過兩遍,白不存。」又問:「你兒子有了親事?」王婆道:「還不曾與他尋,他跟客人淮上,來家這一年多,家中胡亂積賺了些小本經紀,買個驢兒,胡亂磨些麵兒,賣來度日。慢慢替他尋一個兒與他。」因問:「老爹不在家了?」婦人道:「他爹今日往門外與撫按官送行去了。他大娘也不在家。有甚話說?」王婆道:「老九有樁事,央及老身來對老爹說。他兄弟何十,乞賊攀着,現拿在提刑院老爹手裏問。攀他是窝主。本等與他無干,望乞老爹案下與他分豁分豁:等賊若指攀,只不准他就是了。何十出來,到明日買禮來重謝老爹。有個説帖兒在此。」一面遞與婦人。婦人看了,説道:「你留下,等你老爹來家,我與他瞧。」婆子道:「老九在前邊伺候着哩,明日教他來討話罷。」婦人一面呌秋菊看茶來。须臾,秋菊㧱了一盞茶來,與王婆喫了。那婆子坐着説道:「娘子,你這般受福夠了!」婦人道:「甚麽夠了!不惹氣便好!成日嘔氣不了在這裏。」那婆子道:「我的奶奶,你飯來張口,水來濕手。這等插金帶銀,呼奴使婢,又惹甚麽氣?」婦人道:「常言道説得好,三窝兩塊,大婦小妻。一個碗内兩張匙,不是湯着就抹着,如何沒些氣兒?」婆子道:「好奶奶,你比那個不聰明?趁着老爹這等好時月,你受用到那裏是那裏!」說道:「我明日使他來討話罷。」於是拜辭起身。婦人道:「老王,你多坐囘去不是?」那婆子道:「難為老九只顧等我,不坐罷,改日再來看你。」那婦人也不留他留兒,就放出他來了。到了門首,又叮嚀玳安。玳安道:「你老人家去,我知道。等俺爹來家,我就禀。」何九道:「安哥,我明日早來討話罷。」於是和王婆一路去了。

至晚,西門慶來家,玳安便把此事禀知西門慶。西門慶到金蓮房看了帖子,交付與答應的收着:「明日到衙門中稟我。」一面又令陳經濟發初三日請人帖兒。瞞着春梅,又使琴童兒送了一兩銀子並一盒點心到韓道國家,對着他説:「是與申二姐的,教他休惱。」那王六兒笑嘻嘻接了,說:「他不敢惱。多上覆爹娘:冲撞他春梅姑娘!」俱不在言表。

至晚,月娘來家,穿着銀鼠皮襖,遍地金襖兒,錦藍裙,坐大轎,打着兩個燈籠,到家先拜見大妗子,衆人然後相見。西門慶正在上房喫酒,道了萬福。當下告訴:「夏大人娘子見了我去,好不喜歡,多謝重禮。今日也有許多親鄰堂客。原來夏大人有書來了,也有與你的書,明日送來與你。也只在這初六七起身,僱車搬取家小上京去也。說了又說:好歹教賁四送他家到京,就囬來。賁四的那孩子長兒,今日與我磕頭,好不出挑了,好個身段兒!嗔道他傍邊捧着茶,把眼只顧偸瞧我。我也忘了他!倒是夏大人娘子呌他:——改換了名字,呌做瑞雲——『過來與你西門奶奶磕頭。』他纔放下茶托兒,與我磕了四個頭。我與了他兩枝金花兒。如今夏大人娘子好不喜歡擡擧他,也不把他當房裏人,只做親兒女一般看他。」西門慶道:「還是這孩子有福,若是別人家手裏,怎麽容得?不罵奴才,撒椒末兒,又肯擡擧他?」被月娘瞅了一眼,説道:「硶說嘴的貨,是我罵了你心愛的小姐兒!」那西門慶笑了,說道:「他借了賁四押家小去,我線舖子教誰看?」月娘道:「関兩日也罷了。」西門慶道:「關兩日阻了買賣。近年節,紬絹絨線銷正快,如何關閉了舖子?到明日等再䖏。」說畢,月娘進裏間脫衣裳摘頭,走到那邊房内,和大妗子坐的,家中大小都來參見磕頭。是日,西門慶在後邊雪娥房中歇了一夜,早往衙門中去了。

只見何九走來,問玳安討信,與了玳安一兩銀子。玳安如此這般:「昨日爹來家,就替你説了。今日到衙門中,就開出你兄弟來放了。你往衙門首伺候。」這何九聽言,滿心歡喜,一直走衙門前去了。西門慶到衙門裏坐廳,提出強盜來,每人又是一夾二十大板,打的順腿鲜血迸流。把何十開出來放了,另㧱了弘化寺一名和尚頂缺,說強盜曾在他寺内宿了一夜。世上有如此不公事?正是:張公喫酒李公醉,桑樹上喫刀柳樹上暴。有詩為證:

宋朝氣運已將終,執掌提刑忒不公。

畢竟難逃天地眼,那堪激濁與揚清。

那日,西門慶家中叫了四個唱的:吴銀兒、鄭愛月兒、洪四兒、齊香兒,日頭向午就來了,都拿着衣裳包兒,齊到月娘房内,與月娘大妗子衆人磕了頭。月娘在上房擺茶與他們喫了。正彈着樂器唱曲兒與大妗子月娘衆人聽,忽見西門慶従衙門中來家,進房來。四個唱的都放了樂器,笑嘻嘻向前一齊與西門慶插燭也磕了頭坐下,月娘便問:「你怎的衙門中這咱纔來?」西門慶告訴:「今日問理好幾樁事情。」因望着金蓮說:「昨日王媽媽來説何九那兄弟,今日我已開除來放了。那兩名強盜還攀扯他,教我每人打了二十,夾了一夾,拿了門外寺裏一個和尚頂缺,明日做文書送過東平府去。又是一起奸情事,丈母養女婿的。那女婿年小,不上三十多歲,名喚宋得,原與這家是養老不歸宗女婿。落後親丈母死了,娶了個後丈母周氏。不上一年,把丈人死了。這周氏年小,守不得,就與他這女婿常時言笑自若,漸漸在家嚷的人知道,住不牢。一日,送他這丈母往鄉裏娘家去,周氏便向宋得說:『你我本没事,枉躭其名。今日在此山野空地,咱兩個成其夫妻罷!』這宋得就把周氏姦訖一度。以後娘家囬還,遂通姦不絶。後因為責使女,被使女傳於兩隣,纔首告官。今日取了供招,都一日送過去了。這一到東平府,姦妻之母,係緦麻之親,兩個都是絞罪!」潘金蓮道:「要着我,把學舌的奴才打的爛糟糟的。問他個死罪也不多!你穿着青衣抱黑柱,一句話就把主子弄了!」西門慶道:「也喫我把奴才拶了幾拶子好的。為你這奴才,一時小節不完,丧了兩個人性命!」月娘道:「大不正則小不敬。母狗不掉尾,公狗不上身!大凡還是女婦人心邪,若是那正氣的,誰敢犯他?」連四個唱的都笑道:「娘說的是。就是俺裏邊唱的,接了孤老的朋友,還使不的,休說外頭人家。」說畢,擺飯與西門慶喫了。

忽聽前廳鼓樂響,荆都監老爹來了。西門慶連忙冠带出迎,接至廳上叙禮,謝其厚賜,分賓主坐下。茶罷,如此這般告說:「宋巡按收了說帖,已面慨許,執事恭喜必然在邇。」荆都監聽了,又轉身下坐作揖致謝:「老翁費心,提㩦之力,銘刻難忘。」西門慶又說起:「周老總兵,生亦薦言一二,宋公必有主意。」談話間,忽報劉薛二内相到,鼓樂迎接進來。西門慶降堦相讓入廳,兩個叙禮。二位内相皆穿青縲絨蟒衣,寳石縧環,正中間坐下。次後周守禦到了,一處叙話。荆都監又向周守禦說:「四泉厚情,昨日宋公在尊府擺酒與侯公送行,曾稱頌公之才猷,宋公已留神於中,高轉在即。」周守禦亦欠身致謝不盡。落後張團練、何千户、王三官、范千户、吴大舅、喬大戶,陸續都到了。喬大户冠帶,青衣四個伴當跟隨。進門見畢諸公,與西門慶拜了四拜。衆人問其恭喜之事,西門慶道:「舍親家在本府援例,新受恩榮義官之職。」周守禦道:「四泉令親,吾輩亦當奉賀。」喬大户道:「蒙列位老爹盛情,豈敢動勞!」說畢,各分次序坐下。遍坐遞上一道茶畢,然後收拾上座。錦屏前玳筵羅列,畫堂内寳玩爭輝,堦前動一派笙歌,席上堆滿盤異菓。良久,遞酒安席畢,各家僮僕上來接去衣服,歸席坐下。王三官再三不肯上來坐。西門慶道:「尋常罷了,今日在舍,權借一日陪諸公上座。」王三官迫不得已,左邊垂首坐了。須臾上罷湯飯,廚役上來割道燒鵝,獻小割。下邊教坊囬數隊舞吊畢,撮弄雜耍百戯院本之後,四個唱的慢慢纔上來,拜見過了。個個粉粧花貌,人人珠翠僊裳,銀箏玉阮放嬌聲,倚翠偎紅頻笑語。正是:

舞裙歌板逐時新,散盡黄金只此身。

寄語富兒休暴殄,儉如良薬可醫貧。

不説當日劉内相坐首席,也賞了許多銀子。飲酒作歡至一更時分方散。西門慶打發楽工賞錢出門,四個唱的都在月娘房内彈唱。月娘留下吴銀兒過夜,打發三個唱的去。臨去,見西門慶在廳上,拜見拜見。西門慶吩咐鄭愛月兒:「你明日就拉了李桂姐兩個,還來唱一日。」那鄭愛月兒就知今日有王三官兒,不呌李桂姐來唱。笑道:「爹你兵馬司倒了牆——賊走了!」又問:「明日請誰喫酒?」西門慶道:「都是親朋。」鄭月兒道:「有應二那花子,我不來。我不要見那醜寃家怪物!」西門慶道:「明日沒有他。」愛月兒道:「沒有他纔好。若有那怪攮刀子的,俺們不來。」說畢,磕了頭,揚長去了。西門慶看着收了家伙,上李瓶兒那邊和如意兒睡了,一宿晚景題過。

次日早往衙門,送問那兩起人犯過東平府去。囬來家中擺酒,請吴道官、吴二舅、花大舅、沈姨夫、韓姨夫、任醫官、溫秀才、應伯爵並會中人、李智、黄四、杜三哥,並家中二個夥計,十二張桌兒。席間止是李桂姐、吴銀兒、鄭愛月兒,三個粉頭遞酒,李銘吴惠鄭奉三個小優兒彈唱。正遞酒中間,忽平安來報:「雲二叔新襲了職,來拜爹,送禮來。」西門慶聽言,連忙道:「有請。」只見雲離守穿着青紵絲補服員領,冠冕着,腰繫金帶,後邊伴當擡着禮物,先遞上揭帖與西門慶觀看,上寫:「新襲職山東清河右衛指揮同知,門下生雲離守頓首百拜。謹具土儀:貂鼠十個,海魚一尾,蝦米一包,臘鵝四隻,臘鴨十隻,油紙簾二架,少申芹敬。」西門慶即令左右收了,連忙致謝。雲離守道:「在下昨日纔來家,今日特來拜老爹。」於是磕頭,四雙八拜,說道:「蒙老爹莫大之恩,些少土儀,表意而已。」然後又與衆人叙禮拜見。西門慶見他居官,就待他不同,安他與吴二舅一桌坐了。連忙安下鍾筯,下了湯飯,脚下人俱打發攢盤酒肉。因問起發喪替職之事,這雲離守一一叙言:「蒙兵部余爺憐其家兄在鎮病亾,祖職不動,還與了個本衛現任僉書。」西門慶歡喜道:「恭喜,恭喜!容日一定來賀。」當日衆人席上每位奉陪一盃,又令三個唱的奉酒,須臾把雲離守灌的醉了。那應伯爵在席上,如線兒提的一般,起來坐下;又鬦李桂姐和鄭月兒,彼此互相戲罵不絶。這個罵他怪門神,白臉子,少根基的貨!那個罵他是醜冤家,怪物勞,猪八戒坐在冷舖裏。伯爵罵道:「我把你這兩個女又十撇!鴉胡石影子布兒朶朶雲兒,丁口惡心。」不說當日酒筵笑聲,花攢錦簇,觥籌交錯,耍頑至二更時分方纔席散。打發三個唱的去了,西門慶歸上房宿歇。

到次日起來遲,正在上房擺粥喫了,穿衣要拜雲離守。只見玳安來説:「賁四在前邊請爹說話。」西門慶就知因為夏龍溪送家小之事,一面出來廳上。只見賁四向袖中取出夏指揮書來呈上,説道:「夏老爹要敎小人送送家小往京裏去,不久就囬。小人禀問過老爹,去不去?」西門慶看了書中言語,無非是叙其闊別,謝其早晚看顧家小,又借賁四携送家小之事。因說道:「他旣央你,你怎的不去?」因問:「幾時起身?」賁四道:「今早他大官兒呌了小人去,吩咐初六日家小准上車起身。小人也得月半纔囬來。」説畢,把獅子街舖内鑰匙,交遞與西門慶。西門慶道:「你去,我教你吳二舅來,替你開兩日舖子罷。」那賁四方纔拜辭出門,往家中收拾行裝去了。這西門慶就冠冕着出門,僕従跟隨,乘馬拜雲指揮去了。

那日是大妗子家去,呌下轎子門首伺候。也是合當有事,月娘裝了兩盒子茶食,點心下飯,上房管待。大妗子出門首上轎,只見畫童兒小廝躲在門傍鞍子房兒大哭不止。那平安兒只顧扯他,那小夥子越扯越哭起來,被月娘等聽見。送出大妗子上轎去了,便問平安兒:「賊囚,你平白拉他怎的?惹的他恁怪哭!」平安道:「溫師父那邊呌他,他白不去,只是罵小的。」月娘道:「你教他好好去罷。」因問道:「小廝,你師父那邊呌,去就是了,怎的哭起來?」那畫童嚷平安道:「又不管你事,我不去罷了,你扯我怎的!」月娘道:「你因何不去?」那小廝又不言語。金蓮道:「這賊小囚兒就是個肉佞賊,你大娘問你,怎的不言語?」被平安向前打了一個嘴巴,那小廝越發大哭了。月娘道:「怪囚根子,你平白打他怎的?你好好教他說,怎的不去?」

正問着,只見玳安騎了馬進來,月娘問道:「你爹來了?」玳安道:「被雲叔留住喫酒哩。使我送衣裳來了,帶毡巾去。」看見畫童兒哭,便問:「小大官兒,怎的號啕痛、剜牆拱?」平安道:「對過溫師父叫着,他不去,反哭罵起我來了。」玳安道:「我的哥哥,溫師父呌,你仔細!他有名的溫屁股,一日没屁股也成不的。你每常怎麽挨他的,今日如何又躱起來了?」月娘罵道:「怪囚根子,怎麽溫屁股?」玳安道:「娘自問他就是了。」那潘金蓮得不的風兒就是雨兒,一面呌過畫童兒來,只顧問他:「小奴才,你實說,他呼你做甚麽?你不說,看我教你大娘打你。」逼問那小廝急了,說道:「他又要哄着小的,把他行貨子放在小的屁股裏,弄的脹脹的疼起來。我說你還不快拔出來,他又不肯拔,只顧來囬動。且教小的拿出來,跑過來。他又來呌小的。」月娘聽了,便喝道:「怪賊小奴才兒,還不與我過一邊去!也有這六姐,只管好審問他,說的硶死了!我不知道,還當好話兒側着耳朶兒聽他!這蠻子也是個不上蘆葦的行貨子!人家小廝與你使,却背地幹這個營生!」那金蓮道:「大娘,那個上蘆葦的肯幹這營生?冷舖睡的花子纔這般所為!」孟玉樓道:「這蠻子他有老婆,怎生這等没廉耻?」金蓮道:「他來了這一向,俺們就沒見他老婆怎生樣兒。」平安道:「怎麽樣兒,娘們會勝看不見他。他但往那裏去,就鎖了門。住了這半年,我只見他坐轎子往娘家去了一遭,沒到晚就來家了。每常幾時出個門兒來?只好晚夕門首出來倒榪子走走兒罷了。」金蓮道:「他那老婆也是個不長俊的行貨子!嫁了他,怕不的也没見個天日兒,敢每日只在屋裏坐天牢哩?」說了囬,月娘同衆人囘後邊去了。

西門慶約莫日落時分來家,到上房坐下。月娘問道:「雲夥計留你坐來?」西門慶道:「他在家,見我去,甚是無可不可,旋放桌兒留我坐,打開一罈酒陪我喫。如今衛中荆南崗陞了,他就挨着掌印。明日連他和喬親家,就是兩分賀禮。衆同僚都說了,要與他挂軸子。少不的教溫葵軒做兩篇文章,早些買軸子寫下。」月娘道:「還纏甚麽溫葵軒、鳥葵軒哩!平白安扎恁樣行貨子,沒廉耻!傳出去教人家知道,把醜來出盡了!」西門慶聽言,唬了一跳,便問:「怎麽的?」月娘道:「你别要來問我,你問你家小廝去。」西門慶道:「是那個小廝?」金蓮道:「情知是誰,畫童賊小奴才!俺送大妗子去,他正在門首哭。如此這般,溫蠻子弄他來!」這西門慶聽了,還有些不信。便道:「你呌那小奴才來,等我問他。」一面使玳安兒前邊把畫童兒呌到上房跪下,西門慶要拿拶子拶他,便道:「賊奴才,你實說,他呌你做甚麽?」畫童兒道:「他呌小的,要灌醉了小的,要幹小營生兒。今日小的害疼,躱出來了,不敢去。他只顧使平安呌,又打小的。教娘出來看見了。他常時問爹家中各娘房裏的事,小的不敢說。昨日爹家中擺酒,他又教唆小的偸銀器兒家伙與他。又某日,他望俺倪師父去,拿爹的書稿兒與倪師父瞧,倪師父又與夏老爹瞧。」這西門慶不聽便罷,聽了便道:「畫虎畫皮難畫骨,知人知面不知心。我把他當個人看,誰知是人皮包狗骨東西,要他何用?」一面喝令畫童兒起去,吩咐:「再不消過那邊去了。」那畫童磕了頭起來,往前邊去了。西門慶向月娘:「怪道前日翟親家說我,『機事不密則害成。』我想來沒人,原來是他把我的事透泄與人,我怎得曉的!這樣狗骨秃東西,平白養在家做甚麽!」月娘道:「你和誰說!你家又沒孩子上學,平白招攬個人在家養活着,寫禮帖兒。我家有這些禮帖書柬寫?饒養活着他,還教他弄乾坤兒!怪不的你我家裏底事往外打探。」西門慶道:「不消說了,明日教他走道兒就是了。」一面叫將平安來了,吩咐:「對過對他說,家老爹要房子堆貨,教溫師父轉尋房兒便了。等他來見我,你在門首只囬我不在家。」那平安兒應諾去了。

西門慶告月娘說:「今日賁四來辭我,初六日起身,與夏龍溪送家小往東京去。我想來線舖子没人,倒好敎他二舅來,替他開兩日兒。左右與來昭一遞三日上宿,飯倒都在一䖏喫,好不好?」月娘道:「好不好,隨你呌他去,我不管你,省的人又說照顧了我的兄弟。」西門慶不聽,於是使棋童兒:「請你二舅來。」不一時,請吴二舅到,在前廳陪他坐的喫酒,把鑰匙交付與他,明日同來昭早往獅子街開舖子去,不在話下。

却説溫秀才見畫童兒一夜不過來睡,心中着恐。到次日,平安走來說:「家老爹多上覆溫師父,早晚要這房子堆貨,教師父別尋房兒罷。」這溫秀才聽了,大驚失色,就知畫童兒有甚話說。穿了衣巾,要見西門慶說話。平安兒道:「俺爹往衙門中去了,還未來哩。」比及來,這溫秀才又衣巾過來伺候,具了一篇長柬,遞與琴童兒,琴童又不敢接,說道:「俺爹纔従衙門中來家辛苦,後邊歇去了,俺們不敢禀。」這溫秀才就知疎遠他,一面走到倪秀才家商議,還搬移家小往舊處住去了。正是:誰人汲得西江水,難洗今朝一面羞。

靡不有初鮮克終,交情似水淡長濃。

自古人無千日好,果然花無摘下紅。

畢竟未知後來如何,且聽下回分解。

