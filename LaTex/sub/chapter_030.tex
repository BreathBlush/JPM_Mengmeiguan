\includepdf[pages={59,60},fitpaper=false]{tst.pdf}
\chapter*{第三十囬 \\來保押送生辰擔 西門慶生子喜加官}
\addcontentsline{toc}{chapter}{第三十囬 來保押送生辰擔 西門慶生子喜加官}
\markboth{第三十囬 來保押送生辰擔 西門慶生子喜加官}{第三十囬 來保押送生辰擔 西門慶生子喜加官}
\thispagestyle{empty}

\begin{myquote}
得失榮枯總是閑,機關用盡也徒然!

人心不足蛇吞象,世事到頭螳捕蟬。

無薬可延卿相壽,有錢難買子孫賢。

家常守分隨緣過,便是消遙自在僊。
\end{myquote}

話説西門慶與潘金蓮兩個洗畢澡,就睡在房中。春梅坐在穿廊下一張涼椅兒上衲鞋。只見琴童兒在角門首探頭舒腦的觀看。春梅問道:「你有甚話説?」那琴童又見秋菊頂着石頭跪在院内,只顧用手往來指。春梅罵道:「怪囚根子,你有甚麽話,説就是了,指手畫脚怎的?」那琴童笑了半日,方纔説:「有看墳的張安兒,在外邊等爹説話哩。」春梅道:「賊囚根子,張安就是了,何必大驚小怪見鬼也似!悄悄兒的,爹和娘在屋裏睡着了,驚醒他,你就是死。你且教張安在外邊等等兒。」那琴童兒走出來外邊,約等够半日,又走來角門首踅探,問:「姐,爹起來了不曾?」春梅道:「怪囚,失張冒勢,恁唬我一跳。有要沒緊,兩頭來囬遊魂哩!」琴童道:「張安等爹出去見了,説了話,還要趕出門去,怕天晚了。」春梅道:「爹娘正睡的甜甜兒的,誰敢攪擾他。你教張安且等着去,十分晚了,教他明日去罷。」

正説着,不想西門慶在房裏聽見,便呌春梅進房,問誰説話。春梅道:「琴童小廝進來説,墳上張安兒在外邊,見爹説話哩。」西門慶道:「㧱衣我穿,等我起去。」春梅一面打發西門慶穿衣裳,金蓮便問:「張安來説甚麽話?」西門慶道:「張安前日來説,喒家墳隔壁趙寡婦家莊子兒連地要賣,價錢三百兩銀子,我只還他二百五十兩銀子,教張安和他講去。若成了,我敎賁四和陳姐夫去兑銀子。裏面一眼井,四個井圈打水。我買了這莊子,展開合為一䖏,裏面蓋三間捲棚、三間廳房、疊山子花園、松牆,槐樹棚、井亭、射箭廳、打毬場、耍子去䖏,破使幾兩銀子收拾也罷。」婦人道:「也罷,喒買了罷。明日你娘們上墳,到那裏好遊玩耍子。」説畢,西門慶往前邊和張安説話去了。

金蓮起來,向鏡臺前重匀粉臉,再整雲鬟。出來院内,要打秋菊。那春梅旋去外邊呌了琴童兒來掉板子。金蓮便問道:「教你㧱酒,你怎的㧱冷酒與你爹喫?原來你家沒大小,説着你,還釘嘴鐵舌兒的!」喝聲呌琴童兒:「與我老實打與這奴才二十板子。」那琴童纔打到十板子上,多虧了李瓶兒笑嘻嘻走過來勸住了,饒了他十板。金蓮敎與李瓶兒磕了頭。放他起來,廚下去了。李瓶兒道:「老馮領了個十五歲的丫頭,後邊二姐姐買了房裏使喚,要送與他去哩,要七兩五錢銀子。請你過去瞧瞧。」這金蓮遂與李瓶兒一同後邊去了。李嬌兒果然問了西門慶,用七兩銀子買了,丫頭改名夏花兒,房中使喚,不在話下。

安下一頭,却説一䖏。單表來保同吴主管押送生辰擔,自従離了清河縣,一路朝登紫陌,暮踐紅塵,飢餐渴飲,夜住曉行。正值大暑炎蒸天氣,爍石流金之際,路上十分難行。評話捷説,有日到了東京萬壽門外,尋客店安下。到次日,齎擡馱箱禮物,逕到天漢橋蔡太師府門前伺候。來保教吴主管押着禮物,他穿上青衣,逕向守門官吏唱了個喏。那守門官吏問道:「你是那裏來的?」來保道:「我是山東清河縣西門員外家人,來與老爺進獻生辰禮物。」官吏罵道:「賊少死野囚軍!你那裏便興你東門員外西門員外?俺老爺當今一人之下,萬人之上,不論三台八位,不論公子王孫,誰敢在老爺府前這等稱呼?趂早靠後!」内中有認的來保的,便安撫來保説道:「此是新參的守門官吏,纔不多幾日,他不認的你,休怪。你要禀見老爺,等我請出翟大叔來。」這來保便向袖中取出一包銀子,重一兩,遞與那人。那人道:「我倒不消。你再添一份,與那兩個官吏,休和他一般見識。」來保連忙㧱出三包銀子來,每人一兩,都打發了。那官吏纔有些笑容兒,説道:「你旣是清河縣來的,且畧候候,等我領你先見翟管家。老爺纔従上清寳籙宫進了香囬來,書房内睡。」

良久,請到翟管家出來,穿着涼鞋淨襪,青絲絹道袍。來保見了,先磕下頭去。翟管家答禮相還,説道:「前者累你。你來與老爺進生辰擔禮來了?」來保先遞上一封揭帖,脚下人捧着一對南京尺頭,三十兩白金,説道:「家主西門慶,多上覆翟爹:無物表情,這些薄禮,與翟爹賞人。前者鹽客王四之事,多蒙翟爹費心。」翟謙道:「此禮我不當受。罷罷!我且收下。」來保又遞上太師壽禮帖兒,看了,還付與來保,吩咐把禮擡進來,到二門裏首伺候。原來二門西首有三間倒座,來往雜人都在那裏待茶。須臾,一個小童㧱了兩盞茶來,與來保吴主管喫了。

少頃,太師出廳。翟謙先禀知太師,太師然後令來保吴主管進見,跪於階下。翟謙先把壽禮揭帖,呈遞與太師觀看。來保吳主管各捧獻禮物。但見:

\begin{myquote}
黄烘烘金壺玉盞,白晃晃揀銀僊人,良工製造費工夫,巧匠鑽鑿人罕見;錦綉蟒衣,五彩奪目;南京紵緞,金碧交輝;湯羊羙酒,盡貼封皮;異菓時新,高堆盤榼。
\end{myquote}

太師如何不喜?便道:「這禮物决不好受的,你還將囬去。」於是慌了來保等,在下叩頭説道:「小的主人西門慶沒甚孝順,些小微物,進獻老爺賞人便了。」太師道:「旣是如此,令左右收了。」傍邊左右祇應人等,把禮物盡行收下去。太師又道:「前日那滄州客人王四等之事,我已差人下書與你巡撫侯爺説了,可見了分上不曾?」來保道:「蒙老爺天恩,書到,衆鹽客都牌提到鹽運司,與了勘合,都放出來了。」太師因向來保説道:「禮物我故收了。累次承你主人費心,無物可伸,如何是好?你主人身上可有甚官役?」來保道:「小的主人一介鄉民,有何官役!」太師道:「旣無官役,昨日朝廷欽賜了我幾張空名告身劄付,我安你主人在你那山東提刑所做個理刑副千户,頂補千户賀金的員缺,好不好?」來保慌的叩頭謝道:「蒙老爺莫大之恩,小的家主擧家粉首碎身,莫能報答。」於是喚堂候官擡書案過來,即時僉押了一道空名告身劄付,把西門慶名字填註上面,列銜「金吾衛衣左所副千户、山東等處提刑所理刑」。向來保道:「你二人替我進獻生辰禮物,多有辛苦。」因問:「後邊跪的,是你甚麽人?」來保纔待説是夥計,那吴主管向前道:「小的是西門慶舅子,名喚吳典恩。」太師道:「你旣是西門慶舅子,我觀你倒好個儀表。」喚堂候官取過一張劄付:「我安你在本䖏清河縣做個馹丞,倒也去的。」那吴典恩慌的磕頭如搗蒜。又取過一張劄付來,把來保名字填寫山東鄆王府,做了一名校尉。俱磕頭謝了,領了劄付。吩咐:「明日早晨,吏兵二部掛號,討勘合,限日上任應役。」又吩咐翟謙:「西廂房管待酒飯。討十兩銀子,與他二人做路費。」不在話下。

看官聽説:那時徽宗天下失政,奸臣當道,讒佞盈朝。高楊童蔡四個奸黨在朝中賣官鬵獄,賄賂公行,懸秤陞官,指方補價。夤緣鑽刺者,驟陞羙任;賢能廉直者,經歲不除。以致風俗頽敗,贓官汚吏,遍滿天下。役煩賦重,民窮盜起,天下騷然。不因奸佞居臺輔,合是中原血染人!

當下翟謙把來保吴主管邀到廂房管待,廚下大盤大碗,肉賽花糕,酒如琥珀,湯飯點心齊上,飽餐了一頓。翟謙向來保説:「我有一件事,央及你爹替我䖏處,未知你爹肯應承我否?」來保道:「翟爹説那裏話!蒙你老人家這等老爺前扶持看顧,不揀甚事,但肯吩咐,無不奉命。」翟謙道:「不瞞你説,我答應老爺,每日身邊止賤荆一人,常有疾病,通無所出。我年也將及四十,央及你爹,只説你那貴䖏有好人材女子,不拘十五六上下,替我尋一個送來。該多少財禮,我一一奉過去。」於是將一封人事並回書付與來保,又體己送二人五兩盤纏。來保再三不肯受,説道:「剛纔老爺上頭已賞過了,翟爹還收囬去。」翟謙道:「那是老爺的,此是我的,不必推辭。」當下喫畢酒飯。翟謙道:「如今我這裏替你着個辦事官,同你到下䖏,明早好往吏兵二部掛號,就領了勘合,好起身。省的你明日又來,途間往返了。我吩咐了去,部裏不敢遲滯了你文書。」即時喚了個辦事官,名喚李中友:「你與二位明日同到部裏,掛了號,討勘合,來囬我話。」那員官與來保吳典恩作辭,出的府門,來到天漢橋街上太白酒店内會話。來保管待酒飯,又與了李中友三兩銀子,約定明日絶早先到吏部,然後到兵部,都掛號,討了勘合。——聞得是太師老爺府裏,誰敢遲滯,顛倒奉行?金吾衛太尉朱勔,即時使印,僉了票帖,行下頭司,把來保填註在本䖏山東鄆王府當差。又㧱了個拜帖,囬翟管家。不消兩日,把事情幹得完備。有日僱頭口起身,星夜囬清河縣來報喜。正是:富貴必因奸巧得,功名全仗鄧通成!

且説一日三伏天氣,十分炎熱。西門慶在家中聚景堂中大捲棚内賞玩荷花,避暑飲酒。吴月娘與西門慶居上坐,諸妾與大姐都兩邊列坐。春梅迎春玉簫蘭香一般兒四個家楽,在傍彈唱。怎見的當日酒席?但見:

\begin{myquote}
盆栽綠草,瓶插紅花。水晶簾捲蝦鬚,雲母屏開孔雀。盤堆麟脯,佳人笑捧紫霞觴;盆浸冰桃,羙女高擎碧玉斝。食烹異品,菓獻時新。絃管謳歌,奏一派聲清韻羙;綺羅珠翠,擺兩行舞女歌兒。當筵象板撒紅牙,遍體舞裙補錦綉。消遣壶中閒日月,遨遊身外醉乾坤。
\end{myquote}

妻妾正飲酒中間,坐間不見了李瓶兒。月娘向綉春説道:「你娘往屋裏做甚麽哩,怎的不來喫酒?」綉春道:「我娘害肚裏痛,屋裏ぞ着哩!便來也。」月娘道:「還不快對他説去,休要ぞ着,來這裏坐着,聽一囬唱罷。」西門慶便問月娘怎的。月娘道:「李大姐忽然害肚裏痛,屋裏躺着哩。我剛纔使小丫頭請他去了。」因向玉樓道:「李大姐七八臨月,只怕攪撒了。」潘金蓮道:「大姐姐,他那裏是這個月!約他是八月裏孩子,還早哩。」西門慶道:「旣是早哩,使丫頭請你六娘來聽唱。」不一時,只見李瓶兒來到。月娘道:「只怕你掉了冷氣,你喫上鍾熱酒,管情就好了。」不一時,各人面前斟滿了酒。西門慶吩咐春梅:「你們唱個『人皆畏夏日』我聽。」那春梅等四個方纔箏排雁柱,阮跨鮫綃,啟朱唇,露皓齒,唱「人皆畏夏日」云云。

那李瓶兒在酒席上,只是把眉頭忔皺着,也沒等的唱完了,囬房中去了。月娘聽了詞曲,躭着心,使小玉房中瞧去。囬來報説:「六娘害肚裏疼,在炕上打滚哩!」慌了月娘道:「我説是時候,這六姐還強説早哩。還不喚小廝來,快請老娘去!」西門慶即令來安兒:「風跑,快請蔡老娘去。」於是連酒也喫不成,都來李瓶兒房中問他。月娘問道:「李大姐,你心裏覺怎的?」李瓶兒囬道:「大娘,我只心口連小肚子往下憋墜着疼。」月娘道:「你起來,休要睡着,只怕滚壞了胎。老娘請去了,便來也。」少頃,漸漸李瓶兒疼的緊了,月娘又問:「使了誰請老娘去了,這喒還不見來?」玳安道:「爹使了來安去了。」月娘罵道:「這囚根子!你還不快迎迎去?平白沒算計,使那小奴才去,有緊沒慢的。」西門慶叫玳安快騎了騾子趕了去。月娘道:「一個風火事,還像尋常慢條斯禮兒的。」

那潘金蓮見李瓶兒待養孩子,心中未免有幾分氣。在房裏看了一回,把孟玉樓拉出來,兩個站在西稍間簷柱兒底下那裏歇涼,一䖏説話。説道:「耶嚛嚛!緊着熱剌剌的擠了一屋子裏人,也不是養孩子,都看着下象胎哩!」

良久,只見蔡老娘進門,望衆人道:「那位主家奶奶?」李嬌兒道:「這位大娘哩。」那蔡老娘倒身磕頭下去。月娘道:「姥姥,生受。你怎的這喒纔來?」蔡老娘道:「你老人家聽我告訴:

\begin{myquote}
我做老娘姓蔡,兩隻脚兒能快。

身穿怪綠喬紅,各樣䯼髻歪戴。

嵌絲環子鮮明,閃黄手帕符㩟。

入門利市花紅,坐下就要管待。

不拘貴宅嬌娘,那管皇親國太。

教他任意端詳,被他腿衣㓦劃。

横生就用刀割,難產須將拳揣。

不管臍帶胞衣,着忙用手撕壞。

活時來洗三朝,死了走的偏快。

因此主顧偏多,請的時常不在。」
\end{myquote}

月娘道:「你且休閑説。請看這位娘子,敢待生養也。」蔡老娘向牀前摸了摸李瓶兒身上,説道:「是時候了。」問大娘:「預備下绷褯草紙不曾?」月娘道:「有。」便教小玉:「往我房中快取去。」

且説玉樓見老娘進門,便向金蓮説:「蔡老娘來了,喒不往屋裏看看去?」那金蓮一面不是一面説道:「你要看你去,我是不看他。他是有孩子的姐姐,又有時運,人怎的不看他?頭裏我自不是,説了句話兒,見他不是這個月的孩子,只怕是八月裏的,教大姐姐白搶白相我,想起來好沒來由,倒惱了我這半日。」玉樓道:「我也只説他是六月裏孩子。」金蓮道:「這囬連你也韶刀了!我和你恁算:他従去年八月來,又不是黄花女兒,當年懷、入門養,一個後婚老婆,漢子不知見過了多少,也一兩個月纔生胎,就認做是喒家孩子!我説差了?若是八月裏孩兒,還有喒家些影兒。若是六月的,䠕小板凳兒糊險道神——還差着一帽頭子哩!失迷了家鄉——那裏尋犢兒去?」正説着,只見小玉抱着草紙绷接並小褥子兒來。孟玉樓道:「此是大姐姐預備下他早晚臨月用的物件兒,今日且借來應急兒。」金蓮道:「一個是大老婆,一個是小老婆,明日兩個對養。十分養不出來,零碎出來也罷。俺們是買了個母鷄不下疍,莫不殺了我不成?」又道:「仰着合着,沒的狗咬尿胞——虚喜歡!」玉樓道:「五姐是甚麽話!」以後見他説話兒出來有些不防頭腦,只低着頭弄裙子,並不作聲應答他。潘金蓮用手扶着庭柱兒,一隻脚跐着門檻兒,口裏嗑着瓜子兒。只見孫雪娥聽見李瓶兒前邊養孩子,後邊慌慌張張一步一跌走來觀看。不防黑影裏被臺基險些不曾絆了一跤。金蓮看見,叫玉樓:「你看,獻勤的小婦奴才!你慢慢走,慌怎的?搶命哩!黑影子拌倒了,磕了牙也是錢。姐姐,賣蘿蔔的拉鹽擔子攘——鹹嘈心!養下孩子來,明日賞你這小婦一個紗帽戴。」

良久,只聽房裏呱的一聲,養下來了。蔡老娘道:「對當家的老爹説,討喜錢,分娩了一位哥兒。」吳月娘報與西門慶。西門慶慌的連忙洗手,天地祖先位下滿爐降香,告許一百二十分清醮,要祈子母平安,臨盆有慶,坐草無虞。這潘金蓮聽見生下孩子來了,合家歡喜亂成一塊,越發怒氣生,走去了房裏,自閉門户,向牀上哭去了。時宣和四年戊申六月廿一日也。正是:不如意處常八九,可與人言無二三。這蔡老娘收拾孩兒,咬去臍带,埋畢衣胞,熬了些定心湯,打發李瓶兒喫了,安頓孩兒停當。月娘讓老娘後邊管待酒飯。臨去,西門慶與了他五兩一錠銀子,許洗三朝來還與他一疋緞子。這蔡老娘千恩萬謝出門。

當日西門慶進房去,見一個滿抱的孩子,生的甚是白淨,心中十分歡喜,合家無不欣悦。晚夕就在李瓶兒牀房中歇了,不住來看孩兒。次日,巴天不明早起來,㧱十副方盒,使小廝各親戚鄰友處分投送喜麵。應伯爵謝希大聽見西門慶生了子,送喜麵來,慌的兩步做一步走來賀喜。西門慶留他捲棚内喫麵。剛打發去了,正在廳上亂着,使小廝呌媒人來尋養娘看奶孩兒。忽有薛嫂兒領了個奶子來,原是小人家媳婦兒,年三十歲,新近丢了孩兒,不上一個月。男子漢當軍,過不的,恐出征去無人養贍,只要六兩銀子,要賣他。月娘見他生的乾淨,對西門慶説,兌了六兩銀子留下,起名如意兒,教他早晚看奶哥兒。又把老馮呌來暗房中使喚,每月與他五錢銀子,管顧他衣服。

正熱鬧,一日忽有平安報:「來保吴主管在東京囬還,現在門首下頭口。」不一時,二人進來,見了西門慶報喜。西門慶問:「喜従何來?」二人悉把到東京見蔡太師進禮一節,従頭至尾訴説一遍:「老爺見了禮物甚喜,説道:『我累次受你主人禮太多,無可補報。』因問爹原祖上有甚差事。小的説一介鄉民,並無寸役在身。太師老爺説:朝廷欽賞了他幾張空名誥身劄付,與了爹一張,填寫爹名字在上,填註在金吾衛副千户之職,就委差的在本䖏提刑所理刑,頂補賀老爹員缺。把小的做了鐵鈴衛校尉,填註鄆王府當差。吴主管陞做本縣馹丞。」於是把一樣三張印信劄付並吏兵二部勘合,並誥身都取出來放在桌上,與西門慶觀看。西門慶看見上面銜着許多印信,朝廷欽依事例,果然他是副千户之職,不覺歡従額角眉尖出,喜向腮邊笑臉生。便把朝廷明降,㧱到後邊與吴月娘衆人觀看説:「太師老爺擡擧我,陞我做金吾衛副千户,居五品大夫之職。你頂受五花官誥,坐七香車,做了夫人。又把吴主管㩦帶做了馹丞,來保做了鄆王府校尉。吴神僊相我不少紗帽戴,有平地登雲之喜,今日果然。不上半月,兩樁喜事都應驗了。」又對月娘説:「李大姐養的這孩兒,甚是脚硬,到三日洗了三,就起名呌做官哥兒罷。」把朝廷明降與月娘看了。來保進來與月娘衆人磕頭,説了囬話。西門慶吩咐:「明日早把文書下到提刑所衙門裏與夏提刑知會了。」吴主管明日早下文書到本縣,作辭西門慶囬家去了。

到次日洗三畢,衆親鄰朋友一概都知西門慶第六個娘子新添了娃兒,未過三日,就有如此羙事:官祿臨門,平地做了千户之職,誰人不來趨附?送禮慶賀,人來人去,一日不断頭。常言:時來誰不來,時不來誰來?正是:時來頑鐵有光輝,運退真金無豔色!

畢竟未知後來如何,且聽下囬分解。

\part*{夢梅館本《金瓶梅詞話》卷之四}
\addcontentsline{toc}{part}{夢梅館本《金瓶梅詞話》卷之四}

