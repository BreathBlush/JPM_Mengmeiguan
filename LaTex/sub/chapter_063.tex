\includepdf[pages={125,126},fitpaper=false]{tst.pdf}
\chapter*{第六十三囬 親朋祭奠開筵宴 西門慶觀戲感李瓶}
\addcontentsline{toc}{chapter}{第六十三囬 親朋祭奠開筵宴 西門慶觀戲感李瓶}
\markboth{第六十三囬 親朋祭奠開筵宴 西門慶觀戲感李瓶}{第六十三囬 親朋祭奠開筵宴 西門慶觀戲感李瓶}

十二瑤臺七寳欄,瓊花落後再開難!

龍鬚煮薬醫無效,熊膽為丸晒未乾。

蓉帳夜愁紅燭冷,紙窗秋暮翠衾寒。

應憐失伴孤飛雁,霜落風高一影單。

話說當日應伯爵勸解了西門慶一囬,拭淚而止。令小廝後邊看飯去了。不一時,吳大舅吴二舅都到了。靈前行畢禮,與西門慶作揖,道及煩惱之意。請至廂房中,與衆人同坐。玳安走至後邊,向月娘説:「如何?我説娘們不信,怎的應二爹來了,一席話説的爹就喫飯了?」金蓮道:「你這賊,積年久慣的囚根子!鎮日在外邊替他做牵頭,有個㧱不住他性兒的!」玳安道:「従小兒答應主子,不知心腹?」月娘問道:「那幾個在廂房子裏坐着,陪他喫飯?」玳安道:「大舅二舅剛纔來,和溫師父,連應二爹、謝爹、韓夥計、姐夫共爹八位人哩。」月娘道:「請你姐夫來後邊喫罷了,也擠在上頭?」玳安道:「姐夫坐下了。」月娘吩咐:「你和小廝往廚房裏㧱飯去。你另㧱甌兒㧱粥與他喫。怕清早晨不喫飯。」玳安道:「再有誰,止我在家!都使出報丧、燒紙、買東西。王經又使他往張親家爹那裏借雲板去了。」月娘道:「書童那奴才,和他㧱去是的,怕打了他紗帽展翅兒?」玳安道:「書童和畫童兩個,在靈前一個打磬,一個伺候焚香燒紙哩。春鴻爹又使他跟賁四換絹去了——嫌絹不好,要換六錢一疋的絹破孝。」月娘道:「論起來,五錢銀子的也罷,又巴巴兒換去!」又道:「你呌下畫童兒那小奴才,和他快㧱去,只顧還挨磨甚麽?」玳安於是和畫童兩個大盤大碗㧱到前邊,安放八僊桌席。衆人正喫着飯,只見平安㧱進手本來禀:「衙門中夏老爹,差冩字的送了三班軍衛來這裏答應,討囬帖。」西門慶看了放下,吩咐:「討三錢銀子賞他。寫期服生雙囬帖兒,囬你夏老爹:多謝了!」

一面喫畢飯,收了家伙。只見來保請的畫師韓先生來到。西門慶與他行畢禮,説道:「煩先生揭白傳個神子兒。」那韓先生道:「小人理會得了。」吴大舅道:「動手遲了些,倒只怕面容改了。」韓先生道:「也不妨,就是揭白也傳得。」正喫茶畢,忽見平安來報:「門外花大舅來了。」西門慶陪花子由靈前哭涕了一回,見畢禮數,與衆人一䖏。因問:「甚麽時候?」西門慶道:「正丑時断氣。臨死,還伶伶俐俐説話兒。剛睡下,丫頭起來瞧,就沒了氣兒。」因見韓先生傍邊小童㧱着屏插,袖中取出描筆顔色來,花子由道:「姐夫如今要傳個神子?」西門慶道:「我心裏疼他,少不的留了個影像兒,早晚看着題念他題兒。」一面吩咐後邊堂客躲開,掀起帳子,領韓先生和花大舅衆人到跟前。這韓先生用手揭起千秋旛,用五輪八寳玩着兩點神水,打一觀看,見李瓶兒勒着鴉青手帕,雖故久病,其顔色如生,姿容不改,黄懨懨的,嘴唇兒紅潤可愛。那西門慶由不的掩淚而哭。當下來保與琴童在傍捧着屏插、顔色,韓先生一見就知道了。衆人圍着他瞧畫,應伯爵便道:「先生,此是病容,平昔好時,比此面容飽滿,姿容秀麗。」韓先生道:「不須尊長吩咐,小人知道。不敢就問老爹:此位老夫人,前者五月初一日,曾在嶽廟裏燒香,親見一面,可是否?」西門慶道:「正是。那時還好哩。先生,你用心想着,傳畫一軸大影,一軸半身,靈前供養。我送先生一疋緞子,上蓋十兩銀子。」韓先生道:「老爹吩咐,小人無不用心。」須臾,描染出個半身來,端的玉貌幽花秀麗,肌膚嫩玉生香。㧱與衆人瞧,就是一幅羙人圖兒。西門慶看了,吩咐玳安:「㧱到後邊與你娘們瞧瞧去,看好不好。有那些兒不是,説來好改。」這玳安㧱到後邊,向月娘道:「爹説教娘們瞧瞧六娘這影,看畫的如何。那些兒不像,説出去敎韓先生好改。」月娘道:「成精鼓搗,人也不知死到那裏去了,又描起影來了,看畫的那些兒不像?」潘金蓮接過來道:「那個是他的兒女?畫下影、傳下神來,好替他磕頭禮拜!到明日六個老婆死了,畫下六個影纔好。」孟玉樓和李嬌兒㧱過來觀看,説道:「大娘你來看,李大姐這影,倒像似好時那等模樣,打扮的鮮鮮兒,只是嘴唇略扁了些兒。」月娘道:「這左邊額頭略低了些兒。他的眉角,比這眉角兒還彎些。虧這漢子揭白,怎的畫來!」玳安道:「他在廟上曾見過六娘一面,剛纔想着,就畫到這等模樣。」

少頃,只見王經進來説道:「娘們看了快敎㧱出去。喬親家爹來了,等喬親家爹瞧哩。」玳安走到前邊,吩咐韓先生道:「這裏邊説來,嘴唇略扁了些,左額角稍低,眉還略放彎着些兒。」韓先生道:「這個不打緊。」隨即取描筆改正了,呈與喬爹瞧。喬大戶道:「親家母這幅尊像,是畫得通,只是少了口氣兒!」西門慶滿心歡喜,一面遞了三鍾酒與韓先生,管待了酒飯;紅漆盤捧出一疋尺頭、十兩白金與韓先生,敎他:「先趲造出半身來,就要挂;大影不悮出殯就是了。俱要用大青大綠,珠翠圍髮冠,大紅通袖五彩遍地金袍兒、百花裙。衢花綾裱,象牙軸頭。」韓先生道:「不必吩咐,小人知道。」領了銀子,敎小童㧱着插屏,拜辭出門。喬大户與衆人又看了一囬做成的棺木,便道:「親家母今日小殮罷了。」西門慶道:「如今仵作行人來,就小殮。大殮還等到三日。」喬大户喫畢茶,就告辭起身去了。

不一時,仵作行人來伺候,紙劄打捲,鋪下衣衾。西門慶要親與他開光明,強着陳經濟做孝子,與他抿了目。西門慶旋尋出一顆胡珠,安放在他口裏。登時小殮停當,照前停放端正,放下帳子,合家大小哭了一場。來興又早冥衣舖裏,做了四座堆金瀝粉侍奉的捧盆巾盥櫛毛女兒,都是珠子纓絡兒,銀鑲墜兒,似真的色綾衣服,一邊兩座擺下。靈前供養的彝爐、商瓶、燭臺、香盒,教錫匠打造停當,擺在桌上,耀日爭輝。又兑了十兩銀子,教銀匠打了三付銀爵盞。又在廂房中與應伯爵定管丧禮簿籍:先兑了五百兩銀子、一百弔錢來,委付與韓夥計管帳;賁四與來興兒專管大小買辦,兼管外廚房;應伯爵謝希大溫秀才甘夥計四人,輪番陪侍往來弔客;崔本專管付孝帳;來保管外庫房;王經管酒房;春鴻與畫童專管靈前伺候;平安逐日與四名排軍,單管人來打雲板,捧香紙;又是一個寫字的,帶領四名排軍,在大門首記門簿,值念經日期打傘相搭挑旛幢,無事把門。都派委已定,寫了告示,貼在影壁上,各遵守去訖。只見皇莊上薛内相差人送了六十根杉條、三十條毛竹、三百領蘆蓆、一百條麻䋲,㧱帖兒與西門慶瞧。連忙賞了來人五錢銀子,㧱期服生囬帖兒,打發去了。吩咐搭採匠把棚起脊,搭大着些,㽞兩個門走,把影壁夾在中間。前廚房内還搭三間罩棚,大門首紮七間榜棚,請報恩寺十二衆僧人,先念〈倒頭經〉。每日兩個茶酒,在茶坊内伺候茶水。外廚房兩名廚役,答應各項飯食。花大舅吴二舅坐了一囬,起身去了。西門慶教溫秀才起孝帖兒,要開刊去,令寫:「荆婦奄逝。」溫秀才悄悄㧱與應伯爵看,伯爵道:「這個理上説不通。現有如今吴家嫂子在正室,如何使得?這一個出去,不被人議論,就是吴大哥心内也不自在。等我慢慢再與他講,你且休要寫着。」陪坐至晚,各散歸家去了。西門慶晚夕也不進後邊去,就在李瓶兒靈傍邊裝起一張凉牀,㧱圍屏圍着,鋪陳停當,獨自宿歇。有春鴻、書童兒,近前伏侍。天明便往月娘房裏梳洗。穿戴了裁縫做的白唐巾、孝冠、孝衣、白絨襪、白履鞋,絰帶隨身。

第二日清晨,夏提刑就來探丧弔問,慰其節哀。西門慶還禮畢,溫秀才相陪,待茶而去。到門首吩咐寫字的:「好生在此答應!查有不到的排軍,呈來衙門内懲治。」説畢,騎馬往衙門中去了。西門慶令溫秀才發帖兒,差人請各親眷,三日做齋誦經,早來赴會。後晌鋪排來收拾道場,懸掛佛像,不必細説。那日院中吴銀兒打聽得知,坐轎子來靈前哭泣上紙。引去到後邊,月娘相接,吳銀兒與月娘磕頭,哭道:「六娘沒了,我通一字不知。就沒個人兒和我説聲兒,可憐傷感人也!」孟玉樓道:「你是他乾女兒,他不好了這些時,你就不來看他看兒?」吴銀兒道:「好三娘,我但知道,有個不來看的?説句假就死了。委實不知道!」月娘道:「你不來看你娘,他還掛牵着你,留了件東西兒與你做一念兒,我替你收着哩!」因令小玉:「你取出來與銀姐兒看。」那小玉走到裏間,取出包袱,内包着一套緞子衣服、兩根金頭簪兒,一件金花兒。把吴銀兒哭的淚人也相似,説道:「我早知他老人家不好,也來伏侍兩日兒!」説着,一面拜謝了月娘。月娘待茶與他喫,留他過了三日去。

到三日,和尚打起磬子,揚旛,道場誦經,挑出紙錢去。合家大小都披麻帶孝。陳經濟穿重孝,絰巾,佛前拜禮。街坊鄰舍,親朋官長,來弔問上紙祭奠者,不計其數。陰陽徐先生早來伺候大殮。祭告已畢,擡屍入棺。西門慶教吳月娘,又尋出他四套上色衣服來裝在棺内,四角安放了四錠小銀子兒依着。花子由説:「姐夫,倒不消安他在裏面。金銀日久定要出世,倒非久遠之計。」西門慶不肯,安放如故。放下一七星板,閣上紫蓋。仵作四面用長命釘一齊釘起來,一家大小放聲號哭。西門慶亦哭的獃了,口口聲聲哭呌:「我的年少的姐姐,再不得見你了!」良久哭畢,管待徐先生齋饌,打發去了。洒花米,貼「神燈安真」四個大字在靈前。親朋夥計人等,都是巾帶孝服。行香之時,門首一片皆白。溫秀才擧薦北邊杜中書來題銘旌,名子春,號雲野,原侍真宗寜和殿,今坐閑在家。西門慶備金幣請來,在捲棚内備菓盒,西門慶親遞三盃酒。應伯爵與溫秀才相陪,鋪大紅官紵題旌。西門慶要寫:「詔封錦衣西門恭人李氏柩」十一字。伯爵再三不肯,説:「現有正室夫人在,如何使得?」杜中書道:「旣曾生過子,於禮也無礙。」講了半日,去了「恭」字,改了「室人」。溫秀才道:「恭人係命婦,有爵;室人乃室内之人,只是個渾然通常之稱。」於是用白粉題畢,「詔封」二字貼了金,懸於靈前;又題了神主。叩謝杜中書,管待酒饌,拜辭而去。

那日喬大户、吳大舅、花大舅、門外韓姨夫、沈姨夫,各家都是三牲祭桌來燒紙。喬大户娘子並吴大妗子、二妗子、花大妗子,坐轎子來弔丧,祭祀哭泣。月娘等皆孝髻、頭鬚繫腰、麻布孝裙,出來囬禮擧哀,讓後邊待茶擺齋。惟花大妗子與花大舅便是重孝,直身道袍兒,餘者都是輕孝。那日院中李桂姐打聽得知,坐轎子也來上紙。看見吴銀兒在這裏,説道:「你幾時來的?怎的也不會我會兒?好人兒,原來只顧你!」吴銀兒道:「我也不知道娘沒了,早知道也來看看兒。」月娘後邊管待,俱不必細説。

須臾過了三日,看看到首七。正是報恩寺十六衆上僧,黄僧官為首座,引領做水陸道場,誦〈法華經〉,拜三昧水懺。親朋夥計,無不畢集。那日,玉皇廟吴道官來上紙弔孝,攬二七經。西門慶留在捲棚内,衆人喫齋。忽見小廝來報:「韓先生送半身影來。」衆人觀看,但見:頭戴金翠圍冠,雙鳳珠子挑牌,大紅粧花袍兒,白馥馥臉兒,儼然如生時一般。西門慶見了,滿心歡喜,懸掛棺材頭上。衆人無不誇獎:「只少口氣兒!」一面讓捲棚喫齋,囑付:「大影比這還要加工夫些。」韓先生道:「小人隨筆潤色,豈敢粗心。」西門慶厚賞而去。午間,喬大户那邊來上祭:猪羊祭品,喫看桌面,高頂簇盤,五老錠勝,方糖樹菓,减碟湯飯,五牲看碗,金山、銀山,緞帛綵繒,冥紙炷香,共約五十餘擡,地弔高蹺,鑼鼓細楽吹打,纓絡打挑喧闐而至。官堂客約許多人,陰陽生讀祝。西門慶與陳經濟穿孝衣在靈前還禮。應伯爵、謝希大,與溫秀才、甘夥計等,迎待賓客。那日喬大户邀了尚擧人、朱臺官、吴大舅、劉學官、范千户、段親家七八位親朋,各在靈前上香。三獻已畢,俱跪聽讀祝文曰:

「維政和七年,歲次丁酉,九月庚申朔,越二十二日辛巳,眷生喬洪等,謹以剛鬣柔毛庶羞之奠,致祭於

故親家母西門孺人李氏之靈曰:嗚呼,孺人之性,寬裕温良,治家勤儉,御衆慈祥。克全婦道,譽動鄉邦。閨閫之秀,蘭蕙之芳。夙配君子,效聘鸞凰。撫字子性,以義以方。效顰大德,以柔以良。施懿範於家室,悚和粹於娣嫜。藍玉已種,浦珠已光。正期諧琴瑟於有永,享彌壽於無疆。胡為一疾,夢断黄粱。善人之殁,孰不哀傷!弱女襁褓,沐愛姻嬙。不期中道,天不従願,鴛伴失行。恨隔幽冥,莫覩行藏。悠悠情誼,寓此一觴。靈其有知,來格來歆。尚饗!」

官客祭畢,回禮畢,讓捲棚内,自有桌席管待,不在話下。然後喬大户娘子、崔親家母、朱臺官娘子、尚擧人娘子、段大姐,衆堂客女眷祭奠,地弔鑼鼓,靈前弔鬼判隊舞,戧將響楽。吳月娘陪着哭畢,請去後邊待茶設席,三湯五割,俱不必細説。

西門慶正在捲棚内陪人喫酒,忽聽前邊打的雲板響,答應的慌慌張張進來禀報:「本府胡爺上紙來了,在門首下轎子。」慌的西門慶連忙穿孝衣,靈前伺候。即使溫秀才衣巾素服出迎,前廳伺候換衣裳。左右先捧進香紙,然後胡府尹素服金帶纔進來。許多官吏圍隨,扶衣搊帶,奔走不暇。到於靈前,春鴻跪着,捧的香高高的。上了香,展拜兩禮。西門慶便道:「老先生請起,多有勞動!」連忙下來回了禮。胡府尹道:「弔遲、弔遲!令夫人幾時沒了?學生昨日纔知。」西門慶道:「不想簉室一疾不救,辱承老先生枉弔!」溫秀才在傍作揖畢,與西門慶兩邊列坐。待茶一盃,胡府尹起身。溫秀才送出大門,上轎而去。上祭人喫至後晌時分方散。

到第二日,院中鄭愛月兒家來上紙。愛月兒下了轎子,穿着白雲絹對衿襖兒,藍羅裙子,頭上勒着珠子箍兒,白挑線汗巾子,進至靈前燒了紙。月娘見他擡了八盤餅饊,三牲湯飯來祭奠,連忙討了一疋整絹孝裙與他。——吴銀兒與李桂姐都是三錢奠儀。告西門慶説,西門慶道:「值甚麽,每人都與他一疋整絹頭鬚繫腰。」月娘邀到後邊房兒裏擺茶管待,過夜。

晚夕,親朋夥計來伴宿,呌了一起海鹽子弟搬演戲文。李銘、吴惠、鄭奉、鄭春,都在這裏答應。晚夕西門慶在大棚内放十五張桌席,為首的就是喬大户、吳大舅、吴二舅、花大舅、沈姨夫、韓姨夫、倪秀才、溫秀才、任醫官、李智、黃四、應伯爵、謝希大、祝日念、孫寡嘴、白來創、常時節、傅自新、韓道國、甘出身、賁地傳、吳舜臣兩個外甥,還有街坊六七位人,都是十菜五菓開桌兒。點起十數枝高檠大燭來,廳上垂下簾。堂客便在靈前圍着圍屏,放桌席,往外觀戲。當時衆人祭奠畢,西門慶與經濟回畢禮,安席上坐。下邊戲子打動鑼鼓,搬演的是「韋臯玉簫女兩世姻緣」〈玉環記〉。西門慶分派四名排軍單管下邊㧱盤,琴童、棋童、畫童、來安,四個單管下菓兒,李銘、吴惠、鄭奉、鄭春,四個小優兒席上斟酒。不一時弔場,生扮韋臯,唱了一回下去。貼旦扮玉簫,又唱了一回下去。廚房裏廚役上湯飯、割鵝。應伯爵便向西門慶説:「我聞的院裏姐兒三個在這裏,何不請出來與喬老親家老舅席上遞盃酒兒?他到是會看戲,又倒便益了他!」西門慶便使玳安進入説去,請他姐兒三個出來!喬大户道:「這個却不當,他來弔丧,如何敎他遞起酒來?」伯爵道:「老親家你不知。像這樣小淫婦兒,别要閒着他。快與我牽出來,你説應二爹説,六娘沒了,只當行孝順,也該與俺每人遞盃酒兒。」玳安進去半日説:「聽見應二爹在坐,都不出來哩。」伯爵道:「旣恁説,我去罷。」走了兩步,又囬坐下。西門慶笑道:「你怎的又囬了?」伯爵道:「我有心待要扯那三個小淫婦出來,等我罵兩句,出了我氣,我纔去。」落後又使了玳安請了一遍,那三個纔慢條條出來,都一色穿着白綾對衿襖兒,藍緞裙子,向席上不端不正拜了拜兒,笑嘻嘻立在傍邊。應伯爵道:「俺們在這裏,你如何只顧推三阻四,不肯出來?」那三個也不答應,向上邊遞了囬酒,另設一席坐着。下邊鼓楽響動,関目上來,生扮韋臯,淨扮包知水,同到勾欄裏玉簫家來。那媽兒出來迎接。包知水道:「你去呌那姐兒出來。」媽云:「包官人,你好不看輕人,俺女兒等閒不便出來,説不的一個請字兒,你如何説『呌他出來』?」那李桂姐向席上笑道:「這個姓包的就和應花子一般,就是個不知趣的蹇衛兒!」伯爵道:「小淫婦!我不知趣,你家媽兒喜歡我?」桂姐道:「他喜歡你?過一邊兒。」西門慶道:「且看戲罷,且説甚麽!再言語,罰一大盃酒。」那伯爵纔不言語了。那戲子又做了一囬,並下。

這裏廳内左邊弔簾子看戲的,是吴大妗子、二妗子、楊姑娘、潘媽媽、吴大姨、孟大姨、吴舜臣媳婦鄭三姐、段大姐,並本家月娘衆姊妹,右邊弔簾子看戲的,是春梅、玉簫、蘭香、迎春、小玉,都擠着觀看。那打茶的鄭紀,正㧱着一盤菓仁泡茶従簾下頭過。被春梅呌住,問道:「㧱茶與誰喫?」鄭紀道:「那邊大妗子娘們要喫。」這春梅取一盞在手。不想小玉聽見下邊扮戲的旦兒名子也叫玉簫,便把玉簫拉着説道:「淫婦,你的孤老漢子來了,鴇子呌你接客哩。你還不出去!」使力往下一推,直推出簾子外。春梅手裏㧱着茶,推潑一身。罵玉簫:「怪淫婦,不知甚麽張致,都頑的這等,把人的茶都推潑了。早是沒曾打碎盞兒。」西門慶聽得,使下來安兒來問:「誰在裏面喧嚷?」春梅坐在椅上道:「你去就説:玉簫浪淫婦面見了漢子,這等浪相。」那西門慶問了一囬,亂着席上遞酒,就罷了。月娘便走過那邊數落小玉:「你出來這一日,也往屋裏瞧瞧去。都在這裏,屋裏有誰?」小玉道:「大姐剛纔後邊去的。兩位師父也在屋裏坐着。」月娘道:「敎你們賊狗胎在這裏看看,就恁惹是招非的!」春梅見月娘過來,連忙立起身來説道:「娘,你問他,都一個個只像有風病來,狂的通沒些成色兒,嘻嘻哈哈,也不顧人看見。」那月娘數落了一囬,仍過那邊去了。

那時喬大户與倪秀才先起身去了。沈姨夫與任醫官、韓姨夫,也要起身,被應伯爵攔住道:「東家,你也説聲兒。俺們倒是朋友,不敢散;一個親家却要去?沈姨夫又不隔門,韓姨夫與任大人、花大舅,都在門外,這咱纔三更天氣,門也還未開,慌的甚麽?都來大坐囬兒,左右関目還未了哩。」西門慶又令小廝,提四罈麻姑酒放在面前,説:「列位,只了此四罈酒,我也不留了。」因㧱大賞鍾,放在吴大舅面前,説道:「那位離席破坐説起身者,任大舅擧罰。」於是衆人又復坐下了。西門慶令書童催促子弟快弔関目上來,吩咐:「揀着熱鬧䖏唱罷。」須臾打動鼓板,扮末的上來,請問西門慶:「小的『寄眞容』的那一摺,唱罷?」西門慶道:「我不管你,只要熱鬧。」貼旦扮玉簫,唱了一囬。西門慶看唱到「今生難會,因此上寄丹青」一句,忽想起李瓶兒病時模樣,不覺心中感觸起來,止不住眼中淚落,袖中不住取汗巾兒擦拭。又早被潘金蓮在簾内冷眼看見,指與月娘瞧,説道:「大娘,你看他,好個沒來頭的行貨子。如何喫着酒,看見扮戲的哭起來!」孟玉樓道:「你聰明一場,這些兒就不知道了?楽有悲歡離合,想必看見那一段兒觸着他心,他覷物思人,見鞍思馬,纔落淚來。」金蓮道:「我不信。打談的掉眼淚,替古人躭憂,這個都是虚。他若唱的我淚出來,我纔算他好戲子!」月娘道:「六姐,悄悄兒,咱們聽罷。」玉樓因向大妗子道:「俺六姐不知怎的,只好快説嘴。」

那戲子又做了一囬,約有五更時分,衆人齊起身。西門慶㧱大盃攔門遞酒,款留不住,俱送出門。看收了家伙,「留下戯箱,說明日有劉公公薛公公來祭奠,白日坐,還做一日。」衆戲子答應,管待了酒飯,歸下處歇去了。李銘等四個亦歸家不題。西門慶見天色已將曉,就歸後邊歇息去了。正是:得多少紅日映窗寒色淺,淡煙籠竹曙光微。

畢竟後來如何,且聽下囬分解。

