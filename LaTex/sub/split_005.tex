\chapter*{新鐫金瓶梅詞話}
\addcontentsline{toc}{chapter}{新刻金瓶梅詞話·詞曰·四貪詞}
\markboth{\titlename}{新刻金瓶梅詞話·詞曰·四貪詞}


\section*{詞曰}

\begin{myquote0}
閬苑瀛洲,金谷陵樓,筭不如茅舍清幽。野花綉地,莫也風流。也宜春,也宜夏,也宜秋。

酒熟堪あ,客至須留,更無榮無辱無憂。退閒一步,着甚來由。但倦時眠,渴時飲,醉時謳。

短短横牆,矮矮疎窻,忔い兒小小池塘。高低疊峯,綠水邊傍。也有些風,有些月,有些凉。

日用家常,竹几藤床,靠眼前水色山光。客來無酒,清話何妨。但細烹茶,熱烘盞,淺澆湯。

水竹之居,吾愛吾盧,石磷磷床砌堦除。軒窻隨意,小巧規模。却也清幽,也瀟灑,也寬舒。

懶散無拘,此等何如:倚闌干臨水觀魚。風花雪月,贏得功夫,好炷心香,說些話,讀些書。

淨掃塵埃,惜取蒼苔,任門前紅葉鋪堦。也堪圖畫,還也奇哉。有數株松,數竿竹,數枝梅。

花木栽培,取次教開,明朝事天自安排。知他富貴幾時來。且優遊,且隨分,且開懷。
\end{myquote0}

\newpage\section*{四貪詞}
%\addcontentsline{toc}{chapter}{四貪詞}
%\markboth{\titlename}{四貪詞}

\hspace*{1em}酒

\begin{myquote0}
酒損精神破喪家,語言無狀鬧喧嘩。疎親慢友多由你,背義忘恩盡是他。

切須戒,飲流霞。若能依此寳無差。失却萬事皆因此,今後逄賔只待茶。
\end{myquote0}

\hspace*{1em}色

\begin{myquote0}
休愛綠髩羙朱顔,少貪紅粉翠花鈿。損身害命多嬌態,傾國傾城色更鮮。

莫戀此,養丹田。人能寡慾壽長年。従今罷却閒風月,紙帳梅花獨自眠。
\end{myquote0}

\hspace*{1em}財

\begin{myquote0}
錢帛金珠籠内收,若非公道少貪求。親朋道義因財失,父子懷情為利休。

急縮手,且抽頭。免使身心晝夜愁。兒孫自有兒孫福,莫與兒孫作遠憂。
\end{myquote0}

\hspace*{1em}氣

\begin{myquote0}
莫使強梁逞技能,揮拳捰袖弄精神。一時怒發無明火,到後憂煎祸及身。

莫太過,免災迍。勸君凡事放寬情。合撒手時須撒手,得饒人處且饒人。
\end{myquote0}

\part*{夢梅館校本《金瓶梅詞話》卷之一}
\addcontentsline{toc}{part}{夢梅館校本《金瓶梅詞話》卷之一}

