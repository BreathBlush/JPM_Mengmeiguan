\includepdf[pages={15,16},fitpaper=false]{tst.pdf}
\chapter*{第八囬 \\潘金蓮永夜盼西門慶 燒夫靈和尚聽淫聲}
\addcontentsline{toc}{chapter}{第八囬 潘金蓮永夜盼西門慶 燒夫靈和尚聽淫聲}
\markboth{{\titlename}卷之一}{第八囬 潘金蓮永夜盼西門慶 燒夫靈和尚聽淫聲}


\begin{myquote}
靜悄房櫳獨自猜,鴛鴦失伴信音乖。

臂上粉香猶未泯,牀頭揪面暗塵埋。

芳容消瘦虚鸞鏡,雲鬢鬅鬆墜玉釵。

駿驥不來勞望眼,空餘鴛枕淚盈腮。
\end{myquote}

話說西門慶自従娶了玉樓在家,燕爾新婚,如膠似漆。又遇着陳宅那邊使了文嫂兒來通信,六月十二日就要娶大姐過門。西門慶促忙促急,趲造不出牀來,就把孟玉樓陪來的一張南京描金彩漆拔步牀,陪了大姐。三朝九日,足亂了約一個月多,不曾徃潘金蓮家去。把那婦人每日門兒倚遍,眼兒望穿。使王婆徃他門首去了兩遍。門首小廝常見王婆,知道是潘金蓮使來的,都不理他,只說:「大官人不得閒哩。」婦人盼他急的緊,只見婆子囬了婦人,婦人又打罵小女迎兒街上去尋他。那小妮子怎敢入他那深宅大院裏去,只在門首踅探了一兩遍,不見西門慶,就囬來了。來家又被婦人噦罵在臉上,打在臉上,怪他没用,便要教他跪着。餓到晌午,又不與他飯喫。

那時正値三伏,天道十分炎熱。婦人在房中害熱,吩咐迎兒熱下水,伺候澡盆,要洗澡。又做了一籠夸餡肉角兒,等西門慶來喫。身上只着薄纊短衫,坐在小杌上,盼不見西門慶來到,嘴谷都的罵了幾句負心賊。無情無緒,悶悶不語。用纖手向脚上脱下兩隻紅繡鞋兒來,試打一個相思卦,看西門慶來不來。正是:逢人不敢高聲語,暗卜金錢問遠人。有〔山坡羊〕爲證:

\begin{myquote}
凌波羅襪,天然生下,紅雲染就相思卦。似藕生芽,如蓮卸花。怎生纏得些娘大!柳條兒比來剛半扠。他,不念咱;咱,想念他!

簾兒私下,門兒悄呀,空教奴被兒裏呌着他那名兒罵:你怎戀煙花,不來我家!奴眉兒淡淡教誰畫?何處綠楊拴繫馬?他,辜負咱;咱,念戀他!
\end{myquote}

當下婦人打了一囘相思卦,見西門慶不來了,不覺困倦上來,就歪在牀上盹睡着了。約一個時辰醒來,心中正沒好氣。迎兒問:「熱了水,娘洗澡也不洗?」婦人便問:「角兒蒸熟了?拏來我看。」迎兒連忙拏到房中。婦人用纖手一數,原做下一扇籠三十個角兒,翻來覆去只數了二十九個,少了一個角兒。便問:「徃那裏去了?」迎兒道:「我並没看見,只怕娘錯數了。」婦人道:「我親數了兩遍,三十個角兒,要等你爹來喫。你如何偸喫了一個?好嬌態淫婦奴才!你害饞癆饞痞,心裏要想這個角兒喫?你大碗小碗𠳹搗不下飯去?我做下的孝順你來!」於是不由分說,把這小妮子跣剝去了身上衣服,㧱馬鞭子下手打了二三十下,打的妮子殺猪般也似叫。問着他:「你不承認,我定打下百數!」打的妮子急了,說道:「娘休打,是我害餓的慌,偸喫了一個。」婦人道:「你偸了,如何賴我錯數了?眼看着就是個牢頭祸根淫婦!有那亡八在時,輕學重告;今日徃那裏去了?還在我跟前弄神弄鬼!我只把你這牢頭淫婦打下你下截來!」打了一囬,穿上小衣,放他起來,吩咐在旁打扇。打了一囬扇,口中說道:「賊淫婦,你舒過臉來,等我掐你這皮臉兩下子。」那妮子眞個舒着臉,被婦人尖指甲掐了兩道血口子,纔饒了他。

良久,走到鏡臺前,従新粧點,出來門簾下站立。也是天假其便,只見西門慶家小廝玳安,夾着毡包,騎着馬,打婦人門首過的。婦人叫住他,問他徃何處去來。那小廝平日說話乖覺,常跟西門慶在婦人家行走,婦人常與他浸潤,他有甚不是,在西門慶面前替他說方便,以此和婦人徃來熟滑。一面下馬來,說道:「俺爹使我送人情,徃守備府裏去來。」婦人叫進門來問他:「你爹家中有甚事?如何一向不來傍個影兒看我一看?想必另續上了一個心甜的姊妹,把我做個網巾圈兒打靠後了。」玳安道:「俺爹再沒續上姊妹,只是這幾日家中事忙,不得脫身來看得六姨。」婦人道:「就是家中有事,那裏丢我恁個半月,音信不送一個兒!只是不放在心兒上。」因問玳安:「有甚麽事?你對我說。」那小廝嘻嘻只是笑,不肯說:「有樁事兒罷了,六姨只顧吹毛求疵問怎的?」婦人道:「好小油嘴兒!你不對我說,我就惱你一生。」小廝道:「我對六姨說,六姨休對爹說是我說的。」婦人道:「我不對他說便了。」玳安如此這般,把家中娶孟玉樓之事従頭至尾,告訴了一遍。這婦人不聽便罷,聽了由不的那裡眼中淚珠兒順着香腮流將下來。玳安慌了,便道:「六姨,你原來這等量窄。我故此不對你說;對你說,便就如此。」婦人倚定門兒,長歎了一口氣,說道:「玳安,你不知道,我與他從前已徃那樣恩情,今日如何一旦抛閃了!」止不住紛紛落下淚來。玳安道:「六姨,你何苦如此?家中俺娘也不管着他。」婦人便道:「玳安,你聽告訴!」另有前腔為證:

\begin{myquote}
「喬才心邪,不來一月,奴繡鴛衾曠了三十夜。他俏心兒别,俺癡心兒獃,不合將人十分熱。常言道,容易得來容易捨。興,過也;緣,分也!」
\end{myquote}

說畢,又哭了。玳安道:「六姨,你休哭,俺爹怕不的也只在這兩日頭,他生日待來也。你寫幾個字兒,等我替你捎去,與俺爹瞧看了,必然就來。」婦人道:「是必累你請的他來,到明日我做雙好鞋與你穿。我這裏也要等他來,與他上壽哩。他若不來,都在你小油嘴身上。他若是問起你來這裏做什麽,你怎生囬答他?」玳安道:「爹若問小的,只説在街上飲馬,六姨使王奶奶叫了我去,捎了這個柬帖兒,多上覆爹,好歹請爹過去哩。」婦人笑道:「你這小油嘴!倒是再來的紅娘,倒會成合事兒哩。」說畢,令迎兒把桌上蒸下的角兒裝了一碟兒,打發玳安兒喫茶。一面走入房中,取過一幅花箋,又輕撚玉管,款弄羊毛,須臾寫了一首〔寄生草〕。詞曰:
\begin{myquote}
「將奴這知心話,付花箋,寄與他。想當初結下青絲髮,門兒倚遍簾兒下,受了些沒打弄的躭驚怕。你今果是負了奴心,不來還我香羅帕!」
\end{myquote}

寫就,疊成一個方勝兒,封停當,付與玳安兒收了:「好歹多上覆他,待他生日,千萬走走,奴這裏來專望。」那玳安喫了點心,婦人又與數十文錢。臨出門上馬,婦人道:「你到家見你爹,就說六姨好不罵你。他若不來,你就說六姨到明日坐轎子親自來哩。」玳安道:「六姨!自喫你賣粉團的撞見了敲板兒蠻子叫寃屈——麻飯疙瘩的帳!騎着木驢兒嗑瓜子兒——瑣碎昏昏。」說畢,騎上馬去了。

那婦人每日長等短等,如石沉大海一般,那裏得個西門慶影兒來。看看七月將盡,到了他生辰。這婦人挨一日似三秋,盼一夜如半夏,等了一日,杳無音信;盼了多時,寂無形影。不覺銀牙暗咬,星眼流波。至晚,旋叫王婆來,安排酒肉,與他喫了。向頭上拔下一根金頭銀簪子與他,央徃西門慶家走走,去請他來。王婆道:「這早晚來,茶前酒後,他定也不來。待老身明日侵早,徃大官人宅上請他去罷。」婦人道:「乾娘是必記心,休要忘了。」婆子道:「老身管着那一門兒來,肯悞了勾當!」當下這婆子非錢而不行,得了這根簪子,喫得臉紅紅,歸家去了。原來婦人在房中,香薰鴛被,款剔銀燈,睡不着,短歎長吁,翻來覆去。正是:得多少琵琶夜久慇勤弄,寂寞空房不忍彈。於是獨自彈着琵琶,唱一個〔綿搭絮〕為證:

\begin{myquote}
「當初奴愛你風流,共你剪髮燃香,兩態雲蹤兩意投。背親夫和你情偸,怕甚麽傍人講論,覆水難收。你若負了奴眞情,正是緣木求魚空自羞!」
\end{myquote}

又
\begin{myquote}
「誰想你另有了裙釵,氣的奴似醉如癡,斜傍定帷屏故意兒猜。不明白怎生丢開!傳書寄柬你又不來。你若負了奴的恩情,人不為仇天降災!」
\end{myquote}

又
\begin{myquote}
「奴家又不曾愛你錢財,只愛你可意的寃家,知重知輕性情兒乖。奴本是朶好花兒園内初開,蝴蝶餐破再也不來。我和你那様恩情,前世裏姻緣今世裏該。」
\end{myquote}

又
\begin{myquote}
「心中猶豫展轉成憂,常言婦人癡心,惟有情人意不周。是我迎頭和你把情偸,鮮花付與怎肯干休?你如今另有知心,海神廟裏和你把狀投!」
\end{myquote}

原來婦人一夜翻來覆去,不曾睡着。到天明,使迎兒過間壁:「瞧那王奶奶請你爹去了不曾?」迎兒去了不多時,說:「王奶奶老早就出去了。」

且說那婆子,早晨梳洗出門,來到西門慶門首,問門上:「大官人在家?」都説不知道。在對門牆角下等不夠多時,只見傅夥計來開舖子。婆子走向前來,道個萬福:「動問一聲,大官人在家麽?」傅夥計道:「你老人家尋他怎的?早是來問着我,第二個人也不知他。」因說:「大官人昨日壽日,在家請客喫酒。喫了一日酒,到晚拉衆朋友徃院裏去了,一夜通沒來家。你徃那裏尋他去。」這婆子拜辭出縣前,來到東街口,正徃勾欄那條巷去。只見西門慶騎馬遠遠從東來,兩個小廝跟隨,喫的醉眼摩娑,前合後仰。被婆子高聲叫道:「大官人,少喫些兒怎的。」向前一把手,把馬嚼環扯住。西門慶醉中問道:「你是王乾娘?你來有甚話說?」那婆子向他耳畔低言,道不數句,西門慶道:「小廝來家對我說來,我知道六姐惱我哩,我如今就去。」那西門慶一面跟着他,兩個一遞一句,整説了一路話。

比及到婦人門首,婆子先入去報道:「大娘子!且喜還虧老身去了,沒半個時辰,把大官人請得來了!」婦人聽見他來,連忙叫迎兒收拾房中乾凈,一面出房來迎接。西門慶搖着扇兒進來,带酒半酣,進入房來,與婦人唱喏。婦人還了萬福,說道:「大官人,貴人稀見面,怎的把奴來丢了,一向不來傍個影兒?家中新娘子陪伴,如膠似漆,那裏想起奴家來!還説大官人不變心哩。」西門慶道:「你休聽人胡說,那討甚麽新娘子來?只因小女出嫁,忙了幾日,不曾得閒工夫來看你。就是這般話。」婦人道:「你還哄我哩!你若不是憐新棄舊,再不外邊另有别人,你指着旺跳身子說個誓,我方信你。」那西門慶道:「我若負了你情意,生碗來大疔瘡,害三五年黄病,扁擔大蛆ご口袋!」婦人道:「賊負心的,扁擔大蛆ご口袋,管你甚事?」一手向他頭上把帽兒撮下來,望地下只一丢。慌的王婆地下拾起來,見一頂新纓子瓦楞帽兒,替他放在桌上,說道:「大娘子只怪老身不去請大官人,來就是這般的!還不與他帶上,看篩了風。」婦人道:「那怕負心強人陰寒死了,奴也不痛他!」一面向他頭上拔下一根簪兒,拏在手裏觀看,卻是一點油金簪兒,上面鈒着兩溜子字兒:「金勒馬嘶芳草地,玉樓人醉杏花天。」卻是孟玉樓帶來的。婦人猜是那個唱的與他的,奪了放在袖子裏不與他,說道:「你還不變心哩,奴與你的簪兒那裏去了!卻帶着那個的這根簪子?」西門慶道:「你那根簪子,前日因喫酒醉了,跌下馬來,把帽子落了,頭髮散開,尋時就不見了。」婦人道:「你哄三歲小孩兒也不信。哥哥兒,你醉的眼花恁様了,簪子落地下,就看不見?」王婆在傍插口道:「大娘子,你休怪大官人。他離城四十里見蜜蜂兒拉屎,出門教獺象絆了一跤,原來覷遠不覷近。」西門慶道:「緊自他麻煩人,你又自作耍!」婦人因見手中拏着一把紅骨細洒金金釘鉸川扇兒,取過來迎亮處只一照。原來婦人久慣知風月中事,見扇兒多是牙咬的碎眼兒,就疑是那個妙人與他的扇子。不由分說,兩把折了。西門慶救時,已是扯的爛了,說道:「這扇子是我一個朋友卜志道送我的。今日纔拏了三日,被你扯爛了。」那婦人奚落了他一囬。只見迎兒拏茶來,叫迎兒放下茶託,與西門慶磕頭。王婆道:「你兩口子聐聒了這半日,也夠了,休要悞了勾當,老身廚下收拾去也。」

婦人一面吩咐迎兒房中放桌兒,預先安排下與西門慶上壽的酒餚,無非是燒鷄燒鵝鮮魚肉鮓菓品之類。須臾,安排停當,拏到房中,擺在桌上。婦人向箱中取出與西門慶做下上壽的物事,用盤託盛着,擺在面前,與西門慶觀看:一雙玄色緞子鞋;一雙挑線密約深盟隨君膝下香草邊闌松竹梅花歲寒三友醬色緞子護膝;一條紗綠潞紬、永祥雲嵌八寳水光絹裏兒、紫線帶兒、裏面裝着排草玫瑰花兜肚;一根並頭蓮瓣簪兒,簪兒上鈒着五言四句詩一首云:「奴有並頭蓮,贈與君關髻。凡事同頭上,切勿輕相棄。」西門慶一見,滿心歡喜,把婦人一手摟過,親了個嘴,說道:「那知你有如此一段聰慧,少有!」婦人教迎兒執壺,斟一盃與西門慶,花枝招颺、插燭也似磕了四個頭。那西門慶連忙拖起來。兩個並肩而坐,交盃換盞飲酒。那王婆陪着喫了幾盃酒,喫的臉紅紅的,告辭回家去了。二人自在取樂頑耍。迎兒打發王婆出去,關上大門,廚下坐的。婦人陪伴西門慶飲酒多時,看看天色晚來,但見:
\begin{myquote}
密雲迷晚岫,暗霧鎖長空。羣星與皓月爭輝,綠水共青天鬬碧。僧投古寺,深林中嚷嚷鴉飛;客奔荒村,閭巷内汪汪犬吠。枝上子規啼夜月,園中粉蝶戯花來。
\end{myquote}

當下西門慶吩咐小廝囬馬家去,就在婦人家歇了。到晚夕,二人如癲狂鷂子相似,儘力盤桓,淫慾無度。

常言道:樂極悲生,泰極否來。光陰迅速,單表武松自從領了知縣書禮,離了清河縣,送禮物馱擔到東京朱太尉處,下了書禮,交割了箱馱,街上各䖏閒行了幾日,討了囬書,領一行人取路向山東大路而來。去時三四月天氣,囬來卻淡暑新秋。路上雨水連綿,遲了日限。前後徃囬也有三個月光景。在路上臥坐住行,只覺得神思不安,身心恍惚,趕囬要看哥哥。不免差了一個土兵,預先報與知縣相公。又私自寄了一封家書,與他哥哥武大,說他也不久——只在八月内囬還。那土兵先下了知縣相公禀帖,然後逕奔來找尋武大家。可可天假其便,王婆正在門首。那土兵見武大家門關着,纔要叫門,婆子便問:「你是尋誰的?」土兵道:「我是武都頭差來,下書與他哥哥。」婆子道:「武大郎不在家,都上墳去了。你有書信,交與我就是了,等他歸來,我遞與他也是一般。」那土兵向前唱了一個喏,便向身邊取出家書來,交與王婆,忙忙促促騎上頭口,飛的一般去了。

這王婆拏着那封書,従後門走過婦人家來。迎兒開了門,婆子入來。原來婦人和西門慶狂了半夜,約睡至飯時,還不起來。王婆呌道:「大官人娘子起來!匆匆有句話和你們說。如今如此如此,這般這般,武二差土兵寄了書來,他與哥哥說,他不久就到。我接下,幾句話兒打發他去了。你們不可遲滯,早䖏長便。」那西門慶不聽萬事皆休,聽了此言,正是:分開八塊頂梁骨,傾下半桶冰雪來。一面與婦人都起來,穿上衣服,請王婆到房内坐了,取出書來與西門慶看了。武松書中寫着,不過中秋回家。二人都慌了手脚,説道:「如此怎了?乾娘遮藏我們則個,恩有重報,不敢有忘!我如今與大姐情深似海,不能相捨;武二那廝囬來,便要分散,如何是好?」婆子道:「大官人,有什麽難處之事!我前日已說過了,初嫁由爹娘,後嫁由自己。古來叔嫂不通問。如今已是大郎百日來到,大娘子請上幾位僧衆來把這靈牌子燒了,趁武二未到家來,大官人一頂轎子娶了家去。等武二那廝囬來,我自有話說。他敢怎的!自此你二人自在一生,無些鳥事。」西門慶便道:「乾娘說的是。」正是:人無剛骨,安身不牢。當日西門慶和婦人用畢早飯,約定:八月初六日是武大郎百日,請僧念佛燒靈;初八日晚,擡娶婦人家去。三人計議已定。不一時,玳安拏馬來接回家,不在話下。

光陰似箭,日月如梭,又早到八月初六日。西門慶拏了數兩散碎銀錢、二斗白米齋襯,來婦人家,教王婆報恩寺請了六個僧,在家做水陸超度武大升天,晚夕除靈。道人頭五更就挑了經擔來,鋪陳道場懸掛佛像。王婆伴廚子在灶上安排整理齋供。西門慶那日就在婦人家歇了。不一時,和尚來到,搖響靈杵,打動鼓鈸,宣扬諷誦,咒演《法華經》,禮拜《梁王懺》,早晨發牃,請降三寳,證盟功德,請佛獻供;午刻召亡施食。不必細說。

且說潘金蓮怎肯齋戒,陪伴西門慶睡到日頭半天,還不起來。和尚請齋主撚香簽字,證盟禮佛。婦人方纔起梳洗,喬素打扮,來到佛前參拜。那衆和尚見了武大這個老婆,一個個都昏迷了佛性禪心,一個個都關不住心猿意馬,都七顛八倒,酥成一塊。但見:

\begin{myquote}
班首輕狂,念佛號不知顛倒;維那昏亂,誦經言豈顧高低。燒香行者,推倒花瓶;秉燭頭陀,錯拏香盒。宣盟表白,大宋國稱做大唐;懺罪闍黎,武大郎念為大父。長老心忙,打鼓借拏徒弟手;沙彌心蕩,磬槌打破老僧頭。従前苦行一時休,萬個金剛降不住。
\end{myquote}

那婦人佛前燒了香,簽了字,拜禮佛畢,囬房去了。依舊陪伴西門慶做一䖏,擺上酒席葷腥來,自去取樂。西門慶吩咐王婆:「有事你自答應便了,休教他來聒噪六姐。」婆子哈哈笑道:「大官人你倒放心,由着老娘和那秃廝纏,你兩口兒是會受用!」

看官聽説:世上有德行的高僧,坐懷不亂的少。古人有云:一個字便是「僧」,二個字便是「和尚」,三個字是個「鬼樂官」,四個字是「色中餓鬼」。蘇東坡又云:不秃不毒,不毒不秃;轉毒轉秃,轉秃轉毒。此一篇議論,專説這為僧戒行。住着這高堂大厦、佛殿僧房,喫着那十方檀越錢糧,又不耕種,一日三餐,又無甚事縈心,只專在這色慾上留心。譬如在家俗人,或士農工商,富貴長者,小相俱全,每被利名所絆,或人事徃來,雖有羙妻少妾在旁,忽想起一件事來關心,或探探甕中無米,囤内少柴,早把興來沒了。卻輸與這和尚們許多。有詩為證:
\begin{myquote}
色中餓鬼獸中狨,壞教貪淫玷祖風。

此物只宜林下看,不堪引入畫堂中。
\end{myquote}

當時這衆和尚見了武大這個老婆喬模喬樣,都記在心裏。到午齋徃寺中歇晌囬來,婦人正和西門慶在房裏飲酒作歡。原來婦人臥房,正在佛堂一䖏,止隔一道板壁。有一個僧人先到,走在婦人窻下水盆裏洗手,忽然聽見婦人在房裏顫聲柔氣,呻呻吟吟,哼哼唧唧,恰似有人在房裏交媾一般。於是推洗手,立住了脚,聽夠多時。只聽婦人口裏喘聲呼叫西門慶:「達達,你休只顧𢵞打到幾時?只怕和尚來聽見。饒了奴,快些丢了罷!」西門慶道:「你且休慌!我還要在蓋子上燒一下兒哩!」不想都被這秃廝聽了個不亦樂乎。落後衆和尚都到齊了,吹打起法事來,一個傳一個,都知道婦人有漢子在屋裏,不覺都手之舞之,足之蹈之。臨佛事完滿,晚夕送靈化財出去,婦人又早除了孝髻,換了一身艷衣服,在簾裏與西門慶兩個並肩而立,看着和尚化燒靈座。王婆舀將水,點一把火來,登時把靈牌並佛旛燒了。那賊秃冷眼瞧見簾子裏,一個漢子和婆娘影影綽綽並肩站立,想起白日裏聽見那些勾當,只顧亂打鼓𢵞鈸不住。被風把長老的僧伽帽刮在地上,露見青旋旋光頭,不去拾,只顧𢵞鈸打鼓,笑成一塊。王婆便叫道:「師父,紙馬也燒過了,還只顧𢵞打怎的?」和尚答道:「還有紙爐蓋子上沒燒過。」西門慶聽見,一面令王婆快打發襯錢與他。長老道:「請齋主娘子謝謝。」婦人道:「王婆說:免了罷。」衆和尚道:「不如饒了罷。」一齊笑的去了。正是:遺蹤堪入時人眼,不買胭脂畫牡丹。有詩為證:
\begin{myquote}
淫婦燒靈志不平,和尚竊壁聽淫聲。

果然佛道能消罪,亡者聞之亦慘魂。
\end{myquote}

畢竟未知後來何如,且聽下囬分解。

