\includepdf[pages={85,86},fitpaper=false]{tst.pdf}
\chapter*{第四十三囬 \\為失金西門慶駡金蓮 因結親月娘會喬太太}
\addcontentsline{toc}{chapter}{第四十三囬 為失金西門慶駡金蓮 因結親月娘會喬太太}
\markboth{{\titlename}卷之五}{第四十三囬 為失金西門慶駡金蓮 因結親月娘會喬太太}


\begin{myquote}
細推今古事堪愁,貴賤同歸土一丘:

漢武玉堂人豈在,石家金谷水空流!

光陰自旦還將暮,草木従春又到秋。

閑事與時俱不了,且將暫入醉鄉遊。
\end{myquote}

話説西門慶歸家已有三更時分。到於後邊,吳月娘還未睡,正和吳吳大妗子衆人坐着説話兒,李瓶兒還伺候着與他遞酒。大妗子見西門慶來家,就過那邊屋裏去了。月娘見他有酒了,打發他脱了衣裳,只教李瓶兒與他磕了頭,同坐下,問了回今日酒席上話。玉簫點茶來喫。因有大妗子在,就往孟玉樓房中歇了一夜。

到次日,廚役早來收拾治辦酒席。西門慶先到衙門中拜牌,大發放。夏提刑見了,致謝日昨房下厚擾之意。西門慶道:「日昨甚是簡慢,恕罪恕罪!」來家,有喬大戶家使了孔嫂兒,引了喬五太太那裏家人送禮來了,一壜南酒、四樣餚品。西門慶收了,管待家人酒飯。孔嫂兒進裏邊月娘房裏坐的。吳舜臣媳婦兒鄭三姐轎子先來了,拜了月娘衆人,都陪着孔嫂兒喫茶。

正値李智黄四關了一千兩香蠟銀子,賁四従東平府押了來家。應伯爵打聽得知,亦走來幫扶交與。西門慶令陳經濟拿天平,在廳上盤秤,兑明白收了,還欠五百兩本钱,一百五十兩利息。當日黄四拿出四錠金鐲兒來,重三十兩,算一百五十之數,别的搗換了合同。西門慶吩咐二人:「你等過燈節再來計較,我連日家中有事。」那李智黄四老爹長老爹短,千恩萬謝出門。應伯爵因記掛着二人許了他些業障兒,趂此機會好問他要。正要跟隨同去,又被西門慶叫住説話。西門慶因問:「昨日你們三個,怎的三不知不和我説就走了?我使小廝落後趕你不着了。」伯爵道:「昨日甚是深擾哥。本等酒夠了,我見哥也有酒了,今日嫂子家中擺酒,一定還等哥説話。俺們不走了,還只顧纏到多咱?我猜哥今日也沒得往衙門裏去,本等連日辛苦。」西門慶道:「我昨日來家已有三更天氣。今日還早到衙門,拜了牌,坐廳大發放,理了回公事。如今家中治料堂客之事。今日觀裏打上元醮,拈了香囬來,還趕了往周南軒家喫酒去,不知到多咱纔得來家。」伯爵道:「還是虧哥好神思,你的大福。不是面獎,若是第二個,也成不的!」兩個説了一囬,西門慶要留伯爵喫飯,伯爵道:「我不喫飯,去罷!」西門慶又問:「嫂子怎的不來?」伯爵道:「房下轎子已叫下來,便來也。」舉手作辭出門,一直趕往李智黄四去了。正是:假饒駕霧騰雲術,取火鑽冰只要錢。

卻説西門慶打發伯爵去了,把手中拿着黄烘烘四錠金鐲兒,心中甚是可愛。口中不言,心裏暗道:「李大姐生的這孩子甚是脚硬,一養下來,我平地就得此官。我今日與喬家結親,又進這許多財。」於是用袖兒抱着那四錠金鐲兒,也不到後邊,逕往花園内李瓶兒房裏來。正往潘金蓮角門首所過,只見金蓮正出來,看見叫住,問道:「你手裏托的是什麽東西兒?過來我瞧瞧。」那西門慶道:「等我囬來與你瞧。」托着一直往李瓶兒那邊去了。那婦人見叫不回他來,心中就有幾分羞訕,説道:「什麽罕稀貨,忙的這等?唬人子剌剌的,不與我瞧罷!賊跌折腿的三寸貨強盜,這麽逩喪進他門去,正走着,矻齊的把那兩條腿𢱉折了,纔現報了我的眼!」

卻説西門慶拿着金子,走入李瓶兒房裏。見李瓶兒纔梳了頭,奶子正抱着孩子頑耍。西門慶一徑裏把那四個金鐲兒抱着,教他手兒撾弄。李瓶兒道:「是那裏的?只怕冰了他手。」西門慶悉把李智黄四今日還銀子,準折利錢,約這金子一節説了。這李瓶兒生怕冰着他,取了一方通花汗巾兒與他裹着耍子。

只見玳安走來,説道:「雲夥計騎了兩疋馬來,在外邊,請爹出去瞧。」西門慶道:「雲夥計他是那裏的馬?」玳安道:「他説是他哥雲參將邊上捎來的馬,只説會行。」正説着,只見後邊李嬌兒、孟玉樓,陪着大妗子并他媳婦兒鄭三姐,都來李瓶兒房裏看官哥兒。西門慶丢下那四錠金子,就往外邊大門首看馬去了。李瓶兒見衆人來到,只顧與衆人見禮讓坐,也就忘記了孩子拿着這金子。弄來弄去,少了一錠。只見奶子如意兒問李瓶兒説道:「娘沒曾收哥兒耍的那錠金子?只三錠,少了一錠了。」李瓶兒道:「我沒曾收,我把汗巾子替他裹着哩!」如意兒道:「汗巾子也落在地下了,我抖來,那裏得那錠金子來?」屋裏就亂起來,奶子問迎春,迎春就問老馮,老馮道:「耶嚛,耶嚛!我老身就瞎了眼,也沒看見。老身在這裏恁幾年,就是折針我也不敢動。娘他老人家知道我,就是金子我老身也不愛。你們守着哥兒,沒的寃枉起我來了!」李瓶兒笑道:「你看這媽媽子説混話。這裏不見的,不是金子卻是什麽?」又罵迎春:「賊臭肉,平白亂的是些什麽?等你爹進來,等我問他,只怕是你爹收了。怎的只收一錠兒?」孟玉樓問道:「是那裏金子?」李瓶兒道:「是他爹外邊拿來的,與孩子耍。誰知道是那裏的!」

不想西門慶在門首看了一囬馬,衆夥計家人都在跟前,教小廝來回騎溜了兩趟。西門慶道:「雖是兩疋東路來的馬,鬃尾醜,不十分會行,論小行也罷了。」因問雲夥計道:「此馬你令兄那裏要多少銀子?」雲離守道:「兩疋只要七十兩。」西門慶道:「也不多,只是不會行。你還牽了去,另有好馬騎來,倒不説銀子。」説畢,西門慶進來。只見琴童來請:「六娘房裏請爹哩!」於是走入李瓶兒房裏來。李瓶兒問他:「金子你收了一錠去了?如何只三錠在這裏?」西門慶道:「我丢下就出來了,外邊看馬,誰收那錠來?」李瓶兒道:「你沒收,卻往那裏去了?尋了這一日沒有。奶子推老馮。急的那老馮賭身罰咒,只是哭。」西門慶道:「端的是誰拿了?由他,慢慢兒尋罷!」李瓶兒道:「頭裏要尋,因後邊和大妗子娘兒兩個來時,亂着,就忘記了。我只説你收了出去,誰知你也沒收,就兩耽了。尋起來,唬的他們都走了。」於是把那三錠還交與西門慶收了。正値賁四傾了一百兩銀子來交,西門慶往後邊收兑銀子去。

且說潘金蓮聽見李瓶兒這邊嚷不見了孩子耍的一錠金鐲子,得不的風兒就是雨兒,就先走來房裏告月娘説:「姐姐,你看三寸貨幹的營生。隨你家怎的有錢,也不該拿金子與孩子耍!」月娘道:「剛纔他們告我説,他房裏好不反亂,説不見了金鐲子。端的不知那裏的金鐲子。」金蓮道:「誰知他是那裏的!你還沒見,他頭裏従外邊拿進來,那等用襖子袖兒托着,恰似八蠻進寳的一般!我問他是什麽,拿過來我瞧瞧。頭兒也不回,一直奔命往屋裏去了。遲了一囬,反亂起來,説不見了一錠金子。乾淨就是他!學三寸貨説,『不見了,由他,慢慢兒尋罷。』你家就是王十萬,也使不的!一錠金子,至少重十來兩,也値個五六十兩銀子。平白就罷了?瓮裏走了鱉,左右是他家一窝子。再有誰進他屋裏去?」

正説着,只見西門慶進來兌收賁四傾的銀子。把剩的那三錠金子,交與月娘收了。因告訴月娘:「此是李智黄四還的。這四錠金子拿到與孩子耍了耍,就不見了一錠。」吩咐月娘:「你與我把各房裏丫頭叫出來審問審問。我使小廝街上買狼觔去了。早拿出來便罷,不然,我就教狼觔抽起來!」月娘道:「論起來,這金子也不該拿與孩子,沉甸甸冰着他,怕一時砸了他手脚,怎了?」潘金蓮在旁,接過來説道:「不該拿與孩子耍?只恨拿不到他屋哩!頭裏叫着,想囬頭也怎的?恰似紅眼軍搶將來的,不敎一個人兒知道。這囬不見了金子,虧你怎麽有臉兒來對大姐姐説,教大姐姐替你查考各房裏丫頭。教各房裏丫頭,口裏不笑,ず窿子也笑!」幾句説的西門慶急了,走向前把金蓮按在月娘炕上,提起拳來罵道:「恨殺我罷了!不看世界面上,把你這小歪剌骨兒就一頓拳頭打死了!單管嘴尖舌快的,不管你事也來插一脚。」那潘金蓮就假做喬張致,哭將起來,説道:「我曉的你倚官仗勢,倚財為主,把心來横了,只欺負的是我。你説你這般把這一個半個人命兒打死了不放在意裏,那個攔着你手兒哩不成!你打不是!有的是我,隨你怎麽打,難得只打的有這口氣兒在着,若沒了,愁我家那病媽媽子來不問你要人?隨你家怎麽有錢有勢,和你家一遞一狀。你説你是衙門裏千戶便怎的?無過只是個破砂帽債殼子窮官罷了,能禁的幾個人命?可就不是做皇帝,敢殺下人也怎的?」幾句説的西門慶反呵呵笑了,説道:「你看原來小歪剌骨兒這等刁嘴!我是破紗帽窮官,教丫頭取我的紗帽來,我這紗帽那塊兒放着破?這裏清河縣問聲,我少誰家銀子,你説我是債殼子!」金蓮道:「你怎的叫我是歪剌骨來?」因蹺起一隻脚來,「你看,老娘這脚那些兒放着歪?你怎罵我是歪剌骨,那剌骨也不怎的!」月娘在旁笑道:「你兩個銅盆撞了鐵刷帚。常言:惡人自有惡人磨,見了惡人沒奈何!自古嘴強的爭一步。六姐,也虧你這個嘴頭子,不然嘴鈍些兒也成不的。」

那西門慶見奈何不過他,穿了衣裳,往外去了。迎見玳安來說:「周爹家差人邀來了。備馬了,請問爹先往打醮處去,往周爺家去?」西門慶吩咐:「打醮䖏,教你姐夫去罷。到了那裏拈了香,快來家裏看着。伺候馬,我往你周爺家喫酒去就是了!」説着,書童兒拿冠带過來,打發穿了,繫上帶。只見王皇親家扮戲兩個師父,率衆過來與西門慶叩頭。西門慶敎書童看飯與他喫,説:「今日你等用心唱,伏侍衆奶奶,我自有重賞。休要上邊打箱去。」那師父跪下説道:「小的們若不用心答應,豈敢討賞?」西門慶因吩咐書童:「他唱了兩日,連賞賜封下五兩銀子賞他。」書童應諾:「小的知道了。」西門慶就上馬,往周守備家喫酒去了。

單表潘金蓮在上房陪吳妗子坐的,吳月娘便説:「你還不往屋裏勻勻那臉去?揉的恁紅紅的,等住囬人來看着什麽張致。誰教你惹他來!我倒替你捏兩把汗。若不是我在跟前勸着,綁着鬼也有幾下子打在身上。漢子家臉上有狗毛,不知好歹,只顧下死手的和他纏起來了!不見了金子,隨他不見去,尋不尋不在你。又不在你屋裏不見了,平白扯着脖子和他強怎麽?你也丢了這口氣兒罷!」幾句説的金蓮閉口無言,往屋裏勻臉去了。

不一時,只見李瓶兒和吳銀兒都打扮出來,到月娘房裏。月娘問他:「金子怎的不見了?剛纔惹得他爹和六姐兩個在這裏好不拌了這回嘴,差些兒沒曾拌惱了打起來!乞我勸開了,他爹便往人家喫酒去了。吩咐小廝買狼觔去了,等他晚上來家,要把各房丫頭抽起來。你屋裏丫頭老婆管着那一門兒來?就看着孩子耍,便不見了他一錠金子!是一個半個錢的東西兒也怎的?」李瓶兒道:「平白他爹拿進四錠金子來,與孩子耍,我亂着陪大妗子和鄭三姐並他二娘坐着説話,誰知就不見了一錠。如今丫頭推奶子,奶子推老馮。急的那媽媽哭哭啼啼,只要尋死。無眼難明勾當,如今冤誰的是?」吳銀兒道:「天麽天麽!每常我還和哥兒耍子,早是今日我在娘這邊屋裏梳頭,沒曾過去。不然,難為我了。雖然爹娘不言語,你我心上何安?誰人不愛錢?俺裏邊人家最忌叫這個名聲兒,傳出去醜聽!」

正説着,只見韓玉釧兒董嬌兒兩個,提着衣包兒進來,笑嘻嘻先向月娘大妗子李瓶兒磕了頭,起來,望着吳銀兒拜了一拜,説道:「銀姐昨已來了,沒家去?」吳銀兒道:「你兩個怎的曉得?」董嬌兒道:「昨日俺兩個都在燈巿街房子裏唱來,大爹對俺們説,教俺今日來唱,伏侍奶奶。」一面月娘讓他兩個坐下。須臾,小玉拿了兩盞茶來。那韓玉釧兒董嬌兒連忙立起身來接茶,還望小玉拜了一拜。吳銀兒因問:「你兩個昨日唱多咱散了?」韓玉釧道:「俺們到家也有二更多了。同你兄弟李銘都一路去來。」説了一囬話,月娘吩咐玉簫:「早些打發他們喫了茶罷!等住囬,只怕那邊人來忙了。」一面放下桌兒,兩方春槅,四盒茶食。月娘使小玉:「你二娘房裏請了桂姐來,同喫了茶罷。」不一時,桂姐和他姑娘來到,兩個各道了禮數,坐下同喫了茶,收過家活去。

忽見迎春打扮着,抱了官哥兒來。頭上戴着金梁緞子八吉祥帽兒,身穿大紅氅衣兒,下邊白綾襪兒、緞子鞋兒,胸前項牌符索,手上小金鐲兒。李瓶兒看見,説道:「小大官兒,沒人請你,來做甚麽?」一面接過來,放在膝蓋上。看見一屋裏人,把眼不住的看了這頭,看那一個。桂姐坐在月娘炕上笑,引鬦他耍子,道:「哥子只看我這裏,想必只要我抱他。」於是用手引了他引兒,那孩子就撲到懷裏教他抱着。吳大妗子笑道:「恁點小孩兒,他也曉的愛好。」月娘接過來説:「他老子是誰?到明日大了,管情也是小嫖頭兒。」孟玉樓道:「若做了小嫖頭兒,教大媽媽就打死了。」那李瓶兒道:「小廝,你姐姐抱,只休溺了你姐姐衣服,我就忙死了。」那桂姐道:「耶嚛,怕怎麽!溺了也罷,不妨事。我心裏要抱哥兒耍耍兒。」於是與他兩個嘴揾嘴兒耍子。只見潘金蓮也來了,董嬌兒韓玉釧兒下來行禮畢,坐下説道:「俺兩個來了這一日,還沒曾唱個兒與娘們聽。」因叫小玉:「姐,你取楽器來,等俺唱。」那小玉便取箏和琵琶,遞與他二人。當下韓玉釧兒琵琶,董嬌兒彈箏,吳銀兒也在旁邊陪唱;於是唱了一套「繁花滿目開」〔金索掛梧桐〕。唱出一句來,端的有落塵遶梁之聲,裂石流雲之響。把官哥兒唬的在桂姐懷裏只磕倒着,再不敢擡頭出氣兒。月娘看見,便呌:「李大姐,你接過孩子來,教迎春抱的屋裏去罷。好個不長俊的小廝,你看唬的那臉兒!」這李瓶兒連忙接過來,教迎春掩着他耳朶,抱的往那邊房裏去了。於是四個唱的,齊合着聲兒,唱這一套詞道:

\begin{myquote}
「繁花滿目開,錦被空閑在。劣性寃家悞得我忒毒害!我前生少欠他今世裏相思債。廢寢忘餐,倚定門兒待。房櫳靜悄如何捱?」

{\markfont〔駡玉郎〕}「冷清清房櫳靜悄如何捱?獨自把幃屏倚,知他是甚情懷?想當初同行同坐同歡愛,到如今孤另另怎㓦劃?愁戚戚酒倦釃,羞慘慘花慵戴。」

{\markfont〔東甌令〕}「花慵戴,酒倦釃,如今燕約鶯期不見來,多應是他在那裏那裏貪歡愛。物在人何在?空勞魂夢到陽臺,只落得淚盈腮。」

{\markfont〔感皇恩〕}「呀,只落得兩淚盈腮,多應是命裏合該!莫不是你緣薄咱分淺,都應是一般運拙時乖。怎禁那攪閒人是非,施巧計裁排。撕撏碎合歡帶,硬分開鸞鳳釵,水淹浸楚陽臺。」

{\markfont〔針線箱〕}「把一床絃索塵埋,兩眉峯不展開。香ざ瘦損愁無奈,懶刺繡傍粧臺。舊恨新愁教我如何捱?我則怕蝶使蜂媒不再來。臨鸞鏡也,問道朱顔未改,他又早先改。」

{\markfont〔採茶歌〕}「改朱顔瘦了形骸,冷清清怎生捱?我則怕梁山伯不戀我這祝英臺。他若是背義忘恩尋罪責,我將那盟山誓海説的明白。」

{\markfont〔解三酲〕}「頓忘了盟山誓海,頓忘了音書不寄來,頓忘了枕邊許多恩和愛,頓忘了素體相挨,頓忘了神前兩下千千拜,頓忘了表記香羅紅繡鞋。説將起,旁人見了珠淚盈腮。」

{\markfont〔烏夜啼〕}「俺如今相離三月,如隔數載,要相逢甚日何年再?則我這瘦伶仃形體如柴,甚時節還徹了相思債!又不見青鳥書來,黄犬音乖。每日家病懨懨懶去傍粧臺。得團圓,便把神羊賽。意廝投,心相愛,早成了鸞交鳳友,省的着蝶笑蜂猜。」

{\markfont〔尾聲〕}「把局兒牢鋪擺,情人終久再歸來,羙滿夫妻百歲諧。」
\end{myquote}

四個唱的正唱着,只見玳安進來。月娘便問:「你邀請的衆奶奶們怎的這咱還不見來?」玳安道:「小的到喬親家娘那邊邀來,朱奶奶尚舉人娘子都過喬親家娘家來了,只等着喬五太太。到了,就往咱這裏來。」月娘吩咐:「你就説與平安兒小廝,説教他在大門首看着。等奶奶們轎子到了,就先進來説。」玳安道:「大門前邊大廳上,鼓楽迎接哩,娘們都收拾伺候就是了。」月娘吩咐玳安,後廳明間鋪下錦毯,安放坐位,捲起簾來,金鈎雙控,蘭麝香飄。春梅迎春玉簫蘭香都打扮起來,家人媳婦都插金戴銀,披紅垂綠,準備迎接新親。只見應伯爵娘子兒應二嫂先到了,應寳跟着轎子。月娘等迎接進來,見了禮數,明間内坐下。向月娘拜了又拜,説:「俺家的常時打擾這裏,多蒙看顧。」月娘道:「姑娘好説,常時累你二爹。」

良久,只聞喝道之聲漸近,前廳鼓楽響動。平安兒先進來報道:「喬太太轎子到了。」須臾黑壓壓一羣人,跟着五頂大轎,落在門首。惟喬五太太轎子在頭裏,轎上是垂珠銀頂,天青重沿銷金走水轎衣,使藤棍唱路。後面家人媳婦坐小轎跟隨。四名校尉擡衣箱火爐。兩個青衣家人騎着小馬,後面隨從。其餘者,就是喬大戶娘子、朱臺官娘子、尚舉人娘子、崔大官媳婦段大姐,並喬通媳婦也坐着一頂小轎,跟來收疊衣裳。吳月娘這裏穿大紅五彩遍地錦百獸朝麒麟緞子通袖袍兒,腰束金鑲寳石鬧粧;頭上寳髻巍峩,鳳釵雙插,珠翠堆滿;胸前繡帶垂金,項牌錯落;裙邊禁步明珠,與李嬌兒孟玉樓潘金蓮李瓶兒孫雪娥,一個個打扮的似粉粧玉琢,錦繡耀目,都出二門迎接。只見衆堂客簇擁着喬五太太進來,生的五短身材,約七旬多年紀,戴着疊翠寳珠冠,身穿大紅宫繡袍兒。近而視之,鬢髮皆白。正是:眉分八道雪,髻綰一窝絲;眼如秋水微渾,鬢似楚山雲淡。接入後廳,先與吳大妗子叙畢禮數,然後與月娘等廝見。月娘再三請太太受禮,太太不肯。讓了半日,止受了半禮。次與喬大戶娘子,又叙其新親家之禮。彼此道及款曲,謝其厚儀。已畢,然後向錦屏正面,設放一張錦裀座位,坐了喬五太太。其次坐就讓喬大戶娘子。喬大戶娘子再三辭説:「姪婦不敢與五太太上僭。」讓朱臺官尚舉人娘子,兩個又不肯。彼此讓了半日,喬五太太坐了首座,其餘客東主西,兩分頭坐了。當中大方爐火箱籠起火來,堂中氣煖如春。春梅迎春玉簫蘭香,一般兒四個丫頭都打扮起來,身上一色都大紅粧花緞襖兒,藍織金裙,綠遍地金比甲兒,在跟前遞茶。

良久,喬五太太對月娘説:「請西門大人出來拜見,叙叙親情之禮。」月娘道:「拙夫今日衙門中理公事去了,還未來家哩。」喬五太太道:「大人居於何官?」月娘道:「乃一介鄉民,蒙朝廷恩例,實授千戶之職,現掌刑名。寒家與親家那邊結親,實是有玷。」喬五太太道:「娘子説那裏話?似大人這等崢嶸也夠了!昨日老身聽得舍姪女與府上做親,心中甚喜。今日我來會會,到明日席上好廝見。」月娘道:「只是有玷老太太名目。」喬五太太道:「娘子是甚麽説話,想朝廷還與庶民做親哩!老身説起來話長。如今當今東宫貴妃娘娘,係老身親侄女兒。他父母都沒了,止有老身。老頭兒在時,曾做世襲指揮使。不幸五十歲故了,身邊又無兒孫輪着,輪着別門姪另替了。手裏沒錢,如今倒是做了大戶。我這個姪兒,雖是差役立身,頗得過的日子,庶不玷汚的門戶。」説了一囬,吳大妗子對月娘説:「抱孩子出來與老太太看看,討討壽。」李瓶兒慌的走去,到房裏吩咐奶子抱了官哥來,與太太磕頭。喬太太看了,誇道:「好個端正的哥哥!」即叫過左右,連忙向毡包内打開,捧過一端宫中紫閃黄錦緞,并一付鍍金手鐲與哥兒戴。月娘連忙下來拜謝了,請去房中換了衣裳。須臾,前邊捲棚内安放四張桌席,擺下茶。每桌四十碟,都是各樣茶菓甜食,羙口菜蔬,蒸酥點心,細巧油酥餅饊之類。兩邊家人媳婦丫頭侍奉伏侍,不在話下。喫了茶,月娘就引去後邊山子花園中,開了門,遊玩了一囬下來。那時陳經濟打醮去,喫了午齋囬來了,和書童兒、玳安兒,又早在前廳擺放桌席齊整,請衆奶奶們遞酒上來。端的好筵席!但見:

\begin{myquote}
屏開孔雀,褥隱芙蓉。盤堆異菓奇珍,瓶插金花翠葉。爐焚獸炭,香裊龍涎。器列象州之古玩,簾開合浦之明珠。白玉碟高堆麟脯,紫金壺滿貯瓊槳。煮猩唇,燒豹胎,果然下筯了萬錢;烹龍肝,炮鳳髓,端的獻時品滿座。梨園子弟,簇捧着鳳管鸞簫;内院歌姬,緊按定銀箏象板。進酒佳人雙洛浦,分香侍女兩嫦娥。正是:兩行珠翠列階前,一派笙歌臨座上。
\end{myquote}

須臾,吳月娘與李瓶兒遞酒。階下戲子鼓樂嚮罷,喬太太與衆親戚又親與李瓶兒把盞祝壽。李桂姐吳銀兒韓玉釧兒董嬌兒四個唱的,在席前錦瑟銀箏,玉面琵琶,紅牙象板,彈唱起來,唱了一套「壽比南山」。下邊鼓楽響動,戲子呈上戲文手本。喬五太太吩咐下來,教做《王月英元夜留鞋記》。廚役上來獻小割燒鵝,賞了五錢銀子。比及割凡五道,湯陳三獻,戲文四摺下來,天色已晚。堂中畫燭流光,肴如山疊,各樣花燈都點起來。錦帶飄飄,彩䋲低轉。一輪明月従東而起,照射堂中,燈光掩映。來興媳婦惠秀與來保媳婦惠祥,每人拿着一方盤菓餡元宵,都是銀鑲茶鍾,金杏葉茶匙,放白糖玫瑰,馨香羙口;走到上邊,春梅迎春玉簫蘭香四人分頭照席捧遞,甚是禮數周詳,舉止沉穩。階下動楽,琵琶箏ぬ,笙簫笛管,吹打了一套燈詞〔畫眉序〕「花月滿春城」。唱畢,喬太太和喬大戶娘子叫上戲子,賞了兩包一兩銀子;四個唱的,每人二錢。月娘又在後邊明間内擺設下許多菓碟兒,留後座,四張桌子都堆滿了。唱的唱,彈的彈,又喫了一囬酒。喬太太再三説晚了,要起身。月娘衆人款留不住,送在大門首;又攔了遞酒,看放煙火。兩邊街上看的人,鱗次蜂排一般,平安兒同衆排軍執棍攔擋再三,還湧擠上來。須臾,放了一架煙火,兩邊人散了。喬太太和衆娘子方纔拜辭月娘等起身上轎去了。那時已有三更天氣。然後又送應二嫂起身。

月娘衆姊妹歸到後邊來,吩咐陳經濟來興書童玳安兒看着廳上收拾家活,管待戲子並兩個師範酒飯,與了五錢銀子唱錢,打發去了。月娘吩咐出來,剩攢下一桌餚饌半罈酒,請傳夥計賁四陳姐夫,説:「他們管事辛苦,大家喫鍾酒。就在大廳上安放一張桌兒,你爹不知多咱纔回。」於是還有殘燈不盡,當下傳夥計賁四經濟來保上座,來興書童玳安平安打横,把酒來斟。來保叫平安兒:「你還委個人大門首,怕一時爹囬,沒人看門。」平安道:「我教畫童看着哩!不妨事。」於是八個人猜枚飲酒。經濟道:「你們休猜枚,大驚小唱的,惹後邊聽見。咱不如悄悄行令兒耍子。每人要一句,説的出免罰,説不出罰一大盃酒。」該傅夥計先説:「堪笑元宵景物。」賁四道:「人生歡楽有數。」經濟道:「趂此月色燈光。」來保道:「咱且休要辜負。」來興道:「纔約嬌兒不在。」書童道:「又學大娘吩咐。」玳安道:「雖然剩酒殘燈。」平安道:「也是春風一度。」衆人念畢,呵呵笑了。正是:飲罷酒闌人散後,不知明月轉梅梢。

畢竟未知後來如何,且聽下囬分解。

