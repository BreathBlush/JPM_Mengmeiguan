\includepdf[pages={189,190},fitpaper=false]{tst.pdf}
\chapter*{第九十五囘 \\平安偸盜假當物 薛嫂喬計說人情}
\addcontentsline{toc}{chapter}{第九十五囘 平安偸盜假當物 薛嫂喬計說人情}
\markboth{\titlename}{第九十五囘 平安偸盜假當物 薛嫂喬計說人情}


格言

\begin{myquote}
有福莫享盡,福盡身貧窮;

有勢莫倚盡,勢盡寃相逢。

福宜常自惜,勢宜常自恭。

人間勢與福,有始多無終。
\end{myquote}

話説孫雪娥賣在洒家店為娼不題。話分兩頭,却說吴月娘自從大姐死了,告了陳經濟一狀到官,大家人來昭也死了,他妻一丈青,帶着小鐵棍兒也嫁人去了,來興兒看守門户。房中綉春,與了王姑子做了徒弟,出家去了。那來興兒自從他媳婦惠秀死了,一向没有妻室。奶子如意兒,要便引着孝哥兒在他屋裏頑耍喫東西,來興兒又打酒和奶子喫。兩個嘲戲,勾來勾去,就刮剌上了,非止一日。但來前邊,歸入後邊就臉紅。月娘察知其事,罵了一頓,家醜不可外揚,與了他一套衣裳,四根簪子,一件銀壽字兒,一件梳背兒,揀了個好日子,就與了來興兒完房,做了媳婦子。白日上竃、看哥兒、後邊扶持,到夜間往前邊他屋裏睡去。

一日,八月十五日,月娘生日。有吴大妗、二妗子,並三個姑子,都來與月娘做生日,在後邊堂屋裏喫酒。晚夕,都在孟玉樓住的廂房内,吴大妗、二妗子、三個姑子,同在一處睡,聽宣卷,到二更時分,中秋兒便在後邊竃上看茶,由着月娘呌,都不應。月娘親自走到上房裏,只見玳安兒正按着小玉,在炕上幹得好。看見月娘推開門進來,慌的湊手脚不迭。月娘便一聲兒也沒言語,只說得一聲:「賊臭肉,不在後邊看茶去!那屋裏師父宣了這一日卷,要茶喫,且在這裏做甚麽哩!」那小玉道:「中秋兒竃上我敎他炖茶哩。」低着頭,往後邊去。玳安便走出儀門,往前邊來。過了兩日,大妗子、二妗子、三個女僧,都家去了。這月娘把來興兒房騰出,收拾了與玳安住。却教來興兒搬到來昭屋裏,看守大門去了。替玳安做了兩牀鋪蓋,做了一身裝新衣服,盔了一頂新網新帽,做了雙新靴襪;又替小玉張了一頂䯼髻,與了他幾件金銀首飾,四根金頭銀脚簪,環墜戒指之類,兩套緞絹顔色衣服。擇日完房,就配與玳安兒做了媳婦。白日裏還進來,在房中答應月娘,只晚夕臨關儀門時便出去,和玳安歇去。這丫頭揀好東好西,甚麽不拿出來和玳安喫?這月娘當看見,只推不看見。

常言道:溺愛者不明,貪得者無厭。羊酒不均,駟馬奔陳;處家不正,奴婢抱怨。却說平安兒見月娘把小玉配與玳安做了媳婦兒,與了他一間房住,衣服穿戴勝似别人;他比玳安倒大兩歲,今年二十二歲,倒不與他妻室,一間房住!一日在假當舖,看見傅夥計當了人家一副金頭面,一柄鍍金鈎子,當了三十兩銀子。那家只把銀子使了一個月,加了利錢,就來贖討。傅夥計和玳安尋出來,放在舖子大橱櫃内。不隄防這平安兒見財起心,就連匣兒偸了,走去南瓦子裏開坊子的武長脚家,有兩個私窠子,一個呌薛存兒,一個呌伴兒,在那裏歇了兩夜。忘八見他使錢兒猛大:匣子蹙着金頭面,撅着銀鋌子打酒與鴇兒買東西,戳與土番,就把他截在屋裏,打了兩個耳刮子,就拿了。

也是合當有事,不想吴典恩新陞巡檢,騎着馬,頭裏打着一對板子,正従街上過來,看見問:「拴的甚麽人?」土番跪下禀說:「如此這般,拐帶出來,瓦子裏宿娼,拿金銀頭面行使。小的可疑,拿了。」吳典恩吩咐:「與我帶來審問!」一面拿到巡檢廳兒内。吳典恩坐下,兩邊弓皂排列。土番拴平安兒到跟前,認的是吴典恩,當初是他家夥計,「一定見了我就放的!」開口就說:「小的是西門慶家平安兒。」吴典恩道:「你旣是他家人,拿這金東西在這坊子裏做甚麽?」平安道:「小的大娘借與親戚家頭面戴,使小的取去,來晚了,城門閉了。小的投在坊子權借宿一夜,不料被土番拿了。」吴典恩罵道:「你這奴才胡說!你家自是這般頭面多金銀廣,教你這奴才把頭面拿出來老婆家歇宿行使?想必是你偸盗出來頭面。趂早説來,免我動刑!」平安道:「委的親戚家借去頭面,家中大娘使我討去來,並不敢說謊。」吴典恩大怒,罵道:「此奴才眞賊,不打如何肯認?」喝令左右:「與我拿夾棍夾這奴才!」一面套上夾棍夾起來,夾的小廝猶如殺猪呌,呌道:「爺休夾小的,放小的實説了罷。」吴典恩道:「你只實說,我就不夾你。」平安兒道:「小的偸的假當舖當的人家一副金頭面,一柄鍍金鈎子。」吴典恩問道:「你因甚麽偸出來?」平安道:「小的今年二十二歲,大娘許了替小的娶媳婦兒,不替小的娶。家中使的玳安兒小廝,纔二十歲,倒把房裏丫頭配與他完了房。小的因此不憤,纔偸出假當舖這頭面走了。」吴典恩道:「想必是這玳安兒小廝與吴氏有奸,纔先把丫頭與他配了妻室。你只實説,沒你的事,我便饒了你。」平安兒道:「小的不知道。」吴典恩道:「你不實說,與我拶起來。」左右套上拶子。慌的平安兒沒口子說道:「爺休拶小的,等小的說就是了。」吴典恩道:「可又來,你只說了,須没你的事!」一面放了拶子。那平安説:「委的俺大娘與玳安兒有奸。先要了小玉丫頭,俺大娘看見了,就沒言語,倒與了他許多衣服首飾東西,配與他完房。」這吴典恩一面令吏典上來抄了他口詞,取了供狀,把平安監在巡檢司,等着出牌提吴氏玳安小玉來審問這件事。

那日却説解當舖橱櫃裏不見了頭面,把傅夥計唬慌了,問玳安,玳安説:「我在生薬舖子裏看,你在這邊喫飯,我不知道。」傅夥計道:「我把頭面匣子放在橱裏,如何不見了?」一地裏尋平安兒,尋不着,急的傅夥計插香賭誓。那家子討頭面,傅夥計只推:「還沒尋出來哩。」那人走了幾遍,見沒有頭面,只顧在門前嚷鬧說:「我當了兩個月,本利不少你的,你如何不與我?頭面鈎子值七八十兩銀子!」傅夥計見平安兒一夜没來家,就知是他偸出去了,四下使人找尋不着,那討頭面主兒又在門首嚷亂。對月娘說,賠他五十兩銀子,那人還不肯,說:「我頭面值六十兩,鈎子連寳石珠子鑲嵌共值十兩,該賠七十兩銀子!」傅夥計又添了他十兩,還不肯,定要與傅夥計合口。正鬧時,有人來報說:「你家平安兒偸了頭面,在南瓦子養老婆,被吴巡檢拿在監裏,還不敎人快認贓去?」這吴月娘聽見吴典恩做巡檢,是咱家舊夥計,一面請吴大舅來商議,連忙寫了領狀,第二日教傅夥計領贓去:「有了原物在,省得兩家賴,敎人家人在門前放屁!」傅夥計拿狀子到巡檢司,實承望吴典恩看舊時分上,領得頭面出來。不想反被吴典恩老狗老奴才儘力罵了一頓,叫皂隸拉倒要打,褪去衣裳,把屁股脫了,半日饒放起來。說道:「你家小廝在這裏供出吴氏與玳安許多奸情來!我這裏申過府縣,還要行牌提取吴氏來對證。你這老狗骨頭,還敢來領贓?」倒喫他千奴才萬老狗,罵將出來,唬的往家中走不迭。來家不敢隱諱,如此這般,對月娘說了。月娘不聽便罷,聽了正是:分開八塊頂梁骨,傾下半桶冰雪來。慌的手脚麻木。又見那討頭面人在門前大嚷大鬧,説道:「你家不見了我頭面,又不與我原物,又不賠我銀子,只反哄着我兩頭來囘走!今日哄我去領贓,明日等領頭面,端的領的在那裏?這等不合理!」那傅夥計賠下情,將好言央及安撫他:「畧從容兩日,就有頭面出來了。若無原物,加倍賠你!」那人說:「等我囘聲當家的去。」說畢去了。

這吴月娘憂上加憂,眉頭不展,使小廝請吴大舅來商議,敎他尋人情對吴典恩說,掩下這樁事罷。吴大舅說:「只怕他不受人情,要些賄賂打點他。」月娘道:「他當初這官,還是咱家照顧他的。還借咱家一百兩銀子,文書俺爹也没收他的,今日反恩將仇報起來?」吴大舅說:「姐姐,說不的那話了!従來忘恩背義纔一個兒也怎的?」吴月娘道:「累及哥哥,上緊尋個路兒,寜可送他幾十兩銀子罷。領出頭面來,還了人家,省得合口費舌。」打發吴大舅喫了飯去了。

月娘送哥哥到大門首,也是合當事情凑巧,只見薛嫂兒提着花箱兒,領着一個小丫鬟過來。月娘呌住便問:「老薛,你往那裏去?怎的一向不來俺這裏走走?」薛嫂道:「你老人家倒且說的好!這兩日好不忙哩,偏有許多頭緒兒。咱家小奶奶那裏,使牢子大官兒叫了好幾遍,還不得空兒去哩。」月娘道:「你看媽媽子撒風,他又做起俺小奶奶來了!」薛嫂道:「如今不做小奶奶,倒做了大奶奶了!」月娘道:「他怎的做大奶奶?」薛嫂道:「你老人家還不知道?他好小造化兒,自従生了哥兒,大奶奶死了,守備老爺就把他扶了正房,做了封贈娘子!正經二奶奶孫氏,不如他。手下買了兩個奶子、四個丫頭伏侍。又是兩個房裏得寵學唱的姐兒,都是老爺收用過的。要打時就打他躺棍兒!老爺敢做的主兒?自恁還恐怕氣了他!那日不知因甚麽,把雪娥娘子打了一頓,把頭髮都撏了,半夜呌我去領出來,賣了八兩銀子。如今孫二娘房裏,使着個荷花丫鬟。他手裏倒使着四五個,又是兩個奶子,還言人少!二娘又不敢言語,成日奶奶長奶奶短,只哄着他。前日對我說:『老薛,你替我尋個小丫頭來我使。』嫌那小丫頭不會做生活,只會上竃,他屋裏事情冗雜。今日我還睡哩,大清早晨又早使牢子叫了我兩遍,教我快往宅裏去。問我要兩副大翠重雲子鈿兒,又要一副九鳳鈿銀根兒,一個鳳口裏啣一串珠兒,下邊墜着青紅寳石金牌兒。先與了我五兩銀子,銀子不知使的那裏去了,還沒送與他生活去哩!這一見了我,還不知怎生罵我哩。我如今就送這丫頭去。」月娘道:「你到後邊,等我瞧瞧怎樣翠鈿兒。」一面讓薛嫂到後邊明間内坐下。薛嫂打開花箱,取出與吴月娘看。果然做的好樣範!約四指寬,通掩過䯼髻來,金翠掩映,翡翠重疊,背面貼金;那九鳳鈿,每個鳳口内啣着一掛寳珠牌兒,十分奇巧。薛嫂道:「只這副鈿兒,做着本錢三兩五錢銀子,那付重雲子的,只一兩五錢銀子,還没尋他的錢。」

正說着,只見玳安兒走來,對月娘說:「討頭面的又來在前邊嚷哩,說等不的領贓,領到幾時?若明日没頭面,要和傅二叔打了,到個去處理會哩!傅二叔心裏不好,往家去了。那人嚷了囘去了。」薛嫂問:「是甚麽勾當?」月娘便長吁了一口氣,如此這般,告訴薛嫂說:「平安兒奴才偸去印子舖人家當的一付金頭面,一個鍍金鈎子,走在城外坊子裏養老婆,被吴巡檢拿住,監在監裏。人家來討頭面,沒有,在門前嚷鬧。吴巡檢又勒掯刁難,不容俺家領贓,打夥計將來,要錢。白尋不出個頭腦來,如何是好?死了漢子,敗落一齊來,就這等被人欺負,好苦也!」說着,那眼中淚紛紛落將下來。薛嫂道:「好奶奶,放着路兒不會尋!咱家小奶奶,你這裏寫個帖兒,等我對他說聲,教老爺差人吩咐巡檢司。莫說一副頭面,就十副頭面,也討去了。」月娘道:「周守備他是武職官,他管的着這巡檢司?」薛嫂道:「奶奶,你還不知道,如今周爺,朝廷新與他的勅書,好不管的事情寬廣!地方河道,軍馬錢糧,都在他手裏打卯遞手本。又河東水西,捉拿強盗,賊情正在他手裏!」月娘聽了便道:「旣然管着,老薛,就累你多上覆龐大姐說聲,一客不煩二主,敎他在周爺面前羙言一句兒。問巡檢司討出頭面來,我破五兩銀子謝你!」薛嫂道:「好奶奶,錢恁中使!我見你老人家剛纔悽惶,我倒下意不去。你敎人寫了帖兒,不喫茶罷。等我到府裏和小奶奶說,成了,隨你老人家;不成,我還來囘你老人家話。」這吴月娘一面叫小玉擺茶與薛嫂喫。薛嫂兒道:「這早晚了,不喫罷。你只教大官兒寫了帖兒,我拿了去罷。你不知,我一身的事在我身上哩!」月娘道:「我曉的,你也出來這半日了,喫了點心兒去。」小玉即便放桌兒,擺上茶食來。月娘陪他喫茶。薛嫂兒遞與丫頭兩個點心喫。月娘問:「丫頭幾歲了?」薛嫂道:「今年十二歲了。」不一時,玳安兒前邊寫了說帖兒。薛嫂兒喫了茶,放在袖内,作辭月娘,提着花箱出門。轉彎抹角,逕到守備府中。

春梅還在煖炕上睡,還沒起來哩。只見大丫鬟月桂進來說:「老薛來了。」春梅便叫小丫頭翠花把裏面窗寮開了,日色照的紗窗十分明亮。薛嫂進去說道:「奶奶這咱還未起來?」放下花箱便磕下頭去。春梅道:「不當家化化的,磕甚麽頭。」說道:「我心裏不自在,今日起來的遲些。」問道:「你做的那翠雲子和九鳳鈿兒拿了來不曾?」薛嫂道:「奶奶這兩副鈿兒,好不費手。昨日晚夕,我纔打翠花舖子裏討將來。今日要送來,不想奶奶又使了牢子去。」一面取出來,與春梅過目。春梅還嫌翠雲子做的不十分現撇,還安放在紙匣兒内,交與月桂收了,看茶與薛嫂兒喫。薛嫂便叫小丫鬟進來,「與奶奶磕頭。」春梅問:「是那裏的?」薛嫂兒道:「二奶奶和我說了好幾遍,說荷花只做的飯,敎我替他尋個小孩子,學做些針指。我替他領了這個孩子來了。到是鄉裏人家女孩兒,今年纔十二歲,正是養材兒,只好拘束着學做生活。」春梅道:「你一發替他尋個城裏孩子,還伶便些。這鄉裏孩子,曉的甚麽?也是前日一個張媽子,領了兩個鄉裏丫頭子來,一個十一歲,那一個十二歲了。一個呌生金,一個呌活寳。兩個且是不善,都要五兩銀子,娘老子就在外頭等着要銀子。我說,『且留他住一日兒,試試手兒,會答應不會,教他明日來領銀子罷。』死活留下他一夜。丫頭們不知好歹,與了他些肉湯子泡飯喫了。到第二日天明,只見丫頭們嚷亂起來。我便罵:『賊奴才,亂的是甚麽?』原來那生金撒了被窝屎;那活寳溺的褲子提溜不動。把我又是那笑,又是那砢硶。等的張媽子來,還敎他領的去了。」因問:「這丫頭要多少銀子?」薛嫂兒道:「要不多,只四兩銀子,他老子要投軍使。」春梅敎海棠:「你領到二娘房裏去,明日兑銀子與他罷。」又叫月桂:「拿大壺,内有金華酒,篩來與薛嫂兒喫,盪寒。再有甚點心,拿上一盒子與他喫。省得他又說大清早晨拿寡酒灌他。」薛嫂道:「桂姐,且不要篩上來,等我和奶奶説了話着。剛纔在那裏也喫了些甚麽來了。」春梅道:「你對我説,在誰家喫甚來?」薛嫂道:「剛纔大娘那頭,留我喫了些甚麽來了。如此這般,望着我好不哭哩!說平安兒小廝偸了印子舖内人家當的金頭面,還有一把鍍金鈎子,在外面養老婆,喫番子拿在巡檢司拶打。這裏人家要頭面嚷亂,使傅夥計領贓。那吴巡檢舊日是咱那裏夥計,有爹在日,照顧他的官。今日一旦反面無恩,夾打小廝,攀扯人。又不容這裏領贓,要錢纔准,把夥計打罵將來。唬的夥計不好了,躱的往家去了。央我來多多上覆你老人家,不知咱家老爺管的着這巡檢司。可憐見擧眼兒無親的,敎你替他對老爺說聲,領出頭面來,交付與人家去了,大娘親來拜謝你老人家。」春梅問道:「有個帖兒没有?不打緊,有你爺。出巡去了,怕不的今晚來家,等我對你爺説!」薛嫂兒道:「他有說帖兒在此。」向袖中取出。這春梅看了,順手就放在窗户檯上。

不一時,托盤内拿上四樣嗄飯菜蔬,月桂拿大銀鍾,滿滿斟了一鍾,流沿兒遞與薛嫂。薛嫂道:「我的奶奶,我原捱的了這大行貨子?」春梅笑道:「比你家老頭子那大貨差些兒。那個你倒捱了,這個你倒捱不的?好歹與我捱了。要不喫,月桂,你與我捏着鼻子灌他!」薛嫂道:「你且拿了點心與我,打了底兒着。」春梅道:「這老媽子單管說謊!你纔說在那裏喫了來,這囘又說沒打底兒?」薛嫂道:「喫了他兩個茶食,這咱還有哩?」月桂道:「薛媽媽,你且喫了這大鍾酒。我拿點心與你喫,俺奶奶又怪我沒用,要打我哩。」這薛嫂沒奈何,只得喫了。被他灌了一鍾,覺心頭小鹿兒劈劈跳起來。那春梅のの個嘴兒,又呌海棠斟滿一鍾教他喫。薛嫂推過一邊,說:「我的好娘,人家却一點兒也喫不的了。」海棠道:「你老人家捱了月桂姐一下子,不捱我一下子,奶奶要打我!」那薛嫂兒慌的直撅兒跪在地下。春梅道:「也罷,你拿過那餅與他喫了,敎他好喫酒。」月桂道:「薛媽媽,誰似我恁疼你,留下恁好玫瑰果餡餅兒與你喫!」就拿過一大盤子頂皮酥玫瑰餅兒來。那薛嫂兒只喫了一個,别的春梅都敎他袖在袖子裏:「到家捎與你家老王八喫。」薛嫂兒喫酒蓋着臉兒,把一盤子火薰肉、醃臘鵝,都用草紙包、布子裹,塞在袖内。海棠使氣白賴又灌了半鍾酒,見他嘔吐上來,纔收過家伙去,不要他喫了。春梅吩咐:「明日來討話說,兑丫頭銀子與你。」又使海棠問孫二娘去,囘來說:「丫頭留下罷,教大娘娘與他銀子。」臨出門拜辭,春梅吩咐:「媽媽,休推聾裝啞,那翠雲子做的不好,明日另帶兩副好的我瞧。」薛嫂道:「我知道。奶奶叫個大姐送我送,看狗咬了我腿。」春梅笑道:「俺家狗都有眼,只咬到骨秃跟前就住了。」一面使蘭花送出角門來。

話休饒舌。周守備至日落時分,牌兒馬藍旗作隊,叉槊後隨,出巡來家。進入後廳,左右丫鬟接了冠服。進房見了春梅,小衙内,心中歡喜。坐下,月桂海棠拿茶喫了,將出巡之事告訴一遍。不一時,放桌兒擺飯。飯罷,掌上燭,安排盃酌飲酒,因問:「前邊沒甚事?」春梅一面取過薛嫂拿的帖兒來與守備看,說吴月娘那邊如此這般,「小廝平安兒偸了頭面,被吴巡檢拿住監禁,不容領贓,只拷打小廝,攀扯誣賴吳氏奸情,索要銀兩,呈詳府縣」等事,守備看了說:「此事正是我衙門裏事,如何呈詳府縣?吴巡檢那廝,這等可惡!我明日出牌連他都提來發落。」又說:「我聞得這吴巡檢是他門下夥計,只因往東京與蔡太師進禮,帶挈他做了這個官,如何倒要誣害他家!」春梅道:「正是這等說,你替他明日䖏䖏罷。」一宿晚景題過。

次日,旋教吴月娘家補了一紙狀,當廳出了個大花欄批文,用一個封套裝了,上面批:「山東守禦府為失盜事,仰巡檢司官連人解繳。右差虞侯張勝李安,准此。」當下二人領出公文來,先到吴月娘家。月娘管待了酒飯,每人與了一兩銀子鞋脚錢。傅夥計家中睡倒了,吴二舅跟隨到巡檢司。吴巡檢見平安監了兩日,不見西門慶家中人來打點,正教吏典做文書申呈府縣。只見守禦府中兩個公人到了,拿出批文來與他。見封套上硃紅筆標着「仰巡檢司官連人解繳」,拆開見裏面吴氏狀子,唬慌了,反賠下情,與李安張勝每人二兩銀子。隨即做文書,解人上去,到於守備府前伺候。半日,待的守備升廳,兩邊軍牢排下,然後带進人去。這吴巡檢把文書呈遞上去,守備看了一遍說:「此正是我這衙門裏事,如何不申解前來我這裏發送?只顧延捱監滞,顯有情獘!」那吴巡檢禀道:「小官纔待做文書,申呈老爺案下,不料老爺鈞批到了。」守備喝道:「你這狗官可惡!多大官職,這等欺玩法度,抗違上司。我欽奉朝廷勅命,保障地方,巡捕盗賊,提督軍門,兼管河道,職掌開載已明。你如何拿了起件,不行申解?妄用刑杖拷打犯人,誣攀無辜,顯有情弊!」那吴巡檢聽了,摘去冠帽,在堦前只顧磕頭。守備道:「本當參治你這狗官,且饒你這遭。下次再若有犯,定行參究!」一面把平安提到廳上,說道:「你這奴才,偷盗了財物,還肆言謗主!人家都似你恁如此,也不敢使奴才了。」喝令左右:「與我打三十大棍放了!將贓物封貯,教本家人來領去。」一面喚進吴二舅來,遞了領狀。守備這裏還差張勝拿帖兒,同送到西門慶家,見了分上。吳月娘打發張勝酒飯,又與了一兩銀子。走來府裏,囘了守備春梅話。那吴巡檢乾拿了平安兒一塲,倒折了好幾兩銀子。

月娘還了那人家頭面、鈎子,見是他原物,一聲兒沒言語去了。傅夥計到家,傷寒病睡倒了。只七日光景,調治不好,嗚呼哀哉死了。月娘見這等合氣,把印子舖只是收本錢,贖討,再不假當出銀子去了。止是敎吴二舅同玳安在門首生薬舖子,日逐賺得來家中盤纏。此事表過不題。

一日,吴月娘叫將薛嫂兒來,與了三兩銀子。薛嫂道:「不要罷,傳的府裏小奶奶怪我。」月娘道:「天不使空人,多有累你。我見他不題出來就是了。」於是買了四盤下飯,宰了一口鮮猪,一罈南酒,一疋紵絲尺頭,薛嫂押着,來守備府中致謝春梅。玳安穿着青絹褶兒,用描金匣兒盛着禮帖兒,逕到裏邊見春梅。薛嫂領着到後堂。春梅出來,戴了金梁冠兒,金釵梳,鳳鈿,上穿繡襖,下着錦裙,左右丫鬟養娘侍奉。玳安兒趴倒地下磕頭。春梅吩咐放桌兒,擺茶食與玳安喫。説道:「沒甚事,你奶奶免了罷,如何又費心送這許多禮來?你周爺一定不肯受。」玳安道:「家奶奶說:前日平安兒這場事,多有累周爺周奶奶費心。沒甚麽,些小微禮兒,與爺奶奶賞人便了。」春梅道:「如何好受的?」薛嫂道:「你老人家若不受,惹那頭又怪我。」春梅一面又請進守備來計較了,止受了猪酒下飯,把尺頭囘將來了。與了玳安一方手帕,三錢銀子。擡盒人二錢。春梅因問:「你奶奶哥兒好麽?」玳安説:「哥兒好不耍子兒哩!」又問玳安兒:「你幾時籠起頭去、包了網巾?幾時和小玉完房來?」玳安道:「是八月内來。」春梅道:「到家多頂上你奶奶,多謝了重禮。待要請你奶奶來坐坐,你周爺早晚又出巡去。我到過年正月裏,哥兒生日,我往家裏走走。」玳安道:「你老人家若去,小的到家就對俺奶奶說,到那日來接奶奶。」說畢,打發玳安出門。薛嫂便向玳安兒說:「大官兒,你先去罷,奶奶還要與我說話哩。」

那玳安兒押盒擔來家,見了月娘,說如此這般,「守備只受了猪酒下飯,把尺頭囘將來了。春梅姐讓到後邊,管待茶食喫。問了囘哥兒好,家中長短。與了我一方手帕,三錢銀子。擡盒人二錢銀子。多頂上奶奶,多謝重禮。都不受來,被薛嫂兒和我再三說了,纔受了下飯猪酒,擡囘尺頭。要不是,請奶奶過去坐坐,一兩日周爺出巡去。他只到過年正月孝哥生日,來家裏走走。」又告說:「他住着五間正房,穿着錦裙繡襖,戴着金梁冠兒。出落的越發胖大了。手下好少丫頭奶子侍奉!」月娘問:「他眞實說明年往咱家來?」玳安兒道:「委的對我說來。」月娘道:「到那日咱這邊使人接他去。」因問:「薛嫂怎的還不來?」玳安道:「我出門,他還坐着説話,教我先來了。」自此兩家交往不絶。正是:世情看冷暖,人面逐高低。有詩為證:

\begin{myquote}
得失榮枯命裏該,皆因年月日時栽。

胸中有志應須至,囊裏無財莫論才。
\end{myquote}

畢竟未知後來何如,且聽下囬分解。

