\includepdf[pages={101,102},fitpaper=false]{tst.pdf}
\chapter*{第五十一囬 \\月娘聽演金剛科 桂姐躲在西門宅}
\addcontentsline{toc}{chapter}{第五十一囬 月娘聽演金剛科 桂姐躲在西門宅}
\markboth{\titlename}{第五十一囬 月娘聽演金剛科 桂姐躲在西門宅}


\begin{myquote}
羞看鸞鏡惜朱顔,手托香腮懶去眠。

瘦損纖腰寬翠帶,淚流粉面落金鈿。

薄倖惱人愁切切,芳心撩亂恨綿綿。

何時借得東風便,刮得檀郎到枕邊。
\end{myquote}

話説潘金蓮見西門慶拿了淫器包兒在李瓶兒房裏歇了,足惱了一夜没睡,懷恨在心。到第二日,打聽西門慶往衙門裏去了,李瓶兒在屋裏梳頭,老早走到後邊,對月娘説:「李瓶兒背地好不説姐姐哩。説姐姐『會那等虔婆勢,喬作衙,別人生日,喬作家管。你漢子吃醉了,進我屋裏來,我又不曾在前邊,平白對着人羞我,望着我丢臉兒。教我惱了,走到前邊把他爹趍到後邊來。落後他怎的也不在後邊?還往我房裏來了!』咱兩個黑夜説了一夜梯己話兒,只有心腸五臟沒曾倒與我罷了。」這月娘聽了,如何不惱!因向大妗子孟玉樓説:「早是你昨日也在跟前看着,我又沒曾説他甚麽!小廝交灯籠進來,我只問了一聲:『你爹怎的不進來?』小廝倒説往六娘屋裏去了。我便説:『你二娘這裏等着,恁沒槽道,却不進來。』論起來也不傷他,怎的説我虔婆勢喬作衙?我是淫婦老婆?我還把他當好人看承,原來知人知面不知心,那裏看人去!乾淨是個綿裏針、肉裏刺的貨!還不知背地在漢子跟前架的甚麽舌兒哩?怪道他昨日決烈的就往前走了。儍姐姐,那怕漢子成日在你那屋裏不出門,休想我這心動一動兒。一個漢子丢與你們,隨你們去,守寡的不過!想着一娶來之時,賊強人和我門裏門外不相逢,那等怎麽過來?」大妗子在傍勸道:「姑娘罷麽,都看着孩兒的分上罷。自古宰相肚裏好行舡,當家人是個惡水缸兒,好的也放在你心裏,歹的也放在心裏。」月娘道:「不拘幾時,我也要對這兩句話,等我問着他:我怎麽虔婆勢,喬作衙?」金蓮慌的没口子説道:「姐姐寬恕他罷!常言大人不責小人過,那個小人没罪過?他在屋裏背地調唆漢子,俺們這幾個誰没吃他排説過?我和他緊隔着壁兒,要與他一般見識起來倒了不成,行動只倚逞着孩子降人!他還説的好話兒哩,説他的孩兒到明日長大了,有恩報恩,有仇報仇。俺們都是餓死的數兒,你還不知道哩!」吴大妗子道:「我的奶奶,那裏有此話説!」月娘一聲兒也没言語。

常言路見不平,也有向燈向火。不想西門大姐平日與李瓶兒最好,常没針線鞋面,李瓶兒不拘好綾羅緞帛就與之;好汗巾手帕兩三方背地與大姐,銀錢是不消説。當日聽了此話,如何不告訴他?李瓶兒正在屋裏,與孩子做那端午戴的那絨線符牌兒,及各色紗小粽子兒,幷解毒艾虎兒,只見大姐走來,李瓶兒讓他坐,同看做生活。李瓶兒教迎春:「拿茶與你大姑娘吃。」一面吃了茶,大姐道:「頭裏請你吃茶,你怎的不來?」李瓶兒道:「打發他爹出門,我趕早涼兒,與孩子做這戴的碎生活兒來。」大姐道:「有樁事兒,我也不是舌頭,敢來告你説。學説你説俺娘虔婆勢,你沒曾惱着五娘?他在後邊對着俺娘如此這般,説了你一篇是非。如今俺娘要和你對話哩!你別要説我對你説,教他怪我。你須預備些話兒打發他。」這李瓶兒不聽便罷,聽了此言,手中拿着那針兒通㧱不起來,兩隻胳膊都軟了,半日説不出話來。對着大姐掉眼淚,説道:「大姑娘,我那裏有一字兒閒話!昨晚我在後邊,聽見小廝説他爹往我這邊來了,我就來到前邊催他往後邊去了,再誰説一句話兒來?你娘恁覷我一場,莫不我恁不識好歹,敢説這個話?設使我就説,對着誰説來,也有個下落!」大姐道:「他聽見俺娘説不拘幾時要對這話,他如何就慌了?要着我,你兩個當面鑼對面鼓的對不是!」李瓶兒道:「我對的過他那嘴頭子?自憑天罷了!他左右晝夜算計的我。只是俺娘兒兩個,到明日裏料吃他算計了一個去,也是了當!」説畢哭了。大姐坐着勸了一囬,只見小玉來請六娘,大姑娘吃飯,就後邊去了。李瓶兒丢下針指,同大姐到後邊,也不曾吃飯,回來房中,倒在床上就睡着了。西門慶衙門中來家,見他睡,問迎春,迎春道:「俺娘一日飯也還沒吃哩!」慌了西門慶,向前問道:「你怎的不吃飯?你對我説。」又見他哭的眼紅紅的,只顧問:「你心裏怎麼的?對我説!」那李瓶兒連忙起來,揉了揉眼,説道:「我害眼疼,不怎的。今日心裏懶待吃飯。」並不題出一字兒來。正是:滿懷心腹事,盡在不言中。有詩為證:

\begin{myquote}
莫道佳人總是癡,惺惺伶俐沒便宜。

只因會盡人間事,惹得閒愁滿肚皮!
\end{myquote}

大姐在後邊對月娘説:「我問他來,他説没有此話,『我對着誰説來?』且是好不賭身罰咒,望着我哭哩。説娘這般看顧他,他肯說此話?」吴大妗子道:「我就不信。李大姐好個人兒,他原肯説這等謊?」月娘道:「想必兩個不知怎的有些小節不足,哄不動漢子,走來後邊戳無路兒,沒的拿我墊舌根。我這裏還多着個影兒哩!」大妗子道:「大姑娘,今後你也别要虧了人。不是我背他説,潘五姐一百個不及他!為人心地兒又好,來了咱家恁二三年,要一些歪樣兒也沒有。」

正説着,只見琴童兒藍布大包袱背進來。月娘問:「是甚麽?」琴童道:「是三萬鹽引。韓夥計和崔本纔従関上掛了號來。爹説打發飯與他二人吃。如今兌銀子打包,後日二十是好日子起身,打發他三個往揚州去。」吴大妗子道:「只怕姐夫進來,我和二位師父往他二娘房裏坐去罷。」剛説未畢,只見西門慶掀簾子進來,慌的吴妗子和薛姑子王姑子,往李嬌兒屋裏走不迭。早被西門慶看見,問月娘:「那個是薛姑子?賊胖秃淫婦,來我這裏做什麽?」月娘道:「你好恁枉口拔舌!不當家化化的,罵他怎的!他惹着你來?你怎的知道他姓薛?」西門慶道:「你還不知他弄的乾坤兒哩!他把陳參政家小姐,七月十五日,吊在地藏菴兒裏,和一個小夥阮三偸奸。不想那阮三就死在女子身上。他知情,受了十兩銀子。事發拿到衙門裏,被我褪衣打了二十板,教他嫁漢子還俗。他怎的還不還俗?好不好,拿到衙門裏,再與他幾拶子!」月娘道:「你有要没緊,恁毀神謗佛的!他一個佛家弟子,想必善根還在,他平白還甚麽俗?你還不知,他好不有道行!」西門慶道:「你問他,有道行一夜接幾個漢子?」月娘道:「你就休汗邪,又討我那没好口的罵你!」因問:「幾時打發他三個起身?」西門慶道:「我剛纔使來保會喬親家去了。他那裏出五百兩,我這裏出五百兩。二十是個好日子,打發他們起身去罷了。」月娘道:「線舖子却教誰開?」西門慶道:「且教賁四替他開着罷。」説畢,月娘開箱子拿出銀子,一面兑了出來交付與三人,正在捲棚内看着打包。每人兑與他五兩銀子,叫他家中收拾衣装行李,不在話下。

只見應伯爵走到捲棚裏,見西門慶看着打包,便問:「哥打包做甚麽?」西門慶因把二十日打發來保等往揚州支鹽去一節,告訴一遍。伯爵擧手道:「哥,恭喜!此去囬來,必有大利息。」西門慶一面讓他坐,喚茶來吃了。因問:「李三黄四銀子幾時関?」應伯爵道:「也只不出這個月裏就関出來了。他昨日對我説,如今東平府又派下二萬香來了,還要問你挪五百兩銀子,接濟他這一時之急。如今関出這批的銀子,一分也不動,都擡過這邊來。」西門慶道:「倒是你看見,我這裏打發揚州去,還沒銀子,問喬親家那裏借了五百兩在裏頭。那討銀子來?」伯爵道:「他再三央及待我對你説,一客不煩二主。你不接濟他這一步兒,教他又問那裏借去?」那西門慶道:「門外街東徐四舖少我銀子,我那裏挪五百兩銀子與他罷。」伯爵道:「可知好哩!」

正説着,只見平安兒拿進帖兒來説:「夏老爹家差了夏壽送來,請爹明日坐坐。」西門慶展開柬帖云云,道:「曉得了。」伯爵道:「我今敢來有樁事兒來報與哥。你知道院裏李桂兒勾當?他没來?」西門慶道:「他従正月去了,再幾時來?我並不知道甚麽勾當。」伯爵因説起:「王招宣府裏第三的,原來是東京六黃太尉侄女兒女婿,従正月往東京拜年,老公公賞了一千兩銀子與他兩口兒過節。你還不知,六黃太尉這侄女兒生的怎麽標緻,上畫兒委的只畫半邊兒也没恁俊俏相的!你只守着你家裏的罷了,每日被老孫、祝麻子、小張閑,三四個摽着在院裏撞,把二條巷齊家那小丫頭子齊香兒梳籠了,又在李桂兒家走。把他娘子兒的頭面都拿出來當了,氣的他娘子兒家裏上吊。不想前日,這月裏老公公生日,他娘子兒到東京,只一説,老公公惱了,將這幾個人的名字送與朱太尉。朱太尉批行東平府,着落本縣拿人。昨日把老孫祝麻子與小張閑都従李桂兒家拿的去了。李桂兒便躱在隔壁朱毛頭家過了一夜。今日説來你這裏央及你來了。」西門慶道:「我説正月裏都摽着他走,這裏誆人家銀子,那裏誆人家銀子,那祝麻子還對着我搗生鬼!」説畢,伯爵道:「我去罷,等住囬,只怕李桂兒來,你管他不管他,他又説我來串作你。」西門慶道:「你且坐着,我還和你説哩。李三你且別要許他,等我門外討銀子出來,和你説話去。」伯爵道:「我曉的。」剛走出大門首,只見李桂姐轎子在門首,又早下轎進去了。

西門慶正吩咐陳經濟,教他騎騾子往門外徐四家催銀子去,只見琴童兒走到捲棚内請西門慶,道:「大娘後邊請。有李桂姨來了。」這西門慶走到後邊,只見李桂姐身穿茶色衣裳,也不搽臉,用白挑線汗巾子搭着頭,雲鬟不整,花容淹淡,與西門慶磕着頭哭起來,説道:「爹!可怎麽樣兒好,恁造化低的營生!正是関着門兒家裏坐,祸従天上來。一個王三官兒,俺們又不認的他,平日的祝麻子孫寡嘴領了來俺家來討茶吃。俺姐姐又不在家,依着我説,別要招惹他。那些兒不是俺這媽,越發老的韶刀了。就是來宅裏與俺姑娘做生日的這一日,你上轎來了就是了,見祝麻子打旋磨兒跪着,従新又囬去。對我説,姐姐,你不出來待他鍾茶兒,却不難為囂了人了。他便往爹這裏來了,教我把門插了不出來。誰想従外邊撞了一夥人來,把他三個不由分説都拿的去了。王三官兒便奪門走了,我便走在隔壁人家躲了。家裏有個人牙兒?纔使保兒來這裏接的他家去。到家,把媽唬的魂兒也没了,只要尋死。今日縣裏皂隸,又拿着票喝囉了一清早,起身去了。如今坐名兒只要我往東京囬話去。爹,你老人家不可憐見救救兒,却怎麽樣兒的?娘在傍邊也替我説説兒。」西門慶笑道:「你起來。」因問:「票上還有誰的名字?」桂姐道:「還有齊香兒的名字,他梳籠了齊香兒,在他家使錢着,便該當。俺家若見了他一個錢兒,就把眼睛珠子掉了!若是沾他沾身子兒,一個毛孔兒裏生一個天疱瘡!」月娘對西門慶道:「也罷,省的他恁説誓剌剌的,你替他説説罷。」西門慶道:「如今齊香兒拿了不曾?」桂姐道:「齊香兒他在王皇親宅裏躲着哩。」西門慶道:「既是恁的,你且在我這裏住兩日。倘人來尋你,我就差人往縣裏替你説去。」於是就叫書童兒:「你快寫個帖兒,往縣裏見你李老爹,就説桂姐常在我這裏答應,看怎的免提他罷。」書童應諾,穿青絹衣服去了。

不一時,拿了李知縣回帖兒來。書童道:「李老爹説:多上覆你老爹,別的事無不領命,這個卻是東京上司行下來批文,委本縣拿人;縣裏只拘的人在。既是你老爹分上,我這裏且寬限他兩日。要免提,還往東京上司處説去。」西門慶聽了,只顧沉吟,説道:「如今來保一兩日起身,東京沒人去。」月娘道:「也罷,你打發他兩個先去,存下來保,替桂姐往東京説了這勾當,教他隨後邊趕了去,也是不遲。你看唬的他那腔兒!」那桂姐連忙與月娘和西門慶磕頭。

西門慶隨使人叫將來保來,吩咐:「二十日你且不去罷,教他兩個先去。你明日且往東京替桂姐説説這勾當來,見你翟爹,如此這般,好歹差人往衛裏説説。」桂姐連忙就與來保下禮。慌的來保頂頭相還,説道:「桂姨,我就去。」西門慶一面教書童兒寫就一封書,致謝翟管家:「前日曾巡按之事,甚是費心。」又封了二十兩折節禮銀子,連書交與來保。桂姐便歡喜了,拿出五兩銀子來,與來保路上做盤纏,説道:「回來俺媽還重謝保哥。」西門慶不肯,還教桂姐收了銀子。教月娘另拿五兩銀子與來保盤纏。桂姐道:「也没這個道理!我央及爹這裏説人情,又教爹出盤纏?」西門慶道:「你笑話我没這五兩銀子盤纏了,要你的銀子?」那桂姐方纔收了。向來保拜了又拜,説道:「累保哥,明日好歹起身罷,只怕遲了。」來保道:「我明日早五更就走道兒了。」於是領了書信,又走到獅子街韓道國家。

王六兒正在屋裏替他縫小衣兒哩,打窗眼看見是來保,忙道:「你有甚説話?請房裏坐。他不在家,往裁縫那裏討衣裳去了,便來也。」便叫錦兒:「還不往對過徐裁家叫你爹去!你説保大爺在這裏。」來保道:「我敢來説聲,我明日且去不成,又有樁業障鑽出來。當家的㽞下,教我往東京替院裏李桂姐説人情去哩。他剛纔在爹跟前再三磕頭禮拜央及我。娘和爹説:『也罷,你且替他往東京走一遭,説説這勾當。且叫韓夥計和崔大官兒先去。你囬來再趕了去,也是不遲。』我明日早起身了,剛纔書也有了。」因問:「嫂子,你做的是甚麽?」王六兒道:「是他的小衣裳兒。」來保道:「你教他少帶衣裳。到那去處,是出紗羅緞絹的窝兒裏,愁没衣裳穿?」正説着,韓道國來了,兩個唱了喏,因把前事説了一遍。因説:「我到明日揚州那裏尋你們?」韓道國道:「老爹吩咐,教俺們馬頭上投經紀王伯儒店裏下。説過世老爹曾和他父親相交,他店内房屋寬廣,下的客商多,放財物不躭心。你只往那裏尋俺們就是了。」又説:「嫂子,我明日東京去,你沒甚鞋脚東西捎進府裏,與你大姐去?」王六兒道:「沒甚麽,只有他爹替他打的兩對簪兒,幷他兩雙鞋,起動保叔捎捎進去與他。」於是用手帕包縫停當,遞與來保。一面教春香看菜兒篩酒,婦人連忙丢下生活,就放桌兒。來保道:「嫂子,你休費心,我不坐。我到家還收拾了褡褳,明日好起身。」王六兒笑嘻嘻道:「耶嚛,你怎的上門怪人家!夥計家,自恁與你餞行,也該吃鍾兒。」因説韓道國:「你好老實,桌兒不穩,你也撒撒兒讓保叔坐,只像沒事的人兒一般兒!」於是拿上菜兒來,斟酒遞與來保,王六兒也陪在傍邊。三人坐定吃酒。

來保吃了幾鍾,説道:「我家去罷。晚了,只怕家裏関門早。」韓道國問道:「你頭口僱下了不曾?」來保道:「明日早僱罷了。」因説:「舖子裏鑰匙並帳簿,都交與賁四罷了,省的你又上宿去。家裏歇息歇息好走路兒。」韓道國道:「夥計説的是。我明日就交與他。」王六兒又斟了一甌子,説道:「保叔,你只吃這一鍾,我也不敢留你了。」來保道:「嫂子,你既要我吃,再篩熱着些。」那王六兒連忙歸到壺裏,交錦兒炮熱了,傾在盞内,雙手遞與來保,説道:「沒甚好菜兒與保叔下酒。」來保道:「嫂子,好説,家無常禮。」拿起酒來,與婦人對飲,一吸而同乾,方纔作辭起身。王六兒便把女兒鞋脚遞與他,説道:「累保叔,好歹到府裏問聲孩子好不好,我放心些。」於是道了萬福,兩口兒齊送出門來。不説來保到家收拾行李,第二日起身東京去了,不題。

單表月娘上房擺茶與桂姐吃。吴大妗子、楊姑娘、兩個姑子,都做一處坐。有吴大舅前來對西門慶説:「有東平府行下文書來,派俺本衙兩所掌印千户管工脩理社倉,題准旨意,限六月工完,陞一級;違限,聽巡按御史查參。姐夫有銀子,借得幾兩工上使用。待関出工價來,一一奉還。」西門慶道:「大舅用多少,只顧拿去。」吴大舅道:「姐夫下顧,與二十兩罷。」一面進入後邊,見了月娘説了話,教月娘拿二十兩出來交與大舅,又吃了茶,出來。因後邊有堂客,不好坐的,月娘教西門慶㽞大舅大廳上吃酒。

正飲酒中間,只見陳經濟走來囬話説:「門外徐四家銀子,頂上爹,再讓兩日兒。」西門慶道:「胡説!我這裏等銀子使,再讓兩日兒?照舊還去,罵那狗弟子孩兒!」經濟應諾。吴大舅讓:「姐夫坐的!」陳經濟作了揖,打横坐了,琴童兒連忙安放了鍾筯。這裏前邊吃酒。且説後邊大妗子、楊姑娘、李嬌兒、孟玉樓、潘金蓮、李瓶兒、大姐,都伴桂姐在月娘房裏吃酒。先是郁大姐數了囬《張生遊寳塔》,放下琵琶。孟玉樓在傍斟酒布菜兒與他吃,説道:「賊瞎拽磨的,唱了這一日,又説我不疼你!」那潘金蓮又大筯子夾腿肉,放在他鼻子上,戲弄他頑耍。桂姐因叫玉簫:「姐,你遞過那郁大姐琵琶來,我唱個曲兒與姑奶奶和大妗子聽。」月娘道:「桂姐,你心裏熱剌剌的,不唱罷。」桂姐道:「不妨事,等我唱。見爹娘替我説人情去了,我這回不焦了。」孟玉樓笑道:「李桂姐倒還是院中人家娃娃,做臉兒快,頭裏一來時,把眉頭忔縐着,焦的茶兒也吃不下去。這回説也有,笑也有。」當下桂姐輕舒玉指,頓撥氷弦,唱了一囬。

正唱着,只見琴童兒收進家活來。月娘便問道:「你大舅去了?」琴童兒道:「大舅去了。」吳大妗子道:「只怕姐夫進來,俺們活變活變兒。」琴童道:「爹不往後邊來了,往五娘房裏去了。」這潘金蓮聽見往他屋裏去了,就坐不住,趨趄着脚兒只要走,又不好走的。月娘也不等他動身,説道:「他往你屋裏去了,你去罷,省的你欠肚兒親家似的!」那潘金蓮嚷:「可可兒的起來!」口兒裏硬着,那脚步兒且是去的快。來到前邊,入房來,西門慶已是吃了胡僧薬,教春梅脱了衣裳,在床上帳子裏坐着哩。金蓮看見笑道:「我的兒,今日好呀!不等你娘來就上床了。俺們剛纔在後邊陪大妗子楊姑娘吃酒,被李桂姐唱着,灌了我幾鍾好的。獨自一個兒,黑影子裏一步高,一步低,不知怎的就走的來了!」叫春梅:「你有茶,倒甌子我吃。」那春梅眞個點了茶來。金蓮吃了,撇了個嘴與春梅,那時春梅就知其意,那邊屋裏早已替他熱下水。婦人抖些檀香白礬在裏面,洗了牝。向燈下摘了頭,止撇着一根金簪子。拿過鏡子來,従新把嘴唇抹了些胭脂,口中噙着香茶,走過這邊來。春梅床頭上取過睡鞋來與他換了,帶上房門出來。

這婦人便將燈臺挪近床邊桌上放着,一手放下半邊紗帳子來。褪去紅褌,露見玉體。西門慶坐在枕頭上,那話帶着兩個托子,一會弄的大大的,露出來與他瞧。婦人燈下看見,唬了一跳,一手揝不過來,紫巍巍,沉甸甸,約有虎二。便眤瞅了西門慶一眼,説道:「我猜你沒別的話,一定吃了那和尚薬,弄聳的恁般大,一會要來奈何老娘。好酒好肉,王里長吃的去;你在誰人跟前試了新,這回剩了些殘軍敗將,纔來我這屋裏來了?俺們是雌剩ぎぐ㒲的,你還説不偏心哩!嗔道那一日我不在屋裏,三不知把那行貨包子偸的往他屋裏去了。原來晚夕和他幹這個營生,他還對着人撇清搗鬼哩!你這行貨子,乾淨是個沒挽回的三寸貨。想起來,一百年不理你纔好!」西門慶笑道:「小淫婦兒!你過來。你若有本事把他咂過了,我輸一兩銀子與你。」婦人道:「汗邪了你了,你吃了甚麽行貨子,我禁的過他!」於是把身子斜軃在袵席之上,雙手執定那話,用朱唇吞裹,説道:「好大行貨子!把人的口也撑的生疼的。」説畢,出入嗚咂,或舌尖挑弄蛙口,舐其龜弦,或用口噙着,往來哺摔;或在粉臉上偎㨪,百般搏弄,那話越發堅硬た崛起來,裂瓜頭凹眼圓睜,絡腮鬍挺身直豎。西門慶垂首窺見婦人香肌,掩映於紗帳之内,纖手捧定毛都魯那話往口裏吞放。燈下一來一往動彈。不想傍邊蹲踞着一個白獅子貓兒,看見動彈,不知當做甚物件兒,撲向前用爪兒來撾。這西門慶在上,又將手中拿的洒金老鴉扇兒只顧引鬦他耍子。被婦人奪過扇子來,把貓儘力打了一扇把子,打出帳子外去了。眤向西門慶道:「怪發訕的寃家,緊着這咂咂的不得人意,又引鬦他恁上頭上臉的,一時間撾了人臉,却怎樣的?好不好我就不幹這營生了!」西門慶道:「怪小淫婦兒,會張致死了!」婦人道:「你怎的不教李瓶兒替你咂來?我這屋裏,儘着教你掇弄!不知吃了甚麽行貨子,咂了這一日,一發咂了没事没事。」西門慶於是向汗巾兒上小銀盒兒裏,用挑牙挑了些粉紅膏子薬兒,抹在馬口内。仰卧於上,教婦人騎在身上。婦人道:「等我げ着,你往裏放。」龜頭昂大,濡硏半晌,僅沒龜稜。婦人在上,將身左右捱擦,似有不勝隱忍之態,因叫道:「親逹逹,裏邊緊,澀住了,好不難捱。」一面用手摸之。燈下窺見麈柄已被牝户吞進半截,撑的兩邊皆滿,無復作往來。婦人用唾津塗抹牝户兩邊,已而稍寬滑落,頗作往來,一擧一坐,漸沒至根。婦人因向西門慶説:「你每常使的顫聲嬌,在裏頭只是一味熱癢不可當,怎如和尚這薬,使進去従子宮冷森森直掣到心上。這一回把渾身上下都酥麻了。我曉的,今日這命死在你手裏了,好難捱忍也!」西門慶笑道:「五兒,我有個笑話兒説與你聽,是應二哥説的。一個人死了,閻王就拿驢皮披在身上,教他變驢。落後判官查簿籍,還有他十三年陽壽,又放囬來了。他老婆看見渾身都變過來了,只有陽物還是驢的,未變過來。那人道:『我往陰間换去。』他老婆慌了,説道:『我的哥哥,你這一去,只怕不放你回來怎了?由他,等我慢慢兒的挨罷。』婦人聽了,笑將扇把子打了一下子,説道:「怪不得應二老婆捱慣了驢的行貨,硶説嘴的貨,我不看世界,這一下打的你!」兩個足纏了一個更次,西門慶精還不過。他在下合着眼,由着婦人蹲踞在上,極力抽提,提的龜頭刮答刮答怪響。提夠良久,又掉過身子去,朝向西門慶。西門慶雙足擧其股,没稜露腦而提之,往來甚急。西門慶雖身接目視,而猶如無物。良久,婦人情極,轉過身子來,兩手摟定西門慶脖項,合伏在身上,舒舌頭在他口裏。那話直抵牝中,只顧揉搓,沒口子呌:「親達達,罷了!五兒的死了。」須臾一陣昏迷,舌尖冰冷,泄訖一度。西門慶覺牝中一股熱氣,直透丹田,心中翕翕然美快不可言也。已而淫津溢出,婦人以帕抹之,兩個相摟相抱,交頭疊股,鳴咂其舌,那話通不拽出來。睡的没半個時辰,婦人淫情未足,爬上身去,兩個又幹起來。婦人一連丢了兩遭,身子亦覺稍惓。西門慶只是佯佯不睬,暗想胡僧之薬通神。看看窗外鷄鳴,東方漸白。婦人道:「我的心肝,你不過卻怎樣的?到晚夕你再來,等我好歹替你咂過了罷。」西門慶道:「就咂也不得過,管情只一樁事兒就過了。」婦人道:「告我説是那一樁兒?」西門慶道:「法不傳六耳,待我晚夕來對你説。」

早晨起來梳洗,春梅打發穿上衣裳,韓道國崔本又早外邊伺候。西門慶出來,燒了紙,打發起身,交付二人兩封書:「一封到揚州馬頭上,投王伯儒店裏下;這一封就往揚州城内,找尋苗青,問他的事情下落,快來囬報我。如銀子不夠,我後邊再教來保捎去。」崔本道:「還有蔡老爹書没有?」西門慶道:「你蔡老爹書還不曾寫,教來保後邊捎了去罷。」二人拜辭,上頭口去了,不在話下。西門慶冠帶了,就往衙門中來,與夏提刑相會,道及日昨多承見招之意。夏提刑道:「今日奉屈長官一叙,再無他客。」發放已畢,各分散來家。吳月娘又早上房擺下菜蔬,請西門慶吃粥。

只見一個穿青衣皂隸,騎着快馬,夾着毡包,走的滿面汗流,到大門首問平安:「此是問刑西門老爹家?」平安道:「你是那裏來的?」那人即便下了馬作揖,便説:「我是督催皇木的安老爹先差來送禮與老爹。俺老爹與管磚廠黃老爹,如今都往東平府胡老爹那裏吃酒,順便先來拜老爹這裏,看老爹在家不在。」平安道:「有帖兒没有?」那人向毡包内取出,連禮物都遞與平安。平安拿進去與西門慶看,見禮帖上寫着:浙紬二端,湖綿四斤,香帶一束,古鏡一圓。吩咐:「包五錢銀子,拿回帖打發來人,就説在家拱候老爹!」那人急急去了。

西門慶一面家中預備酒菜,等至日中,二位官員喝道而至,皆乘轎,張蓋甚盛。先令人投拜帖,一個是「侍生安忱拜」,一個是「侍生黄葆光拜」。都是青雲白鷴補子,烏紗皂履,下轎揖讓而入。西門慶出大門迎接,至廳上敍禮。各道契闊之情,分賓主坐下。黃主事居左,安主事居右,西門慶主位相陪。先是黃主事擧手道:「久仰賢名,盛德芳譽,學生拜遲。」西門慶道:「不敢。辱承老先生先事枉駕,當容踵叩,敢問尊號?」安主事道:「黃年兄號泰宇,取『宇泰定者發乎天光』之意。」黃主事道:「敢問尊號?」西門慶道:「學生賤號四泉,因小莊有四眼井之説。」安主事道:「昨日會見蔡年兄,說他與宋松原都在尊府打攪。」西門慶道:「因承雲峯尊命,又是敝邑公祖,敢不奉迎?小价在京,已知鳳翁榮選,未得躬賀。」又問:「幾時家中起身來?」安主事道:「自去歲尊府別後,學生到家續了親,過了年,正月就來京了。選在工部,備員主事。欽差督運皇木,前往荆州。回來道經此䖏,敢不奉謁?」西門慶又説:「盛儀感謝不盡!」說畢,因請寬衣,令左右安放桌席。黃主事就要起身。安主事道:「實告,我與黃年兄如今還往東平胡大尹那裏赴席。因打尊府過,敢不奉謁?容日再來取擾。」西門慶道:「就是往胡公處,去路尚許遠。縱二公不餓,其如従者何?學生不敢具酌,只備一飯在此,以犒手下従者。」於是先打發轎子攢盤。廳上安放桌席,珍羞異品,極時之盛。就是湯飯點心,海鮮羙味,一齊上來。西門慶將小金鍾只奉了三盃,連桌兒擡下去,管待親隨家人吏典。少頃,兩位官人拜辭起身,向西門慶道:「生輩明日有一小柬到,奉屈賢公,到我這黃年兄同僚劉老太監莊上一敍,未審肯命駕否?」西門慶道:「既蒙寵招,敢不趨命!」説畢,送出大門,上轎而去。

只見夏提刑差人來邀。西門慶説道:「我就去。」一面吩咐備馬。走到後邊換了衣服,出來上馬,玳安琴童跟隨,排軍喝道,打着黑扇,逕往夏提刑家來。到廳上敘禮,説道:「適有工部督皇木安主政和磚廠黃主政來拜,㽞坐了半日,去了。不然也來的早。」見畢禮數,接了衣服下來,玳安叫排軍褶了,連帶放在氈包内。見廳上面設放兩張桌席,讓西門慶居左,其次就是西賓倪秀才。座間因敍起來,問道:「老先生尊號?」倪秀才道:「學生賤名倪鵬,字時遠,號桂巖,現在府庠備數。在我這東主夏老先生門下設館,教習賢郎大先生學業。友道之間,實有多愧。」説話間,兩個小優兒上來磕頭。吃罷湯飯,廚役上來割道。西門慶喚玳安拿賞賜賞了廚役,吩咐:「取巾來戴,把冠帶衣服送囘家去,晚上來接罷。」玳安應諾,吃了點心,回馬家來不題。

且説潘金蓮従打發西門慶出來,直睡到晌午纔爬起來。甫能起來,又懶待梳頭。恐怕到後邊人説他,月娘請他吃飯也不吃,只推不好。大後晌纔出房門,來到後邊。月娘因西門慶不在,要聽薛姑子講説佛法,演頌《金剛科儀》。正在明間内安放一張經桌兒,焚下香。薛姑子與王姑子兩個一對坐,妙趣妙鳳兩個徒弟立在兩邊,接念佛號。大妗子、楊姑娘、吴月娘、李嬌兒、孟玉樓、潘金蓮、李瓶兒、孫雪娥和李桂姐,一個不少,都在跟前,圍着他坐的,聽他演誦。先是薛姑子道:

\begin{myquote}
「蓋聞電光易滅,石火難留。落花無返樹之期,逝水絶歸源之路。畫堂繡閣,命盡有若風燈;極品高官,祿絶猶如作夢。黄金白玉,空為祸患之資;紅粉輕裘,總是塵勞之費。妻孥無百載之歡,黑暗有千重之苦。一朝枕上,命掩黃泉。空榜揚虚假之名,黄土埋不堅之骨。田園百頃,其終被兒女爭奪;綾錦千箱,死後無寸絲之分。青春未半,而白髮來侵;賀者纔聞,而吊者隨至。苦苦苦,氣化清風塵歸土!點點輪迴喚不回,改頭換面無遍數。

南無盡虚空遍法界過見未來佛法僧三寳。

無上甚深微妙法,百千萬劫難遭遇。

我今見聞得受持,願解如來眞實義!」
\end{myquote}

王姑子道:「當時釋伽牟尼佛,乃諸佛之祖,釋教之主。如何出家?願聽演説。」薛姑子便唱〔五供養〕:

\begin{myquote}
「釋伽佛,梵王子!捨了江山雪山去,割肉喂鷹鵲巢頂。只修的,九龍吐水混金身,纔成南無大乘大覺釋伽尊。」
\end{myquote}

王姑子又道:「釋伽佛,旣聽演説。當日觀音菩薩,如何修行,纔有莊嚴百億化身,有大道力,願聽其説。」薛姑子又道:

\begin{myquote}
「大莊嚴,妙善主!辭別皇宫香山住,天人送供跏趺坐。只修的,五十三參變化身,纔成南無救苦救難觀世音。」
\end{myquote}

王姑子道:「觀音菩薩,旣聽其法。昔日有六祖禪師傳燈佛,教化行西域,東歸不立文字。如何苦功,願聽其詳。」薛姑子又道:

\begin{myquote}
「達磨師,盧六祖!九年面壁功行苦,蘆芽穿膝伏龍虎。只修的,隻履折蘆任往來,纔成了南無大慈大願昆盧佛。」
\end{myquote}

王姑子道:「六祖傳燈,旣聞其詳。敢問昔日有個龐居士,捨家私送寳船歸海,以成正果。如何説?」薛姑子道:

\begin{myquote}
「龐居士,善知識!放債來生濟貧苦,驢馬夜間私相語。只修的,抛妻棄子上法舡,纔成了南無妙乘妙法伽藍耶。」
\end{myquote}

月娘正聽到熱鬧處,只見平安兒慌慌張張走來説道:「巡按宋爺家,差了兩個快手一個門子送禮來。」月娘慌了,説道:「你爹往夏家吃酒去了,誰人打發他?」正亂着,只見玳安兒放進毡包來,説道:「不打緊,等我拿帖兒對爹説去。教姐夫且讓那門子進來,管待他些酒飯兒着。」這玳安交下氈包,拿着帖子,騎馬雲飛般走到夏提刑家,如此這般説了:「巡按宋老爺送禮來。」西門慶看了帖子,上面寫着:鮮猪一口,金酒二尊,公紙四刀,小書一部。」下書「侍生宋喬年拜」。連忙吩咐:「到家教書童快拿我的官銜雙摺手本回去。門子答賞他三兩銀子、兩方手帕,擡盒的每人與他五錢。」玳安來家,到處尋書童兒,那裏得來?急的只遊回磨轉。陳經濟又不在,教傅夥計陪着人吃酒。玳安旋打後邊樓房裏討了手帕銀子出來,又沒人封,自家在櫃上彌封停當,教傅夥計寫了,大小三包。因問平安兒道:「你就不知他往那去了?」平安道:「頭裏姐夫在家時,他還在家來。落後姐夫往門外討銀子去了,他也不見了!」玳安道:「別要題,一定秫秫小廝在外邊胡行亂走的,養老婆去了!」正在急噪之間,只見陳經濟與書童兩個,疊騎着騾子纔來。被玳安駡了幾句,教他寫了官銜手本,打發送禮人去了。玳安道:「賊秫秫小廝,仰げ着掙了,合縫着丢!爹不在,家裏不看,跟着人養老婆兒去了!爹又没使你和姐夫門外討銀子,你平白跟了去做甚麽?看我對爹説不説!」書童道:「你説不是,我怕你?你不説,就是我的兒!」玳安道:「賊狗攮的秫秫小廝,你賭個兒眞個!」走向前,一個潑脚撇翻倒,兩個就磆碌成一塊子。那玳安得手,吐了他一口唾沫,纔罷了。説道:「我接爹去。等我來家,和淫婦算帳!」騎馬一直去了。

月娘在後邊,打發兩個姑子吃了些茶食兒,又聽他唱佛曲兒,宣念偈子兒。那潘金蓮不住在傍,先拉玉樓,不動,又扯李瓶兒,又怕月娘説。月娘便道:「李大姐,他呌你,你和他去不是,省的急的他在這裏恁有㓦劃沒使䖏的!」那李瓶兒方纔同他出來。被月娘瞅了一眼,説道:「拔了蘿蔔地皮寬。敎他去了,省的他在這裏跑兔子一般。原不是那聽佛法的人!」

這潘金蓮拉着李瓶兒走出儀門,因説道:「大姐姐好幹這營生!你家又不死人,平白教姑子家中宣起卷來了!都在那裏圍着他怎的?咱們出來走走,就看看大姐在屋裏做甚麽哩!」於是一直走出大廳來。只見廂房内點着燈,大姐和經濟正在裏面絮聒,説不見了銀子了。被金蓮向窗櫺上打了一下,説道:「後面不去聽佛曲兒,兩口子且在房裏拌的甚麽嘴兒?」陳經濟出來,看見二人,説道:「早是我沒曾罵出來!原來是五娘六娘來了。請進來坐。」金蓮道:「你好膽子,罵不是?」進來見大姐正在燈下衲鞋,説道:「這早晚,熱剌剌的,還衲鞋?」因問:「你兩口子嚷的是些甚麼?」陳經濟道:「你問他!爹使我門外討銀子去。他與了我三錢銀子,就教我替他捎銷金汗巾子來。不想到那裏,袖子裏摸銀子没了,不曾捎得來。來家他説我那裏養老婆,和我嚷罵了這一日,急的我賭身發咒。不想丫頭掃地,地下拾起來。他把銀子收了不與,還敎我明日買汗巾子來。你二位老人家説,却是誰的不是?」那大姐便罵道:「賊囚根子,别要説嘴!你不養老婆,平白带了書童兒去做甚麽?剛纔教玳安甚麽不駡出來。想必兩個打夥兒養老婆去來,去到這早晚纔來!你討的銀子在那裏?」金蓮問道:「有了銀子了不曾?」大姐道:「有了,銀子剛纔丫頭地下掃地拾起來,我拿着哩。」金蓮道:「不打緊處,我與你銀子,明日也替我带兩方銷金汗巾子來。」李瓶兒便問:「姐夫,門外有賣銷金汗巾兒,也捎幾方兒與我。」經濟道:「門外手帕巷有名王家,專一發賣各色花樣銷金點翠手帕汗巾兒,隨你便多少也有。你老人家要甚顔色?銷甚花樣?早説與我,明日一齊都替你带來了。」李瓶兒道:「我要一方老金黃銷金點翠穿花鳳汗巾。」經濟道:「六娘,老金黃銷上金,不現。」李瓶兒道:「你別要管我。我還要一方銀紅綾銷江牙海水嵌八寳汗巾兒;又是一方閃色芝蔴花銷金汗巾兒。」經濟便道:「五娘,你老人家要甚花樣?」金蓮道:「我沒銀子,只要兩方兒夠了。要一方玉色綾瑣子地兒銷金汗巾兒。」經濟道:「你又不是老人家,白剌剌的,要他做甚麽?」金蓮道:「你管他怎的?戴不的,等我往後吃孝戴!」經濟道:「那一方是甚顔色?」金蓮道:「那一方,我要嬌滴滴紫葡萄顔色四川綾汗巾兒,上銷金,間點翠,十樣錦,同心結,方勝地兒,一個方勝兒裏面一對兒喜相逢,兩邊欄子兒都是纓絡珎珠碎八寳兒。」經濟聽了,説道:「耶嚛,耶嚛!再沒了?賣瓜子兒開箱子打㖒噴——瑣碎一大堆!」那金蓮道:「怪短命,有錢買了稱心貨,隨各人心裏所好,你管他怎的?」李瓶兒便向荷包裏拿出一塊銀子兒,遞與經濟,説:「連你五娘的,都在裏頭哩。」那金蓮搖着頭兒,説道:「等我與他罷。」李瓶兒道:「都一答兒的教姐夫捎來,你又起個窖兒?」經濟道:「就是連五娘的,這銀子還多着哩。」一面取等子稱了,一兩九錢。李瓶兒道:「剩下的就與大姑娘捎兩方來。」那大姐連忙道了萬福。金蓮道:「你六娘替大姐買了汗巾兒,把那三錢銀子拿出來,你兩口兒鬦葉兒,賭個東道兒罷。少,便叫你六娘貼些兒出來,明日等你爹不在了,買燒鴨子白酒咱們吃。」經濟道:「旣是五娘説,拿出來。」大姐遞與金蓮,金蓮交付與李瓶兒收着。拿出紙牌來,燈下大姐與經濟鬦。金蓮又在傍替大姐指點,登時贏了經濟三桌。

忽聽前邊打門,西門慶來家,金蓮同李瓶兒纔囬房去了。經濟出來迎接西門慶,回了話説:「徐四家銀子,後日先送二百五十兩來,餘者出月交還。」西門慶罵了幾句,酒帶半酣,也不到後邊,逕往金蓮房裏來。正是:自有内事迎郎意,何怕明朝花不開。

畢竟未知後來何如,且聽下囬分解。

