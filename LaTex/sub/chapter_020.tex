\includepdf[pages={39,40},fitpaper=false]{tst.pdf}
\chapter*{第二十囬 \\孟玉樓義勸吴月娘 西門慶大鬧麗春院}
\addcontentsline{toc}{chapter}{第二十囬 孟玉樓義勸吴月娘 西門慶大鬧麗春院}
\markboth{{\titlename}卷之二}{第二十囬 孟玉樓義勸吴月娘 西門慶大鬧麗春院}


\begin{myquote}
在世為人保七旬,何勞日夜弄精神?

世事到頭終有悔,浮華過眼恐非眞。

貧窮富貴天之命,得失榮華隙裏塵。

不如且放開懷楽,莫使蒼然兩鬢侵。
\end{myquote}

話説西門慶在房中,被李瓶兒幾句柔情軟話,感觸的囬嗔作喜,拉他起來,穿上衣裳,兩個相摟相抱,極盡綢繆。一面令春梅進房放桌兒,往後邊取酒去。

且説金蓮和孟玉樓,従西門慶進他房中去,站在角門首打聽消息。他這邊門又閉着,止是春梅一人在院子裏伺候。金蓮拉玉樓兩個打門縫兒望裏張覷,只見房中掌着燈燭,裏邊説話,卻聽不見。金蓮道:「俺不如春梅賊小肉兒,他倒聽得伶俐。」那春梅便在窗下潛聽。一囬春梅走過來,金蓮悄問他房中怎的動靜,這春梅聽了,便隔門告訴與二人説:「俺爹怎的敎他脱衣裳跪着,他不脱。爹惱了,抽了他幾馬鞭子。」金蓮問道:「打了他,他脱了不曾?」春梅道:「他見爹惱了,纔慌了,就脱了衣裳,跪在地平上。爹如今問他話哩!」玉樓恐怕西門慶聽見,便道:「五姐,咱過那邊去罷。」拉金蓮來西角門首站立。那時八月二十頭,月色纔上來。站在黑影裏,金蓮吃瓜子兒,兩個一處説話,等着春梅出來問他話。潘金蓮便向玉樓道:「我的姐姐,説好食菓子,一心只要來這裏。頭兒沒動,下馬威討了這幾下在身上!俺這個好不順臉的貨兒,你若顺他順兒,他倒罷了;屬扭股兒糖的,你扭扭兒也是錢,不扭也是錢。想着先前,乞小婦奴才和那一行院壓枉造舌,我陪下十二分小心,還乞他奈何的我那等哭哩。姐姐,你來了幾時,還不知他性格哩!」

二人正説話之間,少頃只聽開的角門響,春梅出來,一直徑往後邊走。不防他娘站在黑影處叫他,問道:「小肉兒,那去?」那春梅笑着只顧走。那金蓮道:「怪小肉兒,你過來,我問你話。慌走怎的?」那春梅方纔立住了脚,方説如此這般,「他哭着對俺爹説了許多説話哩。爹喜歡抱起他來,令他穿上衣裳,教我放了桌兒,如今往後邊取酒去。」金蓮聽了,便向玉樓説道:「賊沒廉耻的貨!頭裏那等雷聲大雨點小,打哩亂哩。及到其間,也不怎麽的。我猜,也沒的想,管情取了酒來,敎他遞。賊小肉兒,沒他房裏丫頭,你替他取酒去?到後邊,又叫雪娥那小婦奴才ず聲浪顙,我又聽不上。」春梅道:「爹使我,管我腿事!」於是笑嘻嘻去了。金蓮道:「俺的小肉兒,正經使着他,死了一般懶得動彈。不知怎的,聽見幹貓兒頭差事,鑽頭覓縫幹辦了要去,去的那快!現他房裏兩個丫頭,你替他走,管你腿事!賣蘿蔔的跟着鹽擔子走——好個閒嘈心的小肉兒!」玉樓道:「可不是的,俺大丫頭蘭香,我正經使他做活兒,他像大石,直不動;他爹使他行鬼頭兒,聽人的話兒你看他走的那快!」

正説着,只見玉簫自後邊驀地走來,便道:「三娘還在這裏?我來接你來了。」玉樓道:「怪狗肉,唬我一跳!」因問:「你娘知道你來不曾?」玉簫道:「我打發娘睡下這一日了,我來前邊瞧瞧,剛纔看見春梅後邊要酒菓去了。」因問:「俺爹到他屋裏,怎樣個動靜兒?」金蓮接過來道:「進他屋裏去,尖頭醜婦磞到毛司牆上——齊頭故事。」玉簫又問玉樓,玉樓便一一告他説。玉簫道:「三娘,眞個敎他脱了衣裳跪着,打了他五馬鞭子來?」玉樓道:「你爹因他不跪,纔打他。」玉簫道:「帶着衣服打來,去了衣裳打來?虧他那瑩白的皮肉兒上怎麽挨得!」玉樓笑道:「怪小狗肉兒!你倒替古人耽憂!」正説着,只見春梅和小玉取了酒菜來。春梅拿着酒,小玉拿着方盒,逕往李瓶兒那邊去。金蓮道:「賊小肉兒,不知怎的,聽見幹恁個勾當兒,雲端裏老鼠——天生的耗。」吩咐:「快送了來,敎他家丫頭伺候去。你不要管他,我要使你哩!」那春梅笑嘻嘻,同小玉進去了。一面把酒菜擺在桌上,這春梅和小玉就出來了,只是迎春綉春在房答應。玉樓金蓮問了他話。玉簫道:「三娘,咱後邊去罷。」二人一路去了。金蓮敎春梅関上角門,歸進房來,獨自宿歇,不在話下。正是:可惜團圝今夜月,清光咫尺别人圓。

不説金蓮獨宿,單表西門慶與李瓶兒兩個,相憐相愛,飲酒説話到半夜,方纔被伸翡翠,枕設鴛鴦,上牀就寢。燈光掩映,不啻鏡中之鸞鳳和鳴;香氣薰籠,好似花間之蝴蝶對舞。正是:今宵賸把銀缸照,祇恐相逢是夢中。有詞為證:

\begin{myquote}
淡畫眉兒斜插梳,不忻拈弄倩工夫。雲窗霧閣深深許,蕙性蘭心款款呼。

相憐愛,倩人扶,神僊標格世間無。従今罷却相思調,美滿恩情錦不如。
\end{myquote}

兩個睡到次日飯時,李瓶兒恰待起來臨鏡梳頭。只見迎春後邊拿將來四小碟甜醬瓜茄,細巧菜蔬,一甌炖爛鴿子鶵兒,一甌黄韮乳餅,并醋燒白菜,一碟火燻肉,一碟紅糟鰣魚,兩銀鑲甌兒白生生軟香稻粳米飯兒,兩雙牙筯。婦人先漱了口,陪西門慶吃上半盞兒,就敎迎春:「将昨日剩的銀壺裏金華酒篩來。」拿甌子陪着西門慶每人吃了兩甌子,方纔洗臉梳粧。一面開箱子,打點細軟首飾衣服,與西門慶過目。拿出一百顆西洋珠子與西門慶看,原是昔日梁中書家帶來之物。又拿出一件金鑲鴉青帽頂子,説是過世老公公的。起下來上等子秤,四錢八分重。李瓶兒敎西門慶拿與銀匠,替他做一對墜子。又拿出一頂金絲䯼髻,重九兩。因問西門慶:「上房他大娘衆人,有這䯼髻沒有?」西門慶道:「他們銀絲䯼髻倒有兩三頂,只沒編這金䯼髻。」婦人道:「我不好带出來的。你替我拿到銀匠家毀了,打一件金九鳳鈿根兒,每個鳳嘴啣一掛珠兒;剩下的再替我打一件,照依他大娘正面戴的金鑲玉觀音滿池嬌分心。」西門慶收了,一面梳頭洗臉,穿了衣服出門。李瓶兒吩咐:「那邊房子裏沒人,你好歹過去看看,委付個人兒看守,替了小廝天福兒來家使喚。那老馮老行貨子,啻啻磕磕的,獨自在那裏,我又不放心。」西門慶道:「你吩咐,我知道了。」袖着䯼髻和帽頂子出門,一直往外走。

不防金蓮鬅着頭,還未梳洗,站在東角門首,叫道:「哥,你往那去?這咱纔出來,看雀兒撞眼兒!」那西門慶道:「我有勾當去。」婦人道:「怪行貨子,你還來,慌走怎的?我和你説話。」那西門慶見他叫的緊,只得囬來。被婦人引到房中,婦人便坐在椅子上,把他兩隻手拉着,説道:「我不好罵出來的,怪火燎腿三寸貨!那個拿長鍋鑊吃了你,慌往外搶的是些甚的?你過來,我且問你。」西門慶道:「罷麽,小淫婦兒,只顧問甚麽!我有勾當哩,等我囬來説。」説着,往外走。婦人摸見他袖子裏重重的,道:「是甚麽?拿出來我瞧瞧。」西門慶道:「是我的銀子包。」婦人不信。伸手進去袖子裏就掏,掏出一頂金絲䯼髻來,説道:「這是他的䯼髻,你拿那去?」西門慶道:「他問我,知你們沒有這䯼髻,到銀匠家替他毀了,打兩件頭面戴。」金蓮問道:「這䯼髻多少重?他要打甚麽?」西門慶道:「這䯼髻重九兩,他要打一件九鳳鈿兒,一件照依上房戴的正面那一件,金鑲玉觀音滿池嬌分心。」金蓮道:「一件九鳳鈿兒,滿破使個三兩五六錢金子夠了;大姐姐那件分心,我秤只重一兩六錢;把剩下的,好歹你替我照依他也打一件九鳳鈿兒。」西門慶道:「滿池嬌他要搗實枝梗的。」金蓮道:「就是搗實枝梗,使個三兩金子滿篡。綁着鬼還落他二三兩金子,夠打個鈿兒了。」西門慶笑駡道:「你這小淫婦兒!單管愛小便益兒,隨處也掐個尖兒。」金蓮道:「我兒,娘説的話你好歹記着。你不替我打將來,我和你答話!」那西門慶袖了䯼髻,笑着出門。金蓮戲道:「哥兒,你幹上了。」西門慶道:「我怎的幹上了?」金蓮道:「你既不幹,昨日那等雷聲大雨點小,要打着敎他上吊。今日拿出一頂䯼髻來,使的你狗油嘴鬼推磨,不怕你不走!」西門慶笑道:「這小淫婦兒,單只管胡説!」説着往外去了。

却説吳月娘和孟玉樓李嬌兒在房中坐的,忽聽見外邊小廝一片聲尋來旺兒,尋不着。只見平安來掀簾子,月娘便問:「尋他做甚麽?」平安道:「爹緊等着哩。」月娘半日纔説:「我使了他有勾當去了。」原來月娘早晨吩咐下他,往王姑子庵裏送香油白米去了。平安道:「小的囬爹,只説娘使了他有勾當去了。」月娘罵道:「怪奴才!隨你怎麽回去!」平安唬的不敢言語一聲兒,往外走了。月娘便向玉樓衆人説道:「我開口,又説我多管;不言語,我又憋的慌。一個人也拉剌將來了,那房子賣掉了就是了。平白扯淡,搖鈴打鼓的看守甚麽!左右有他家馮媽媽子在那裏,再派一個沒老婆的小廝,晚夕同在那裏上宿睡就是了,怕走了那房子也怎的?作養娘抱,巴巴叫來旺兩口子去!自他媳婦子七病八病,一時病倒了在那裏,上床誰扶持他?」玉樓便道:「姐姐在上,不該我説。你是個一家之主,不爭你與他爹兩個不説話,就是俺們不好張主的,下邊孩子們也沒投奔。他爹這兩日,隔二偏三的,也甚是没意思。看姐姐恁的,依俺們一句話兒,與他爹笑開了罷。」月娘道:「孟三姐,你休要起這個意。我又不曾和他兩個嚷鬧,他平白的使性兒。那怕他使的那臉も,休想我正眼看他一眼兒!他背地對人駡我不賢良的淫婦,我怎的不賢良你來?如今聳六七個在屋裏,纔知道我不賢良!自古道:順情説好話,戅直惹人嫌。我當初大説攔你,也只為你來。你既收了他許多東西,又買了他房子,今日又圖謀他老婆,就着官兒也看喬了;何况他孝服不滿,你不好娶他的。誰知道人在背地裏把圈套做的成成的,每日行茶過水,只瞞我一個兒,把我合在缸底下。今日也推在院裏歇,明日也推在院裏歇,誰想他只當把個人兒『歇』了家裏來。端的好個在院裏歇!他只吃人在他跟前那等花麗狐哨,喬龍畫虎的兩面刀哄他,就是千好萬好了。似俺們這等依老實,苦口良言,着他理你理兒!你到如今反被為仇。正是前車倒了千千輛,後車到了亦如然;分明指與平川路,錯把忠言當惡言!你不理我,我想求你?一日不少我三頓飯。我只當沒漢子,守寡在這屋裏!隨我去,你們不要管他。」幾句話,説的玉樓衆人訕訕的。

良久,只見李瓶兒梳妝打扮,上穿大紅遍地金對衿羅衫兒,翠藍拖泥粧花羅裙,迎春抱着銀湯瓶,綉春拿着茶盒,走來上房,與月娘衆人遞茶。月娘叫小玉安放座兒與他坐。落後孫雪娥也來到,都遞了茶,一處坐的。潘金蓮嘴快,便呌道:「李大姐,你過來,與大姐下個禮兒。實和你説了罷,大姐姐和他爹那些時兩個不説話,因為你來!俺們剛纔替你勸了恁一日。你改日安排一席酒兒,央及央及大姐姐,敎他兩個老公婆笑開了罷。」李瓶兒道:「姐姐吩咐,奴知道。」於是向月娘面前,花枝招展,綉帶飄飄,插燭也似磕了四個頭。月娘道:「李大姐,他哄你哩。」又道:「五姐,你們不要來攛掇。我已是賭下誓,就是一百年也不和他在一答兒哩!」以此衆人再不敢復言。

金蓮在傍拿把抿子與李瓶兒抿頭,見他頭上戴着一副金玲瓏草蟲兒頭面,並金纍絲松竹梅歲寒三友梳背兒,因説道:「李大姐,你不該打這碎草蟲頭面,只是有些抓住了頭髮。不如大姐姐頭上戴的這金觀音滿池嬌,是搗實枝梗的好。」這李瓶兒老實,就説道:「奴也照樣兒要教銀匠打恁一件哩!」落後小玉玉簫來跟前遞茶,都亂戲他。先是玉簫問道:「六娘,你家老公公,當初在皇城内那衙門來?」李瓶兒道:「先在惜薪司掌廠,御前班直,後陞廣南鎭守。」玉簫笑道:「嗔道你老人家昨日挨的好柴!」小玉又道:「去年城外澇鄉,許多里長老人好不尋你,敎你往東京去。」婦人不知道甚麽,説道:「他尋我怎的?」小玉笑道:「他説你老人家會告的好水災!」玉簫又道:「你老人家鄉裏媽媽拜千佛,昨日磕頭磕夠了。」小玉又説道:」朝廷昨日差了四個夜不收,請你老人家往口外和番,端的有這話麽?」李瓶兒道:「我不知道。」小玉笑道:「説你老人家會叫的好達達!」把玉樓金蓮笑的不了。月娘便道:「怪臭肉們,幹你那營生去,只顧奚落他怎的?」於是把個李瓶兒羞的臉上一塊紅,一塊白,站又站不得,坐又坐不住,半日囬房去了。

良久,西門慶進房來,囬他顧銀匠家打造生活。就與他計較,明日發柬,二十五日請官客吃會親酒,少不的拿帖兒請請花大哥。李瓶兒道:「他娘子三日來,再三説了。也罷,你請他請罷。」李瓶兒又説:「那邊房子左右有老馮看守,你這裏再叫一個,和天福兒輪着晚夕上宿就是,不消教旺官去罷。上房姐姐説,他媳婦兒有病,去不的。」西門慶道:「我不知道。」即叫平安近前吩咐:「你和天福兒兩個輪,一遞一日獅子街房子裏上宿。」不在言表。

話休饒舌,不覺到二十五日,西門慶家中吃會親酒,插花筵席,四個唱的,一起雜耍步戲。頭一席,花大舅吴大舅;第二席是吴二舅沈姨夫;第三席應伯爵謝希大;第四席祝日念孫天化;第五席常時節吴典恩;第六席雲離守白來創;西門慶主位,其餘傅自新、賁地傳、女婿陳經濟,兩邊列位。先是李桂姐、吴銀兒、董玉僊、韓金釧兒,従晌午時分,坐轎子就來了,在月娘上房裏坐的。官客在新蓋捲棚内坐的吃茶,然後到齊了,大廳上坐。席上都有桌面,某人居上,某人居下。先吃小割海青捲兒,八寳攢湯。頭一道割燒鵝大下飯。楽人撮弄雜耍囬數,就是笑楽院本。下去,李銘吴惠兩個小優上來彈唱,間着清吹。下去,四個唱的出來,筵外遞酒。

應伯爵在席上先開言,説道:「今日哥的喜酒,是兄弟不當斗膽,請新嫂子出來拜見拜見,足見親厚之情。俺們不打緊,花大尊親並二位老舅沈姨丈在上,今日為何來?」西門慶道:「小妾醜陋,不堪拜見,免了罷。」謝希大道:「哥,你這話難説。當初已言在先,不為嫂子,俺們怎麽兒來?何况這個嫂子,現有我尊親花大哥在上,先做友,後做親,又不同别人。請出來見見,怕怎的?」那西門慶笑,不動身。應伯爵道:「哥,你不要笑。俺們都拿着拜見錢在這裏,不白敎他出來見。」西門慶道:「你這狗才,單管胡説。」乞他再三逼迫不過,叫過玳安來,敎他後邊説去。半日,玳安出來囬説:「六娘道,免了罷。」應伯爵道:「就是你這小狗骨朵兒的鬼!你幾時往後邊去,就來哄我?賭個兒眞個,我就後邊去了!」玳安道:「小的莫不哄應二爹,二爹進去問不是?」伯爵道:「你量我不敢進去?左右花園中熟徑,好不好,我走進去,連你那幾位娘都拉了出來。」玳安道:「俺家那大猱獅狗好不利害。倒沒的把應二爹下半截撕下來。」伯爵故意下席,趕着玳安踢兩脚,笑道:「好小狗骨秃兒!你傷的我好!趂早與我後邊請去。請不將來,打二十欄杆。」把衆人四個唱的都笑了。

那玳安到下邊,又走來立着,把眼看着他爹不動身。西門慶無法可處,只淂叫過玳安,近前吩咐:「對你六娘説,收拾了出來見見罷。」那玳安去了半日出來,復請了西門慶進去。然後纔把脚下人趕出去,関上儀門。四個唱的,都往後邊彈楽器,簇擁婦人上拜。孟玉樓潘金蓮百方攛掇,替他抿頭,戴花翠,打發他出來。廳上又早鋪下錦毡綉毯,麝蘭靉靆,絲竹和鳴,四個唱的,導引前行。婦人身穿大紅五彩通袖羅袍兒,下着金枝緑葉沙綠百花裙,腰裏束着碧玉女帶,腕上籠着金壓袖;胸前項牌瓔珞,裙邊環珮玎璫,頭上珠翠堆盈,鬢畔寳釵半卸;紫瑛金環,耳邊低掛;珠子挑鳳,髻上雙插;粉面宜貼翠花鈿,湘裙越顯紅鴛小。正是:恍似嫦娥離月殿,猶如神女到筵前。四個唱的,琵琶箏絃,簇擁婦人,花枝招颭,綉帶飄飄,望上朝拜。慌的衆人都下席來還禮不迭。

却説孟玉樓、潘金蓮、李嬌兒,簇擁着月娘,都在大廳軟壁後聽覷,聽見唱〔喜得功名遂〕,唱到「天之配合一對兒,如鸞似鳳,夫共妻。」,直到「笑吟吟慶喜,高擎着鳳凰盃。象板銀箏間玉笛,列盃盤,水陸排佳會。」,直至「永團圓,世世夫妻」跟前,金蓮向月娘説道:「大姐姐,你聽唱的!小老婆今日不該唱這一套,他做了一對魚水團圓、世世夫妻,把姐姐放到那裏?」那月娘雖故好性兒,聽了這兩句,未免有幾分動意,惱在心中。又見應伯爵謝希大這夥人,見李瓶兒出來上拜,恨不的生出幾個口來誇獎奉承,説道:「我這嫂子,端的寰中少有,蓋世無雙!休説德性溫良,舉止沉重;只這一表人物,普天之下,也尋不出來。那裏有哥這樣大福?俺們今日得見嫂子一面,明日死也淂好處!」因喚玳安兒:「快請你娘囬房裏,只怕勞動着,倒値了多的。」吴月娘衆人聽了,駡「扯淡輕嘴的囚根子」不絶。良久,李瓶兒下來。四個唱的見他手裏有錢,都亂趨捧着他,娘長娘短,替他拾花翠,疊衣服,無所不至。

月娘歸房,甚是悒怏不楽。只見玳安平安接了許多拜錢,也有尺頭、衣服並人情禮,盤子盛着,拿到月娘房裏。月娘正眼也不看,駡道:「賊囚根子!拿送到前頭就是了,平白㧱進我屋裏來做甚麽?」玳安道:「爹吩咐拿到娘房裏來。」月娘敎玉簫接了,掠在牀上去。

不一時,吴大舅吃了第二道湯飯,走進後邊來見月娘。月娘見他哥進房來,連忙花枝招颭,與他哥哥行禮畢,坐下。吴大舅道:「昨日你嫂子在這裏打攪,又多謝姐夫送了桌面去。到家對我説,你與姐夫兩個不説話。我執着要來勸你,不想姐夫今日請。——姐姐,你若這等,把你従前一塲好都沒了。自古癡人畏婦,賢女畏夫。三従四德,乃婦道之常。今後姐姐,他行的事,你休要攔他。料姐夫他也不肯差了,落得你還做好好先生,纔顯出你賢德來。」月娘道:「早賢德好來,不敎人這般憎嫌。他有了他富貴的姐姐,把俺這窮官兒家丫頭只當亡故了的算帳。你也不要管他,左右是我,隨他把我怎麽的罷!賊強人,従幾時這等變心來?」説着,月娘就哭了。吴大舅道:「姐姐,你這個就差了。你我不是那等人家,快休如此。你兩口兒好好的,俺們走來也有光輝些!」勸月娘一囬。小玉拿了茶來,吃畢茶,吩咐放桌兒,留吴大舅房裏吃酒。吴大舅道:「姐姐沒的説,我適纔席上酒飯都吃的飽飽的,來看看姐姐。」坐了一囬,只見前邊使小廝來請,吴大舅便作辭月娘出來。當下衆人吃至掌燈以後,就起身散了。那日四個唱的,李瓶兒每人都是一方銷金汗巾兒,五錢銀子,歡喜回家。

自此西門慶一連在瓶兒房裏歇了數夜。別人都罷了,只是潘金蓮惱的了不的,背地唆調吴月娘與李瓶兒合氣。對着李瓶兒,又説月娘許多不是,説月娘容不的人。李瓶兒尚不知墮他計中,每以姐姐呼之,與他親厚尤密。正是:逢人且説三分話,未可全抛一片心。

西門慶自従娶李瓶兒過門,又兼得了兩三場横財,家道營盛,外莊内宅,煥然一新。米麥成倉,騾馬成羣,奴僕成行。把李瓶兒帶來小廝天福兒,改名琴童。又買了兩個小廝,一名來安兒,一名棋童兒。把金蓮房中春梅,上房玉簫,李瓶兒房中迎春,玉樓房中蘭香,一般兒四個丫鬟,衣服首飾粧束出來,在前廳西廂房,敎李嬌兒兄弟楽工李銘來家,敎演習學彈唱。春梅琵琶,玉簫學箏,迎春學絃子,蘭香學胡琴。每日三茶六飯,管待李銘,一月與他五兩銀子。又打開門面二間,兌出二千兩銀子來,委傅夥計賁地傳開解當舖。女婿陳經濟只掌管鑰匙,出入尋討,不拘薬材當物。賁地傳只是寫帳目,秤發貨物。傅夥計便督理生薬、解當兩個舖子,看銀色,做買賣。潘金蓮這樓上,堆放生薬;李瓶兒那邊樓上,鑲成架子,擱解當庫衣服、首飾、古董、書畫、玩好之物。一日也嘗當許多銀子出門。

陳經濟每日起早睡遲,带着鑰匙,同夥計查點出入銀錢,收放寫算皆精。西門慶見了,喜歡的了不的。一日,在前廳與他同桌兒吃飯,説道:「姐夫,你在我家這等會做買賣,就是你父親在東京知道,他也心安,我也得托了。常言道:有兒靠兒,無兒靠婿。姐夫是何人?我家姐姐是何人?我若久後没出,這份兒家當,都是你兩口兒的。」那陳經濟説道:「兒子不幸,家遭官事,父母遠離,投在爹娘這裏。蒙爹娘抬擧,莫大之恩,生死難報。只是兒子年幼,不知好歹,望爹娘躭待便了,豈敢非望!」這西門慶聽見他會説話兒,聰明乖覺,越發滿心歡喜。但凡家中大小事務,出入書柬禮帖,都敎他寫;但凡人客到,必請他席側相陪。吃茶吃飯,一時也少不的他。誰知這小夥兒,綿裏之針,肉裏之刺,常向綉簾窺賈玉,每従綺閣竊韓香。有詩為證:

\begin{myquote}
東牀嬌婿實堪憐,况遇青春美少年。

待客每令席側坐,尋常只在便門穿。

家前院後明嘲戲,呆裏撒乖暗做奸。

空在人前稱半子,従來骨肉不牽連。
\end{myquote}

光陰似箭,日月如梭。才見中秋賞月,忽然菊綻東籬。空中寒鴈向南飛,不覺雪花滿地。一日,十一月下旬天氣,西門慶在友人常時節家會茶飲酒,散的早,未等掌燈時分就起身,同應伯爵謝希大祝日念三個並馬而行。剛出了常時節門,只見天上彤雲密布,又早紛紛揚揚飄下一天雪花兒來。應伯爵便説道:「哥,咱這時候就家去,家裏也不收。我們知你許久不曾進裏邊看看桂姐,今日趂着天氣落雪,只當孟浩然踏雪尋梅,咱望他望去。」祝日念道:「應二哥説的是。你每月風雨不阻,出二十兩銀子包錢包着他,你不去,落得他自在。」西門慶於是吃三人你一言我一句,説的把馬逕往東街勾攔那條路來了。來到了李桂姐家,已是天氣將晚。只見客位裏掌起燈燭,丫頭正掃地不迭。老媽并李桂卿出來見畢,上面列四張校椅,四人坐下。老虔婆便道:「前者桂姐在宅裏來晚了,多有打攪;又多謝六娘賞汗巾、花翠。」西門慶道:「那日空過他。我恐怕晚了他們,客人散了就打發他來了。」説着,虔婆一面看茶吃了,丫鬟就安放桌兒,設放案酒。西門慶道:「怎麽桂姐不見?」虔婆道:「桂姐連日在家伺候姐夫,不見姐夫來到。不想今日他五姨媽生日,拿轎子接了,與他五姨媽做生日去了。」

看官聽説:原來世上,惟有和尚道士并唱的人家這三行人,不見錢眼不開;嫌貧取富,不説謊調詖也成不的。原來李桂姐也不曾往五姨媽家做生日。近日見西門慶不來,又接了杭州販紬絹的丁相公兒子丁二官人,號丁雙橋;販了千兩銀子紬絹,在客店裏安下,瞞着他父親來院中敲嫖。頭上拿十兩銀子、兩套杭州重絹衣服請李桂姐,一連歇了兩夜。適纔正和桂姐在房中吃酒,不想西門慶到,老虔婆敎桂姐連忙陪他後邊第三層一間僻淨小房那裏坐去了。當下西門慶聽信虔婆之言,便道:「既是桂姐不在,老媽快看酒來,俺們慢慢等他。」這老虔婆在下邊一力攛掇,酒餚菜蔬齊上,湏臾,堆滿桌席。李桂卿不免箏排雁柱,歌按新腔,衆人席上猜枚行令。正飲酒在熱鬧處,不防西門慶往後邊更衣去。也是合當有事,忽聽東耳房有人笑聲。西門慶更畢衣,走到窗下偷眼觀覷,正見李桂兒在房内陪着一個戴方巾的蠻子飲酒。由不的心頭火起,走到前邊,一手把吃酒桌子掀倒,碟兒盞兒打的粉碎。喝令跟馬的平安、玳安、畫童、琴童,四個小廝上來,不由分説,把李家門窗户壁牀帳都打碎了。應伯爵、謝希大、祝日念,向前拉勸不住。西門慶口口聲聲只要採出蠻囚來,和粉頭一條䋲子墩鎖在門房内。那丁二官兒又是個小膽之人,外邊嚷鬧起來,唬的藏在裏間牀底下,只呌:「桂姐救命!」桂姐道:「呸!好不好,還有媽哩!不妨事。隨他發作,怎的叫嚷,你休要出來。」且説老虔婆兒見西門慶打的不像模樣,不慌不忙拄拐而出,説了幾句閑話。西門慶心中越怒起來,指着罵道,有〔滿庭芳〕為證:

\begin{myquote}
「虔婆你不良:迎新送舊,靠色為娼。巧言詞將咱誑,説短論長。我在你家使夠,有黄金千兩,怎禁賣狗懸羊?我罵你句眞伎倆,媚人狐黨,衠一片假心腸!」
\end{myquote}

虔婆亦答道:

\begin{myquote}
「官人聽知:你若不來,我接下別的。一家兒指望他為活計。吃飯穿衣,全憑他供柴糴米。沒來由暴叫如雷,你怪俺全無意。不思量自己,不是你憑媒娶的妻!」
\end{myquote}

西門慶聽了,心中越怒,險些不曾把李老媽媽打起來。多虧了應伯爵、謝希大、祝日念,三個死勸活喇喇,拉開了手。西門慶大鬧了一場,賭誓再不踏他門來,大雪裏上馬囬家。正是:

\begin{myquote}
宿盡閑花萬萬千,不如歸去伴妻眠。

雖然枕上無情趣,睡到天明不要錢。
\end{myquote}

又曰:

\begin{myquote}
女不織兮男不耕,全憑賣俏做營生。

任君斗量并車載,難滿虔婆無底坑!
\end{myquote}

又曰:

\begin{myquote}
假意虚脾恰似眞,花言巧語弄精神。

幾多伶俐遭他陷,死後應知拔舌根。
\end{myquote}

畢竟未知後來何如,且聽下回分解。

\part*{夢梅館校本《金瓶梅詞話》卷之三}
\addcontentsline{toc}{part}{夢梅館校本《金瓶梅詞話》卷之三}

