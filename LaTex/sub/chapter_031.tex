\includepdf[pages={61,62},fitpaper=false]{tst.pdf}
\chapter*{第三十一囬 \\琴童藏壺覷玉簫 西門慶開宴喫喜酒}
\addcontentsline{toc}{chapter}{第三十一囬 琴童藏壺覷玉簫 西門慶開宴喫喜酒}
\markboth{{\titlename}卷之四}{第三十一囬 琴童藏壺覷玉簫 西門慶開宴喫喜酒}


\begin{myquote}
家富自然身貴,逢人必讓居先。貧寒敢仰上官憐?彼此都看錢面。\\
婚嫁專尋勢要,通財邀結豪英。不知興廢在心田,只靠眼前知見。
\end{myquote}

話説西門慶次日使來保提邢所本縣下文書,一面使人做官帽。又喚趙裁率領四五個裁縫,在家來裁剪尺頭,趲造衣服。又呌了許多匠人,釘了七八條都是四指寬玲瓏雲母、犀角、鶴頂紅、玳瑁、魚骨香帶。

不説西門慶家中熱亂。且説吴典恩那日走到應伯爵家,把做馹丞之事,再三央及伯爵,要問西門慶借銀子上下使用,許伯爵:「借銀子出來,把十兩銀子買禮物謝老兄。」説着,跪在地下。慌的伯爵一手拉起,説道:「此是成人之羙。大官人照顧你東京走了這遭,㩦帶你得此前程,也不是尋常小可。」因問:「你如今所用多少够了?」吴典恩道:「不瞞老兄説,我家活人家,一文錢也沒有。到明日上任,參官贄見之禮,連擺酒並治衣類鞍馬,少説也得七八十兩銀子,那裏區處?如今我寫了一紙文書在此,也沒敢下數兒。望老兄好歹扶持小人,在旁加羙言。事成恩有重報,不敢有忘。」伯爵看了文書,因説:「吴二哥,你説借出這七八十兩銀子來,也不夠使。依我,取筆來寫上一百兩,恒是看我面不要你利錢,你且得手使了,到明日做上官兒,慢慢陸續還他也是不遲。常言俗語説得好,借米下得鍋,討米下不的鍋。哄了一日是兩晌,何況你又在他家曾做過買賣,他那裏把你這幾兩銀子放在心上?」那吴典恩聽了,謝了又謝。於是把文書上填寫了一百兩之數。

當下兩個喫了茶,一同起身,來到西門慶門首。伯爵問守門平安兒:「你爹起來了不曾?」平安兒道:「俺爹起來了,在捲棚看着匠人釘帶哩。待小的禀去。」於是一直走來報西門慶説:「應二爹和吳二叔來了。」西門慶道:「請進。」不一時,二人進入裏面,見有許多裁縫匠人七手八脚做生活。西門慶帶着小帽錦衣,和陳經濟在穿廊下看着寫見官手本謁帖。見二人,作揖讓坐。伯爵問:「哥的手本劄付,下了不曾?」西門慶道:「今早使小价往提刑府下劄付去了。今有手本還未往東平府並本縣下去。」説畢,小廝畫童兒㧱上茶來。喫畢茶,那應伯爵並不題吳主管之事,走下來且看匠人釘帶。西門慶見他㧱起带來看,一徑賣弄,説道:「你看我尋的這幾條帶如何?」伯爵極口稱讚誇獎説道:「虧哥那裏尋的,都是一條賽一條的好帶!難得這般寬大。别的倒也罷了,只這條犀角帶並鶴頂紅,就是滿京城㧱着銀子也尋不出來。不是面獎,就是東京衛主老爺玉帶金帶空有,也沒這條犀角帶。這是水犀角,不是旱犀角。旱犀角不值錢。水犀角號作通天犀,你不信,取一碗水,把犀角安放在水内,分水為兩處,此為無價之寳。又夜間燃火照千里,火光通宵不滅。」因問:「哥,你使了多少銀子尋的?」西門慶道:「你們試估估價值。」伯爵道:「這個有甚行款,我們怎麽估得出來!」西門慶道:「我對你説了罷,此帶是大街上王招宣府裏的帶。昨日晚間,一個人聽見我這裏要帶,巴巴來對我説。我着賁四㧱了七十兩銀子,再三囬了他這條帶來。他家還張致不肯,定要一百兩。」伯爵道:「且難得這等寬樣好看。哥,你到明日繫出去,甚是霍綽。就是你同僚間見了也愛。」於是誇羙了一囬,坐下。

西門慶便向吳主管問道:「你的文書下了不曾?」伯爵道:「吳二哥文書還未下哩。今日巴巴的他央我來激煩你。雖然蒙你照顧他往東京押生辰擔,蒙太師與了他這個前程,就是你擡擧他一般,也是他各人造化,説不的。一品至九品都是朝廷臣子,況他如今家中無錢。他告我説,就是如今上任見官擺酒並治衣服之類,也要許多銀子使。一客不煩二主,那處活變去?沒奈何,哥看我面,有銀子借與幾兩扶持他,賙濟了這些事兒。他到明日做上官,就啣環結草也不敢忘了哥大恩人。休説他舊是喒府中夥計,在哥門下出入,就是従前已後外京外府官吏,哥不知拔濟了多少。不然,你教他那裏區䖏去?」因説道:「吴二哥,你㧱出那符兒來與你大官人瞧。」這吴典恩連忙向懷中取出,遞與西門慶觀看。見上面借一百兩銀子,中人就是應伯爵,每月行利五分。西門慶取筆把利錢抹了,説道:「旣是應二哥作保,你明日只還我一百兩本錢就是了。我料你上下也得這些銀子攪纏。」於是把文書收了。

纔待後邊取銀子去,忽有提刑所夏提刑㧱帖兒差了一名寫字的㧱手本,三班送了十二名排軍來答應,就問討上任日期,討問字號,——衙門同僚具公禮來賀。西門慶教陰陽徐先生擇定七月初二日青龍、金匱黄道,宜辰時到任,㧱拜帖兒囬夏提刑,賞了寫字的五錢銀子,俱不必細說。

應伯爵和吴典恩正在捲棚内坐的,只見陳經濟㧱着一百兩銀子出來,西門慶交與吴主管説:「吴二哥,你明日只還我本錢便了。」那吴典恩一面接了銀在手,叩頭謝了。西門慶道:「我不留你坐罷,你家中執你的事去。留下應二哥,我還和你説句話兒。」那吴典恩㧱着銀子歡喜出門。看官聽説:後來西門慶死了,家中時敗勢衰,吴月娘守寡,把小玉配與玳安為妻。家中平安兒小廝又偸盜出解當庫頭面,在南瓦子裏宿娼。被吳馹丞㧱住,痛刑拶打,教他指攀月娘與玳安有奸,要羅織月娘出官,恩將讐報。此係後事,表過不題。正是:不結子花休要種,無義之人不可交!

那時賁四往東平府並本縣下了手本來囬話,西門慶留他和應伯爵陪陰陽徐先生擺飯。正喫着飯,只見西門慶舅子吳大舅來拜望。徐先生就起身。良久,應伯爵也作辭,出門來到吴主管家。吳典恩又早封下十兩保頭錢,雙手遞與伯爵,磕下頭去。伯爵道:「若不是我那等取巧説着,他會勝不肯借與你。這一百兩銀子與你,隨你上下還使不了這些,還落一半,家中盤纏。」那吴典恩酬謝了伯爵,治辦官帶衣類,擇日見官上任不題。

那時本縣正堂李知縣,會了四衙同僚,差人送羊酒賀禮來。又㧱帖兒送了一名小郎來答應,年方一十六歲,本貫蘇州府常熟縣人,喚名小張松。原是縣中門子出身,生的清俊,面如傅粉,齒白唇紅。又識字會寫,善能歌唱南曲。穿着青綃直裰,京鞋淨襪。西門慶一見小郎伶俐,滿心歡喜。就㧱拜帖囬覆李知縣,留下他在家答應,改換了名字,呌做書童兒。與他做了一身衣裳,新靴新帽。不教他跟馬,教他專管書房收禮帖,㧱花園門鑰匙。祝日念又擧保了一個十四歲小廝來答應,亦改名棋童,每日派定和琴童兒兩個背書袋、夾拜帖匣,跟馬。

到了上任日期,在衙門中擺大酒席桌面,出票拘集三院楽工俳色長承應,吹打彈唱,後堂飲酒,日暮時分散歸。每日騎着大白馬,頭戴烏紗,身穿五彩洒線猱頭獅子補子員領,四指大寬萌金茄楠香帶,粉底皂靴,排軍喝道,張打着大黑扇,前呼後擁,何止十數人跟隨,在街上搖擺。上任回來,先拜本府縣帥府都監並清河左右衛同僚官,然後親朋鄰舍,何等榮耀施為!家中收禮接帖子,一日不断。正是:

\begin{myquote}
白馬血纓彩色新,不來親者強來親。

時來頑鐵皆光彩,運去良金不發明。
\end{myquote}

西門慶自従到任以來,每日坐提刑院衙門中陞廳畫卯,問理公事。光陰迅速,不覺李瓶兒坐褥一月將滿。吴大妗子、二妗子、楊姑娘、潘姥姥、吴大姨、喬大户娘子,許多親鄰堂客女眷,都送禮來,與官哥兒做彌月。院中李桂姐、吴銀兒,見西門慶做了提刑所千户,家中又生了子,亦送大禮,坐轎子來慶賀。西門慶那日在前邊大廳上擺設筵席,請堂客飲酒。春梅迎春玉簫蘭香都打扮起來,在席前與月娘斟酒執壶,侍堂客飲酒。

原來西門慶每日従衙門中來,只到外邊廳上,就脫了衣服,教書童疊了,安在書房中,止戴着冠帽進後邊去。到次日起身,旋使丫鬟來書房中取。新近收拾大廳西廂房一間做書房,内安牀几、桌椅、屏幃、筆硯、琴書之類。書童兒晚夕只在牀脚踏板上搭着鋪睡,未曾西門慶出來,就收拾頭腦,打掃書房乾淨,伺候答應。或是在那房裏歇,早晨就使出那房裏丫鬟來前邊取衣服。取來取去,不想這小郎本是門子出身,生的伶俐乖覺又清俊,二者又與各房丫頭打牙犯嘴慣熟,於是暗和上房裏玉簫兩個嘲戲上了。

那日也是合當有事。這小郎正起來,在書房牀地平上插着棒兒香,正在窗户臺上擱着鏡兒梳頭,㧱紅䋲扎頭髮。不料上房玉簫推開門進來,看見説道:「好賊囚,你這喒還來描眉畫眼兒的,爹喫了粥便出來。」書童也不理,只顧扎包髻兒。那玉簫道:「爹的衣服疊了,在那裏放着哩?」書童道:「在牀南頭安放着哩。」玉簫道:「他今日不穿這一套。他吩咐我,教問你要那件玄色匾金補子、絲布圓領、玉色襯衣穿。」書童道:「那衣服在廚櫃裏。我昨日纔收了,今日又要穿他?姐,你自開門取了去。」那玉簫且不㧱衣服,走來跟前看着他扎頭,戲道:「怪賊囚,也像老婆般㧱紅䋲扎着頭兒,梳的鬢這虚籠籠的。」因見他白滚紗漂白布汗掛兒上,繫着一個銀紅紗香袋兒,一個綠紗香袋兒,問他要:「你與我這個銀紅的罷。」書童道:「人家個愛物兒,你就要。」玉簫道:「你小廝家帶不的這銀紅的,只好我帶。」書童道:「早是這個罷了,他要是個漢子兒,你也愛他罷?」被玉簫故意向他肩膊上擰了一把,説道:「賊囚,你夾道賣門神——看出來的好畫兒!」不由分說,把兩個香袋子等不的解,都揪断繫兒放在袖子内。書童道:「你好不尊貴,把人的帶子也揪断。」被玉簫發訕,一拳一把戯打在身上,打的書童急了,説:「姐,你休鬼混我,待我扎上這頭髮着。」玉簫道:「我且問你,沒聽見爹今日往那去?」書童道:「爹今日與縣中三宅華主簿老爹送行,在皇庄薛公公那裏擺酒,來家早。也下午時分。我聽見會下應二叔,今日兑銀子,要買對門喬大户家房子,那裏喫酒罷了。」玉簫道:「等住囬你休往那去了,我來和你説話。」書童道:「我知道。」玉簫於是與他約會下,㧱衣服一直往後邊去了。

少頃,西門慶出來,就呌書童吩咐:「在家,别往那去了。先寫十二個請帖兒,都用大紅紙封套,二十二日請官客喫慶官哥兒酒;教來興兒買辦東西,添廚役茶酒,預備桌面齊整;玳安和兩名排軍送帖兒,呌唱的;留下琴童兒在堂客面前管酒。」吩咐畢,西門慶上馬送行去了。那吴月娘衆姊妹請堂客到齊了,先在捲棚擺茶,然後大廳上屏開孔雀,褥隱芙蓉,上坐。席間叫了四個妓女彈唱。果然西門慶到午後時分來家。家中安排一食盒酒菜,邀了應伯爵和陳經濟,擡了七百兩銀子,往對門喬大户家成房子去了。

堂客正飲酒中間,只見玉簫㧱下一銀執壶酒,並四個梨、一個柑子,逕來廂房中送與書童兒喫。推開門,不想書童兒不在裏面。恐人看見,連壶放下就出來了。可霎作怪,琴童兒正在上邊看酒,冷眼睃見玉簫進書房去,半日出來,只知有書童兒在裏邊,三不知扠進去瞧。不想書童兒外邊去,不曾進來。一壶熱酒和菓子還放在牀底下。這琴童連忙把菓子藏袖裏,將那一壶酒影着身子一直提到李瓶兒房裏。迎春和婦人都在上邊,不曾下來。止有奶子如意兒和綉春在屋裏看哥兒。那琴童進門就問:「姐在那裏?」綉春道:「他在上邊與娘斟酒哩,你問他怎的?」琴童兒道:「我有個好的兒,敎他替我收着。」綉春問他甚麽,他又不㧱出來。正説着,迎春從上邊㧱下一盤子燒鵝肉,一碟玉米面玫瑰菓餡蒸餅兒與奶子喫,看見便道:「賊囚,你在這裏笑甚麽,不在上邊看酒?」那琴童方纔把壶従衣裳底下㧱出來,教迎春:「姐,你與我收了。」迎春道:「此是上邊篩酒的執壶,你平白㧱來做甚麽?」琴童道:「姐,你休管他。此是上房裏玉簫,和書童兒小廝七個八個,偷了這壶酒和些柑子梨,送到書房中與他喫。我趕眼不見,戲了他的來。你只與我好生收着,隨問甚麽人來找尋,休㧱出來。我且拾個白財兒着。」因把梨和柑子掏出來與迎春瞧,説道:「我看篩了酒,今日該我獅子街房子差,我上宿去也。」迎春道:「等住囬找尋壶反亂,你就承當!」琴童道:「我又沒偸他的壶。各人當場者亂,隔壁心寬,管我腿事!」説畢,揚長去了。迎春把壶藏放在裏間桌上不題。

至晚,酒席上人散,查收家伙,少了一把壶。玉簫往書房中尋,那裏得來?再有一把也沒了。問書童,説:「我外邊有事去,不知道。」那玉簫就慌了,一口推在小玉身上。小玉罵道:「㒲昏了你這淫婦!我後邊看茶,你抱着執壶在席上與娘斟酒。這囬不見了壶兒,你來賴我!」向各處都找尋不着。良久,李瓶兒到房來,迎春如此這般告訴:「琴童兒㧱了一把進來,敎我替他收着。」李瓶兒道:「這囚根子,他做甚麽㧱進他這把壺來?後邊為這把壺好不反亂。玉簫推小玉,小玉推玉簫,急的那大丫頭賭身發咒,只是哭。你趂早還不快替他送進去哩,遲囬管情就賴在你這小淫婦兒身上。」那迎春方纔取出壶,要送入後邊來。後邊玉簫和小玉兩個正亂這把壶不見了,兩個嚷到月娘面前。月娘道:「賊臭肉,還敢嚷的是些甚麽!你們管着那一門兒?把壶不見了!」玉簫道:「我在上邊跟着娘遞酒,他守着銀器家伙,不見了,如今賴我。」小玉道:「大妗子要茶,我不往後邊替他取茶去?你抱着執壺兒,怎的不見了?敢屁股大掉了心了也怎的!」月娘道:「我着恐今日席上再無閒雜人,怎的不見了東西?等住囬看這把壷従那裏出來。等住囬嚷的你主子囬來,沒這壶,管情一家一頓。」玉簫道:「爹若打了我,我把這淫婦饒了也不算!」

正亂着,只見西門慶自外來,問因甚嚷亂。月娘把不見壶一節説了一遍。西門慶道:「慢慢尋就是了,平白嚷的是些甚麽?」潘金蓮道:「若是喫一遭酒,不見了一把,不嚷亂,你家是王十萬!頭醋不酸——到底兒薄。」看官聽説:金蓮此話譏諷李瓶兒首先生孩子,滿月就不見了壶,也是不吉利。西門慶明聽見,只不做聲。只見迎春送壶進來。玉簫便道:「這不是壶有了!」月娘問迎春:「這壶端的在那裏來?」迎春悉把「琴童從外邊㧱到俺娘屋裏收着,不知在那裏來。」月娘因問:「琴童兒那奴才如今在那裏?」玳安道:「他今日該獅子街房子差,上宿去了。」金蓮在旁,不覺鼻子裏笑了一聲。西門慶便問:「你笑怎的?」金蓮道:「琴童兒是他家人,放壶他屋裏,想必要瞞昧這把壶的意思。要呌我,使小廝如今呌將那奴才來,老實打着,問他個下落。不然,頭裏就賴他那兩個,正是走殺金剛坐殺佛!」西門慶聽了,心中大怒,睜眼看着金蓮説道:「看着你恁説起來,莫不李大姐他愛這把壷?旣有了,丟開手就是了,只管亂甚麽!」那金蓮把臉羞的飛紅了,便道:「誰説姐姐手裏沒錢!」説畢,走過一邊使性兒去了。西門慶就被陳經濟來請,説有管磚廠劉太監差人送禮來。往前去看了。金蓮和孟玉樓站在一處,罵道:「恁不逢好死三等九做賊強盜!這兩日作死也怎的?自從養了這種子,恰似他生了太子一般,見了俺們如同生刹神一般,越發通沒句好話兒説了。行動就睜着兩個𣭈窟礲吆喝人!誰不知姐姐有錢?明日慣的他們小廝丫頭養漢做賊,把人㒲遍了也休要管他!」這裏金蓮使性兒不題。

且説西門慶走到前邊,劉太監差了家人送了一罈内酒、一牽羊、兩疋金緞、一盤壽桃、一盤壽麵、四樣嘉餚,一者祝壽,二者來賀。西門慶厚賞來人,打發去了。到後邊,有李桂姐吴銀兒兩個拜辭要家去。西門慶道:「你們兩個再住一日兒,到二十八日我請你帥府周老爹和提刑夏老爹、都監荆老爹、管皇庄薛公公和磚廠劉公公,有院中雜耍扮戯的,教你二位只專遞酒。」桂姐道:「旣留下俺們,我敎頂人家中回媽聲,放心些。」於是把兩人轎子都打發去了,不在話下。

只見西門慶坐了一囬,往前邊去了。孟玉樓道:「你還不去?他管情往你屋裏去了。」金蓮道:「可是他説的,有孩子屋裏熱鬧,俺們沒孩子的屋裏冷清。」正説着,只見春梅従外來。玉樓道:「我説他往你屋裏去了,你還不信哩!這春梅來叫你來了。」一面呌過春梅來問他。春梅道:「我來問玉簫要汗巾子來。他今日借了我汗巾子帶來。」玉樓問道:「你爹在那裏?」春梅道:「爹往六娘房裏去了。」這金蓮聽了,心上如攛上一把火相似,駡道:「賊強人,到明日永世千年,就跌折脚,也别要進我那屋裏。踹踹門檻兒,敎他牢拉的囚根子把懷子骨𢱉折了。」玉樓道:「六姐,你今日怎的下恁毒口咒他?」金蓮道:「不是這等説。賊三寸貨強盜,那鼠腹雞腸的心兒只好有三寸大!一般都是你老婆,無故只是多有了這點尿胞種子罷了,難道怎麽樣兒的?做甚麽恁擡一個滅一個,把人躧到泥裏?」正是:大風刮倒梧桐樹,自有旁人話短長。

次日,西門慶在大廳上錦屏羅列,綺席鋪陳,預先發柬請官客飲酒,因前日在皇莊見管磚廠劉公公,故送了禮來,西門慶這裏發柬請他與薛内相,又邀了應伯爵謝希大兩個相陪。従飯時,二人衣帽齊整,又早先到了。西門慶讓他捲棚内坐,待茶。伯爵因問:「今日哥席間請那幾客?」西門慶道:「有劉薛二内相、帥府周大人、都監荆南崗、敝同僚夏提刑、團練張總兵、衛上范千户、吴大哥、吴二哥。喬老便今日使人來囬了不來。連二位,通只數客。」説畢,適有吴大舅二舅到,作了揖,同坐下。左右放桌兒擺飯。喫畢,應伯爵因問:「哥兒滿月,抱出來不曾?」西門慶道:「也是因衆堂客要看,房下説且休教孩兒出來,恐風篩着他。他奶子説不妨事。教奶子用被裹出來,他大媽屋裏走了遭,應了個日子兒,就進屋去了。」伯爵道:「那日嫂子這裏請去,房下也要來走走。百忙裏他舊時那疾又擧發了,起不的炕兒,心中急的了不的。如今趂人未到,哥倒好説聲,抱哥兒出來,俺們同看一看。」西門慶一面吩咐後邊:「慢慢抱哥兒出來,休要唬着他。對你娘説,大舅二舅在這裏,和應二爹謝爹要看一看。」月娘教奶子如意兒用紅綾小被兒裹的緊緊的,送到捲棚角門首,玳安兒接抱到捲棚内。衆人睜眼觀看,官哥兒穿着大紅緞毛衫兒,生的面白紅唇,甚是富態,都喝采誇獎不已。伯爵與希大,每人袖中掏出一方錦緞兜肚,上着一個小銀墜兒。惟應伯爵與一柳五色線,上穿着十數文長命錢。敎與玳安兒好生抱回房去,休要驚唬哥兒。説道:「相貌端正,天生的就是個戴紗帽胚胞兒!」西門慶大喜,作揖謝了他二人重禮。伯爵道:「哥沒的説,惶恐表意罷了。」

説話中間,忽報劉公公薛公公來了。慌的西門慶穿上衣,儀門迎接。二位内相坐四人轎,穿過肩蟒,纓鎗隊喝道而至。西門慶先讓至大廳上拜見,叙禮捧茶。落後周守備荆都監夏提刑等衆武官,都是錦繡官服,藤棍大扇,軍牢喝道,僚椽跟隨,須臾都到了。門首黑壓壓的許多伺候,裏面鼓樂喧天,笙簫迭奏。上坐遞酒之時,劉薛二内相相讓。廳正面設十二張桌席,都是お拴錦带,花插金瓶。桌上擺着簇盤定勝,地下鋪着錦裀繡毯。西門慶先把盞讓坐次。劉薛二内相再三讓遜:「還有列位大人。」周守備道:「二位老太監齒德俱尊。常言:三歲内宦,居於王公之上。這個自然首坐,何消泛講?」彼此讓遜了一囬,薛内相道:「劉哥,旣是列位不肯,難為東家,喒坐了罷。」於是羅圈唱了個喏,打個躬,劉内相居左,薛内相居右,每人膝下放一條手巾,兩個小廝在傍打扇,就坐下了。其次者纔是周守備荆都監衆人。須臾,堦下一派簫韶,動起楽來。怎見的當日好筵席?但見:食烹異品,菓獻時新。須臾,酒過五巡,湯成三獻。廚役上來割了頭一道小割燒鵝,先首位劉内相,賞了五錢銀子。

教坊司俳官跪呈上大紅紙手本,下邊簇擁一段笑樂的院本,當先是:

\begin{myquote}
{\marktext\small(外扮節級上開)}法正天心順,官清民自安。妻賢夫祸少,子孝父心寬。小人不是別人,乃是上廳節級是也。手下管着許多長行楽俑匠。昨日市上買了一架圍屏,上寫着滕王閣的詩,訪問人請問,人説是唐朝身不滿三尺王勃殿試所作。只説此人下筆成章,廣有學問,乃是個才子。我如今呌副末找尋,若請得他來,見他一見,有何不可。副末的在那裏?{\marktext\small(末云)}堂上一呼,堦下百諾。禀復節級,有何使令?{\marktext\small(外云)}我昨日見那圍屏上寫的滕王閣詩甚好,聞説乃是唐朝身不滿三尺王勃殿試所作。你如今將這個樣板去,限即時就替我請去。請得來,一錢賞賜;請不得來,二十麻杖,決打不饒。{\marktext\small(末云)}小人理會了。{\marktext\small(轉下去)}節級糊塗。那王勃殿試,従唐時到如今,何止千百餘年,教我那裏找尋他去?不免來來去去,到於文廟門首,遠遠望見一位飽學秀才過來,不免動問他一聲。先生,你是做滕王閣詩的身不滿三尺王勃殿試麽?{\marktext\small(淨扮秀才,笑云)}王勃殿試乃唐朝人物,今時那裏有?試哄他一哄。我就是那王勃殿試,滕王閣的詩是我做的。我先念兩句你聽:「南昌故郡,洪都新府。星分翼軫,文光射斗牛之墟;人傑地靈,徐孺下陳蕃之榻。」{\marktext\small(末云)}俺節級與了我這副樣板,身只要三尺,差一指也休請去。你這等身軀,如何充得過?{\marktext\small(淨云)}不打緊,道在人為。你見那裏,又一位王勃殿試來了。{\marktext\small(背粧矮子,末將樣板比,淨越縮。末笑云)}可充得過了。{\marktext\small(淨云)}一件,見你節級切記,好歹小板凳兒要緊。來來去去,到節級門首。{\marktext\small(末令淨)}外邊伺候。{\marktext\small(淨云)}小板凳兒要緊!等進去禀報節級。{\marktext\small(外云)}你請得那王勃殿試來了?{\marktext\small(末云)}現請在門外伺候。{\marktext\small(外云)}你與説,我在中門相待。榛松泡茶,割肉水飯。{\marktext\small(相見科,外云)}此眞乃王勃殿試也!一見尊顏,三生有幸!{\marktext\small(磕下頭)(淨慌科)}小板凳在那裏?{\marktext\small(外又云)}亘古到今,難逢難遇。聞名不曾見面。今日見面勝若聞名。{\marktext\small(再磕下頭去,那淨慌科)}小板凳在那裏?{\marktext\small(末躲過一邊去了。外云)}聞公博學廣記,筆底龍蛇,真才子也!在下如渴思槳,如熱思涼,多拜兩拜。{\marktext\small(淨急了,説道)}你家爺好,你家媽好,你家姐和妹子一家兒都好!{\marktext\small(外云)}都好。{\marktext\small(淨云)}狗㒲娘的,你旣一家大小都好,也教我直直腰兒着!正是:

百寳粧ん帶,珍珠絡臂鞲。

笑時花近眼,舞罷錦纏頭。
\end{myquote}

筵前遞酒,席上衆官都笑了。薛内相大喜,呌上來賞了一兩銀子,磕頭謝了。須臾,李銘吴惠兩個小優兒上來彈唱了。一個ち箏,一個琵琶。周守備先擧手讓兩位内相説:「老太監,吩咐賞他二人唱那套詞兒?」劉太監道:「列位請先。」周守備道:「老太監自然之理,不必計較。」劉太監道:「兩個子弟,唱個『嘆浮生有如一夢裏』。」周守備道:「老太監,此是這歸隱嘆世之詞,今日西門大人喜事,又是華誕,唱不的。」劉太監又道:「你會唱『雖不是八位中紫綬臣,管領的六宫中金釵女』?」周守備道:「此是《陳琳抱粧盒》雜記,今日慶賀,唱不的。」薛太監道:「你呌他二人上來,等我吩咐他。你記的〔普天樂〕『想人生最苦是離别』?」夏提刑大笑道:「老太監,此是離别之詞,越發使不的。」薛太監道:「俺們内官的營生,只曉的答應萬歲爺,不曉的詞曲中滋味,憑他們唱罷。」夏提刑倒還是金吾執事人員,倚仗他刑名官,一樂工上來,吩咐:「你唱套〔三十腔〕。今日是你西門老爹加官進祿,又是好的日子,又是弄璋之喜,宜該唱這套。」薛内相問:「這怎的弄璋之喜?」周守備道:「二位老太監,此日又是西門大人公子彌月之辰,俺們同僚都有薄禮慶賀。」薛内相道:「我等……」因向劉太監道:「劉家,喒們明日都補禮來慶賀。」西門慶謝道:「學生生一豚犬,不足為賀,倒不必老太監費心。」説畢,喚玳安裏邊呌出吴銀兒李桂姐席前遞酒。兩個唱的打扮出來,花枝招颺,望上不端不正插燭也似磕了四個頭兒。起來執壺斟酒,逐一敬奉。兩個楽工又唱一套新詞,歌喉宛囀,真有遶樑之聲。當夜前歌後舞,錦簇花攢,直飲至更餘時分,方纔薛内相起身説道:「生等一者過蒙盛情,二者又值喜慶,不覺留連暢飲,十分擾極。學生告辭。」西門慶道:「盃茗相邀,得蒙光降,頓使蓬蓽增輝。幸再寬坐片時,以畢餘興。」衆人俱出位説道:「生等深擾,酒力不勝。」各躬身施禮相謝。西門慶再三款留不住,只得同吳大舅吴二舅等一齊送至大門。一派鼓樂喧天,兩邊燈火燦爛,前遮後擁,喝道而去。正是:得多少歌舞歡娱嫌日短,故燒高燭照紅粧。

畢竟後項未知如何,且聽下囬分解。

