\includepdf[pages={179,180},fitpaper=false]{tst.pdf}
\chapter*{第九十囬 \\來旺盜拐孫雪娥 雪娥官賣守備府}
\addcontentsline{toc}{chapter}{第九十囬 來旺盜拐孫雪娥 雪娥官賣守備府}
\markboth{{\titlename}卷之九}{第九十囬 來旺盜拐孫雪娥 雪娥官賣守備府}


\begin{myquote}
花開花落開又落,錦衣布衣更換着。

豪家未必常富貴,貧人未必常寂寞;

扶人未必上青天,推人未必塡溝壑:

勸君凡事莫怨天,天意與人無厚薄。
\end{myquote}

話說吳大舅領着月娘等一簇男女,離了永福寺,順着大樹長堤前來。玳安又早在杏花村酒樓下邊,人煙熱鬧,揀高阜去處,那裏幕天席地設下酒餚,等候多時了。遠遠望月娘衆人轎子到了,問道:「如何這咱纔來?」月娘又把永福寺中遇見春梅告訴一遍。不一時,斟上酒來。衆人坐下正飲酒,只見樓下香車綉轂,往來人煙喧雜,車馬轟雷,笙歌鼎沸。月娘衆人躧着高阜,把眼觀看,看見人山人海圍着,都看敎師走馬耍解的。

原來是本縣知縣相公兒子李衙内,名喚李拱璧,年約三十餘歲,現為國子上舍,一生風流博浪,懶習詩書,專好鷹犬走馬,打毬蹴踘,常在三瓦兩巷中走,人稱他為「李棍子」。那日穿着一弄兒輕羅軟滑衣裳,頭戴金頂纏棕小帽,脚踏乾黄靴,納綉襪口,同廊吏何不韋,帶領二三十好漢,拏彈弓、吹筒、毬棒,在於杏花莊大酒樓下,看教師李貴走馬賣解:豎肩樁,隔肚带,輪鎗舞棒,做各樣技藝頑耍。有這許多男女,圍着哄笑。那李貴諢名號為「山東夜叉」,頭戴萬字巾,腦後撲匾金環,身穿紫窄衫,銷金裹肚,脚上八搭腿絣,乾黄䩺靴,五彩飛魚襪口,坐下銀鬃馬,手執朱紅桿明鎗,提招風令字旗,在街心扳鞍上馬,高聲説念一篇道:

\begin{myquote}
「我做敎師世罕有,江湖遠近揚名久。

雙拳打下如鎚鑽,兩脚入來如飛走。

南北兩京打類臺,東西兩廣無敵手。

分明是個鐵嘴行,自家本事何曾有?

少林棍,只好打田鷄;董家拳,只好嚇小狗。

撞對頭不敢喊一聲,沒人處專會誇大口。

騙得銅錢放不牢,一心要折章臺柳。

虧了北京李大郎,養我在家為契友:

蘸生醬喫了半畦蒜,捲春餅𠳹了兩擔韮。

小人自來生得饞,寅時喫酒直到酉。

牙齒疼,把來剉一剉;肚子脹,將來扭一扭。

充饑喫了三斗米飯,點心喫了一大缸酒。

多虧了此人未得酬,來世做隻看家狗。

若有賊來掘壁洞,把他陰囊咬一口。

問君何故咬他囊?動不的手來只動口!」
\end{myquote}

當下李衙内正看處,忽擡頭看見一簇婦人在高阜處飲酒,一見那長挑身材婦人,不覺心搖目蕩,觀之不足,看之有餘,口中不言,心内暗道:「不知誰家婦女,有男子沒有?」一面呌過手下答應的小張閒架兒來,悄悄吩咐:「你去那高坡上打聽那三個穿白的婦人是誰家的?訪得是實,告我知道。」那小張閒掩口應諾,雲飛跑去。不多時,走到跟前附耳低言,囬報說:「如此這般,是縣門前西門慶家妻小。一個年老的姓吳,是他妗子。一個五短身材,是他大娘子吳月娘。那個長挑身材,有白麻子的,是第三個娘子,姓孟,名喚玉樓。如今都守寡在家。」這李衙内聽了,獨看上孟玉樓,重賞小張閒,不在話下。

吳大舅和月娘衆人,觀看了半日,見日色銜山,令玳安收拾了食盒,攛掇月娘上轎囘家。一路上得多少錦轡郎搖羅袖醉,綺羅人揭綉簾看。有詩為證:

\begin{myquote}
柳底花陰壓路塵,一囘遊賞一囘新。

有緣千里來相會,無緣對面不相親。
\end{myquote}

這月娘衆人囘家,不題。卻說那日孫雪娥與西門大姐在家,午後時分無事,都出大門首站立。也是天假其便,不想一個搖驚閨的過來。那時賣胭脂粉花翠生活,磨鏡子,都搖驚閨。大姐說:「我鏡子昏了,使平安兒呌住那人,與我磨磨鏡子。」那人放下擔兒,說道:「我不會磨鏡子,我賣些金銀生活,首飾花翠。」站立在門前,只顧眼上眼下看着雪娥。雪娥便道:「那漢子,你不會磨鏡子,去罷,只顧看我怎的?」那人說:「雪姑娘,大姑娘,不認的我了?」大姐道:「眼熟,急忙想不起來。」那人道:「我是爹手裏出去的來旺兒。」雪娥便道:「你這幾年在那裏來?怎的不見?出落得恁胖了!」來旺兒道:「我離了爹門,到原籍徐州,家裏閒着沒營生,投跟了個老爹上京來做官。不想到半路裏,他老爺兒死了,丁憂家去了。我便投在城内顧銀舖,學會了此銀行手藝,揀鈒大器頭面,各樣生活。這兩日行市遲,顧銀舖敎我挑副擔兒出來,街上發賣些零碎。看見娘們在門首,不敢來相認,恐怕踅門瞭戶的。今日不是你老人家呌住,還不敢相認。」雪娥道:「原來敎我只顧認了半日,白想不起。旣是舊兒女,怕怎的?」因問:「你擔兒裏賣的是甚麽生活?挑進裏面,等俺們看一看。」那來旺兒一面把擔兒挑入裏邊院子裏來,打開箱子,用匣兒託出幾件首飾來,金銀鑲嵌不等,打造得十分奇巧。但見:

\begin{myquote}
孤雁啣蘆,雙魚戯藻。牡丹巧嵌碎寒金,貓眼釵頭火焰蠟。也有獅子滚毬,也有駱駝獻寳。滿冠擎出廣寒宫,掩鬢鑿成桃源境。左右圍髮,利市相對荔枝叢;前後分心,觀音盤膝蓮花座。也有寒雀爭梅,也有孤鸞戯鳳。正是:縧環平安祖母綠,帽頂高嵌佛頭青。
\end{myquote}

看了一囬,問來旺兒:「你還有花翠?拏出來。」那來旺兒又取一盒子各樣大翠鬢花,翠翹滿冠,並零碎草蟲生活來。大姐揀了他兩對鬢花,這孫雪娥便留了他一對翠鳳,一對柳穿金魚兒。大姐便稱出銀子來與他,雪娥兩件生活,欠他一兩二錢銀子,約下他:「明日早來取罷。今日你大娘不在家,同你三娘和哥兒都往坟上與你爹燒紙去了。」來旺道:「我去年在家裏,就聽見人說爹死了,大娘生了哥兒,怕不的好大了?」雪娥道:「你大娘孩兒,如今纔周半兒,一家兒大大小小,如寳上珠一般,全看他過日子哩。」說話中間,來昭妻一丈青出來,傾了盞茶與他喫,那來旺兒接了茶,與他唱了個喏。來昭也在跟前,同敍了囬話,吩咐:「你明日來見見大娘。」那來旺兒挑擔出門。

到晚上,月娘衆人轎子來家。雪娥大姐衆人丫鬟接着,都磕了頭。玳安跟盒擔,赶不上,僱了疋驢兒騎來家,打發擡盒人去了。月娘告訴雪娥大姐説今日寺裏遇見春梅一節:「原來他把潘家的就葬在寺後首,俺們也不知他來替他娘燒紙,悞打悞撞遇見他,娘兒們又認了囘親。先是寺裏長老擺齋喫了,落後又放下兩張桌席,教伴當擺上他家的四五十攢盒,各樣菜蔬下飯,篩酒上來,通喫不了。他看見哥兒,又與了一對簪兒,好不和氣。起解行三坐五,坐着大轎子,許多跟隨。又且是出落的比舊時長大了好些,越發白胖了。」吳大妗子道:「他倒也不改常忘舊。那咱在咱家時,我見他比衆丫鬟行事兒正大,說話兒沉穩,就是個材料兒。你看今日福至心靈,恁般造化!」孟玉樓道:「姐姐没問他,我問他來,果然半年没洗換,身上懷着喜事哩。也只是八九月裏孩子,守備好不喜歡哩!薛嫂兒說的倒不差。」說了一囬,雪娥提起:「今日娘不在,我和大姐在門首看見來旺兒。原來又在這裏學會了銀匠,挑着擔兒賣金銀生活花翠,俺們就不認得他了,買了他幾枝花翠。他問娘來,我說往坟上燒紙去了。」月娘道:「你怎的不敎他等着我來家?」雪娥道:「俺們呌他明日來。」

正坐着説話,只見奶子如意兒向前對月娘説:「哥哥來家,這半日只是昏睡不醒,口中出冷氣,身上燙燒火熱的。」這月娘聽見慌了,向炕上抱起孩兒來,口搵着口兒,果然出冷汗,渾身發熱。罵如意兒:「好淫婦,此是轎子冷了孩兒了!」如意兒道:「我拏小被兒裹的嚴嚴的,怎得凍着?」月娘道:「再不是,抱了往那死鬼坟上唬了他來了!那等吩咐,敎你休抱他去,你不依,浪着抱的去了!」如意兒道:「早是小玉姐看着,抱了他到那裏,看看就來了,幾時唬着他來?」月娘道:「別要說嘴!看那看兒,便怎的卻把他唬了?」即忙呌來安兒:「快請劉婆子去。」不一時,劉婆來到,看了脉息,摸摸身上說:「着了些驚寒,撞見祟祸了。」留了兩服硃砂丸,用姜湯灌下去。吩咐奶子抱着他熱炕上睡。到半夜出了些冷汗,身上纔凉了。於是管待劉婆子喫了茶,與了他三錢銀子,呌他明日還來看看。一家子慌的了不的,開門闔戶,整亂了半夜。

卻説來旺次日依舊挑將生活擔兒,來到西門慶門首,與來昭唱喏説:「昨日雪姑娘留下我些生活,許下今日敎我來取銀子,就見見大娘。」來昭道:「你且去着,改日來。昨日大娘來家,哥兒不好,呌醫婆太醫看下薬,整亂一夜,好不心焦。今日纔好些,那得工夫稱銀子與你?」正說着,只見月娘玉樓雪娥送出劉婆子來,到大門首,看見來旺兒。那來旺兒趴在地下,與月娘玉樓磕了兩個頭。月娘道:「幾時不見你,就不來這裏走走!」來旺兒悉將前事說了一遍:「要來,不好來的。」月娘道:「舊兒女人家,怕怎的?你爹又沒了。當初只因潘家那淫婦,一頭放火,一頭放水架的舌,把個好媳婦兒生逼臨的弔死了,將有作没,把你墊發了去。今日天也不容他,往那去了!」來旺兒道:「也說不的,只是娘心裏明白就是了。」説了囘話,月娘問他:「賣的是甚樣生活?拏出來瞧。」揀了他幾件首飾,該還他三兩二錢銀子,都用等子稱了與他。呌他進入儀門裏面,吩咐小玉取一壺酒來,又是一盤點心,敎他喫。那雪娥在厨上一力攛掇,又熱了一大碗肉出來與他。喫的酒飯飽了,磕頭出門。月娘玉樓衆人歸到後邊去,雪娥獨自悄悄和他打話:「你常常來走着,怕怎的?奴有話敎來昭嫂子對你說。我明日晚夕,在此儀門裏紫牆兒跟前耳房内等你!」兩個遞了眼色,這來旺兒就知其意,說:「這儀門晚夕關不關?」雪娥道:「如此這般,你來先到來昭屋裏,等到晚夕,踩着梯櫈,越過牆,順着遮隔,我這邊接你下來。咱二人會合一面,還有底細話與你説。」這來旺得了此話,正是歡従額起,喜向腮生。作辭雪娥,挑擔兒出門。正是:不着家神,弄不得家鬼。有詩為證:

\begin{myquote}
閒來無事倚門闌,偶遇多情舊日緣。

對人不敢高聲語,故把秋波送幾番。
\end{myquote}

這來旺兒歡喜囘家,一宿無話。到次日,也不挑擔兒出來賣生活,慢慢踅來西門慶門首,等來昭出來,與他唱喏。那來昭便說:「旺兒希罕,好些時不見你了。」來旺兒說:「沒事,閒來走走。裏邊雪姑娘少我幾錢生活銀,討討。」來昭道:「既如此,請來屋裏坐。」把來旺兒讓到房裏坐下。來旺兒道:「嫂子怎不見?」來昭道:「你嫂子今日後邊上竃哩。」那來旺兒拏出一兩銀子,遞與來昭說:「這幾星銀子,取壶酒來和哥嫂喫。」來昭道:「何消這許多!」即叫他兒子鐵棍兒過來,那鐵棍弔起頭去,十五歲了,拏壺出來,打了一大注酒,使他後邊叫一丈青來。不一時,一丈青蓋了一錫鍋熱飯,一大碗雜熬下飯,兩碟菜蔬,說道:「好呀,旺官兒在這裏。」來昭便拏出銀子與一丈青瞧,說:「兄弟破費,也打壶酒咱兩口兒喫。」一丈青笑道:「無功消受,怎生使得?」一面放了炕桌,讓來旺炕上坐,擺下酒菜,把酒來斟。來旺兒先傾頭一盞遞與來昭,次斟一盞與一丈青,深深唱喏,說:「一向不見哥嫂,這盞水酒,孝順哥嫂。」一丈青便說:「哥嫂不道酒肉喫傷了!你對眞人,休説假話。裏邊雪姑娘昨日已央及逹知我了,你兩個舊情不斷,託俺們兩口兒如此這般周全。你們休推睡裏夢裏,要問山下路,且得過來人。你若入港相會,有東西出來,休要獨喫,須把些汁水敎我呷一呷,俺替你們須躭許多利害。」那來旺便跪下説:「只是望哥嫂周全,並不敢有忘。」說畢,把酒喫了。一囘,一丈青往後邊和雪娥答了話,出來對他說,約定晚上來,來昭屋裏窝藏,待夜裏関上儀門,後邊人歇下,越牆而過,於中取事。有詩為證:

\begin{myquote}
報應本無私,影響皆相似!

要知祸福因,但看所為事。
\end{myquote}

這來旺得了此言,囘來家,巴不到晚,踅到來昭屋裏,打酒和他兩口兒喫。至更深時分,更無一人覺的,直待的大門關了,後邊儀門上了拴,家中大小歇息定了,彼此都有個暗號兒,只聽牆内雪娥咳嗽之聲。這來旺兒躧着梯櫈,黑影中爬過粉牆,順着遮陽排子,雪娥那邊用櫈子接着。兩個在西耳房堆馬鞍子去處,兩個相摟相抱,雲雨做一處。彼此都是曠夫寡女,慾心如火。那來旺兒纓鎗強壯,儘力盤弄了一囘,楽極精來,一泄如注。事畢,雪娥遞與他一包金銀首飾,幾兩碎銀子,兩件緞子衣服。吩咐:「明日晚夕你再來,我還有些細軟與你。你外面尋下安身去處。往後這家中過不出好來,不如我和你悄悄出去,外邊尋下房兒,成其夫婦。你又會銀行手藝,愁過不的日子!」來旺兒便說:「如今東門外細米巷有我個姨娘,有名收生的屈老娘。他那裏曲彎小巷倒避眼,咱兩個投奔那裏去。遲些時,看無動靜,我帶你往原籍家裏,買幾畝地種去也好。」兩個商量已定。這來旺兒作别雪娥,依舊爬過牆來,到來昭屋裏,等至天明,開了大門,挨身出去。到黄昏時分,又來門首,踅入來昭屋裏。晚夕,依舊跳過牆去,兩個幹事。朝來暮往,非止一日,也抵盜了許多細軟東西,金銀器皿,衣服之類。來昭兩口子也得抽分好些肥己,俱不必細說。

一日,後邊月娘看孝哥兒出花兒,心中不快,睡得早。這雪娥房中使女中秋兒,原是大姐使的,因李嬌兒房中元宵兒被經濟要了,月娘就把中秋兒與了雪娥,把元宵兒扶侍大姐。那一日,雪娥打發中秋兒睡下,房裏打點一大包釵環頭面,裝在一個匣内,用手帕蒙蓋了頭,隨身衣服,約定來旺兒在來昭屋裏等候,兩個要走。這來昭便說:「不爭你走了,我看守大門管放水鴨兒?若大娘知道,問我要人,怎了?不如你二人打房上去,就躧破些瓦,還有蹤跡。」來旺兒道:「哥也說得是!」雪娥又留一個銀折盂、一根金耳斡、一件青綾襖、一條黄綾裙,謝了他兩口兒。直等五更鼓,月黑之時,隔房爬過去。來昭夫婦又篩上兩大鍾煖酒,與來旺雪娥喫,說:「喫了好走,路上壯膽些!」喫到五更時分,每人拏着一根香,躧着梯子,打發兩個爬上房去,一步一步走,把房上瓦也跐破許多。比及爬到房簷跟前,街上人還未行走。聽巡捕的聲音,這來旺兒先跳下去,後卻敎雪娥躧着他肩背,接摟下來。兩個往前邊走,到十字路口上,被巡捕的攔住,便說:「往那裏去的男女?」雪娥便唬慌了手脚,這來旺兒不慌不忙,把手中官香彈了一彈,說道:「俺是夫婦二人,前往城外岳廟裏燒香,起的早了些,長官勿怪。」那人問:「背的包袱内是甚麽?」來旺兒道:「是香燭紙馬。」那人道:「既是兩口兒岳廟燒香,也是好事,你快去罷。」這來旺得不的一聲,拉着雪娥往前飛走。走到城下,城門纔開。打人鬧裏挨出城去,轉了幾條街巷。

原來細米巷在個僻靜去䖏,住着不多幾家人家,都是矮房低廈,後邊就是大水穴沿子。到於屈姥姥家,屈姥姥還未開門。呌了半日,屈姥姥纔起來開了門,見來旺兒領了個婦人來。原來來旺兒本姓鄭,名喚鄭旺,說:「這婦人是我新尋的妻小,姨娘這裏有房子,且借一個寄住些時,再尋房子。」遞與屈姥姥三兩銀子,敎買柴米。那屈姥姥見這金銀首飾,來因可疑。他兒子屈鏜,因他娘屈姥姥安歇鄭旺夫妻,二人帶此東西,夜晚見財起意,掘開房門,偷盗出來耍錢,致被捉獲,具了事件,拏去本縣見官。李知縣見係賊贜之事,贜物執證見在,差人押着屈鏜到家,把鄭旺孫雪娥一條索子都拴了。那雪娥唬的臉蠟渣也似黄了,換了滲淡衣裳,帶着眼紗,把手上戒指都勒下來打發了公人,押去見官。當下烘動了一街人觀看,有認得的,說是:「西門慶家小老婆,今被這走出去的小廝來旺兒,今改名鄭旺,通姦拐盗財物,在外居住。又被這屈鏜掏摸了,今事發見官。」當下一個傳十,十個傳百個,路人行人口似飛。

月娘家中自従雪娥走了,房中中秋兒見箱内細軟首飾都沒了,衣服丢的亂三攪四,報與月娘。月娘喫了一驚,便問中秋兒:「你跟着他睡,走了,你豈會不知?」中秋兒便說:「他要便晚夕悄悄偷走出外邊,半日方囘,不知詳細。」月娘又問來昭:「你看守大門,人出去你怎不曉的?」來昭便說:「大門每日上鎖,莫不他飛出去?」落後看見房上瓦躧破許多,方知越房而去了。又不敢使人躧訪,只得按納含忍。不想本縣知縣當堂問理這件事,先把屈鏜夾了一頓,追出金頭面四件、銀首飾三件、金環一雙、銀鍾二個、碎銀五兩、衣服二件、手帕一個、匣一個;向鄭旺名下,追出銀三十兩、金碗簪一對、金仙子一件、戒指四個;向雪娥名下,追出金挑心一件、銀鐲一付、金鈕五付、銀簪四對、碎銀一包;屈姥姥名下追出銀三兩。就將來旺兒問擬奴婢因奸盗取財物,屈鏜係竊盗,俱係雜犯死罪,准徒五年,贜物入官。雪娥孫氏係西門慶妾,與屈姥姥當下都當官拶了一拶。屈姥姥供明放了,雪娥責令本縣差人,到西門慶家敎人遞領狀領孫氏。那吳月娘呌吳大舅來商議:「已是出醜,平白又領了來家做甚麽?沒的玷辱了家門,與死的裝幌子。」打發了公人錢,囘了知縣話。知縣拘將官媒人來,當官變賣。

卻説守備府中春梅,打聽得知,説西門慶家中孫雪娥,如此這般,被來旺兒拐出,盗了財物去,在外居住,事發到官,如今當官變賣。這春梅聽見,要買他來家上竃,要打他嘴,以報平昔之仇。對守備說:「雪娥善能上竃,會做的好茶飯湯水,買來家中伏侍。」這守備即便差張勝李安,拏帖兒對知縣説。知縣自恁要做分上,只要八兩銀子官價。交完銀子,領到府中,先見了大奶奶並二奶奶孫氏,次後到房中來見春梅。春梅正在房裏縷金床錦帳之中纔起來。手下丫鬟領雪娥見面。那雪娥見是春梅,不免低頭進見,望上倒身下拜,磕了四個頭。這春梅把眼瞪一瞪,喚將當直的家人媳婦上來:「與我把這賤人,撮去了䯼髻,剝了上蓋衣裳,打入厨下,與我燒火做飯!」這雪娥聽了,口中只呌苦。自古世間打牆板兒翻上下,討米卻做管倉人!旣在他簷下,怎敢不低頭?孫雪娥到此地步,只得摘了髻兒,換了艷服,滿臉悲慟,往厨下去了。有詩為證:

\begin{myquote}
布袋和尚到明州,策杖芒鞋任意遊。

饒你化身千百億,一身還有一身愁。
\end{myquote}

畢竟未知後來如何,且聽下囘分解。

\part*{夢梅館校本《金瓶梅詞話》卷之十}
\addcontentsline{toc}{part}{夢梅館校本《金瓶梅詞話》卷之十}

