\includepdf[pages={109,110},fitpaper=false]{tst.pdf}
\chapter*{第五十五囬 \\西門慶東京慶壽旦 苗員外揚州送歌童}
\addcontentsline{toc}{chapter}{第五十五囬 西門慶東京慶壽旦 苗員外揚州送歌童}
\markboth{{\titlename}卷之六}{第五十五囬 西門慶東京慶壽旦 苗員外揚州送歌童}


\begin{myquote}
千歲蟠桃帶露攜,擕來黄閣祝期頤。

八僊下降稱觴日,七鳳團花織錦時。

六合五溪輸賀軸,四夷三島獻珍奇。

羲和莫遣兩丸速,願壽中朝帝者師。
\end{myquote}

卻説任醫官看了脉息,依舊到廳上坐下。西門慶便開言道:「不知這病症看得何如?没的甚事麽?」任醫官道:「夫人這病,原是産後不愼調理,因此得來。目下惡露不淨,面帶黃色,飲食也沒些要緊,走動便覺煩勞。依學生愚見,還該謹愼保重。大凡婦人產後,小兒痘後,最難調理,略有些差池,便種了病根。如今夫人兩手脉息,虚而不實,按之散大,卻又軟不能自固。這病症,都只為火炎肝腑,土虚木旺,虚血妄行。若今番不治,他後邊一發了不的了。」説畢。西門慶道:「如今該用甚薬纔好?」任醫官道:「只是用些清火止血的薬,黃栢知母為君,其餘只是地黃黄岑之類,再加減些,吃下若住,就好了。」西門慶聽了,就呌書童封了一兩銀子,送任醫官做薬本。任醫官作謝去了。不一時,送將薬來。李瓶兒屋裏煎服,不在話下。

且説西門慶送了任醫官去,回來與應伯爵坐地。想起東京蔡太師壽旦已近,先期曾差玳安往杭州買辦龍袍錦繡金花寳貝上壽禮物,俱已完備,即日要自往東京拜賀,算來日期已近,自山東來到東京,也有半個月的路程,連夜收拾行李進發,剛剛正好,再遲不的了。便進房來和月娘説知,如此這般。月娘道:「這早時不説,如今忙匆匆的,你擇定幾時起身?」西門慶道:「明日起身也纔夠到哩,還得十幾個日頭。」西門慶説畢,就走出外來,吩咐玳安、書童、琴童、畫童,打點衣服行李,明日跟隨東京走一遭。四個小廝各各收拾行李不迭。月娘便教小玉:「去請你各房娘,都來收拾你爹行李。」當下只有李瓶兒,一來有了孩子,二來服了薬,不出房來。其餘各房,孟玉樓、潘金蓮,一齊都到,走來的都動手,把皮箱涼箱裝了蟒衣龍袍緞匹上壽等物,共有二十多扛,又整頓了應用冠帶衣服等件,一齊完了。晚夕,三位娘子擺設酒餚和西門慶送行,席上西門慶各人叮囑了幾句,自進月娘房裏宿歇。次日把二十扛行李,先打發出門。又發了一張通行馬牌,仰經過驛遞起夫馬迎送。各各停當,然后進李瓶兒房裏來,看了官哥兒,與李瓶兒説了幾句話,教他好好調理,「我不久便來家看你。」那李瓶兒閣着淚道:「路上小心保重。」直送出廳來,和月娘、玉樓、金蓮,打夥兒送出了大門。

西門慶乘了涼轎,四個小廝騎了頭口,望東京進發。迤邐行來,卻走了百里路程。那時日已傍晚,西門慶吩咐駐劄。驛官廝見,送供應,過了一宵。明日天早,西門慶催趲人馬,扛箱快行。一路看了些山明水秀。午牌時,打中火,又行。路上相遇的,無非各路文武官員進京慶賀壽旦的,也有進生辰摃的,不計其數。又行了十來日,算前途路已不多,趲到剛剛凑巧。宿了一晚,又行夠兩日,早到東京,進了萬壽城門。那時天色將晚,趕到龍德街牌樓底下,就投翟家屋裏去住歇。那翟管家聞知西門慶到了,忙的出來迎接,各叙寒喧。吃了茶,西門慶叫玳安專管行李,一一交盤進了翟家裏來。翟謙教府幹收了,就擺酒和西門慶洗塵。不一時,只見剔犀官桌上列着幾十樣大菜,幾十樣小菜,都是珍饈美味,燕窝魚翅,絶好下飯,只沒有龍肝鳳髓;其餘奇巧富麗,便是蔡太師自家受用也不過如此。當直的拿着通天犀杯,斟上麻姑酒兒,遞與翟謙。接過滴了天,然後又斟上來,把盞與西門慶。西門慶也囘敬了。兩人坐下,糖菓熱碟按酒之物,流水也似遞將上來。酒過兩巡,西門慶便對翟謙道:「學生此來,單為老太師慶壽,聊備些微禮,孝順太師,想不見卻。只是學生向有相攀的心,欲求親家預先稟過,但拜太師門下做個乾生子,也不枉了一生一世。不知可以啟口帶攜的學生麽?」翟謙道:「這個有何難哉?我們主人雖是朝廷大臣,卻也極好奉承,今日見了這般盛禮,自然還要陞選官爵,不惟拜做乾子,定然允哩!」西門慶聽説,不勝之喜。飲夠多時,西門慶便推:「不吃酒罷!」翟管家道:「再請一杯,怎的不吃了?」西門慶道:「明日有正經事,卻不敢多飲。」再四相勸,只得又吃了一杯。翟管家賞了隨従人酒食,吩咐呌把牲口牽到後槽去。

當下收過了家活,就請西門慶到後邊書房裏安歇。早安排下好描金暖床,鮫綃帳兒,把銀鉤掛起,露出一床好錦被,香噴噴的。一班小廝,扶侍西門慶脱衣脱襪上床。獨宿孤眠,西門慶一生不慣,那一晚好難捱過也。巴到天明,正待起身,那翟家門戶重掩着,那裏討水來淨臉?直挨到巳牌時分,纔有個人把匙鑰一路開將出來。隨后一個小廝拿着手巾,一個捧着銀面盆,傾了香湯,進書房來。西門慶梳洗完畢,戴上忠靖冠,穿着外蓋衣服,一個在書房裏坐。只見翟管家出來,和西門慶廝見了坐下,當直的託出一個朱紅盒子,裏邊有三十來樣美味,一把銀壺。斟上酒來,吃早飯。翟謙道:「請用過早飯,學生先進府去,和主翁説過,然後親家搬禮物進來。」西門慶道:「多勞費心。」酒過數盃,就拿早飯來吃了,收過家活。翟管家道:「且權坐一回,學生進府去便來。」翟謙去不多時,忙跑來家向西門慶説:「老爺正在書房梳洗,外邊滿朝文武官員,都各伺候拜壽,未得廝見哩。學生已對老爺説過了,如今先進去拜賀,省的住會混雜,學生也隨後便到了。」西門慶不勝歡喜,便教跟隨人拉同翟家幾個伴當,先把那二十扛金銀緞疋,擡到太師府前,一行人應聲去了。西門慶冠帶,乘了轎來,只見亂哄哄的挨肩擦背,都是大小官員來上壽的。西門慶遠遠望見一個官員也乘着轎進龍德坊來。西門慶仔細一認,倒是揚州苗員外。卻不想苗員外也望見西門慶了。兩個同下轎作揖,敍別來寒温。原來這苗員外是第一個財主,他身上也現做個散官之職。向來結交在蔡太師門下,那時也來上壽,恰遇了故人。當下兩個忙匆匆路次話了幾句,問了寓處,分手而别。西門慶來到太師府前,但見:

\begin{myquote}
堂開綠野,彷彿雲霄;閣起凌煙,依稀星斗。門前寬綽堪旋馬,閥閲嵬峩好竪旗。錦綉叢中,風送到畫眉聲巧;金銀堆裏,日映出琪樹花香。旃檀香,截成梁棟;醒酒石,滿砌階除。左右肉屏風,一個個夷光紅拂;滿堂羅寳翫,一件件周鼎商彝。明晃晃懸掛着明珠十二,黑夜裏何用燈油;貌堂堂招致得珠履三千,彈短鋏盡皆名士。任他九州四海大小官員,都來慶賀;就是六部尚書三邊總督,無不低頭。正是:除卻萬年天子貴,只有當朝宰相尊。
\end{myquote}

西門慶恭身進了大門,只見中門關着不開,官員都打從角門而入。西門慶便問:「為何今日大事,卻不開大門?」翟管家道:「原來中門曾經官家行幸,因此人不敢打這門出入。」西門慶和翟管家進了幾重門,門上都是武官把守,一些兒也不混亂。見了翟謙,一個個都欠身問:「管家従何䖏來?」翟管家答道:「舍親打山東來拜壽老爺的。」説罷,又走過幾座門,轉幾個彎,無非是畫棟雕梁,金張甲第。隱隱聽見鼓楽之聲,如在天上的一般。西門慶又問道:「這裏民居隔絶,那裏來的鼓楽喧嚷?」翟管家道:「這是老爺教的女樂。一班共二十四人,也曉得天魔舞、霓裳舞、觀音舞,凡老爺早膳、中飯、夜燕,都是奏的。如今想是早膳了。」西門慶聽言未了,又鼻子裏覺得異香馥馥,楽聲一發近了。翟管家道:「這裏老爺書房將到了,脚步兒放鬆些。」轉個迴廊,只見一座大廳,如寳殿僊宫。廳前僊鶴孔雀,種種珍禽,又有那瓊花曇花佛桑花,四時不謝,開的閃閃爍爍,應接不暇。西門慶還未敢闖進,教翟管家先進去了,然后挨挨排排,走到堂前。堂上虎皮太師交椅上,坐一個大猩紅蟒衣的,是太師了。屏風後列有二三十個美女,一個個都是宫樣粧束,執巾執扇,捧擁着他。翟管家也站在一邊。西門慶朝上拜了四拜,蔡太師也起身,就狨毯上回了個禮。這是初相見了。落後翟管家走近蔡太師耳邊,暗暗説了幾句話下來。西門慶理會的是那話了,又朝上拜四拜,蔡太師便不答禮。這四拜是認乾爺了。因受了四拜,後來都以父子相稱。西門慶開言道:「孩兒沒甚孝順爺爺,今日華誕,家裏備的幾件菲儀,聊表千里鵝毛之意。願老爺壽比南山!」蔡太師道:「這怎的生受!」便請坐下。當直的拿了把椅子上來,西門慶朝上作了個揖道:「告坐了。」就西邊坐地,吃茶。翟管家慌跑出門來,叫擡禮物的都進來。二十來扛禮物,揭開了涼箱蓋,呈上一個禮目:大紅蟒袍一套、官綠龍袍一套、漢錦二十疋、蜀錦二十疋、火浣布二十疋、西洋布二十疋、其餘花素尺頭共四十疋、獅蠻玉帶一圍、金鑲奇南香帶一圍、玉盃犀盃各十對、赤金攢金爵盃八隻、明珠十顆、又梯己黃金二百兩,送上蔡太師做贄見的禮。蔡太師看了禮目,又瞧了擡上二十來扛,心下十分歡喜,連聲稱「多謝」不迭。便敎翟管家:「收進庫房去罷。」一面吩咐擺酒款待。西門慶因見忙冲冲,推事故辭别了蔡太師。太師道:「既如此,下午早早來罷。」西門慶作個揖起身,蔡太師送了幾步便不送了。西門慶依舊和翟管家同出府來。翟管家府内有事,也作別進去。西門慶徑回到翟家來,脱下冠帶,又整的好飯吃了一頓。回到書房,打了個瞌睡,恰好蔡太師差舍人邀請赴席。西門慶謝了些扇金,着先去:「隨後就來了。」便重整冠帶,預先叫玳安封下許多賞封,做一拜匣盛了,跟隨着四個小廝,乘轎望太師府來,不題。

且説蔡太師那日滿朝文武官員來慶賀的,各各請酒。自次日為始,分做三停:第一日是皇親内相,第二日是尚書顯要衙門官員,第三日是内外大小等職。只有西門慶一來遠客,二來送了許多禮物,蔡太師倒十分歡喜他。因此就是正日,獨獨請他一個。見説請到了新乾子西門慶,忙走出軒下相迎。西門慶再四謙遜,讓爺爺先行,自家屈着背,輕輕跨入檻内。蔡太師道:「遠勞駕従,又損隆儀,今日畧坐,少表微忱。」西門慶道:「孩兒戴天履地,全賴爺爺洪福,些小敬意,何足掛懷?」兩個喁喁笑語,眞似父子一般。二十個羙女一齊奏楽。府幹當直的斟上酒來,蔡太師要與西門慶把盞,西門慶力辭不敢,只領的一盞,立飲而盡,隨即坐了筵席。西門慶教書童取過一隻黃金桃盃,斟上一盃滿滿,走到蔡太師席前,雙膝跪下道:「願爺爺千歲!」蔡太師滿面歡喜道:「孩兒起來。」接過便飲個完。西門慶纔起身,依舊坐下。那時相府華筵,珍奇萬狀,都不必説。西門慶直飲到黄昏時候,拿賞封賞了諸執役人,纔作謝告別道:「爺爺貴冗,孩兒就此叩謝,後日不敢再來求見了。」出了府門,仍到翟家安歇。

次日,要拜苗員外,着玳安跟尋了一日,卻在皇城後李太監房中住下。玳安拿着帖子通報了,苗員外來出迎道:「學生一個兒坐着,正想個知心的朋友講講,恰好來得凑巧。」就留西門慶筵燕。西門慶推卻不過,只得便住了。當下山餚海錯,不記其數。又有兩個歌童,生的眉清目秀,開喉音唱幾套曲兒。西門慶指着玳安、琴童、書童、畫童,向苗員外説道:「那班蠢材,只顧吃酒飯,卻怎地比的那兩個!」苗員外笑道:「只怕伏侍不的老先生。若愛時,就送上也何難!」西門慶謙謝:「不敢奪人之好。」飲到更深,別了苗員外,依舊來翟家歇。

那幾日内,相府管事的,各各請酒,留連了八九日。西門慶歸心如箭,便叫玳安收拾行李。那翟管家苦死留住,只得又吃了一夕酒,重敍姻親,極其眷戀。次日,早起辭別,望山東而行。一路水宿風餐,不在話下。

且説自従西門慶往東京慶壽,姊妹們眼巴巴望西門慶囬來,多有懸掛。在屋裏做些針指,通不出來閒耍。只有那潘金蓮打扮的如花似玉,喬模喬樣,在丫環夥裏,或是猜枚,或是抹牌,説也有,笑也有,狂的通没些成色,嘻嘻哈哈,也不顧人看見,只想着與陳經濟勾搭,便心上亂亂的焦燥起來。多少長吁短嘆,託着腮兒獃登登。本待要等經濟回來,和他做些營生,又不道,經濟每日在店裏沒的閒。欲要自家出來尋着他,又有許多丫頭,往來不方便。日裏便似熬盤上蟻子一般,跑進跑出,再不坐在屋裏。那一日正是風和日暖,那金蓮身邊帶着許多麝香合香,走到捲棚後面,只望着雪洞裏。那經濟日在店裏,那得脫身進來?望了一回不見,只得來到屋裏,把筆在手,吟哦了幾聲,便寫一封書,封着,呌春梅:「逕送與陳姐夫。」經濟接着,拆開従頭一看,卻不是書——一個曲兒。經濟看罷,慌的丢了買賣,跑到捲棚後面看。只見春梅回房,去對潘金蓮説了。不一時,也跑到捲棚下,兩個遇着,就如餓眼見瓜皮一般,禁不的一身直鑽到經濟懷裏來,捧着經濟臉,一連親了幾個嘴,咂的舌頭一片聲響道:「你負心的短命賊囚!自従我和你在屋裏,被小玉撞破了去後,如今一向都不得相會,這幾日你爺爺上東京去了,我一個兒坐炕上,淚汪汪只想着你,你難道耳根兒也不熱的?我仔細想來,你恁地薄情,便丢着也索罷休。只到了其間,又丢你不的。常言癡心女子負心漢,只你也全不留些情!」正在熱鬧處,不想那玉樓冷眼瞧破。忽然擡頭看見,順手一推,險些兒經濟跌了一跤。慌忙驚散,不題。

那日吳月娘、孟玉樓、李瓶兒,同一處坐地,只見玳安慌慌的跑進門來,見月娘磕了個頭,道:「爹回來了。小的一路騎頭口,拿着馬牌先行,因此先到家。爹這時節也差不上二十里遠近了。」月娘道:「你曾吃飯沒有?」玳安道:「従早上吃來,卻不曾吃中飯。」月娘便教玳安廚下吃飯去。又教整飯待大官人回來,自和六房姊妹同夥兒到廳上迎接。正是:詩人老去鶯鶯在,公子歸時燕燕忙。四人閒話多時,卻早西門慶到門前下轎了。衆妻妾一齊相迎進去。西門慶先和月娘廝見畢,然后孟玉樓、李瓶兒、潘金蓮,依次見了。西門慶和六房妻小各叙寒温。落后書童、琴童、畫童,也來磕了六房的頭,自去廚下吃飯。西門慶把路上辛苦,並到翟家住下多日,蔡太師厚情,與内相日日吃酒事情,備細説了一遍。因問李瓶兒:「孩子這幾時好麽?你身子怎地調理?吃的任醫官薬,有些應驗麽?我雖則往東京,一心只丢不下家事哩!店裏又不知怎樣,因此急忙回來。」李瓶兒道:「孩子也沒甚事,我身子,吃薬後畧覺好些。」月娘一面教衆人收好行李及蔡太師送的下程,一面做飯與西門慶吃。到晚,又設酒和西門慶接風。西門慶晚就在月娘房裏歇了,兩個是久旱逢甘雨,他鄉遇故知,懽愛之情都不必説。次日,陳經濟和大姐來廝見了,説了些店裏的帳目。應伯爵和常時節打聽的大官人來家,都來望。西門慶出門廝見畢,兩個一齊説:「哥哥一路辛苦。」西門慶便把東京富麗的事情,及太師管待情分,備細説了一遍,兩人只顧稱羡不已。當日西門慶留二人吃了一日酒,常時節臨起身,向西門慶道:「小弟有一事相求,不知哥可照顧麽?」説着只是低了臉,半含半吐。西門慶道:「但説不妨。」常時節道:「實為住的房子不方便,待要尋間房子安身,卻沒有銀子,因此要求哥周濟些兒。日後少不的加些利錢,送還哥哥。」西門慶道:「相處中説甚利錢!我如今忙忙地,那討銀子?且待到韓夥計貨船來家,自有個䖏。」説罷,常時節、應伯爵,作謝去了,不在話下。

且説苗員外自與西門慶相會在太師府前,便請了一席酒,席上又把兩個歌童許下了。那一日,西門慶歸心如箭,卻不曾作別的他,徑自歸來了。員外還道西門慶在京,伴當來翟家問着,那翟家説:「三日前西門大官家去了。」伴當回話,苗員外纔曉的,卻忖道:「君子一言,快馬一鞭。不送去罷,他不和我合着氣?只後邊説不的話了。」便叫過兩個歌童,吩咐道:「我前日請山東西門大官人,席上把你兩個許下他。如今他離東京回家去了,我目下就要送你們過去。你們早收拾包裹,待我捎下書打發你們。」那兩個歌童一齊跪告道:「小的們伏侍的員外多年了,卻為何今日閃的小的們不好?又不知西門大官人性格怎地,今日還要員外做主。」員外道:「你們卻不曉得,西門大官人家裏豪富潑天,金銀廣布,身居着右班武職,現在蔡太師門下做個乾兒子。就是内相、朝官,那個不與他心腹往來?家裏開着兩個綾緞舖,如今又要開個標行,進的利錢也委的無數。况兼他性格温柔,吟風弄月,家裏養着七八十個丫頭,那一個不穿綾着襖?後房裏擺着五六房娘子,那一個不插珠挂金?那些小優們,戲子們,個個借他錢鈔,服他差使;平康巷青水巷這些角伎,人人受他恩惠,這也不消説的。只是咱前日酒席之中,已把小的子許下他了。如今終不成改個口哩?」那歌童又説道:「員外這幾年上不知費盡多少心力,教的俺們彈唱哩。如今纔曉得些絃索,卻不㽞下自家歡楽,怎地倒送與別人快活?」説罷,不覺地撲簌簌哩掉下淚來。那員外也覺慘然不樂,説道:「小的子,你也説的是,咱也何苦定要是這等?只是『人而無信,不知其可也。』那孔聖人説的話,怎麽違得?如今也由不得你。待咱修書一封,差個伴當送你去,教他把隻眼兒好生看覷你們。你到那邊快活,也強似在我這裏一般。」就叫那門館先生寫着一封通候的八行書信,後面又寫那相送歌童,求他青目的語兒;又寫個禮單兒,把些尺頭書帕做個通問的禮兒。差了苗秀苗實,齎擎書信,護送兩個歌童。一霎時拴上了頭口,帶了被囊行李,直到山東西門慶家來。

那兩個歌童當時忍不住腮邊淚滴,又是主命難違,只得插燭也似磕了幾個頭,謝辭了員外,翻身上馬。迤邐行來,見那青山環馬首,綠水繞行鞭,酒簾深樹裏,草舍落霞前。止為那遏行雲歌聲絶代,不覺的辭恩主跋涉風煙。這兩個,思鄉念主,把那些檀板風流陽春白雪兒都忘卻;這兩個,忙投急趂,止思量早完公事,披星戴月的夜忘眠。正是:朝為苗府清歌客,暮作西門侑酒人。遠遠望見緑樹林中,挂着一個望子。那歌童道:「哥,走了這一日了,肚裏有些饑了,且吃盃酒兒去。」只見四個人兒滚鞍下馬,走入店中。那招牌上面寫的好,説:「神僊留玉佩,卿相解金貂。」眞個是好酒店也!四人坐下,喚過賣打上兩角酒來,攮個蔥兒蒜兒、大賣肉兒、豆腐菜兒,舖上幾碟,正待舒懷暢飲。忽地裏回頭看時,止見粉壁上飛白字寫着兩行,説道:「千里不為遠,十年歸未遲;總在乾坤内,何湏嘆别離!」正對着兩個歌童眼兒,不覺的薬賣有病的了,動人心處,撲簌簌流下兩行淚來,説道:「哥,我們隨着員外,指望一蒂兒到底。誰想酒席中間一言兩句,竟把我們送與别人。人離鄉賤,未知去後若何?」那苗秀苗實把好言知慰了一番,吃了飯,上馬又走。四個牲口,十六個蹄兒,端的是走的好。不多幾個日頭,就到東平府清河縣地面。四人拴了牲口,下馬訪問端的,一直地竟到紫石街西門慶家府裏投下。

卻説那西門慶自従東京到家,每日忙不迭,送禮的,請酒的,日日三朋四友。既要與大娘兒接風,又要與各房兒繾綣,朝朝殢雨尤雲,以此不曾到衙門裏去走,連那告假的帖兒也不曾消的。那日清閒無事,且到衙門裏升堂畫卯,把那些解到的人犯,也有姦情的、鬦毆的、賭賻的、竊盗的,一一重問一番。又把那些投到文書,一一押到日,僉押了一會。乘了一乘涼轎,幾個牢子喝道子簇擁來家。只見那苗秀苗實與那兩個歌童,已是候的久了,就跟着西門慶的轎子,隨到前廳,雙膝跪下稟説:「小的是揚州苗員外家人,有書拜候老爺。」磕個頭起在一邊。那西門慶擧個手,説道:「起來。」就把苗員外別來的行徑,寒暄的套語,問了一會。就叫書童把銀剪子剪開護封,拆了内涵封袋,打開副啟,細細看時,只見那苗秀苗實依先跪下,奉過那許多禮物説道:「這是俺員外一點孝心,求老爹俯納。」西門慶喜之不勝,連忙叫玳安收起禮物,請起苗秀苗實,説道:「我與你員外千里相逢,不想就蒙員外情投意合,十分相愛,就把歌童相許。那時酒中説話,咱也忘卻多時。因為那歸的忙促,不曾叩府辭别,正在想着。不意一諾千金,遠蒙員外記憶。我記得那古人交誼,止有那范張結契,千里相從,古今以為美談。如今你們那個員外,委的也是難得!」稱長道好,細細又感謝了一番。只見那兩個歌童,従新走過,又磕幾個頭,説道:「員外着小的們伏侍老爺,萬求老爺青目。」西門慶見兩個兒生得清秀,眞眞嫋嫋媚媚,雖不是兩節穿衣的婦人,卻勝似那唇紅齒白的妮子。歡天喜地,就請四位管家前廳茶飯。一面整辦厚禮,綾羅細軟,修書答謝員外,一面收拾房間,就叫兩個歌童在于書房伺候着。

只見那應伯爵諸人聞知此事,通來探望。西門慶就叫玳安裏邊討出菜蔬、嗄飯、點心、小酒,擺着八僊桌兒,就與諸人燕飲,就叫兩個歌童前來唱,只見捧着檀板,拽起歌喉,唱一個:

\begin{myquote}
{\markfont〔新水令〕}「小園昨夜放江梅,另一番動人風味。梨花迎笑臉,楊柳妬腰圍。試問荼䕷,開到海棠未?」

{\markfont〔駐馬聽〕}「野徑疎籬,陣陣香風來燕子;小園幽砌,紛紛晴雨過林西。芳心不與蝶潛知,暗香未許蜂先覺。闌遍倚,不知多少傷心處!」

{\markfont〔雁兒落带得勝令〕}「我則見碧陰陰西施鎖眉翠,紅點點鶗鴂抛珠淚;舞僊僊砑光帽上簪,虚飄飄金谷樓前墜。尚兀是芳氣襲人衣,豔質易霑泥。落䖏魚堪驚,飛來蝶欲迷。尋思,憑誰寄?還悲,花源未可期。」
\end{myquote}

那西門慶點着頭道:「果然唱得好!」那兩個歌童打個半跪兒,跪將下去道:「小的們還學得些小詞兒,一發歌與老爹聽。」西門慶説道:「這卻更好。」便教歌詞:

\begin{myquote}
「試裂齊紈,施鉛槧爰圖春牧。草淺淺細舖平野,散騎黄犢。一卷殘書牛背穩,數聲短笛煙光綠。想按圖題詠賦新詞,勞心曲。

文章妙,傳芸局;音調促,偕絲竹。倚清歌追和,〔陽春〕難續。一代風流誇好事,可堪膾炙人爭錄。羡先生想像賦《高唐》,情詞足。」
\end{myquote}

又:

\begin{myquote}
「畫出耕圖,郊原外東阡西陌。町疃曲,羣山環翠,岸塍聯絡。綠遍田疇多黍稌,麥旂纂纂蚕盈箔。彷彿有溪水繞柴門,山如削。

扶藜杖,徑丘壑;穿林藪,聽猿鶴。子耕耘妻饁,服勞耕作。喬木陰森流憩處,皤然捫腹舒雙足。羨先生想像詠《豳風》,村田楽。」
\end{myquote}

又:

\begin{myquote}
「寫就丹青,新圖好溪山環繞。隱隱遍,沙汀水岸,綠蘋紅蓼。一派秋光連浦溆,短簑篛笠煙波渺。看此時網得幾鮮鱗,鱸魚小。

漁唱起,飛鴻杳;江月白,歸雲少。倚蓬窻試覓,舊盟鷗鳥。借問忘機當日事,何如此際心情悄。羡先生想像詠《滄浪》,起塵表。」
\end{myquote}

又:

\begin{myquote}
「四野雲垂,冰花碎平舖茅屋。紅爐暖,妻煨山芋,自斟醽醁。課僕採薪去外戶,呼兒引鶴翻平陸。攬此景寫入畫圖中,娛心目。

鍾貴富,天之祿;懼盛滿,吾之欲。騁姸奇攄寫,好詞盈軸。愧我倡酬才思澀,輸他文采機關熱。羡先生想像楽桑楡,顏如玉。」
\end{myquote}

果然是聲遏行雲,歌成《白雪》,引的那後邊娘子們吳月娘、孟玉樓、潘金蓮、李瓶兒,都來聽着,十分歡喜。齊道:「唱的好。」只見潘金蓮在人叢裏,雙眼直射那兩個歌童,口裏暗暗低言道:「這兩個小夥子不但唱的好,就他容貌也標致的緊。」心下便已有幾分喜他了。當下西門慶打發兩個歌童東廂房安下,一面叫擺飯與苗秀苗實吃,一面整頓禮物回書,答謝苗員外。

畢竟未知何如,且聽下囬分解。

