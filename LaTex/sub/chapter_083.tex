\includepdf[pages={165,166},fitpaper=false]{tst.pdf}
\chapter*{第八十三囬 \\秋菊含恨泄幽情 春梅寄柬諧佳會}
\addcontentsline{toc}{chapter}{第八十三囬 秋菊含恨泄幽情 春梅寄柬諧佳會}
\markboth{第八十三囬 秋菊含恨泄幽情 春梅寄柬諧佳會}{第八十三囬 秋菊含恨泄幽情 春梅寄柬諧佳會}
\thispagestyle{empty}

\begin{myquote}
堪笑西門識未通,惹將桃李笑春風。

滿牀錦被藏賊睡,三頓珍羞養大蟲。

愛物只圖夫婦好,貪財常把丈人坑。

更有一件堪觀處,穿房入屋弄乾坤。
\end{myquote}

話說潘金蓮見陳經濟天明越牆過去了,心中又後悔。次日却是七月十五日,吴月娘坐轎子出門,往地藏庵薛姑子那裏,替西門慶燒盂蘭盆會箱庫去,金蓮衆人都送月娘到大門首囬來。孟玉樓、孫雪娥、西門大姐,都往後邊去了,獨金蓮落後,走到前廳儀門首,撞遇經濟,正在李瓶兒那邊樓上尋了解當庫衣物抱出來。金蓮呌住,便向他說:「昨日我說了你幾句,你如何使性兒今早就跳撲出來了,莫不真個和我罷了?」經濟道:「你老人家還說哩!一夜誰睡着來,險些兒一夜没曾把我麻犯死了!你看把我臉上肉也撾的去了!」婦人罵道:「賊短命,旣不與他有首尾,賊人膽兒虚,你平白走怎的?」經濟向袖中取出了紙帖兒來。婦人打開觀看,却是〈寄生草〉一詞,說道:

\begin{myquote}
「動不動將人罵,一徑把臉兒上撾。千般做小伏低下。但言語便要和咱罷。罷字兒說的人心怕。忘恩失義俏冤家,你眉兒淡了教誰畫?」
\end{myquote}

金蓮一見笑了,說道:「旣無此事,你今晚來後邊,我慢慢再問你。」經濟道:「乞你麻犯了人一夜,誰合眼兒來!等我白日裏睡一覺兒去。」婦人道:「待不去,和你算帳!」說畢婦人囬房去了。

經濟拿衣物往舖子裏來,做了一囬買賣。歸到廂房,歪在牀上睡了一覺。盼望天色晚來,要往金蓮那邊去。不想比及到黄昏時分,天氣一陣陰黑來,窗外簌簌下起雨來。正是:蕭蕭庭院黄昏雨,點點芭蕉不住聲。這經濟見那雨下得緊,說道:「好個不做羙的天!他甫能教我對證話去,今日不想又下起雨來,好悶倦人也。」於是長等短等,那雨不住,簌簌直下到初更時分,下的房簷上流水。這小郎君等不的雨住,披着一條茜紅氁子臥單在身上。那時吴月娘來家,大姐與元宵兒都在後邊没出來。於是鎖了房門,従西角門大雨裏走入花園金蓮那邊。推了推角門,——婦人知他今日晚必來,早已吩咐春梅,灌了秋菊幾鍾酒,同他在炕房裏先睡了,以此把角門虚掩。——這經濟推了推角門,見掩着,便挨身而入。進入婦人臥房,見紗窗半啟,銀蠟高燒,桌上酒菓已陳,金尊滿泛。兩個並肩疊股而坐。婦人便問:「你既不曾與孟三兒勾搭,這簪子怎得到你手裏?」經濟道:「本是我昨日在花園荼䕷架下拾的。若哄你,便促死促滅!」婦人道:「旣無此事,還把這根簪子與你關頭,我不要你的。只要把我與你的簪子香囊帕兒物事收好,若少了我一件兒,我與你答話。」兩個喫酒下棋,到一更方上牀就寢。顛鸞倒鳳,整狂了半夜。婦人把昔日西門慶枕邊風月,一旦盡付與情郎身上。

却說秋菊在那邊屋裏,夜間聽見這邊房裏恰似有男子聲音說話,更不知是那個了。到天明鷄叫時分,秋菊起來溺尿,忽聽那邊房内開的門響,朦朧月色,雨尚未止,打窗眼看,見一人披着紅臥單,従房中出去了,恰似陳姐夫一般:「原來夜夜和我娘睡!我娘自來人前會撇清,乾淨暗裏養着女婿!」次日逕走到後邊厨房裏,就如此這般對小玉說。不想小玉和春梅好,又告訴與春梅:「你那邊秋菊,說你娘養着陳姐夫,昨日在房裏睡了一夜,今早出去了。大姑娘和元宵又沒在前邊睡。」這春梅歸房,一五一十對婦人説:「娘不打與這奴才幾下!敎他騙口張舌,葬送主子就是一般!」金蓮聽了大怒,就叫秋菊到面前跪着,罵道:「敎你煎煎粥兒,就把鍋來打破了。你屁股大,掉了心也怎的?我這幾日沒曾打你,這奴才骨朶癢了!」於是拿棍子向他脊背上儘力狠抽了三十下,打的殺猪也似呌,身上都破了。春梅走將來說:「娘沒的打他這幾下兒,與他撾痒痒兒哩!旋剝了,呌將小廝來,拿大板子儘力砍與他二三十板,看他怕不怕!湯他這幾下兒,打水不渾的,只像鬦猴兒一般,他好小膽兒,你想他怕也怎的!做奴才,裏言不出,外言不入。都似這般,養出家生哨兒來了!」秋菊道:「誰說甚麽來?」婦人道:「還說嘴哩!賊破家誤主兒的奴才,還說甚麽!」幾聲喝的秋菊往廚下去了。正是:蚊蟲遭扇打,只為嘴傷人。

一日,八月中秋時分,金蓮夜間暗約經濟賞月飲酒,和春梅同下鱉棋兒。晚夕貪睡失曉,至茶時前後還未起來,頗露圭角。不想被秋菊睃到眼裏,連忙走到後邊上房,對月娘說。不想月娘正梳頭,小玉在上房門首。秋菊拉過他一邊,告他說:「俺姐夫如此這般,昨日又在我娘房裏歇了一夜,如今還未起來哩。前日為我告你説,打了我一頓。今日眞實看見,我須不賴他。請奶奶快去瞧去!」小玉罵道:「張眼露睛奴才,又來葬送主子!俺奶奶梳頭哩,還不快走哩!」月娘便問:「他說甚麽?」小玉不能隱諱,只說:「五娘使秋菊來請奶奶說話。」更不題出别的事。這月娘梳了頭,輕移蓮步,驀然來到前邊金蓮房門首。早被春梅看見,慌的先進來報與金蓮。金蓮與經濟兩個還在被窝内未起。聽見月娘到,兩個都喫了一驚,慌做手脚不迭。連忙藏經濟在牀身子裏,用一牀錦被遮蓋的嚴嚴的。教春梅放小桌兒在牀上,拿過珠花來,且穿珠花。不一時,月娘到房中坐下,說:「六姐,你這咱還不見出門,只道你做甚,原來在屋裏穿珠花哩。」一面㧱在手中觀看,誇道:「且是穿得好!正面芝蔴花,兩邊槅子眼方勝兒,周圍蜂赶菊。你看,着的珠子一個挨一個兒,凑的同心結且是好看。到明日,你也替我穿恁條箍兒戴。」婦人見月娘說好話兒,那心頭小鹿兒纔不跳了。一面令春梅:「倒茶來,與大娘喫。」少頃,月娘喫了茶,坐了囬去了,說:「六姐,快梳了頭,後邊坐。」金蓮道:「知道。」打發月娘出來,連忙攛掇經濟出港,往前邊去了。春梅與婦人整揑兩把汗。婦人說:「你大娘等閒無事,他不來我這屋裏來。無甚事,他今日大清早晨來做甚麽?」春梅道:「左右是咱家這奴才戳的來!」不一時,只見小玉走來,如此這般:「秋菊後邊說去,說姐夫在這屋裏,明睡到夜,夜睡到明。被我罵喝了他兩聲,他還不動。俺奶奶問,我沒的說,只說五娘請奶奶説話,方纔來了。你老人家只放在心裏,大人不見小人過,只隄防着這奴才就是了。」

看官聽說:雖是月娘不信秋菊說話,只恐金蓮少女嫩婦,沒了漢子,日久一時心邪,着了道兒。恐傳出去,被外人唇齒:西門慶為人一場,沒了多時光兒,家中婦人都弄的七顛八倒!恰似我養的這孩子,也來路不明一般。香香噴噴在家裏,臭臭烘烘在外頭。又以愛女之故,不教大姐遠出門,把李嬌兒廂房挪與大姐住,教他兩口兒搬進後邊儀門裏來。遇着傅夥計家去,教經濟輪番在舖子裏上宿。取衣物薬材,同玳安兒出入。各處門户都上了鎖鑰,丫鬟婦女無事不許往外邊去,凡事都嚴緊。這潘金蓮與經濟,兩個熱突突恩情都間阻了。正是:世間好事多間阻,就裏風光不久長。有詩為證:

\begin{myquote}
幾向天臺訪玉真,三山不見海沉沉。

侯門一入深如海,従此蕭郎是路人。
\end{myquote}

潘金蓮自被秋菊泄露姦情之後,月娘雖不見信,晚夕把各處門户都上了鎖,西門大姐搬進李嬌兒房中居住;經濟尋取薬材衣物,同玳安或平安眼同出入;二人恩情都間阻了,約一個多月不曾相會一處。金蓮每日難挨,繡帷孤枕,怎禁畫閣凄凉?未免害些木邊之目,田下之心,脂粉懶匀,茶飯頓减,帶圍寬褪,懨懨瘦損,每日只是思睡,扶頭不起。有春梅向前問道:「娘,你這兩日怎的不去後邊坐,或是往花園中散心走走?每日短嘆長吁,端的為些甚麽?」婦人道:「你不知道,我與你姐夫相交,有〈雁兒落〉為證:

\begin{myquote}
我與他好似並頭蓮一䖏生,比目魚纏成塊。初相逢熱似粘,乍離别難禁耐。好是怪奇哉,這兩日他不進來。大娘又把門上鎖,花園中狗兒乖。難猜,奴婢們ぢっ的怪;傷懷,這相思實難解!」
\end{myquote}

春梅道:「娘,你放心,不妨事!塌了天,還有四個大漢扶着哩。昨日大娘留下兩個姑子,今晚夕宣卷,後邊關的儀門早。晚夕,我推往前邊馬坊内,取草裝填枕頭,等我往前邊舖子裏叫他去。你寫下個柬帖兒與我㧱着,我好歹呌了姐夫,和娘會一面。娘心下如何?」婦人道:「我的好姐姐,你若肯可憐見,呌得他來,我恩有重報,不可有忘。我的病兒好了,替你做雙滿臉花鞋兒!」春梅道:「娘説的是那裏話!你和我是一個人,爹又沒了,你明日往前復進,我情願跟娘去,咱兩個還在一處。」婦人道:「你有此心,可知好哩。」婦人於是輕拈象管,欵拂花箋,寫就一個柬帖兒,彌封停當。到於晚夕,婦人先在後邊月娘前,假托心中不自在,得了個金蟬脱殼,歸到前邊房中。没事,月娘後邊儀門老早關了,丫鬟婦女都放出來聽尼僧宣卷。金蓮央及春梅,遞與他柬帖,說道:「好姐姐,你快些請他去。有〈河西六娘子〉為證:

央及春梅好姐姐,你放寬洪海量些。俺團圓,只在今宵夜。嗏,你把脚步兒快走些些,我這裏錦被兒重薰等待者!」

春梅道:「等我先把秋菊那奴才,與他幾鍾酒灌醉了,倒扣他在廚房内。我方拿了筐,推往前邊馬坊中取草來填枕頭,就叫他來。」於是篩了兩大碗酒,打發秋菊喫,好扣他在廚房内。拿了婦人柬帖兒出門。有〈雁兒落〉為證:

\begin{myquote}
我往馬坊中推取草,到前邊就把他來呌。歸來把狗兒藏,門上將鎖兒掏。尊前酒兒篩,牀上燈兒罩。帳煖度春霄,准備鳳鸞交。休教人知覺,把秋菊灌醉了。聽着,花影動知他到;今宵,管您兩個成就了!
\end{myquote}

春梅走到前邊,撮了一筐草,到印子舖門首呌門。正值傅夥計不在舖中,往家去了。獨有經濟在炕上,纔歪下。忽見有人呌門,問是那個。春梅道:「是你前世娘,散相思五瘟使!」經濟開門,見是他,滿臉笑道:「原來是小大姐!沒人,請裏面坐。」進入房内,見桌上點着燭,問小廝們在那裏,經濟道:「玳安和平安在那裏生薬舖中睡哩,獨我一個在此。受孤悽,挨冷淡,就是小生!」春梅道:「俺娘多上覆你好人兒:這幾日就門邊兒也不傍,往俺那屋裏走走去?說你另有了對門主顧兒了,不希罕俺娘兒們了!」經濟道:「那裏話!自従那日因些閒話,見大娘緊門緊户,所以不耐煩走動。」春梅道:「俺娘為你,這幾日心中好生不快。逐日無心無緒,茶飯懶喫,做事沒入脚處。今日大娘留他後邊聽宣卷也沒去,就來了。一心只是牽掛想你。巴巴使我捎寄了一柬帖在此,好歹教你快去哩!」這經濟接過柬帖,見封的甚密,拆開觀看,却是〈寄生草〉一詞,說道:

\begin{myquote}
「將奴這桃花面,只因你憔瘦損。不是因惜花愛月傷春困。則是因今春不减前春恨,常則是淚珠兒滴盡相思症。恨的是繡幃燈照影兒孤,盼的是書房人遠天涯近。」
\end{myquote}

經濟一見了此詞,連忙向春梅躬身深深地唱喏,說道:「多有起動起動!我並不知他不好,沒曾去看的你娘兒們,休怪休怪!你且先走一步,我收拾了,如今就去。」一面開橱門,取出一方白綾汗巾,一副銀三事挑牙兒答贈。和春梅兩個摟抱,按在炕上且親嘴咂舌,不勝歡謔。正是:無緣得會鶯鶯面,且把紅娘去解饞。有詩為證:

\begin{myquote}
淡畫眉兒斜插梳,不忺拈弄繡工夫。

雲窗霧閣深深處,靜拂雲箋學草書。

多豔麗更清姝,神僊標格世間無。

當初只說梅花似,細看梅花却不如。
\end{myquote}

當下兩個相戯了一囬,春梅先㧱着草歸到房來,一五一十對婦人說:「姐夫我呌了,他便來也!他看了你那柬帖兒,好不喜歡,與我深深作揖,與了我一方汗巾,一副銀挑牙兒相謝。」婦人便呌春梅:「你去外邊看看,只怕他來,休敎狗咬。」春梅道:「我把狗藏過一邊。」原來那時正值中秋八月十六七,月色正明。且說陳經濟旋到那邊生薬舖,叫過平安兒來這邊歇。他一個獵古調兒——前邊花園門關了,打後邊角門——走入金蓮那邊,搖木槿花為號。春梅隔牆看見花梢動,且連忙以咳嗽應之,報婦人。經濟推開門挨身進入到房中,婦人迎門接着,笑語說道:「好人兒,就不進來走走兒!」經濟道:「彼此怕是非,躲避兩日兒。不知你老人家不快,有失問候!」婦人道:「有〈四換頭〉詞為證:

\begin{myquote}
赤緊的因些閒話,把海樣恩情一旦差。你這兩日門兒不抹我心兒掛。關情的我兒你怎生便撇的下!」
\end{myquote}

兩個坐下,春梅關上角門,房中放桌兒擺上酒肴。婦人和經濟並肩疊股而坐,春梅打横,把酒來斟。傳盃換盞,倚翠偎紅,喫了一囬,擺下棋子,三人同下鱉棋兒。喫得酒濃上來,婦人嬌眼乜斜,烏雲半軃,取出西門慶淫器包兒,裏面包着相思套、顫聲嬌、銀托子、勉鈴,一弄兒淫器,教經濟使。在燈光影下,婦人便赤身露體,仰臥在一張醉翁椅兒上,經濟亦脱的上下沒條絲,也對坐一椅,拿春意二十四解本兒,在燈下着照樣兒行事。婦人便呌春梅:「你在後邊推着你姐夫,只怕他身子乏了。」春梅真個在身後推送,經濟那話,插入婦人牝中,往來抽送,十分暢美,不可盡言。

却表秋菊在後邊厨下,睡到半夜裏,起來淨手。見房門倒扣着,推不開。於是伸手出來,拔開了弔兒,大月亮地裏,躡足潛踪,走到前房窗下,潤破窗紙,望裏張看。見房中掌着明晃晃燈燭,三個喫的大醉,都光赤着身子正做得好:兩個對面坐着椅子,春梅便在後邊推車,三人串作一處。但見:

\begin{myquote}
一個不顧夫主名分,一個那管上下尊卑。一個氣喘吁吁,猶如牛吼柳影;一個嬌聲嚦嚦,猶似鶯嘲花間。一個椅上逞雨意雲情,一個耳畔説山盟海誓。一個寡婦房内,翻為快活道場;一個丈母跟前,變作行淫世界。一個把西門慶枕邊風月,盡付與嬌婿;一個將韓壽偸香手段,悉送與情娘。正是:寫成今世不休書,結下來生歡喜帶。
\end{myquote}

當時都被秋菊看到眼裏,口中不說,心中暗道:「他們還只在人前撇清,要打我,今日却真實被我看見了。到明日對大娘說,莫非又說騙口張舌,賴他不成!」於是瞧了個不亦楽乎,依舊還往厨房中睡去了。

三個整狂到三更時分纔睡。春梅未曾天明先起來,走到厨房,見厨房門開了,便問秋菊。秋菊道:「你還說哩!我尿急了,往那裏溺?我拔開了弔,出來院子裏溺尿來。」春梅道:「成精奴才,屋裏放着榪子,溺不是?」秋菊道:「我不知榪子在屋裏!」兩個後邊聒譟。經濟天明起來,早往前邊去了。正是:兩手劈開生死路,翻身跳出是非門。婦人便問春梅:「後邊亂甚麽?」這春梅如此這般,告説秋菊夜裏開門一節。婦人發恨要打秋菊。這秋菊早晨又走來後邊,報與月娘知道。被月娘喝了一聲,罵道:「賊葬弄主子的奴才!前日平空走來輕事重報,說他主子窝藏陳姐夫在屋裏,明睡到夜,夜睡到明。呌了我去。他主子正在牀上放炕桌兒穿珠花兒,那得陳姐夫來?落後陳姐夫打前邊來。恁一個弄主子的奴才!一個大人放在屋裏,端的是糖人兒,木頭兒,不拘那裏安放了?一個漢子,那裏發落,莫非放在ず眼裏面不成?傳出去,知道的,是你這奴才們葬送主子;不知道的,只說西門慶平昔要的人強佔多了,人死了多少時兒,老婆們一個個都弄的七顛八倒!恰似我的這孩子,也有些基根兒不正一般。」於是要打秋菊,唬的秋菊往前邊疾走如飛,再不敢來後邊說去了。婦人聽見月娘喝出秋菊,不信其事,心中越發放下膽子來了。於是與經濟作一詞以自快,有〈紅繡鞋〉為證:

\begin{myquote}
會雲雨風般疎透,閒是非屁似休瞅。那怕無縫鎖上十字扭。輪鍬的閃了手腕,散楚的叫破咽喉。咱兩個關心的情越有!
\end{myquote}

西門大姐聽見此言,背地裏審問陳經濟,經濟道:「你信那汗邪了的奴才!我昨日見在舖子上宿,幾時往花園那邊去了?花園門成日又關着。」西門大姐罵道:「賊囚根子,你别要說嘴!你若有風吹草動到我耳朶内,惹娘說我,你就信信脱脱去了,再也休想在這屋裏了!」經濟道:「是非終日有,不聽自然無。怪不的說舌的奴才到明日得不了好,大娘眼見,不信他!」西門大姐道:「得你這般說就好了。」正是:誰料郎心輕似絮,那知妾意亂如絲。

畢竟未知後來何如,且聽下囬分解。

