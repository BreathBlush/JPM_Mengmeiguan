\includepdf[pages={183,184},fitpaper=false]{tst.pdf}
\chapter*{第九十二囬 \\陳經濟被陷嚴州府 吳月娘大鬧授官廳}
\addcontentsline{toc}{chapter}{第九十二囬 陳經濟被陷嚴州府 吳月娘大鬧授官廳}
\markboth{{\titlename}卷之十}{第九十二囬 陳經濟被陷嚴州府 吳月娘大鬧授官廳}


\begin{myquote}
暑往寒來春復秋,夕陽西下水東流。

雖然富貴皆由命,運去貧窮亦有由。

事遇機關須進步,人逢得意早囘頭。

將軍戰馬今何在,野草閑花滿地愁。
\end{myquote}

話説當日李衙内打了玉簪兒一頓,即時呌了陶媽媽來,領出賣了八兩銀子,買了個十八歲使女,名喚滿堂兒上竈。不在話下。

卻表陳經濟,自従西門大姐來家,交還了許多牀帳粧奩,箱籠家伙,三日一場嚷,五日一場鬧,問他娘張氏要本錢做買賣。他母舅張團練,來問他母親借了五十兩銀子,復謀管事,被他喫醉了,往在張舅門上罵嚷,他張舅受氣不過,另問别處借了銀子,幹成管事,還把銀子交還將來。他母親張氏,着了一場重氣,染病在身,日逐臥牀不起,終日服薬,請醫調治。喫他逆慪不過,兌出二百兩銀子交他,叫陳定在家門首打開兩間房子,開布舖做買賣。逐日結交朋友陸三郎楊大郎,狐朋狗黨,在舖中彈琵琶、抹骨牌、打雙陸、喫半夜酒,看看把本錢弄下去了。陳定對張氏說:「他每日飲酒花費。」張氏聽信陳定言語,不託他。經濟反說陳定染布去尅落了錢,把陳定兩口兒攆出來外邊居住,卻搭了楊大郎做夥計。這楊大郎名喚楊光彦,綽號為鐵指甲,專一糶風賣雨,架謊鑿空,撾着人家本錢就使。他祖貫係沒州脱空縣拐帶村無底鄉人氏,他父親呌做楊不來,母親白氏,他兄弟叫楊二風。他師父是崆峒山拖不洞火龍庵精光道人,那裏學的謊。他渾家是沒驚着小姐,生生喫謊唬死了。他許人話如捉影撲風,騙人財似探囊取物。這經濟問娘又要出三百兩銀子來添上,共凑了五百兩銀子,信着他往臨清販布去。

這楊大郎到家收拾行李,沒底兒褡褳裝着些軟斯金楡錢兒,拿一張黑心鵰弓,騎一疋白眼龍馬,跟着經濟従家中起身,前往臨清馬頭上尋缺貨去。三里抹過沒州縣,五里來到脫空村,有日到於臨清。這臨清閘上,是個熱鬧繁華大馬頭去處,商賈往來,船隻聚會之所,車輛輻輳之地,有三十二條花柳巷,七十二座管絃樓。這經濟終是年小後生,被這鐵指甲楊大郎領着遊娼樓,串酒店,每日睡睡,終宵蕩蕩,貨物倒販得不多。因走在一娼樓館上,見了一個粉頭,名喚馮金寳,生的風流俏麗,色藝雙全。問青春多少,鴇子說:「姐兒是老身親生之女,止是他一人掙錢養活,今年青春纔交二九一十八歲。」經濟一見,心目蕩然,與了鴇子五兩銀子房金,一連和他歇了幾夜,楊大郎見他愛這粉頭,留連不捨,在旁花言說念,就要娶他家去。鴇子開口要銀一百五十兩,講到一百兩上,兌了銀子,娶到來家。一路上擡着,楊大郎和經濟押着貨物車走。一路上揚鞭走馬,那樣懽喜,正是:

\begin{myquote}
多情燕子樓,馬足空囘首。

載得武陵春,陪作鸞鳳友。
\end{myquote}

他娘張氏,見經濟貨倒販得不多,把本錢倒娶了一個唱的來家,又着了口重氣,嗚呼哀哉,斷氣身亡。這經濟不免買棺裝殮,念經做七,停放了一七光景,發送出門,祖塋合葬。他母舅張團練看他娘面上,亦不和他一般見識。這經濟坟上覆墓囘來,把他娘正房三間,中間供養靈位,那兩間收拾與馮金寳住,大姐倒住着耳房。又替馮金寳買了丫頭重喜兒伏侍。門前楊大郎開着舖子,家裏大酒大肉買與唱的喫。每日只和唱的睡,把大姐丢着不去瞅睬。

一日,打聽孟玉樓嫁了李知縣兒子李衙内,帶過許多東西去。三年任滿,李知縣陞在浙江嚴州府,做了通判,領憑起身,打水路赴任去了。這陳經濟因想起昔日在花園中拾了孟玉樓那根簪子,喫醉又被金蓮所得,落後還與了他收到如今。就把這根簪子做個證見把柄,趕上嚴州去,只說玉樓先與他有了姦,與了他這根簪子,不合又帶了許多東西嫁了李衙内,都是昔日楊戩寄放金銀箱籠應沒官之物,「那李通判一個文官,多大湯水,聽見這個利害聲口,不怕不敎他兒子雙手把老婆奉與我。我那時取將來家,與馮金寳又做一對兒,落得好受用。」正是:計就月中擒玉兔,謀成日裏捉金烏。經濟不來倒好,此這一來,正是:失曉人家逢五道,溟泠餓鬼撞鍾馗。有詩為證:

\begin{myquote}
趕到嚴州訪玉人,人心難忖似石沉。

侯門一入深如海,従此蕭郎落陷坑。
\end{myquote}

卻說一日,陳經濟打點他娘箱中,尋出一千兩金銀。留下一百兩與馮金寳家中盤纏,把陳定復叫進來看家,幷門前舖子發賣零碎布疋。他與楊大郎又帶了家人陳安,押着九百兩銀子,従八月中秋起身,前往湖州販了半船絲綿紬絹,來到清江浦江口馬頭上,灣泊住了船隻,投在個店主人陳二店内。夜間點上燈光,教陳二郎殺鷄取酒,與楊大郎共飲。飲酒中間,和楊大郎說:「夥計,你暫且看守船上貨物,在二郎店内畧住數日。等我和陳安㧱些人事禮物,往浙江嚴州府看家姐,嫁在府中。多不上五日,少只三日期程就來。」楊大郎道:「哥去只顧去,兄弟情願店中等候。哥到日一同起身。」

這陳經濟千不合萬不合,和陳安身邊帶了些銀兩、人事禮物,有日取路逕到嚴州府。進入城内,投在寺中安下。打聽李通判到任一個月,家小船隻纔到三日光景。這陳經濟不敢怠慢,買了四盤禮物,兩疋紵絲尺頭,兩罈酒,陳安押着。他便揀選衣帽齊整,眉目光鮮,逕到府衙門前,與門吏作揖道:「煩報一聲,說我是通判李老爹衙内新娶娘子的親,孟二舅來探望。」這門吏聽了,不敢怠慢,隨即稟報進去。衙内正在書房中看書,聽見是婦人兄弟,令左右先把禮物擡進來,一面忙整衣冠道:「有請!」把陳經濟請入府衙廳上,敘禮分賓主坐下,說道:「前日做親之時,怎的不會二舅?」經濟道:「在下因在川廣販貨,一年方囘,不知家姐嫁與府上,有失親近。今日敬備薄禮來看看家姐。」李衙内道:「一向不知,失禮,恕罪恕罪!」須臾,茶湯已罷,衙内令左右:「把禮帖幷禮物取進去,對你娘說:二舅來了。」孟玉樓正在房中坐的,只聽小門子進來報說:「孟二舅來了。」玉樓道:「一二年不曾囘家,再有那個孟舅?莫不是我二哥孟鋭來家了,千山萬水來看我?」只見伴當拿進禮物和帖兒來,上面寫着「眷生孟銳」,就知是他兄弟,一面道:「有請!」令蘭香收拾後堂乾凈。玉樓裝點打扮,伺候出見。只見衙内讓進來,玉樓在簾内觀看,可霎作怪,不是他兄弟,卻是陳姐夫:「他來做甚麽?等我出去,見他怎的說話。常言親不親故鄉人,羙不羙鄉中水。雖然不是我兄弟,也是我女婿人家。」一面整裝出來拜見。那經濟說道:「一向不知姐姐嫁在這裏,沒曾看得。」正說得這句,不想門子來請衙内,外邊有客來了。這衙内吩咐玉樓:「管待二舅!」就出去待客去了。

玉樓見經濟磕下頭,連忙還禮,說道:「姐夫免禮。那陣風兒刮你到此處?」叙畢禮數,讓坐,叫蘭香看茶出來。喫了茶,彼此敍了些家常話兒,玉樓因問:「大姐好麽?」經濟就把従前西門慶家中出來,幷討箱籠的一節話,告訴玉樓。玉樓又把清明節上坟,在永福寺遇見春梅在金蓮坟上燒紙的話告訴他。又說:「我那時在家中,也常勸你大娘:疼女兒,就疼女婿;親姐夫,不曾養活了外人。他聽信小人言語,把姐夫打發出來,落後姐夫討箱子,我就不知道。」經濟道:「不瞞你老人家說,我與六姐相交,誰人不知!生生喫他信奴才言語,把他打發出去,纔乞武松殺了。他若在家,那武松有七個頭八個膽,敢往你家來殺他?我這仇恨,結的有海來深。六姐死在陰司裏也不饒他!」玉樓道:「姐夫也罷,丢開了手的事!自古寃仇只可解,不可結。」說話中間,丫鬟放下桌兒,擺上酒來,盃盤餚品,堆滿春檯。玉樓斟上一盃酒,雙手遞與經濟,說:「姐夫遠路風塵,無事破費,且請一盃兒水酒。」這經濟用手接了,唱了喏,亦斟一盃囘奉婦人,敍禮坐下。因見婦人姐夫長姐夫短呌他,口中不言,心内暗道:「這淫婦怎的不認範,只呌我姐夫?等我慢慢的探他。」

當下酒過三巡,餚添五道,彼此言來語去,說得入港。這經濟酒蓋着臉兒,——常言酒情深似海,色膽大如天。見無人在跟前,先丢的幾句邪言說入去,說道:「我兄弟思想姐姐,如渴思漿,如熱思涼,想當初在丈人家,怎的在一處下棋抹牌,同坐雙雙,似油瓶蓋一般。誰承望今日各自分散,你東我西!」玉樓笑道:「姐夫好說。自古清者清而渾者渾,久而自見。」這經濟笑嘻嘻向袖中取出一包雙人兒的香茶,遞與婦人説:「姐姐,你若有情,可憐見兄弟,喫我這個香茶兒。」説着,就連忙跪下。那婦人登時一點紅従耳畔起,把臉飛紅了,一手把香茶包兒掠在地下,說道:「好不識人敬重!奴好意遞酒與你喫,倒戯弄我起來!」就撇了酒席,往房裏去了。經濟見他不就範,一面拾起香茶來,發話道:「我好意來看你,你倒變了卦兒。你敢說你嫁了通判兒子,好漢子不睬我了!你當初在西門慶家做第三個小老婆,沒曾和我兩個有首尾?」因向袖中取出舊時那根金頭銀簪子,拿在手内說:「這個物是誰人的?你旣不和我有姦,這根簪兒怎落在我手裏?上面還刻着『玉樓』名字。!你和大老婆串同了,把我家寄放的八箱子金銀細軟玉帶寳石東西——都是當朝楊戩寄放應沒官之物,都帶來嫁了漢子。我敎你不要謊,到八字八は兒上和你答話!」玉樓見他發話,拿的簪子委的他頭上戴的金頭蓮瓣簪兒,「昔日在花園中不見,怎的落在這短命手裏?」恐怕嚷的家下人知道,須臾變作笑吟吟臉兒,走將出來,一把手拉經濟說道:「好姐夫,奴鬦你耍子,如何就惱起來?」因觀看左右無人,悄悄說:「你旣有心,奴亦有意。」兩個不由分說,摟着就親嘴。這陳經濟把舌頭似蛇吐信子一般,就舒到他口裏,教他咂。說道:「你叫我聲親親的丈夫,纔算你有我之心。」婦人道:「且禁聲,只怕有人聽見。」經濟悄悄向他說:「我如今治了半船貨,在清江浦等候。你若肯下顧時,如此這般,到晚夕假扮門子私走出來,跟我上船家去,成其夫婦,有何不可?他一個文職官,怕是非,莫不敢來找尋你不成?」婦人道:「旣然如此,也罷。」約會下:「你今晚在府牆後等着,奴有一包金銀細軟,打牆上繫過去,與你接了。然後奴纔扮做門子,打門裏出來,跟你上船去罷。」

看官聽說:正是佳人有意,那怕粉牆高萬丈;紅粉無情,總然共坐隔千山!當時孟玉樓若嫁得個癡蠢之人,不如經濟,經濟便下得這個鍬鐝着。如今嫁了李衙内,有前程,又是人物風流,青春年少,恩情羙滿,他又勾你做甚?休說平日又無連手。這個郎君,也是合當倒運,就吐實話泄機與他,倒喫婆娘哄賺了。正是:花枝葉下猶藏刺,人心難保不懷毒。

當下二人會下話。這經濟喫了幾盃酒,少頃,告辭囘去。李衙内連忙送出府門,陳安跟隨而去。衙内便問婦人:「你兄弟住那裏下處?我明日囘拜他去,送些嗄程與他。」婦人便説:「那裏是我兄弟,他是西門慶家女婿。如此這般,來勾搭,要拐我出去。奴已約下他,今晚夜至三更,在後牆相等。咱好不好將計就計,把他當賊拿下,除其後患如何?」衙内道:「叵耐這廝無端!自古無毒不丈夫,不是我去尋他,他自來送死!」一面走出外邊,叫過左右伴當心腹快手,如此這般,預備去了。

這陳經濟不知機變,至半夜三更,果然帶領家人陳安,來府衙後牆下,以咳嗽為號。只聽牆内玉樓聲音,打牆上掠過一條索子去,那邊繫過一大包銀子來。原來是庫内拿的二百兩贓罰銀子。這經濟纔待教陳安拿着走,忽聽一聲梆子響,黑影裏閃出四五條漢,呌聲:「有賊了!」登時把經濟連陳安都綁了。禀知李通判,吩咐都且押送牢裏去,明日問理。

原來嚴州府正堂知府姓徐,名喚徐葑,係陝西臨洮府人氏,庚戍進士,極是個清廉剛正之人。次日早升堂,左右排兩行官吏。這李通判上去畫了公座,庫子呈禀賊情事,帶陳經濟上去說:「昨夜至三更時分,有先不知名、今知名賊人二名陳經濟陳安,鍬開庫門鎖鑰,偸出贓銀二百兩,越牆而過,致被捉獲,來見老爺。」徐知府喝令:「帶上來!」把陳經濟並陳安揪簇採擁,驅至當廳跪下。知府見經濟年小清俊,便問:「這廝是那裏人氏?因何來我這府衙公廨,夜晚做賊,偸盗官庫贓銀數多,有何理說?」那陳經濟只顧磕頭聲寃。徐知府道:「你做賊如何聲寃?」李通判在旁欠身便道:「老先生不必問他,眼見得贓證明白,何不加起刑來!」徐知府即令左右拿下去打二十板。李通判道:「人是苦虫,不打不成。不然,這賊便要展轉。」當下兩邊皂隸,把經濟陳安拖翻,大板打將下來。這陳經濟口内只罵:「誰知淫婦孟三兒陷我至此,寃哉,苦哉!」這徐知府終是黄堂出身官人,聽見這一聲,必有緣故,纔打到十板上,喝令:「住了!且收下監去,明日再問。」李通判道:「老先生不該發落他。常言人心似鐵,官法如爐,従容他一夜不打緊,就翻異口詞。」徐知府道:「無妨,吾自有主意。」當下獄卒把經濟陳安押送監中去訖。

這徐知府心中有些疑忌,即喚左右心腹近前,如此這般,「下監中探聽經濟所犯來歷,即便囘報。」這幹事人假扮做犯人,和經濟晚間在一㭱上睡,問其所以:「我看哥哥青春年少,不是做賊的。今日落在此刑憲,打屈官司!」經濟便說:「一言難盡。小人本是清河縣西門慶女婿,這李通判兒子新娶的婦人孟氏,是俺丈人的小,舊與我有姦的,今帶過我家老爺楊戩寄放十箱金銀寳玩之物來他家,我來此間問他索討,反被他如此這般欺負,把我當賊拿了。苦打成招,不得見其天日,是好苦也!」這人聽了,走來退廳,告報徐知府。知府道:「如何?我說這人聲寃叫孟氏,必有緣故。」

到次日升堂,官吏兩旁侍立,這徐知府把陳經濟陳安提上來,摘了口詞,取了張無事的供狀,喝令釋放。李通判在旁邊不知,還再三說:「老先生,這廝賊情旣的,不可放他。」反被徐知府對佐貳官儘力數說了李通判一頓說:「我居本府正官,與朝廷幹事,不該與你家官報私仇,誣陷平人作賊!你家兒子娶了他丈人西門慶妾孟氏,帶了許多東西,應没官贓物金銀箱籠來。他是西門慶女婿,逕來索討前物,你如何假揑賊情,拿他入罪,教我替你家出力?做官養兒養女也要長大,若然如此,公道何堪?」當廳把李通判數說的滿面羞,垂首喪氣而不敢言。陳經濟與陳安便釋放出去了,良久,徐知府退廳。

這李通判囘到本宅,心中十分焦燥。夫人便問:「相公每常退衙歡天喜地,今日這般心中不快,何說?」那李通判大喝一聲:「你女婦人家,曉得甚麽!養的好不肖子,今日喫徐知府當堂對衆同僚官吏,儘力上落了我一頓,可不氣殺我也!」夫人慌了,便問甚麽事。李通判即把兒子呌到跟前,喝令左右:「拿大板子來,氣殺我也!」說道:「你當初為娶這個婦人來家,今時他家女婿因這婦人帶了許多裝奩金銀箱籠,口口聲聲稱是當朝逆犯楊戩寄放應沒官之物,來問你要。說你假盜出庫中官銀,當賊情拿他。我通一字不知,反被正宅徐知府對衆數說了我這一頓。此是我頭一日官未做,你照顧我的。我要你這不肖子何用!」即令左右,雨點般大板打將下來。可憐打得這李衙内皮開肉綻,鮮血迸流。夫人見打得不像模樣,在旁哭泣勸解。孟玉樓又在後廳角門首掩淚潛聽。當下打了三十大板,李通判吩咐左右押着衙内,「即時與我把婦人打發出門,令他任意改嫁,免惹是非,全我名節。」那李衙内心中怎生捨得離異,只顧在父母跟前哭啼哀告:「寜把兒子打死爹爹跟前,並捨不的婦人。」李通判把衙内用鐵索墩鎖在後堂,不放出去。只要囚禁死他。夫人哭道:「相公,你做官一塲,年紀五十餘歲,也只落得這點骨肉。不爭為這婦人,你囚死他,往後你年老休官,倚靠何人?」李通判道:「不然。他在這裏,須帶累我受人氣。」夫人道:「你不容他在此,打發他兩口兒上原籍眞定府家去便了。」通判依聽夫人之言,放了衙内,限三日就起身。打點車輛,同婦人歸棗強縣家裏攻書去了。

卻表陳經濟與陳安出離嚴州府,到寺中取了行李,逕往清江浦陳二店中來尋楊大郎。陳二說:「三日前往府前尋你去,說你監在牢中,他收拾了貨船,起身往家中去了。」這經濟未信,向河下覓船隻,撲了空,說道:「這天殺的,如何不等我來就起身去了!」况新打監中出來,身邊盤纏已無,和陳安不免搭在人船上,把衣衫解當,討喫歸家。忙忙似喪家之犬,急急如漏網之魚,隨路找尋楊大郎,並無踪跡。那時正値秋暮天氣,樹木凋零,金風搖落,甚是凄凉。有詩八句,單道這秋天行人最苦:

\begin{myquote}
柄柄芰荷枯,葉葉梧桐墜。

蛩鳴腐草中,鴈落平沙地。

細雨濕青林,霜重寒天氣。

不是路行人,怎曉秋滋味。
\end{myquote}

有日經濟到家,陳定正在門首,看見經濟來家,衣衫襤褸,面貌黧黑,唬了一跳。接到家中,問貨船到於何處。經濟氣得半日不言,把嚴州府遭官司一節說了,「多虧正宅徐知府放了我,不然性命難保。今被楊大郎這天殺的,把我貨物不知拐的往那裏去了。」先使陳定往他家探聽,他家說還不曾來家。陳經濟又親去問了一遭,並沒下落,心中着慌。走入房來,那馮金寳又和西門大姐扭南面北。自従經濟出門,兩個合氣直到如今。大姐便說:「馮金寳拿着銀子錢,轉與他鴇子去了,他家保兒成日來,瞞藏背掖,打酒買肉在屋裏喫。家中要的沒有,睡到晌午,諸事兒不買,只熬俺們。」馮金寳又說:「大姐成日横草不拈,竪草不動,偸米換燒餅喫。又把煮的醃肉,偸在房裏和丫頭元宵兒同喫。」這陳經濟就信了,反罵大姐:「賊不是材料淫婦!你害饞癆饞痞了,偸米出去換燒餅喫?又和丫頭打夥兒偸肉喫!」把元宵兒打了一頓,把大姐踢了幾脚。這大姐急了,趕着馮金寳兒撞頭,罵道:「好養漢的淫婦!你抵盗的東西與鴇子不値了,倒學舌與漢子説我偸米偸肉!犯夜的倒拿住巡更的了,教漢子踢我!我和你這淫婦擯兑了罷,要這命做甚麽!」這經濟道:「好淫婦,你擯兑他?你還不値他個脚指頭兒哩!」也是合當有事,祸便是這般起——於是一把手採過大姐頭髮來,用拳撞、脚踢、拐子打,打得大姐鼻口流血,半日甦醒過來。這經濟便歸唱的房裏睡去了,由着大姐在下邊房裏嗚嗚咽咽只顧哭泣。元宵兒便在外間睡着了。可憐大姐到半夜,用一條索子懸梁自縊身死,亡年二十四歲。

到次日早晨,元宵起來,推裏間不開。上房經濟和馮金寳還在被窝裏,使他丫頭重喜兒來呌大姐門,取木盆洗坐脚,只顧推不開。經濟還罵:「賊淫婦,如何還睡?這早晚不起來!我這一跥開門進去,把淫婦鬢毛都拔淨了。」重喜兒打窗眼内望裏張看,說道:「他起來了,且在房裏打鞦韆耍子兒哩!」又說:「他提偶戯耍子兒。」只見元宵瞧了半日,叫道:「爹,不好了,俺娘吊在牀頂上吊死了!」這小郎纔慌了,和唱的齊起來,跥開房門,向前解卸下來,灌救了半日,那得口氣兒來?原來不知多咱時分,嗚呼哀哉死了。正是:不知眞性歸何處,疑在行雲秋水中。

陳定聽見大姐死了,恐怕連累,先走去西門慶家中報知月娘。月娘聽見大姐吊死了,經濟娶唱的在家!正是:冰厚三尺,不是一日之寒,率領家人小廝丫鬟媳婦,七八口往他家來。見了大姐屍首吊的直挺挺的,哭喊起來,將經濟拿住,揪採亂打,渾身錐子眼兒也不計數。唱的馮金寳躱在牀底下,採出來也打了個臭死。把門窗户壁都打得七零八落,房中牀帳裝奩都還搬的去了。歸家請將吳大舅二舅來商議。大舅說:「姐姐,你趂此時咱家人死了不到官,到明日他過不的日子還來纏要箱籠!人無遠慮,必有近憂。不如到官處斷開了,庶杜絶後患。」月娘道:「哥見得是。」一面寫了狀子。次日,月娘親自出官,來到本縣授官廳下,遞上狀去。原來新任知縣姓霍,名大立,湖廣黄崗縣人氏,舉人出身,為人鯁直。聽見係人命重事,即升廳受狀。見狀上寫着:

\begin{myquote}[\markfont]
「告狀人吳氏,年三十四歲,係已故千户西門慶妻。狀告為惡婿欺凌孤孀,聽信娼婦,熬打逼死女命,乞憐究治,以存殘喘事。比有女婿陳經濟,遭官事投來氏家,潛住數年。平日喫酒行兇,不守本分,打出吊入。是氏懼法,逐離出門。豈期經濟懷恨,在家將氏女西門氏,時常熬打,一向含忍。不料伊又娶臨清娼婦馮金寳來家,奪氏女正房居住,聽信唆調,將女百般痛辱熬打,又採去頭髮,渾身踢傷。受忍不過,比及將死。於本年八月廿三日三更時分,方纔將女上吊縊死。若不具告,切思經濟恃逞兇頑,欺氏孤寡,聲言還要持刀殺害等語,情理難容。乞賜行拘到案,嚴究女死根因,盡法如律。庶兇頑知警,良善得以安生,而死者不為含寃矣!為此,具狀上告

本縣青天老爺 施行。」
\end{myquote}

這霍知縣在公座上看了狀子,又見吳月娘身穿縞素,腰繫孝裙,係五品職官之妻,生的容貌端莊,儀容閑雅,欠身起來説道:「那吳氏起來,我據看你也是個命官娘子,這狀上情理,我都知了。你請囘去,不必在這裏。今後只令一家人在此伺候就是了。我就出牌去拿他。」那吳月娘連忙拜謝了知縣,出來坐轎子囬家,委付來昭廳下伺候。須臾批了呈狀,委的兩個公人,一面白牌,行拘陳經濟娼婦馮金寳,並兩隣保甲,正身赴官聽審。這經濟正在家裏亂丧事,聽見月娘告下狀來,縣中差公人發牌來㧱他,唬的魂飛天外,魄喪九霄。那馮金寳已被打的渾身疼痛,睡在牀上,聽見人拿他,唬的勢不知有無。陳經濟沒高低使錢,打發公人喫了酒飯,一條䋲子連唱的都拴到縣裏,左隣范綱,右隣孫紀,保甲王寬兒。霍知縣聽見拿了人來,即時升廳。來昭跪在上首,陳經濟馮金寳一行人跪在階下。知縣看了狀子,便叫經濟上去説:「你是陳經濟?」又問那是馮金寳。那馮金寳道:「小的是馮金寳。」知縣因問經濟:「你這廝可惡!因何聽信娼婦,打死西門氏,方今上吊?有何理説?」經濟磕頭告道:「望乞青天老爺察情。小的怎敢打死他?因為搭夥計在外,被人坑陷了資本,着了氣來家,問他要飯喫,他不曾做下飯,委被小的踢了兩脚。他到半夜,自縊身死了。」知縣喝道:「你旣娶下娼婦,如何又問他要飯喫?尤說不通。吳氏狀上說你打死他女兒,方纔上吊,你還不招認?」經濟道:「吳氏與小的有仇,故此誣賴小的,望老爺察情。」知縣大怒,說:「他女兒現死了,還推賴那個!」喝令左右:「拿下去,打二十大板!」提馮金寳上來,拶了一拶,敲一百敲,令公人帶下收監。次日,委典史臧不息,帶領吏書保甲隣人等,前至經濟家,擡出屍首當場檢驗。身上都有青傷,脖項間亦有䋲痕,生前委因經濟踢打傷重,受忍不過,自縊身死。取供具結,塡圖解繳,囘報縣中。知縣大怒,褪衣又打了經濟金寶十板,問陳經濟夫毆妻至死者絞罪;馮金寳遞决一百,發囘本司院當差。

這陳經濟慌了,監中寫出帖子,對陳定説:「把布舖中本錢,連大姐頭面,共凑了一百兩銀子,暗暗送與知縣。」知縣一夜把招卷改了,止問了個逼令身死,係雜犯,准徒五年,運灰贖罪。吳月娘再三跪門哀告。知縣把月娘叫上去,説道:「娘子,你女兒項上見䋲痕,如何問他毆殺條律?人情莫非忒偏向麽?你怕他後邊纏擾你,我這裏替你取了他杜絕文書,令他再不許上你門就是了。」一面把經濟提到跟前,吩咐道:「我今日饒你一死,務要改過自新,不許再去吳氏家纏擾。再犯到我案下,决然不饒!即便把西門氏買棺裝殮,發送葬埋來囘話。我這裏好申文書往上司去。」這經濟得了個饒,交納了贖罪銀子,歸到家中,抬屍入棺,停放一七,念經送葬埋城外。前後坐了半個月監,使了許多銀兩,唱的馮金寳也去了,家中所有的都乾凈了,房兒也典了,剛刮剌出個命兒來,再也不敢聲言丈母了。正是:祸福無門人自招,須知楽極有悲來。有詩為證:

\begin{myquote}
風波平地起蕭牆,義重恩深不可忘。

水溢藍橋應有會,雙星權且作參商。
\end{myquote}

畢竟未知後來如何,且聽下囘分解。

