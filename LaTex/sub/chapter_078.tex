\includepdf[pages={155,156},fitpaper=false]{tst.pdf}
\chapter*{第七十八囬 \\西門慶兩戰林太太 吳月娘翫燈請藍氏}
\addcontentsline{toc}{chapter}{第七十八囬 西門慶兩戰林太太 吳月娘翫燈請藍氏}
\markboth{{\titlename}卷之八}{第七十八囬 西門慶兩戰林太太 吳月娘翫燈請藍氏}


\begin{myquote}
黄鍾應律好風催,陰伏陽生淑歲囘。

葵影便移長至日,梅花先趂大寒開。

八神表日占和歲,六管吹葭動細灰。

已有岸傍迎臘柳,參差又欲領春來。
\end{myquote}

話說當日西門慶陪大舅飲酒,到晚囬家。到次日,荆都監早晨騎馬來拜謝,說道:「昨日見旨意下來,下官不勝欣喜,足見老翁愛厚費心之至,實為啣結難忘!范大人便老了,張菊軒指望陞轉他一步兒,照舊也罷了,還虧他些。」説畢,茶湯兩換,荆都監起身,因問:「雲大人到幾時請俺們喫酒?」西門慶道:「近節這兩日也是請不成,直到正月間罷了。」送至大門,上馬而去。西門慶這裏宰了一口鮮猪,兩壜浙江酒,一疋大紅絨金豸員領,一疋黑青粧花紵絲員領,一百菓餡金餅,謝宋御史。就差春鴻拿帖兒,送到察院去。門吏入報進去,宋御史喚至後廳火房内,賞茶喫。等寫了囬帖,裝於套内封了,又賞了春鴻三錢銀子。來見西門慶,拆開觀看,上寫着:

\begin{myquote}[\markfont]
「兩次造擾華府,悚愧殊甚!今又辱承厚貺,何以克當?外令親荆子事,已具本矣,想已知悉。連日渴仰丰標,容當面悉。使旋謹謝。

\raggedleft{{\kaishu(下書)}侍生宋喬年拜}

\raggedright{大錦衣西門先生大人門下。」}
\end{myquote}

宋御史隨即差人送了一百本曆日,四刀紙,一口猪來囬禮。

一日,上司行下文書來,令吳大舅本衛到任管事。西門慶拜去,就與吳大舅三十兩銀子,四疋京緞,教他上下使用。到二十四日稍閒,封了印來家,又備羊酒、花紅、軸文,邀請親朋,等吳大舅従衛中上任囬來,迎接到家,擺大酒席,與他作賀。又是何千戶東京家眷到了,西門慶寫月娘名字,送茶過去。到二十六日,玉皇廟吳道官十二個道衆,在家與李瓶兒念百日經,十囘度人,整做法事,大吹大打,道場行香。各親朋都來送茶,請喫齋供,至晚方散,俱不在言表。至廿七日,西門慶打發各家禮畢;又是應伯爵、謝希大、常時節、傅夥計、甘夥計、韓道國、賁地傳、崔本,每家半口猪,半腔羊,一壜酒,一包米,一兩銀子;院中李桂姐、吳銀兒、鄭愛月兒,每人一套杭州絹衣服,三兩銀子。吳月娘又與菴裏薛姑子打齋,令來安兒送香油米麵銀錢去,不在言表。

看看到年除之日,窻梅横月,簷雪滚風,竹爆千門,燈燃萬戶,家家貼春勝,處䖏掛桃符。西門慶燒了紙,又到於李瓶兒房靈前祭奠已畢,置酒於後堂,合家大小團聚。西門慶與月娘上坐,等李嬌兒、孟玉樓、潘金蓮、孫雪娥、西門大姐並女婿陳經濟都遞了酒,兩旁列坐。先是春梅、迎春、玉簫、蘭香、如意兒,五個磕頭,然後小玉、繡春、小鸞兒、元宵兒、中秋兒、秋菊磕頭。其次者來昭妻一丈青惠慶、來保妻惠祥、來興妻惠秀、來爵妻惠元,一般兒四個家人媳婦磕頭。然後纔是王經、春鴻、玳安、平安、來安,棋童兒、琴童兒、畫童兒、來昭兒子鐵棍兒、來保兒子僧寳兒、來興女孩兒年兒來磕頭。西門慶與吳月娘俱有手帕汗巾銀錢賞賜。

到次日,重和元年新正月元旦,西門慶早起,冠冕穿大紅,天地上炷了香,燒了紙,喫了點心,備馬就出去拜巡按,賀節去了。月娘與衆婦人,早起來施朱傅粉,插花插翠,錦裙繡襖,羅襪弓鞋,粧點妖嬈,打扮可喜,都來後邊月娘房内,廝見行禮。那平安兒與該日節級,在門首接拜帖、落門簿,答應往來官長士夫。玳安與王經穿着新衣裳、新靴新帽,在門首踢毽子兒,放炮𤍤,又嗑瓜子兒,袖香桶兒,戴鬧蛾兒。衆夥計主管,門下底人,伺候見節者不計其數,都是陳經濟一人在前邊客位管待。後邊大廳,擺設錦筵桌席,單管待親朋;花園捲棚,放下毡幃煖簾,鋪陳錦裀繡毯,獸炭火盆,放着十桌,都是銷金桌幃,粧花椅墊,盤粧菓品,瓶插金花,筵開玳瑁,專一留待士大夫官長。約晌午間,西門慶往府縣拜了人囬來,剛下馬,招宣府王三官兒衣巾,有四五個人跟隨,就來拜。到廳上拜了西門慶四雙八拜,然後請吳月娘出來見。西門慶請到後邊,與月娘見了,出來前廳留坐。纔拿起酒來喫了一盞,只見何千戶來拜。西門慶就教陳經濟管待陪王三官兒,他便往捲棚内陪何千戶坐去了。王三官喫了一囬,告辭起身。陳經濟送出大門,上馬而去。落後又是荆都監、雲指揮、喬大戶,皆絡繹而至。

西門慶待了一日人,已酒帶半酣。至晚打發人去了,歸到上房,歇了一夜。到次日早,又出去賀節。直至晚歸家來。家中韓姨夫、應伯爵、謝希大、常時節、花子由來拜,陳經濟陪侍在廳上坐的。候至已久,西門慶到了,見畢禮,従新擺上酒菜點心來飲酒。韓姨夫與花子由隔門,先起身去了。只見伯爵、希大、常時節坐着,如定油兒一般,還不去。又撞見吳二舅來了,見了禮,又往後邊拜見月娘,出來一處坐的。直喫到掌燈已後方散。西門慶已喫的酩酊大醉,送出伯爵等到門首,衆人去了。西門慶見玳安在旁站立,揑了一把手。玳安就知意,說道:「他屋裏沒人。」這西門慶就撞入他房内,老婆早已在封門裏,迎接進去。兩個也無閒話,走到裏間内,老婆脱衣解帶,仰𢵞炕上。西門慶褪下褲子,扛起腿來,那話使有銀託子,就幹起來。原來老婆好並着腿幹,兩隻手𢵞着,只教西門慶攮他心子。那浪水熱熱一陣流出來,把牀褥皆濕。西門慶龜頭蘸了薬,攮進去。兩手扳着腰,只顧兩相揉搓,麈柄盡入至根,不容毫髮。婦人瞪目,口中只呌親爺。那西門慶問他:「你小名叫甚麽?說與我。」老婆道:「奴娘家姓葉,排行五姐。」這西門慶口中喃喃呐呐,就呌:「葉五兒!不知道口裏會㒲不會?」那老婆原來奶子出身,與賁四私通,被拐出來,占為妻子;五短身材,兩個べべ胎眼兒,今年也是屬兔的,三十二歲了,甚麽事兒不知道?口裏如流水連叫親爺不絶,情濃一泄如注。西門慶扯出麈柄要抹,婦人攔住:「休抹,等淫婦下去替你吮淨了罷!」這西門慶滿心歡喜。婦人眞個蹲下身子,雙手捧定那話,吮咂的乾乾淨淨,纔繫上褲子。因問西門慶:「他怎的去恁些時不來?」西門慶道:「我這裏也盼他哩,只怕京中夏大人留住他使。」又與了老婆二三兩銀子盤纏。因說:「我待與你一套衣服,恐賁四知道,不好意思。不如與你些銀子兒,你自家治買罷。」開門送出來。玳安又早在鋪子裏掩門等候,待西門慶進來,方纔関上拴。西門慶便往後邊去了。

看官聽說:自古上梁不正則下梁歪,此理之自然也。如人家主子行苟且之事,家中使的奴僕,皆效尤而行。原來賁四這個老婆,不是守本分的,先與玳安有姦,落後又把西門慶勾引上了。這玳安剛打發西門慶進去了,傅夥計又没在鋪子裏上宿,他與平安兒打了兩大壺酒,就在賁四老婆屋裏,喫到有二更時分,平安在鋪子裏歇了,他就和老婆在屋裏睡了一宿。有這等的事?正是:時人不用穿針待,那得工夫送巧來!有詩為證:

\begin{myquote}
滿眼風流滿眼迷,殘花何事濫如泥?

捨琴暫息商陵操,惹得山禽遶樹啼。
\end{myquote}

卻說賁四老婆晚夕對玳安說:「只怕隔壁韓嫂兒傳嚷得後邊知道,也似韓夥計娘子,一時被你娘們說上幾句,羞人答答的,怎好相見?」玳安道:「如今家中除了俺大娘和五娘不言語,别的不打緊。俺大娘倒也罷了,只是五娘快出尖兒。你依我,節間買些甚麽兒進去,孝順俺大娘;别的不稀罕,他平昔好喫蒸酥,你買一錢銀子菓餡蒸酥、一盒好大壯瓜子送進去。這初九日是俺五娘生日,你再送些禮去,你到明日進來磕頭,梯己再送一盒瓜子與俺五娘。管情就掩住許多口嘴。」這賁四老婆眞個依着玳安之言,第二日趕西門慶不在家,玳安就替他買了盒子,掇進後邊月娘房中。月娘便道:「是那裏的?」玳安道:「是賁四嫂送這盒點心瓜子與娘喫。」月娘道:「男子漢又不在家,那討個錢來?又教他費心!」連忙收了,又囘出一盒饅頭,一盒菓子與他說:「多上覆,多謝了。」

那日西門慶拜人囬家早,有玉皇廟吳道官來拜,在廳上留坐喫酒。剛打發吳道官去了,西門慶脱了衣服,使玳安:「你騎了馬,問聲文嫂兒去。俺爺今日要來拜拜太太,看他怎的説。」玳安道:「爺且不消去。頭裏小的撞見文嫂兒,騎着驢子打門首過去了。他說明日初四,王三官兒起身往東京,與六黄公公磕頭去了。太太說,教爺初六日過去見節,他那裏伺候着哩。」西門慶便道:「他眞個這等說來?」玳安道:「莫不小的敢說謊?」這西門慶就入後邊去了。剛到上房坐下,忽有來安兒來報:「大舅來了。」只見吳大舅冠冕着,束着金帶,進入後堂,先拜西門慶,說道:「一言難盡。我吳鎧多蒙姐夫擡擧看顧,又破費姐夫了,多謝厚禮。日昨姐夫下降,我又不在家,失迎!空慢姐夫來了。今日敬來與姐夫磕個頭兒,恕我遲慢之罪!」説着,磕下頭去。西門慶慌忙平頭相還下來,說道:「大舅恭喜,自然之道理,至親何必計較!」吳大舅於是拜畢西門慶,月娘出來,與他哥磕頭。頭戴翡白縐紗金梁冠兒,海獺臥兔,白綾對衿襖兒,沉香色遍地金比甲,玉色綾寬襴裙。耳邊二珠環兒,金鳳釵梳,胸前帶着金三事㩟領兒,裙邊紫遍地金八條穗子的荷包,五色鑰匙線帶兒,紫遍地金扣花白綾高底鞋兒,打扮的鮮鮮兒的,向前花枝招颭,繡帶飄飄,插燭也似磕了四個頭。慌的大舅忙還半禮,說道:「姐姐,兩禮兒罷!」説道:「哥哥嫂嫂不識好歹,常來擾害你兩口兒。你哥老了,看顧看顧罷。」月娘道:「一時不到,望哥躭帶便了。」吳大舅道:「姐姐没的說,累你兩口兒還少哩!」拜畢,西門慶留吳大舅坐,說道:「這早晚了,料大舅也不拜人了。寬了衣裳,咱房裏坐罷。」不想孟玉樓與潘金蓮兩個都在屋裏,聽見嚷吳大舅進來,連忙走出來與大舅磕頭:都是海獺臥兔兒,白綾襖兒,玉色挑線裙子;一個綠遍地金比甲兒;一個是紫遍地金比甲兒;頭上戴的都是䯼髻,玉樓帶的是環子,金蓮是青寶石墜子;下邊尖尖趫趫,顯露金蓮。與吳大舅磕了頭,逕往各人房裏去了。

西門慶讓大舅房内坐的,騎火盆安放桌兒,擺上春盛菓盒,各樣熱碗嗄飯,大饅頭、點心,八寳攢湯,一齊拿上來。小玉玉簫都來與大舅磕頭。須臾,喫了湯飯,月娘用小金镶玳瑁鍾兒斟酒遞與大舅,西門慶主位相陪。吳大舅讓道:「姐姐,你也來坐的。」月娘道:「我就來。」又往裏間房内,拿出數樣配酒的菓菜來,都是冬笋、銀魚、黄鼠、鱘鮓、海蜇、天花菜、蘋婆、螳螂、鮮柑、石榴、風菱、雪梨之類。飲酒之間,西門慶便問:「大舅的公事都了畢停當了?」吳大舅道:「蒙姐夫擡擧,年前任便到了,上下人事倒也都周給的七八,還有屯所裏未曾去到到任。明日是個好日期,衛中開了印,來家整理了些盒子,須得擡到屯所裏到任,行牌拘將那屯頭來參見,吩咐吩咐。前官丁大人壞了事情,已是被巡撫侯爺參劾去了任。如今我接管承行,須得也要振刷在册花戶,警勵屯頭,務要把這舊管新增開報明白,到明日秋糧夏稅,纔好下屯徵收。」西門慶道:「通共約有多少屯田?」吳大舅道:「這屯田,不瞞姐夫說,太祖舊例,練兵衛因田養兵,省轉輸之勞,纔立下這屯田。那時只是上納屯田秋糧,又不問民地。後喫宰相王安石立青苗法,增上這夏稅。而今這濟州管内,除了拋荒葦場港隘,通共二萬七千畝屯地。每頃秋税夏税,只徵收一兩八錢,不上五百兩銀子。到年終纔傾齊了,往東平府交納,轉行招商,以備軍糧馬草作用。」西門慶又問:「還有羡餘之利?」吳大舅道:「雖故還有些抛零人戶不在册者,鄉民頑滑,若十分追徵緊了,等秤斛斗重,恐聲口致起公論。」西門慶道:「若是有些敷餘兒也罷,難道說全徵?若徵收些出來,斛斗等秤上也夠咱們上下攪給。」吳大舅道:「不瞞姐夫說,若會管此屯,現一年也有百十兩銀子尋。到年終,人戶們還有些鷄鵝豚米面見相送。那個是各人取覓,不在數内的。只是多賴姐夫力量扶持。」西門慶道:「得夠你老人家攪給,也盡我一點之心。」正說着,月娘也走來旁邊陪坐,三人飲酒到掌燈已後,吳大舅纔起身去了。西門慶那日就在前邊金蓮房中歇了一夜。到次日,早往衙門中開印,陞廳畫卯,發放公事。先是雲離守家發帖兒,初五日請西門慶并合衛官員喫慶官酒。何千戶娘子藍氏下帖兒,初六日請月娘姊妹相會。

且說那日,西門慶同應伯爵、吳大舅三人,起身到雲離守家。原來旁邊又典了人家一所房子,三間客位内擺酒,叫了一起吹打鼓楽迎接,都有桌面,喫至晚夕來家。巴不到次日,月娘往何千戶家喫酒去了。西門慶打選衣帽齊整,袖着賞賜包兒,騎馬带眼紗,玳安琴童跟隨,午後時分,逕來王招宣府中拜節。王三官兒不在,留下帖兒。文嫂兒又早在那裏,接了帖兒,連忙報與林太太,隨說出來:「請老爺後邊坐。」轉過大廳,到於後邊,進入儀門。五間住房,掀起明簾子,上面供養着先公王景崇影像,陳設兩桌春臺菓酌,朱紅公座,虎皮交椅。脚下氍毹匝地,簾幙垂紅。少頃,林氏穿着大紅通袖襖兒,珠翠盈頭,粉粧膩臉,與西門慶見畢禮數,留坐待茶,吩咐大官把馬牽於後槽喂養。茶湯罷,讓西門慶寬衣房内坐,説道:「小兒従初四日往東京與他叔岳父黄太尉磕頭去了,直過了元宵纔來。」這西門慶一面喚玳安脱去上蓋,裏邊穿着白綾襖子,天青飛魚氅衣,粉底皂靴,十分綽耀。婦人房裏安放桌席。黄銅四方獸面火盆,生着炭火。朝陽房屋,日色照窻,房中十分明亮。須臾,丫鬟拿酒菜上來。盃盤羅列,餚饌堆盈,酒汎金波,茶烹玉蕊。婦人錦裙繡襖,皓齒明眸,玉手傳盃,秋波送意。猜枚擲骰,笑語烘春。話良久,意洽情濃;飲多時,目邪心蕩。看看日落黄昏,又早高燒銀燭。玳安琴童下邊耳房放桌兒,自有文嫂兒主張酒饌點心管待。三官兒娘子,另在那邊角門内一所屋裏居住,自有丫鬟養娘伏侍,等閒不過這邊來。婦人又倒扣角門,僮僕誰敢擅入。酒酣之際,兩個共入裏間房内,掀開繡帳,関上窻戶。丫鬟輕剔銀釭,佳人忙掩朱戶,男子則解衣就寢,婦人即洗脚上牀。枕設寳花,被翻紅浪。原來西門慶家中磨鎗備劍,带了淫器包兒來,安心要鏖戰這婆娘,早把胡僧薬用酒喫在腹中,那話上使着雙託子。在被窝中,架起婦人兩股,縱麈柄入牝中,擧腰展力,那一陣掀騰鼓搗,其聲猶若數鰍行泥淖中相似,連聲響亮。婦人在下,没口叫達達如流水。正是:照海旌幢秋色裏,擊天鼙鼓月明中。有長詞一篇,道這場交戰。但見:

\begin{myquote}
錦屏前迷魂陣擺,繡幃下攝魄旗開。迷魂陣上,閃出一員酒金剛、色魔王:頭戴肉紅盔、錦兜鍪,身穿烏油甲、絳紅袍、纏觔縧、魚皮带、没縫靴,使一柄黑纓鎗,帶的是虎眼鞭、皮包頭流星槌、没す箭,跨一疋捲毛凹眼渾紅馬,打一面覆雨翻雲大帥旗。攝魄旗下,擁一個粉骷髏、花狐狸:頭戴雙鳳翹、珠絡索,身穿素羅衫、翠裙腰、白練襠、凌波襪、鮫綃帶、鳳頭鞋,使一條隔天邊、話絮刀、不得箭、淚偸錘、容瘦鐧、粉面撾、羅幃棒,騎一疋百媚千嬌玉面す,打一柄倒鳳顛鸞遮日傘。須臾,這陣上撲鼕鼕鼓震春雷,那陣上鬧挨挨麝蘭靉靆;這陣上暖溶溶被翻紅浪,那陣上刷剌剌帳控銀鉤。被翻紅浪精神健,帳控銀鉤情意乖。這一個急展展二十四解任徘徊,那一個忽剌剌一十八滚難掙扎;一個是慣使的紅綿套索鴛鴦扣,一個是好耍的拐子流星鷄心槌。一個火忿忿桶子鎗,恨不的扎夠三千下;一個顫巍巍肉膀牌,巴不得搨夠五十囬。這一個善貫甲披袍戰,那一個能奪精吸髓華。一個戰馬叭蹋蹋蹅翻歌舞地,一個征人軟濃濃塞滿密林崖。一個醜搊搜剛硬形骸,一個俊嬌嬈杏臉桃腮。一個施展他久戰熬場法,一個賣弄他鶯聲燕語諧。一個鬬良久,汗浸浸釵横鬢亂;一個戰多時,喘吁吁枕欹裀歪。頃刻間,只見這内襠縣乞砲打成堆,個個皆腫眉し眼;霎時下,則望那莎草塲被鎗扎倒底,人人肉綻皮開。正是:愁雲託上九重天,一派敗兵沿地滚;幾番鏖戰貪淫婦,不似今番這一遭。
\end{myquote}

當下西門慶就在這婆娘心口與陰戶,燒了兩炷香,許下明日家中擺酒,使人請他同三官兒娘子去看燈耍子。這婦人一段身心已是被他拴縛定了,於是滿口應承都去。這西門慶滿心歡喜,起來與他留連痛飲,至二更時分,把馬従後門牽出,作别方囬家去。正是:不愁明日盡,自有暗香來。有詩為證:

\begin{myquote}
盡日思君倚畫樓,相逢不捨又頻留。

劉郎莫謂桃花老,浪把輕紅逐水流。
\end{myquote}

卻說西門慶到家,有平安迎門禀説:「今日有薛公公家,差人送請帖兒,請爺明早往門外皇莊看春。又是雲二叔家,差人送了五個帖兒,請五位娘喫節酒。帖兒都交進去了。」西門慶聽了,沒言語。進入後邊月娘房來,只見孟玉樓潘金蓮都在房内坐的。月娘従何千戶家赴了席來家,已摘了首飾花翠,止戴着䯼髻,撇着六根金簪子,勒着珠子箍兒,上着藍綾襖,下着軟黄綿紬裙子,坐着說話。西門慶進來,連忙道了萬福。西門慶就在正面椅上坐下。問道:「你今日往那裏,這咱纔來?」西門慶只得說:「我在應二哥家留坐到這咱晚。」月娘便説起今日何千戶家酒席上事:「原來何千戶娘子還年小哩,今年纔十八歲。生的燈人兒也似一表人物,好標致!知今博古,透靈兒還強十分。見我去,恰似會了幾遍,好不喜狎。嫁了何大人二年光景,房裏倒使着四個丫頭,兩個養娘,兩房家人媳婦。」西門慶道:「他是内府御前生活所藍太監姪女兒,與他陪嫁了好少錢兒!」月娘又道:「小廝對你説來?明日雲夥計家又請俺們喫節酒,送了五個帖兒,在揀粧上擱着。連薛内相家帖子,都放在一䖏。」因令玉簫:「拿過來與你爹瞧。」這西門慶看了薛内相家帖兒,又看雲離守家帖兒,下書他娘子兒:「雲門范氏歛袵拜請」。西門慶説:「你們明日收拾了,都去走走。」月娘道:「留雪姐在家罷,只怕大節下,一時有個人客驀將來,他們沒處撾撓。」西門慶道:「也罷,留雪姐在家裏,你們四個去吧。明日我也不往那裏去,薛太監請我門外看春,我也懶待去。這兩日春氣發也怎的,只害這邊腰腿疼。」月娘道:「你腰腿疼,只怕是痰火,問任醫官討兩服薬喫不是,只顧挨着怎的?」那西門慶道:「不妨事,由他,一發過了這兩日喫,心淨些。」因和月娘計較:「到明日燈節,咱少不得置席酒兒,請請何大人娘子,連周守備娘子、荆南崗娘子、張親家母、喬親家母、雲二哥娘子,連王三官兒母親和大妗子、崔親家母,這幾位都會會。也只在十二三,掛起燈來。還叫王皇親家那起小廝扮戯耍一日。爭耐去年還有賁四在家,扎了幾架煙火放,今年他東京去了,只顧不見來了,卻教誰人看着扎?」那金蓮在旁插口道:「賁四去了,他娘子兒扎也是一般。」這西門慶就瞅了金蓮道:「這個小淫婦兒,三句話就說下道兒去了。」那月娘玉樓也不睬顧,就罷了。因說道:「那三官兒娘,咱們與他没有大會過,人生面不熟的,怎麽好請他?只怕他也不肯來。」西門慶道:「他旣認我做親,咱送個帖兒與他,來不來,隨他就是了。」月娘又道:「我明日不往雲家去罷,懷着個臨月身子,只管往人家撞來撞去的,敎人家唇齒!」玉樓道:「姐姐,没的説,怕怎麽的?你身子懷的又不顯,怕還不是這個月的孩子,不妨事。大節下,自恁散心去走走兒纔好。」説畢,西門慶喫了茶,就往後邊孫雪娥房裏去了。那潘金蓮見他往雪娥房中去,告過大姐姐,也就往前邊去了。西門慶到於雪娥房中,晚間教他打腿捏身上,捏了半夜。一宿晚景題過。

到次日早晨,只見應伯爵走來借衣服頭面,對西門慶說:「昨日雲二嫂送了個帖兒,今日請房下陪衆嫂子坐。家中舊時有幾件衣服兒,都倒塌了。大正月出門入戶,不穿件好衣服,惹的人家笑話!敢來上覆嫂子,有上蓋衣服,借的兩套兒;頭面簪環,借的幾件兒,教他穿戴了去。」西門慶令王經:「你裏邊對你大娘說去。」伯爵道:「應寳在外邊拿着毡包並盒哩,哥哥累你拿進去,就包出來罷。」那王經接毡包進去。良久抱出來,交與應寳,說道:「裏面兩套上色緞子織金衣服,大小五件頭面,一雙二珠環兒。」應寳接的,往家去了。

西門慶陪着伯爵喫茶,說道:「昨日房下在何大人家喫酒,來晚了。今日不想雲二哥娘子送了五個帖兒,又請房下們都會會兒。大房下又有臨月身孕,懶待去。我說他旣來請,大節下你等走走去罷。我又連日不得閒,只昨日纔把人事拜了。前日咱們在雲二哥家喫了酒來,昨日我又出去有些小事,來家晚了。今日薛内相又請我門外看春,怎麽得工夫去?吳親家廟裏又送帖兒,初九日年例打醮,也是去不成,教小婿去了罷。這兩日不知酒多了也怎的,只害腰疼,懶待動彈。」伯爵道:「哥,你還是酒之過。濕痰流注在這下部,也還該忌忌。」西門慶道:「這節間到人家,誰是肯輕放了你我的,怎麽忌的住!」伯爵又問:「今日那幾位嫂子去?」西門慶道:「大房下和第二第三第五的房下四人去,我在家且歇息兩日兒罷。」正說着,只見玳安拿進盒兒來,説道:「何老爹家差人送請帖兒來,初九日請喫節酒。」西門慶道:「早是你看着,人家來請,你不去?」於是看盒兒内放着三個請書兒,一個宛紅簽兒寫着:「大寅丈四泉翁老先生大人」,一個寫「大都閫吳老先生大人」,一個寫着「大鄉望應老先生大人」,俱是「侍教生何永壽頓首拜」。玳安説:「他那裏説不認的,教咱這裏轉送送兒罷。」伯爵一見,便說:「這個卻怎樣兒好?我還没送禮兒去與他,他來請我,怎好去?」西門慶道:「我這裏替你封上分帕禮兒,你差應寳早送去就是了。」一面令王經:「你封二錢銀子,一方手帕,寫你應二爹名字,與你應二爹。」因說:「你把這請帖兒袖了去,省的我又教人送。」只把吳大舅的差來安兒送去了。須臾,王經封了帕禮,遞與伯爵。伯爵打恭說道:「謝哥,容另還。我後日早來會你,咱一同起身。」説畢,作辭去了。

卻表吳月娘等午間打扮停當,一頂大轎,三頂小轎,後面又帶着來爵媳婦兒惠元收疊衣服,一頂小轎兒;四名排軍喝道,琴童、春鴻、棋童、來安,四個跟隨,往雲指揮家來喫酒。正是: 

\begin{myquote}
翠眉雲鬢畫中人,嬝娜宫腰迴出塵。

天上嫦娥元有種,嬌羞釀出十分春。
\end{myquote}

不説月娘與李嬌兒、孟玉樓、潘金蓮,都往雲離守家喫酒去了。西門慶吩咐大門上平安兒:「隨問甚麽人,只說我不在。有帖兒接了就是了。」那平安經過一遭,那裏再敢離了左右,只在門首坐的。但有人客來望,只回不在家。西門慶那日,只在李瓶兒房中圍爐坐的。自從李瓶兒沒了,月娘教如意兒休勒上奶去,每日只喂奶來興女孩兒城兒。連日西門慶害腿疼,猛然想起任醫官與他延壽丹,用人乳喫。於是來到房中,教如意兒擠奶。那如意兒,節間頭上戴着黄霜霜簪環,滿頭花翠,勒着翠藍銷金汗巾,藍紬子襖兒,玉色雲緞披襖兒,黄綿紬裙子,脚下沙緑潞紬白綾高底鞋兒,粧點打扮比昔時不同;手上戴着四個烏銀戒指兒,坐在傍邊打發喫了薬,又與西門慶斟酒布菜兒。迎春打發喫了飯,走過隔壁,和春梅下棋去了,要茶要水,自有綉春在厨下打發。西門慶見丫鬟都不在屋裏,在炕上斜靠着背,扯開白綾吊的絨褲子,露出那話來,帶着銀託子,教他用口吮咂。一面傍邊放着菓酌,斟酒自飲。因呼道:「章四兒,我的兒!你用心替達達咂。我到明日,尋出件好粧花緞子比甲兒來,你正月十二日穿。」老婆道:「着,爹可憐見。」咂弄夠一頓飯時,西門慶道:「我兒,我心裏要在你身上燒炷香兒。」老婆道:「隨爹你揀着燒炷香兒。」西門慶令他関上房門,把裙子脱了,上炕來仰臥在枕上,底下穿着新做的大紅潞紬褲兒,褪下一隻褲腿來。西門慶袖内還有燒林氏剩下的三個燒酒浸的香馬兒,撇去他抹胸兒,一個坐在他心口内,一個坐在他小肚兒底下,一個安在他𣭈蓋子上,用安息香一齊點着。那話下邊便插進牝中,低着頭看着拽,只顧沒稜露腦,往來送進不已。又取過鏡臺來,傍邊照看。須臾,那香燒到肉跟前,婦人蹙眉齧齒,忍其疼痛,口裏顫聲柔語,哼成一塊,沒口子呌:「達達爹爹,罷了,我了,好難忍也!」西門慶便叫道:「章四兒淫婦,你是誰的老婆?」婦人道:「我是爹的老婆。」西門慶敎與他:「你說是熊旺的老婆,今日屬了我的親達達了。」那婦人囬應道:「淫婦原是熊旺的老婆,今日屬了我的親達達了!」西門慶又問道:「我會㒲不會?」婦人道:「達達會㒲𣭈。」兩個淫聲艷語,無般言語不說出來。西門慶那話粗大,撑的婦人牝戶滿滿,使往來出入,帶的花心紅如鸚鵡舌,黑似蝙蝠翅一般,翻覆可愛。西門慶於是把他兩股扳抱在懷内,四體交匝,兩相迎凑,那話盡没至根,不容毫髮。婦人瞪目失聲、淫水流下。西門慶情濃楽極,精邈如湧泉。正是:不知已透春消息,但覺形骸骨節鎔。有詩為證:

\begin{myquote}
任君隨意薦霞盃,滿腔春事浩無涯。

一身徑藉東君愛,不管牀頭墜寳釵。
\end{myquote}

當日西門慶燒了這老婆身上三處香,開門尋了一件玄色緞子粧花比甲兒與他。至晚月娘衆人來家,對西門慶說:「原來雲二嫂也懷着個大身子。俺兩個今日酒席上都遞了酒,說過到明日兩家若分娩了,若是一男一女,兩家結親做親家;若都是男子,同堂攻書;若是女兒,拜做姐妹,一處做針指,來往同親戚兒耍子。應二嫂做保證。」西門慶聽了笑。

話休饒舌,到第二日,卻是潘金蓮上壽。西門慶早起往衙門中去了,吩咐小廝們擡出燈來,收拾揩抹乾淨,大廳捲棚各處掛燈,擺設錦帳圍屏,呌來興買下鮮菓,呌了小優,晚夕上壽。這潘金蓮早晨打扮出來,花粧粉抹,翠袖朱唇,走來大廳上看。見玳安與琴童站着高凳在那裏掛燈——那三大盞珠子吊掛燈,笑嘻嘻說道:「我道是誰在這裏,原來是你們在這裏掛燈哩。」琴童道:「今日是五娘上壽,爹吩咐下俺們掛了燈,明日娘的生日好擺酒。晚夕小的們與娘磕頭,娘一定賞俺們哩。」婦人道:「要打便有,要賞可沒有!」琴童道:「耶嚛!娘怎的沒打不說話,行動只把打放在頭裏?小的們是娘的兒女,娘看顧看顧兒便好,如何只說打起來!」婦人道:「賊囚,别要説嘴!你與他好生仔細掛那燈,沒的例兒撦兒的拿不牢掉將下來。前日年裏為崔本來,説你爹大白日裏不見了,險不險赦了一頓打,沒曾打。這遭兒可打成了!」琴童道:「娘只説破話,小的命兒薄薄的,又唬小的!」玳安道:「娘也不打聽,這個話兒娘怎得知?」婦人道:「宫外有株松,宫内有口鐘——鐘的聲兒,樹的影兒,我怎麽有個不知道的!昨日可是你爹對你大娘説,去年有賁四在家,還扎了幾架煙火放。今年他不在家,就沒人會扎。乞我說了兩句:『他不在家,左右有他老婆會扎,教他扎不是!』」玳安道:「娘説的甚麽話?一個夥計家,那裏有此事?」婦人道:「甚麽話,檀木靶!有此事,眞個的。畫一道兒,只怕㒲過界兒去了!」琴童道:「娘也休聽人説,只怕賁四來家知道。」婦人道:「瞞那儍王八千來個!我只說那王八也是明王八,怪不的他往東京去的放心,丢下老婆在家,料莫他也不肯把𣭈閒着!賊囚根子們,別要說嘴!打夥兒替你爹做牽頭,勾引上了道兒,你們好圖躧狗尾兒,説的是也不是?敢說我不知道!嗔道賊淫婦買禮來,與我也罷了,又送蒸酥與他大娘!另外又送一大盒瓜子兒與我,小買住我的嘴頭子,他是會養漢兒!我就猜沒别人,就知道是玳安兒這賊囚根子替他鋪謀定計。」玳安道:「娘屈殺小的,小的平白管他這勾當怎的?小的等閒也不往他屋裏去,娘也少聽韓囬子老婆說話。他兩個為孩子好不嚷亂!常言:『要好不能夠,要歹登時就』、『一片房倒壓不殺人,一片舌頭倒壓殺人』、『聽者有不聽者無』。論起來,賁四娘子為人和氣,在咱門首住着,家中大小,沒曾惡識了一個人。誰人不在他屋裏討茶喫?莫不都養着,倒沒放處!」金蓮道:「我見那水眼淫婦,矮着個靶子,像是半頭磚兒也似的個兒,把那水濟濟眼擠着,七八拿杓兒舀。好個怪淫婦!他便和那韓道國老婆那長大摔瓜淫婦,我不知怎的掐了眼兒不待見他!」

正說着,只見小玉走來説:「俺娘請五娘,潘姥姥來了,要轎子錢哩。」金蓮道:「我在這裏站着,他従多咱進去了?」琴童道:「姥姥打夾道裏,我送進去了。一來時擡轎的說該他六分銀子轎子錢。」金蓮道:「我那得銀子?來人家怎不帶轎子錢兒來!」一面走到後邊,見了他娘,只顧不與他轎子錢,只說沒有。月娘道:「你與姥姥一錢銀子,寫帳就是了。」金蓮道:「我是不惹他。他的銀子都有數兒,只教我買東西,没教我打發轎子錢!」坐了一囬,大眼看小眼。外邊擡轎子的催着要去。玉樓見不是事,向袖中拿出一錢銀子來,打發擡轎的去了。不一時,大妗子、二妗子、大師父來了。月娘擺茶喫了。潘姥姥歸到前邊他女兒房内來,被金蓮儘力數落了一頓,説道:「你沒轎子錢,誰教你來了?恁出醜㓦劃的,教人家小看!」潘姥姥道:「姐姐,你沒個錢兒與我來,老身那討個錢兒來?好容易賙辦了這分禮兒來!」婦人道:「指望問我要錢,我那裏討個錢兒與你?你看着,睜着眼在這裏七個窟窿到有八個眼兒!今後你有轎子錢,便來他家來;沒轎子錢,别要來。料他家也沒少你這個窮親戚,休要做打嘴的獻世包!関王買荳腐——人硬貨不硬!我又聽不上人家那等𣭈聲顙氣。前日為你去了,和人家大嚷大鬧的,你知道便罷了,驢糞毬兒面前光——卻不知裏面受悽惶!」幾句説的潘姥姥嗚嗚咽咽哭起來了。春梅道:「娘今日怎的只顧説起姥姥來了!」一面安撫老人家在裏邊炕上坐的,連忙點了盞茶與他喫。潘姥姥氣的在炕上睡了一覺,只見後邊請陪大妗子喫飯,纔起來往後邊去了。

西門慶従衙門中來家,正在上房擺飯,忽有玳安拏進帖兒來說:「荆老爹陞了東南統制,來拜爹。」西門慶見帖兒上寫:「新陞東南統制兼督漕運總兵官荆忠頓首拜。」慌的西門慶令擡開飯桌,連忙穿衣冠帶,迎接出來。只見荆總制穿着大紅麒麟補服、渾金帶進來,後面跟着許多僚掾軍牢。一面讓至大廳上,叙禮畢,分賓主而坐。茶湯上來,待茶畢,荆統制説道:「前日陞官勅書纔到,還未上任,逕來拜謝老翁。」西門慶道:「老總兵榮擢,恭喜!大才必有大用,自然之道。吾輩亦有光矣,容當拜賀。」一面請寬尊服:「少坐一飯。」即令左右放桌兒。荆統制再三致謝道:「學生奉告老翁,一家尚未拜,還有許多薄冗,容日再來請教罷。」便徑起身。西門慶那裏肯放,隨令左右上來,寬去衣服,登時打抹春臺,收拾酒菓上來。獸炭頓燒,煖簾低放;金壺斟玉液,翠盞貯羊羔。纔斟上酒來,只見鄭春王相兩個小優兒來到,趴在面前磕頭。西門慶道:「你兩個如何這咱纔來?」問鄭春:「那一個叫甚名字?」鄭春道:「他喚王相,是王柱的兄弟。」西門慶即令拿楽器上來,彈唱與他荆爺聽。須臾,兩個小優安放樂器停當,歌唱了一套「霽景融和」。左右拿上兩盤攢盒點心嗄飯,兩瓶酒,打發馬上人等。荆統制道:「這等就不是了。學生叨拜,下人又蒙賜饌,何以克當!」即令上來磕頭。西門慶道:「一二日房下還要潔誠請尊正老夫人賞燈一敍,望乞下降。在座者惟老夫人、張親家夫人、同僚何天泉夫人,還有兩位舍親,再無他人。」荆統制道:「若老夫人尊票到,賤荆一定趨赴。」又問起:「周老總兵怎的不見陞轉?」荆統制道:「我聞得周南軒也只在三月間,有京營之轉。」西門慶道:「這也罷了。」坐不多時,荆統制告辭起身。西門慶送出大門,看着上馬喝道而去。

晚夕,潘金蓮上壽,後廳小優彈唱,遞了酒,西門慶便起身往金蓮房中去了。月娘陪着大妗子、潘姥姥、女兒郁大姐、兩個姑子,在上房坐的飲酒。潘金蓮便陪西門慶在他房内,從新又安排上酒來,與西門慶梯己遞酒磕頭。落後潘姥姥來了,金蓮打發他李瓶兒這邊歇臥。他便陪着西門慶自在飲酒,作歡頑耍做一䖏。

卻説潘姥姥到那邊屋裏,如意迎春讓他熱炕上坐着。先是姥姥看見明間内靈前供擺着許多獅僊五老定勝、樹菓柑子、石榴蘋婆、雪梨鮮菓、蒸酥點心、饊子蔴花,滿爐焚着末子香蠟,點着長明燈,桌上拴着銷金桌幃,旁邊掛着他影,穿大紅遍地金袍兒,錦裙繡襖,珠子挑牌,向前道了個問訊,說道:「姐姐好䖏生天去了!」因坐在炕上,向如意兒迎春道:「你娘夠了,官人這等費心追薦,受這般大供養,夠了!他是有福的。」如意兒道:「前日娘的百日,請姥姥怎的不來?門外花大妗子和大妗子,都在這裏來。十二個道士念經,好不大吹大打,揚旛道場,水火煉度,晚上纔去了。」潘姥姥道:「幫年逼節,丢着個孩子在家,我來,家中沒人,所以就不曾來。今日你楊姑娘怎的不見?」如意兒道:「姥姥還不知道,楊姑娘老病死了。從年裏俺娘念經就沒來。俺娘們都往北邊與他上祭去了。」潘姥姥道:「可傷!他大如我,我還不曉的他老人家沒了!嗔道今日怎的不見他!」說了一囬楊姑娘。如意兒道:「姥姥,有鍾兒甜酒兒,你老人家用些兒?」一面教迎春:「姐,你放小桌兒在炕上,篩甜酒與姥姥喫盃。」不一時取到。飲酒之間,婆子又提起李瓶兒來:「你娘好人,有仁義的姐姐,熱心腸兒。我但來這裏,沒曾把我老娘當外人看承。到就是熱茶熱水與我喫,還只恨我不喫。夜間和我坐着說話兒。我臨家去,好歹包些甚麽兒與我拿了去,再沒曾空了我。不瞞姐姐你們說,我身上穿的這披襖兒,還是你娘與我的!正經我那寃家,半個折針兒也迸不出來與我。我老身不打誑語,阿彌陀佛,水米不打牙,他若肯與我一個錢兒,我滴了眼睛在地!你娘與了我些甚麽兒,他還說我小眼薄皮,愛人家的東西。想今日為轎子錢,你大包家拿着銀子,就替老身出幾分便怎的?咬定牙兒,只說他沒有。倒教後邊西房裏姐姐,拿出一錢銀子來,打發擡轎的去了。歸到屋裏,還數落了我一頓:到明日,有轎子錢便教我來;沒轎子錢,休教我上門走!我這去了,不來了!來到這裏,沒的受他的氣。隨他去,有天下人心狠,不似俺這短壽命!姐姐,你們聽着我說,老身若死了,他到明日不聽人說,還不知怎麽收成結果哩!想着你従七歲沒了老子,我怎的守你到如今?従小兒教你做針指,往余秀才家上女學去,替你怎麽纏手縛脚兒的。你天生就是這等聰明伶俐?到得這步田地,他把娘喝過來断過去,不看一眼兒!」如意兒道:「原來五娘従小兒上學來,嗔道恁提起來就會,識字深!」潘姥姥道:「他七歲兒上女學,上了三年,字倣也曾寫過;甚麽詩詞歌賦唱本上字不認的!」

正說着,只見打的角門子響。如意兒道:「是誰呌門?」使綉春:「二姐,你去瞧瞧去。」那綉春走來說:「是春梅姐來了。」如意兒連忙捏了潘姥姥一把手,就說道:「姥姥悄悄的,春梅來了。」潘姥姥道:「老身知道。他與我那寃家一條腿兒。」只見春梅進來,頭上翠花雲髻兒,羊皮金沿的珠子箍兒,藍綾對衿襖兒,黄綿紬裙子,金燈籠墜子,貂鼠圍脖兒,走來見衆人陪着潘姥姥喫酒,說道:「姥姥還沒睡哩?我來瞧瞧姥姥來了。」如意兒讓他坐。這春梅把裙子摟起,一屁股坐在炕上。迎春便緊挨着他坐。如意坐在右邊炕頭上,潘姥姥坐在當中。因問:「你爹和你娘睡了不曾?」春梅道:「剛纔喫了酒,打發他兩個睡下了。我來這邊瞧瞧姥姥,有幾樣菜兒,一壺兒酒,取了來和姥姥坐的。」因央及綉春:「你那邊敎秋菊掇了來,我已是攢下了。」那綉春不一時走過那邊取了來。秋菊放盒内掇着菜兒,綉春提了一錫瓶金華酒。春梅吩咐秋菊:「你往房裏聽着,若呌我,來這裏對我說。」那秋菊把嘴谷都着去了。一面擺酒在炕桌上,都是燒鴨、火腿、薰臘腸、細鮓、糟魚、菓仁、鹹酸、蜜食、海味之類,堆滿春臺。綉春関上角門,走進在旁邊陪坐。於是篩上酒來,春梅先遞了一鍾與潘姥姥,然後遞一鍾如意兒,一鍾與迎春。綉春在旁邊炕兒上坐的,共五人坐定,把酒來斟。春梅護衣碟兒内每樣揀出遞與姥姥衆人喫,說道:「姥姥,這個都是整菜,你用些兒。」那婆子道:「我的姐姐,我老身喫。」因說道:「就是你娘,従來也沒費恁個心兒管待我管待兒。姐姐,你倒有惜孤愛老的心。你到明日管情好,一步一步自高。敢是俺那寃家,沒人心,沒人義!幾遍為他心齷齪,我也勸他,他就扛的我失了色。今日早是姐姐你看着,我來你家討冷飯喫來了?你下老實那等扛我!」春梅道:「姥姥,罷,你老人家只知其一,不知其二。俺娘他爭強不伏弱的性兒,不同的六娘錢自有。他本等手裏沒錢,你只說他不與你。別人不知道,我知道。像俺爹,雖是抄的銀子放在屋裏,俺娘正眼兒也不看他的。若遇着買花兒東西,明公正義問他要,不恁瞞藏背掖的。教人看小了他,他怎麽張着嘴兒說人!他本沒錢,姥姥怪他,就虧了他了。莫不我護他?也要個公道!」如意兒道:「錯怪了五娘。自古親兒骨肉,五娘有錢,不孝順姥姥,再與誰?常言道:要打看娘面,千朶桃花一樹兒生。到明日你老人家黄金入櫃,五娘他也沒個貼皮貼肉的親戚,就如死了俺娘樣兒!」婆子道:「我有今年沒明年,知道今死明日死?我也不怪他。」春梅見婆子喫了兩鍾酒,韶刀上來了。便呌迎春:「二姐,你拿骰盆兒來,咱們擲個骰兒搶紅耍子兒罷。」不一時,取了四十個骰兒的骰盆兒來。春梅先與如意兒擲,擲了一囬,又與迎春擲,都是賭大鍾子。你一盞,我一鍾,須臾竹葉穿心,桃花上臉,把一錫瓶酒喫的罄凈。迎春又㧱上半罈麻姑酒來,也都喫了。約莫到二更時分,那潘姥姥老人家,熬不的,又早前靠後仰打起盹來,方纔散了。

春梅便歸這邊來。推了推角門,開着;進入院内,只見秋菊正在明間板壁縫兒内,倚着春櫈兒,聽他兩個在屋裏行房,怎的作聲喚,口中呼呌甚麽。正聽在熱鬧,不防春梅走來到跟前,向他腮頰上儘力打了個耳刮子,罵道:「賊少死的囚奴,你平白在這裏聽甚麽!」打的秋菊睜睜的,說道:「我這裏打盹,誰聽甚麽來?你就來打我!」不想房内婦人聽見,便問春梅:「他和誰說話?」春梅道:「沒有人。我使他関門,他不動。」於是替他摭過了。秋菊揉着眼,関上房門。春梅走到炕上,摘頭睡了,不在話下。正是:鶬鷓有意留殘景,杜宇無情戀晚暉。

一宿晚景題過。次日,潘金蓮生日,有傅夥計、甘夥計、賁四娘子、崔本媳婦段大姐、吳舜臣媳婦鄭三姐、吳二妗子,都在這裏。西門慶約會吳大舅、應伯爵,整衣冠,尊瞻視,騎馬喝道,往何千戶家赴席。那日也有許多官客,四個唱的,一起雜耍,周守禦同席。飲酒至晚囬家,就在前邊和如意兒歇了。到初十日,發帖兒請衆官娘子喫酒。月娘便向西門慶說:「趂着十二日看燈酒,把門外他孟大姨和俺大姐,也帶着請來坐坐,省的教他知道惱,請人不請他。」西門慶道:「早是你說。」吩咐陳經濟:「再寫兩個帖,差琴童兒請去。」這潘金蓮在旁聽着多心,走到屋裏,一面攛掇把潘姥姥就要起身。月娘道:「姥姥,你慌去怎的?再消住一日兒是的。」金蓮道:「姐姐,大正月裏,他家裏丢着孩子沒人看,敎他去罷。」慌的月娘裝了兩個盒子點心茶食,又與了他一錢轎子錢,管待打發去了。金蓮因對着李嬌兒說:「他明日請他有錢的大姨兒來看燈喫酒,一個老行貨子,觀眉觀眼的,不打發去了,平白敎他在屋裏做甚麽?待要說是客人,沒好衣服穿;待要說是燒火的媽媽子,又不似,倒沒的敎我惹氣!」西慶使玳安兒送了四個請書兒往招宣府,一個請林太太,一個請王三官兒娘子黄氏。又使他院中早呌李桂姐、吳銀兒、鄭愛月兒、洪四兒,四個唱的,李銘、吳惠、鄭奉,三個小優兒。

不想那日賁四従東京來家,梳洗頭臉,打選衣帽齊整,來見西門慶,磕頭,遞上夏指揮囬書。西門慶問他:「如何住這些時不來?」賁四具言在京感冒打寒一節,直到正月初二日,纔收拾起身囬來,「夏老爹多上覆老爹,多承看顧。」西門慶照舊還把鑰匙交與他管絨綿鋪。另打開一間,教吳二舅開鋪子賣紬絹。到明日松江貨船到,都卸在獅子街房内,同來保發賣。且教賁四叫花兒匠在家,趲造兩架煙火,十二日要放與堂客看。早約下應伯爵、謝希大、吳大舅、常時節四位,白日在廂房内坐的,晚夕賞燈飲酒。

只見應伯爵領了李三見西門慶,先道外日承擕之事。坐下喫畢茶,方纔說起:「李三哥來,今有一宗買賣與你說,你做不做?」西門慶道:「端的甚麽買賣,你說來。」李三道:「今有朝廷東京行下文書,天下十三省,每省要幾萬兩銀子的古器。咱這東平府,坐派着二萬兩,批文在巡按處,還未下來。如今大街上張二官府,破二百兩銀子幹這宗批要做,都看有一萬兩銀子尋。小人會了二叔,敬來對老爹說。老爹若做,張二官府拿出五千兩來,老爹拿出五千兩來,兩家合着做這宗買賣。左右沒別人,這邊是二叔和小人與黄四哥,他那邊還有兩個夥計,二八分錢使。未知老爹意下何如?」西門慶問道:「是甚麽古器?」李三道:「老爹還不知,如今朝廷皇城内新蓋的艮嶽,改為壽岳,上面起蓋許多亭臺殿閣,又建上清寳籙宫、會眞堂、璇神殿;又是安妃娘娘梳粧閣;都用着這珍禽奇獸、周彝商鼎、漢篆秦爐、宣王石鼓、歷代銅鞮、僊人掌承露盤,並希世古董翫器擺設,好不大興工程,好少錢糧!」西門慶聽了,說道:「比是我與人家打夥兒做,我自家做了罷。敢量我拿不出這一二萬銀子來?」李三道:「得老爹全做,又好了!俺們就瞞着他那邊了。左右這邊二叔和俺們兩個,再沒人。」伯爵道:「哥,家裏還添個人兒不添?」西門慶道:「到跟前,再添上賁四替你們走跳就是了。」西門慶又問道:「批文在那裏?」李三道:「還在巡按上邊,沒發下來哩。」西門慶道:「不打緊,我這差人寫封書,封些禮,問宋松原討將來就是了。」李三道:「老爹若討去,不可遲滯。自古兵貴神速,先下米的先喫飯,誠恐遲了,行到府裏,乞別人家幹的去了。」西門慶笑道:「不怕他。設使就行到府裏,我也還敎宋松原拿囬去;就是胡府尹,我也認的。」於是留李三伯爵同喫了飯,約會:「我如今就寫書,明日差小价去。」李三道:「又一件,宋老爹如今按院不在這裏了。従前日起身,往兖州府盤查去了。」西門慶道:「你明日就同小价往兖州府走遭。」李三道:「不打緊,等我去,來囘破五六日罷了。老爹差那位管家?等我會下,有了書,教他往我那裏歇,明日我同他好早起身。」西門慶道:「別人你宋老爹不認的,他常喜的是春鴻,教春鴻來爵一起去罷。」於是呌他二人到面前,會了李三,晚夕往他家宿歇。伯爵道:「這等纔好。事要早幹。多才疾足者得之!」於是與李三喫畢飯,告辭而去。西門慶隨即教陳經濟寫了書,又封了十兩葉子黄金在書帕内,與春鴻來爵二人,吩咐路上仔細:「若討了批文,即便早來。若是行到府裏,問你宋老爹討張票,問府裏要。」來爵道:「爹不消吩咐,小的曾在兖州答應過徐參議,小的知道。」於是領了書禮,打在身邊,逕往李三家去了。

不說十一日來爵春鴻同李三早僱了長行頭口,往兖州府去了。卻說十二日,西門慶家中請各官堂飲酒,那日在家不出門,約下吳大舅、應伯爵、謝希大、常時節四位,晚夕來在捲棚内賞燈飲酒。王皇親家楽小廝,従早晨就挑了箱子來了,在前邊廂房做戯房。堂客到,打銅鑼銅鼓迎接。周守禦娘子有眼疾,不得來,差人來囬。又是荆統制娘子、張團練娘子、雲指揮娘子,並喬親家母、崔親家母、吳大姨、孟大姨,都先到了。只有何千戶娘子,王三官母親林太太並王三官娘子不見到。西門慶使排軍、玳安琴童兒來囬催邀了兩三遍,又使文嫂兒催邀。午間,只見林氏一頂大轎,一頂小轎跟了來。見了禮,請西門慶拜見。問:「怎的三官娘子不來?」林氏道:「小兒不在,家中沒人。」拜畢下來。止有何千戶娘子直到晌午大錯纔來,坐着四人大轎,一個家人媳婦坐小轎跟隨,排軍擡着衣箱,又是兩位青衣家人,緊扶着轎竿,到二門裏纔下轎。前邊鼓楽吹打迎接。吳月娘衆姊妹迎至儀門首。西門慶悄悄在西廂房放下簾來偸瞧見這藍氏年紀不上二十歲,生的長挑身材,打扮的如粉粧玉琢,頭上珠翠堆滿,鳳翹雙插,身穿大紅通袖五彩粧花四獸麒麟袍兒,繫着金鑲碧玉帶,下襯着花錦藍裙,兩邊禁步叮㖦,麝蘭香噴。但見:

\begin{myquote}
儀容嬌媚,體態輕盈。姿性兒百伶百俐,身段兒不短不長。細彎彎兩道蛾眉,直侵入鬢;滴溜溜一雙鳳眼,來往踅人。嬌聲兒似囀日流鶯,嫩腰兒似弄風楊柳。端的是綺羅隊裏生來,卻厭豪華氣象;珠翠叢中長大,倒堪雅淡梳粧。開遍海棠花,也不問夜來多少;飄殘楊柳絮,竟不知春色如何。要知他半點眞情,除非是穿綺窻皓月;能悉他一腔心事,卻便似翻繡幌清風。輕移蓮步,有蕊珠僊子之風流;欵蹙湘裙,似水月觀音之態度。正是:比花花解語,比玉玉生香!
\end{myquote}

這西門慶不見則已,一見魂飛天外,魄丧九霄,未曾體交,精魄先失。少頃,月娘等迎接進入後堂,相見叙禮已畢,請西門慶拜見。西門慶得不的這一聲,連忙整衣冠行禮,恍若瓊林玉樹臨凡,神女巫山降下,躬身施禮,心搖目蕩,不能禁止。拜見畢,下來。先在捲棚内放桌兒擺茶,極盡希奇美饌。然後大廳上坐,陳水陸珍羞。但見:

\begin{myquote}
正面設石崇錦帳圍屏,四下鋪玳筵廣席。花燈高挑,綵䋲半拽。雕梁錦帶低垂,畫燭齊明寳蓋。魚龍山戲,恍一片珠璣;殿閣樓臺,簇千團翡翠。左邊廂九姊十妹羙人,圖畫丹青;右首下九曜八洞神僊,粧成金碧。喫的是龍肝鳳髓、熊掌駝峯;歌的是錦瑟銀箏、鳳簫象管。鼉鼓鼕鼕驚過鳥,歌喉囀囀遏行雲。席上嬌嬈,盡是珠圍翠繞;堦下脚色,皆按離合悲歡。正是:得多少進酒丫鬟雙洛浦,獻羹侍妾兩嫦娥。
\end{myquote}

當下林太太上席,戲文扮的是《小天香半夜朝元記》,唱了兩摺下來。李桂姐、吳銀兒、鄭月兒、洪四兒,四個唱的上去彈唱,唱燈詞「錦繡花燈半空挑」。孟大姨門外,先起身去了。西門慶在捲棚内,自有吳大舅、應伯爵、謝希大、常時節,李銘吳惠鄭奉三個小優兒彈唱飲酒,不住下來大廳格子外往裏觀覷。這各家跟轎子家人伴當,自有酒饌,前廳管待,不必用說。

看官聽説:次第明月圓,容易彩雲散,楽極悲生,否極泰來,自然之理。西門慶但知爭名奪利,縱意奢淫,殊不知天道惡盈,鬼錄來追,死限臨頭。到晚夕,堂中點起燈來,小優兒彈唱燈詞。還未到起更時分,西門慶正陪着人坐的,就在席上齁齁的打起睡來。伯爵便行令猜枚鬼混他,説道:「哥,你今日沒高興,怎的只打睡?」西門慶道:「我昨日沒曾睡。不知怎的,今日只是沒精神,打睡。」只見四個唱的下來。伯爵教兩個唱燈詞,兩個遞了酒。當下洪四兒與鄭月兒兩個彈着箏琵琶唱,吳銀兒與李桂姐遞酒。正耍在熱鬧䖏,忽玳安來報:「王太太與何老爹娘子起身了。」這西門慶就下席來,黑影裏走到二門裏首,偸看着他上轎。月娘衆人送出來,前邊天井内看放煙火。藍氏穿着大紅遍地金貂鼠皮襖,翠藍遍地金裙;林太太是白綾襖兒,貂鼠披風,大紅裙,帶着金鐸玉珮,家人打着燈籠,簇擁上轎而去。這西門慶正是餓眼將穿,饞涎空嚥,恨不能就要成雙。見藍氏去了,悄悄従夾道進來。當時沒巧不成話,姻緣凑合,可霎作怪,不想來爵兒媳婦見堂客散了,正従後邊歸來,開他房門,不想頂頭撞見西門慶,沒處藏躱。原來西門慶見媳婦子生的喬樣,安心已久,雖然不及來旺妻宋氏風流,也頗充得過第二。於是乘着酒興兒,雙関摟進他房中親嘴。這老婆當初在王皇親家,因是養了主子,被家人不忿攘鬧,打發出來。今日又撞着這個道路,如何不従了?一面就遞舌頭在西門慶口中。兩個解衣褪褲,就按在炕沿子上,掇起腿來,被西門慶就聳了個不亦楽乎。正是:未曾得遇鶯娘面,且把紅娘去解饞。有詩為證:

\begin{myquote}
燈月交輝浸玉壺,分得清光照綠珠。

莫道使君終有婦,教人桑下覓羅敷。
\end{myquote}

畢竟未知後來何如,且聽下囬分解。

