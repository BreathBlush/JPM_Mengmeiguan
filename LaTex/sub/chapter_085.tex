\includepdf[pages={169,170},fitpaper=false]{tst.pdf}
\chapter*{第八十五囬 \\月娘識破金蓮奸情 薛嫂月夜賣春梅}
\addcontentsline{toc}{chapter}{第八十五囬 月娘識破金蓮奸情 薛嫂月夜賣春梅}
\markboth{{\titlename}卷之九}{第八十五囬 月娘識破金蓮奸情 薛嫂月夜賣春梅}


\begin{myquote}
人家養女甚無聊,倒踏來家更不合。

口稱爹媽虚情意,權當為兒假做作。

入户只嫌恩愛少,出門翻作怨仇多。

若有一些不到處,一日一場罵老婆。
\end{myquote}

話説吴大舅保月娘,有日取路來家,不題。

單表潘金蓮,自従月娘不在家,和陳經濟兩個,家中前院後庭,如鷄兒趕彈兒相似,纏做一處,無一日不會合。一日,金蓮眉黛低垂,腰肢寬大,終日懨懨思睡,茶飯懶嚥。呌經濟到房中說:「奴有件事告你說,這兩日眼皮兒懶待開,腰肢兒漸漸大,肚腹中捘捘跳,茶飯兒怕待喫,身子好生沉困。有你爹在時,我求薛姑子符薬衣胞,那等安胎,白沒見個踪影。今日他沒了,和你相交多少時兒,便有了孩子。我従三月内洗換身上,今方六個月,已有半肚身孕。往常時我排磕人,今日卻輪到我頭上,你休推睡裏夢裏,趁你大娘未來家,那裏討貼墜胎的薬,趁早打落了。這胎氣離了身,奴走一步也伶俐。不然弄出個怪物來,我就尋了無常罷了,再休想擡頭見人!」經濟聽了,便道:「咱家舖中諸樣薬都有,倒不知那幾樁兒墜胎,又沒方修合。你放心,不打緊處,大街坊胡太醫,他大小方脉婦人科,都善治,常在咱家看病。等我問他那裏贖取兩貼與你喫,打下胎便了。」婦人道:「好哥哥,你上緊快去,救奴之命!」

這陳經濟包了三錢銀子,逕到胡太醫家呌門。胡太醫正在家,出來相見聲喏。認的經濟是西門大官人女婿,讓坐說:「一向稀面。動問到舍有何見教?」經濟道:「別無干凟……」向袖中取出白金三星,「充薬資之禮,敢求下良劑一二貼,足見盛情。」胡太醫說道:「我家醫道,大方脉、婦人科、小兒科、内科、外科、加減十三方、壽域神方、海上方、諸般雜症方,無不通曉。又專治婦人胎前産後。且婦人以血為本,藏於肝,流於臟,上則為乳汁,下則為月水,合精而成胎氣。女子十四而天癸至,任脉通,故月候按時而行,常以三旬一見則無病。一或血氣不調,則陰陽愆伏。過於陽,則經水先期而來;過於陰,則經水後期而至。血性得熱而流,寒則凝滞。過與不及,皆致病也。冷則多白,熱則多赤。冷熱不調則赤白帶。大抵血氣和平,陰陽調順,其精血聚而胞胎成。心腎二脉,應手而動。精盛則為男,血勝則為女。此自然之理也。胎前必須以安胎為本,如無他疾,不可妄服薬餌,待十月分娩之時,尤當謹護。不然,恐生産後諸疾。愼之,愼之!」經濟笑道:「我不要安胎,我今只用墜胎薬。」胡太醫道:「天地之間,以好生為本。人家十個九個,只要安胎的薬,你如何倒要墜胎?没有沒有!」經濟見他掣肘,又添了二錢薬資說:「你休管他,各人家自有用處。此婦子女生落不順,情願下胎。」這胡太醫接了銀子,說道:「不打緊,我與你一服紅花一掃光。喫下去,如人行五里,其胎自落矣。有〔西江月〕為證:

\begin{myquote}
牛膝蟹瓜甘遂,定磁大戟芫花。斑螯赭石與碙砂,水銀芒硝研化。又加桃仁通草,麝香文帶凌花。更釅醋煮好紅花,管取孩兒落下。」
\end{myquote}

經濟於是討了兩貼紅花一掃光,作辭胡太醫,到家遞與婦人,一五一十說了。到晚夕,煎紅花湯喫下去,登時滿肚裏生疼,睡在炕上,敎春梅按住身,只情揉揣,可霎作怪,須臾坐淨桶,把孩子打下來了。只說身上來,令秋菊攪草紙倒將東淨毛司裏。次日掏坑的漢子挑出去,一個白胖的小廝兒!常言好事不出門,惡事傳千里,不消幾日,家中大小都知金蓮養女婿,偸出私肚子來了。

卻説吴月娘有日來家。——往囬泰安州,去了半個月光景,來時正値十月天氣。家中大小接着,如天上落下來的一般。月娘到家中,先到天地佛前炷了香,然後西門慶靈前拜罷,就對孟玉樓衆姊妹家中大小,把岱嶽廟中及山寨上的事,従頭告訴一遍,因大哭一場。合家大小都來參見了。月娘見奶子抱孝哥兒到跟前,子母相會在一處。燒紙,置酒管待吴大舅囬家。晚夕,衆姊妹與月娘接風,俱不在話下。到第二日,月娘路上風霜跋涉,着了辛苦,又喫了驚怕,身上疼痛沉困,整不好了兩三日。

那秋菊在家,把金蓮經濟兩人幹的勾當,聽的滿耳滿心。要去上房告月娘,說二人怎生偸出私肚子來,傾在毛司裏,乞掏坑的掏出去,何人不看見!又說被婦人怎生打罵,含恨正没發付處。走到上房門首,又被小玉噦駡在臉上,大耳刮子打在臉上,駡道:「賊說舌的奴才,趁早與我走!俺奶奶遠路來家,身子不快活,還未起來。趁早與我走,氣了他,倒値了多的!」罵的秋菊忍氣吞聲,喏喏而退。

一日,也是合當有事,經濟進來尋衣裳,婦人和他又在玩花樓上兩個做得好。被秋菊走到後邊,叫了月娘來看,說道:「奴婢兩番三次告大娘説不信。娘不在,兩個在家明睡到夜,夜睡到明,偸出私肚子來,與春梅兩個都打成一家。今日兩人又在樓上幹歹事,不是奴婢說謊,娘快些瞧去!」月娘急忙走到前邊,兩個正幹的好,還未下樓。不想金蓮房簷籠内馴養得個鸚哥兒會説嘴,高聲呌:「大娘來了!」春梅正在房中,聽見迎出來,見是月娘,比及上樓呌婦人。先是經濟拿衣服下樓往外走,被月娘喝罵了幾句,說:「小孩兒没記性,有要没緊進來撞甚麽?」經濟道:「舖子内人等着,没人尋衣裳。」月娘道:「我那等吩咐,教小廝進來取,如何又進來寡婦房裏,有要沒緊做甚麽?没廉耻!」幾句罵的經濟往外金命水命,走投無命。婦人羞的半日不敢下來。然後下來,被月娘儘力數說了一頓,説道:「六姐,今後再休這般沒廉恥!你我如今是寡婦,比不的有漢子。香噴噴在家裏,臭烘烘在外頭,盆兒罐兒都有耳朶。你有要沒緊和這小廝纏甚麽?教奴才們背地排說的硶死了!常言道:男兒没性,寸鐵無鋼;女人無性,爛如麻糖。其身正,不令而行;其身不正,雖令不行。你有長進正條,肯敎奴才排說你?在我跟前說了幾遍,我不信,今日親眼看見,說不的了!我今日說過,要你自家立志,替漢子爭氣。像我進香去,兩番三次,被強人擄掠逼勒,若是不正氣的,也來不到家了。」金蓮喫月娘數說,羞的臉上紅一塊白一塊,口裏說一千個沒有,只說:「我在樓上燒香,陳姐夫自去那邊尋衣裳,誰和他說甚話來?」當下月娘亂了一囬,歸後邊去了。

晚夕,西門大姐在房内又罵經濟:「賊囚根子,敢說又沒眞贜實犯拿住你?你還那等嘴巴巴的!今日兩個又在樓上做甚麽?說不的了!兩個弄的好硶兒,只把我合在缸底下一般。那淫婦要了我漢子,還在我跟前拿話兒栓縛人,毛司裏磚兒又臭又硬,恰似降伏着那個一般。他便羊角葱靠南牆老辣已定,你還在這屋裏雌飯喫!」經濟罵道:「淫婦,你家收着我銀子,我雌你家飯喫?」使性往前邊來了。自此已後,經濟只在前邊,無事不敢進入後邊來。取東取西,只是玳安平安兩個往樓上取去。每日飯食,晌午還不拿出來,把傅夥計餓的只拿錢街上盪麪喫。正是:龍鬦虎爭,苦了小獐。各處門户,日頭半天老早關了。由是與金蓮兩個恩情又間阻了。經濟那邊陳宅房子,一向教他母舅張團練看守居住。張團練革任在家閑住,經濟早晚往那裏喫飯去,月娘亦不追問。

兩個隔别,約一月不得會面。婦人獨在那邊,挨一日似三秋,過一宵如半夏,怎禁這空房寂靜,慾火如蒸?要見他一面難上之難。兩下音信不通,這經濟無門可入。忽一日,見薛嫂兒打門首所過,有心要托他寄一紙柬兒到那邊與金蓮,訴其間阻之事,表此肺腑之情。一日推門外討帳,騎頭口逕到薛嫂家。拴了騾子,掀簾便問:「薛媽在家?」有他兒子薛紀媳婦兒金大姐,抱孩子在炕上,伴着人家賣的兩個使女,聽見有人叫薛媽,出來問:「是誰?」經濟道:「是我。問薛媽在家不在?」金大姐道:「姑夫請家來坐!俺媽往人家兑了頭面,討銀子去了。有甚話說,使人呌去。」連忙點茶與經濟喫。少坐片時,只見薛嫂兒來了。同經濟道了萬福說:「姑夫,那陣風兒吹來我家!」呌金大姐:「倒茶與姑夫喫。」金大姐道:「剛纔喫了茶了。」經濟道:「無事不來。如此這般,我與五娘勾搭日久,今被秋菊丫頭戳舌,把俺兩個姻緣拆散。大娘與大姐甚是疎淡我。我與六姐拆散不開,二人離別日久,音信不通,欲捎寄數字進去與他,無人得到内裏。須央及你,如此這般,通個消息。」向袖中取出一兩銀子來,「這些微禮,權與薛媽買茶喫。」那薛嫂一聞其言,拍手打掌笑起來,說道:「誰家女婿戯丈母?世間那裏有此事!姑夫,你實對我說,端的你怎麽得手來?」經濟道:「薛媽禁聲,且休取笑。我有這柬帖封好在此,好歹明日替我送與他去。」薛嫂一手接了說:「你大娘従進香囬來,我還沒看他去,兩當一節,我去走走。」經濟道:「我在那裏討你信?」薛嫂道:「往舖子裏尋你囬話。」說畢,經濟騎頭口來家。

次日,卻說薛嫂提着花箱兒,先進西門慶家上房看月娘;坐了一囬,又到孟玉樓房中;然後纔到金蓮這邊。金蓮正放桌兒喫粥。春梅見婦人悶悶不楽,說道:「娘,你老人家也少要憂心。僊姑人說日日有夫,是非來入耳,不聽自然無。古昔僊人,還有小人不足之處,休說你我。如今爹也没了,大娘他養出個墓生兒來,莫不也來路不明?他也難管我你暗地的事。你把心放開,料天塌了,還有撑天大漢哩。人生在世,且風流了一日是一日。」於是篩上酒來,遞一鍾與婦人,說:「娘,且喫一盃兒煖酒,解解愁悶!」因見堦下兩隻犬兒交戀在一處,說道:「畜生尚有如此之楽,何况人而反不如此乎?」

正飲酒,只見薛嫂來到,向前道了萬福,笑道:「你娘兒兩個好受用。」因觀二犬戀在一處,又笑道:「你家好祥瑞!你娘兒們看看,怎不解許多悶?」於是又道個萬福。婦人道:「那陣風兒今日刮你來,怎的一向不來走走?」一面讓薛嫂坐。薛嫂兒道:「我鎭日不知幹的甚麽,只是不得閑。大娘頂上進了香來,也不曾看的他,剛纔好不怪我。西房三娘也在跟前,留了我兩對翠花,一對大翠圍髮,好快性,就秤了八錢銀子與我。只是後邊住的雪姑娘,従八月裏要了我二對線花兒,該二錢銀子來,説一些沒有支用着,白不與我。好慳吝的人!我對你說,怎的不見你老人家?」婦人道:「我這兩日身子有些不快,不曾出去走動。」春梅一面篩了一鍾酒,遞與薛嫂兒,薛嫂連忙道萬福說:「我進門就喫酒。」婦人道:「你到明日,養個好娃娃。」薛嫂兒道:「我養不的。俺家兒子媳婦兒金大姐,倒新添了個娃兒,纔兩個月來。」又道:「你老人家没了爹,終久這般冷清清了。」婦人道:「說不得,有他在好了。如今弄得俺娘兒們,一折一磨的。不瞞老薛說,如今俺家中人多舌頭多,他大娘自従有了這孩兒,把心腸兒也改變了,姊妹不似那咱親熱了。這兩日,一來我心裏不自在,二來因些閑話,没曾往那邊去。」春梅道:「都是俺房裏秋菊這奴才,大娘不在,霹空架了俺娘一篇是非,把我也扯在裏面,好不亂哩。」薛嫂道:「就是房裏使的那大姐?他怎的倒弄主子?自古穿青衣抱黑柱,這個使不的!」婦人使春梅:「你瞧瞧那奴才,只怕他來覷聽。」春梅道:「他在厨下揀米哩!這破包簍奴才,在這屋就是走水的槽,單管屋裏事兒往外學舌。」薛嫂道:「這裏沒人,咱娘兒們說話。嗔道昨日陳姐夫到我那裏,如此這般告訴我,乾淨是他戳犯你們的事兒了。陳姐夫說:他大娘數說了他,各處門户都緊了,不託他進來取衣裳拿薬材;又把大姐搬進東廂房裏住;每日晌午還不拿飯出去與他喫,餓的他只往他母舅張老爹那裏喫去。一個親女婿,不托他,倒托小廝,有這個道理?他有好一向沒得見你老人家,巴巴央及我,捎了個柬兒,多多拜上你老人家:少要焦心,左右爹也是沒了,爽利放倒身大做一做,怕怎的?點根香怕出煙兒,放把火倒也罷了!」於是取出經濟封的柬帖兒遞與婦人。拆開觀看,别無甚話,上冩〔紅繡鞋〕一詞:

\begin{myquote}
「祅廟火燒着皮肉,藍橋水淹過咽喉。緊按納風聲滿南州。畢罷了終是染污,成就了倒是風流。不恁麽也道有!

六姐(妝次)

\raggedleft{{\marktext(下書)}經濟百拜上。」}
\end{myquote}

婦人看畢,收了入袖中。薛嫂兒道:「他教你囘個記色與他,寫幾個字兒捎了去,方信我送的有個下落。」婦人教春梅陪着薛嫂喫酒,他進入房,半晌拿了一方白綾帕,一個金戒子兒。帕兒上也寫着一詞在上,說道:

\begin{myquote}
「我為你耽驚受怕,我為你折挫渾家。我為你脂粉不曾搽。我為你在人前抛了些見識,我為你奴婢上使了些鍬筏。咱兩個一雙憔悴殺!」
\end{myquote}

婦人冩了,封得停當,交與薛嫂,便說:「你上覆他,教他休要使性兒往他母舅張家那裏喫飯,惹他張舅唇齒,說你在丈人家做買賣,卻來我家喫飯!顯得俺們都是沒處活的一般,教他張舅怪。或是未有飯喫,教他舖户裏拿錢,買些點心和夥計喫便了。你使性兒不進來,和誰賭憋氣哩?恰似賊人膽兒虚一般!」薛嫂道:「等我對他說。」婦人又與薛嫂五錢銀子,作别出門。來到前邊舖子裏,尋見經濟。兩個走到僻靜處說話,把封的物事遞與他:「五娘說:教他休使性兒賭憋氣,教他常進來走走,休往你張舅家喫飯去,惹人家怪!」因拿出五錢銀子與他瞧:「此是裏面與我的,六眼不藏私,久後你兩個愁不會在一答裏對出來,我臉放在那裏?」經濟道:「老薛,多有累你。」深深與他唱喏。那薛嫂走了兩步,又囬來說:「我險些忘了一件事。剛纔我出來,大娘又使丫頭綉春呌進我去,呌我晚上來領春梅,要打發賣他。說他與你們做牽頭,和他娘通同養漢。敢就因這件事!」經濟道:「薛媽,你只顧領在家,我改日到你家見他一面,有話問他。」

那薛嫂說畢,囘家去了。果然到晚夕月上的時分,走來領春梅。到月娘房中,月娘開口說:「那咱原是你手裏十六兩銀子買的,你如今拿十六兩銀子來就是了。」吩咐小玉:「你看着,到前邊收拾了,敎他罄身兒出去,休要他帶出衣裳去了。」那薛嫂兒到前邊,向婦人如此這般:「他大娘教我領春梅姐來了。對我說,他與你老人家通同作獘,偸養漢子。不管長短,只問我要原價。」婦人聽見說領賣春梅,就睜了眼半日說不出話來,不覺滿眼落淚,呌道:「薛嫂兒,你看我娘兒兩個沒漢子的好苦也!今日他死了多少時兒,就打發他身邊人?他大娘這般沒人心仁義,自恃他身邊養了個尿胞種,就放人躧到泥裏!李瓶兒孩子週半還死了哩,花巴痘疹未出,知道天怎麽算計,就心高遮了太陽!」薛嫂道:「孩兒出了痘疹了没曾?」婦人道:「何曾出來了,還不到一週兒哩。」薛嫂道:「春梅姐說爹在日曾收用過他?」婦人道:「只收用過二字兒?死鬼把他當心肝肺腸兒一般看待!說一句聽十句,要一奉十,正經成房立紀老婆且打靠後。他要打那個小廝十棍兒,他爹不敢打五棍兒!」薛嫂道:「可又來,大娘差了!爹收用的恁個出色姐兒,打發他?箱籠兒也不與,又不許帶一件衣服兒,只教他罄身兒出去,隣舍也不好看的!」婦人道:「他對你說,休教帶出衣裳去?」薛嫂道:「大娘吩咐小玉姐,便來。教他看着,休教帶衣裳出去。」那春梅在傍,聽見打發他,一點眼淚也没有。見婦人哭,說道:「娘,你哭怎的!奴去了,你耐心兒過,休要思慮壞了。你思慮出病來,沒人知你疼熱的。等奴出去,不與衣裳也罷,自古好男不喫分時飯,好女不穿嫁時衣!」

正說着,只見小玉進來,說道:「五娘,你信我奶奶倒三顛四的!小大姐扶持你老人家一場,瞞上不瞞下,你老人家拿出他箱子來,揀上色的包與他兩套,敎薛嫂兒替他拿了去,做個一念兒,也是他番身一場。」婦人道:「好姐姐,你到有點仁義!」小玉道:「你看誰人保得常無事!蝦蟆促織兒,都是一鍬土上人。兔死狐悲,物傷其類。」一面拿出春梅箱子來,凡是戴的汗巾兒、翠簪兒,都教他拿去。婦人揀了兩套上色羅緞衣服鞋脚,包了一大包;婦人梯己與了他幾件釵梳簪墜戒子,小玉也頭上拔下兩根簪子來,遞與春梅。餘者珠子纓絡、銀絲雲髻、遍地金粧花裙襖,一件兒沒動,都擡到後邊去了。春梅當下拜辭婦人、小玉,洒淚而别。臨出門,婦人還要他拜辭拜辭月娘衆人,只見小玉搖手兒。這春梅跟定薛嫂,頭也不囘,揚長决裂出大門去了。小玉和婦人送出大門囘來。小玉到上房囘大娘,只說:「罄身子去了,衣服都留下没與他。」這金蓮歸進房中,往常有春梅,娘兒兩個相親相熱說知心話兒,今日他去了,丢得屋裏冷冷落落,甚是孤悽,不覺放聲大哭。有詩為證:

\begin{myquote}
耳畔言猶在,於今恩愛分。

房中人不見,無語自消魂。
\end{myquote}

畢竟未知後來如何,且聽下囘分解。

