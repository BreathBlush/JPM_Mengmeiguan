\includepdf[pages={187,188},fitpaper=false]{tst.pdf}
\chapter*{第九十四囬 \\劉二醉毆陳經濟 洒家店雪娥為娼}
\addcontentsline{toc}{chapter}{第九十四囬 劉二醉毆陳經濟 洒家店雪娥為娼}
\markboth{{\titlename}卷之十}{第九十四囬 劉二醉毆陳經濟 洒家店雪娥為娼}


\begin{myquote}
花開不擇貧家地,月照山河到處明。

世間只有人心歹,萬事還敎天養人。

癡聾瘖瘂家豪富,伶俐聰明卻受貧:

年月日時該載定,算來由命不由人。
\end{myquote}

話説陳經濟自從陳三兒引到謝家大酒樓上,見了馮金寳,兩個又勾搭上前情。往後沒三日不和他相會,或一日經濟廟中有事不去,金寳就使陳三兒捎寄物事,或寫情書來叫他去。一次,或五錢,或一兩。以後日間供其柴米,納其房錢。歸到廟中便臉紅,任道士問他何䖏喫酒來,經濟只說:「在米舖和夥計暢飲三盃解辛苦來。」他師兄金宗明,又替他遮掩,晚夕和他一䖏盤弄那勾當,是不必說。朝來暮往,把任道士囊篋中細軟財本也抵盗出大半,花費了不知覺。

一日,也是合當有事。這洒家店的劉二,有名坐地虎。他是帥府周守備府中親隨張勝的小舅子,專一在馬頭上開娼店,倚強凌弱,擧放私債與窠窝中各娼門人使用,加三討利。有一日不給,搗換文書,將利作本,利上加利。嗜酒行兇,人不敢惹他。就是打粉頭的班頭,欺酒客的領袖。因見陳經濟是晏廟任道士的徒弟,白臉小廝,在謝家大酒樓上把粉頭鄭金寳兒包占住了,喫的楞楞睜睜,提着碗來大小拳頭,走來謝家樓下,問金寳在那裏。慌的謝三郎連忙聲喏,說道:「劉二叔,他在樓上第二個閣兒裏便是。」這劉二大扠步上樓來。經濟正與金寳在閣兒裏面,兩個飲酒,做一處快活,只把房門關閉,外邊簾子掛着。被劉二一把手扯下簾子,大叫:「金寳兒出來!」唬的陳經濟鼻口内氣兒也不敢出。這劉二用脚把門跥開,金寳兒只得出來相見説:「劉二叔叔,有何說話?」劉二罵道:「賊淫婦,你少我三個月房錢,卻躱在這裏,就不去了!」金寳笑嘻嘻說道:「二叔叔,你家去,我使媽媽就送房錢來。」被劉二只摟心一拳,打了老婆一跤,把頭顱搶在堦沿下磕破,血流滿地。罵道:「賊淫婦,還等甚送來,我如今就要!」看見陳經濟在裏面,走向前,把桌子只一掀,碟兒打得粉碎。那經濟便道:「阿呀!你是甚麽人,走來撒野!」劉二罵道:「我㒲你道士秫秫娘!」手採過頭髮來,按在地下,拳捶脚踢無數。那樓上喫酒的人看着,都立睜了。店主人謝三郎初時見劉二醉了,不敢惹他,次後見打得人不像模樣,上樓來解勸說道:「劉二叔,你老人家息怒。他不曉得你老人家大名,悞言衝撞,休要和他一般見識。看小人薄面,饒他去罷。」這劉二那裏依従,儘力把經濟打個發昏章第十一。叫將地方保甲,一條䋲子,連粉頭都拴在一處墩鎖,吩咐:「天明早解到老爺府裏去!」原來守備勑書上命他保障地方,巡捕盗賊,兼管河道。這裏㧱了經濟,任道士廟中還尚不知,只說他晚夕米舖中上宿未囬。

卻説次日,地方保甲巡河快手押解經濟金寳,僱頭口騎上,趕清晨早到府前伺候。先遞手本與兩個管事張勝李安看了,說是劉二叔地方喧鬧一起,晏公廟道士一名陳經濟,娼婦鄭金寳。衆軍牢都問他要錢,說道:「俺們是廳上動刑的,一班十二人,隨你罷。正徑兩位管事的,你倒不可輕視了他!」經濟道:「身邊銀錢倒有,都被夜晚劉二打我時,被人掏摸的去了。身上衣服都扯碎了,那得錢來?止有頭上關頂一根銀簪兒,拔下來與二位管事的罷。」衆牢子拿着那根簪子,走來對張勝李安如此這般說:「他一個錢兒不拿出來,止與了這根簪兒,還是鬧銀的。」張勝道:「你叫他近前,等我審問他。」衆軍牢不一時推擁他到跟前跪下問:「你是任道士第幾個徒弟?」經濟道:「第三個徒弟。」又問:「你今年多大年紀?」經濟道:「廿四歲了。」張勝道:「你這等年少,只宜在廟中做道士,習學經典,許你在外宿娼飲酒喧嚷?你把俺老爺帥府衙門,當甚麽些小衙門,不拿個錢兒來?這根簪子,打水不渾,要他做甚?」還掠與他去。吩咐牢子:「等住囬老爺升廳,把他放在頭一起!眼看這狗男女道士,就是個吝錢的。只許你白要四方施主錢糧?休説你為官事,你就來喫酒赴席,也帶方汗巾兒揩嘴。等動刑時,着實加力拶打這廝!」又把鄭金寳呌上去。鄭家有忘八跟着,上下打發了三四兩銀子。張勝說:「你係娼門,不過趂熟覓些衣飯為主,沒甚大事。看老爺喜怒不同,若惱,只是一兩拶子;若喜懽,只恁放出來也不定。」旁邊那個牢子說:「你再把與我一錢銀子,等若拶你,待我饒你兩個大指頭。」李安吩咐:「你带他遠些伺候,老爺將次出廳。」不一時,只見裏面雲板響,守備升廳,兩邊僚掾軍牢森列,甚是齊整。但見:

\begin{myquote}
緋羅繳壁,紫綬卓圍。當廳額掛茜羅,四下簾垂翡翠。勘官守正,戒石上刻御製四行;人従謹廉,鹿角旁插令旗兩面。軍牢沉重,僚掾威儀。執大棍授事立堦前,挾文書廳旁聽發放。雖然一路帥臣,果是滿堂神道!
\end{myquote}

當時没巧不成話。也是五百劫寃家聚會,姻緣合當凑着。春梅在府中,從去歲八月間,已生了個哥兒小衙内;今方半歲光景,貌如冠玉,唇若塗硃。守備喜似掌上之珍,愛如無價之寳。未幾大奶奶下世,守備就把春梅册正,做了夫人,就住着五間正房。買了兩個養娘抱奶哥兒,一名玉堂,一名金匱;兩個小丫鬟伏侍,一個名喚翠花,一個名喚蘭花。又有兩個身邊得寵彈唱的姐兒,都十六七歲,一名海棠,一名月桂,都在春梅房中侍奉。那孫二娘房中,止使着一個丫鬟,名喚荷花兒,不在話下。此時小衙内只要張勝懷中抱他外邊頑耍,遇着守備升廳,在旁邊觀看。

當日守備升廳坐下,放了告牌出去,各地方解進人來。頭一起正叫上陳經濟並娼婦鄭金寳兒去。守備看了呈狀,又見經濟面上帶傷,説道:「你這廝是個道士,不守那清規,如何宿娼飲酒,騷擾我地方?行止有虧!左右拿下去打二十棍,追了度牒還俗。那娼婦鄭氏,拶一拶,敲五十敲,責令歸院當差。」兩邊軍牢向前,纔待扯翻經濟,攤去衣服,用䋲索綁起,輪起棍來,兩邊招呼打時,可霎作怪,張勝抱着小衙内正在廳前月臺上站立,走過來觀看,那小衙内看見打經濟,便在懷裏攔不住撲着要經濟抱。張勝恐怕守備看見,忙走過來,那小衙内一發大哭起來,直哭到後邊春梅跟前。春梅問他怎的哭,張勝便說:「老爺廳上發放事,打那晏公廟姓陳道士,他就撲着要他抱,小的走下來,他就哭了。」這春梅聽見是姓陳的,不免輕移蓮步,欵蹙湘裙,走到軟屏後面,探頭觀覷,「廳下打的那人,聲音模樣,倒好似陳姐夫一般。他因何出家做了道士?」又叫過張勝,問他:「此人姓甚名誰?」張勝道:「這道士供狀上年廿四歲,俗名呌陳經濟。」春梅暗道:「正是他了!」一面使張勝:「請下你老爺來。」這守備廳上打經濟,纔打到十棍,一邊還拶着唱的,忽聽後邊夫人有請,吩咐牢子把棍且擱住休打,一面走下廳來,春梅説道:「你打的那道士,是我姑表兄弟,看奴面上,饒了他罷。」守備道:「夫人不早説,我已打了他十棍,怎生奈何?」一面出來吩咐牢子:「都與我放了。」唱的便歸院去了。守備悄悄使張勝:「叫那道士囬來,且休去。問了你奶奶,請他相見。」這春梅纔待使張勝請他到後堂相見,忽然想起一件事來,口中不言,心内暗道:「剜去眼前瘡,安上心頭肉。眼前瘡不去,心頭肉如何安得上?」於是吩咐張勝:「你且叫那人去着,等我慢慢再呌他。」度牒也不曾追。

這陳經濟打了十棍,出離了守備府,還奔來晏公廟。不想任道士聽見人來說:「你那徒弟陳宗羙,在大酒樓上包着唱的鄭金寳兒,惹了洒家店坐地虎劉二,打得臭死,連老婆都拴了,解到守備府裏去了。行止有虧,便差軍牢來拿你去審問,追度牒還官!」這任道士聽了,一者年老的着了驚怕,二者身體胖大,因打開囊篋内又沒了細軟東西,着了口重氣,心中痰湧上來,昏倒在地。衆徒弟慌忙向前扶救,請將醫者來,灌下薬去,通不省人事,到半夜嗚呼斷氣身亡,亡年六十三歲。第二日陳經濟來到,左近隣人說:「你還敢廟裏去?你師父因為你如此這般,得了口重氣,昨夜三更鼓死了!」這經濟聽了,唬的忙忙似喪家之犬,急急如漏網之魚,復囘清河縣城中來。正是:鹿隨鄭相應難辨,蝶化莊周未可知!

話分兩頭,卻説春梅一見經濟,方待留他,忽然心上想起一件事來,還使出張勝來,教經濟且去着。走歸房中,摘了冠兒,脱了繡服,倒在牀上,一面捫心撾被,聲疼叫喚起來。唬的合宅大小都慌了。下房孫二娘來問道:「大奶奶纔好好的,怎的來就不好起來?」春梅説:「你們且去,休管我。」落後守備退廳進來,見他躺在牀上叫喚,也慌了,扯着他手兒問道:「你心裏怎的來?」也不言語。又問:「那個惹着你來?」也不做聲。守備道:「莫不剛纔見我打了你兄弟,你心内惱麽?」亦不應答。這守備無計奈何,自出外邊麻犯起張勝李安來了:「你那兩個,早知他是你奶奶兄弟,如何不早對我說?卻敎我打了他十下,惹的你奶奶心中不自在起來。我曾敎你留下他,請你奶奶相見,你如何又放他去了?你這廝們都討分曉!」張勝說:「小的曾禀過奶奶來,奶奶説且敎他去着,小的纔放他去了。」一面走入房中,哭啼哀告春梅:「望乞奶奶在爺前方便一言,不然,爺要見責小的們哩。」這春梅睜圓星眼,剔起蛾眉,叫過守備近前說:「我自心中不好,干他們甚事?那廝他不守本分,在外邊做道士。且奈他些時,等我慢慢招認他。」這守備纔不麻犯張勝李安了。

守備見他只顧聲喚,又使張勝請下醫官來看脉,說:「老夫人染了六慾七情之病,着了重氣在心。」討將薬來,又不喫,都放冷了。丫頭們都不敢向前説話,請將守備來看着喫薬,只呷了一口,就不喫了。守備出去了,大丫鬟月桂拿過薬來:「請奶奶喫薬。」被春梅拿過來,劈臉只一潑,罵道:「賊浪奴才,你只顧拿這苦水來灌我怎的!我肚子裏有甚麽!」敎他跪在面前。孫二娘走來問道:「月桂怎的,奶奶敎他跪着?」海棠道:「奶奶因他拿薬與奶奶喫來。奶奶說,『我肚子裏有甚麽,㧱這藥來灌我!』敎他跪着。」孫二娘道:「奶奶,你委的今一日沒曾喫甚麽,這月桂他不曉得。奶奶休打他,看我面上,饒他這遭罷。」吩咐海棠:「你往厨下熬些粥兒來,與你奶奶喫口兒。」春梅於是把月桂放起來。

那海棠走到厨下,用心用意,熬了一小鍋粳小米濃濃的粥兒,定了四碟小菜兒,用甌兒盛着,象牙筷兒,熱烘烘拿到房中。春梅躺在牀上,面朝裏睡,又不敢呌,直待他翻身,方纔請他:「有個粥兒在此,請奶奶喫粥。」春梅把眼合着,不言語。海棠又呌道:「粥晾冷了,請奶奶起來喫粥。」孫二娘在旁説道:「大奶奶,你這半日沒喫甚麽。這囘你覺好些?且起來喫些個,有柱戧些。」那春梅一𥑮碌子爬起來,敎奶子拿過燈來,取粥在手,只呷了一口,往地下只一推,早是不曾把家伙打碎,被奶子接住了,就大吆喝起來,向孫二娘説:「你平白呌我起來喫粥,你看賊奴才熬的好粥,我又不坐月子,熬這照面湯來與我喫怎麽?」吩咐奶子金匱:「你與我把這奴才臉上,把與他四個嘴巴!」當下眞個把海棠打了四個嘴巴。孫二娘便道:「奶奶,你不喫粥,卻喫些甚麽兒?卻不餓着你!」春梅道:「你呌我喫,我心内攔着喫不下去。」良久,叫過小丫鬟蘭花兒來吩咐道:「我心内想些鷄尖湯兒喫。你去厨房内,對着淫婦奴才,敎他洗手做碗好雞尖湯兒與我喫口兒。敎他多放些酸笋,做的酸酸辣辣的我喫。」孫二娘便說:「奶奶,吩咐他敎雪娥做去。你心下想喫的,就是薬。」

這蘭花不敢怠慢,走到厨下對雪娥說:「奶奶敎你做鷄尖湯,快些做,等着要喫哩!」原來這鷄尖湯,是雛鷄脯翅的尖兒,碎切的做成湯。這雪娥一面洗手剔甲,旋宰了兩隻小鷄,退刷乾淨,剔選翅尖,用快刀碎切成絲,加上椒料、葱花、芫荽、酸笋、油醬之類,揭成清湯。盛了兩甌兒,用紅漆盤兒,熱騰騰,蘭花㧱到房中。春梅燈下看了,呷了一口,怪呌大罵起來:「你對那淫婦奴才說去,做的甚麽湯!精水寡淡,有些甚味!你們只敎我喫,平白敎我惹氣!」慌的蘭花生怕打,連忙走到廚下,對雪娥說:「奶奶嫌湯淡,好不罵哩。」這雪娥一聲兒不言語,忍氣吞聲,從新坐鍋,又做了一碗。多加了些椒料,香噴噴敎蘭花拿到房裏來。春梅又嫌忒鹹了,拿起來照地下只一潑,——早是蘭花躱得快,險些兒潑了一身,——罵道:「你對那奴才說去,他不憤氣做與我喫,這遭做的不好,敎他討分曉哩!」這雪娥聽見,千不合萬不合,悄悄說了一句:「姐姐幾時這般大了,就抖摟起人來!」不想蘭花囘到房裏,告春梅說了。這春梅不聽便罷,聽了此言,登時柳眉剔豎,星眼圓睜,咬碎銀牙,通紅了紛面,大呌:「與我採將那淫婦奴才來!」須臾,使了養娘丫鬟三四個,登時把雪娥拉到房中,春梅氣狠狠的,一手扯住他頭髮,把頭上冠子跥了,罵道:「淫婦奴才,你怎的說『幾時這般大』?不是你西門慶家擡擧的我這般大!我買將你來,伏侍我,你不憤氣,敎你做口子湯,不是精淡,就是苦丁子鹹!你倒還對着丫頭說我『幾時恁般大起來,摟搜索落我!』要你何用?」一面請將守備來:「採雪娥出去,當天井跪着!前邊叫將張勝李安,旋剝褪去衣裳,打三十大棍!」兩邊家人點起明晃晃燈籠,張勝李安各執大棍伺候。那雪娥只是不肯脫衣裳。守備恐怕氣了他,在跟前不敢言語。孫二娘在旁邊再三勸道:「隨大奶奶吩咐打他多少,免褪他小衣罷!不爭對着下人脱去他衣裳,他爺體面上不好看的!只望奶奶高擡貴手,委的他的不是了!」春梅不肯,定要去他衣服打,說道:「那個攔我,我把孩子先摔殺了,然後我也一條繩子吊死就是了!留着他便是了。」於是也不打了,一頭撞倒在地,就直挺挺的昏迷,不省人事。守備唬的連忙扶起說道:「隨你打罷,沒的氣着你!」當下可憐把這孫雪娥拖翻在地,褪去衣服,打了三十大棍,打的皮開肉綻。一面使小牢子半夜叫將薛嫂兒來,即時罄身領出去變賣。春梅把薛嫂兒叫在背地吩咐:「我只要八兩銀子,將這淫婦奴才好歹與我賣與娼門!隨你賺多少,我不管你。你若賣在别處,我打聽出來,只休要見我!」那薛嫂兒道:「我靠那裏過日子,卻不依你說?」當夜領了雪娥來家。

那雪娥悲悲切切,整哭到天明。薛嫂便勸道:「你休哭了,也是你的晦氣,寃家撞在一處。老爺見你到罷了,只恨你與他有些舊仇舊恨,折挫你。那老爺也做不得主兒,見他有孩子,須也依隨他。正經下邊孫二娘,還讓他幾分。常言:討米倒做了倉官,說不的了!你休氣哭。」雪娥收淚謝薛嫂:「只望早晚尋個好頭腦我去,自有飯喫罷。」薛嫂道:「他千萬吩咐,只教我把你送在娼門。我養兒養女,也要天理。等我替你尋個單夫獨妻,或嫁個小本經紀人家,養活得你來也罷。」那雪娥千恩萬福,謝了薛嫂。

過了兩日,只見隣住一個開店張媽,走來叫薛媽:「你這壁廂有甚娘子,怎的哭的悲切?」薛嫂便道:「張媽,請進來坐。」說道:「便是這位娘子,他是大人家出來的,因和大娘子合不着,打發出來,在我這裏嫁人。情願尋個單夫獨妻,免得惹氣。」張媽媽道:「我那邊下着一個山東賣綿花客人,姓潘,排行第五,年三十七歲,幾車花菓,常在老身家安下。前日說他家有個老母有病,七十多歲,死了渾家半年光景,没人扶侍。再三和我說,替他保頭親事,並無相巧的。我看來,這位娘子年紀到相當,嫁與他做個娘子罷。」薛嫂道:「不瞞你老人家說,這位娘子,大人家出身,不拘粗細都做的。針指女工,鍋頭竃腦,自不必說,又做的好湯水。今纔三十五歲。本家只要三十兩銀子,倒好保與他罷。」張媽媽道:「有箱籠没有?」薛嫂道:「止是他隨身衣服簪環之類,並無箱籠。」張媽媽道:「旣是如此,老身囘去,對那人說,敎他自家來看一看。」説畢,喫茶,坐囘去了。晚夕對那人說了,次日飯罷以後,果然領那人來相看。一看,見了雪娥好模樣兒,年小,一口就還了二十五兩,另外與薛嫂一兩媒人錢。薛嫂也沒爭兢,就兑了銀子,寫了文書。晚夕過去,次日就上車起身。薛嫂敎人改換了文書,只兑了八兩銀子交到府中,春梅收了,只說賣與娼門去了。

那人娶雪娥到張媽家止過得一夜,到第二日五更時分,謝了張媽媽,作别上了車,逕到臨清去了。此是六月天氣,日子長,到馬頭上,纔日西時分,到於洒家店。那裏有百十間房子,都下着各處遠方來的窠子行院唱的。這雪娥一領進入一個門户,半間房子,裏面打着土炕,炕上坐着個五六十歲的婆子,還有個十七八頂老丫頭,打着盤頭揸髻,抹着鉛粉紅唇,穿着一弄兒軟絹衣服,在炕邊上彈弄琵琶。這雪娥看見,只呌得苦。纔知道那漢子潘五是個水客,買他來做粉頭,起了他個名兒叫玉兒。這小妮子名喚金兒,每日拿廝鑼兒出去酒樓上接客供唱,做這道路營生。這潘五進門不問長短,把雪娥先打了一頓,睡了兩日,只與他兩碗飯喫。敎他楽器,學彈唱,學不會又打,打得身上青紅遍了。引上道兒,方與他好衣穿,粧點打扮,門前站立,倚門獻笑,眉目嘲人。正是:遺踪堪入時人眼,不買胭脂畫牡丹!有詩為證:

\begin{myquote}
窮途無奔更無投,南去北來休便休。

一夜彩雲何䖏散,夢隨明月到青樓。
\end{myquote}

這雪娥在洒家店,也是天假其便,一日,張勝被守備差遣,往河下買幾十石酒麯,宅中造酒。這洒家店坐地虎劉二,看見他姐夫來,連忙打掃酒樓乾淨,在上等閣兒裏安排酒殽盃盤,各樣時新菓品,好酒活魚,請張勝坐在上面飲酒。酒博士保兒篩酒,近前跪下:「禀問二叔,下邊叫那幾個唱的上來遞酒?」劉二吩咐:「叫王家老姐兒、趙家嬌兒、潘家金兒、玉兒,四個上來伏侍你張姑夫。」酒博士保兒應諾下樓。不多時,只聽得胡梯畔笑聲,見一般兒四個唱的頂老,打扮得如花似朶,都穿着輕紗軟絹衣裳,上的樓來,望上一面花枝招颭,繡帶飄飄,拜了四拜,立在旁邊。這張勝猛睜眼觀看,内中一個粉頭,可霎作怪:「到像老爺宅裏小奶奶打發出來,厨下做飯的那雪娥娘子。他如何做這道路在這裏?」那雪娥亦眉眼掃見是張勝,都不做聲。這張勝便問劉二:「那個粉頭是誰家的?」劉二道:「不瞞姐夫,他是潘五屋裏玉兒金兒,這個是王老姐。一個是趙嬌兒。」張勝道:「王老姐兒我認的。這潘家玉兒,我有些眼熟。」因呌他近前,悄悄問他:「你莫不是老爺宅裏雪姑娘麽?怎生到於此䖏?」那雪娥聽見他問,便簇地兩行淚下,便道:「一言難盡!」如此這般,具說一遍:「被薛嫂擸瞞,把我賣了二十五兩銀子,賣在這裏供筵習唱,接客迎人!」這張勝平昔見他生的好,終是懷心。這雪娥席前殷勤勸酒,兩個說得入港。雪娥和金兒不免拿過琵琶來,唱了個詞兒,與張勝下酒,名〔四塊金〕:

\begin{myquote}
「前生想咱,少欠下他相思債。中途漾卻,綰不住同心帶。說着教我淚滿腮,悶來愁似海。萬誓千盟,到今何在?不良才,怎生消磨了我許多時恩愛!」
\end{myquote}

當下唱畢,彼此傳盃換盞,倚翠偎紅。常言:世財紅粉歌樓酒,誰為三般事不迷?喫得酒濃時,這張勝就把雪娥來愛了。兩個晚夕,留在閣兒裏就一處睡了。這雪娥枕邊風月,耳畔山盟,和張勝儘力盤桓,如魚似水,百般難述。次日起來,梳洗了頭面,劉二又早安排酒餚上來,與他姐夫扶頭,大盤大碗,饕食一頓。收起行裝,喂飽頭口,裝載米麵,伴當跟隨,臨出門與了雪娥三兩銀子,吩咐劉二:「好生看顧他,休教人欺負!」自此以後,張勝但來河下,就在洒家店與雪娥相會。往後走來走去,每月與潘五幾兩銀子,就包住了他,不許接人。那劉二自恁要圖他姐夫歡喜,連房錢也不問他要了。各窠窝刮刷將來,替張勝出包錢,包定雪娥柴米來。有詩為證:

\begin{myquote}
豈料當年縱意為,貪淫倚勢把心欺。

祸不尋人人自取,色不迷人人自迷。
\end{myquote}

畢竟未知後來如何,且聽下囘分解。

