\includepdf[pages={117,118},fitpaper=false]{tst.pdf}
\chapter*{第五十九囬 \\西門慶摔死雪獅子 李瓶兒痛哭官哥兒}
\addcontentsline{toc}{chapter}{第五十九囬 西門慶摔死雪獅子 李瓶兒痛哭官哥兒}
\markboth{{\titlename}卷之六}{第五十九囬 西門慶摔死雪獅子 李瓶兒痛哭官哥兒}


\begin{myquote}
日落水流西復東,春風不盡折何窮。

巫娥廟裏低含雨,宋玉門前斜帶風。

莫將楡莢共爭翠,深感杏花相映紅。

灞上漢南千萬樹,幾人遊宦別離中。
\end{myquote}

話説孟玉樓和潘金蓮,在門首打發磨鏡叟去了。忽見従東一人帶着大帽眼紗,騎着騾子,走得甚急,逕到門首下來。慌的兩個婦人往後走不迭。落後揭開眼紗,卻是韓夥計來家了。平安忙問道:「貨車到了不曾?」韓道國道:「貨車進城了。禀問老爹,卸在那裏?」平安道:「爹不在家,往周爺府裏喫酒去了。收拾了,教卸在對門樓上哩。你老人家請進裏邊去。」不一時,陳經濟出來,陪韓道國入後邊見了月娘。出來廳上,拂去塵土,把行李搭褳教王經送到家去。月娘一面打發出飯來,與他喫了。不一時,貨車纔到。經濟拿鑰匙開了那邊樓上門,就有卸車的小脚子領籌搬運,貨一箱箱堆卸在樓上。十大車緞貨,連家用酒米,直卸到掌燈時分。崔本也來幫扶照管。堆卸完畢,查數鎖門,貼上封皮,打發小脚錢出門。早有玳安往守備府報西門慶去了。西門慶聽見家中卸貨,喫了幾鍾酒,約掌燈以後就來家。韓夥計等着見了,在廳上坐的,悉把前後往回事説了一遍。西門慶因問:「錢老爹書下了?也見些分上不曾?」韓道國道:「全是錢老爹這封書,十車貨少使了許多税錢。小人把緞箱兩箱併一箱,三停只報了兩停,都當茶葉馬牙香櫃上税過來了。通共十大車貨,只納了三十兩五錢鈔銀子。錢老爹接了報單,也沒差巡攔下來查點,就把車喝過來了。」西門慶聽言,滿心歡喜。因説:「到明日,少不的重重買一分禮,謝那錢老爹!」於是吩咐陳經濟陪韓夥計崔大哥坐,後邊拿菜出來,㽞喫了一囬酒,方纔各散囬家。

王六兒聽見韓道國來了,王經替他馱行李搭褳來家,連忙接了行李,因問:「你姐夫來了麽?」王經道:「俺姐夫看着卸行李,還等着見俺爹纔來哩。」這婦人吩咐丫頭春香錦兒,伺候下好茶好飯。等的晚上韓道國到家,拜了家堂,脱了衣裳,淨了面目,夫妻二人各訴離情一遍。韓道國悉把買賣得意一節,告訴老婆。老婆又見搭褳内沉沉重重,許多銀兩,因問他;替己又帶了一二百兩貨物酒米,卸在門外店裏,慢慢發賣了銀子來家。老婆滿心歡喜道:「聽見王經説,又尋了個甘夥計做賣手,喒們和崔大哥與他同分利錢使,這個又好了。到出月開舖子。」韓道國道:「這裏使着了人做賣手,南邊還少個人立莊置貨。老爹一定還裁派我去。」老婆道:「你看貨材料,自古能者多勞。你若不會做買賣,那老爹託你麽?常言:不將辛苦藝,難得世人財。你外邊走上三年,……你若懶得去,等我對老爹説了,教姓甘的和保官兒打外,你便在家賣貨就是了。」韓道國道:「外邊走熟了,也罷了。」老婆道:「可又來,你先生迷了路,在家也是悶。」説畢,擺上酒來,夫婦二人飲了幾盃闊别之酒,收拾就寢。是夜歡娛無度,不必用説。次日卻是八月初一日,韓道國早到。西門慶教同崔本甘夥計在房子内看着收卸磚瓦木石,收拾裝修土庫,不在話下。

卻説西門慶卸完貨物,家中無事,忽然心中想起,要往鄭愛月兒家去。暗暗使玳安兒送了三兩銀子、一套紗衣服與他。鄭家鴇子聽見西門老爹來請他家姐兒,如天上落下來的一般,連忙收了禮物,沒口子向玳安道:「你多頂上老爹,就説他姐兒兩個都在家裏伺候老爹。請老爹早些兒下降。」玳安走來家中書房内回了西門慶話。西門慶約午後時分,吩咐玳安收拾着涼轎,頭上戴着坡巾,身上穿青緯羅暗補子直身,粉底皂靴。先走出房子,看了一囬裝修土庫,然後起身。坐上凉轎,放下斑竹簾來,琴童玳安跟隨,留王經在家,止着春鴻背着直袋,逕往院中鄭月兒家來。正是:

\begin{myquote}
天僊執手整香羅,入午光涵雪一窝。

不獨桃源能問渡,卻來月窟伴嫦娥。
\end{myquote}

卻説鄭愛香兒頭戴着銀絲䯼髻,梅花鈿兒,周圍金纍絲簪兒,打扮的粉面油頭,花容月貌;上着藕絲裳,下着湘紋裙,見西門慶到,笑吟吟在半門裏首迎接進去。到於明間客位,道了萬福。西門慶坐下,就吩咐小廝琴童:「把轎囬了家去,晚夕騎馬來接。」琴童跟轎家去不題,止留玳安和春鴻兩個伺候。良久,只見鴇子出來拜見,説道:「外日姐兒在宅内多有打攪。老爹家中悶的慌,來這裏自恁散心走走罷了,如何多計較又見賜將禮來?又多謝與姐兒的衣服。」西門慶道:「我那日呌他,怎的不去?只認王皇親家了!」鴇子道:「俺們如今還怪董嬌兒和李桂兒。不知是老爹生日呌唱,他們都有了禮,只俺們姐兒沒有。若早知時,也不答應王皇親家唱,先往老爹宅裏去了。老爹那裏呌唱在後,喒姐兒纔待收拾起身,只見王家人來,把姐兒的衣包拿的去。落後老爹那裏又差了人來,他哥子鄭奉又説:『你若不去,一時老爹動意,怒了。』慌的老身背着王家人,連忙攛掇姐兒,打後門起身上轎去了。」西門慶道:「先日我在他夏老爹家酒席上,已定下他了。他若那日不去,我不消説的就惱了。怎的他那日不言不語,不做喜歡,端的是怎的説?」鴇子道:「小行貨子家,自従梳弄了,那裏好生出去供唱去!到老爹宅内,見人多,不知唬的怎樣的。他従小是恁不出語,嬌養慣了。你看,甚時候纔起來!老身該催促了幾遍,説:『老爹今日來,你早些起來收拾了罷。』他不依,還睡到這早晚。」不一時,丫鬟拿茶上來,鄭愛香兒向前遞了茶,喫了。鴇子道:「請老爹到後邊坐罷。」原來鄭愛香兒家,門面四間,到底五層房子。轉過軟壁,就是竹槍籬,三間大院子,兩邊四間廂房。上首一明兩暗,三間正房,就是鄭愛月兒的房。——他姐姐愛香兒的房,在後邊第四層住。——但見簾櫳香靄,進入明間内,供養着一軸海潮觀音;兩旁掛四軸羙人,按春夏秋冬:惜花春起早,愛月夜眠遲,掬水月在手,弄花香滿衣。上面挂着一聯:「捲簾邀月入,諧瑟待雲來。」上首列四張東坡椅,兩邊安二條琴光漆春櫈。西門慶坐下,看見上面楷書「愛月軒」三字。

坐了半日,忽聽簾櫳響處,鄭愛月兒出來:不戴䯼髻,頭上挽着一窝絲杭州攢,梳的黑鬖鬖光油油的烏雲,露着四鬢,雲鬢堆縱,猶若輕煙密霧;都用飛金巧貼,帶着翠梅花鈿兒,周圍金纍絲簪兒齊插,後鬢鳳釵半卸;耳邊帶着紫瑛石墜子;上着白藕絲對衿僊裳,下穿紫綃翠紋裙,脚下露一雙紅鴛鳳嘴,胸前搖琱璫寳玉玲瓏。正面貼三顆翠面花兒,越顯那芙蓉粉面;四周圍香風縹緲,偏相襯楊柳纖腰。正是:若非道子觀音畫,定然延壽羙人圖。望上不當不正,與西門慶道了萬福,就用灑金扇兒掩着粉臉,坐在傍邊。西門慶注目停視,比初見時節兒越發齊整。不覺心搖目蕩,不能禁止。不一時,丫鬟又拿一道茶來。這粉頭輕搖羅袖,微露春纖,取一鍾茶過來,抹去盞邊水漬,雙手遞與西門慶。然後與愛香各取一鍾相陪。喫畢,收下盞託去,請寬衣服房裏坐。西門慶呌玳安上來,把上蓋青紗衣寬了,搭在椅子上,進入粉頭房中。但見:

\begin{myquote}
瑤窻以素紗罩,淡月半浸;綉幕以夜明懸,祥光高燦。正面黑漆縷金牀,牀上帳懸綉錦,褥隱華裀;旁設褆紅小几,几上博山小篆,靄沉檀香。文錦囊掛樓鼻壁上,象窯瓶插紫笋其中。牀前設兩張綉墊矮椅,旁邊放一對鮫綃錦帨。雲母屏,模寫淡濃之筆;鴛鴦榻,高閣古今之書。
\end{myquote}

西門慶坐下,但覺異香襲人,極其清雅,眞所謂神僊洞府,人跡不可到者也。彼此攀話之間,語言調笑之際,只見丫鬟進來安放桌兒。四個小翠碟兒,都是精製銀絲細菜,割切香芹,鱘絲、鰉鮓、鳳脯、鸞羹。然後拿上兩筯賽團圓、如明月、薄如紙、白如雪、香甜羙口、酥油和蜜餞麻椒鹽荷花細餅。鄭愛香兒與鄭愛月兒親手楝攢各樣菜蔬肉絲,捲就安放小泥金碟兒内,遞與西門慶喫。旁邊燒金翡翠甌兒,斟上苦艷艷桂花木樨茶。須臾,姊妹二人陪喫了餅,收下家伙去。揩抹桌席,鋪茜紅氈條,牀几上取了一個沉香雕漆匣,内盛象牙牌三十二扇,兩個與西門慶抹牌。當下西門慶出了個天地分——劔行十道,那愛香兒出了個地牌——花開蝶滿枝,那愛月兒出了個人牌——搭梯望月。須臾收過去,擺上酒來。但見盤堆異菓,酒泛金波。桌上無非是鵝鴨鷄蹄,烹龍炮鳳。珍菓人間少有,佳餚天上無雙。正是:舞回明月墜秦樓,歌遏行雲遮楚館。鴛鴦盃,翡翠盞,飲玉液,泛瓊漿。姊妹二人遞上酒去,在旁箏排雁柱,款跨鮫綃,當下鄭愛香兒彈箏,愛月兒琵琶,唱了一套「兜的上心來」。端的詞出佳人口,有裂石遶梁之聲。唱畢,又是十二碟菓仁減碟,細巧品類。姊妹兩個,促席而坐,㧱骰盆兒,二十個骰兒,與西門慶搶紅猜枚。

飲夠多時,鄭愛香兒推更衣出去了。獨有愛月兒陪着西門慶喫酒。先是西門慶向袖中取出白綾雙欄子汗巾兒,上一頭栓着三事挑牙兒,一頭束着金穿心盒兒。鄭愛月兒只道是香茶,便要打開。西門慶道:「不是香茶,是我逐日喫的補薬。我的香茶不放在這裏面,只用紙包兒包着。」於是袖中取出一包香茶桂花餅兒,遞與他。那月兒不信,還伸手往他這邊袖子裏掏。又掏出個紫縐紗汗巾兒,上栓着一副揀金挑牙兒,拿在手中觀看,甚是可愛。説道:「我見桂姐和吳銀兒都拿着這樣汗巾兒,原來是你與他的。」西門慶道:「是我揚州船上帶來的。不是我與他,誰與他的?你若愛,與了你罷。到明日,再送一副與你姐姐。」説畢,西門慶就着鍾兒裏酒,把穿心盒兒内薬喫了一服。把粉頭摟在懷中,兩個一遞一口兒飲酒咂舌,無所不至。西門慶又舒手向他身上摸弄他香乳兒,緊緊就就,賽麻團滑膩。一面攤開衫兒觀看,白馥馥猶如瑩玉一般。揣摩良久,淫心輒起,腰間那話,突然而興。解開褲帶,令他纖手籠揝。粉頭見其偉長粗大,唬的吐舌害怕。雙手摟定西門慶脖心,説道:「我的親親,你我今日初會,將就我,只放半截兒罷;若都放進去,我就死了。你敢喫薬養的這等大!不然,如何天生恁怪剌剌兒的,紅赤赤、紫漒漒,好呵磣人子!」西門慶笑道:「我的兒,你下去替我品品。」愛月兒道:「慌怎的,往後日子多如樹葉兒。今日初會,人生面不熟。再來,等我替你品。」説畢,西門慶欲與他媾歡。愛月兒道:「你不喫酒了?」西門慶道:「我不喫了,喒睡罷。」愛月兒便呌丫鬟把酒桌擡過一邊,與西門慶脱靴,打發先上牀睡;炷了香,放在薰籠内。他便就往後邊更衣澡牝去了。西門慶脱靴時,還賞了丫頭一塊銀子。良久婦人進房,問西門慶:「你喫茶不喫?」西門慶道:「我不喫。」一面掩上房門,放下綾綃來,將絹兒安在褥下,解衣上牀。兩個枕上鴛鴦,被中鸂鶒。西門慶見粉頭脱了衣裳,肌膚纖細,牝淨無毛,猶如白麵蒸餅一般,柔嫩可愛。抱了抱,腰肢未盈一掬,誠為軟玉溫香,千金難買。於是把他兩隻白生生銀條般嫩腿兒,來夾在兩邊腰眼間。那話上使了託子,向花心裏頂入。龜頭昂大,濡攪半晌,方纔沒稜。那鄭月兒把眉頭縐在一處兒,兩手攀閣在枕上,隱忍難挨,朦朧着星眼,低聲説道:「今日你饒了鄭月兒罷。」西門慶於是扛起他兩隻金蓮於肩膀上,肆行抽送,不勝歡娛。正是:得多少春點碧桃紅綻蕊,風欺楊柳緑翻腰。有詩為證:

\begin{myquote}
帶雨籠煙匝樹奇,妖嬈身勢似難支。

紅推西國無雙色,春占河陽第一枝。

濃豔正宜吟鄭子,功夫何用寫王維。

含情欲把芳心束,留住東風不放歸。
\end{myquote}

當下西門慶與鄭愛月兒留戀至三更方纔囬家。到次日,吳月娘打發他往衙門中去了,和玉樓金蓮李嬌兒都在上房坐的。只見玳安進來上房取尺頭匣兒,往夏提刑家送生日禮去:四樣鮮餚,一壜酒,一疋金緞。月娘因問玳安:「你爹昨日坐轎子往誰家喫酒,喫到那早晚纔來家?想必又在韓道國家,望他那老婆去來?原來賊囚根子成日只瞞着我,背地替他幹這等繭兒!」玳安止道:「不是。他漢子來家,爹怎好去的。」月娘道:「不是那裏,卻是誰家?」那玳安又不説,只是笑。取了緞匣,送禮去了。潘金蓮道:「娘,你不消問這賊囚根子,他也不肯實説。我聽見説蠻小廝昨日也跟他爹去來。你只呌了蠻小廝來問他就是了。」一面把春鴻呌到跟前。金蓮問:「你昨日跟了你爹轎子去,在誰家喫酒來?你實説便罷,不實説,如今你大娘就要打你。」那春鴻跪下便道:「娘休打小的,待小的説就是來。小的和玳安琴童哥三個,跟俺爹従一座大門樓進去,轉了幾條街巷,到個人家,只半截門兒,都用鋸齒兒鑲了。門裏立着個娘娘,打扮的花花黎黎的。」金蓮聽見笑了,説道:「囚根子,一個院裏半門子也認不的了,趕着粉頭叫娘娘起來!」金蓮問道:「那個娘娘怎麽模樣?你認的他不認的?」春鴻道:「我不認的他。生的像菩薩樣,也像娘們頭上戴着這個假壳。進入裏面,一個年老白頭的阿婆出來,望俺爹拜了一拜。落後請到大後邊,竹籬笆進去,又是一位年小娘娘出來,不戴假壳。生的銀盆臉,瓜子面,搽的嘴唇紅紅的,陪着俺爹喫酒。」金蓮道:「你們都在那裏坐來?」春鴻道:「我在俺玳安琴童哥,便在阿婆房裏,陪着俺們喫酒並肉兜子來。」把月娘玉樓笑的了不得。因問道:「你認的他不認的?」春鴻道:「那一個好似在喒家唱的。」玉樓笑道:「就是李桂姐了。」月娘道:「原來摸到他家去了!」李嬌兒道:「俺家沒半門子,也沒竹槍籬。」金蓮道:「只怕你不知道。你家新安的半門子是的。」問了一囬,西門慶來家,往夏提刑家拜壽去了。

卻説潘金蓮,房中養活的一隻白獅子貓兒,渾身純白,只額兒上帶龜背一道黑,名喚「雪裏送炭」,又名「雪獅子」。又善會口啣汗巾兒,拾扇兒。西門慶不在房中,婦人晚夕常抱着他在被窝裏睡。又不撒尿屎在衣服上。婦人喫飯,常蹲在肩上喂他飯,呼之即至,揮之即去。婦人常喚他是「雪賊」。每日不喫牛肝乾魚,只喫生肉半斤,調養得十分肥壯,毛内可藏一鷄疍。甚是愛惜他,終日抱在膝上摸弄。不是生好意:因李瓶兒官哥兒平昔怕貓,尋常無人處,在房裏用紅絹裹肉,令貓撲而撾食。也是合當有事,官哥兒心中不自在,連日喫劉婆子薬,畧覺好些。李瓶兒與他穿上紅緞衫兒,安頓在外間炕上,鋪着小褥子兒頑耍。迎春守着,奶子便在旁拿着碗喫飯。不料金蓮房中這雪獅子,正蹲在護炕上。看見官哥兒在炕上,穿着紅衫兒一動動的頑耍。只當平日哄喂他肉食一般,猛然望下一跳,撲將官哥兒,身上皆抓破了。只聽那官哥兒呱的一聲,倒咽了一口氣,就不言語了,手脚俱被風搐起來。慌的奶子丢下飯碗,摟抱在懷,只顧唾噦,與他收驚。那貓還來趕着他要撾,被迎春打出外邊去了。

如意兒實承望孩子搐過一陣好了。誰想只顧常連,一陣不了一陣搐起來。李瓶兒人在後邊。一面使迎春:「後邊請娘去!哥兒不好了,風搐着哩,叫娘快來!」那李瓶兒不聽便罷,聽了正是驚損六葉連肝肺,唬壞三毛七孔心,連月娘慌的兩步做一步走,逕撲到房中。見孩子搐的兩隻眼直往上吊,通不見黑眼睛珠兒,口中白沫流出,咿咿猶如小鷄呌,手足皆動。一見,心中猶如刀割鎗刺一般,連忙摟抱起來,臉搵着他嘴兒,大哭道:「我的哥哥,我出去好好兒,怎麽的搐起來!」迎春與奶子悉把被五娘房裏貓所唬一節説了。那李瓶兒越發哭起來,説道:「我的哥哥,你緊不可公婆意,今日你只當脱不了打這條路兒去了!」月娘聽了,一聲兒沒言語。一面呌將金蓮來問他説:「是你屋裏的貓唬了孩子?」金蓮問:「是誰説的?」月娘指着:「是奶子和迎春説來。」金蓮道:「你看這老婆子這等張睛!俺貓在屋裏好好兒的臥着不是?你們亂道怎的,把孩子唬了,沒的賴人起來。爪兒只揀軟處捏,俺們這屋裏是好纏的!」月娘道:「他的貓,怎得來這屋裏?」迎春道:「每常也來這邊屋裏走跳。」那金蓮接過來道:「早是你説,每常怎的不撾他?可可今日兒就撾起來?你這丫頭,也跟着他恁張眉瞪眼兒六説白道的!將就些兒罷了,怎的要把弓兒扯滿了,可可兒俺們是恁沒時運來!」於是使性子抽身往房裏去了。

看官聽説:常言道,花枝葉下猶藏刺,人心怎保不懷毒?這潘金蓮平日見李瓶兒従有了官哥兒,西門慶百依百隨,要一奉十,每日爭姸競寵,心中常懷嫉妬不平之氣。今日故行此陰謀之事,馴養此貓,必欲唬死其子,使李瓶兒寵衰,教西門慶復親於己,就如昔日屠岸賈養神獒,害趙盾丞相一般。正是:

\begin{myquote}
湛湛青天不可欺,未曾擧意早先知。

休道眼前無報應,古往今來放過誰?
\end{myquote}

月娘衆人見孩子只顧搐起來,一面熬姜湯灌他,一面使來安兒快呌劉婆去。不一時,劉婆子來到,看了脈息,只顧跌脚,説道:「此遭驚唬重了,是驚風,難得過來。」急令快熬燈心薄荷金銀湯,取出一丸金箔丸來,向鍾兒内硏化。見牙關緊閉,月娘連忙拔下金簪兒來,撬開口,灌下去。劉婆道:「過得來便罷,如過不來,告過主家奶奶,必湏要灸幾蘸纔好。」月娘道:「誰敢躭?必湏還等他爹來,問了他爹。不然灸了,惹他來家吆喝。」李瓶兒道:「大娘,救他命罷!若等來家,只恐遲了。若是他爹罵,等我承當就是了。」月娘道:「孩兒是你的孩兒,隨你灸,我不敢張主。」當下劉婆子把官哥兒眉攢脖根兩手關尺並心口,共灸了五蘸,放他睡下。那孩子昏昏沉沉,直睡到日暮時分,西門慶來家,還不醒。那劉婆見西門慶來家,月娘與了他五錢銀子薬錢,一溜煙従夾道内出去了。西門慶歸到上房,月娘把孩子風搐不好對西門慶説了。西門慶連忙走到前邊來看視。見李瓶兒哭的眼紅紅的,問:「孩兒怎的風搐起來?」李瓶兒滿眼落淚,只是不言語。問丫頭奶子,都不敢説。西門慶又見官哥兒手上皮兒去了,灸的滿身火艾,心中焦燥,又走到後邊問月娘。月娘隱瞞不住,只得把金蓮房中貓驚唬之事説了:「劉婆子剛纔看,説是急驚風。若不針灸,難過得來。若等你來,又恐怕遲了。他娘母子主張,教他灸了孩兒身上五蘸。纔放下他睡了,這半日還未醒。」西門慶不聽便罷,聽了此言,三尸暴跳,五臟氣衝,怒従心上起,惡向膽邊生,直走到潘金蓮房中,不由分説,尋着貓,提溜着脚,走向穿廊,望石臺基輪起來只一摔,只聽響亮一聲,腦漿迸萬朶桃花,滿口牙零噙碎玉。正是:不在陽間擒鼠耗,卻歸陰府作狸僊。那潘金蓮見他拿出貓去摔死了,坐在炕上風紋也不動。待西門慶出了門,口裏喃喃呐呐罵道:「賊作死的強盜,把人裝出去殺了纔是好漢!一個貓兒碍着你𠳹屎,兇神也似走的來摔死了。他到陰司裏,明日還問你要命,你慌怎的!賊不逢好死變心的強盜!」

這西門慶走到李瓶兒房裏,因説奶子迎春:「我教你好生看着孩兒,怎的教貓唬了他,把他手也撾了?又信劉婆子那老淫婦,平白把孩子灸的恁樣的!若好便罷;不好,把這老淫婦拿到衙門裏,與他個兩拶!」李瓶兒道:「你看孩兒緊自不得命,你又是恁樣的。孝順是醫家,他也巴不得要好哩。」當下李瓶兒只指望孩兒好來,不料被艾火把風灸返於内,變為慢風。内裏抽搐的腸肚兒皆動,尿屎皆出,大便屙出五花顏色,眼目忽睜忽閉,終朝只是昏沉不省,奶也不喫了。李瓶兒慌了,到處求神問卜打卦,皆有兇無吉。月娘瞞着西門慶,又請劉婆子來家跳神。又請小兒科太醫來看,卻用接鼻散試之。「若吹在鼻孔内打喷㖒還看得;若無喷㖒出來,則看陰騭守他罷了。」於是吹下去,茫然無知,並無一個喷㖒出來。越發晝夜守着哭涕不止,連飲食都減了。

看看到八月十五日將近,月娘因他不好,連自家生日都回了不做。親戚内眷就送禮來,也不請。家中止有吳大妗子楊姑娘並大師父來相伴。那薛姑子和王姑子兩個在印經䖏爭分錢不平,又使性兒,彼此互相揭調。十四日賁四同薛姑子催討,將經卷挑將來,一千五百卷都完了。李瓶兒又與了一弔錢買紙馬香燭,十五日同陳經濟早往岳廟裏進香紙。把經來看着都散施盡了,走來回李瓶兒話。喬大戶家一日一遍使孔嫂兒來看。又擧薦了一個看小兒的鮑太醫來看,説道:「這個變成天弔客忤,治不得了。」白與了他五錢銀子,打發去了。灌下薬去也不受,還吐出來了。只是把眼合着,口中咬的牙格支支響。李瓶兒通衣不解帶,晝夜只摟在懷中,眼淚不乾的只是哭。西門慶也不往那裏去,每日衙門中來家,就進來看孩兒。

那時正値八月下旬天氣。李瓶兒守着官哥兒睡在牀上。桌上點着銀燈。丫鬟養娘都睡熟了。覷着滿窻月色,更漏沉沉,見那孩兒只是昏昏不省人事,一向愁腸萬結,離思千端。正是:人逢喜事精神爽,悶來愁腸磕睡多。但見:

\begin{myquote}
銀河耿耿,玉漏迢迢。穿窻皓月耿寒光,透戶涼風吹夜氣。鴈聲嘹喨,孤眠才子夢魂驚;蛩韻凄凉,獨宿佳人情緒苦。譙樓禁鼓,一更未盡一更敲;別院寒砧,千搗將殘千搗起。畫簷前叮噹鐵馬,敲碎仕女情懷;銀臺上閃爍燈光,偏照佳人長嘆。一心只想孩兒好,誰料愁來怪夢多。
\end{myquote}

當下李瓶兒臥在牀上,似睡不睡,夢見花子虚従前門外來,身穿白衣,恰活時一般。見了李瓶兒,厲聲駡道:「潑賊淫婦,你如何抵盜我財物與西門慶?如今我告你去也!」被李瓶兒一手扯住他衣袖,央及道:「好哥哥,你饒恕我則個!」花子虚一頓,撒手驚覺,卻是南柯一夢。醒來,手裏扯着卻是官哥兒的衣衫袖子。連噦了幾口道:「怪哉,怪哉!」一聽更鼓時,正打三更三點。這李瓶兒唬的渾身冷汗,毛髮皆竪起來。到次日西門慶進房來,把夢中之事告訴與西門慶。西門慶道:「知道他死到那裏去了!此是你夢想舊境。只把心來放正着,休要理他。你休害怕,如今我使小廝拿轎子接了吳銀兒,晚夕來與你做伴兒;再把老馮叫來,伏侍你兩個。」玳安打院裏接了吳銀兒來。

那消到日西時分,那官哥兒在奶子懷裏,只搐氣兒了。慌的奶子叫李瓶兒:「娘,你來看,哥哥這黑眼睛珠兒只往上翻。口裏氣兒,只有出來的,沒有進去的!」這李瓶兒走來,抱到懷中,一面哭起來,叫丫頭:「快請你爹去,你説孩子待断氣也!」可好常時節又走來説話,告訴「房子兒尋下了,門面兩間二層,大小四間,只要三十五兩銀子。」西門慶聽見後邊官哥兒重了,就打發常時節起身,説:「我不送你罷!改日我使人拿銀子和你看去。」急急走到李瓶兒房中。月娘衆人連吳銀兒大妗子都在房裏瞧着。那孩子在他娘懷裏,把嘴一口口搐氣兒。西門慶不忍看他,走到明間椅子上坐着,只長吁短嘆。那消半盞茶時,官哥兒嗚呼哀哉,断氣身亡。時八月廿三日申時也,只活了一年零兩個月。合家大小,放聲號哭。

那李瓶兒撾耳撓腮,一頭撞在地下,哭的昏過去,半日方纔甦省。摟着他大放聲哭叫道:「我的沒救星兒,心疼殺我了!寜可我同你一答兒裏死了罷!我也不久活於世上了!我的抛閃殺人的心肝,撇的我好苦也!」那奶子如意兒和迎春在旁,哭的言不得,動不得。西門慶即令小廝收拾前廳西廂房乾淨,放下兩條寬櫈,要把孩子連枕席被褥擡出去那裏挺放。那李瓶兒躺在孩兒身上,兩手摟抱着,那裏肯放。口口聲聲直叫:「沒救星的寃家,嬌嬌的兒,生揭了我的心肝去了!撇的我枉費辛苦,乾生受一場,再不得見你了,我的心肝!」月娘衆人哭了一囬,在旁勸他不住。西門慶走來,見他把臉抓破了,滚的寳髻鬅鬆,烏雲散亂,便道:「你看蠻的!他既然不是你我的兒女,乾養活他一場,他短命死了,哭兩聲丢開罷了。如何只顧哭不完?又哭不活他!你的身子也要緊。如今擡出去,好呌小廝請陰陽來看。那是甚麽時候?」月娘道:「這個也有申時前後。」玉樓道:「我頭裏怎麽説來,他管情還等他這個時候纔囬去。原是申時生,還是申時死。日子又相同,都是二十三日。只是月分差些,圓圓的一年零兩個月。」李瓶兒見小廝們伺候兩旁要擡他,又哭了。説道:「慌擡他出去怎麽的?大媽媽,你伸手摸摸,他身上還熱的。」呌了一聲:「我的兒嚛,你教我怎生割捨的你去?坑得我好苦也!」一頭又撞倒在地下,放聲哭道,有〔山坡羊〕為證:

\begin{myquote}
「呌一聲,青天你,如何坑陷了奴性命!叫一聲我的嬌兒呵,恨不的一聲兒就要把你呌應!也是前緣前世那世裏少欠下你寃家債不了,輪着我今生今世為你眼淚也抛流不盡。每日家吊膽提心,費殺了我心!従來我又不曾坑人陷人,蒼天如何恁不睜眼?非是你無緣,必是我那些兒薄倖。撇的我四不着地樹倒無陰來呵,竹籃打水勞而無功。呌了一聲痛腸的嬌生,奴情願和你陰靈路上一處兒行!」
\end{myquote}

當下李瓶兒哭了一囬,把官哥兒擡出停在西廂房内。月娘向西門慶計較:「還對親家那裏,並他師父廟裏説聲去。」西門慶道:「他師父廟裏明早去罷。」一面使玳安往喬大戶家説了。一面使人請了徐陰陽來批書。又拿出十兩銀子與賁四,教他快擡了一付平頭杉板,令匠人隨即趲造了一具小棺槨兒,就要入殮。喬宅那裏一聞來報,隨即喬大戶娘子就坐轎子來,進門就哭。月娘衆人都陪着大哭了一場,告訴前事一遍。不一時請了陰陽徐先生來到,看了説道:「哥兒還是正申時永逝。」月娘吩咐出來,教與他看看黑書。徐先生掐指尋紋,又檢閲了陰陽秘書,瞧了一囬,説道:「哥兒生時八字,生於政和丙申六月廿三日申時,卒於政和丁酉八月廿三日申時,月令丁酉,日干壬子,犯天地重丧,本家卻要忌忌哭聲。親人不忌。入殮之時,蛇龍鼠兔四生人避之則吉。又黑書上云:『壬子日死者,上應寳瓶宫,下臨齊地。』他前生曾在衮州蔡家作男子,曾倚力奪人財物,喫酒落魄,不敬天地六親,横事牽連,遭氣寒之疾,久臥牀蓆,穢汚而亡。今生為小兒,亦患風癇之疾。十日前被六畜驚去魂魄,又犯土司太歲,先亡攝去魂,死託生往鄭州王家為男子,後作千戶,壽六十八歲而終。」湏臾,徐先生看了黑書:「請問老爹,明日出去,或埋或化?」西門慶道:「明日如何出得去!三日念了經,到五日出去,墳上埋了罷。」徐先生道:「二十七日丙辰,合家本命都不犯。宣正午時掩土。」批畢書,一面就收拾入殮,已有三更天氣。李瓶兒哭着往房中尋出他幾件小道衣道髻鞋襪之類,替他安放在棺槨内。釘了長命釘,合家大小又哭了一場,打發陰陽去了。

次日,西門慶亂着,也沒往衙門中去。夏提刑打聽得知,早晨衙門散時,就來弔問,致賻慰懷。又差人對吳道官廟裏説知。到三日,請報恩寺八衆僧人在家誦經。吳道官廟裏並喬大戶家,俱備折桌三牲來祭奠。吳大舅、沈姨夫,門外韓姨夫、花大舅,都有三牲祭桌來燒紙。應伯爵、謝希大、溫秀才、常時節、韓道國、甘出身、賁地傳、李智、黄四,都鬦了分資,晚夕來與西門慶伴宿。打發僧人去了,呌了一起提偶的,先在哥兒靈前祭畢。然後西門慶在大廳上放桌席,管待衆人。那日院中李桂姐吳銀兒並鄭月兒三家都有人情來上紙。

李瓶兒思想官哥兒,每日黄懨懨,連茶飯兒都懶待喫。提起來只是哭涕,把喉音都哭啞了。西門慶怕他思想孩兒,尋了拙智,白日裏吩咐奶子丫鬟和吳銀兒相伴他,不離左右。晚夕西門慶一連在他房中歇了三夜,枕上百般解勸。薛姑子夜間又替他唸《楞嚴經》、《解寃咒》,勸他:「休要哭了,經上不説的好,改頭換面輪迴去,來世機緣莫想他。當世他不是你的兒女,都是宿世寃家債主,託生來,化財化物,騙劫財物。或一歲而亡,二歲而亡,三六九歲而亡。一日一夜,萬死萬生。《陀羅經》上不説的好:昔日有一婦人,常持《佛頂心陀羅經》,日以供養不缺。乃於三生之前,曾置毒薬殺害他命。此寃家不曾離於前後,欲求方便,致殺其母。遂以託蔭此身,向母胎中,抱母心肝,令母至生産之時,分解不得,萬死千生。及至生産下來,端正如法。不過兩歲,即便身亾。母思憶之,痛切號哭。遂即把他孩兒,抛向水中。如是三遍,託蔭此身,向母腹中,欲求方便,致殺其母。至第三遍,准前得生,向母胎中,百千計較,抱母心肝,令其母千生萬死,悶絶呌喚。准前得生下,特地端嚴,相貌具足。不過兩歲,又以身亾,母既見之,不覺放聲大哭。是何惡業因緣?准前抱孩兒直至江邊,已經數時,不忍抛棄。感得觀世音菩薩遂化作一僧,身披百衲,直至江邊。乃謂此婦人曰:『不用啼哭。此非是你男女,是你三生前寃家,三度託生,欲殺母不得。為緣你常持誦《佛頂心陀羅經》,並供養不缺,故殺汝不得。若你要見這寃家,但隨貧僧手指看之。』道罷,以神通力一指,其兒遂化作一夜叉之形,向水中而立。報言:『緣汝曾殺我來,我今故來報冤。蓋緣汝有大道心,常持《佛頂心陀羅經》,善神日夜擁護,所以殺汝不得。我已蒙觀世音菩薩受度了,従今永不與汝為寃。』道畢,沉水中不見。此女人兩淚交流,禮拜菩薩。歸家益修善事,後壽至九十七歲而終,轉女成男。不該我貧僧説,今你這兒子,必是宿世寃家,託來你蔭下,化物化財,要惱害你身。為緣你供養修持,即時捨了此經一千五百卷,有此功行,他投害你不得,今此離身,到明日再生下來纔是你兒女。」這李瓶兒聽了,終是愛緣不断。但提起來,輒流涕不止。

須臾過了五日光景,到廿七日早晨,僱了八名青衣白帽小童,大紅銷金棺輿,旛幢雲蓋,玉梅雪柳圍隨,前首大紅銘旌,題着「西門冢男之柩」。吳道官廟裏,又差了十二衆青衣小道童兒來,遶棺轉咒生神玉章,動清楽送殯。衆親朋陪西門慶穿素服走至大街東口,將及門上,纔上頭口。西門慶恐怕李瓶兒到墳上悲慟,不呌他去。只是吳月娘、李嬌兒、孟玉樓、潘金蓮、大姐,家裏五頂轎子,陪喬親家母、大妗子,和李桂姐、鄭月兒、吳舜臣媳婦鄭三姐,往山頭去。留下孫雪娥、吳銀兒,並個姑子在家,與李瓶兒做伴兒。那李瓶兒見不放他去,見棺材起身,送出到大門首,趕着棺材大放聲,一口一聲只呌:「不來家虧心的兒嚛!」呌的連聲氣破了。不防一頭撞在門底下,把粉額磕傷,金釵墜地。慌了吳銀兒與孫雪娥,向前搊扶起來,勸歸後邊去了。到了房中,見炕上空落落的,只有他耍的那壽星博浪鼓兒,還掛在牀頭上。一面想將起來,拍了桌子,由不的又哭了。〔山坡羊〕前腔為證:

\begin{myquote}
「進房來,四下靜,由不的我悄嘆。想嬌兒,哭的我肝腸兒氣断。想着生下你來我受盡了千辛萬苦,説不的偎乾就濕,成日把你躭心兒來看。教人氣破了心腸,和我兩個結寃。實承望你與我做主兒,團圓久遠。誰知道天無眼又把你殘生丧了,撇的我前不着村後不着店。明知我不久也命丧在黄泉來呵,喒娘兒兩個鬼門關上一處兒眠。呌了一聲我嬌嬌的心肝!皆因是前世裏無緣,你今生壽短!」
\end{myquote}

那吳銀兒在旁,一面拉着他手,勸説道:「娘,少哭了。哥哥已是拋閃了你去了,那裏再哭得活?你湏自解自嘆,休要只顧煩惱了。」雪娥道:「你又年少青春,愁到明日養不出來也怎的?這裏墻有縫,壁有眼,俺們不好説的。他使心用心,反累己身。誰不知他氣不忿你養這孩子?若果是他害了哥哥,來世教他一還一報,問他要命。不止你,我也被他話埋了幾遭哩!只要漢子常守着他便好。到人屋裏睡一夜兒,他就氣生氣死。早是前者你們都知道,漢子等閒不到我後邊。到了一遭兒,你看背地都亂唧喳成一塊。對着他姐兒們説我長道我短。拿個紙包兒裏包着哩!俺們也不言語,每日洗着眼兒看着他。這個淫婦,到明日還不知怎麽死哩!」李瓶兒道:「罷了,我也惹了一身病在這裏,不知在今日明日死也!和他也爭執不得了,隨他罷!」正説着,只見奶子如意兒向前跪下,哭道:「小媳婦有句話,不敢對娘説。今日哥兒死了,乃是小媳婦沒造化,只怕往後爹和大娘打發小媳婦出去。小媳婦男子漢又沒了,那裏投奔?」李瓶兒見他這般説,又心中傷痛起來,説:「我有那寃家在一日占用他一日,他豈有此話説?」便道:「怪老婆,你放心,孩子便沒了,我還沒死哩。總然我到明日死了,你恁在我手下一場,我也不教你出門。往後你大娘身子若是生下哥兒小姐來,你就接了奶,就是一般了。你慌亂的是些甚麽?」那如意兒方纔不言語了。這李瓶兒良久又悲慟哭起來。前腔:

\begin{myquote}
「想嬌兒,想的我,無顛無倒。盼嬌兒,除非是夢兒中來到。白日裏覩物傷情如刀剜了肺腑,到晚間睡醒來,再不見你在我這懷兒裏抱,由不的珎珠望下抛!你再不來在描金牀兒上睡着頑耍,你再不來在我手掌兒上引笑。你再不來相靠着我胸膛兒來呵,生把這熱突突心肝割上一刀。奴為你乾生受枉費了徒勞,稱願了別人,撇的我無有個下梢!」
\end{myquote}

雪娥與吳銀兒兩個在旁解勸了一囬,説道:「你肚中喫了些甚麽兒,這般只顧哭不完!」一面綉春後邊拿了飯來,擺在桌上,陪他喫。那李瓶兒怎生嚥得下去?只喫了半甌兒,就丢下不喫了。

西門慶在墳上,教徐先生畫了穴,把官哥兒就埋在先頭陳氏娘懷中,抱孫葬了。那日喬大戶山頭,並衆親戚,都有祭祀。就在新蓋捲棚管待,飲酒一日。來家,李瓶兒與月娘喬大戶娘子大妗子磕着頭,又哭了,向喬大戶娘子説道:「親家,誰似奴養的孩兒不氣長,短命死了。旣死了,你家姐姐做了望門寡,勞而無功。親家休要笑話。」那喬大戶娘子説道:「親家怎的這般説話?孩兒們各人壽數,誰人保得後來的事!常言:先親後不改。親家們又不老,往後愁沒子孫?須得慢慢來,親家也少要煩惱了。」説畢,作辭囬家去了。西門慶在前廳教徐先生灑掃,各門上都貼辟非黄符,「死者煞高三丈,向東北方而去,遇日遊神冲囘,不出,斬之則吉。親人勿避。」西門慶拿出一疋大布、二兩銀子,謝了徐先生,管待出門。晚夕入李瓶兒房中,陪他睡。夜間百般言語溫存。見官哥兒的戯耍物件都還在跟前,恐怕李瓶兒看見,思想煩惱,都令迎春拿到後邊去了。正是:

\begin{myquote}
思想嬌兒晝夜啼,寸心如割命懸絲。

世間萬般哀苦事,除非死別共生離。
\end{myquote}

畢竟未知後來何如,且聽下囬分解。

