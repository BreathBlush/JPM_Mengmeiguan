\includepdf[pages={1,2},fitpaper=false]{tst.pdf}
\chapter*{第一囬 \\景陽岡武松打虎 潘金蓮嫌夫賣風月}
\addcontentsline{toc}{chapter}{第一囬 景陽岡武松打虎 潘金蓮嫌夫賣風月}
\markboth{{\titlename}卷之一}{第一囬 景陽岡武松打虎 潘金蓮嫌夫賣風月}


詞曰:

\begin{myquote}
丈夫隻手把吳鉤,欲斬萬人頭。如何鐵石打成心性,卻為花柔?

請看項籍並劉季,一似使人愁。只因撞着虞姬戚氏,豪傑都休!
\end{myquote}

此一隻詞兒,單說着「情」、「色」二字,乃一體一用。故色絢於目,情感於心,情色相生,心目相視。亙古及今,仁人君子,弗能忘之。晉人云:情之所鍾,正在我輩。如磁石吸鐵,隔礙潛通。無情之物尚爾,何况為人终日在情色中做活計一節。湏而「丈夫隻手把吳鉤」,吳鉤,乃古劔也。古有干將、莫釾、太阿、吳鉤、魚腸、躅鏤之名。言丈夫心腸如鐵石,氣槪貫虹蜺,不免屈志於女人。

題起當時西楚霸王,姓項名籍,單名羽字。因秦始皇無道,南修五嶺,北築長城,東塡大海,西建阿房,並吞六國,坑儒焚典。因與漢王劉邦,單名季字,時二人起兵,席捲三秦,滅了秦國,指鴻溝為界,平分天下。因用范增之謀,連敗漢王七十二陣。只因寵着一個婦人,名喚虞姬,有傾城之色,載於軍中,朝夕不離。一旦被韓信所敗,夜走陰陵,爲追兵所逼。霸王欲向江東取救,因捨虞姬不得,又聞四面皆楚歌,事發,嘆曰:「力拔山兮氣蓋世,時不利兮騅不逝,騅不逝兮可奈何,虞兮虞兮奈若何!」歌畢,淚下數行。虞姬曰:「大王莫非以賤妾之故,有廢軍中大事?」霸王曰:「不然,吾與汝不忍相捨故耳!況汝這般容色,劉邦乃酒色之君,必見汝而納之。」虞姬泣曰:「妾寜以義死,不以苟生。」遂請王之寳劍,自刎而死。霸王因大慟,尋以自剄。史官有詩嘆曰:

\begin{myquote}
「拔山力盡霸圖隳,倚劍空歌不逝騅。

明月滿營天似水,那堪囬首别虞姬。」
\end{myquote}

那漢王劉邦,原是泗上亭長,提三尺劍,硭碭山斬白蛇起手,二年亡秦,五年滅楚,掙成天下。只因也是寵着個婦人,名喚戚氏夫人,所生一子,名趙王如意。因被呂后妬害,心甚不安。一日,高祖有疾,乃枕戚夫人腿而臥。夫人哭曰:「陛下萬歲後,妾母子何所託?」帝曰:「不難。吾明日出朝,廢太子而立爾子,意下如何?」戚夫人乃收淚謝恩。呂后聞之,密召張良謀計。良擧薦商山四皓,下來辅佐太子。一日,同太子入朝。高祖見四人鬚鬢皎白,衣冠甚偉,各問姓名。一名東園公,一名綺里季,一名夏黄公,一名甪里先生。因大驚問:「朕昔求聘諸公,如何不至,今日乃從吾兒所遊?」四皓答曰:「太子乃守成之主也。」高祖聞之,愀然不悦。比及四皓出殿,乃召戚夫人指示之曰:「我欲廢太子,况彼四人輔佐,羽翼已成,卒難搖動矣。」戚夫人遂哭泣不止。帝乃作歌以解之:

\begin{myquote}
「鴻鵠高飛兮,一擧千里。羽翼已就兮,横絶四海。横絶四海兮,又可奈何?雖有矰繳兮,尚安所施!」
\end{myquote}

歌訖,後遂不果立趙王矣。高祖崩世,呂后酒酖殺趙王如意,人彘了戚夫人,以除其心中之患。

詩人评此二君,评到個去處,說劉項者,固當世之英雄,不免爲二婦人以屈其志氣。雖然,妻之視妾,名分雖殊,而戚氏之祸,尤慘於虞姬。然則妾婦之道,以色事其丈夫,而欲保全首領於牖下,難矣。觀此二君,豈不是「撞着虞姬戚氏,豪傑都休」?有詩爲證:

\begin{myquote}
劉項佳人絕可憐,英雄無策庇嬋娟。

戚姬葬處君知否?不及虞姬有墓田。
\end{myquote}

說話的如今只愛説這「情」、「色」二字做甚?故士矜才則德薄,女衒色則情放。若乃持盈愼滿,則為端士淑女。豈有殺身之祸?今古皆然,貴賤一般。如今這一本書,乃虎中羙女,後引出一個風情故事來。一個好色的婦女,因與個破落戶相通,日日追歡,朝朝迷戀。後不免屍横刀下,命染黄泉,永不得着綺穿羅,再不能施朱傅粉。靜而思之,着甚來由?況這婦人他死有甚事?貪他的,斷送了堂堂六尺之軀;愛他的,丢了潑天關産業。驚動了東平府,大鬧了清河縣。端的不知誰家婦女?誰的妻小?後日乞何人占用?死於何人之手?正是:說時華嶽山峯歪,道破黄河水逆流!

話說宋徽宗皇帝政和年間,朝中寵信高楊童蔡四個奸臣,以致天下大亂,黎民失業,百姓倒懸,四方盗賊蜂起。罡星下生人間,攪亂大宋花花世界,四處反了四大寇。那四大寇?山東宋江、淮西王慶、河北田虎、江南方臘。皆轟州劫縣,放火殺人,僭稱王號。惟有宋江替天行道,專報不平,殺天下贜官汚吏豪惡刁民。

那時山東陽谷縣,有一人姓武,名植,排行大郎。有個嫡親同胞兄弟,名喚武松。其人身長七尺,膀闊三停。自幼有膂力,學得一手好槍棒。他的哥哥武大,生的身不滿三尺,爲人懦弱,又頭腦濁蠢可笑。平日本分,不惹是非。因時遭荒饉,將祖房兒賣了,與兄弟分居,搬移在清河縣居住。這武松因酒醉打了童樞密,單身獨自逃在滄州横海郡小旋風柴進莊上,——他那裏招攬天下英雄豪傑,仗義疎财,人號他做「小孟嘗君」柴大官人,乃是周朝柴世宗嫡派子孫,——那裏躱逃。柴進因見武松是一條好漢,收攬在莊上。不想武松就害起瘧疾來,住了一年有餘,因思想哥哥武大,告辭歸家。在路上行了幾日,來到清河縣地方。

那時山東界上,有一座景陽崗,山中有一隻弔睛白額虎,食得路絶人稀。官司杖限獵戶,擒捉此虎。崗子路上兩邊都有榜文,可敎過往經商,結夥成群於巳午未三個時辰過崗,其餘不許過崗。這武松聽了,呵呵大笑。就在路旁酒店内喫了幾碗酒。壯着膽,横拖着防身梢棒,踉踉蹌蹌大扠步走上崗來。不半里之地,見一座山神廟,門首貼着一張印信榜文。武松看時,上面寫道:

\begin{myquote}[\markfont]
「景陽崗上,有一隻大蟲,近來傷人甚多。現今立限各鄉里正並獵戶人等,打捕住時,官給賞銀三十兩。如有過往客商人等,可於巳午未三個時辰,結夥過崗。其餘時分,及單身客旅,白日不許過崗,恐被傷害性命不便。各宜知悉。」
\end{myquote}

武松喝道:「怕甚麽鳥!且只顧上崗去,看有甚大蟲。」武松將棒綰在脅下,一步步上那崗來。囘看那日色,漸漸下山。此正是十月間天氣,日短夜長,容易得晚。武松走了一會,酒力發作,遠遠望見亂樹林子,直奔過樹林子來。見一塊光撻撻的大青臥牛石,把那棒倚在一邊,放翻身體,恰待要睡,但見青天忽然起一陣狂風,看那風時,但見:

\begin{myquote}
無形無影透人懷,四季能吹萬物開,

就地撮將黄葉去,入山推出白雲來。
\end{myquote}

原來雲生從龍,風生從虎。那一陣風過處,只聽得亂樹背後黄葉刷刷的響,撲地一聲,跳出一隻弔睛白額斑斕猛虎來,猶如牛來大。武松見了,叫聲「阿呀」時,從青石上翻身下來,便提梢棒在手,閃在青石背後。那大蟲又饑又渴,把兩隻爪在地上跑了一跑,打了個歡翅,將那條尾剪了又剪,半空中猛如一個焦霹靂,滿山滿嶺盡皆振響。這武松被那一驚,把肚中酒都變做冷汗出了。說時遲,那時快,武松見大蟲撲來,只一閃,閃在大蟲背後。原來猛虎項短,回頭看人較難,便把前爪搭在地下,把腰胯一伸,掀將起來;武松只一躱,躱在側邊。大蟲見掀他不着,吼了一聲,把山崗也振動。把這鐵棒也似虎尾倒豎起來,只一剪,武松卻又閃過一邊。原來虎傷人,只是一撲、一掀、一剪,三般捉不着時,氣力已自没了一半。武松見虎没力,翻身囘來,雙手輪起梢棒,盡平生氣力,只一棒,——只聽得一聲響,簌簌地將那樹枝帶葉打將下來。原來不曾打着大蟲,正打在樹枝上,磕磕把那條棒折做兩截,只拿一半在手裏。這武松心中,也有幾分慌了。那虎便咆哮性發,剪尾弄風起來,向武松又只一撲,撲將來。武松一跳,卻跳回十步遠。那大蟲撲不着武松,把前爪搭在武松面前。武松將半截棒丢在一邊,乘勢向前,兩隻手撾住大蟲頂花皮,使力只一按。那虎急要掙扎,早沒了氣力。武松儘力撾定那虎,那裏肯放鬆。一面把隻脚望虎面上眼睛裏只顧亂踢。那虎咆哮,把身底下扒起兩堆黄泥,做了一個土坑。武松按在坑裏,騰出右手,提起拳頭來只顧狠打。儘平生氣力,不消半歇兒時辰,把那大蟲打死,躺卧着卻似一個綿布袋,動不得了。有古風一篇,單道景陽崗武松打虎。但見:

\begin{myquote}
景陽崗頭風正狂,萬里陰雲埋日光。

焰焰滿川紅日赤,紛紛遍地草皆黄。

觸目晚霞掛林藪,侵人冷霧滿穹蒼。

忽聞一聲霹靂響,山腰飛出獸中王;

昂頭踴躍逞牙爪,谷裏獐鹿皆奔降。

山中狐兔潛蹤跡,澗内獐猿驚且慌。

卞莊見後魂魄散,存孝遇時心膽亡。

清河壯士酒未醒,忽在崗頭偶相迎。

上下尋人虎饑渴,撞着猙獰來撲人。

虎來撲人似山倒,人去迎虎如巖傾。

臂腕落時墜飛砲,爪牙撾處幾泥坑。

拳頭脚尖如雨點,淋漓兩手鮮血染。

穢汚腥風滿松林,散亂毛鬚墜山崦。

近看千鈞勢未休,遠觀八面威風減。

身横野草錦斑消,緊閉雙睛光不閃。
\end{myquote}
	
當下這隻猛虎,被武松沒頓飯之間,一頓拳脚,打的動不得了。使的這漢子口裏兀自氣喘不息。武松放了手,來松樹邊尋那打折的梢棒。只怕大蟲不死,向身上又打了十數下,那大蟲氣都没了。武松尋思:「我就勢把這大蟲拖下崗子去。」就血泊中雙手來提時,那裏提得動?原來使盡了氣力,手脚都酥軟了。武松正坐在石上歇息,只聽草坡裏刷剌剌響。武松口中不言,心下驚恐:「天色已黑了,倘或又跳出一個大蟲來,我卻怎生鬦得過他?」剛言未畢,只見坡下鑽出兩隻大蟲來,唬得武松大驚道:「阿呀,今番我死也!」只見那兩個大蟲於面前直立起來。武松定睛看時,卻是個人:把虎皮縫做衣裳,頭上帶着虎磕腦。那兩人手裏各拿着一條五股鋼叉,見了武松倒頭便拜,說道:「壯士,你是人也,神也?端的喫了㺀𤝽心、豹子肝、獅子腿,膽倒包了身軀!不然,如何獨自一個,天色漸晚,又沒器械,打死這個傷人大蟲?我們在此觀看多時了。端的壯士,高姓大名?」武松道:「我行不更名,坐不改姓,自我便是陽谷縣人氏,姓武名松,排行第二。」因問:「你兩個是甚麽人?」那兩個道:「不瞞壯士說,我們是本處打獵戶。因為崗前這隻虎,夜夜出來,傷人極多。只我們獵戶也折了七八個,過路客人不計其數。本縣知縣相公,着落我們衆獵戶限日捕捉,得獲時,賞銀三十兩;不獲時,定限喫拷。叵耐這業畜勢大,難近得他,誰敢向前?我們只和數十鄉夫在此,遠遠地安下窝弓藥箭等他。正在這裏埋伏,卻見你大剌剌從崗子上走來,三拳兩脚,和大蟲敵鬦,把大蟲登時打死了。未知壯士身上有多少力!俺衆人把大蟲綣了,請壯士下崗,往本縣去見知縣相公,討賞去來。」於是衆鄉夫獵戶,約凑有七八十人,先把死大蟲擡在前面,將一個兜轎擡了武松,逕投本䖏一個上戶家。那上戶里正都在莊前迎接,把這大蟲扛在草庭上。卻有本縣里老,都來相探,問了武松姓名。因把打虎一節,說了一遍。衆人道:「眞乃英雄好漢!」那衆獵戶先把野味將來,與武松把盞,喫得大醉。打掃客房,武松歇息。

到天明,里老先去縣裏報知。一面合具虎牀,安排花紅軟轎,迎送武松到縣衙前。清河縣知縣使人來接到縣内廳上。那滿縣人民聽得說一個壯士打死了景陽崗上大蟲,迎賀將來,盡皆出來觀看,哄動了那個縣治。武松到廳上下了轎,扛着大蟲在廳前。知縣看了武松這般模様,心中自忖道:「不恁地,怎打得這個猛虎?」便唤武松上廳來。參見畢。將打虎首尾,訴說了一遍。兩邊官吏,都驚獃了。知縣就廳上賜了幾盃酒,將庫中衆上戶出納的賞錢三十兩,就賜與武松。武松禀道:「小人託賴相公的福蔭,偶然僥倖打死了這個大蟲。非小人之能,如何敢受這三十兩賞賜?衆獵戶因這畜生,受了相公許多責罰。何不就把這賞給散與衆人去?也顯相公恩霑,小人義氣。」知縣道:「旣是如此,任從壯士處分。」武松就把這三十兩賞錢,在廳上俵散與衆獵戶去了。知縣見他仁德忠厚,又是一條好漢,有心要擡擧他,便道:「你雖是陽谷縣的人氏,與我這清河縣只在咫尺。我今日就參你在我這縣裏做個巡捕的都頭,專一河東水西擒拿盗賊,你意下如何?」武松跪謝道:「若蒙恩相擡擧,小人終身受賜。」知縣隨即喚押司立了文案,當日便參武松做了巡捕都頭。衆里正大戶,都來與武松作賀慶喜,連連誇官,喫了三五日酒。武松正要陽谷縣找尋哥哥,不料又在清河縣做了都頭。一日在街上閒遊,喜不自勝。傳得東平一府兩縣,皆知武松之名。有詩為證:

\begin{myquote}
壯士英雄藝略芳,挺身直上景陽崗。

醉來打死山中虎,自此聲名播四方。
\end{myquote}
	
按下武松,單表武大。自從與兄弟分居之後,因時遭荒饉,搬移在清河縣紫石街賃房居住。人見他為人懦弱,模様猥衰,起了他個渾名,叫做「三寸丁、谷樹皮」。俗語言其身上粗糙,頭臉窄狹故也。以此人見他這般軟弱樸實,都欺負他。武大並無生氣,常時迴避便了。看官聽說:世上惟有人心最歹,軟的又欺,惡的又怕;太剛則折,太柔則廢。古人有幾句格言說的好:

\begin{myquote}
柔軟立身之本,剛強惹祸之胎。無爭無競是賢才,虧我些兒何礙?青史幾場春夢,紅塵多少奇才。不須計較巧安排,守分而今見在。
\end{myquote}

且說武大終日挑擔子出去街上,賣炊餅度日。不幸把渾家故了,丢下個女孩兒,年方十二歲,名喚迎兒。爺兒兩個過活,那消半年光景,又消折了資本,移在大街坊張大戶家臨街房居住,依舊做買賣。張宅家下人見他本分,常看顧他,照顧他炊餅。閒時在他舖中坐,武大無不奉承。因此張宅家下人個個都歡喜,在大戶面前,一力與他說方便,因此大戶連房錢也不問武大要。

這張大戶家有萬貫家財,百間房産,年約六旬之上,身邊寸男尺女皆無。媽媽余氏,主家嚴厲,房中並無清秀使女。一日,大戶拍胸嘆了一口氣。媽媽問道:「你田産豐盛,資財充足,閒中何故嘆氣?」大戶道:「我許大年紀,又無兒女,雖有家財,終無大用。」媽媽道:「旣然如此說,我教媒人替你買兩個使女,早晚習學彈唱,服侍你便了。」大戶心中大喜,謝了媽媽。過了幾時,媽媽果然敎媒人來,與大戶買了兩個使女,一個叫做潘金蓮,一個喚做白玉蓮。這潘金蓮卻是南門外潘裁的女兒,排行六姐。因他自幼生得有些顔色,纏得一雙好小脚兒,因此小名金蓮。父親死了,做娘的因度日不過,從九歲賣在王招宣府裏,習學彈唱,就會描眉畫眼,傅粉施朱,梳一個纏髻兒,着一件扣身衫兒,做張做勢,喬模喬樣。況他本性機變伶俐,不過十五,就會描鸞刺綉,品竹彈絲,又會一手琵琶。後王招宣死了,潘媽媽爭將出來,三十兩銀子轉賣與張大戶家,與玉蓮同時進門。在大戶家習學彈唱,金蓮學琵琶,玉蓮學箏。玉蓮亦年方二八,乃是樂戶人家女子,生得白凈,小字玉蓮。這兩個同房歇臥。主家婆余氏初時甚是擡擧二人,不令上鍋,聊備洒掃,與他金銀首飾,粧束身子。後日不料白玉蓮死了,止落下金蓮一人,長成一十八歲,出落的臉襯桃花,不紅不白;眉彎新月,又細又彎。張大戶每要收他,只怕主家婆利害,不得手。一日,主家婆鄰家赴席不在,大戶暗把金蓮喚至房中,遂收用了。正是:羙玉無瑕,一朝損壞。珍珠何日,再得完全?

大戶自從收用金蓮之後,不覺身上添了四五件病症。端的那五件?第一,腰便添疼;第二,眼便添淚;第三,耳便添聾;第四,鼻便添涕;第五,尿便添滴。還有一樁兒不可說,白日間只是打盹,到晚來噴㖒也無數。後主家婆頗知其事,與大戶嚷駡了數日,將金蓮甚是苦打。大戶知不容此女,卻賭氣倒賠房奩,要尋嫁得一個相應的人家。大戶家下人,都說武大忠厚,現無妻小,又住着宅内房兒,堪可與他。這大戶早晚還要看覷此女,因此不要武大一文錢,白白的嫁與他為妻。這武大自従娶的金蓮來家,大戶甚是看顧他。若武大沒本錢做炊餅,大戶私與他銀兩:與他做本錢。武大若挑擔兒出去,大戶候無人,便踅入房中與金蓮廝會。武大雖一時撞見,亦不敢聲言。朝來暮往,如此也有幾時。忽一日,大戶得患陰寒病症,嗚呼哀哉死了。主家婆察知其事,怒令家童將金蓮武大即時趕出,不容在房子裏住。武大不免又尋紫石街西王皇親房子,賃内外兩間居住,依舊賣炊餅。原來金蓮自從嫁武大,見他一味老實,人物猥衰,甚是憎嫌,常與他合氣。抱怨大戶:「普天世界斷生了男子,何故將奴嫁與這樣個貨!每日牽着不走,打着倒腿的,只是一味𠳹酒。着緊處卻是錐扎也不動。奴端的那世裏晦氣,卻嫁了他!是好苦也!」常無人處唱個〔山坡羊〕爲證:

\begin{myquote}
「想當初,姻緣錯配,奴把他當男兒漢看覷。不是奴自己誇獎,他烏鴉怎配鸞鳳對?奴眞金子埋在土裏,他是塊高麗銅,怎與俺金色比!他本是塊頑石,有甚福抱着我羊脂玉體?好似糞土上長出靈芝。奈何?隨他怎樣,倒底奴心不羙!聽知:奴是塊金磚,怎比泥土基!」
\end{myquote}

看官聽說:但凡世上婦女,若自己有些顔色,所禀伶俐,配個好男子便罷了,若是武大這般,雖好煞,也未免有幾分憎嫌。自古佳人才子相凑着的少,買金偏撞不着賣金的。武大每日自挑炊餅擔兒出去賣,到晚方歸。婦人在家,别無事幹,一日三餐喫了飯,打扮光鮮,只在門前簾兒下站着,常把眉目嘲人,雙睛傳意。左右街坊,有幾個奸詐浮浪子弟,睃見了武大這個老婆,打扮油樣,霑風惹草,被這干人在街上撒謎語,往來嘲戯唱叫:「這一塊好羊肉,如何落在狗口裏!」人人只知武大是個懦弱之人,卻不知他娶得這個婆娘在屋裏,風流伶俐,諸般都好,為頭的一件好偷漢子。有詩為證:

\begin{myquote}
金蓮容貌更堪題,笑蹙春山八字眉。

若遇風流清子弟,等閒雲雨便偸期。
\end{myquote}

這婦人每日打發武大出門,只在簾子下嗑瓜子兒,一徑把那一對小金蓮故露出來,勾引的這夥人日逐在門前彈胡博詞、扠兒機,口裏油似滑言語,無般不說出來。因此武大在紫石街住不牢,又要往别䖏搬移,與老婆商議。婦人道:「賊混沌,不曉事的!你賃人家房住,淺房淺屋,可知有小人囉唣!不如凑幾兩銀子,看相應的典上他兩間住,卻也氣概些,免受人欺負。你是個男子漢,倒擺布不開,常敎老娘受氣!」武大道:「我那裏有錢典房?」婦人道:「呸!濁材料!把奴的釵梳凑辦了去,有何難處?過後有了,再治不遲。」武大聽了老婆這般說,當下凑了十數兩銀子,典得縣門前樓,上下兩層,四間房屋居住。第二層是樓,兩個小小院落,甚是乾凈。

武大自從搬到縣西街上來,照舊賣炊餅。一日,街上所過,見數隊纓鎗,鑼鼓喧天,花紅軟轎,簇擁着一個人,卻是他嫡親兄弟武松。因在景陽崗打死了大蟲,知縣相公擡擧他,新陞做了巡捕都頭。街上里老人等作賀他,送他下處去。卻被武大撞見,一手扯住,叫道:「兄弟,你今日做了都頭,怎不看顧我?」武松囬頭,見是哥哥。二人相會,兄弟大喜,一面邀請到家中,讓至樓上坐。房裏喚出金蓮來,與武松相見。因說道:「前日景陽岡打死了大蟲的,便是你小叔。今新充了都頭,是我一母同胞兄弟。」那婦人叉手向前,便道:「叔叔萬福!」武松施禮,倒身下拜。婦人扶住武松道:「叔叔請起,折殺奴家。」武松道:「嫂嫂受禮。」兩個相讓了一囬,都平磕了頭,起來。少頃,小女迎兒拿茶,二人喫了。武松見婦人十分妖嬈,只把頭來低着。不多時,武大安排酒飯,管待武松。說話中間,武大下樓買酒菜去了,丢下婦人,獨自在樓上陪武松坐的。看了武松身材凜凜,相貌堂堂,身上恰似有千百斤氣力。——不然,如何打得那大蟲?心裏尋思道:「一母所生的兄弟,又這般長大,人物壯健,奴若嫁得這個,胡亂也罷了。你看我家那身不滿尺的『丁樹』,三分似人,七分似鬼。奴那世裏遭瘟,直到如今!據看武松又好氣力,何不教他搬來我家住?誰想這段姻緣,卻在這裏!」那婦人一面臉上堆下笑來,問道:「叔叔,你如今在那裏居住?每日飯食,誰人整理?」武松道:「武二新充了都頭,逐日答應上司,别䖏住不方便,胡亂在縣前尋了個下處,每日撥兩個土兵服事做飯。」婦人道:「叔叔何不搬來家裏住,省的在縣前土兵服事,做飯腌臢。一家裏住,早晚要些湯水喫時,也方便些。就是奴家親自安排與叔叔喫,也乾凈。」武松道:「深謝嫂嫂。」婦人又道:「莫不别䖏有嬸嬸,可請來廝會也。」武松道:「武二並不曾婚娶。」婦人道:「叔叔青春多少?」武松道:「虚度二十八歲。」婦人道:「原來叔叔倒長奴三歲。叔叔今番従那裏來?」武松道:「在滄州住了一年有餘,只想哥哥在舊房居住,不想搬在這裏!」婦人道:「一言難盡。自從嫁得你哥哥,喫他忒善了,被人欺負,纔得到這裏。若似叔叔這般雄壯,誰敢道個不字。」武松道:「家兄從來本分,不似武松撒潑。」婦人笑道:「怎的顛倒說?常言人無剛強,安身不牢。奴家平生快性,看不上這樣三打不囘頭,四打連身轉的人。」有詩為證,詩曰:

\begin{myquote}
叔嫂萍蹤得偶逢,嬌嬈偏逞秀儀容。

私心便欲成歡會,暗把邪言釣武松。
\end{myquote}

原來這婦人甚是言語撇清。武松道:「家兄不惹祸,免嫂嫂憂心。」二人只在樓上説話未了,只見武大買了些肉菜果餅歸來,放在廚下。走上樓來呌道:「大嫂,你且下來安排則個。」那婦人應道:「你看那不曉事的!叔叔在此,無人陪侍,卻敎我撇了下去?」武松道:「嫂嫂請方便。」婦人道:「何不去間壁,請王乾娘來安排便了。只是這般不見便?」武大便自去央了間壁王婆子來,安排端正,都拿上樓來,擺在桌子上,無非是些魚肉、果菜、點心之類,隨即盪上酒來。武大教婦人坐了主位,武松對席,武大打横,三人坐下,把酒來斟。武大篩酒在各人面前。那婦人拿起酒來道:「叔叔休怪,没甚管待,請盃兒水酒。」武松道:「感謝嫂嫂,休這般說。」武大只顧上下篩酒,那裏來閒事。那婦人笑容可掬,滿口兒呌:「叔叔,怎的肉果兒也不揀一筯兒?」揀好的遞將過來。武松是個直性漢子,只把做親嫂嫂相待。誰知這婦人是個使女出身,慣會小意兒;亦不想這婦人一片引人心。那武大又是善弱的人,那裏會管待人。婦人陪武松喫了幾盃酒,一雙眼只看着武松身上。武松乞他看不過,只低了頭不理他。喫了一歇,酒闌了,便起身。武大道:「二哥,没事再喫幾盃兒去。」武松道:「生受!我再來望哥哥嫂嫂罷。」都送下樓來。出的門外,婦人便道:「叔叔是必上心,搬來家裏住!若是不搬來,俺兩口兒也喫别人笑話。親兄弟,難比别人,與我們爭口氣,也是好處!」武松道:「既是吾嫂厚意,今晚有些行李,便取來。」婦人道:「叔叔是必記心者,奴這裏專候!」正是:滿前野意無人識,幾點碧桃春自開。有詩為證:

\begin{myquote}
可怪金蓮用意深,包藏淫行蕩春心。

武松正大原難犯,耿耿清名抵萬金。
\end{myquote}

當日這婦人情意十分慇勤。卻說武松到縣前客店内,收拾行李鋪蓋,敎土兵挑了,引到哥家。那婦人見了,強如拾了金寶一般歡喜。旋打掃一間房,與武松安頓停當。武松吩咐土兵囬去,當晚就在哥家宿歇。次日早起,婦人也慌忙起來,與他燒湯淨面。武松梳洗裹幘,出門去縣裏畫卯,婦人道:「叔叔畫了卯,早些來家喫飯,休去别處喫了。」武松應諾。到縣裏畫卯已畢,伺候了一早晨,囬到家中。那婦人又早齊齊整整,安排下飯,三口兒同喫了飯。婦人雙手便捧一盃茶來,遞與武松。武松道:「教嫂嫂生受,武松寢食不安!明日縣裏撥個土兵來使喚。」那婦人連聲叫道:「叔叔,卻怎生這般計較?自家骨肉,又不服事了別人!雖然有這小丫頭迎兒,奴家見他拿東拿西蹀里蹀斜,也不靠他。就是撥了土兵來,那廝上鍋上竈不乾凈,奴眼裏也看不上這等人。」武松道:「恁的,卻生受嫂嫂了!」有詩為證:

\begin{myquote}
武松儀表甚搊搜,嫂嫂淫心不可收。

籠絡歸來家裏住,要同雲雨會風流。
\end{myquote}

話休絮煩。自従武松搬來哥家裏住,取些銀子出來與武大,教買餅饊茶果,請那兩邊鄰舍。衆憐舍都鬦分子,來與武松人情。武大又安排了囬席,都不在話下。過了數日,武松取出一疋彩色緞子,與嫂嫂做衣服。那婦人堆下笑來,便道:「叔叔,如何使得!旣然賜與,奴家不敢推辭,只得接了。」道個萬福。自此武松只在哥家歇宿。武大依前上街挑賣炊餅。武松每日自去縣裏承差應事,不論歸遲歸早,婦人炖羹炖飯,歡天喜地服事武松。武松倒安身不得:那婦人時常把些言語來撥他,武松是個硬心的直漢。

有話即長,無話即短,不覺過了一月有餘,看看十一月天氣,連日朔風緊起。只見四下彤雲密布,又早紛紛揚揚,飛下一天瑞雪來。但見: 

\begin{myquote}
萬里彤雪密佈,空中祥瑞飄簾,瓊花片片舞前簷。剡溪當此際,濡滯子猷船。頃刻樓臺都壓倒,江山銀色相連。飛鹽撒粉漫連天,當時呂蒙正,窯内嗟無錢。
\end{myquote}

當日這雪直下到一更時分,卻似銀粧世界,玉碾乾坤。次日,武松早去縣裏畫卯,直到日中未歸。武大被婦人早趕出去做買賣,央及間壁王婆,買了些酒肉;去武松房裏,簇了一盆炭火。心裏自想道:「我今日着實撩鬦他一鬦,不怕他不動情!」那婦人獨自冷冷清清立在簾兒下,望見武松正在雪裏踏着那亂瓊碎玉歸來。婦人推起簾子,迎着笑道:「叔叔寒冷。」武松道:「感謝嫂嫂掛心!」入得門來,便把毡笠兒除將下來,那婦人將手去接。武松道:「不勞嫂嫂,生受!」自把雪來拂了,掛在壁子上。隨即解了纏帶,脱了身上鸚哥綠紵絲衲襖,入房内搭了。那婦人便道:「奴等了一早晨,叔叔怎的不歸來喫早飯?」武松道:「早間有一相識請我喫飯了,卻纔又有一個作盃,我不耐煩,一直走到家來。」婦人道:「旣恁的,請叔叔向火。」武松道:「正好。」便脫了油靴,換了一雙襪子,穿了暖鞋,掇條凳子,自近火盆邊坐的。那婦人早令迎兒把前門上了閂,後門也關了。卻搬些煮酒菜蔬入房裏來,擺在桌子上。武松問道:「哥哥那裏去了?」婦人道:「你哥哥每日自出去做些買賣,我和叔叔自喫三盃。」武松道:「一發等哥來家,喫也不遲。」婦人道:「那裏等的他!」說猶未了,只見迎兒小女早煖了一注酒來。武松道:「不必嫂嫂費心,待武二自斟。」婦人也掇一條凳子,近火邊坐了。桌上擺着盃盤,婦人拿盞酒擎在手裏,看着武松:「叔叔滿飲此盃!」武松接過酒去,一飲而盡。那婦人又篩一盃來,說道:「天氣寒冷,叔叔飲個成雙的盞兒。」武松道:「嫂嫂自飲。」接來又一飲而盡。武松卻篩一盃酒,遞與婦人。婦人接過酒來呷了,卻拿注子再斟酒,放在武松面前。那婦人一徑將酥胸微露,雲鬟半軃,臉上堆下笑來,說道:「我聽得人説,叔叔在縣前街上養着個唱的,有這話麽?」武松道:「嫂嫂休聽別人胡說,我武二従來不是這等人!」婦人道:「我不信,只怕叔叔口頭不似心頭。」武松道:「嫂嫂不信時,只問哥哥就是了。」婦人道:「呵呀,你休說,他那裏曉得甚麽?如在醉生夢死一般。他若知道時,不賣炊餅了。叔叔且請一盃!」連篩了三四盃飲過。那婦人也有三盃酒落肚,烘動春心,那裏按納得住?慾心如火,只把閒話來說。武松也知了八九分,自己只把頭來低了,卻不來兜攬。婦人起身去盪酒,武松自在房内,卻拿火筯簇火。婦人良久煖了一注子酒,來到房裏,一隻手拿着注子,一隻手便去武松肩上只一捏,說道:「叔叔,只穿這些衣服,不寒冷麽?」武松已有五七分不自在,也不理他。婦人見他不應,劈手便來奪火筯,口裏道:「叔叔你不會簇火,我與你撥火!只要一似火盆來熱便好。」武松有八九分焦躁,只不做聲。這婦人也不看武松焦躁,便丢下火筯,卻篩一盞酒來,自呷了一口,剩下大半盞酒,看着武松道:「你若有心,喫我這半盃兒殘酒。」乞武松劈手奪過來,潑在地下。說道:「嫂嫂,不要恁的不識羞耻!」把手只一推,爭些兒把婦人推了一跤。武松睜起眼來說道:「武二是個頂天立地的噙齒戴髮的男子漢,不是那等敗壞風俗傷人倫的猪狗!嫂嫂休要這般不識羞耻,為此等的勾當,倘有些風吹草動,我武二眼裏認的是嫂嫂,拳頭卻不認的是嫂嫂!再來休要如此所爲。」婦人喫他幾句,搶得通紅了面皮,便叫迎兒收拾了碟盞家伙。口裏指着說道:「我自作耍子,不値得便當眞起來,好不識人敬!」收了家伙,自往廚下去了。有詩爲證:

\begin{myquote}
潑賤操心太不良,貪淫無耻壞綱常。

席間尚且求雲雨,反被都頭駡一場。
\end{myquote}

這婦人見勾搭武松不動,反被他搶白了一場好的。武松自在房中氣忿忿的,自己尋思。天色卻早申牌時分,武大挑着擔兒大雪裏歸來。推開門,放下擔兒,進的房來,見婦人一雙眼哭的紅紅的,便問道:「你和誰鬧來?」婦人道:「都是你這不不爭氣的,教外人來欺負我!」武大道:「誰敢來欺負你?」婦人道:「情知是誰!爭奈武二那廝。我見他大雪裏歸來,好意安排些酒飯與他喫,他見前後没人,便把言語來調戲我。便是迎兒眼見,我不賴他!」武大道:「我兄弟不是這等人,從來老實。休要高聲,乞鄰舍聽見笑話!」武大撇了婦人,便來武松房裏。叫道:「二哥,你不曾喫點心,我和你喫些個。」武松只不做聲。尋思了半晌,脱了絲鞋,依舊穿上油臘靴,着了上蓋,戴上毡笠兒。一面繋纏帶,一面出大門。武大叫道:「二哥,你那裏去?」也不答,一直只顧去了。武大囘到房内,問婦人道:「我呌他,又不應,只顧往縣前那條路去了。正不知怎的了!」婦人罵道:「賊混沌蟲!有甚難見䖏?那廝羞了,没臉兒見你,走了出去。我猜他一定呌個人來搬行李,不要在這裏住。卻不道你留他!」武大道:「他搬了去,須乞别人笑話。」婦人罵道:「混沌魍魎,他來調戲我,到不乞别人笑話?你要,便自和他過去,我卻做不的這樣人。你與了我一紙休書,你自留他便了!」武大那裏再敢開口,被這婦人倒數罵了一頓。

正在家兩口兒絮聒,只見武松引了個土兵,拿着條扁擔,徑來房内,收拾行李便出門。武大走出來,呌道:「二哥,做甚麽便搬了去?」武松道:「哥哥不要問,說起來裝你的幌子。只由我自去便了!」武大那裏再敢問備細,由武松搬了出去。那婦人在裏面喃喃呐呐駡道:「卻也好!只道是親難轉債,人只知道一個兄弟做了都頭怎的養活了哥嫂,卻不知反來嚼咬人!正是花木瓜,空好看。搬了去,倒謝天地,且得冤家離眼前。」武大見老婆這般言語,不知怎的了,心中只是放它不下。

自從武松搬去縣前客店宿歇,武大自依前上街賣炊餅。本待要去縣前尋兄弟說話,卻被這婦人千叮萬囑,吩咐教不要去兜攬他,因此武大不敢去尋武松。有詩為證:

\begin{myquote}
雨意雲情不遂謀,心中誰信起戈矛。

生將武二搬離去,骨肉翻令作寇讐!
\end{myquote}

畢竟未知後來何如,且聽下囘分解。

