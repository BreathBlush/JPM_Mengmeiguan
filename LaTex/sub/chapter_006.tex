\includepdf[pages={11,12},fitpaper=false]{tst.pdf}
\chapter*{第六囬 \\西門慶買囑何九 王婆打酒遇大雨}
\addcontentsline{toc}{chapter}{第六囬 西門慶買囑何九 王婆打酒遇大雨}
\markboth{\titlename}{第六囬 西門慶買囑何九 王婆打酒遇大雨}


\begin{myquote}
可怪狂夫戀野花,因貪淫色受波喳。

亡身喪命皆因此,破業傾家總為他。

半晌風流有何益,一般滋味不須誇。

一朝祸起蕭牆内,虧殺王婆先做牙。
\end{myquote}

却說西門慶便對何九說去了。且說王婆拿銀子來買棺材冥器,又買些香燭紙錢之類,歸來與婦人商議,就於武大靈前點起一盞隨身燈。鄰舍街坊都來看望,那婦人虚掩着粉臉假哭。衆街坊問道:「大郎得何病患便死了?」那婆娘答道:「拙夫因害心疼得慌,不想一日日越重了,看看不能夠好。不幸昨夜三更鼓死了,好是苦也!」又哽哽咽咽假哭起來。衆鄰舍明知道此人死的不明,不敢只顧問他。衆人盡勸道:「死是死了,活的自要安穩過。娘子省煩惱,天氣暄熱。」那婦人只得假意兒謝了,衆人各自散去。王婆擡了棺材來,又去請仵作團頭何九。但是入殮用的都買了,并家裏一應物件也都買了。就於報恩寺叫了兩個禪和子,晚夕伴靈拜懺。不多時,何九先撥了幾個火家整頓。

且說何九,到巳牌時分,慢慢的走來,到紫石街巷口,迎見西門慶,叫道:「老九何往?」何九答道:「小人只去前面,殮這賣炊餅的武大郎屍首。」西門慶道:「且借一步說話。」何九跟着西門慶,來到轉角頭一個小酒店裏,坐下在閣兒内。西門慶道:「老九請上坐。」何九道:「小人是何等之人,敢對大官人一䖏坐的!」西門慶道:「老九何故見外?且請坐!」二人讓了一囬,坐下。西門慶吩咐酒保:「取瓶好酒來。」酒保一面鋪下菜蔬菓品案酒之類,一面盪上酒來。何九心中疑忌,想道:「西門慶自來不曾和我喫酒,今日這盃酒必有蹺蹊。」兩個飲夠多時,只見西門慶自袖子裏摸出一錠雪花銀子,放在面前,説道:「老九休嫌輕微,明日另有酬謝!」何九叉手道:「小人無半點用功効力之䖏,如何敢受大官人見賜銀兩?若是大官人有使令,小人也不敢辭。」西門慶道:「老九休要見外,請收過了。」何九道:「大官人便說不妨。」西門慶道:「别無甚事。少刻他家自有些辛苦錢。只是如今殮武大的屍身,凡百事周全,一牀錦被遮蓋則個!餘不多言。」何九道:「我道何事!這些小事,有甚打緊,如何敢受大官人銀兩?」西門慶道:「老九!你若不受時,便是推却。」何九自來懼西門慶是個刁徒,把持官府的人,只得收了銀子。又喫了幾盃酒,西門慶呼酒保來:「記了帳目,明日來我舖子内支錢。」兩個下樓,一同出了店門。臨行,西門慶道:「老九,是必記心,不可泄漏。改日另有補報!」吩咐罷,一直去了。

何九心中疑忌:「我殮武大身屍,他何故與我這十兩銀子?此事必有蹺蹊。」一面來到武大門首,只見那幾個火家正在門首伺候;王婆也等的火裏火發在那裏。何九便問火家:「這武大是甚病死了?」火家道:「他家説害心疼病死了。」何九入門,揭起簾子進來。王婆接着,道:「久等多時了,陰陽也來了半日,老九如何這咱纔來?」何九道:「便是。有些小事絆住了脚,來遲了一步。」只見那婦人穿着一件素淡衣裳,白布䯼髻,從裏面假哭出來。何九道:「娘子省煩惱,大郎已是歸天去了。」那婦人虚掩着淚眼道:「說不得的苦!我夫心疼症候,幾個日子便把命丢了。撇得奴好苦!」這何九一面上上下下看了婆娘的模樣,心裏自忖的道:「我従來只聽得人說武大娘子,不曾認得他。原來武大郎討得這個老婆在屋裏。西門慶這十兩銀子使着了!」一面走向靈前,看武大屍首。陰陽宣念經畢,揭起千秋旛,扯開白絹,用五輪八寳玩着那兩點神水定睛看時,見武大指甲青,唇口紫,面皮黄,眼皆突出,就知是中毒。傍邊那兩個火家說道:「怎的臉也紫了,口唇上有牙痕,口中出血?」何九道:「休得胡說!兩日天氣十分炎熱,如何不走動些?」一面七手八脚葫蘆提殮了,裝入棺材内,兩下用長命釘釘了。王婆一力攛掇,拿出一吊錢來與何九。打發衆火家去了,就問:「幾時出去?」王婆道:「大娘子説,只三日便出殯,城外燒化。」衆火家各分散了。

那婦人當夜擺着酒請人。第二日,請四個僧念經。第三日早五更,衆火家都來扛擡棺材,也有幾個鄰舍街坊,吊孝相送。那婦人帶上孝,坐了一乘轎子,一路上口内假哭養家人。來到城外化人場上,便教擧火燒化棺材,并武大屍首燒得乾乾淨淨,把骨殖撒在池子裏。原來那日齋堂管待,一應都是西門慶出錢整頓。那婦人歸到家中,樓上去設個靈牌,上寫:「亡夫武大郎之靈」。靈牀子前點一盞琉璃燈,裏面貼些經旛、錢紙、金銀錠之類。那日却和西門慶做一處,打發王婆家去,二人在樓上任意縱横取樂,不比先前在王婆茶坊裏,只是偸鷄盗狗之歡。如今武大已死,家中無人,兩個恣情肆意,停眠整宿。初時西門慶恐鄰舍瞧破,先到王婆那邊坐一囬;今武大死後,帶着跟隨小廝,徑從婦人家後門而入。自此和婦人情沾肺腑,意密如膠,常時三五夜不曾歸去,把家中大小丢的七顛八倒,都不喜歡。原來這女色坑陷得人,有成時必有敗!有詩為證:

\begin{myquote}
色膽如天不自由,情深意密兩綢繆。

貪歡不管生和死,溺愛誰將身體修?

只為恩深情欝欝,多因愛闊恨悠悠。

要將吴越寃仇解,地老天荒難歇休。
\end{myquote}

光陰迅速,日月如梭。西門慶刮剌那婦人,將兩月有餘。一日將近端陽佳節,但見:
\begin{myquote}
綠楊裊裊垂絲碧,海榴點點胭脂赤。兩兩亂鶯啼,毶毶梧竹齊。微微風動幔,颯颯凉侵扇。䖏處過端陽,家家共擧觴。
\end{myquote}

西門慶自岳廟上囬來,到王婆茶坊裏坐下。那婆子連忙點一盞茶來,便問:「大官人往那裏去來?怎的不過去看看大娘子?」西門慶道:「今日往廟上走走。大節間,記掛着,來看看大姐。」婆子道:「今日他娘潘媽媽在這裏,怕還未去哩。等我過去看看,囬大官人。」這婆子一面走過婦人後門看時,婦人正陪潘媽媽在房裏喫酒,見婆子來,連忙讓坐。婦人撮下笑來道:「乾娘來得正好!請陪俺娘,且喫個進門盞兒,到明日養個好娃娃!」婆子笑道:「老身又没有老伴兒,那裏得養出來?你年小少壯,正好養哩!」婦人道:「常言小花不結,老花兒結。」婆子便看着潘媽媽嘈道:「你看,你女兒這等傷我,說我是老花子。到明日,還用着我老花子哩!」說罷,潘媽道:「他従小兒是這等快嘴,乾娘休要和他一般見識。」原來這婆子撮合得西門慶和這婦人刮剌上了,早晚替他通事殷勤兒,提壺打酒,靠些油水養口。一面對他娘潘媽說:「你家這姐姐,端的百伶百俐,不枉了好個婦女。到明日,不知什麽有福的人受的他!」潘媽媽道:「乾娘既是撮合山,全靠乾娘作成則個。」一面安下鍾筯,婦人斟酒在他面前。婆子一連陪了幾盃酒,喫得臉紅紅的,又怕西門慶在那邊等候,連忙丢了個眼色與婦人,告辭歸去。婦人就知西門慶來了,於是一力攛掇他娘起身去了。將房中收拾乾凈,燒些異香,従新把娘的殘饌撇去,另安排一席齊整酒肴,預備陪侍。

西門慶従月臺上過來,婦人從梯凳接着。到房中,道個萬福,坐下。原來婦人自従武大死後,怎肯带孝?樓上把武大靈牌丢在一邊,用一張白紙蒙着,羹飯也不瞅睬。每日只是濃粧豔抹,穿顔色衣服,打扮嬌樣,陪伴西門慶做一䖏,作歡頑耍。因見西門慶兩日不來,就罵:「負心的賊!如何撇閃了奴,又往那家另續上心甜的了?把奴冷丢,不來瞅睬!」西門慶道:「便是家中小妾昨日没了,殯送忙了兩日。今日往廟上去,替你置了些首飾珠翠衣服之類。」那婦人滿心歡喜。西門慶一面喚過小廝玳安來,氈包内取出,一件件把與婦人。婦人方纔拜謝收了。小女迎兒,尋常被婦人打怕的,以此不瞞他,令他拿茶與西門慶喫。一面婦人安放桌兒,陪西門慶喫茶。西門慶道:「你不消費心,我已與了乾娘銀子,買酒肉嗄飯菓子去了。大節間,正要和你坐一坐。」婦人道:「此是待俺娘的,奴存下這桌整菜兒。等到乾娘買來,且有一囬躭擱。咱且喫着。」婦人陪西門慶,臉兒相貼,腿兒相壓,並肩一處飲酒。

且說婆子提着個籃子,拿着一條十八兩秤,走到街上打酒買肉。那時正值五月初旬天氣,大雨時行。只見紅日當天,忽一塊濕雲過處,大雨傾盆相似。但見:
\begin{myquote}
烏雲生四野,黑霧鎖長空。刷剌剌漫空障日飛來,一點點擊得芭蕉聲碎。狂風相助,侵天老檜掀翻;霹靂交加,泰華嵩嶠震動。洗炎驅暑,潤澤田苗。洗炎驅暑,佳人貪其賞玩;潤澤田苗,行人忘其泥濘。正是:江淮河濟添新水,翠竹紅榴洗濯清。
\end{myquote}

那婆子正打了一瓶酒,買了一籃魚肉鷄鵝菜蔬菓品之類,在街上遇見這大雨,慌忙躲在人家房簷下,用手帕裹着頭,把衣服都淋濕了。等了一歇,那雨脚慢了些,大步雲飛來家。進入門來,把酒肉放在廚房下。走進房來,看婦人和西門慶飲酒,笑嘻嘻道:「大官人和大娘子好飲酒!你看把婆子身上衣服都淋濕了,到明日就叫大官人賠我!」西門慶道:「你看老婆子,就是個賴精。」婆子道:「我不是賴精,大官人少不得賠我一疋大海青!」婦人道:「乾娘,你且飲個盪熱酒盞兒。」那婆子陪着飲了三盃,說道:「老身往廚下烘乾衣裳去。」一面走到廚下,把衣服烘乾。那鷄鵝嗄飯,割切安排停當,用盤碟盛了菓品之類,都擺在房中,盪上酒來。西門慶與婦人重斟羙酒,共設佳肴,交盃疊股而飲。西門慶飲酒中間,看見婦人壁上掛着一面琵琶,便道:「久聞你善彈,今日好夕彈個曲兒我下酒。」婦人笑道:「奴自幼粗學一兩句,不十分好,官人休要笑耻。」西門慶一面取下琵琶來,摟婦人在懷,看他放在膝兒上,輕舒玉笋,款弄冰絃,慢慢彈着,唱了一個〈兩頭南調兒〉:
\begin{myquote}
「冠兒不戴懶梳粧,髻挽青絲雲鬢光。金釵斜插在烏雲上。喚梅香,開籠箱,穿一套素縞衣裳,打扮的西施模樣。出繡房,梅香,你與我捲起簾兒,燒一炷兒夜香。」
\end{myquote}

西門慶聽了,喜歡的沒入脚處。一手摟過婦人粉項來,就親了個嘴,稱誇道:「誰知姐姐你有這段兒聰明!就是小人在勾欄三街兩巷相交唱的,也沒你這手好彈唱!」婦人笑道:「蒙官人擡擧,奴今日與你百依百隨,是必過後休忘了奴家!」西門慶一面捧着他香腮,說道:「我怎肯忘了姐姐!」兩個殢雨尤雲,調笑玩耍。少頃,西門慶又脱下他一隻繡花鞋兒,擎在手内,放一小盃酒在内,喫鞋盃耍子。婦人道:「奴家好小脚兒,官人休要笑話。」不一時,二人喫得酒濃,掩閉了房門,解衣上牀頑耍。王婆把大門頂着,和迎兒在廚房中動彈。由着二人在房内顛鸞倒鳳,似水如魚,取樂歡娱。那婦人枕邊風月,比娼妓尤甚,百般奉承;西門慶亦施逞鎗灋打動。兩個女貌郎才,俱在妙齡之際。有詩單道其態,詩曰:
\begin{myquote}
寂靜蘭房簟枕涼,佳人才子至妙頑。

纔去倒澆紅臘燭,忽然又棹夜行船。

偸香粉蝶餐花萼,戲水蜻蜓下下旋。

樂極情濃無限趣,靈龜口内吐清泉。
\end{myquote}

當日西門慶在婦人家盤桓至晚,欲囬家,留下幾兩散碎銀子,與婦人做盤纏。婦人再三挽留不住。西門慶帶上眼罩,出門去了。婦人下了簾子,關上大門,又和王婆喫了一囬酒,各散去了。正是:倚門相送劉郎去,煙水桃花去路迷。

畢竟未知後來何如,且聽下囬分解。

