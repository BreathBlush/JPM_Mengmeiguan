\includepdf[pages={173,174},fitpaper=false]{tst.pdf}
\chapter*{第八十七囘 \\王婆子貪財受報 武都頭殺嫂祭兄}
\addcontentsline{toc}{chapter}{第八十七囘 王婆子貪財受報 武都頭殺嫂祭兄}
\markboth{{\titlename}卷之九}{第八十七囘 王婆子貪財受報 武都頭殺嫂祭兄}


\begin{myquote}
平生作善天加福,若是剛強定祸殃。

舌為柔和終不損,齒因堅硬必遭傷。

杏桃秋到多零落,松栢冬深愈翠蒼。

善惡到頭終有報,高飛遠走也難藏。
\end{myquote}

話説陳經濟僱頭口起身,呌了張團練一個伴當跟隨,早上東京去不題。

卻表吳月娘打發潘金蓮出門,次日使春鴻叫薛嫂兒來,要賣秋菊。這春鴻正走到大街,撞見應伯爵,呌住問春鴻:「你往那裏去?」春鴻道:「家中大娘使小的叫媒人薛嫂兒去。」伯爵問:「呌媒人做甚麽?」春鴻道:「賣五娘房裏秋菊丫頭。」伯爵又問:「你五娘為甚麽打發出來?在王婆子家住着,說要尋人家嫁人,端的有此話麽?」這春鴻便如此這般:「因和俺姐夫有些說話,大娘知道了,先打發了春梅小大姐,然後打了俺姐夫一頓,赶出往家去了。昨日纔打發出俺五娘來。」伯爵聽了,點了點頭兒,說道:「原來你五娘和你姐夫有楂兒!看不出人來。」又向春鴻說:「孩兒,你爹已是死了,你只顧還在他家做甚麽,終是沒出産!你心裏還要歸你南邊去?這裏尋個人家跟罷,心下如何?」春鴻道:「便是這般說。老爹已是没了,家中大娘好不嚴緊,各處買賣都收了,房子也賣了,琴童兒畫童兒都走了,也攬不過這許多人口來。小的待囬南邊去,又沒順便人帶去;這城内尋個人家跟,又沒個門路。」伯爵道:「儍孩兒,人無遠見,安身不牢。千山萬水,又往南邊去做甚?誰人帶去?你肚裏會幾句唱,愁這城内尋不出主兒來答應?我如今擧保個門路與你。如今大街坊張二老爹家,有萬萬貫家財,百間房屋,現頂補了你爹,在提刑院做掌刑千戶。如今你二娘,又在他家做了二房。我把你送到他宅中答應,他見你會唱南曲,管情一箭就上垜。留下你做個親隨大官兒,又不比在你這家裏?他性兒又好,年紀小小,又倜儻,又愛好,你就是個有造化的!」這春鴻趴到地下,就磕了個頭:「有累二爹!小的若見了張老爹,得一步之地,買禮與二爹磕頭。」伯爵一把手拉着春鴻說:「儍孩兒,你起來。我無有個不作成人的,肯要你謝?你那得錢兒來?」春鴻道:「小的去了,只怕家中大娘找尋小的,怎了?」伯爵道:「這個不打緊。我問你張二老爹討個帖兒,封一兩銀子與他家。他家銀子不敢受,不怕把你不雙手兒送了去!」說畢,春鴻往薛嫂兒家,叫了薛嫂兒見月娘,領秋菊出來,只賣了五兩銀子,交與月娘。不在話下。

卻說應伯爵領春鴻到張二官宅裏見了。張二官見他生的清秀,又會唱南曲,就留下他答應。使拿拜帖兒,封了一兩銀子,往西門慶家討他箱子。那日,吳月娘家中正陪雲離守娘子范氏喫酒。先是雲離守襲過哥雲參將指揮,補在清河左衛做同知,見西門慶死了,吳月娘守寡,手裏有東西,就安心有垂涎圖謀之意。此日正買了八盤羹菓禮物,來看月娘。見月娘生了孝哥,范氏房内亦有一女,方兩月兒,要與月娘結親。那日喫酒,遂兩家割衫襟,做了兒女親家,留下一雙金環為定禮。聽見玳安兒拿進張二官府帖兒,並一兩銀子,說春鴻投在他家答應去了,使人來討他箱子衣服。月娘見他現做提刑官,不好不與他,銀子也不曾收,只得把箱子與將出來。

初時,應伯爵對張二官說:「西門慶第五娘子潘金蓮,生的標致,會一手琵琶,百家詞曲,雙陸象棋,無不通曉,又會寫字。因為年小守不的,又和他大娘子合氣,今打發出來,在王婆家聘嫁。」這張二官一替兩替使家人拿銀子往王婆家相看,王婆只推他大娘子吩咐,不倒口要一百兩銀子。那人來回講了幾遍,還到八十兩上,王婆還不吐口兒。落後春鴻到他宅内,張二官聽見春鴻說,婦人在家養着女婿,因為如此打發出來,這張二官就不要了,對着伯爵說:「我家現放着十五歲未出幼兒子上學攻書,要這樣婦人來家做甚!」又聽見李嬌兒說,金蓮當初用毒薬擺佈死了漢子,被西門慶占將來家,又偷小廝,把第六個娘子生了兒子,娘兒兩個生生喫他害殺了,以此張二官就不要了。

話分兩頭,卻說春梅賣到守備府中,守備見他生的標致伶俐,擧止動人,心中大喜,與了他三間房住,手下使一個小丫鬟,就一連在他房中歇了三夜。三日,替他裁了兩套衣裳。薛嫂兒去,賞了薛嫂五錢銀子。又買了個使女扶侍他,立他做二房。大娘子一目失明,喫長齋念佛,不管閒事。還有生姐兒孫二娘,在東廂房住。春梅在西廂房,各處鑰匙都敎他掌管,甚是寵愛他。一日,聽薛嫂兒說,潘金蓮出來,在王婆家聘嫁,這春梅晚夕啼啼哭哭對守備說:「俺娘兒兩個,在一處廝守這幾年,他大氣兒不曾呵着我,把我當親女兒一般看承。只知拆散開了,不想今日他也出來了。你若肯娶將他來,俺娘兒們還在一處過好日子。」又說他怎的好模樣兒,「諸家詞曲都會,又會彈琵琶。聰明俊俏,百伶百俐。屬龍的,今纔三十二歲兒。他若來,奴情願做第三的也罷。」於是把守備念轉了,使手下親隨張勝李安,封了兩方手帕、二錢銀子,往王婆家相看。果然生的好個出色的婦人。王婆開口指稱:「他家大娘子要一百兩銀子。」張勝李安講了半日,還了八十兩,那王婆還不肯。走來囘守備,又添了五兩,復使二人拿着銀子和王婆子說。王婆子只是假推他大娘子不肯,不轉口兒要一百兩:「媒人錢要不要罷,天也不使空人!」這張勝李安只得又拿囘銀子來禀守備,丢了兩日。怎禁這春梅晚夕哭哭啼啼:「好歹再添幾兩銀子,娶了來和奴做伴兒,死也甘心。」守備見春梅只是哭泣,只得又差了大管家周忠同張勝李安,氈包内拿着銀子,打開與婆子看,又添到九十兩上。婆子越發張致起來,說:「若九十兩,到不的如今,提刑張二老爹家擡的去了!」這周忠就惱了,吩咐李安,把銀子包了,說道:「三隻脚蟾沒處尋,兩脚老婆愁那裏尋不出來!這老淫婦連人也不識。你說那張二官府怎的,俺府裏老爺管不着你?不是新娶的小夫人,再三在老爺跟前說念要娶這婦人,平白出這些銀子要你何用?」李安道:「勒掯俺兩番三次來囘走,賊老淫婦,越發鸚哥兒了!」拉周忠說:「管家哥,咱去來。到家囘了老爺,好不好敎牢子拿去,拶與他一頓好拶子!」這婆子終是貪着陳經濟那口食,由他罵,只是不言語。二人到府中,囘禀守備說:「已添到九十兩,還不肯。」守備說:「明日兑與他一百兩,拿轎子擡了來罷。」周忠說:「爺就添了一百兩,王婆子還要五兩媒人錢。且丢他兩日,他若張致,拿到府中,且拶與他一頓拶子,他纔怕!」看官聽說:大段潘金蓮生有地兒死有處,不爭被周忠說這兩句話,有分教,這婦人従前作過事,今朝沒興一齊來!有詩為證:

\begin{myquote}
人生雖未有前知,祸福因由更問誰?

善惡到頭終有報,只爭來早與來遲。
\end{myquote}

按下一頭,卻說一人。單表武松,自従西門慶墊發孟州牢城充軍之後,多虧小管營施恩看顧。次後施恩與蔣門神爭奪快活林酒店,被蔣門神打傷,央武松出力,反打了蔣門神一頓。不想蔣門神妹子玉蘭,嫁與張都監為妾,賺武松去,假捏賊情,將武松拷打,轉又發安平寨充軍。這武松走到飛雲浦,又殺了兩個公人,復囘身殺了張都監蔣門神全家老小,逃躱在施恩家。施恩寫了一封書,皮箱内封了一百兩銀子,敎武松到安平寨與知寨劉高,敎看顧他。不想路上聽見太子立東宫,放郊天大赦,武松就遇赦囘家,到清河縣下了文書,依舊在縣當差,還做都頭。來到家中,尋見上隣姚二郎,交付迎兒。那時迎兒已長大,十九歲了,收攬來家,一處居住,打聽西門慶已死,「你嫂子出來了,如今還在王婆家,早晚嫁人!」這漢子聽了,舊仇在心,正是:踏破鐵鞋無處覓,得來全不費工夫!

次日,裹幘穿衣,逕出門來到王婆門首。金蓮正在簾下站着,見武松來,連忙閃入裏間去。武松掀開簾子來問:「王媽媽在家?」那婆子正在磨上掃麵,連忙出來應道:「是誰呌老身?」見是武松,道了萬福。武松深深唱喏。婆子道:「武二哥且喜,幾時囘家來了?」武松道:「遇赦囘家,昨日纔到。一向多累媽媽看家,改日相謝。」婆子笑嘻嘻道:「武二哥比舊時保養,鬍子楂兒也有了,且是好身量,在外邊又學得這般知禮。」一面讓上坐,點茶喫了。武松道:「我有一樁事和媽媽說。」婆子道:「有甚事,武二哥只顧說。」武松道:「我聞的人說,西門慶已是死了,我嫂子出來,在你老人家這裏居住。敢煩媽媽對嫂子說,他若不嫁人便罷,若是嫁人,如今迎兒大了,娶得嫂子家去,看管迎兒,早晚招個女婿,一家一計過日子,庶不敎人笑話。」婆子初時還不吐口兒,便道:「他是在我這裏,倒不知嫁人不嫁人。」次後聽見武松重謝他,便道:「等我慢慢和他說。」那婦人便簾内聽見武松言語,要娶他看管迎兒;又見武松在外,出落得長大,身材胖了,比昔時又會說話兒,舊心不改,心下暗道:「這段姻緣,還落在他家手裏。」就等不得王婆呌他,自己出來,向武松道了萬福,說道:「旣是叔叔還要奴家去顧管迎兒,招女婿成家,可知好哩。」王婆道:「又一件,如今他家大娘子,要一百兩雪花銀子纔嫁人。」武松道:「如何要這許多?」王婆道:「西門大官人當初為他使了許多,就打恁個銀人兒也夠了。」武松道:「不打緊,我旣要請嫂嫂家去,就使一百兩也罷。另外破五兩銀子,謝你老人家。」這婆子聽見,喜歡的屁滚尿流,沒口說:「還是武二哥知禮,這幾年江湖上見的事多,眞是好漢!」婦人聽了此言,走到屋裏,又濃點了一盞瓜仁泡茶,雙手遞與武松喫了。婆子問道:「如今他家要發脫的緊,又有三四處官戶人家爭着娶,都囘阻了,價錢不兑。你這銀子,作速些便好。常言先下米先喫飯。千里姻緣着線牽,休要落在別人手内。」婦人道:「旣要娶奴家,叔叔上緊些。」武松便道:「明日就來兌銀,晚夕請嫂嫂過去。」那王婆還不信武松有這些銀子,胡亂答應去了。

到次日,武松打開皮箱,拿出小管營施恩與知寨劉高那一百兩銀子來,又另外包了五兩碎銀子,走到王婆家,拿天平兑起來。那婆子看見白晃晃擺了一桌銀子,口中不言,心内暗道:「雖是陳經濟許下一百兩上東京去取,不知幾時到來,仰着合着,我現鐘不打卻打鑄鐘?」又見五兩謝他,連忙收了。拜了又拜,說道:「還是武二哥曉禮,知人甘苦!」武松道:「媽媽收了銀子,今日就請嫂嫂過門。」婆子道:「武二哥且是好急性,門背後放花兒,你等不到晚了!也待我往他大娘子那裏交了銀子,纔打發他過去。」又道:「你今日帽兒光光,晚夕做個新郎!」那武松緊着心中不自在,那婆子不知好歹,又奚落他。打發武松出門,自己尋思:「他家大娘子只交我發脱,又沒和我砸定價錢,我今胡亂與他一二十兩銀子滿纂,綁着鬼也落他多一半養家。」一面把銀鑿下二十兩銀子,往月娘家裏交割明白。月娘問:「甚麽人家娶了去了?」王婆道:「兔兒沿山跑,還來歸舊窝!嫁了他小叔,還喫舊鍋裏粥去了!」月娘聽了,暗中跌脚。常言仇人見仇人,分外眼睛明,與孟玉樓說:「往後死在他小叔子手裏罷了!那漢子殺人不斬眼,豈肯干休?」

不說月娘家中嘆息,卻表王婆交了銀子到家,下午時,敎王潮先把婦人箱籠桌兒送過去。這武松在家又早收拾停當:打下酒肉,安排下菜蔬。晚上婆子領婦人進門,換了孝,戴着新䯼髻,身穿紅衣服,搭着蓋頭。進門來,見明間内明亮亮點着燈燭,武大靈牌供養在上面,先自有些疑忌,由不的髮似人揪,肉如鉤搭。進入門來,到房中,武松吩咐迎兒把前門上了拴,後門也頂了。王婆見了,說道:「武二哥,我去罷,家裏沒人。」武松道:「媽媽請進房裏喫盞酒!」武松敎迎兒拿菜蔬擺在桌上,須臾盪上酒來,請婦人和王婆喫酒。那武松也不讓,把酒斟上,一連喫了四五碗酒。婆子見他喫得惡,便道:「武二哥,老身酒夠了,放我去,你兩口兒自在喫盞兒罷。」武松道:「媽媽且休得胡說!我武二有句話問你!」只聞颼的一聲響,向衣底掣出一把二尺長刄薄背厚扎刀子來,一隻手籠着刀靶,一隻手按住掩心,便睜圓怪眼,倒豎剛鬚,便道:「婆子休得喫驚!自古寃有頭債有主,休推睡裏夢裏,我哥哥性命都在你身上!」婆子道:「武二哥,夜晚了,酒醉拿刀弄杖,不是耍處!」武松道:「婆子休胡說,我武二就死也不怕!等我問了這淫婦,慢慢來問你這老猪狗。若動一動步兒,身上先喫我五七刀子。」一面囬過臉來,看着婦人罵道:「你這淫婦聽着!我的哥哥怎生謀害了?従實說來,我便饒你。」那婦人道:「叔叔如何冷鍋中豆兒爆,好沒道理。你哥哥自害心疼病死了,干我甚事!」説猶未了,武松把刀子忔楂的插在桌子上,用左手揪住婦人雲髻,右手劈胸提住,把桌子一脚踢翻,碟兒盞兒都落地打得粉碎。那婦人能有多大氣脉,被這漢子隔桌子輕輕提將過來,拖出外間靈桌子前。

那婆子見頭勢不好,便走奔前門走,前門又上了拴。被武松大扠步趕上,揪番在地,用腰間纏帶解下來,四手四脚綑住,如猿猴獻菓一般,便脱身不得,口中只呌:「都頭不消動意,大娘子自做出來,不干我事。」武松道:「老猪狗,我都知了,你賴那個?你敎西門慶那廝墊發我充軍去,今日我怎生又囘家了,西門慶那廝卻在那裏?你不說時,先剮了這個淫婦,後殺你這老猪狗!」提起刀來,便望那婦人臉上撇兩撇。婦人慌忙呌道:「叔叔且饒!放我起來,等我說便了。」武松一提,提起那婆娘,旋剝凈了,跪在靈桌子前。武松喝道:「淫婦快說!」那婦人唬得魂不附體,只得従實招說,將那時收簾子打了西門慶起,並做衣裳入馬通姦,後怎的踢傷了武大心窝,用何人薬,王婆怎地教唆下毒,撥置燒化,又怎的娶到家去,一五一十,従頭至尾說了一遍。王婆聽見,只是暗地呌苦說:「儍材料,你實說了,卻敎老身怎的支吾!」這武松一面就靈前一手揪着婦人,一手澆奠了酒,把紙錢點着,說道:「哥哥,你陰魂不遠,今日武二與你報仇雪恨!」那婦人見頭勢不好,纔待大叫,被武松向爐内撾了一把香灰塞在他口,就叫不出來了,然後劈腦揪翻在地,那婦人掙扎,把䯼髻簪環都滚落了。武松恐怕他掙扎,先用油靴只顧踢他肋肢,後用兩隻脚踏他兩隻胳膊,便道:「淫婦,只說你伶俐,不知你心怎麽生着,我試看一看!」一面用手去攤開他胸脯,説時遲,那時快,把刀子去婦人白馥馥心窝内只一剜,剜了個血窟嚨,那鮮血就邈出來。那婦人就星眸半閃,兩隻脚只顧登踏。武松口噙着刀子,雙手去斡開他胸脯,撲忔的一聲,把心肝五臟生扯下來,血瀝瀝供養在靈前,後方一刀割下頭來,血流滿地。迎兒小女在旁看見,唬的只掩了臉。武松這漢子端的好狠也!可憐這婦人,正是三寸氣在千般用,一日無常萬事休!亡年三十二歲。但見:

\begin{myquote}
手到處青春喪命,刀落時紅粉亡身。七魄悠悠,已赴森羅殿上;三魂渺渺,應歸枉死城中。星眸緊閉,直挺挺屍横光地下;銀牙半咬,血淋淋頭在一邊離。好似初春大雪壓折金線柳,臘月狂風吹折玉梅花。這婦人嬌媚不知歸何處,芳魂今夜落誰家?
\end{myquote}

古人有詩一首,單悼金蓮死的好苦也:

\begin{myquote}
堪悼金蓮誠可憐,衣服脫去跪靈前。

誰知武二持刀殺,只道西門綁腿頑。

往事堪嗟一場夢,今身不値半文錢。

世間一命還一命,報應分明在眼前。
\end{myquote}

當下武松殺了婦人,那婆子看見,大呌:「殺人了!」武松聽見他呌,向前一刀,也割下頭來,拖過屍首。一邊將婦人心肝五臟,用刀插在樓後房簷下。那時也有初更時分,倒扣迎兒在屋裏,迎兒道:「叔叔,我也害怕!」武松道:「孩兒,我顧不得你了!」武松跳過王婆家來,還要殺他兒子王潮兒。不想王潮合當不該死,聽見他娘這邊呌,就知武松行兇。推前門不開,呌後門也不應,慌的走去街上呌保甲。那兩隣明知武松兇惡,誰敢向前?武松跳過牆來,到王婆房内,只見點着燈,房内一人也沒有。一面打開王婆箱籠,就把他衣服撒了一地,那一百兩銀子,止交與吳月娘二十兩,還剩下八十五兩,並些釵環首飾,武松一股皆休,都包裹了。提了朴刀,越後牆,赶五更挨出城門,投十字坡張青夫婦那裏躱住,做了頭陀,上梁山為盗去了。正是:平生不作縐眉事,世上應無切齒人。

畢竟未知後來如何,且聽下囘分解。

