\includepdf[pages={23,24},fitpaper=false]{tst.pdf}
\chapter*{第十二囬 \\潘金蓮私僕受辱 劉理星魘勝貪財}
\addcontentsline{toc}{chapter}{第十二囬 潘金蓮私僕受辱 劉理星魘勝貪財}
\markboth{\titlename}{第十二囬 潘金蓮私僕受辱 劉理星魘勝貪財}


\begin{myquote}
堪笑西門暴富,有錢便是主顧。

一家歪斯胡纏,那討綱常禮數!

狎客日日來往,紅粉夜夜陪宿。

不是長久夫妻,也算春風一度。
\end{myquote}

話說西門慶在院中,貪戀住桂姐姿色,約半月不曾來家。吴月娘使小廝一連拿馬接了數次,李家把西門慶衣帽都藏過一邊,不放他起身。丢的家中這些婦人都閑靜了。倒别人猶可,惟有潘金蓮這婦人,青春未及三十歲,慾火難禁一丈高。每日和孟玉樓兩個,打扮粉粧玉琢,皓齒朱唇,無一日不走在大門首倚門而望,等到黄昏時分。到晚來,歸入房中,粲枕孤幃,鳳臺無伴。睡不着,走來花園中款步花臺。月漾水底,猶恐西門慶心性難拿;怪玳瑁貓兒交歡,鬦的我芳心迷亂。當時玉樓帶來一個小廝,名喚琴童,年約十六歲,纔㽞起頭髮。生的眉目清秀,乖滑伶俐。西門慶敎他拿鑰匙看管花園打掃,晚夕就在花園門前一間小耳房内安歇。潘金蓮和孟玉樓白日裏常在花園中亭子上坐在一處做針指,或下棋。這小廝專一通小殷勤,常觀見西門慶來,就先來告報。以此婦人喜他,常叫他入房,賞酒與他吃。兩個朝朝暮暮,眉來眼去,都有意了。

不想將近七月廿八日,西門慶生日來到。吴月娘見西門慶在院中留戀煙花,不想囬家,一面使小廝玳安拿馬往院中接西門慶。這潘金蓮暗暗修了一柬帖,交付玳安,教「悄悄遞與你爹,説五娘請爹早些家去罷。」這玳安不敢怠慢,騎馬一直到勾欄李家。只見應伯爵、謝希大、祝日念,孫寡嘴、常時節衆人正在那裏相伴着西門慶,摟着粉頭,花攢錦簇,歡楽飲酒。西門慶看見玳安來到,便問:「你來怎麽?家中沒事?」玳安道:「家中沒事。」西門慶道:「前邊各項銀子叫傅二叔討討,等我到家算帳。」玳安道:「這兩日傅二叔討了許多,等爹到家上帳。」西門慶道:「你桂姨那一套衣服捎來不曾?」玳安道:「已捎在此。」便向毡包内取出一套紅衫藍裙,遞與桂姐。桂姐桂卿道了萬福,收了。連忙吩咐下邊,管待玳安酒飯。那小廝吃了酒飯,復走來上邊伺候。悄悄向西門慶耳邊附耳低言,説道:「家中五娘使我捎了個帖兒在此,請爹早些家去。」西門慶纔待用手去接,早被李桂姐看見。只道是西門慶前邊那婊子寄來的情書,一手撾過來,拆開觀看,卻是一幅迴文邊錦箋,上寫着幾行墨跡。桂姐遞與祝日念,敎念與他聽。這祝日念見上面寫詞一首,名〈落梅風〉,對衆朗誦了一遍:

\begin{myquote}
「黃昏想,白日思,盼殺人多情不至。因他為他憔悴死,可憐也繡衾獨自!

燈將殘,人睡也,空㽞得半窗明月。孤眠衾硬渾似鐵,這凄涼怎捱今夜?」
\end{myquote}

\begin{quotation}
\begin{flushright}
下書「愛妾潘六兒拜」。
\end{flushright}
\end{quotation}

那桂姐聽畢,撇了酒席,走入房中,倒在床上,面朝裏邊睡了。且説西門慶見桂姐惱了,把帖子扯的稀爛,衆人前把玳安踢了兩靴脚。請桂姐兩遍不來,慌的西門慶親自進房内抱出他來,到酒席上説道:「吩咐帶馬囬去,家中那個淫婦使你來,我這一到家,都打個臭死!」不説玳安含淚囬家。西門慶道:「桂姐,你休惱,這帖子不是别人的,乃是舍下第五個小妾投寄,請我到家,有些事兒計較,再無別故。」祝日念在旁又戲道:「桂姐,你休聽,他哄你哩!這個潘六兒,乃是那邊院裏新叙的一個婊子,生的一表人物,你休放他去。」西門慶笑趕着打,説道:「你這賊天殺的,單管弄死了人。緊着他恁麻犯人,你又胡説!」李桂卿道:「姐夫差了!既然家中有人拘管,就不消在前邊梳籠人家粉頭,自守着家裏那人兒便了。纔相伴了多少時?便就要抛離了去!」應伯爵插口道:「説的有理。」便道:「大官人,你依我,你也不消家去;桂姐也不必惱。今日説過,那個再恁惱了,每人罰二兩銀子,買酒肉咱大家吃。」到是這四五個嫖客,説的説,笑的笑,在席上猜枚行令,頑耍飲酒,把桂姐窝盤住了。西門慶把桂姐摟在懷中陪笑,一遞一口兒飲酒。少頃只見鮮紅漆丹盤拿了七鍾茶來,雪綻般茶盞,杏葉茶匙兒,鹽笋、芝蔴、木樨泡茶,馨香可掬,每人面前一盞。應伯爵道:「我有〈朝天子〉兒,單道這茶好處:

\begin{myquote}
這細茶的嫩芽,生長在春風下。不揪不採葉兒楂,但煮着顔色大。絶品清奇,難描難畫,口裏兒常時呷。醉了時想他,醒來時愛他,原來一簍兒千金價。」
\end{myquote}

謝希大笑道:「大官人使錢費物,不圖這『一摟兒』,却圖些甚的?如今每人有詞的唱詞,不會詞,每人説個笑話兒,與桂姐下酒。」該謝希大先,説:「有一個泥水匠,在院中墁地。老媽兒怠慢着他些兒,他暗暗把陰溝内堵上塊磚。落後天下雨,積的滿院子都是水。老媽慌了,尋的他來,多與他酒飯,還秤了一錢銀子,央他打水平。那泥水匠吃了酒飯,悄悄去陰溝内把那個磚拿出,把水登時出的罄盡。老媽便問:『作頭,此是那裏的病?』泥水匠回道:『這病與你老人家病一樣,有錢便流,無錢不流。』」原來把桂姐家來傷了。桂姐道:「我也有個笑話,囬奉列位:有一孫眞人,擺着筵席請人,卻教座下老虎去請。那老虎把客人一個個都路上吃了。眞人等至天晚,不見一客到。人都説:『你那老虎都把客人路上吃了。』不一時,老虎來,真人便問:『你請的客人都往那裏去了?』老虎口吐人言:『告師父得知,我従來不曉得請人,只會白嚼人,就是一能。』」當下把衆人都傷了。應伯爵道:「何見的俺們只是白嚼你家孤老,就還不起個東道?」於是向頭上拔下一根鬧銀耳斡兒來,重一錢;謝希大一對鍍金網巾圈,秤了秤,只九分半;祝日念袖中掏出一方舊汗巾兒,算二百文長錢;孫寡嘴腰間解下一條白布男裙,當兩壶半壜酒;常時節無以為敬,問西門慶借了一錢成色錢子:都遞與桂卿置辦東道,請西門慶和桂姐。那桂卿將銀錢都付與保兒,買了一錢螃蠏,打了一錢銀子猪肉,宰了一隻鷄,自家又賠出些小菜兒來。廚下安排停當。大盤小碗拿上來。衆人坐下,説了一聲「動筯吃」時,説時遲,那時快,但見:

\begin{myquote}
人人動嘴,個個低頭。遮天映日,猶如蝗蝻一齊來;擠眼掇肩,好似餓牢纔打出。這個搶風膀臂,如經年未見酒和餚;那個連三筷子,成歲不逢筵與席。一個汗流滿面,却似與鷄骨朵有冤仇;一個油抹唇邊,恨不把猪毛皮連唾嚥。吃片時,盃盤狼藉;啖良久,筯子縱横。盃盤狼籍,如水洗之光滑;筯子縱横,似打磨之乾淨。這個稱為食王元帥,那個號作淨盤將軍。酒壶翻晒又重斟,盤饌已無還去探。正是:珎羞百味片時休,果然都送入五臟廟。
\end{myquote}

當下衆人吃得個淨光王佛。西門慶與桂姐吃不上兩鍾酒,揀了些菜蔬,還被這夥人吃的去了。那日把席上椅子坐折了兩張。前邊跟馬的那小廝,不得上來掉嘴吃,把門前供養的土地翻倒來,使促恰剌了一泡む谷都的熱屎。臨出門來,孫寡嘴把李家明間内供養的鍍金銅佛塞在褲腰裏;應伯爵推鬦桂姐親嘴,把頭上金啄針兒戲了;謝希大把西門慶川扇兒藏了;祝日念走到桂卿房裏照臉,溜了他一面水銀鏡子;常時節借的西門慶一錢八成銀子,竟是寫在嫖帳上了。原來這起人,只伴着西門慶頑耍,好不快活。有詩為證:

\begin{myquote}
勾欄妓者媚如猱,只堪乘興暂時留。

若要死貪無足厭,家中金鑰教誰收?
\end{myquote}

按下這裡衆人簇擁着西門慶歡楽飲酒。單表玳安小廝囬馬到家,吴月娘和孟玉樓潘金蓮在房坐的,見了玳安,便問:「你接了爹來了不曾?」玳安哭的兩眼紅紅的,如此這般:「被爹踢罵了小的來了,説道那個再使人接,來家都要駡!」月娘便道:「你看,恁不合理!不來便了,如何去罵小廝來?如何狐迷變心這等的!」孟玉樓道:「你踢將小廝便罷了,如何連俺們都駡將來?」潘金蓮道:「十個九個院中淫婦,和你有甚情實?常言説的好:船載的金銀,填不滿煙花寨。」金蓮只知説出來,不妨路上説話,草裏有人。李嬌兒従玳安自院中來家時分,走來窗下潛聽。見潘金蓮對着月娘駡他家千淫婦萬淫婦,暗暗懷恨在心。従此二人結仇,不在話下。正是:

\begin{myquote}
甜言美語三冬煖,惡語傷人六月寒。

金蓮只曉爭先話,那料旁人起祸端。
\end{myquote}

不説李嬌兒與金蓮結仇。單表金蓮這婦人,歸到房中,捱一刻似三秋,盼一時如半夏。知道西門慶不來家,把兩個丫頭打發睡了。推在花園中遊玩,將琴童叫進房,與他酒吃,把小廝灌醉了,掩閉了房門,褪衣解帶,兩個就幹做在一處。正是:色膽如天怕甚事,鴛幃雲雨百年情。但見:

\begin{myquote}
一個不顧綱常貴賤,一個那分上下高低。一個色膽歪邪,管甚丈夫利害;一個淫心蕩漾,從他律犯明條。一個氣喑眼瞪,好似牛吼柳影;一個言嬌語澀,渾如鶯囀花間。一個耳畔訴雨意雲情,一個枕邊説山盟海誓。百花園内,翻為快活排場;主母房中,變作行楽世界。霎時一滴驢精髓,傾在金蓮玉體中。
\end{myquote}

自此為始,每夜婦人便叫這小廝進房中如此。未到天明,就打發出來。背地把金裹頭簪子兩三根帶在頭上,又把裙邊帶的錦香囊股子葫蘆兒也與了他,繫在身底下。豈知這小廝不守本分,常常和同行小廝在街吃酒耍錢,頗露出圭角。

常言若要人不知,除非己莫為。有一日,風聲吹到孫雪娥李嬌兒耳朵内,説道:「賊淫婦,往常言語假撇清,如何今日也做出來了?偷養小廝!」齊來告月娘。月娘再三不信,説道:「不爭你們和他合氣,惹的孟三姐不怪?只説你們擠撮他的小廝。」説的二人無言而退。落後,婦人夜間和小廝在房中行事,忘記関廚房門,不想被丫頭秋菊出來淨手看見了。次日傳與後邊小玉,小玉對雪娥説,雪娥同李嬌兒又來告訴月娘。——正值七月廿七日西門慶上壽,従院中來家。二人如此這般:「他屋裏丫頭親口説出來,又不是俺們葬送他。大娘不説,俺們對他爹説;若是饒了這個淫婦,只除非饒了蝎子娘是的!」月娘道:「他纔來家,又是他好日子。你們不依我,只顧説去;等住囬亂將起來,我不管你。」二人不聽月娘之言,約的西門慶進入房中,齊來告訴,説金蓮在家養小廝一節。這西門慶不聽萬事皆休,聽了怒従心上起,惡向膽邊生。走到前邊坐下,一片聲叫琴童兒。早有人報與潘金蓮。金蓮慌了手脚,使春梅忙叫小廝到房中,囑付千萬不要説出來。把頭上簪子都要過來收了,着了慌就忘解下了香囊葫蘆下來。——被西門慶叫到前廳跪下,吩咐三四個小廝,選大板子伺候。西門慶問道:「賊奴才,你知罪麽?」那琴童半日不敢言語。西門慶令左右:「除了帽子,拔下他簪子來我瞧!」見沒撇着金裹頭銀簪子,因問:「你戴的金裹頭銀簪子往那裏去了?」琴童道:「小的並沒甚銀簪子。」西門慶道:「奴才,還搗鬼!與我旋剥了衣服,拿板子打。」當下兩三個小廝扶持,一個剝去他衣服,扯了褲子,見他身底下穿着玉色絹て兒,て兒帶上露出錦香囊葫蘆兒。西門慶一眼就看見,便叫:「拿上來我瞧!」認的是潘金蓮裙邊帶的物件,不覺心中大怒,就問他:「此物従那裏得來?你實説,是誰與你的?」唬的小廝半日開口不得,説道:「這是小的某日打掃花園,在花園内拾的,並不曾有人與我。」西門慶越怒,切齒喝令:「與我綑起,着實打。」當下把琴童兒綳子綳着,雨點般攔杆打將下來。湏臾打了三十大棍,打得皮開肉綻,鮮血順腿淋漓。又敎大家人來保:「把奴才兩個鬢與我撏了!趕將出去,再不許進門。」那琴童磕了頭,哭哭啼啼出門去了。這小廝,只因昨夜與玉皇殿上掌書僊子廝調戲,今日罪犯天條貶下方。有詩為證:

\begin{myquote}
虎有倀兮鳥有媒,金蓮未必守空閨。

不堪今日私奴僕,自此遭愆更莫追。
\end{myquote}

當下西門慶打畢琴童,趕出去了。潘金蓮在房中聽見,如提在冷水盆内一般。不一時,西門慶進房來,唬的戰戰兢兢,渾身無了脈息,小心在旁扶侍接衣服,被西門慶兜臉打了個耳刮子,把婦人打了一跤。吩咐春梅:「把前後角門頂了,不放一個人進來!」拿張小椅兒坐在院内花架兒底下,取了一根馬鞭子,拿在手裏,喝令:「淫婦,脱了衣裳跪着!」那婦人自知理虧,不敢不跪。到是眞個脱去了上下衣服,跪在面前,低垂粉面,不敢出一聲兒。西門慶便問:「賊淫婦,你休推睡裏夢裏,奴才我纔已審問明白,他一一都供出來了!你實説,我不在家,你與他偷了幾遭?」婦人便哭道:「天麽天麽!可不寃屈殺了我罷了!自従你不在家,半個來月,奴白日裏只和孟三姐做一䖏做針指,到晚夕早関了房門就睡了,沒勾當不敢出這角門邊兒來。你不信,只問春梅便了。有甚私鹽私醋,他有個不知道的?」因叫春梅來:「姐姐你過來,親對你爹説。」西門慶罵道:「賊淫婦!有人説你把頭上金裹頭簪子兩三根都偷與了小廝,你如何不認?」婦人道:「就屈殺了奴罷了!是那個不逢好死的嚼舌根的淫婦,嚼他那旺跳的身子!見你常時進奴這屋裏來歇,他都氣不憤,拿這有天沒日頭的事壓枉奴!就是你與的簪子,都有數兒,一五一十都在,你查不是!我平白想起甚麽來與那奴才?好成器的奴才也不枉説的,恁一個尿不出來的毛奴才,平空把我纂一篇舌頭!」西門慶道:「簪子有沒罷了。」因向袖中取出琴童那香囊來,説道:「這個是你的物件兒,如何打小廝身底下搜出來?你還口漒甚麽?」説着,紛紛的惱了,向他白馥馥香肌上颼的一馬鞭子來,打的婦人疼痛難忍,眼噙粉淚,沒口子叫道:「好爹爹,你饒了奴罷!你容奴説,奴便説,不容奴説,你就打死奴,也只臭煙了這塊地。這個香囊葫蘆兒,你不在家,奴那日同孟三姐在花園裏做生活,因從木香欄下所過,帶繫兒不牢,就抓落在地。我那裏沒尋,誰知這奴才拾了。奴並不曾與他。」只這一句,就合着剛纔琴童前廳上供稱在花園内拾的一樣的話,又見婦人脱的光赤條條,花朵兒般身子,嬌啼嫩語,跪在地下,那怒氣早已鑽入爪哇國去了,把心已囬動了八九分。因叫過春梅,摟在懷中問他:「淫婦果然與小廝有首尾沒有?你説饒了淫婦,我就饒了罷。」那春梅撒嬌撒痴,坐在西門慶懷裏,説道:「這個爹,你好沒的説!和娘成日唇不離腮,娘肯與那奴才?這個都是人氣不憤俺娘兒們,作做出這樣事來。爹,你也要個主張,好把醜名兒頂在頭上,傳出外邊去好聽?」幾句把西門慶説的一聲兒不言語,丢了馬鞭子,一面敎金蓮起來,穿上衣服,吩咐秋菊看菜兒、放桌兒吃酒。這婦人當下滿斟了一盃酒,雙手遞上去。花枝招颭、繡帶飄飄,跪在地下,等他鍾兒。西門慶吩咐道:「我今日饒了你,我若但凡不在家,要你洗心改正,早関了門户,不許你胡思亂想。我若知道,定不饒你!」婦人道:「你吩咐,奴知道了。」到是插燭也似與西門慶磕了四個頭,方纔安座兒,在旁陪坐飲酒。正是:為人莫作婦人身,百年苦楽由他人。潘金蓮這婦人,平日被西門慶寵的狂了,今日討得這塲羞辱在身上。有詩為證:

\begin{myquote}
金蓮容貌更溫柔,恃寵爭姸惹寇仇。

不是春梅當日勸,父娘皮肉怎禁抽。
\end{myquote}

西門慶正在金蓮房中飲酒,忽聽小廝打門,説:「前邊有吴大舅、吴二舅、傅夥計、女兒、女婿、衆親戚,送禮來祝壽。」方纔撇了金蓮,整衣出來前邊陪待賓客。那時,應伯爵謝希大等衆人都有人情。院中李桂姐家,亦使保兒送禮來。西門慶前邊亂着,收人家禮物,發柬請人,不在話下。

且説孟玉樓打聽金蓮受辱,約的西門慶不在家裏,瞞着李嬌兒孫雪娥,走來看望金蓮。見金蓮睡在床上,因問道:「六姐,你端的怎麽緣故,告我説則個。」那金蓮滿眼流淚,哭道:「三姐,你看小淫婦今日在背地裏白唆調漢子,打了我恁一頓。我到明日,和這兩個淫婦冤仇結得有海深!」玉樓道:「你便與他有瑕玷,如何做作着把我的小廝弄出去了!六姐,你休煩惱。莫不漢子就不聽俺們説句話兒?若明日他不進我房裏來便罷,但到我房裏來,等我慢慢勸他。」金蓮道:「多謝姐姐費心。」一面叫春梅看茶來吃。坐着説了囬話,玉樓告辭回房去了。至晚,西門慶因上房吴大妗子來了,走到玉樓房中宿歇。玉樓因説道:「你休枉了六姐心,六姐並無此事。都是日前和李嬌兒孫雪娥兩個有言語,平白把我的小廝扎筏子。你不問個青紅皂白,就把他屈了。你怪六姐,却不難為六姐了?我就替他賭個大誓。若果有此事,大姐姐有個不先説的?」西門慶道:「我問春梅,他也這般説。」玉樓道:「他今在房中不好哩,你不去看他看去?」西門慶道:「我知道,明日到他房中去。」當晚無話。

到第二日,西門慶正生日。有周守備、夏提刑、張團練、吴大舅,許多官客飲酒。拿轎子接了李桂姐並兩個唱的,唱了一日。李嬌兒見他姪女兒來,引着拜見月娘衆人,在上房裏坐吃茶。請潘金蓮見,連使丫頭請了兩遍,金蓮不出來,只説心中不好。到晚夕,桂姐臨家去,拜辭月娘。月娘與他一件雲絹比甲兒、汗巾、花翠之類,同李嬌兒送出到門首。桂姐又親自到他花園角門首:「好歹見見五娘。」那金蓮聽見他來,使春梅把角門関閉得鐵桶相似,就是樊噲也呌不開。説道:「我不開!」這花娘遂羞訕滿面而回。正是:廣行方便,為人何處不相逢?多結寃仇,路逢狹處難囬避。

不題李桂姐囬家去了。單表西門慶至晚進入金蓮房内來。那金蓮把雲鬢不整,花容倦淡,迎接進房。替他脱衣解帶,伺候茶湯脚水,百般殷勤扶持,把小意兒貼戀。到夜裏,枕蓆魚水歡娱,屈身忍辱,無所不至。説道:「我的哥哥,這一家都誰是疼你的?都是露水夫妻,再醮貨兒!惟有奴知道你的心,你知道奴的意。旁人見你這般疼奴,在奴身邊去的多,都氣不憤,背地裏架舌頭,在你跟前唆調。我的儍寃家,你想起甚麽來!中了人的拖刀之計,把你心愛的人兒這等下無情折剉。常言道:家鷄打的團團轉,野雞打的貼天飛。你就把奴打死了,也只在這屋裏,敢往那裏去?就是前日,你在院裏踢罵了小廝來,早是有上房大姐姐孟三姐在跟前,我自不是,説了一聲,也是為你好——恐怕他家裏粉頭掏淥壞了你身子。院中唱的,只是一味愛錢,和你有甚情節,誰人疼你?誰知被有心的人聽見,兩個背地捎成一幫兒算計我。自古人害人不死,天害人纔害死了!往後久而自明,只要你與奴做個主兒便了。」於是幾句把西門慶説的窝盤住了,是夜與他淫慾無度。

到次日,西門慶備馬,玳安平安兩個小廝跟隨,往院中來。卻説李桂姐正打扮着陪人坐的,聽見他來,連忙走進房去,洗了濃粧,除了簪環,倒在床上,裹衾而卧。西門慶去到,坐了半日,還沒一個出來陪侍。只見老媽出來,道了萬福,讓西門慶坐下。虔婆便問:「怎的姐夫連日不進來走走?」西門慶道:「正是因賤日窮冗,家中無人。」虔婆道:「姐兒那日打擾!」西門慶道:「怎的那日姐姐桂卿不來走走?」虔婆道:「桂卿不在家,被客人接去店裏,這幾日還不放了來。」説了半日話,小頂人拿茶來,陪着吃了。西門慶便問:「怎的不見桂姐?」虔婆道:「姐夫還不知哩!小孩兒家不知怎的那日着了惱來家,就不好起來,睡倒了。房門兒也不出,直到如今。姐夫好狠心,也不來看看姐兒。」西門慶道:「真個?我通不知。」因問:「在那邊房裏?我看看去。」虔婆道:「在他後邊臥房裏睡。」慌忙令丫鬟掀簾子。西門慶走到他房中,只見粉頭烏雲散亂,粉面慵粧,裹被便卧在那床上,面朝裏。見了西門慶,不動一動兒。便問道:「你那日來家怎的不好?」也不答應。又問:「你着了誰人惱,你告我説。」問了半日,那桂姐方開言説道:「左右是你家五娘子!你家中既有恁好的迎奸賣俏,又來稀罕俺們這樣淫婦做甚麽?俺們雖是門户中出身,蹺起脚兒,比外邊良人家不成材的貨兒高好些。我前日又不是供唱,我也送人情去。大娘倒見我甚是親熱,又那兩個,與我許多花翠衣服。待要不請你見,又説俺院中沒禮法。只聞知人説,你家有好個五娘子,當請出你拜見,又不出來。家來,同俺姑娘又辭你去,你使丫頭把房門関了。端的好不識人敬重!」西門慶道:「你倒休怪他。他那日本等心中不自在。他若好時,有個不出來見你的?這個淫婦,我幾次因他再三咬羣兒,口嘴傷人,也要打他哩!」這桂姐反手向西門慶臉上一掃,説道:「沒羞的哥兒,你就打他!」西門慶道:「你還不知我手段。除了俺家房下,家中這幾個老婆丫頭,但打起來也不善,着緊二三十馬鞭子還打不下來,好不好還把頭髮都剪了!」桂姐道:「我見砍頭的,没見砍嘴的。你打,三個官兒唱兩個喏,誰見來?你若有本事,到家裏只剪下一柳子頭髮,拿來我瞧,我方信你是本司三院有名的好子弟!」西門慶道:「你敢與我排手?」那桂姐道:「我和你排一百個手!」當日西門慶在院中歇了一夜。到次日黄昏時分,辭了桂姐,上馬回家。桂姐道:「我在這裏眼望旌節旗,耳聽好消息。哥兒,你這一去,沒有這物件就休要見我!」

這西門慶吃他激怒了幾句話,歸家已是酒酣。不往别房裏去,逕到前邊潘金蓮房來。婦人見他有酒了,加意用心伏侍。問他酒飯,都不吃。吩咐春梅:「把床上拭抹凉蓆乾淨。帶上門,出去!」他便坐在床上,令婦人脱靴,那婦人不敢不脱。湏臾脱了靴,打發他上床。西門慶且不睡,坐在一隻枕頭上,令婦人褪了衣服,地下跪着。那婦人唬的揑兩把汗,又不知因為甚麽,於是跪在地下,柔聲大哭道:「我的爹爹,你透與奴個伶俐説話,奴死也甘心!饒奴終夕恁提心吊膽,陪着一千個小心,還投不着你的機會,只㧱鈍刀子鋸處我,教奴怎生吃受?」西門慶駡道:「賤淫婦,你眞個不脱衣裳,我就沒好意了!」因呌春梅:「門背後有馬鞭子,與我取了來!」那春梅只顧不進房來。叫了半日,纔慢條絲禮推開房門進來,——看見婦人跪在床地平上,——向燈前側着身兒下了油。西門慶使他,只不動身。婦人叫道:「春梅,我的姐姐,你救我救兒!他如今要打我。」西門慶道:「小油嘴兒,你不要管他。你只遞馬鞭子與我,打這淫婦!」春梅道:「爹,你怎的恁沒羞!娘幹壞了你的甚麽事兒,你信淫婦言語來?平地裏起風波,要便搜尋娘,還教人和你一心一計哩!你敎人有那眼兒看得上你!」到是也不依他,拽上房門,走在前邊去了。那西門慶無法可處,反呵呵笑了,向金蓮道:「我且不打你,你上來。我問你要樁物兒,你與我不與我?」婦人道:「好親親,奴一身骨朶肉兒都屬了你,隨要甚麽,奴無有不依隨的。不知你心裏要甚麽兒?」西門慶道:「我心要你頂上一柳兒好頭髮。」婦人道:「好心肝,淫婦的身上,隨你怎的揀着燒遍了也依,這個剪頭髮却成不的,可不唬死了我罷了!奴出娘胞兒活了二十六歲,從沒幹這營生。打緊我頂上這頭髮,近來又脱了奴好些,只當可憐見我罷!」西門慶道:「你只嗔我惱我,説的你就不依我?」婦人道:「我不依你再依誰?」因問:「你實對奴説,要奴這頭髮做甚麽去?」西門慶道:「我要做網巾。」婦人道:「你要做網巾,我就與你做。休要拿與淫婦,教他好壓鎮我。」西門慶道:「我不與人便了,要你髮兒做頂線兒。」婦人道:「你既要做頂線,待奴剪與你。」當下婦人分開頭髮,西門慶拿剪刀,按婦人當頂上齊臻臻剪下一大柳來,用紙包放在順袋内。婦人便倒在西門慶懷中,嬌聲哭道:「奴凡事依你,只願你休忘了心腸,隨你前邊和人好,只休抛閃了奴家!」是夜與他歡會異常。

到次日,西門慶起身,婦人打發他吃了飯出門,騎馬逕到院裏。桂姐便問:「你剪的他頭髮在那裏?」西門慶道:「有,在此。」便向茄袋内取出,遞與桂姐。打開觀看,果然黑油也一般好頭髮,就收在袖中。西門慶道:「你看了還與我,他昨日為剪這頭髮,好不費難。吃我變了臉惱了,他纔容我剪下這一柳子來。我哄他只説要做網巾頂線兒,逕拿進來與你瞧。可見我不失信。」桂姐道:「甚麽稀罕貨!慌的你恁個腔兒!等你家去,我還與你。比是你恁怕他,就不消剪他的來了!」西門慶笑道:「那裏是怕他的,我言語不的了。」桂姐一面教桂卿陪着他吃酒,走到背地裏,把婦人頭髮早絮在鞋底下,每日躧踏,不在話下。到是把西門慶纏住,連過了數日,不放來家。

金蓮自従頭髮剪下之後,覺意心中不快,每日房門不出,茶飯慵餐。吴月娘使小廝請了家中常走着的那劉婆子看視,説:「娘子着了些暗氣暗惱在心中,不能囬轉,頭疼惡心,飲食不進。」一面打開薬包來,㽞了兩服黑丸子薬兒:「晚上用薑湯吃。」又説:「我明日叫俺老公來,替你老人家看看今歲流年,有災沒有。」金蓮道:「原來你家老公也會算命?」劉婆道:「他雖是個瞽目人,到會兩三樁本事:第一善陰陽講命,與人家禳保;第二,會針灸收瘡;第三樁兒不可説,單管與人家囬背。」婦人問道:「怎麽是囬背?」劉婆子道:「如有父子不和,兄弟不睦,大妻小妻爭鬦,敎了俺這老公去説了,替他用鎮物安鎮,鎮書符水與他吃了,不消三日,教他父子親熱,兄弟和睦,妻妾不爭。若人家買賣不順溜,田宅不興旺者,常與人開財門、發利市。治病洒掃,禳星告斗都會,因此人都叫他做劉理星。也是一家子新娶個媳婦兒,是小人家女兒,有些手脚兒不穩,常偷盜婆婆家東西往娘家去。丈夫知道,常被責打。俺老公與他囬背,書了二道符,燒灰放在水缸下埋着。渾家大小吃了缸内水,眼看着媳婦偷盜,只像沒看見一般。又放一件鎮物在枕頭内,男子漢睡了那枕頭,也好似手封住了的,再不打他了。」那潘金蓮聽見,遂㽞心,便叫丫頭打發茶湯點心與劉婆吃了。臨去包了三錢薬錢,另外又秤了五錢,敎買紙紮信物,明日早飯時叫劉瞎來燒神紙。

那劉婆子作辭回家。到次日,果然大清早晨,領賊瞎逕進大門,往裏走。那日,西門慶還在院中未來。看門小廝便問:「瞎子往那裏走?」劉婆道:「今日與裏邊五娘燒紙。」小廝道:「既是與五娘燒紙,老劉你領進去,仔細看狗!」這婆子領定,徑到潘金蓮臥房明間内。等到半日,婦人纔出來。瞎子見了禮,坐下。婦人説與他八字。賊瞎子用手掐了掐,説道:「娘子庚辰年、庚寅月、乙亥日、己丑時。初八日立春,已交正月算命。依子平正論,娘子這八字中雖故清奇,一生不得夫星濟,子上有些妨碍。亥中一木,生到正月間,亦作身旺論,不尅當自焚。又兩重庚金,羊刄大重,夫星難為,尅過兩個纔好。」婦人道:「已尅過了。」賊瞎子道:「娘子這命中,休怪小人説,子平雖取煞印格,只吃了亥中有壬水,辰丑中又有癸水,水太多了,冲動了只一重己土,官煞混雜。論來男人煞重掌威權,女子煞重必刑夫。所以主為人聰明機變,得人之寵愛。只有一件,今歲流年甲辰,歲運並臨,災殃必至。命中又犯小耗勾絞兩位星辰打攪,雖不能傷,只是主有比肩不和,小人嘴舌,常沾些啾唧不寜之狀。」婦人聽了,説道:「累先生仔細用心,與我囬背囬背。我這裏一兩銀子相謝,先生買一盞茶吃。奴不求別的,只願得小人離退,夫主愛敬便了。」一面轉入房中,拔了兩件首飾,遞與賊瞎。賊瞎接了,放入袖中,説道:「既要小人囬背,用柳木一塊,刻兩個男女人形像,書着娘子與夫主生時八字。用七七四十九根紅綫,紮在一處。上用紅紗一片,蒙在男子眼中,用艾塞其心,用針釘其手,下用膠粘其足,暗暗埋在睡的枕頭内。又朱砂書符一道,燒火灰,暗暗攪在釅茶内。若得夫主吃了茶,到晚夕睡了枕頭,不過三日,自然有驗。」婦人道:「請問先生,這四樁兒是怎的説?」賊瞎道:「好教娘子得知:用紗蒙眼,使夫主見你一似西施一般嬌豔;用艾塞心,使他心愛到你;用針釘手,隨你怎的不是,使他再不敢動手打你,着緊還跪着你;用膠粘足者,使他再不往那裏胡行。」婦人聽言有這等事,滿心歡喜。當下備了香燭紙馬,替婦人燒了紙。到次日,使劉婆送了符水鎮物與婦人,如法安頓停當。將符燒灰,炖下好茶,待的西門慶家來,婦人叫春梅遞茶與他吃,到晚夕與他共枕同床。過了一日兩,兩日三,似水如魚,歡會異常。看官聽説:但凡大小人家,師尼僧道,乳母牙婆,切記休招惹他。背地裏甚麽事不幹出來?古人有四句格言説得好:

\begin{myquote}
堂前切莫走三婆,後門常鎖莫通和。

院内有井防小口,便是祸少福星多。
\end{myquote}

畢竟未知後來如何,且聽下囬分解。

