\includepdf[pages={95,96},fitpaper=false]{tst.pdf}
\chapter*{第四十八囬 \\曾御史參劾提刑官 蔡太師奏行七件事}
\addcontentsline{toc}{chapter}{第四十八囬 曾御史參劾提刑官 蔡太師奏行七件事}
\markboth{{\titlename}卷之五}{第四十八囬 曾御史參劾提刑官 蔡太師奏行七件事}


格言:

\begin{myquote}
知危識險,終無羅網之門;譽善薦賢,自有安身之地。施恩布德,乃後代之榮昌;懷妬藏奸,為終身之祸患。損人利己,終非遠大之圖;害衆成家,豈是長久之計?改名異體,皆因巧語而生;訟起傷財,蓋為不仁之召。
\end{myquote}

話説安童領着書信,辭了黃通判,往山東大道而來。打聽巡按御史在東昌府察院住扎,姓曾,雙名孝序,乃都御史曾布之子,新中乙未科進士,極是個清廉正氣的官。這安童自思:「我若説下書的,門上人決不肯放。不如我在此等着放告牌出來,我跪門進去,連狀帶書呈上。老爹見了,必然有個决断。」於是早已把狀子寫下,揣在懷裏,在察院門首等候多時。只聽裏面打的雲板響,開了大門二門,曾御史坐廳。頭面牌出來,大書:告親王皇親駙馬勢豪之家;第二面牌出來:告都布按並軍衛有司官吏;第三面牌出來,纔是百姓户婚田土詞訟之事。這安童就隨狀牌進去。待把一應事情發放淨了,方走在丹墀上跪下。兩邊左右問是做甚麽的,這安童方纔把書雙手擧得高高的呈上。只聽公座上曾御史叫:「接上來!」慌的左右吏典下來,把書接上去,安放於書案上。曾公拆開觀看,端的上面寫着甚言詞?書曰:

\begin{myquote}[\markfont]
\hspace*{2em}「寓都下年教生黃羙端肅書奉

大柱史少亭曾年兄先生大人門下:違越光儀,倏忽一載,知己難逢,勝遊易散。此心耿耿,常在左右。去秋忽報瑤章華札,開軸啟函,捧誦之間,而神遊恍惚,儼然長安對面時也。每有感愴,輒一歌之,足舒懷抱矣!未幾,年兄省親南旋,復聞德音,知年兄按巡齊魯,不勝欣慰,叩賀,叩賀!惟年兄忠孝大節,風霜貞操,砥礪其心,耿耿在廊廟,歷歷在士論。今茲出巡,正當摘發官邪,以正風紀之日。區區愛念,尤所不能忘者矣。竊謂年兄平日抱可為之器,當有為之年,値聖明有道之世,老翁在家康健之時,可乘此大展才猷,以振揚法紀,勿使舞文之吏以撓其法;而奸頑之徒以逞其欺。胡乃如東平一府,而有撓大法如苗青者,抱大寃如苗天秀者乎!生不意聖明之世,而有此魍魎!年兄巡歷此方,正當分理寃滯,振刷為之一清可也。去伴安童,持狀告訴,幸垂察。不宣。仲春望後一日具。」
\end{myquote}

這曾御史覽書已畢,便問:「有狀沒有?」左右慌忙下來問道:「老爺問你有狀没有?」這安童向懷中取狀遞上。曾公看了,取筆批:「仰東平府府官,従公查明,驗相屍首,連卷詳報。」喝令安童東平府伺候。這安童連忙磕頭起來,従便門放出。這裏曾公將批詞連狀裝在封套内,鈐了關防,差人賫送東平府來。府尹胡師文見了上司批下來,慌得手脚無措。即調委陽谷縣縣丞狄斯彬。本貫河南舞陽人氏,為人剛而且方,不要錢;問事糊突,人都號他做『狄混』。明文下來,沿河查訪苗天秀屍首下落。

也是合當有事,不想這狄縣丞率領一行人,巡訪到清河縣城西河邊。正行之際,忽見馬頭前起一陣旋風,團團不散,只隨着狄公馬走。狄縣丞道:「怪哉!」遂勒住馬,令左右公人:「你去隨此旋風,務要跟尋個下落。」那公人眞個跟定旋風而來,七八將近新河口而止。走來囬覆了狄公話。狄公即拘了里老來,用鍬掘開岸土,深數尺,見一死屍,宛然頸上有一刀痕,命仵作檢視明白。問其前面是那裏,公人禀道:「離此不遠,就是慈惠寺。」縣丞即令拘寺中僧行問之。皆言:「去冬十月中,本寺因放水燈兒,見一死屍従上流而來,漂入港裏。長老慈悲,故收而埋之。不知為何而死。」縣丞道:「分明是汝衆僧謀殺此人,埋於此䖏。想必身上有財帛,故不肯實説。」於是不由分説,先把長老一箍兩拶,一夾一百敲,餘者衆僧都是二十板,俱令收入獄中。囬覆曾公,再行報看。各僧皆稱寃不服。曾公尋思:「旣是此僧謀死,屍必棄於河中,豈反埋於岸上?」又説:「干碍人衆,此有可疑。」因令將衆僧收監。將近兩月,不想安童來告此狀,即令委官押安童前至屍所,令其認視。這安童見其屍大哭道:「正是我的主人,被賊人所傷,刀痕尚在。」於是檢驗明白,囬報曾公,即把衆僧放囬。一面查刷卷宗,復提出陳三翁八審問,執稱苗青主謀之情。曾公大怒,差人行牌,星夜往揚州提苗青去了。一面寫本參劾提刑院兩員問官受贜賣法。正是:

\begin{myquote}
汚吏贓官濫國刑,曾公判刷雪寃情。

雖然號令風霆肅,萬裏輸贏總未眞。
\end{myquote}

話分兩頭,卻表王六兒自従得了苗青幹事的那一百兩銀子、四套衣服,與他漢子韓道國就白日不閑,一夜沒的睡,計較着要打頭面,治簪環,喚裁縫來裁衣服,従新抽銀絲䯼髻。用十六兩銀子又買了個丫頭,名喚春香使喚,早晚教韓道國收用,不題。一日,西門慶到韓道國家,王六兒接着,裏面喫茶畢,西門慶往後邊淨手去,看見隔壁月臺,問道:「是誰家的?」王六兒道:「是隔壁楽三家月臺。」西門慶吩咐王六兒:「如何教他遮住了這邊風水?你對他説,若不與我即便拆了,不然我叫地方吩咐他!」這王六兒與韓道國説:「鄰舍家,怎好與他説的?」韓道國道:「咱不如瞞着老爹,廟上買幾根木植來,咱這邊也搭起個月臺來。上面曬醬,下邊不拘做馬坊,做個東淨,也是好處。」老婆道:「呸!賊沒算計的!比是搭月臺,買些磚瓦來蓋上兩間廈子卻不好?」韓道國道:「蓋兩間廈子倒不好了,是東子房子了。不如蓋一層兩間小房罷!」於是使了三十兩銀子,又蓋了兩間平房起來。西門慶差玳安擡了許多酒肉燒餅來,與他家犒勞匠人。那條街上,誰人不知。

夏提刑得了幾百兩銀子在家,把兒子夏承恩,年十八歲,幹入武學肄業,做了生員。每日邀結師友習學弓馬。西門慶約會劉薛二内相、周守備、荆都監、張團練,合衛官員,出人情與他掛軸文慶賀,俱不必細説。

西門慶因墳上新蓋了山子捲棚房屋,自従生了官哥,並做了千户,還沒往墳上祭祖。敎陰陽徐先生看了,従新立了一座墳門,砌的明堂神路,門首栽柳,週圍種松柏,兩邊疊的坡峯。清明日上墳,要更換錦衣牌扁,宰猪羊,定桌面。三月初六日清明,預先發柬,請了許多人;推運了東西,酒米、下飯菜蔬。叫的楽工雜耍扮戲的:小優兒是李銘、吳惠、王柱、鄭奉,唱的是李桂姐、吳銀兒、韓金釧、董嬌兒。官客請了張團練、喬大户、吳大舅、吳二舅、花大舅、沈姨夫、應伯爵、謝希大、傅夥計、韓道國、雲離守、賁地傳,并女婿陳經濟等約二十餘人。堂客請了張團練娘子、張親家母、喬大户娘子、朱臺官娘子、尚擧人娘子、吳大妗子、二妗子、楊姑娘、潘姥姥、花大妗子、吳大姨、孟大姨、吳舜臣媳婦鄭三姐、崔本妻段大姐,并家中吳月娘、李嬌兒、孟玉樓、潘金蓮、李瓶兒、孫雪娥、西門大姐、春梅、迎春、玉簫、蘭香,奶子如意兒抱着官哥兒,裏外也有二十四五頂轎子。先是月娘對西門慶説:「孩子且不消教他往墳上去罷。一來還不曾過一周;二者劉婆子説這孩子囟門還未長滿,膽兒小。這一到墳上,路遠,只怕唬着他。依着我,不教他去。留下奶子和老馮在家和他做伴兒。只教他娘母子一個去罷。」西門慶不聽,便道:「比來為何?他娘兒兩個不到墳前與祖宗磕個頭兒去?你信那婆子老淫婦胡説,可可就是孩子囟門未長滿!教奶子用被兒裹着,在轎子裏按的孩兒牢牢的,怕怎的?」那月娘便道:「你不聽人説,隨你。」

従清早晨,堂客都従家裏取齊起身,上了轎子,一路無辭。出南門,到五里原祖墳上,遠遠望見青松鬱鬱,翠柏森森。新蓋的墳門,兩邊坡峯上去,週圍石牆,當中甬路。明堂神臺、香爐、燭臺,都是白玉石鑿的。墳門上新安的牌扁,大書:「錦衣武畧將軍西門氏先塋。」墳内正面土山環抱,林樹交枝。西門慶穿大紅冠帶,擺設猪羊祭品桌席祭奠。官客祭畢,堂客纔祭。響器鑼鼓一齊打起來。那官哥兒唬的在奶子懷裏磕伏着,只倒咽氣,不敢動一動兒。月娘便叫:「李大姐,你還不教奶子抱了孩子往後邊去罷!你看唬的那腔兒!我説且不敎孩兒來罷,恁漒的貨,只當敎抱了他來。你看唬的那孩兒這模樣!」李瓶兒連忙下來,吩咐玳安且叫把鑼鼓住了,連忙攛掇:「掩着孩兒耳朵,快抱了後邊去罷。」須臾祭畢,徐先生唸了祭文,燒了紙。西門慶邀請官客在前客位。月娘邀請堂客在後邊捲棚内:由花園進去,兩邊松墻普築,竹徑欄杆。週圍花草,一望無際。正是:桃紅柳綠鶯梭織,都是東君造化成。當下扮戲的在捲棚内扮與堂客們瞧。兩個小優兒在前廳官客席前唱了一囬,四個唱的輪番遞酒。春梅玉簫蘭香迎春四個,都在堂客上邊執壺斟酒,就立在大姐桌頭同喫湯飯點心。喫了一囬,潘金蓮與玉樓、大姐、李桂姐、吳銀兒,同往花園裏打了囬鞦韆。

原來捲棚後邊,西門慶收拾了一明兩暗三間牀炕房兒。裏邊鋪陳牀帳,擺放桌椅、梳籠、抿鏡、粧臺之類,預備堂客來上墳,在此梳粧歇息,或閒常接了妓者在此頑耍。糊的猶如雪洞般乾淨,懸掛的書畫,琴棋瀟灑。奶子如意兒看守官哥兒,正在那灑金牀炕兒舖着小褥子兒睡。迎春也在傍和他頑耍。只見潘金蓮獨自従花園驀地走來,手中拈着一枝桃花兒。進屋裏,看見迎春,便道:「你原來這一日没在上邊伺候。」迎春道:「有春梅蘭香玉簫在上邊哩。俺娘教我下邊來看哥兒,拿了兩碟下飯點心,與如意兒喫。」金蓮看見那邊桌上放着一碟子鵝肉,一碟蹄子肉,並幾個菓子。奶子見金蓮來,便抱起官哥兒來。金蓮便戲他説道:「小油嘴兒,頭裏見打起鑼鼓來,唬的不則聲,原來這等小膽兒!」於是一面解開藕絲羅襖兒銷金衫兒,接過孩兒,抱在懷裏,與他兩個嘴對嘴親嘴兒。忽有陳經濟掀簾子走入來,看見金蓮鬦孩子頑耍,也鬦那孩子。金蓮道:「小道兒,你也與姐夫個嘴兒。」可霎作怪,那官哥兒便嘻嘻望着他笑。經濟不由分説,把孩子就摟過來,一連親了幾個嘴。金蓮罵道:「怪短命,誰家親孩子把人的鬢都抓亂了!」經濟等戲道:「你還説,早是我沒錯親了哩。」金蓮聽了,恐怕婢子瞧科,便戲發訕將手中拿的扇子,倒過把子來向他身上打了一下,打的經濟鯽魚般跳。罵道:「怪短命,誰和你那等調嘴調舌的!」經濟道:「不是,你老人家摸量惜些情兒。人身上穿着恁單衣裳,就打恁一下!」金蓮道:「我平白惜甚情兒?今後惹着我。只是一味打。」如意兒見他頑的訕,連忙把官哥兒接過來抱着。金蓮與經濟兩個還戲謔一處。金蓮將那一枝桃花兒做了一個圈兒,悄悄套在經濟帽子上。走出去,正値孟玉樓和大姐桂姐三個従那邊來。大姐看見,便問:「是誰幹的營生?」經濟取下來丢了,一聲兒也沒言語。

堂客前戲文扮了四大摺。看看窗外日光彈指過,席前花影座間移,看看天色晚來。西門慶吩咐賁四,先把擡轎子的每人一碗酒,四個燒餅,一盤子熟肉。俵散停當,然後才把堂客轎子起身。官客騎馬在後,來興兒與廚役慢慢的擡食盒煞後。玳安來安畫童棋童兒,跟月娘衆人轎子,琴童並四名排軍,跟西門慶馬。奶子如意兒獨自坐一頂小轎,懷中抱着哥兒,用被裹的緊緊的進城。月娘還不放心,又使囬畫童兒來,叫他跟定着奶子轎子,恐怕進城人亂。

且説月娘轎子進了城,就與喬家那邊衆堂客轎子分路來家,先下轎進去。半日,西門慶陳經濟纔到家下馬。只見平安兒迎門就禀説:「今日掌刑夏老爹親自下馬到廳,問了一遍去了。落後又差人問了兩遍。不知有甚勾當。」西門慶聽了,心中猶豫。到於廳上,只見書童兒在傍接衣服。西門慶因問:「今日你夏老爹來,㽞下甚麽話來?」書童道:「他也沒説出來,只問爹往那去了,『使人請去,我有句要緊話兒説!』小的便道:『今日都往墳上燒紙去了,至晚纔來。』夏老爹説:『我到午上還來。』落後又差人來問了兩遭,小的説還未來哩。」西門慶心中不定,心下轉道:「卻是甚麽?」正疑惑之間,只見平安來報:「夏老爹來了!」那時已有黄昏時分。只見夏提刑便衣坡巾,兩個伴當跟隨,下馬到於廳上,敍禮,説道:「長官今日往寳莊去來?」西門慶道:「今日先塋祭掃。不知長官下降,失迎。恕罪恕罪!」夏提刑道:「敢來有一事報與長官知道。」因説:「咱們往那邊客位内坐去罷。」西門慶令書童開捲棚門,請往那裏説話,左右都令下去。夏提刑道:「今朝縣中李大人到學生那裏,如此這般,説大巡新近有參本上東京,長官與學生俱在參例。學生令人抄了個邸報在此,與長官看。」西門慶聽了,大驚失色,急接過邸報來,燈下觀看。端的上面寫着甚言詞?

\begin{myquote}[\markfont]
「巡按山東監察御史曾孝序一本:參劾貪肆不職武官,乞賜罷黜,以正法紀事。臣聞巡蒐四方,省察風俗,乃

天子巡狩之事也;彈壓官邪,振揚法紀,乃御史糾政之職也。昔《春秋》載天王巡狩而萬邦懷保,民風協矣,王道彰矣,四民順矣,

聖治明矣。臣自去歲奉

命巡按山東齊魯之邦,一年將滿。歷訪方面有司,文武官員賢否,頗得其實。兹當差滿之期,敢不循例甄別,為我

皇上陳之。除參劾有司方面官員,另具疏上請。參照山東提刑所掌刑金吾衛正千户夏延齡:闒茸之材,貪鄙之行,久干物議,有玷班行。昔者典牧

皇畿,大肆科擾,被屬官陰發其私;今省理山東刑獄,復著狼貪,為同僚之所箝制。縱子承恩,冒籍武擧,倩人代考,而士風掃地矣!信家人夏壽,監索班錢,被軍騰詈,而政事不可知乎?接物則奴顔婢膝,時人有『丫頭』之稱;問事則依違兩可,羣下有『木偶』之誚。理刑副千户西門慶:本係市井棍徒,夤緣陞職,濫冒武功,菽麥不知,一丁不識。縱妻妾嬉遊街巷,而帷薄為之不清;携樂婦而酣飲市樓,官箴為之有玷。至於包養韓氏之婦,恣其歡淫,而行檢不修;受苗青夜賂之金,曲為掩飾,而贓跡顯著。此二臣者,皆貪鄙不職,久乖清議,一刻不可居任者也。伏望

聖明垂聽,

勅下該部,再加詳查。如果臣言不謬,將延齡等亟賜罷斥,則官常有賴,而裨

聖德永光矣。」
\end{myquote}

西門慶看了一遍,唬的面面相覷,默默不言。夏提刑道:「長官,似此如何計較?」西門慶道:「常言:兵來將擋,水來土掩。事到其間,道在人為。少不的你我打點禮物,早差人上東京,央及老爺那裏去。」於是夏提刑急急作辭,到家拿了二百兩銀子,兩把銀壺。西門慶這裏是金鑲玉寳石鬧粧一條,三百兩銀子。夏家差了家人夏壽,西門慶這裏是來保。將禮物打包端正,西門慶修了一封書與翟管家,兩個早僱了頭口,星夜往東京幹事去了,不題。

且表官哥兒自従墳上來家,夜間只是驚哭,不肯喫奶,但喫下奶去,就吐了。慌的李瓶兒走來告訴月娘。月娘道:「我那等説,還未到一周的孩子,且休帶他出城門去。獨漒貨他生死不依,只説:『比來今日墳上祭祖,為甚麽來?不教他娘兒兩個走走?』只像那裏攙了分兒一般,睜着眼和我兩個叫。如今卻怎麽好?」李瓶兒正沒法兒擺佈。况西門慶又是因巡按御史參本參了,和夏提刑在前邊説話,往東京打點幹事,心上不遂,家中孩子又不好。月娘使小廝叫劉婆子來看,又請小兒科太醫,開門闔户亂了一夜。劉婆看了説:「哥兒着了些驚氣入肚;又路上撞見五道將軍。不打緊,燒些紙兒,退送退送就好了。又㽞了兩服朱砂丸薬兒,用薄荷燈心湯送下去。那孩兒方纔寜貼。睡了一覺,不驚哭吐奶了,只是身上熱還未退。李瓶兒連忙拿出一兩銀子,教劉婆子備紙去。後晌帶了他老公,還和一個師婆來,在捲棚内與哥兒燒紙跳神。那西門慶早五更打發來保夏壽起身,就亂着和夏提刑往東平府胡知府那裏打聽提苗青消息去了。吳月娘聽見劉婆説孩兒路上着了驚氣,甚是抱怨如意兒,説他不用心看孩兒:「想必路上轎子裏唬了他了。不然,怎的就不好起來?」如意兒道:「我在轎子裏將被兒裹得緊緊的,又沒踮着他。娘使囬畫童兒來跟着轎子,他還好好的,我按着他睡。只進城内七八到家門首,我只覺他打了個冷戰,到家就不喫奶,哭起來了。」

按下這裏家中燒紙與孩子下神。且説來保夏壽一路趲行,只六日就趕到東京城内。到太師府内見了翟管家,將兩家禮物交割明白。翟謙看了西門慶書信,説道:「曾御史參本還未到哩,你且住兩日。如今老爺新近條陳奏了七件事在這裏,旨意還末曾下來。待行下這個本去,曾御史本到,等我對老爺説,教老爺閣中只批與他『該部知道』。我這裏差人再拿我的帖兒,吩咐兵部余尚書把他的本只不覆上來。叫你老爹只顧放心,管情一些事兒沒有。」於是把二人管待了酒飯,還歸到客店安歇,那裏等聽消息。

一日,蔡太師條陳本,聖旨准下來了。來保央府中門吏抄了個邸報,帶囬家與西門慶瞧。端的上面奏行那七件事?

\begin{myquote}[\markfont]
「崇政殿大學士吏部尚書魯國公蔡京一本:陳愚見,竭愚衷,收人才,臻實效,足財用,便民情,以隆

聖治事。

\hspace*{2em}第一曰罷科擧取士,悉由學校陞貢。

竊謂教化凌夷,風俗頽敗,皆由取士不得眞才,而教化無以仰賴。《書》曰:『天生斯民,作之君,作之師。』漢擧孝廉,唐興學校。我

國家始制考貢之法。各執偏陋,以致此輩無眞才,而民之司牧何以賴焉?今

皇上寤寐求才,宵旰圖治。治在於養賢,養賢莫如學校。今後取士,悉遵古由學校陞貢。其州縣發解禮闈,一切罷之。每歲考試上舍,則差知貢擧,亦如禮闈之式,仍立八行取士之科。八行者,謂孝友睦婣任恤忠和也。士有此者,即免試,率相補太學上舍。

\hspace*{2em}二曰罷講議財利司。竊惟

國初定制,都堂置講議財利司,蓋謂人君節浮費、惜民財也。今

陛下即位以來,不寳遠物,不勞逸民,躬行節儉以自奉。蓋天下亦無不可返之俗,亦無不可節之財。惟當事者以俗化為心,以禁令為信,不忽其初,不弛其後,治隆俗美,豐亨豫大,又何講議之為哉!悉罷。

\hspace*{2em}三曰更鹽鈔法。切惟鹽鈔乃

國家之課,以供邊備者也。今合無復遵祖宗之制鹽法者。詔雲中陝西山西三邊上納糧草,關領舊鹽鈔,易東南淮浙新鹽鈔。每鈔折派三分,舊鈔搭派七分。令商人照所派産鹽之地,下場支鹽。亦如茶法,赴官秤騐,納息,請批引,限日行鹽之處販賣。如遇過限,並行拘收,别買新引。增販者俱屬私鹽。如此則國課日增而邊儲不乏矣。

\hspace*{2em}四曰制錢法:切謂錢貨乃

國家之血脈,貴乎流通,而不可淹滯。如有扼阻淹滯不行者,則小民何以變通?而國課何以仰賴矣!自晉末鵝眼錢之後,至

國初瑣屑不堪,甚至雜以鉛鐵夾錫。邊人販於虜,因而鑄兵器,為害不小。合無一切通行禁之也。以

陛下新鑄大錢崇寜大觀通寳,一以當十,庶小民通行,物價不致于踴貴矣。

\hspace*{2em}五曰行結糶俵糴之法。

切惟官糶之法,乃賑恤之義也。近年水旱相仍,民間就食,上始下賑恤之詔。近有户部侍郎韓梠題覆

欽依,將境内所屬州縣,各立社會,行結糶俵糴之法。保之於黨,黨之於里,里之於鄉,倡之結也。每鄉編為三户。按上上、中中、下下。上户者納糧,中户者減半,下户者遞派。糧數關支,謂之俵糶。如此則斂散便民之法得以施行。而

皇上可廣不費之仁矣。惟責守令,覈切擧行,其關係蓋匪細矣。

\hspace*{2em}六曰詔天下州郡納免夫錢。切惟我

國初,寇亂未定,悉令天下軍徭丁壯,集於京師,以供運餽,以壯國勢。今

承平日久,民各安業。合頒

詔行天下州郡,每歲上納免夫錢。每名折錢三十貫,解赴京師,以資邊餉之用。庶兩得其便矣,而民力少蘇矣!

\hspace*{2em}七曰置提擧御前人舡所。切惟

陛下自即位以來,無聲色犬馬之奉。所尚花石,皆山林間物,乃人之所棄者。但有司奉行之過,因而致擾,有傷

聖治。

陛下節其浮濫,仍請作御前提擧人舡所。凡有用悉出内帑,差官取之。庶無擾于州郡。伏乞

聖裁。」

奉

聖旨:「卿言深切時艱,朕心嘉悦,足見忠猷。都依擬行,該部知道。」
\end{myquote}

來保抄了邸報,等的翟管家寫了囬書,與了五兩盤纏,與夏壽取路囬山東清河縣來。有日到家中,西門慶正在家躭心不下。那夏提刑一日一遍來問信。聽見來保二人到了,叫至後邊問他端的。來保對西門慶悉把上項事情訴説一遍:「府中見翟爹,看了爹的信,便説此事不打緊,『敎你爹放心。現今巡按也滿了,另點新巡按下來了。况他的參本還未到。等他本上時,等我對老爺説了,隨他本上參的怎麽重,只批了「該部知道」。老爺這裏再㧱帖兒吩咐兵部余尚書,只把他的本立了案,不覆上去,隨他有撥天關本事,也無妨。』」西門慶聽了。方纔心中放下。因問:「他的本怎倒還不到?」來保道:「俺們一去時,晝夜馬上行去,只五日就趕到京中,可知在他頭裏。俺們囬來,見路上一簇響鈴驛馬過,背着黄包袱,插着兩根雉尾,兩面牙旗,怕不就是巡按衙門進送實封纔到了。」西門慶道:「倒得他的本上的遲,事情就停當了。我只怕去遲了。」來保道:「爹放心,管情沒事。小的不但幹了這件好事,又打聽的兩樁好事來,報爹知道。」西門慶問道:「端的何事?」來保道:「太師老爺新近條陳了七件事,旨意已是准行。如今老爺親家户部侍郎韓爺題准事例:在陝西等三邊,開引種鹽;各府州郡縣設立義倉,官糶糧米。令民間上上之户赴倉上米,討倉鈔,派給鹽引支鹽。舊倉鈔七分,新倉鈔三分。咱舊時和喬親家爹高陽關上納的那三萬糧倉鈔,派三萬鹽引,户部坐派。倒好趁着蔡老爹巡鹽,下場支種了罷,倒有好些利息。」西門慶聽言,問道:「眞個有此事?」來保道:「爹不信,小的抄了個邸報在此。」向書篋中取出來,與西門慶觀看。因見上面許多字樣,前邊叫了陳經濟來唸與他聽。陳經濟唸到中間,只要結住了,——還有幾個眼生字不認的。旋叫了書童兒來唸。那書童倒還是門子出身,蕩蕩如流水不差,直唸到底。端的上面奏着那七件事,云云。西門慶聽了喜甚,又看了翟管家書信,已知禮物交得明白,蔡狀元見朝,已點了兩淮巡鹽,心中不勝歡喜。一面打發夏壽回家,「報與你老爹知道。」一面賞了來保五兩銀子,兩瓶酒,一方肉,囬房歇息,不在話下。正是:樹大招風風損樹,人為名高名丧身。有詩為證:

\begin{myquote}
得失榮枯命裏該,皆因年月日時裁。

胸中有志終湏至,囊内無財莫論才。
\end{myquote}

畢竟不知後來如何,且聽下回分解。

