\includepdf[pages={181,182},fitpaper=false]{tst.pdf}
\chapter*{第九十一囬 \\孟玉樓愛嫁李衙内 李衙内怒打玉簪兒}
\addcontentsline{toc}{chapter}{第九十一囬 孟玉樓愛嫁李衙内 李衙内怒打玉簪兒}
\markboth{{\titlename}卷之十}{第九十一囬 孟玉樓愛嫁李衙内 李衙内怒打玉簪兒}


\begin{myquote}
百歲光陰疾似飛,其間花景不多時。

秋凝白露蛩蟲泣,春老黄昏杜宇啼。

富貴繁華身上孽,功名事跡目中魑。

一塲春夢由人做,自有青天報不欺。
\end{myquote}

話說一日陳經濟聽見薛嫂兒說,西門慶家孫雪娥,被來旺因姦抵盗財物,拐出在外,事發本縣官賣,被守備府裏買了,朝夕受春梅打罵。這陳經濟乘着這個因由,使薛嫂兒徃西門慶家對月娘說:只是經濟風裏言風裏語,在外聲言發話,說不要大姐,寫了狀子,巡撫巡按處要告月娘,說西門慶在日,收着他父親寄放許多金銀箱籠細軟之物。這月娘一來因孫雪娥被來旺兒盗財拐去,二者又是來安兒小廝走了,三者家人來興媳婦惠秀又死了,剛打發出去,家中正七事八事,聽見薛嫂兒來說此話,唬的慌了手脚,連忙僱轎子,打發大姐家去。但是大姐牀奩箱厨陪嫁之物,教玳安僱人都擡送到陳經濟家。經濟說:「這是他隨身嫁我的牀帳粧奩,還有我家寄放的細軟金銀箱籠,須索還我!」薛嫂道:「你大丈母說來,當初丈人在時,止收下這個牀奩嫁粧,並沒見你的别的箱籠。」經濟又要使女元宵兒。薛嫂兒和玳安兒來對月娘說,月娘不肯把元宵與他,說:「這丫頭是李嬌兒房中使女,如今沒人看哥兒,留着早晚看哥兒哩。」把中秋兒打發將來,說原是買了扶侍大姐的。這經濟又不要中秋兒,兩頭來囘只敎薛嫂兒走。他娘張氏便向玳安說:「哥哥,你到家頂上你大娘:你家姐兒們多,豈希罕這個使女看守。既是與了大姐房裏好一向,你姐夫已是收用過他了,你大娘只顧留怎的?」玳安一面到家,把此話對月娘說了。月娘無言可對,只得把元宵兒打發將來。經濟這裏收下,滿心歡喜,說道:「可怎的也打我這條道兒來?正是:饒你奸似鬼,也喫我洗脚水!」

按下一頭,卻表一處。單說李知縣兒子李衙内,自従清明郊外那日在杏花莊酒樓,看見吳月娘孟玉樓,兩口一般打扮,生的俱有姿色,使小張閒打聽,囘報俱是西門慶妻小。衙内有心愛孟玉樓,見生的長挑身材,瓜子面皮,面上稀稀有幾點白麻子兒,模樣兒風流俏麗。原來衙内丧偶,鰥居已久,一向着媒婦各䖏求親,都不遂意。及見玉樓,縱有懷心,無門可入,未知嫁與不嫁,従違如何。不期雪娥緣事在官,已知是西門慶家出來的,周旋委曲,在伊父案前,將各犯用刑硏審,追問贓物數目,冀其來領。月娘害怕,又不使人見官。衙内失望,因此纔將贓物入官,雪娥官賣。至是衙内謀之於廊吏何不韋,徑使官媒婆陶媽媽來西門慶家訪求親事,許說成此門親事,免縣中打卯,還賞銀五兩。這陶媽媽聽了,喜歡的疾走如飛,一直到於西門慶門首。來昭正在門首立,只見陶媽媽向前,道了萬福,說道:「動問管家哥一聲,此是西門老爹家?」那來昭道:「你是那裏來的?這是西門老爹家,老爹下世了,來有甚話說?」陶媽媽道:「累及管家進去禀聲,我是本縣官媒人,名喚陶媽媽,奉衙内小老爹鈞語吩咐,說咱宅内有位奶奶要嫁人,敬來說頭親事。」那來昭喝道:「你這婆子,好不近理!我家老爹沒了一年有餘,止有兩位奶奶守寡,並不嫁人。常言:疾風暴雨,不入寡婦之門。你這媒婆,有要沒緊,走來瞎撞甚親事?還不走快着,惹的後邊奶奶知道,一頓好打。」那陶媽媽笑說:「管家哥,常言:官差吏差,來人不差。小老爹不使我,我來做甚麽?嫁不嫁,起動進去禀聲,我好囬話去。」這來昭道:「也罷,與人方便,自己方便。你少待片時,等我進去。兩位奶奶,一位奶奶有哥兒,一位奶奶無哥兒,不知是哪一位奶奶要嫁人?」陶媽媽道:「衙内小老爹說,是清明那日郊外曾看見來,是面上有幾點白麻子兒的那位奶奶。」

這來昭聽了,走到後邊,如此這般,告月娘說:「縣中使了個官媒人在外面。」倒把月娘喫了一驚,說:「我家裏並沒半個字兒迸出,外邊人怎得曉的?」來昭道:「曾在郊外清明那日見來,說臉上有幾個白麻子兒的那位奶奶。」月娘便道:「莫不孟三姐也臘月裏蘿蔔動了心,忽剌八要徃前進嫁人?」正是:世間海水知深淺,惟有人心難忖量!一面走到玉樓房中,坐下便問:「孟三姐,奴有件事兒來問你。外邊有個保山媒人,說是縣中小衙内,清明那日曾見你一面,說你要徃前進。端的有此話麽?」

看官聽說:當時沒巧不成話,自古姻緣着線牽。那日郊外,孟玉樓看見衙内生的一表人物,風流博浪,兩家年甲多相彷彿,又會走馬撚弓弄箭,彼此兩情四目都有意,已在不言之表,但未知有妻子無妻子。口中不言,心内暗度:「况男子漢已死,奴身邊又無所出,雖故大娘有孩兒,到明日長大了,各肉兒各疼,歸他娘去了,閃的我樹倒無陰,竹籃兒打水。」又見月娘自有了孝哥兒,心腸兒都改變,不似徃時,「我不如徃前進一步,尋上個葉落歸根之處,還只顧儍儍的守些甚麽?到沒的躭閣了奴的青春,辜負了奴的年少!」正在思慕之間,不想月娘進來說此話,正是清明郊外看見的那個人!心中又是歡喜,又是羞愧,口裏雖說:「大娘休聽人胡說,奴並沒此話。」不覺把臉來飛紅了。正是:含羞對衆慵開口,理鬢無言只揾頭。月娘說:「既是各人心裏事,奴也管不的許多。」一面叫來昭:「你請那保山來。」來昭來門首,喚陶媽媽進到後邊。

月娘在上房明間内,正面供養着西門慶靈牀。那陶媽媽施畢禮數,坐下,小丫鬟綉春倒茶喫了,月娘便問:「保山來有甚事?」那陶媽媽便道:「小媳婦無事不登三寳殿,奉本縣正宅衙内吩咐,敬來說咱宅上有一位奶奶要嫁人,講說親事。」月娘道:「是俺家這位娘子嫁人?又沒曾傳出去,你家衙内怎得知道?」陶媽媽道:「俺家衙内說來,清明那日,在郊外親見這位娘子,生的長挑身材,瓜子面皮,臉上有稀稀幾個白麻子兒的,便是這位奶奶。」月娘聽了,「不消說就是孟三姐了!」於是領陶媽媽到玉樓房中,明間内坐下。等夠多時,玉樓梳洗打扮出來。那陶媽媽道了萬福,說道:「就是此位奶奶!果然語不虚傳,人材出衆,蓋世無雙,堪可與俺衙内老爹做得個正頭娘子。你看,従頭看到底,風流實無比;従頭看到脚,風流徃下跑!」玉樓笑道:「媽媽休得亂説。且說你衙内今年多大年紀,原娶過妻小來沒有?房中有人也無?姓甚名誰?鄉貫何䖏?地裏何方?有官身無官身?従實說來,休要搗謊。」陶媽媽道:「天麽,天麽!小媳婦是本縣官媒人,不比外邊媒人快說謊。我有一句說一句,並無虚假。俺知縣老爹,年五十多歲,止生了衙内老爹一人,今年屬馬的,三十一歲,正月二十三日辰時建生,現做國子監上舍,不久就是擧人進士;有滿腹文章,弓馬熟嫻,諸子百家,無不通曉。沒有大娘子二年光景,房内止有一個従嫁使女答應,又不出材兒。要尋個娘子當家,一地裏又尋不着門當戶對的,敬來宅上說此親事。若成,免小媳婦縣中打卯,還重賞銀五兩在外。若是咱宅上肯做這門親事,老爹說來:門面差徭,坟塋地土錢糧,一例盡行蠲免。有人欺負,指名說來,㧱到縣裏任意拶打。」玉樓道:「你衙内有兒女沒有?原籍那裏人氏?誠恐一時任滿,千山萬水帶去,奴親都在此䖏,莫不也要同他去!」陶媽媽道:「俺衙内老爹身邊兒花女花沒有,好不單徑。原籍是咱北京眞定府棗強縣人氏,過了黄河,不上六七百里。他家中田連阡陌,騾馬成羣,人丁無數;走馬牌樓,都是撫按明文,聖旨在上,好不赫耀驚人!如今娶娘子到家做了正房——無正房,入門為正,過後他得了官,娘子便是五花官誥,坐七香車,為命婦夫人,有何不可!」這孟玉樓被陶媽媽一席話,說得千肯萬肯,一面喚蘭香:「放桌兒,看茶食點心與保山喫。」因說:「保山,你休怪我叮嚀盤問。你這媒人們說謊的極多,初時說的天花亂墜,地湧金蓮,及到其間,並無一物,奴也喫人哄怕了!」陶媽媽道:「好奶奶,只要一個比一個,清自清,渾自渾。歹的帶累了好的!小媳婦並不搗謊,只依本分說媒,成就人家好事。奶奶肯了,討個婚帖兒與我,好囘小老爹話去。」玉樓取了一條大紅緞子,使玳安教鋪子裏傅夥計寫了生時八字。吳月娘便說:「你當初原是薛嫂兒說的媒,如今還使小廝叫將薛嫂兒來,兩個同拏了帖兒去說此親事,纔是理。」不多時,使玳安兒叫薛嫂兒來,見陶媽媽,道了萬福。當行見當行,㧱着帖兒出離西門慶家門,徃縣中囘衙内話去。一個是這裏氷人,一個是那頭保山,兩張口四十八個牙,這一去,管取說得月裏嫦娥尋配偶,巫山神女嫁襄王。

陶媽媽在路上問薛嫂兒:「你就是這位娘子的原媒?」薛嫂道:「然者,便是。」陶媽媽問他原先嫁這裏根兒:「是何人家的女兒?嫁這裏是女兒,是再婚兒?」這薛嫂兒便一五一十,把西門慶當初従楊家娶來的話告訴一遍。因見婚帖兒上寫:「女命三十七歲,十一月二十七日子時生」,說:「只怕衙内嫌娘子年紀大些,怎了?他今纔三十一歲,倒大六歲。」薛嫂道:「咱㧱了這婚帖兒,敎個路過的先生算,看年命妨碍不妨碍。若是不對,咱瞞他幾歲兒,不算發了眼。」正走中間,也不見路過響板的先生,只見路南遠遠的一個卦肆,青布帳幔,掛着兩行大字:「子平推貴賤,鐵筆判榮枯;有人來算命,直言不容情。」帳子底下,安放一張桌席,裏面坐着個能寫快算靈先生。這兩個媒人,向前道了萬福,先生便讓坐下。薛嫂道:「有個女人命,累先生算一算。」向袖中㧱出三分命金來,說:「不當輕視,先生權且收了,路過不曾多帶錢來。」先生道:「此是合婚的意思?請說八字。」陶媽媽遞與他婚帖,看上面有八字生日年紀,先生道:「此是合婚。」一面掐指尋紋,把算子搖了一搖,開言說道:「這位女命,今年三十七歲了,十一月廿七日子時生,甲子年,丙子月,辛卯日,庚子時,理取印綬之格。女命逆行,現在壬申運中。丙合辛生,徃後享有威權,執掌正堂夫人之命。四柱中夫星多,雖然財命,益夫發福,受夫寵愛,不久定見妨尅。果然見過了不曾?」薛嫂道:「已尅過兩位夫主了。」先生道:「若見過,後來得個屬馬的。」薛嫂兒道:「他徃後有子沒有?」先生道:「子早哩,命中直到四十一歲纔有一子送老。一生好造化,富貴榮華眞無比。」取筆批下命詞八句:

\begin{myquote}
「花盛菓收奇異時,欣遇良君立鳳池;

嬌姿不失江梅態,三揭紅羅兩畫眉。

㩦手相邀登玉殿,含羞獨步捧金巵。

會看馬首昇騰日,脱卻寅皮任意移。」
\end{myquote}

薛嫂問道:「先生,如何是『會看馬首昇騰日,脱卻寅皮任意移?』這兩句俺們不懂,起動先生,講說講說。」先生道:「馬首者,這位娘子如今嫁個屬馬的夫主,方是貴星,享受榮華。寅皮是尅過的夫主,是屬虎的,雖故受寵愛,只是偏房。徃後一路功名,直到六十八歲,有一子,壽終,夫妻偕老。」兩個媒人收了命狀,說道:「如今嫁的倒果是屬馬的,只怕大了好幾歲,配不來,求先生改少兩歲纔好。」先生道:「既要改,就改做丁卯三十四歲罷。」薛嫂問先生:「三十四歲與屬馬的也合的着?」先生道:「丁火庚金,金逢火煉,定成大器,正好。」當下改做三十四歲。兩個拜辭了先生,出離卦肆,逕到縣中。衙内正坐,門子報入。良久喚進,陶薛二媒人跪下磕頭。衙内便問:「那個婦人是那裏的?」陶媽媽道:「是那邊媒人。」因把親事說成且訴一遍說:「娘子人材無比的好,只爭年紀大些,小媳婦不敢擅便,隨衙内老爹尊意。討了個婚帖在此。」於是遞上去。李衙内看了,上寫着:「三十四歲,十一月廿七日子時生。」說道:「就大三兩歲也罷。」薛嫂兒插口道:「老爹見的多,自古妻大兩,黄金長;妻大三,黄金山。這位娘子人材出衆,性格溫柔,諸子百家,當家理紀,自不必說。」衙内道:「既然好,已是見過,不必再相。命陰陽擇吉日良時,行茶過禮去就是了。」兩個媒人禀說:「小媳婦幾時來伺候?」衙内道:「事不可稽遲,你兩個明日來討話,徃他家說。」吩咐左右:「每人且賞與他一兩銀子做脚步錢。」兩個媒人歡喜出門,不在話下。

這李衙内見親事已成,喜不自勝,即喚廊吏何不韋來,兩個商議,對父親李知縣說了。令陰陽生擇定四月初八日行禮,十五吉日良時,准娶婦人過門。就兌出銀子來,委托何不韋小張閒,買辨茶紅酒禮,不必細說。兩個媒人次日討了日期,徃西門慶家囘月娘孟玉樓話。正是:姻緣本是前生定,曾向藍田種玉來。

四月初八日,縣中備辦十六盤羹菓茶餅、一副金絲冠兒、一副金頭面、一條瑪瑙帶、一副玎璫七事、金鐲銀釧之類,兩件大紅宫錦袍兒、四套粧花衣服、三十兩禮錢,其餘布絹棉花,共約二十餘擡。兩個媒人跟隨,廊吏何不韋押擔,到西門慶家下了茶。十五日,縣中撥了許多快手閒漢來,搬擡孟玉樓牀帳嫁妝箱籠。月娘看着,但是他房中之物,盡數都敎他带去。原舊西門慶在日,把他一張八步彩漆牀陪了大姐,月娘就把潘金蓮房那張螺鈿牀賠了他。玉樓敎蘭香跟他過去,留下小鸞與月娘看哥兒。月娘不肯,說:「你房中丫頭,我怎好留下你的?左右哥兒有中秋兒綉春和奶子,也夠了。」玉樓止留下一對銀囘囘壺,與哥兒耍子,做一念兒,其餘都帶過去了。到晚夕,一頂四人大轎,四對紅紗鐵絡燈籠,八個皂隸跟隨,來娶孟玉樓。玉樓戴着金梁冠兒,插着滿頭珠翠、胡珠環子,身穿大紅通袖袍兒,繫金鑲瑪瑙帶、玎璫七事,下着柳黄百花裙,先辭拜西門慶靈位,然後拜月娘。月娘說道:「孟三姐,你好狠也!你去了,撇的奴孤另另獨自一個,和誰做伴兒?」兩個㩦手哭了一囘。然後家中大小都送出大門,媒人替他帶上紅羅銷金蓋袱,抱着金寳瓶。月娘守寡出不的門,請大姨送親,穿大紅粧花袍兒,翠藍裙,滿頭珠翠,坐大轎,送到知縣衙裏來。

滿街上人看見說:「此是西門大官人第三娘子,嫁了知縣相公兒子衙内,今日吉日良時,娶過門。」也有説好,也有説歹的。說好者道:「當初西門大官人怎的為人做人,今日死了,止是他大娘子守寡正大,有兒子,房中攪不過這許多人來,都敎各人前進來,甚有張主!」有那說歹的,街談巷議,指戳說道:「此是西門慶家第三個小老婆,如今嫁人了!當初這廝在日,專一違天害理,貪財好色,姦騙人家妻子。今日死了,老婆帶的東西,嫁人的嫁人,拐帶的拐帶,養漢的養漢,做賊的做賊,都野鷄毛兒零撏了。常言三十年遠報,而今眼下就報了!」旁人都如此發這等暢快言語。

孟大姨送親到縣衙内,舖陳牀帳停當,留坐酒席來家。李衙内將薛嫂兒陶媽媽叫到跟前,每人五兩銀子,一段花紅利市,打發出門。至晚兩個成親,極盡魚水之歡,曲盡于飛之楽。到次日,吳月娘這邊送茶完飯。楊姑娘已死,孟大妗子、二妗子、孟大姨,都送茶到縣中。衙内這邊下囬書,請衆親戚女眷做三日。扎彩山,喫筵席,都是三院楽人妓女動鼓楽,扮演戲文。吳月娘那日亦滿頭珠翠,身穿大紅通袖袍兒、百花裙,繫蒙金帶,坐大轎,來衙中做三日赴席。在後廳喫酒,知縣奶奶出來陪待。月娘囘家,因見席上花攢錦簇,歸到家中,進入後邊院落,見靜悄悄,無個人接應,想起當初有西門慶在日,姊妹們那樣熱鬧,徃人家赴席來家,都來相見說話,一條板櫈姊妹們都坐不了。如今並無一個兒了!一面撲着西門慶靈牀兒,不覺一陣傷心,放聲大哭。哭了一囬,被丫鬟小玉勸止,住了眼淚。正是:平生心事無人識,只有穿窻皓月知。這裏月娘憂悶不題。

卻說李衙内和玉樓兩個,女貌郎才,如魚似水,正合着油瓶蓋上,每日燕爾新婚,在房中廝守,一步不離。端詳玉樓容貌,觀之不足,看之有餘,越看越愛。又見带了兩個従嫁丫鬟,一個蘭香,年十八歲,會彈唱;一個小鸞,年十五歲,俱有顏色,心中歡喜沒入腳處。有詩為證:

\begin{myquote}
堪誇女貌與郎才,天合姻緣禮所該。

十二巫山雲雨會,兩情願保百年偕。
\end{myquote}

原來衙内房中,先頭娘子丢下一個大丫頭,約三十年紀,名喚玉簪兒。專一搽胭抹粉,作怪成精。頭上打着盤頭揸髻,用手帕苫蓋。周圍勒銷金箍兒,假充作䯼髻,又插着些銅釵蠟片、敗葉殘花;耳朶上帶雙甜瓜墜子;身上穿一套前露臀後露ど怪綠喬紅的裙襖,在人前好似披荷葉老鼠;脚上穿着雙裏外油劉海笑撥舡樣四個眼的剪絨鞋,約尺二長。臉上搽着一面鉛粉,東一塊白,西一塊紅,好似青冬瓜一般。在人跟前輕聲浪顙,做勢㧱班。衙内未娶玉樓來時,他便逐日炖羹炖飯,殷勤扶侍,不說強說,不笑強笑,何等精神。自従娶過玉樓來,見衙内日逐和他牀上睡,如膠似漆般打熱,把他不去瞅睬,這丫頭就有些使性兒起來。一日,衙内在書房中看書,這玉簪兒在厨下炖熱了一盞好菓仁泡茶,雙手用盤兒托來,到書房裏面,笑嘻嘻掀開簾兒,送與衙内。不想衙内看了一囬書,搭伏定書桌,就睡着了。這玉簪兒叫道:「爹,誰似奴疼你,炖了這盞好茶兒與你喫!你家那新娶的娘子,還在被窝裏睡得好覺兒,怎不敎他那小大姐送盞茶來與你喫?」因見衙内打盹,在跟前只顧叫不應。説道:「老花子,你黑夜做夜作,使乏了也怎的,大白日打盹磕睡?起來喫茶!」呌衙内醒了,看見是他,喝道:「怪硶奴才!把茶放下,與我過一邊裏去。」這玉簪兒便臉羞紅了,使性子把茶丢在桌上。出來說道:「好不識人敬重!奴好意用心,大清早晨送盞茶兒來你喫,倒吆喝罵我。常言醜是家中寳,可喜惹煩惱!我醜,你當初瞎了眼?誰敎你要我來使的,直我的那大精𣭈!」被衙内聽見,赶上儘力踢了兩靴脚。

這玉簪兒走出,登時把那付奴臉膀的有房梁高,也不搽臉了,也不炖茶造飯了。趕着玉樓也不叫娘,只你也我也的,無人處,一個屁股就同在玉樓牀上坐。玉樓亦不去理他。他背地又壓伏蘭香小鸞說:「你休趕着我叫姐,只叫姨娘。我與你娘係大小之分。」又說:「你只背地叫罷,休對着你爹呌。你每日跟逐我行,用心做活,你若不聽指敎,老娘㧱煤鍬子請你!」後來幾次見衙内不理他,他就撒懶起來,睡到日頭半天還不起來。飯兒也不做,地兒也不掃。玉樓吩咐蘭香小鸞:「你休靠玉簪兒了,你二人自去厨下做飯,打發你爹喫罷。」他又氣不憤,使性謗氣摔家打活,在厨房内打小鸞,罵蘭香:「賊小奴才,小淫婦兒!碓磨也有個先來後到。先有你娘來,先有我來?都你娘兒們占了罷,不獻這個勤兒也罷了!當原先俺死了那個娘,也沒曾失口叫我聲玉簪兒,你進門幾日,就題名道姓叫我?我是你手裏使的人也怎的?你未來時,我和俺爹同牀共枕,那一日不睡到齋時纔起來。和我兩個如糖拌蜜,如蜜攪酥油一般打熱。房中事,那些兒不打我手裏過?自従你來了,把我蜜罐兒也打碎了,把我姻緣也拆開了,一攆攆到我明間,冷清清支板櫈打官鋪,再不得嘗着俺爹那件東西兒甚麽滋味兒!我這氣苦,正也沒處聲訴。你當初在西門慶家,也曾做第三個小老婆來,你小名兒叫玉樓,敢說老娘不知道?你來在俺家,你識我見,大家膿着些罷了,會那等大廝不道喬張致,呼張喚李!誰是你買到的,屬你管轄不成?」那玉樓在房中聽見,氣的發昏,連套手戰,只是不敢聲言對衙内說。

一日熱天,也是合當有事。晚夕衙内吩咐他厨下熱水,拏浴盆來房中,要和玉樓洗澡。玉樓便說:「你敎蘭香熱水罷,休要使他!」衙内不従,說道:「我偏使他!休要慣了這奴才。」玉簪兒見衙内要水,和婦人洗澡,共浴蘭湯,效魚水之歡,偕于飛之楽,心中正沒好氣,拏浴盆進房,徃地下只一墩,用大鍋燒上一鍋滚水,口内喃喃呐呐說道:「也沒見這浪淫婦,刁鑽古怪,禁害老娘!無過也只是個浪精𣭈,沒三日不拏水洗。像我與俺主子睡,成月也不見點水兒,也不見展汚了甚麽佛眼兒。偏這淫婦,會兩番三次刁蹬老娘!」直罵出房門來。玉樓聽見,也不言語。衙内聽了此言,心中大怒,澡也洗不成,精脊梁靸着鞋,向牀頭取拐子,就要走出來,婦人攔阻住,說道:「隨他罵罷,你好惹氣?只怕熱身子出去,風篩着你,倒値了多的。」衙内那裏按納得住,説道:「你休管他。這奴才無禮!」向前一把手採住他頭髮,拖踏在地下,輪起拐子,雨點打將下來。饒玉樓在旁勸着,也打了二三十下在身。打的這丫頭急了,跪在地下告說:「爹,你休打我,我有句話兒和你說。」衙内罵:「賊奴才,你說!」有〔山坡羊〕為證:

\begin{myquote}
「告爹行,停嗔息怒,你細細兒聽奴分訴。當初你將八兩銀子財禮錢,娶我當家理紀,管着些油鹽醬醋。你喫了飯喫茶,只在我手裏抹布。沒了俺娘,你也把我陞為個署府,咱兩個同鋪同牀何等的頑耍,奴按家伏業,纔把這活來做。誰承望你哄我說不娶了,今日又起這個毛心兒裏來呵,把徃日恩情弄的半星兒也無!呌了聲爹,你忒心毒!我如今不在你家了,情願嫁上個姐夫!」
\end{myquote}

衙内聽了,亦發惱怒起來,又狠了幾下。玉樓勸道:「他旣要出去,你不消打,倒沒得氣了你。」衙内隨令伴當,即時叫將媒人陶媽媽來,把玉簪兒領出去,變賣銀子來交,不在話下。正是:蚊虫遭扇打,只為嘴傷人。有詩為證:

\begin{myquote}
百禽啼後人皆喜,惟有鴉鳴事若何?

見者多嫌聞者唾,只為人前口嘴多。
\end{myquote}

畢竟未知後來何如,且聽下囘分解。

