\includepdf[pages={163,164},fitpaper=false]{tst.pdf}
\chapter*{第八十二囬 \\潘金蓮月夜偸期 陳經濟畫樓雙羙}
\addcontentsline{toc}{chapter}{第八十二囬 潘金蓮月夜偸期 陳經濟畫樓雙羙}
\markboth{{\titlename}卷之九}{第八十二囬 潘金蓮月夜偸期 陳經濟畫樓雙羙}


\begin{myquote}
記得書齋乍會時,雲踪雨跡少人知。

晚來鸞鳳棲雙枕,剔盡銀燈半吐輝。

思往事夢魂迷,今宵喜得效于飛。

顛鸞倒鳳無窮楽,従此雙雙永不離。
\end{myquote}

話說潘金蓮與陳經濟,自從在廂房裏得手之後,兩個人嚐着甜頭兒,日逐白日偸寒,黄昏送煖,或倚肩嘲笑,或並坐調情,掐打揪撏,通無忌憚。或有人跟前不得說話,將心事寫成搓在紙條兒内,丢在地下,你有話傳與我,我有話傳與你。一日,四月天氣,潘金蓮將自己袖的一方銀絲汗巾兒,裹着一個玉色紗挑線香袋兒,裏面裝安息香、排草、玫瑰花瓣兒,並一縷頭髮,又着些松柏兒,一面挑着「松柏長青」,一面是「人面如花」八字,封的停當,要與經濟。不想經濟不在廂房内,遂打窗眼内投進去。後經濟開門進入房中,看見彌封甚厚,打開,卻是汗巾香袋兒,紙上寫一詞,名〔寄生草〕:

\begin{myquote}
「將奴這銀絲帕,並香囊寄與他。當中結下青絲髮。松栢兒要你常牽掛,淚珠兒滴寫相思話:夜深燈照的奴影兒孤,休負了夜深潛等茶䕷架!」
\end{myquote}

這經濟見詞上許他在荼䕷架下等候私會佳期,隨即封了一柄金湘妃竹扇兒,亦寫一詞在上面答他,袖入花園内。不想月娘正在金蓮房中坐着,這經濟三不知,恰進角門就叫:「可意人在家不在?」這金蓮聽見是他語音,恐怕月娘聽見决撒了,連忙走出來,掀起簾子看見是他,佯做擺手兒,說:「我道是誰來,原來是陳姐夫來尋大姐。大姐剛纔在這裏,和他們往花園亭子上摘花兒去了。」這經濟見有月娘在房裏,就把物事暗暗遞與婦人袖了,他就出去了。月娘便問:「陳姐夫來做甚麽?」金蓮道:「他來尋大姐,我囘他往花園中去了。」以此瞞過月娘。不久,月娘起身囘後邊去了。金蓮向袖中取出物事,拆開,卻是湘妃竹白紗扇兒一把,上畫一種青蒲,半溪流水。有〔水僊子〕一首為證:

\begin{myquote}
「紫竹白紗甚逍遙,綠□青蒲巧製成。金鉸銀綫十分妙。妙人兒堪用着,遮炎天少把風招。有人䖏常常袖着,無人處慢慢輕搖。休教那俗人兒偸了!」
\end{myquote}

婦人一見其詞,到於晚夕月上時,早把春梅秋菊兩個丫頭,打發些酒與他喫,關在那邊炕屋睡,然後他便在房中,綠窗半啟,絳燭高燒,收拾牀鋪衾枕,薰香澡牝,獨立木香棚下,專等經濟今晚來赴佳期。卻說西門大姐那日被月娘請去後邊,聽王姑子宣卷去了,止有元宵兒在屋裏,經濟體己與了他一方手帕,安撫他看守房中:「我往你五娘那邊,請我下棋去。等大姑娘進來,你快呌我去。」那元宵兒應諾了。這經濟得手,走來花園中。那花篩月影,參差掩映。走在荼䕷架下,遠遠望着。見婦人摘去冠兒,半挽烏雲,上着藕絲衫,下着翠紋裙,脚襯凌波羅襪,従木香棚下來。這經濟猛然従荼䕷架下突出,雙手把婦人抱住,把婦人唬了一跳,說:「呸!小短命!猛可鑽出來,唬了我一跳。早是我,你摟便將就罷了,若是别人,你也恁大膽摟起來?」經濟喫的半酣兒,笑道:「早是摟了你,就錯摟了紅娘,也是沒奈何!」兩個於是相摟相抱,携手進入房中。房中熒煌煌掌着燈燭,桌上設着酒餚。一面頂了角門,並肩而坐飲酒。婦人便問:「你來,大姐知不知?」經濟道:「大姐後邊聽宣卷去了。我安撫下元宵兒,有事來這裏叫我,只說在這裏下棋哩。」說畢,兩個懽笑做一處。飲酒多時,常言風流茶說合,酒是色媒人,不覺竹葉穿心,桃花上臉,一個嘴兒相親,一個腮兒廝搵,罩了燈上牀交接。婦人摟抱經濟,經濟亦揣摸着婦人。婦人唱〔河西六娘子〕:

\begin{myquote}
「入門來將奴摟抱在懷。奴把錦被兒伸開。俏冤家頑的十分怪。嗏,將奴脚兒擡,脚兒擡!揉亂了烏雲䯼髻兒歪。」
\end{myquote}

經濟亦占囘前詞一首:

\begin{myquote}
「兩意相投情掛牽。休要閃的人孤眠。山盟海誓說千遍:淺情,上放着天,放着天!你又青春咱少年。」
\end{myquote}

兩人雲雨纔畢,只聽得元宵呌門,說:「大姑娘進房中來了。」這經濟慌的穿衣出門去了。正是:狂蜂浪蝶有時見,飛入梨花無處尋。

原來潘金蓮那邊,三間樓上,中間供養佛像,兩邊稍間堆放生薬香料。兩個自此以後,情沾肺腑,意密如膠,無日不相會做一䖏。一日,也是合當有事。潘金蓮早晨梳粧打扮,走來樓上觀音菩薩前燒香。不想陳經濟正拿鑰匙上樓,開庫房門拿薬材香料,撞遇在一處。這婦人且不燒香,見樓上無人,兩個摟抱着親嘴咂舌。一個呌「親親五娘」,一個呼「心肝性命」,說:「趁無人,咱在這裏幹了罷!」一面解褪衣褲,就在一張春櫈上,雙鳧飛肩,靈根半入,不勝綢繆。有生薬名〔水僊子〕為證:

\begin{myquote}
當歸半夏紫紅石,可意檳榔招做女婿。浪蕩根插入蓖麻内。母丁香左右偎,大麻花一陣昏迷。白水銀撲簇簇下,紅娘子心内喜。快活殺兩片陳皮!
\end{myquote}

當初没巧不成話,兩個正幹得好,不防春梅正上樓來拿盒子取茶葉看見,兩個凑手脚不迭,都喫了一驚。春梅恐怕羞了他,連忙倒退囬身子,走下胡梯。慌的經濟兜小衣不迭,婦人正穿裙子,婦人便呌春梅:「我的好姐姐,你上來,我和你說話。」那春梅於是走上樓來。金蓮道:「我的好姐姐,你姐夫不是別人,我今教你知道了罷:俺兩個情孚意合,拆散不開。你千萬休對人說,只放在你心裏!」春梅便說:「好娘,說那裏話!奴伏侍娘這幾年,豈不知娘心腹,肯對人說!」婦人道:「你若肯遮蓋俺們,趁你姐夫在這裏,你也過來和你姐夫睡一睡,我方信你。你若不肯,只是不可憐見俺們了!」那春梅把臉羞的一紅一白,只得依他,卸下湘裙,解開褌帶,仰在櫈上,儘着這小夥兒受用。有這等事?正是明珠兩顆皆無價,可奈檀郎盡得鑽。有〔紅繡鞋〕為證:

\begin{myquote}
假認做女婿親厚,往來和丈母歪偸!人情裏包藏鬼胡油:明講做兒女禮,暗結下燕鶯儔。他兩個現今有。
\end{myquote}

當下經濟耍了春梅,拿茶葉出去了。潘金蓮便與春梅打成一家,與這小夥兒暗約偸期,非止一日,只背着秋菊。婦人偏聽春梅說話,衣服首飾,揀心愛者與之,託為心腹。

六月初一日,金蓮娘潘姥姥老病沒了,有人來說。吴月娘買一張插桌,三牲冥紙,教金蓮坐轎子,往門外探丧祭祀。去了一遭囘來。

到次日,卻是六月初三日,金蓮起來的早,在月娘房裏坐着說了半日話,出來走在大廳院子裏牆根下,急了溺尿。正撩起裙子,蹲踞溺尿。原來西門慶死了,没人客來往,等閒大廳儀門只是關閉不開。經濟在東廂房住,纔起來,忽聽見有人在牆根石榴花樹下溺的尿刷刷的響,悄悄向窗眼裏張看。卻不想是他。便道:「是那個撒野,在這裏溺尿?撩起衣服,看濺濕了裙子了!」這婦人連忙繫上裙子,走到窗下問道:「原來你在屋裏。這咱纔起來,好自在!大姐没在房裏麽?」經濟道:「在後邊幾時出來?昨夜三更纔睡,大娘後邊拉住我宣《紅羅寳卷》與他聽,坐到那早晚,險些兒沒把腰累㿚瘑了!今日白爬不起來。」金蓮道:「賊牢成的,就休搗謊哄我!昨日我不在家,你幾時在上房内聽宣卷來?丫鬟說你昨日在孟三兒屋裏喫飯來!」經濟道:「早是大姐看着,俺們都在上房内,幾時在他屋裏去來!」說着,這小夥兒站在炕上,把那話弄的硬硬的,直豎的一條棍,隔窗眼裏舒過來。婦人一見,笑的了不的,駡道:「怪賊牢拉的短命!猛可舒出你老子頭來,唬了我一跳!你趁早好好抽進去,我好不好拿針刺與你一下子,教你忍痛哩。」經濟笑道:「你老人家這囬兒又不待見他起來!你好歹打發他個好去處,也是你一點陰騭。」婦人罵道:「好個怪牢成久慣的囚根子!」一面向腰裏摸出面青銅小鏡兒來,放在窗欞上,假做匀臉照鏡。一面用朱唇吞裹吮咂他那話,吮咂的這小郎君一點靈犀灌頂,滿腔春意融心。正是:自有内事迎郎意,殷勤愛把紫簫吹。原來婦人做作如此,若有人看見,只說他照鏡匀臉兒,不顯其事,其淫蠱顯然,通無廉恥!正咂在熱鬧處,忽聽的有人走的脚步兒響。這婦人連忙摘下鏡子,走過一邊。經濟便把那話抽囬去。卻不想是來安兒小廝走來說:「傅大郎前邊請姐夫喫飯哩。」經濟道:「教你傅大郎且喫着,我梳頭哩,就來。」來安兒囬去了。婦人便悄悄向經濟說:「晚夕你休往那裏去了,在屋裏。我使春梅呌你,好歹等我,有話和你說。」經濟道:「謹依來命。」婦人說畢,囬房去了。經濟梳洗畢,往舖中自做買賣不題。

不一時,天色晚來。那日月黑星密,天氣十分炎熱。婦人令春梅燒湯熱水,要在房中洗澡,修剪足甲。牀上收拾衾枕,趕了蚊子,放下紗帳子,小篆内炷了香。春梅便呌:「娘不知,今日是頭伏,你不要些鳳僊花染指甲?我替你尋些來。」婦人道:「你那裏尋去?」春梅道:「我直往那邊大院子裏纔有,我去拔幾根來。娘教秋菊尋下杵臼,搗下蒜。」婦人附耳低言,悄悄吩咐春梅:「你就廂房中請你姐夫晚夕來,我和他說話。」這春梅去了。這婦人在房中,比及洗了香ざ,修了足甲,也有好一囬。只見春梅拔了幾棵鳳僊花來,整呌秋菊搗了半夜。婦人又與了他幾鍾酒喫,打發他厨下先睡了。婦人燈光下染了十指春葱,令春梅拿櫈子放在天井内,鋪着凉簟衾枕納凉。約有更闌時分,但見朱户無聲,玉繩低轉,牽牛織女二星隔在天河兩岸;又忽聞一陣花香,幾點螢火。婦人手拈紈扇,正伏枕而待。春梅把角門虚掩。正是:

\begin{myquote}
待月西廂下,迎風户半開;

隔牆花影動,疑是玉人來。
\end{myquote}

原來經濟約定搖木槿花樹為號,就知他來了。婦人見花枝搖影,知是他來,便在院内咳嗽接應。他推開門進來,兩個並肩而坐。婦人便問:「你來,房中有誰?」經濟道:「大姐今日沒出來。我已安撫元宵兒在房裏,有事先來呌我。」因問:「秋菊睡了?」婦人道:「已睡熟了。」說畢,相摟相抱,二人就在院内櫈上,赤身露體,蓆枕交歡,不勝繾綣。但見:

\begin{myquote}
情興兩和諧,摟定香肩臉揾腮。手捻香乳綿似軟,實奇哉。掀起脚兒脫繡鞋。玉體着郎懷,舌送丁香口便開。倒鳳顛鸞雲雨罷,囑多才:明朝千萬早些來!
\end{myquote}

兩個雲雨畢,婦人拿出五兩碎銀子來,遞與經濟說:「門外你潘姥姥死了,棺材已是你爹在日與了他。三日入殮時,你大娘教我去探丧燒紙來了。明日出殯,你大娘不放我去,說你爹熱孝在身,不宜出門。這五兩銀子交與你,明日央你早去門外,發送發送你潘姥姥,打發擡錢,看着下入土内你纔來家,就同我去一般。」這經濟一手接了銀子,說:「這個不打緊。你吩咐我幹事;受人之託,必當終人之事!我明日絶早出門,幹畢事,來囬你老人家。」說畢,恐大姐進房,老早歸廂房中去了。

一宿晚景休題。到次日,到飯時就來家。金蓮纔起來,在房中梳頭。經濟走來囬話,就門外昭化寺裏,㧱了兩枝茉莉花兒來婦人戴。婦人問:「棺材下了葬了?」經濟道:「我管何事?不打發他老人家黄金入了櫃,我敢來囬話!還剩了二兩六七錢銀子,交付與你妹子收了,盤纏度日。千恩萬謝,多多上覆你。」婦人聽見他娘入土,落下淚來,便叫春梅:「把花兒浸在盞内,看茶來與你姐夫喫。」不一時,兩盒兒蒸酥,四碟小菜,打發經濟喫了茶,往前邊去了。由是越發與這小夥兒日親日近。

一日,七月天氣,婦人早晨約下他:「你今日休往那裏去,在房中等着,我往你房裏,和你耍耍。」這經濟答應了。不料那日,被崔本邀了他和幾個朋友,往門外耍子。去了一日,喫的大醉來家,倒在牀上就睡着了,不知天高地下。黄昏時分,金蓮驀地到他房中。見他挺在牀上,行李兒也顧不的,推他推不醒,就知他在那裏喫了酒來。可霎作怪,不想婦人摸他袖子裏,掉出一根金頭蓮瓣簪兒來,上面鈒着兩溜字兒:「金勒馬嘶芳草地,玉樓人醉杏花天。」迎亮一看,就知是孟玉樓簪子:「怎生落在他袖中?想必他也和玉樓有些首尾,不然他的簪子如何他袖着?怪道這短命,幾次在我面上無情無緒。我若不留幾個字兒與他,只說我没來。等我寫四句詩在壁上,使他知道。待我見了,慢慢追問他下落。」於是取筆,在壁上寫了四句詩曰:

\begin{myquote}
「獨步書齋睡未醒,空勞神女下巫雲。

襄王自是無情緒,辜負朝朝暮暮情。」
\end{myquote}

寫畢,婦人囬房中去了。卻說經濟睡起一覺,酒醒過來,房中掌上燈。因想起今日婦人來相會,我卻醉了。囘頭見壁上寫了四句詩在上,墨跡猶新,念了一遍,就知他來到空囬去了,把個送上門的風月兒白丢了!心中懊悔不已:「這咱已起更時分,大姐元宵兒都在後邊,未出來;我若往他那邊去,角門又關了!」走來木槿花下搖花枝為號,不聽見裏面動靜。不免踩着太湖石,爬過粉牆去。

那婦人見他有酒,醉了挺覺,大恨歸房,悶悶在心,就渾衣上牀歪睡。不料半夜他爬過牆來,見院内無人,想丫鬟都睡了,悄悄躡足潛踪,走到房門首,見門虚掩,就挨身進來。窗間月色,照見牀上,婦人獨自朝裏歪着。低聲呌「可意人」數聲,不應。說道:「你休怪我。今日崔大哥衆朋友,邀了我往門外五星原莊上,射箭耍子了一日,來家就醉了,不知你到,有負你之約,恕罪恕罪!」那婦人也不理他。這經濟見他不理,慌了,一面跪在地下,說了一遍又重復一遍。被婦人反手望臉上撾了一下,罵道:「賊牢拉負心短命,還不悄悄的,丫頭聽見!我知道你有個人,把我不放到心上。你今日端的那去來?」經濟道:「我本被崔大哥拉了門外射箭去,灌醉了,來家就睡着了。失悮你約,你休惱我。我看見你留詩在壁上,就知惱了你。」婦人道:「怪搗鬼牢拉的,別要說嘴,與我禁聲!你搗的鬼如泥彈兒圓,我手内放不過你!今日便是崔本呌了你喫酒,醉了來家。你袖子裏這根簪子,卻是那裏的?」經濟道:「本是那日花園中拾的來,今纔兩三日了。」婦人道:「你還㒲神搗鬼,是那花園裏拾的?你再拾一根來,我纔算!這簪子是孟三兒那麻淫婦的頭上簪子,我認千眞萬眞!上面還鈒着他名字,你還哄我?嗔道前日我不在,他呌進你房裏喫飯,原來你和他七個八個!我問着你,還不承認。你不和他兩個有首尾,他的簪子緣何到你手裏?原來把我的事都透露出與他,怪道前日他見了我笑,原來有你的話在頭裏。自今以後,你是你,我是我,綠豆皮兒請退了!」於是急的經濟賭神發咒,継之以哭,道:「我經濟若與他有一字絲麻皂線,靈的是東岳城隍,活不到三十歲,生來碗大疔瘡,害三五年黄病,要湯不見,要水不見!」那婦人終是不信,說道:「你這賊材料,說來的牙疼誓,虧你口内不害硶!」兩個絮聒了一囬,見夜深了,不免解卸衣衫,挨身上牀躺下。那婦人把身子扭過,倒背着他,使個性兒不理他,由着他姐姐長姐姐短,只是反手望臉上撾過去。唬的經濟氣也不敢出一聲兒來,乾霍亂了一夜,就不曾㒲成𣭈頭。天明,恐怕丫頭起身,依舊越墻而過,往前邊廂房中去了。有〔醉扶歸〕詞為證:

\begin{myquote}
我嘴搵着他油䯼髻,他背靠着我胸肚皮。早難送香腮左右偎,只在項窝兒裏長吁氣。一夜何曾見面皮,只覷着牙梳背!
\end{myquote}

看官聽說:往後金蓮還把這根簪子與了經濟。後來孟玉樓嫁了李衙内,往嚴州府去,經濟還拿着這根簪子做證見,認玉樓是姐,要暗中成事。不想玉樓哄逃,反陷經濟牢獄之災。此事表過不題。正是:三光有影遺誰翳,萬事無根只自生。

畢竟後來如何,且聽下囬分解。

