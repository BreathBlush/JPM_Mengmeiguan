\includepdf[pages={121,122},fitpaper=false]{tst.pdf}
\chapter*{第六十一囬 \\韓道國筵請西門慶 李瓶兒苦痛宴重陽}
\addcontentsline{toc}{chapter}{第六十一囬 韓道國筵請西門慶 李瓶兒苦痛宴重陽}
\markboth{{\titlename}卷之七}{第六十一囬 韓道國筵請西門慶 李瓶兒苦痛宴重陽}


\begin{myquote}
去年九日愁何限,重上心來益断腸。

秋色夕陽俱淡薄,淚痕離思共淒涼。

征鴻有隊全無信,黃菊無情卻有香。

自覺近來消瘦了,頻將鸞鏡照容光。
\end{myquote}

話説一日,韓道國晚夕舖中散了,囬家睡到半夜,他老婆王六兒與他商議:「你我被他照顧,此遭掙了恁些錢,就不擺席酒兒請他來坐坐兒?休説他又丢了孩兒,只當與他釋悶,也請他坐半日。他能喫多少?彼此好看些。就是後生小郎看着,到明日就到南邊去,也知財主和你我親厚,比別人不同。」韓道國道:「我心裏也是這等説。明日是初五日,月忌不好。到初六日,呌個廚子,安排酒席,呌兩個唱的,具個柬帖,等我親自到宅内請老爹散悶坐坐。我晚夕便往舖子裏睡去。」王六兒道:「平白又呌甚麽唱的!只怕他酒後要來這屋裏坐坐,不方便。隔壁楽三嫂家常走一個女兒申二姐,年紀小小兒的,打扮又風流,又會唱時興的小曲兒,倒請將他來唱唱罷。等晚夕酒闌上來,老爹若進這屋裏來,打發他過去就是了。」韓道國道:「你説的是。」一宿晚景題過。

到次日,這韓道國走到舖子裏,央及溫秀才寫了個請柬兒,走到對門宅内,親見西門慶。聲喏畢,説道:「老爹明日沒事,小人家裏治了一盃水酒,無事請老爹貴步下臨,散悶坐一日。」因把請柬遞上去。西門慶看了,説道:「你如何又費此心?我明白倒沒事,衙門中囬家就去。」那韓道國作辭出門,來到舖子做買賣。

到次早,㧱銀子呌後生胡秀,㧱籃子往街上買鷄蹄鵝鴨鮮魚嗄飯菜蔬;一面呌廚子在家整理割切。使小廝早㧱轎子接了申二姐來。王六兒同丫鬟伺候下好茶好水,客座内打掃收拾桌椅乾淨,單等西門慶來到。等到午後,只見琴童兒先送了一壜葡萄酒來;然後西門慶坐着涼轎,玳安王經跟隨,到門首下轎;頭戴忠靖冠,身穿青水緯羅直身,粉頭皂靴。韓道國接迎入内,見畢禮數,説道:「又多謝老爹賜將來酒!」正面獨獨安放一張校椅,西門慶坐下。不一時,王六兒打扮出來,頭上銀絲䯼髻,翠藍縐紗羊皮金滚邊的箍兒,週圍插碎金草蟲啄針兒;白杭絹對衿兒,玉色水緯羅比甲兒,鵝黄挑線裙子;脚上老鴉青光素緞子高底鞋兒,羊皮金緝的雲頭兒;耳邊金丁香兒:打扮的十分精緻。與西門慶插燭也似磕了四個頭兒,囬後邊看茶去了。須臾,王經紅漆描金托子,㧱了兩盞八寳青荳木樨泡茶,韓道國先取一盞,擧的高高奉與西門慶,然後自取一盞,旁邊相陪。喫畢,王經接了茶盞下去。韓道國便開言説道:「小人承老爹莫大之恩,一向在外,家中小媳婦蒙老爹看顧,王經又蒙擡擧,呌在宅中答應,感恩不淺。今日與媳婦兒商議,無甚孝順,治了一盃水酒兒,請老爹過來坐坐。前日因哥兒沒了,雖然小人在那裏,媳婦兒因感了些風寒,不曾往宅裏弔問的,恐怕老爹惱。今日一者請老爹解解悶,二者就恕俺兩口兒罪。」西門慶道:「無事又教你兩口兒費心。」説着,只見王六兒也在旁邊小杌兒坐下。因向韓道國道:「你和老爹説了不曾?」道國道:「我還不曾説哩。」西門慶問道:「是甚麽?」王六兒道:「他今日心裏要内邊請兩位姐兒來伏侍老爹,恐怕老爹計較,又不敢請。隔壁樂家常走的一個女兒,姓申,名喚申二姐,諸般大小時樣曲兒連數落都會唱。我前日在宅裏,見那一位郁大姐,唱的也中中的,還不如這申二姐唱的好。教我今日請了他來唱與爹聽,未知你老人家心下何如?若好,到明日呌了宅裏去,唱與他娘們聽。他也常在各人家走。若呌他,預先兩日定下他,他並不敢悞了。」西門慶道:「既是有女兒,一發好了,你請出來我看看。」

不一時,韓道國教玳安上來:「替老爹寬去衣服。」一面安放桌席,胡秀㧱菓菜案酒上來。無非是鴨臘蝦米海味燒骨秃之類。當下王六兒把酒打開,燙熱了,在旁執壺,道國把盞,與西門慶安席坐下。然後纔呌上申二姐來。西門慶睜眼觀看他:高髻雲鬟,插着幾枝稀稀花翠;□□□□,□□□□淡淡釵梳。綠衫紅袖,顯一對金蓮趫趫;桃腮粉臉,描兩道細細春山。青石墜子耳邊垂,糯米銀牙噙口内。望上花枝招颭與西門慶磕了四個頭。西門慶便道:「請起,你今青春多少?」申二姐道:「小的二十一歲了。」又問:「你記得多少小唱?」申二姐道:「小的大小也記百十套曲子。」西門慶令韓道國旁邊安下個坐兒與他坐。那申二姐向前行畢禮,方纔坐下,先㧱箏來唱了一套〔秋香亭〕;然後喫了湯飯,添換上來,又唱了一套「半萬賊兵」。落後酒闌上來,西門慶吩咐:「把箏㧱過去,取琵琶與他,等他唱小詞兒我聽罷。」那申二姐一逕要施逞他能彈擅唱,一面輕搖羅袖,款跨鮫綃,頓開喉音,把絃兒放得低低的,彈了個「四不應」〔山坡羊〕:

\begin{myquote}
「一向來,不曾和寃家面會,肺腑情難捎難寄。我的心誠想着你,你為我懸心掛意。咱兩個相交不分個彼此,山盟海誓心中牢記。你比鶯鶯重生而再有,可惜不在那蒲東寺。不由人一見了眼角留情來呵,玉貌生春你花容無比。呌了聲嬌姿,你敎人目断東牆,把西樓倦倚。

意中人,兩下裏懸心掛意,意兒裏不得和你兩個眉來眼去。去了時強挨孤枕,枕兒寒衾兒冷剩瑤琴獨對。病體如柴瘦損了腰肢。知道你夫人行應難離,倒等的我寸心如醉。最關心伴着這一盞寒燈來呵,又被風弄竹聲只道多情到矣。急忙忙出離了書幃,不想是花影輕搖,月明如水。」
\end{myquote}

唱了兩個〔山坡羊〕,呌了斟酒。那韓道國敎渾家篩酒上來,滿斟一盞,遞與西門慶。因説:「申二姐,你還有好〔鎖南枝〕,唱兩個兒與老爹聽。」那申二姐改了調兒,唱〔鎖南枝〕道:

\begin{myquote}
「初相會,可意人,年少青春不上二旬。黑鬖鬖兩朵烏雲,紅馥馥一點朱唇;臉賽夭桃,手如嫩笋。若生在畫閣蘭堂,端的也有個夫人分。可惜在章臺,出落做下品。但能够改嫁従良,勝強似棄舊迎新。

初相會,可意嬌,月貌花容風塵中最少。瘦腰肢一捻堪描,俏心腸百事難學。恨只恨和他相逢不早。常則願席上樽前,淺斟低唱相偎抱。一覷一個眞,一看一個飽。雖然是半霎歡娛,權且將悶減愁消。」
\end{myquote}

西門慶聽了這兩個〔鎖南枝〕,正打着他初請了鄭月兒那一節事來,心中甚喜,讚他呌了個賞音。王六兒在旁滿滿的又斟上一盞,笑嘻嘻説道:「爹,你慢慢兒的消飲。申二姐這個纔是零頭兒,他還記得好些小令兒哩。到明日閒了,㧱轎子接了,唱與他娘們聽。」又説:「宅中那位唱姐兒?」西門慶道:「那個是常在我家走的郁大姐,這好些年代了。」王六兒道:「管情申二姐到宅裏,比他唱的高。爹到明日呼喚他,早些兒來對我説。我使孩子早㧱轎子去接他,送到宅内去。」西門慶因説:「申二姐,我重陽那日使人來接你,去不去?」申二姐道:「老爹説那裏話,但呼喚小的,怎敢違阻?」西門慶聽見他會説話兒,心中大喜。不一時,交盃換盞之間,王六兒恐席間説話不方便,教他唱了幾套,悄悄向韓道國説:「敎小廝招弟兒,送過他那邊樂三嫂家歇去罷。」臨去拜辭西門慶,西門慶向袖中掏出一包兒三錢銀子,賞賜與他買絃。那申二姐連忙花枝招颭,向西門慶磕頭謝了。西門慶約下:「我初八日使人請你去。」那王六兒道:「爹只教王經來對我説,等這裏教小廝送他去。」那申二姐拜辭了韓道國夫婦,招弟領着往隔壁去了。

那韓道國打發申二姐去了,與老婆説知,就往舖子裏睡去了。只落下老婆在席上,陪西門慶擲骰飲酒。喫了一囬,兩個看看喫的涎將上來,西門慶推起身往後邊更衣,就走入婦人房裏,兩個頂門頑耍。王經便把燈燭㧱出來,在前半間内,和玳安琴童兒三個做一處飲酒。那後生胡秀,不知這多咱時分在後邊廚下偸喫多幾碗酒,打發廚子去了,走在王六兒隔壁半間供養佛祖先堂兒内,地下鋪着一領蓆就睡着了。睡了一覺起來,原來與那邊臥房止隔着一層板壁兒,忽聽婦人房裏聲喚起來。這胡秀只見板壁縫兒透過燈亮兒來,只道西門慶去了,韓道國在房中宿歇,暗暗用頭上簪子取下來,刺破透板縫中糊的紙,打一往那邊張看。見那邊房中亮騰騰點着燈燭,不想西門慶和老婆在屋裏兩個正幹得好。伶伶俐俐,看見把老婆兩隻腿卻是用脚帶弔在牀頂上,西門慶上身止着一件綾襖兒,下身赤露,就在牀沿上,兩個一來一往,一動一靜,𢵞打的連聲響亮。老婆口裏百般言語都呌將出來,淫聲豔語,通做成一塊。良久,只聽老婆説:「我的親逹,你要燒淫婦,隨你心裏揀着那塊,只顧燒,淫婦不敢攔你。左右淫婦的身子屬了你,顧的那些兒了!」西門慶道:「只怕你家裏的嗔是的!」老婆道:「那忘八七個頭八個膽,他敢嗔?他靠着那裏過日子哩!」西門慶道:「你旣是一心在我身上,到明日等賣下銀子,這遭打發他和來保起身,一發留他長遠在南邊立莊,做個買手。家中已有甘夥計發賣,那裏只是缺少個買手,看着置貨。」老婆道:「等走過兩遭兒囬來,卻敎他去。省的閒着在家做甚麽!他説道,倒在外邊走慣了,一心只要外邊去。他江湖従小兒走過,甚麽買賣客貨中事兒不知道?你若下顧他,可知好哩。等他囬來,我房裏替他尋下一個。我也不要他,一心撲在你身上,隨你把我安插在那裏就是了。我若説一句假,把淫婦不値錢身子就爛化了。」西門慶道:「我兒,你快休賭誓!」這裏兩個一動一靜,都被這胡秀聽了個不亦楽乎。

那韓道國先在家中不見胡秀,只説往舖子裏睡去了。走到緞子舖裏,問王顯榮海,説他沒來。韓道國一面又走囬家,呌開門,前後尋胡秀,那裏得來?只見王經陪玳安琴童,三個在前邊喫酒。這胡秀聽見他的語音來家,連忙倒在蓆上,又推睡了。不一時,韓道國點燈尋到佛堂地下,看見他鼻口内打鼾睡,用脚踢醒,駡道:「賊野狗死囚,還不起來!我只説先往舖子裏睡去,你原來在這裏挺的好覺兒。還不起來跟我去?」那胡秀起來,推揉了揉眼,ここ睜睜,跟道國往舖子裏去了。

西門慶弄老婆,直弄夠有一個時辰,方纔了事。燒了王六兒心口裏並ず蓋子上尾停骨兒上共三處香。老婆起來,穿了衣服,敎丫鬟打發舀水淨了手。重篩煖酒,再上佳肴,情話攀盤,又喫了幾鍾,方纔起身上馬。

玳安王經琴童三個跟着,到家中已有二更天氣。走到李瓶兒房中。李瓶兒睡在牀上,見他喫的酣酣兒的進來,説道:「你今日在誰家喫酒來?」西門慶悉言:「韓道國家請我。見我丢了孩子,與我釋悶。他家呌了個女先生申二姐來,年經小小,好不會唱,又不數郁大姐。等到明日重陽,使小廝㧱轎子接他來家唱兩日你們聽,就與你解解悶。你緊自心裏不好,休要只顧思想他了。」説着,就要呌迎春來脫衣裳,和李瓶兒睡。李瓶兒道:「你沒的説,我下邊不住的長流,丫頭火上替我煎着薬哩。你往別人屋裏睡去罷!你看着成日好模樣兒罷了,只有一口遊氣兒在這裏,還來纏我起來。」西門慶道:「我的心肝!我心裏捨不的你,只要和你睡,如之奈何?」李瓶兒瞟了他一眼,笑了笑兒:「誰信你那虚嘴掠舌的,我到明日死了,你也捨不的我罷?」又道:「一發等我好好兒,你再進來和我睡,也是不遲。」那西門慶坐了一囬,説道:「罷罷!你不留我,等我往潘六兒那邊睡去罷。」李瓶兒道:「着來!你去,省的屈着你那心腸兒。他那裏正等的你火裏火發。你不去,卻忙愡兒來我這屋裏纏!」西門慶道:「你恁説,我又不去了。」那李瓶兒微笑道:「我哄你哩,你去麽!」於是打發西門慶過去了。這李瓶兒起來,坐在牀上,迎春伺候他喫薬。㧱起那薬來,止不住撲簌簌従香腮邊滚下淚來,長吁了一口氣,方纔喫那盞薬。正是:心中無限傷心事,付與黄鸝呌幾聲。

不説李瓶兒喫薬睡了。單表西門慶到於潘金蓮房裏。金蓮纔敎春梅罩了燈,上牀睡下。忽見西門慶推開門,進來便道:「我兒,又早睡了?」金蓮道:「稀行!那陣風兒刮你到我這屋裏來?」因問:「你今日往誰家喫酒去來?」西門慶道:「韓夥計打南邊來,見我沒了孩子,一者與我釋悶,二者照顧了他外邊走了這遭,請我坐坐。」金蓮道:「他便在外邊,你在家卻照顧了他老婆了。」西門慶道:「夥計家,那裏有這道理!」婦人道:「夥計家,有這個道理?齊腰拴着根線兒,只怕㒲過界兒去了!你還搗鬼哄俺們哩,俺們知道的不耐煩了!你生日時,賊淫婦他沒在這裏?你悄悄把李瓶兒壽字簪子,黄貓黑尾偸與他,卻敎他戴了來這裏施展。大娘孟三兒這一家子那個沒看見?乞我相問着,他那臉兒上紅了。他沒告訴你?今日又摸到那裏去了,賊沒廉耻的貨,你家外頭還少哩!也不知怎的一個大摔瓜長淫婦,喬眉喬樣,描的那水鬢長長的,搽的那嘴唇鮮紅的,倒像人家那血ず,甚麽好老婆,一個大紫膛色黑淫婦,我不知你喜歡他那些兒!嗔道把忘八舅子也招惹將來,卻一早一晚敎他好往囬捎話兒。」那西門慶堅執不認,笑道:「怪小奴才兒,單管只胡説!那裏有此勾當?今日他男子漢陪我坐,他又沒出來。」婦人道:「你㧱這個話兒來哄我?誰不知他漢子是個明忘八!又放羊,又拾柴,一徑把老婆丢與你,圖你家買賣做,要賺你的錢使。你這儍行貨子,只好四十里聽銃響罷了!」見西門慶脱了衣裳坐在牀沿上,婦人探出手來把褲子扯開,摸見那話軟叮噹的,托子還帶在上面,説道:「可又來!你臘鴨子煮到鍋裏,身子兒爛了嘴頭兒還硬,現放着不語先生在這裏強道!和那淫婦怎麽弄聳,聳到這早晚纔來家?弄的恁軟如鼻涕濃瓜醬的,嘴頭兒還強哩!你賭個兒誓,我敎春梅舀一瓶子涼水,你只喫了,我就算你好膽子。論起來,鹽也是這般鹹,醋也是這般酸,秃子包網巾,饒這一抿子兒也罷了!若是信着你意兒,把天下老婆都耍遍了罷。賊沒羞的貨,一個大眼裏火行貨子!你早是個漢子,若是個老婆,就養遍街,㒲遍巷,屬皮匠的——縫着的就緔。」幾句説的西門慶睜睜的。

上的牀來,敎春梅篩熱了燒酒,把金穿心盒兒内薬,拈了一粒,放在口裏嚥下去。仰臥在枕上。令婦人:「我兒,你下去替你達品品,品起來是你造化。」那婦人一徑做喬張致,便道:「好乾淨兒,你在那淫婦窟礲子裏鑽了來,敎我替你咂,可不臢殺了我!」西門慶道:「怪小淫婦兒,單管胡説白道的,那裏有此勾當?」婦人道:「那裏有此勾當,你指着肉身子賭個誓麽?」亂了一囬,教西門慶下去使水,西門慶不肯下去。婦人旋向袖子裏掏出通花汗巾來,將那話抹展了一囬,方纔用朱唇裹沒,嗚咂半响,登時咂弄的那話奢稜跳腦,暴怒起來。乃騎在婦人身上,縱麈柄自後插入牝中,兩手兜其股,蹲踞而提之,肆行𢵞打,連聲響亮。燈光之下,窺玩其出入之勢。婦人倒伏在枕畔,擧股迎凑者久之,西門慶興猶不愜,將婦人仰臥朝上,那話上使了粉紅薬兒,頂入去,執其雙足,又擧腰沒稜露腦掀騰者將二三百度。婦人禁受不的,瞑目顫聲,沒口子呌:「達達,你這遭兒只當將就我,不使上他也罷了!」西門慶口中呼呌道:「小淫婦兒,你怕我不怕?再敢無禮不敢?」婦人道:「我的達達,罷麽。你將就我些兒,我再不敢了。達達慢慢提,看提撒了我的頭髮。」兩個顛鸞倒鳳,又狂了半夜,方纔體倦而寢。

話休饒舌。又早到重陽令節。西門慶對吳月娘説:「韓夥計家前日請我,席上唱的一個申二姐,生的人材又好,又會唱,琵琶箏都會。我使小廝接他去。等接了他來,留他兩日,敎他唱與你們聽。」於是吩咐廚下,收拾酒菓肴饌。在花園大捲棚聚景堂内,安放大八僊桌席,放下簾來,合家宅眷在那裏飲酒,慶賞重陽佳節。不一時,王經轎子接的申二姐到了。入到後邊,與月娘衆人磕了頭。月娘見他年小,生的好模樣兒,問他套數,倒會不多。若題諸般小曲兒,〔山坡羊〕、〔鎖南枝〕,兼〔數落〕,倒記的有百十來個。一面打發他喫了茶食,先敎在後邊唱了兩套。然後花園擺設下酒席。那日西門慶不曾往衙門中去,在家看着栽了菊花,請了月娘、李嬌兒、孟玉樓、潘金蓮、李瓶兒、孫雪娥,並大姐,都在席上坐的。春梅、玉簫、迎春、蘭香,在旁斟酒伏侍。申二姐先㧱琵琶在旁彈唱。那李瓶兒在房中身上不方便,請了半日,纔請了來,恰似風兒刮倒的一般,強打着精神,陪西門慶坐。衆人讓他酒兒,也不大好生喫。西門慶和月娘見他面帶憂容,眉頭不展,説道:「李大姐,你把心放開,敎申二姐唱個曲兒你聽。」玉樓道:「你説與他,敎他唱甚麽曲兒,他好唱。」那李瓶兒只顧不説。正飲酒中間,忽見王經走來説道:「應二爹常二叔來了。」西門慶道:「請你應二爹常二叔在小捲棚裏坐,我就來。」王經道:「常二叔敎人㧱了兩個盒子在外頭。」西門慶向月娘道:「此是他成了房子,買了些禮來謝我的意思。」月娘道:「少不的安排些甚麽管待他,怎好空了他去?你陪他坐去,我這裏吩咐看菜兒。」西門慶臨出來,又呌申二姐:「你好歹唱個好曲兒,與他六娘聽。」一直往前邊去了。金蓮道:「也沒見這李大姐,隨你心裏説個甚麽曲兒,教申二姐唱個你聽就是了!辜負他爹的心。比來為你呌將他來,你又不言語的。」於是催逼的李瓶兒急了,半日纔説出來:「你唱個『紫陌紅徑』俺們聽聽。」那申二姐道:「這個不打緊,我有。」於是取過箏來,排開鴈柱,調定冰絃,頓開喉音,唱〔折腰一枝花〕:

\begin{myquote}
「紫陌紅徑,丹青妙手難畫成。觸目繁華如鋪錦,料應是春負我,我非是辜負了春。為着我心上人,對景越添愁悶。」

{\markfont〔東甌令〕}「花零亂,柳成陰,正是蝶困蜂迷鶯倦吟。方纔眼睜,心兒裏忘了想。啾啾唧唧呢喃燕,重將舊恨舊恨又題醒。撲撲簌簌,淚珠兒暗傾。」

{\markfont〔四團花〕}「悄悄的庭院深,默默的情掛心。涼亭水閣,果是堪宜宴飲。不見我情人,和誰兩個開樽?把絲絃再理,將琵琶自撥,是奴欲寬悶情,怎如倦聽!」

{\markfont〔東甌令〕}「榴如火,簇紅巾,有焰無煙燒碎我心。懷羞向前,欲待要摘一朶。觸觸拈拈不敢戴,怕奴家花貌不似舊時容。伶伶仃仃,怎宜樣簪?」

{\markfont〔梧桐樹〕}「梧葉兒飄,金風動,漸漸害相思,落入深深井。一日一日夜長,夜長難捱孤枕。懶上危樓望我情人,未必薄情與奴心相應。知他在那裏,那裏貪歡戀飲。」

{\markfont〔東甌令〕}「菊花綻,桂花零,如今露冷風寒秋意漸深。驀聽的窗兒外,幾聲孤飛鴈。悲悲切切如人訴,最嫌花下砌畔小蛩吟。咭咭聒聒,惱碎奴心。」

{\markfont〔浣溪沙〕}「風漸急,寒威澟,害相思最恐怕黄昏。沒情沒緒對着一盞孤燈,窗欞兒數遍還再輪。畫角悠悠聲透耳,一聲聲哽咽難聽。愁來把酒強重斟,酒入悶懷珠淚傾。」

{\markfont〔東甌令〕}「長吁氣,兩三聲,斜倚定幃屏兒思量那個人。一心指望夢兒裏,略略重相見。撲撲簌簌雪兒下,風吹簷馬把奴夢魂驚。叮叮噹噹,攪碎了奴心。」

{\markfont〔尾聲〕}「為多情,牽掛心。朝思暮想淚珠傾,恨殺多才不見影!」
\end{myquote}

唱畢,吳月娘道:「李大姐,你好甜酒兒喫上一鍾兒。」那李瓶兒又不敢違阻了月娘,拿起鍾兒來,咽了一口兒又放下了。強打着精神兒與衆人坐的。坐不多時,下邊一陣熱熱的來,又往屋裏去了。

不説這裏内眷。單表西門慶到於小捲棚翡翠軒,只見應伯爵與常時節在松墻下正看菊花。原來松牆兩邊,擺放二十盆都是七尺高各樣有名的菊花,也有大紅袍、狀元紅、紫袍金帶、白粉西、黄粉西、滿天星、醉楊妃、玉牡丹、鵝毛葡、鴛鴦花之類。西門慶出來,二人向前作揖。常時節即喚跟來人把盒兒掇進來。西門慶一見便問:「又是甚麽?」伯爵道:「常二哥蒙你厚情,成了房子。無甚麽酬答,敎他娘子製造了這螃蟹鮮,並兩雙爐燒鴨兒,邀我來同哥坐坐。」西門慶道:「常二哥,你又費這個心做甚麽?你令正病纔好些,你又禁害他!」伯爵道:「我也是恁説!他説道:『別的東西兒來,恐怕哥不稀罕。』」西門慶令左右打開盒兒觀看,四十個大螃蟹,都是剔剝淨了的,裏邊釀着肉,外用椒料、薑蒜米兒、團粉裹就,香油煠、醬油醋造過,香噴噴酥脆好食。又是兩大隻院中爐燒熟鴨。西門慶看了,即令春鴻王經掇進去。吩咐:「㧱五十文錢賞㧱盒人。」因向常時節謝畢。琴童在旁掀簾,請入翡翠軒坐的。伯爵只顧誇獎不盡好菊花,問:「哥是那裏尋的?」西門慶道:「是管磚廠劉太監送我這二十盆。」伯爵道:「連這盆?」西門慶道:「就連這盆都送與我了。」伯爵道:「花到不打緊,這盆正是官窰雙箍澄漿盆,又喫年代,又禁水漫。都是用絹羅打,用脚跐過泥,纔燒造這個物兒。與蘇州澄漿磚一個樣兒做法,如今那裏尋去?」

誇了一回,西門慶喚茶來喫了。因問:「常二哥幾時搬過去?」伯爵道:「従兑了銀子,三日就搬過去了。那家子已是尋下房子,兩三日就搬了。昨見好日子,買刮了些雜貨兒,門首把舖兒也開了。就是常二嫂兄弟,替他在舖兒裏看銀子兒。」西門慶道:「俺們幾時買些禮來,休要人多了,再邀謝子純、你,三四位。我家裏整理菜兒擡了去,休費煩常二哥一些東西兒。呌兩個妓者,咱們替他暖暖房,耍一日。」常時節道:「小弟有心也要請哥坐坐,算計來不敢請,地方兒窄狹,恐怕哥受屈馳。」西門慶道:「沒的扯淡!那裏又費你的事起來?如今使小廝請將謝子純來,和他説説。」即令琴童兒:「快請你謝爹去。」伯爵因問:「哥,你那日呌那兩個去?」西門慶笑道:「呌你鄭月娘和洪四兒去。洪四兒令打掇鼓兒,唱慢〔山坡羊〕兒。」伯爵道:「哥,你是個人!你請他,就不對我説聲?我怎的也知道了!比李桂兒風月如何?」西門慶道:「通色絲子女不可言。」伯爵道:「他怎的前日你生日時,那等不言語,扭扭的?也是個肉佞賊小淫婦兒!」西門慶道:「等我到幾時再去着,也㩦帶你走走。你月娘兒會打的好雙陸,你和他打兩貼雙陸。」伯爵道:「等我去混那小淫婦兒,休要慣了他!」西門慶道:「你這歪狗才,不要惡識他便好!」

正説着,謝希大到了。聲喏畢,坐下。西門慶道:「常二哥如此這般,新有了華居,瞞着俺們已搬過去了。咱每人隨意出些分資,休要費煩他絲毫。我這裏整治停當,教小廝擡了他府上,我還助兩個妓者,咱耍一日何如?」謝希大道:「哥吩咐每人出多少分資,俺們都送哥這裏來就是了。還有那幾位?」西門慶道:「再沒人,只這三四個兒。每人二星銀子就夠了。」伯爵道:「十分人多了,他那裏沒地方兒。」正説着,只見琴童來説:「吳大舅來了。」西門慶道:「請你大舅這裏來坐。」

不一時,吳大舅進入軒内。先與三人作了揖,然後與西門慶叙禮坐下。小廝㧱茶上來,同喫了茶。吳大舅起身説道:「請姐夫到後邊説句話兒。」西門慶連忙讓大舅到於後邊月娘房裏。月娘還在捲棚内,與衆姊妹喫酒聽唱。聽見小廝説:「大舅來了,爹陪着在後邊坐着説話哩。」一面走到上房見大舅,道了萬福,叫小玉遞上茶來。大舅向袖中取出十兩銀子遞與月娘,説道:「昨日府裏纔領了三錠銀子。姐夫且收下這十兩,餘者待後次再送來。」西門慶道:「大舅,你怎的這般計較?且使着,慌怎的?」大舅道:「我恐怕遲了姐夫的。」西門慶因問:「倉廒修理的也將完了?」大舅道:「還得一個月纔完。」西門慶道:「工完之時,一定撫按有些獎勵。」大舅道:「今年考選軍政在邇,還望姐夫扶持,大巡上替我説説。」西門慶道:「大舅之事,都在於我。」説畢話,月娘道:「請大舅來前邊坐。」大舅道:「我去罷。只怕他三位來有甚話説。」西門慶道:「沒甚麽話。常二哥新近問我借了幾兩銀子,買下了兩間房子,已搬過去了。今日買了些禮兒來謝我。節間留他們坐坐,不想大舅來的正好。」於是讓至前邊坐下。月娘連忙敎廚下打發菜兒上去。

琴童與王經先安放八僊桌席端正,㧱上小菜菓酒上去。西門慶旋教開庫房,㧱出一罈夏提刑家送的菊花酒來。打開碧靛清,噴鼻香,未曾篩,先攙一瓶涼水,以去其蓼辣之性。然後貯於布甑内篩出來,醇厚好喫,又不數葡萄酒。教王經用小金鍾兒斟一盃兒,先與吳大舅嘗了。然後伯爵等每人都嘗訖,極口稱羡不已。湏臾,大盤大碗嗄飯肴品擺將上來,堆滿桌上。先㧱了兩大盤玫瑰菓餡蒸糕,蘸着白砂糖,衆人趂熱搶着喫了一頓。然後纔㧱上釀螃蟹,並兩盤燒鴨子來。伯爵讓大舅喫。連謝希大也不知是甚麽做的,這般有味、酥脆好喫。西門慶道:「此是常二哥家送來的。」大舅道:「我空癡長了五十二歲,並不知螃蟹這般造作,委的好喫!」伯爵又問道:「後邊嫂子都嘗了嘗兒不曾?」西門慶道:「房下們都有了。」伯爵道:「也難為我這常嫂,也這般好手段兒。」常時節笑道:「賤累還恐整理的不堪口,敎列位哥笑話。」喫畢螃蟹,左右上來斟酒。西門慶令春鴻和書童兩個在旁,一遞一個歌唱南曲。

應伯爵忽聽大捲棚内彈箏歌唱之聲,便問道:「哥,今日有李桂姐在這裏?不然,如何這等音樂之聲?」西門慶道:「你再聽,看是不是?」伯爵道:「李桂姐不是,就是吳銀兒。」西門慶道:「你這花子,單管只瞎謅。倒是個女先生!」伯爵道:「不是郁大姐?」西門慶道:「不是他,這個是申二姐,年小哩,好個人材,又會唱。」伯爵道:「眞個這等好?哥怎的不牽出來,俺們瞧瞧,又唱個兒俺們聽。」西門慶道:「今日你衆娘們,大節間呌他來賞重陽頑耍,偏你這狗才耳朵尖聽的見。」伯爵道:「我便是千里眼,順風耳。隨他四十里有蜜蜂兒呌,我也聽見了。」謝希大道:「你這花子,兩耳朶似竹簽兒也似,愁聽不見!」兩個又頑笑了一囬。伯爵道:「哥,你好歹呌他出來,俺們見見。俺們不打緊,敎他只當唱個兒與老舅聽也罷了,休要執古了。」西門慶乞他逼迫不過,一面使王經:「領申二姐出來,唱與大舅聽。」不一時,申二姐來,望上磕了頭,起來,旁邊安放校牀兒,與他坐下。伯爵問申二姐:「青春多少?」申二姐囬道:「屬牛的,二十一歲了。」又問:「會多少小唱?」申二姐道:「琵琶箏上套數小唱,也會百十來個。」伯爵道:「你會許多唱,也夠了。」西門慶道:「申二姐,你㧱琵琶唱小詞兒罷!省的勞動了你。説你會唱『四夢八空』,你唱與大舅聽!」吩咐王經書童兒席間斟上酒。那申二姐款跨鮫綃,微開檀口,唱〔羅江怨〕道:

\begin{myquote}
「懨懨病漸濃,甚日消融?春思夏想秋又冬,滿懷愁悶訴與天公。也。天有知呵,怎不把恩情送?恩多也是個空,情多也是個空,都做了南柯夢。

伊西我在東,何日再逢?花箋慢冩封又封,叮嚀囑付與鱗鴻。也。他也不中,不把我這音書送。思量他也是空,埋怨他也是空,都做了巫山夢。

恩情逐曉風,心意懶慵。伊家做作無始終,山盟海誓一似耳邊風。也。不記當初,多少恩情重。虧心也是空,癡心也是空,都做了蝴蝶夢。

惺惺似懞懂,落伊套中。無言暗把珠淚湧,口心誰想不相同。也。一片眞心,將我廝調弄。得便宜也是空,失便宜也是空,都做了陽臺夢。」
\end{myquote}

不説前邊彈唱飲酒。且説李瓶兒歸到房中,坐淨桶,下邊似尿也一般只顧流將起來,登時流的眼黑了。起來穿裙子,忽然一陣旋暈的,向前一頭拾倒在地。饒是迎春在旁搊扶着,還把額角上磕傷了皮。和奶子搊到炕上,半日不省人事。慌了迎春,使綉春連忙快對大娘説去。那綉春走到席上,報與月娘衆人:「俺娘在房中暈倒了。」這月娘撇了酒席,與衆姊妹慌忙走來看視。見迎春奶子兩個搊扶着他,坐在炕上,不省人事,便問:「他好好的進屋裏,端的怎麽來就不好了?」迎春揭開淨桶與月娘瞧,把月娘唬了一跳,説道:「此是他剛纔只怕喫了酒,助趕的他這血旺了,流了這些。」玉樓金蓮都説:「他幾曾大好生喫酒來?」一面煎燈心薑湯灌他。半晌甦着過來,纔説出話兒來了。月娘問:「李大姐,你怎的來?」李瓶兒道:「我不怎的。坐下桶子,起來穿裙子,只見眼面前黑黑的一塊子,就不覺天旋地轉起來,由不的身子就倒了。」月娘便要使來安兒:「請你爹進來。對他説,敎他請任醫官來看你。」那李瓶兒又嗔敎請去:「休要大驚小怪,打攪了他喫酒。」月娘吩咐迎春:「打舖敎你娘睡罷。」月娘於是也就喫不成酒了,吩咐收拾了家伙,都歸後邊去了。

西門慶陪侍吳大舅衆人,至晚歸到後邊月娘房中。月娘告訴李瓶兒跌倒之事。西門慶慌走到前邊來看視。見李瓶兒睡在炕上,面色蠟渣黄了,扯着西門慶衣袖哭泣。西門慶問其所以。李瓶兒道:「我到屋裏坐榪子。不知怎的,下邊只顧似尿也一般流起來。不覺眼前一塊黑黑的,起來穿裙子,天旋地轉,就跌倒了。恁甚麽就顧不的了!」西門慶見他額上磕傷一道油皮,説道:「丫頭都在那裏,不看你?怎的跌傷了面貌?」李瓶兒道:「還虧大丫頭都在跟前,和奶子搊扶着我。不然,還不知跌得怎樣的。」西門慶道:「我明日還早使小廝請任醫官來看你看。」當夜就在李瓶兒對面牀上睡了一夜。

次日早晨,沒往衙門裏去,旋使琴童騎頭口請任醫官去了。直到晌午纔來。西門慶先在大廳上陪喫了茶,使小廝説進去。李瓶兒房裏收拾乾淨,薰下香,然後請任醫官到房中。診畢脉,走出外邊廳上,對西門慶説:「老夫人脉息,比前番甚加沉重些。七情感傷,肝火太盛,以致木旺土虚,血熱妄行,猶如山崩而不能節制。復使大官兒後邊問去,若所下的血,紫者猶可以調理,若鮮紅者,乃新血也。學生撮過薬來,若稍止則可有望,不然,難為矣!」西門慶道:「望乞老先生留神加減,學生必當重謝!」任醫官道:「是何言語?你我厚間,又是明川情分,學生無不盡心。」西門慶待畢茶,送出門。隨即具一疋杭絹、二兩白金,使琴童兒討將薬來,名曰歸脾湯,乘熱而喫下去,其血越流之不止。西門慶越發慌了。又請大街口胡太醫來瞧。胡太醫説是氣冲血管,熱入血室。亦取將薬來,喫下去,如石沉大海一般。

月娘見前邊亂着請太醫,只留申二姐住了一夜,與了他五錢銀子,一件雲絹比甲兒並花翠,裝了個盒子,打發他坐轎子去了。花子由自従開張那日喫了酒去,聽見李瓶兒不好,至是使了花大嫂買了兩盒禮來看他。見他瘦的黄懨懨兒,不比往時,兩個在屋裏大哭了一囬。月娘後邊擺茶,請他喫了。韓道國説:「東門外住的一個看婦人科的趙太醫,指下明白,極看得好。前歲小姪媳婦月經不通,是他看來。老爹這裏差人,請他來看看六娘,管情就好!」西門慶於是就使琴童同王經兩個疊騎着頭口,往門外請趙太醫去了。西門慶請了應伯爵來,在廂房坐的,和他商議:「第六個房下,甚是不好的重,如之奈何?」伯爵失驚道:「這個……嫂子貴恙,説好些,怎的又不好起來?」西門慶道:「自從小兒沒了,一向着了憂慼,把病來又犯了。昨日重陽,我説接了申二姐,節間你們打夥兒散悶頑耍。他又沒大好生喫酒。誰知走到屋中就不好,暈起來一跤跌倒在地,把臉都磕破了。請任醫官來看,説脉息比前沉重。喫了薬,倒越發血盛了。」伯爵道:「哥,你請胡太醫來看,怎的説?」西門慶道:「胡太醫説是氣冲了血管,喫了他的薬,也不見動靜。今日韓夥計説,門外一個趙太醫,名喚趙龍崗,專科看婦女。我使小廝騎頭口請去了。一向把我焦愁的了不得!生生為這孩子不好,是白日黑夜思慮起這病來了。婦女人家,又不知個囬轉,勸着他,又不依你,敎我無法可䖏!」

正説着,平安來報:「喬親家爹來了。」西門慶一面讓進廳上坐。叙禮已畢,坐下。喬大户道:「聞得六親家母有些不安,昨日舍甥到家,請房下便來奉看。」西門慶道:「便是。一向因小兒沒了,他着了愁慼,身上原有些不調,又感發起來了。蒙親家掛心。」喬大户道:「也曾請人來看不曾?」西門慶道:「常喫任後溪的薬。昨日又請大街胡先生來看,喫薬越發轉盛,今日又請門外專看婦人科趙龍崗去了。」喬大户道:「咱縣門前住的行醫何老人,大小方脉俱精。他兒子何岐軒,現今上了個冠帶醫士。親家何不請他來看看親家母?」西門慶道:「旣是好,等小价請了趙龍崗來看了脉息,看怎的説,再請他來不遲。」喬大户道:「親家,依我愚見,如今請了何老人來看了親家母脉息,講説停當,安在廂房内坐的。待盛价門外請將趙龍崗來,看他診了脉怎麽説,教他兩個細講一講,就論出病源來了。然後下薬,無有個不效之理。」西門慶道:「親家説的是。」一面使玳安:「㧱我拜帖兒,和喬通去請縣門前行醫何老人來。」玳安等應諾去了。西門慶請伯爵到廳上,與喬大户相見,同坐一䖏喫茶。

那消片晌之間,何老人到來。進門與西門慶喬大户等作了揖,讓於上面坐下。西門慶擧手道:「數年不見你老人家,不覺越發蒼髯皓首。」喬大户又問:「令郎先生肆業盛行?」何老人道:「他逐日縣中迎送,也不得閒。倒是老拙常出來看病。」伯爵道:「你老人家高壽了?還這等健朗!」何老人道:「老拙今年癡長八十一歲。」叙畢話,看茶上來喫了。小廝説進去。須臾請至房中,就牀看李瓶兒脉息,旋搊扶起來,坐在炕上。挽着香雲,阻隔三焦,形容瘦的十分狼狽了。但見他:

\begin{myquote}
面如金紙,體似銀條。看看減褪丰標,漸漸消磨精彩。胸中氣急,連朝水米怕沾唇,五臟膨脝,盡日薬丸難下腹。隱隱耳虚聞磐響,昏昏眼暗覺螢飛。六脉細沉,東岳判官催命去;一靈縹緲,西方佛子喚同行。喪門弔客已臨身,扁鵲盧醫難下手。
\end{myquote}

那何老人看了脉息,出來外邊廳上,向西門慶喬大户説道:「這位娘子乃是精冲了血管起,然後着了氣惱,氣與血相博則血如崩。細思當初起病之由,看是也不是?」西門慶道:「是便是,你老人家如何治療?」正相論間,忽報:「琴童和王經門外請了趙先生來了。」何老人便問:「是何人?」西門慶道:「也是夥計擧來一醫者。你老人家只推不知,待他看了脉息出來,你老人家和他兩個相講一講,好下薬。」不一時,趙太醫從外而入。西門慶與他叙禮畢,然後與衆人相見。何喬二老居中,讓他在左,應伯爵在右,西門慶主位相陪。來安兒㧱上茶來喫了,收下盞托去。此人便問:「二位尊長貴姓?」喬大户道:「俺二人一位姓何,一位姓喬。」伯爵道:「在下姓應。敢問先生高姓,尊寓何䖏,治何生理?」其人答道:「不敢。在下小子,家居東門外頭條巷二郎廟三轉橋四眼井住的,有名趙搗鬼便是。平生以醫為業。家祖現為太醫院院判,家父現充汝府良醫。祖傳三輩,習學醫術。每日攻習王叔和、東垣勿聽子,《薬性賦》、《黄帝素問》、《難經》、《活人書》、《丹溪纂要》、《丹溪心法》、《潔古老脉訣》、《加減十三方》、《千金奇效良方》、《壽域神方》、《海上方》,無書不讀,無書不看。薬用胸中活灋,脉明指下玄機。六氣四時,辨陰陽之標格;七表八裏,定関格之沉浮。風虚寒熱之症候,一覽無餘;弦洪芤石之脉理,莫不通曉。小人拙口鈍脗,不能細陳。聊有幾句,道其梗概。」便道:

\begin{myquote}
「我做太醫姓趙,門前常有人呌。

只會賣杖搖鈴,那有眞材實料。 

行醫不按良方,看脉全憑嘴調。

撮薬治病無能,下手取積不妙。

頭疼須用䋲箍,害眼全憑艾醮。

心疼定敎刀剜,耳聾宜將針掏。

得錢一味胡醫,圖利不圖見效。

尋我的少吉多兇,到人家有哭無笑。

正是:半積陰功半養身,古來醫道通僊道。」
\end{myquote}

衆人聽了,都呵呵笑了。何老人道:「你門裏出身,門外出身?」趙太醫道:「門裏出身怎的説?門外出身怎的説?」何老人道:「你門裏出身,有父傳子接脉理之良法。若是門外出身,只可問病下薬而已。」趙太醫道:「老先生你就不知道,古人云:望聞問切,神聖功巧。學生三輩門裏出身,先問病,後看脉,還要觀其氣色。就如同子平兼五星,還要觀手相貌纔看得准,庶乎不差!」何老人道:「旣是如此,請先生進看去。」西門慶即令琴童後邊説去:「又請了趙先生來了。」

不一時,西門慶陪他進入李瓶兒房中。那李瓶兒方纔睡下,安逸一囬,又搊扶起來,靠着枕褥坐着。這趙太醫先診其左手,次診右手,便敎老夫人擡起頭來,看看氣色。那李瓶兒眞個把頭兒揚起來。趙太醫敎西門慶:「老爹,你問聲老夫人,我是誰?」西門慶便問李瓶兒:「你看這位是誰?」那李瓶兒擡頭看了一眼,便低聲説道:「他敢是太醫。」趙先生道:「老爹,不妨事,死不成,還認的人哩!」西門慶笑道:「趙先生你用心看,我重謝你。」一面看視了半日,説道:「老夫人此病,休怪我説:據看其面色,又診其脉息,非傷寒則為雜症,不是産後,定然胎前。」西門慶道:「不是此疾。先生,你再仔細診一診。」先生道:「敢是飽悶傷食,飲饌多了?」西門慶道:「他連日飯食,通不十分進。」趙先生又道:「莫不是黄病?」西門慶道:「不是。」趙先生道:「不是,如何面色這等黄?」又道:「多管是脾虚泄瀉。」西門慶道:「也不是泄疾。」趙先生道:「不泄瀉,卻是甚麽?怎生的害個病也敎人摸不着頭腦!」坐想了半日,説道:「我想起來了。不是便毒魚口,定然是經水不調匀。」西門慶道:「女婦人,那裏便毒魚口來?你説這經事不調,倒有些近理。」趙先生道:「南無佛耶,小人可怎的也猜着一樁兒了!」西門慶問:「如何經事不調匀?」趙先生道:「不是乾血癆,就是血山崩。」西門慶道:「實説與先生,房下如此這般,下邊月水淋漓不止,所以身上都瘦弱了。你有甚急方,合些好薬與他喫,我重重謝你。」趙先生道:「不打緊䖏,小人有薬。等我到前邊寫出個方來,好配薬去。」西門慶一面同他來到前廳。喬大户何老人還未去,問他:「甚麽病源?」趙先生道:「依小人講,只是經水淋漓。」何老人道:「當用何薬以治之?」趙先生道:「我有一妙方,用着這幾味薬材,喫下去,管情就好。聽我説:

\begin{myquote}
「甘草甘遂與碙砂,藜蘆巴豆與芫花。人言調着生半夏,用烏頭杏仁天麻。這幾味兒齊加,葱蜜和丸只一撾,清晨用燒酒送下。」
\end{myquote}

何老人聽了,便道:「這等薬喫了,不薬殺人了?」趙先生道:「自古毒薬苦口利於病。若早得摔手伶俐,強如只顧牽纏。」西門慶道:「這廝俱是胡説。」敎小廝:「與我扠出去!」喬大户道:「夥計旣擧保來一場,醫家休要空了他。」西門慶道:「旣是恁説,前邊舖子裏稱二錢銀子,打發他去罷。」那趙太醫得二錢銀子往家,一心忙似箭,兩脚走如飛。

西門慶見打發趙太醫去了,因向喬大户説:「此人原來不知甚麽。」何老人道:「老拙適纔不敢説。此人東門外有名的趙搗鬼,專一在街上賣杖搖鈴,哄過往之人。他那裏曉的甚脉息病源。」因説:「老夫人此疾,老拙到家撮兩貼薬來。遇緣,若服畢經水少減,胸口稍開,就好用薬:只怕下邊不止,飲食再不進,就難為矣!」説畢起身。

西門慶這裏封白金一兩,使玳安㧱盒兒討將薬來,晚夕與李瓶兒喫了,並不見其分毫動靜。吳月娘道:「你也省可裏與他薬喫。他飲食先阻住了,肚腹中有甚麽兒?只顧㧱薬淘淥他。前者那吳神僊算他二十七歲有血光之災,今年卻不整廿七歲了?你還使人尋這吳神僊去,教替他打算算,這祿馬數上看如何。只怕犯着甚麽星辰,替他禳保禳保。」西門慶這裏旋差人㧱帖兒往周守備府裏問去。那裏説:「吳神僊雲遊之人,來去不定。但來,只在城南土地廟下。今歲従四月裏往武當山去了。要打數算命,眞武廟外有個黃先生,打的好數。一數只要三錢銀子,不上人家門去。一生前後事,都如眼見。」西門慶隨即使陳經濟㧱三錢銀子,逕到北邊眞武廟門首找尋。看黃先生家門上貼着:「妙算先天易數,每命卦金三星。」陳經濟向前作揖,奉上卦金,説道:「有一命,煩先生推算。」説與他八字:「女命,年二十七歲,正月十五日午時。」這黃先生把算子一打,就説:「這女命辛未年,庚寅月,辛卯日,壬午時,理取印綬之格,借四歲行運。四歲己未,十四歲戊午,廿四歲丁巳,三十四歲丙辰。今年流年丁酉,比肩用事,歲傷日干,計都星照命,又犯丧門五鬼,災殺作耗。夫計都者,乃陰晦之星也,其像猶如亂絲而無頭,變異無常。大運逢之,多主暗昧之事,引惹疾病。主正二三七九月病災有損,暗傷財物,小口兇殃。小人所算,口舌是非,主失財物;若是陰人,大為不利。断云:

\begin{myquote}
計都流年臨照,命逢陸地行舟。

必然家主皺眉頭,切記胎前産後。

靜裏躊躇無奈,閒中悲慟無休。

女人犯此問根由:必似亂絲不久。 
\end{myquote}

其數曰:

\begin{myquote}
莫道成家在晚時,止緣父母早先離。

芳姿嬌媚生來羙,百計周全更可思。

傳揚伉儷當龍至,應合屠羊看虎威。

可憐情熱因情失,命入鷄宫葉落裏。」
\end{myquote}

打畢數,卦付與經濟㧱來家。西門慶正和應伯爵溫秀才坐的,見經濟抄了數來,㧱到後邊解説與月娘聽,命中多兇少吉。西門慶不聽便罷,聽了眉頭搭上三黃鎖,腹内包藏萬斛愁。正是:

\begin{myquote}
高貴青春遭夭喪,伶俐惺然卻受貧。

年月日時該載定,算來由命不由人。
\end{myquote}

畢竟未知後來如何,且聽下囬分解。

