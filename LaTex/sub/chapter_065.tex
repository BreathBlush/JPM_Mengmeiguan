\includepdf[pages={129,130},fitpaper=false]{tst.pdf}
\chapter*{第六十五囬 \\吳道官迎殯頒眞容 宋御史結豪請六黃}
\addcontentsline{toc}{chapter}{第六十五囬 吳道官迎殯頒眞容 宋御史結豪請六黃}
\markboth{{\titlename}卷之七}{第六十五囬 吳道官迎殯頒眞容 宋御史結豪請六黃}


\begin{myquote}
齊眉相見喜柔和,誰料參商發浩歌。

殘月雲邊懸破鏡,流光機上擲飛梭;

愁隨草色春深謝,苦入蓮心夜幾何。

試問流乾多少淚,楓林秋色一般多。
\end{myquote}

話説到九月二十八日,李瓶兒死了二七光景,玉皇廟吳道官受齋,請了十六個道衆,在家中揚旛修建青玄救苦二七齋壇。早修之時,有官安郎中來下書。西門慶管待來人去了。吳道官廟中擡了三牲祭器、湯飯簇盤餅饊素食、金銀錠香紙之類,又是一疋尺頭,以為奠儀。道衆遶棺轉咒,吳道官靈前展拜。西門慶與經濟回禮,謝道:「師父多有破費,何以克當?」吳道官道:「小道甚是惶愧,本當該助一經,追薦夫人,爭奈力薄。粗茶飯奠,表意而已,望乞大人笑納。」西門慶祭畢,即收了,打發擡盒人囬去。那日三朝轉經,演生神章,破九幽獄,對靈攝召,拜進救苦朱表,頒告諸眞符命,整做法事,俱不必細說。

第二日,先是門外韓姨夫家來上祭。那時孟玉樓兄弟孟锐,外邊做買賣去了五六年沒來家,昨至是來家,見他姐姐西門慶這邊有喪事,跟隨韓姨夫那邊來上祭,討了一分孝去,送了許多人事。見西門慶叙禮,進入玉樓房中拜見。至者堂客約有十數位人。西門慶這邊亦設席管待,俱不在言表。那日午間,又是本縣知縣李拱極,縣丞錢成,主簿任廷貴,典吏夏恭基,又有陽谷縣知縣狄斯彬,共五員官,都鬦了分,穿孝服來上紙帛弔問。西門慶備席在捲棚内管待,請了吴大舅與溫秀才相陪,三個小優兒彈唱。馬上人俱有攢盤,領下去自有坐喫處。

正飲酒到熱鬧䖏,當時沒巧不成話,忽報:「管磚廠工部黃老爹來弔孝。」慌的西門慶連忙穿孝衣靈前伺候。溫秀才又早迎接至大門外,讓至前廳,換了衣裳,跟従進來。家下人手捧香燭、紙帛、金緞,到靈前,用紅漆丹盤捧過香來,跪下。黃主事上了香,展拜畢。西門慶同經濟下來還禮。黃主事道:「學生不知尊閫沒了,弔遲,恕罪恕罪!」西門慶道:「學生一向欠恭,今又承老先生枉弔,兼辱厚儀,不勝感激。」敍畢禮,讓至棚内上面坐下,西門慶與溫秀才下邊相陪。左右捧茶上來。喫了茶,黄主事道:「昨日宋松原多致意先生,他也聞知令夫人作古,也要來弔問,爭奈有許多事情覊絆。他如今在濟州住劄。先生還不知,朝廷如今營建艮嶽,勅旨令太尉朱勔往江南湖湘採取花石綱,運船陸續打河道中來,頭一運將次到淮上。又欽差殿前六黄太尉來迎取卿雲萬態奇峯,長二丈,闊數尺,都用黃毡蓋覆,張打黄旗,費數號船隻由山東河道而來。况河中沒水,起八郡民夫牽挽。官吏倒懸,民不聊生。宋道長督率州縣,事事皆親身經歷,案牘如山,晝夜勞苦,通不得閒。况黄太尉不久自京而至,宋道長説,必須率三司官員,要接他一接。想此間無可相熟者,委托學生來,敬煩尊府作一東,要請六黃太尉一飯,未審尊意可允否?」因喚左右:「呌你宋老爹承差上來。」有二青衣官吏跪下,毡包内捧出一對金緞,一根沉香,兩根白蠟,一分綿紙。黃主事道:「此乃宋公致賻之儀。那兩封是兩司八府官員辦酒分資:兩司官十二員,每員三兩;府官八員,每員五兩;計二十二分,共一百零六兩。」交與西門慶:「有勞盛使一備之,何如?」西門慶再三辭道:「學生有服在家,奈何奈何!」因問:「迎接在於何時?」黃主事道:「還早哩,也得到出月半頭。黃太監京中還未起身。」西門慶道:「學生十月十二日纔發引,旣是宋公祖老先生吩咐,敢不領命?又兼謝盛儀賻禮且領下,分資決不敢收。該多少桌席,只顧吩咐,學生無不畢具。」黃主事道:「四泉此意差矣。松原委托學生來煩凟,此乃山東一省各官公禮,又非松原之己出,何得見却?如其不納,學生即囬松原,再不敢煩凟矣。」西門慶聽了此言,說道:「學生權且領下。」因令玳安王經接下去。問備多少桌席,黄主事道:「六黄備一張喫看大桌面,宋公與兩司都是平頭桌席。以下府官,散席而已。承應楽人,自有差撥伺候,府上不必再呌。」說畢,茶湯兩換,作辭起身。西門慶款留,黄主事道:「學生還到尚柳塘老先生那裏拜拜。他昔年曾在學生敝䖏作縣令,然後轉成都府推官。如今他令郎兩泉又與學生鄉試同年。」西門慶道:「學生不知老先生與尚兩泉相厚,兩泉亦與學生相交。」黄主事起身。西門慶道:「煩老先生多致意宋公祖,至期寒舍拱候矣。」黄主事道:「臨期松原差人來通報,先生亦不可太奢。」西門慶道:「學生知道。」送出大門,上馬而去。

那縣中官員,聽見黄主事帶領巡按上司人來,唬的都躲在山子下小捲棚内飲酒,吩咐手下把轎馬藏過一邊。當時西門慶回到捲棚,與衆官相見,具說宋巡按率兩司八府,來央煩出月迎請六黄太尉之事。衆官悉言:「正是州縣不勝憂苦這件事!欽差若來,凡一應祇迎、廩餼公宴、器用人夫,無不出於州縣,必取之於民,公私困極,莫此為甚。我輩還望四泉於上司䖏羙言提拔,足見厚愛之至。」言訖,都不久坐,告辭起身,上馬而去。

話休饒舌。到李瓶兒三七,有門外永福寺道堅長老,領十六衆上堂僧來念經。穿雲錦袈裟,戴毘盧帽,大鈸大鼓。早晨取水,轉五方,請三寳,浴佛;午間加持召亡破獄,禮拜《梁皇懺》,談《孔雀》,甚是齊整。晚夕喬大户娘子與衆夥計娘子與月娘等伴宿,在靈前看偶戲。西門慶與應伯爵、吴大舅、溫秀才,在棚内東首另設圍屏飲酒。

十月初八日是四七,請西門外寳慶寺趙喇嘛,亦十六衆,來念番經,結壇,跳沙,灑花米,行香,口誦眞言,齋供都用牛乳茶酪之類。懸掛都是九醜天魔變相,身披纓絡琉璃,項掛髑髏,口咬嬰兒,坐跨妖魅,腰纏蛇螭,或四頭八臂,或手執戈戟,朱髮藍面,醜惡莫比。午齋已後,就動葷酒。西門慶那日不在家,同陰陽徐先生往門外墳上破土開壙去了,後晌方囬。晚夕打發喇嘛散了。次日推運山頭酒米桌面餚品,一應所用之物。又委付主管夥計,莊上前後搭棚,四五䖏酒房廚坊,墳内穴邊,又起三間罩棚。先請附近地隣來坐席面,大酒大肉管待。臨散,皆肩背項負而歸,俱不必細說。

十一日白日,先是歌郎並鑼鼓地弔來靈前參靈,弔《五鬼鬧判》、《張天師着鬼迷》、《鍾馗戲小鬼》、《老子過函関》、《六賊鬧彌勒》、《雪裏梅》、《莊周夢蝴蝶》、《天王降地水火風》、《洞賓飛劍斬黃龍》、《趙太祖千裏送荆娘》,各樣百戲弔罷,堂客都在簾内觀看。參罷靈去了,内眷親戚,都來辭靈燒紙,大哭一場。到次日發引,先絶早擡出銘旌,各項旛亭紙劄。僧道鼓手,細楽人役,都來伺候。西門慶預先問帥府周守備討了五十名巡捕軍士,都帶弓馬,全裝結束。留十名在家看守,四十名跟殯,在材前擺馬道,分兩翼而行。衙門裏又是二十名排軍打路,照管冥器。墳頭又是二十名把門,管收祭祀。那日官員士夫,親隣朋友,來送殯者,車馬喧呼,填街塞巷。本家並親眷堂客轎子也有百十餘頂;三院鴇子粉頭,小轎也有數十。徐陰陽擇定辰時起棺。西門慶留下孫雪娥並二女僧看家,平安兒同兩名排軍把前門。那女婿陳經濟,跪在柩前摔盆。六十四人上扛,有仵作一員,高立於增架上,敲響板,指撥擡材人上肩。先是請了報恩寺朗僧官來起棺,剛轉過大街口望南走,那兩邊觀看的人山人海。那日正値晴明天氣,果然好殯!但見:

\begin{myquote}
和風開綺陌,細雨潤芳塵。東方曉日初升,北陸殘煙乍歛。鼕鼕嚨嚨,出丧鼓不住聲喧;叮叮噹噹,地吊鑼連宵振作。銘旌招颭,大書九尺紅羅;起火軒天,中散半空黄霧。猙猙獰獰開路鬼,斜擔金斧;忽忽洋洋險道神,端秉銀戈。逍逍遙遙八洞僊,龜鶴遶定;窈窈窕窕四毛女,虎鹿相隨。地弔鬼,晃一片鑼篩;煙火架,迸千枝花炮。熱熱鬧鬧採蓮船,撒科打諢;長長大大高蹺漢,貫甲頂盔。清清秀秀小道童十六衆,衆衆都是霞衣道髻,擊坤庭之金,奏八琅之璈,動一派之僊音;肥肥胖胖大和尚二十四個,個個都是雲錦袈娑,排大鈸,敲大鼓,轉五方之法事。一十二座大絹亭,亭亭皆綠舞紅飛;二十四座小絹亭,座座盡珠圍翠繞。左勢下,天倉與地庫相連;右勢下,金山與銀山作隊。掌醢廚,列八珍之罐;香燭亭,供三獻之儀。六座百花亭,現千團錦綉;一乘引魂轎,扎百結黃絲。這邊荷花與雪柳爭輝,那邊寳蓋與銀幢作隊。金字旛、銀字旛,緊護棺輿;白絹繖,綠絹繖,同圍增架。斧符雲氣,一邊三把皆彩畫鮮明;執罐捧巾,兩下侍妾盡梳粧如活。功布招颭,孝眷聲哀,簇捧定五出頭、六歌郎、仰覆運須彌座;六十四名青衣白帽,穩穩擡定五老雲鶴華蓋頂、四垂頭流蘇帶、大紅銷金寳象花棺罩,裏面安着巍巍不動錦綉棺輿。只見那兩邊打路排軍,個個都頭戴孝巾,身穿青衲襖,腰繫孝帶,脚靸腿繃䩺鞋,手執欄杆,前呼後擁。兩邊走解的,頭戴芝蔴羅萬字頭巾,撲匾金環飛於腦後,穿的是兩三領紵絲衲襖,腰繫紫纏帶,足穿鷹爪四縫乾黄靴,襯着五彩翻身搶水獸納紗襪口,賣解猶如鷹鷂,走馬好似猿猴:執着一桿明鎗,題硃紅桿令字藍旗,豎肩樁,打斤斗,隔肚穿錢,金鷄獨立,僊人打過橋,鐙裏藏身。人人喝采,個個爭誇。扶肩擠背,紛紛不辨賢愚;挨覩並觀,攘攘那分貴賤。張三蠢胖,只把氣吁;李四矮矬,頻將脚躧。白頭老叟,盡將拐捧拄髭鬚;綠鬢佳人,也帶兒童來看殯。正是:

鑼鼓鼕鼕靄路塵,花攢錦簇萬人瞻。

哀聲隱隱棺輿過,此殯誠然壓帝京。
\end{myquote}

吳月娘坐大轎在頭裏,後面李嬌兒等本家轎子十餘頂,一字兒緊跟材後走。西門慶總冠孝衣,同衆親朋在材後裏,陳經濟緊扶棺輿。走出東街口,西門慶具禮請玉皇廟吳道官來懸眞。身穿大紅五彩雲霞二十四鶴鶴氅,頭戴九陽玉環雷巾,脚蹬丹舄,手執牙笏,坐在四人肩輿上,迎殯而來,將李瓶兒大影捧於手内。陳經濟跪在面前,那殯停住了。衆人聽他在上高聲宣念:

\begin{myquote}
「兔走烏飛西復東,百年光景似風燈。

時人不悟無生理,到此方知色是空。

恭惟

故錦衣西門恭人李氏之靈,存日陽年二十七歲,元命辛未相正月十五日午時受生,大限於政和七年九月十七日丑時分身故。伏以尊靈:名家秀質,綺閣嬌姝。禀花月之儀容,蘊蕙蘭之佳氣。鬱德柔婉,賦性溫和。配我西君,克諧伉儷。處閨門而賢淑,資琴瑟以好和。曾種藍田,尋嗟楚畹。正宜享福百年,可惜春光三九。嗚呼!明月易缺,好物難全。善類無常,修短有數。今則棺輿載道,丹旆迎風,良夫躃踊於柩前,孝眷哀矜於巷陌。離別情深而難已,音容日遠以日忘。某等謬忝冠簪,愧領玄教,愧無新垣平之神術,恪遵玄元始之遺風。徒展崔徽畫裏之容,難返莊周夢中之蝶。漱甘露而沃瓊漿,超僊識登於紫府;披百寶而面七眞,引淨魄出於冥途。一心無挂,四大皆空。苦苦苦,氣化清風形歸土。一靈眞性去弗迴,改頭換面無遍數。衆聽末後一句,咦!精爽不知歸何處,眞容留與後人傳。」
\end{myquote}

吴道官念畢,端坐轎上,那轎捲坐退下去了。這裏鼓楽喧天,哀聲動地,殯纔起身,迤邐出南門。衆親朋陪西門慶走至門上,方乘馬。陳經濟扶柩,到於山頭五里原。原來坐營張團練帶領二百名軍,同劉薛二内相,又早在墳前高阜處搭帳房,吹響器,打銅鑼銅鼓,迎接殯到。看着裝燒冥器紙劄,煙焰漲天。墳内有十數家收頭祭祀,皆兩院妓女擺列。堂客内眷,自有幃幕。棺輿到,落下扛,徐先生率領仵作,依羅經弔向,巳時祭告后土方隅後,纔下葬掩土。西門慶易服,備一對尺頭,禮請帥府周守備點主。祭畢,衛中官員並衆親朋夥計,皆爭拉西門慶遞酒。鼓楽喧天,煙火匝地。收祭祀者,自有所管人役,再無淆亂。那日待人齋堂,也有四五處。堂客在後捲棚内坐,各有派定人數。熱鬧豐盛,不必細說。喫畢,各又邀去莊院,設席請西門慶收頭飲酒,賞賜亦費許多。

後晌回靈,吳月娘坐魂轎,抱神主魂旛,陳經濟扶靈牀——都是玄色紵絲靈衣,玉色銷金走水,四角垂流蘇。弔挂大影亭、大絹亭、小絹亭、香燭亭,鼓手細楽,十六衆小道童兩邊吹打。吳大舅並喬大户、吴二舅、花大舅、沈姨夫、孟二舅、應伯爵、謝希大、溫秀才,衆主管夥計,都陪着西門慶進城。堂客轎子壓後。到家門首,燎火而入。李瓶兒房中安靈已畢,徐先生前廳祭神洒掃,各門户皆貼辟非黄符。管待徐先生,備一疋尺頭,五兩銀子,相謝出門。各項人役,打發散了。㧱出二十吊錢來,五吊賞巡捕軍人,五吊與衛中排軍,十吊賞營裏人馬。㧱帖兒回謝周守備、張團練、夏提刑,俱不在話下。西門慶還令左右放桌,留喬大户吴大舅衆人坐。衆人都不肯,作辭起身。來保囬説:「搭棚的在外伺候,明日來拆棚。」西門慶道:「棚且不消拆,一發過了你宋老爹擺酒日子來拆罷。」打發搭綵匠去了。後邊花大娘子與喬大户娘子、衆堂客,還等着安畢靈,哭了一場,方纔去了。

西門慶不忍遽捨,晚夕還來李瓶兒房中,要伴靈宿歇。見靈牀安在正面,大影挂在傍邊,靈牀内安着半身,裏面小錦被褥牀几衣服粧奩之類,無不畢具;下邊放着他的一對小小金蓮,桌上香花燈燭,金碟樽俎,般般供養,西門慶大哭不止。令迎春就在對面炕上搭鋪。到半夜,對着孤燈,半窗斜月,翻覆無寐,長吁短歎,思想佳人。有詩為證:

\begin{myquote}
短歎長吁對彼窗,舞鸞孤影寸心傷。

蘭枯楚畹三秋雨,楓落吳江一夜霜。

夙世已違連理願,此生難覓返魂香。

九泉果有精靈在,地下人間兩断腸。
\end{myquote}

白日間供養茶飯,西門慶在房中親看着丫鬟擺下,他便對面桌兒和他同喫,擧起筯兒來:「你請些飯兒!」行如在之禮。丫鬟養娘都忍不住掩淚而哭。奶子如意兒,無人䖏常在跟前遞茶遞水,挨挨搶搶,掐掐捏捏,插話兒應答。那消三夜兩夜,這日,西門慶請了許多官客堂客,並院中李桂姐、吴銀兒、鄭月兒,三個唱的,李銘、吳惠、鄭奉、鄭春,四名小優兒,墳上暖墓,回家。西門慶因陪人喫得醉了,進來,迎春打發歇下。到夜間要茶喫,叫迎春不應。如意兒起來遞茶,因見被拖下炕來,接過茶盞,用手扶起被。西門慶一時興動,摟過脖子就親了個嘴,遞舌頭在他口内。老婆就咂起來,一聲兒不言語。西門慶令脫去衣服上炕,兩個摟接在被窝内,不勝歡娱,雲雨一處。老婆説:「旣是爹擡擧,娘也没了,小媳婦情願不出爹家門,隨爹收用便了。」西門慶便呌:「我兒,你只用心伏侍我,愁養活不過你來?」當下這老婆枕席之間無不奉承,顚鸞倒鳳,隨手而轉,把西門慶歡喜了不的。次日,老婆早晨起來,與西門慶㧱鞋脚疊被褥,就不靠迎春,極盡殷勤,無所不至。西門慶開門,尋出李瓶兒四根簪兒來賞他。老婆磕頭謝了。迎春亦知收用了他,兩個打成一路。老婆自恃得寵,脚跟已牢,無復求告於人,就不同往日,打扮喬模喬樣,在丫鬟夥兒内説也有,笑也有,早被潘金蓮看到眼裏。

早晨,西門慶正陪應伯爵坐的,忽報宋御史老爹差人來送答賀黄太尉一桌金銀酒器:兩把金壺,兩副金臺盞,十副小銀鍾,兩副銀折盂,四副銀賞鍾,兩疋大紅彩蟒,兩疋金緞,十罈酒,兩牵羊。傳報:「太尉船隻,已到東昌地方,煩老爹這裏早先預備酒席,准在十八日迎請。」西門慶收入明白,與了來人一兩銀子,折柬打發回去。隨即兑銀與賁四、來興兒,定桌面,粘菓品,買辦整理,不必細說。因向應伯爵說:「自從他不好起到而今,我再沒一日兒心閒。剛剛打發喪事兒出去了,又鑽出這等勾當來,教我手忙脚亂。」伯爵道:「這個哥不消抱怨,你又不曾掉攬他,他上門兒來央煩你。雖然你這席酒替他賠幾兩銀子,到明日,休説朝廷一位欽差、殿前大太尉來咱家坐一坐,只這山東一省官員,並巡撫、巡按、人馬散級,也與咱門户添許多光輝,壓好些仗氣。」西門慶道:「不是此說。我承望他到二十以外也罷,不想十八日就迎接,忒促急促忙。這十六日又是他五七,我前日已與了吴道官冩法銀子去了,如何又改?不然雙頭火杖,都擠在一䖏,怎亂得過來?」應伯爵道:「這個不打緊,我算來,嫂子是九月十七日沒了,此月二十一日正是五七。你十八日擺了酒,二十日與嫂子念經也不遲。」西門慶道:「你説的是了,我如今就使小廝回吴道官改日子去。」伯爵道:「哥,我又一件。如今趂着東京黄眞人在廟裏住,朝廷差他來泰安州進金鈴吊挂御香,建七晝夜羅天大醮。趁他未起身,倒好敎吳道官請他那日來做高功,領行法事。咱圖他這個名聲也好看。」西門慶道:「只説這黄眞人有道行,少不的那日全堂添二十四衆道士,做一晝夜齋事。爭奈吴道官齋日受他祭禮,出殯又起動他懸眞、道童送殯,沒的酬謝他,敎他念這個經兒表意而已。今又請黄眞人主行,卻不難為他?」伯爵道:「齋一般還是他受,只教他請黄眞人做高功就是了。哥只是多費幾兩銀子,為嫂子,沒曾為了别人。」西門慶一面教陳經濟寫帖子,又多封了五兩銀子寫法,敎他早請黄眞人,改在二十日念經,二十四衆道士,水火煉度一晝夜。即令玳安,騎頭口囬去了。

西門慶打發伯爵去訖,進入後邊,只見吳月娘說:「賁四嫂買了兩個盒兒,他女兒長姐定與人家,來磕頭。」西門慶便問:「誰家?」賁四娘子穿着藍紬襖兒,白絹裙子,青緞披襖;他女兒穿着大紅緞襖兒,黄紬裙子,戴着花翠,插燭向西門慶磕了四個頭。月娘在傍說:「咱也不知道。原來這孩子,與了夏大人房裏擡擧,昨日纔相定下,這二十四日就娶過門,只得了他三十兩銀子。論起來,這孩子倒也好身量,不像十五歲,倒有十六七歲的。多少時不見,就長的成成的!」西門慶道:「他前日在酒席上和我說,要抬擧兩個孩子學彈唱。不知你家孩子與了他。」於是敎月娘讓在房内,擺茶留坐。落後李嬌兒、孟玉樓、潘金蓮、孫雪娥、大姐,都來見禮陪坐。臨走,西門慶月娘與了一套重絹衣服,一兩銀子,李嬌兒衆人都有與花翠、汗巾、脂粉之類。晚上玳安囬話:「吴道官收了銀子,知道了。黄眞人還在廟裏住,過二十頭纔回東京去,十九日早來鋪設壇場。」

西門慶次日家中廚役落作治辦酒席,務要齊整。大門上扎七級彩山,廳前五級彩山。十七日,宋御史差委兩員縣官來觀看筵席。廳正面屏開孔雀,地匝氍毹。都是錦綉桌幃,粧花椅墊。黄太尉便是肘件大飯簇盤、定勝方糖、五老錦豐、堆高頂喫看大插桌,關席兩張小插桌,是巡撫巡按陪坐。兩邊布按三司,有桌席列坐;其餘八府官,都在廳外棚内兩邊,只是五菓五菜平頭桌席。看畢,西門慶待茶,起身回話去了。

到次日,撫按率領多官人馬,早迎到船上,張打黄旗「欽差」二字,捧着勅書在頭裏走。地方統制、守禦、都監、團練,各衛掌印武官,皆戎服甲胄,各領所部人馬尾隨,藍旗纓鎗,叉槊儀杖,擺數里之遠。黄太尉穿大紅五彩雙挂綉蟒,坐八擡八簇銀頂暖轎,張打茶褐傘,後邊名下執事人役跟隨無數,皆駿騎咆哮,如萬花之燦錦,隨路鼓吹而行,黄土墊道,鷄犬不聞,樵採遁迹。人馬過東平府,進清河縣,縣官黑壓壓跪於道傍迎接,左右喝叱起去。隨路傳報,直到西門慶家中大門首。敎坊鼓楽,聲震雲霄,兩邊執事人役,皆青衣排伏,雁翅而列。西門慶青衣冠冕,望塵拱伺。良久,人馬過盡,太尉落下轎進來,後面撫按率領大小官員,一擁而入,到於廳上,廳上又是箏ぬ、方響,雲璈、龍笛、鳳管,細楽響動。為首就是山東巡撫都御史侯蒙、巡按監察御史宋喬年參見,太尉還依禮答之。其次就是山東左布政龔共、左參政何其高、右布政陳四箴、右參政季侃、左參議馮廷鵠、右參議汪伯彥、廉訪使趙訥、採訪使韓文光、提學副使陳正彙、兵備副使雷起元等兩司官參見,太尉稍加優禮。及至東昌府徐崧、東平府胡師文、兖州府凌雲翼、徐州府韓邦奇、濟南府張叔夜、青州府王士奇、登州府黄甲、萊州府葉遷等八府官行廳參之禮,太尉答以長揖而已。至於統制、制置、守禦、都監、團練等官,太尉則端坐。各官聽其發放,各人外邊伺候。然後西門慶與夏提刑上來拜見獻茶,侯巡撫、宋巡按向前把盞。下邊動鼓楽來與太尉簪金花、捧玉斝,彼此酬飲。遞酒已畢,太尉正席坐下,撫按下邊主席,其餘官員並西門慶等各依次第坐了。敎坊伶官遞上手本奏楽,一應呈應彈唱隊舞囬數,各有節次,極盡聲容之盛。當宴搬演的《裴晋公還帶記》,一摺下去,廚役割獻燒鹿花猪,百寳攢湯,大飯燒賣。又有四員伶官,箏、ぬ、琵琶、箜篌,上來清彈小唱,唱了一套〔南吕·一枝花〕:

\begin{myquote}
「官居八輔臣,祿享千鍾近。功存遺百世,名播萬年春。拯溺亨迍,惟治國安邦論,調和鼎鼐新。持義節、率忠貞,都則待報主施恩;乘賢烈、秉正直,也則是清懲化民。」
\end{myquote}

唱畢,湯未兩陳,楽已三奏。下邊跟従執事官身人等,宋御史委差兩員州官,在西門慶捲棚内自有桌席管待。守禦都監等官,西門慶都安在前邊客位,自有坐處。黄太尉令左右㧱十兩銀子來賞賜各項人役,隨即看轎,就要起身。衆官上來再三款留不住,都送出大門。鼓楽笙簧迭奏,兩街儀衛喧闐,清蹕傳道,人馬森列。多官俱上馬遠送,太尉悉令免之,擧手上轎而去。宋御史、侯巡撫,吩咐都監以下軍衛有司,直護送至皇船上來回話。桌面器皿答賀羊酒,具手本差東平府知府胡師文與守禦周秀,親送到船所交割明白。囬至廳上,拜謝西門慶說:「今日不當負累取擾華府,深感深感!分資有所不足,容當奉補。」西門慶慌躬身施禮道:「學生屢承敎愛,累辱盛儀,日昨又蒙賻禮,些小微物,何足挂齒?蜗居卑陋,猶恐有不到䖏,萬望公祖諒宥,幸甚!」宋御史謝畢,即令左右看轎,與侯巡撫一同起身。兩司八府官員皆拜辭而去,各項人役一鬨而散。

西門慶囬至廳上,將伶官楽人賞以酒食,俱令散了,止留下四名官身小優兒伺候。廳内外各官桌面自有本官手下人領,不題。西門慶見天色尚早,收拾家伙停當,攢下四張桌席,佳餚堆滿,使人請吴大舅、應伯爵、謝希大、溫秀才、傅自新、甘出身、韓道國、賁四、崔本,及女婿陳經濟,——従五更起來,各項照管辛苦,坐飲三盃。不一時,衆人來到。吴大舅與溫秀才、應伯爵、謝希大,居上坐,西門慶関席,衆夥計兩邊列坐,左右擺上酒來飲酒。伯爵道:「哥今日落忙,黄太尉坐了多大一回,喜歡不喜歡?」韓道國道:「今日六黄老公公見咱家酒席齊整,無個不喜歡的。巡撫巡按兩位甚是知感不盡,謝了又謝。」伯爵道:「若是第二家擺這席酒也成不的,也沒咱家恁大地方,也沒府上這些人手。今日少說也有上千人進來,都要管待出去。哥就賠了幾兩銀子,咱山東一省也響出名去了。」溫秀才道:「學生宗主提學陳老先生也在這裏預席。」西門慶問其故。溫秀才道:「名陳正彙者,乃諫垣陳了翁先生乃郎,本貫河南鄄城縣人,十八歲科舉,中壬辰進士。今任本處提學副使,極有學問。」西門慶道:「他今年纔二十四歲。」正說着,湯飯上來,衆人喫畢。西門慶呌上四個小優兒,問道:「你四人呌甚名字?」答道:「小的叫周采、梁鐸、馬眞、韓畢。」伯爵道:「你不是韓金釧兒一家?」韓畢跪下說:「金釧兒、玉釧兒,都是小的妹子。」西門慶問:「你們喫了酒飯不曾?」周采道:「小的剛纔都喫過酒飯了。」西門慶一囬想起李瓶兒來,今日擺酒,就不見他,吩咐小優兒:「你們㧱楽器過來,會唱『洛陽花梁園月』不會?唱一個我聽。」韓畢跪下:「小的與周采記的。」一面搊箏撥阮,板排紅牙,唱道:

\begin{myquote}
{\markfont〔普天楽〕}「洛陽花,梁園月。好花湏買,皓月須賒。花倚欄杆看爛熳開,月曾把酒問團圝夜。月有盈虧,花有開謝,想人生最苦離別。花謝了,三春近也;月缺了,中秋到也;人去了,何日來也!」
\end{myquote}

唱畢,應伯爵見西門慶眼裏酸酸的,便道:「哥,別人不知你心,只我略知一二。哥教唱此詞,関係心間之事,莫非想起過世嫂子來?就如同連理枝、比目魚,今分爲兩下,心中怎不想念!」西門慶看見後邊上來菓碟兒,呌:「應二哥,你只嗔我説。有他在,就是他經手整定;従他沒了,隨着丫鬟掇弄,你看都像甚模樣?好應口菜也沒一根我喫。」溫秀才道:「這等盛設,老先生中饋也不謂無人,足可以够了。」伯爵道:「哥休説此話。你心間疼不過,便是這等説。恐一時冷淡了别的嫂子們心。」這裏酒席上説話,不想潘金蓮在軟壁後聽唱,聽見西門慶說此話,走到後邊,一五一十告訴月娘。月娘道:「隨他説去就是了,你如今却怎樣的!前日是不是,他在時即許下把綉春敎伏侍他二娘,他倒睜着眼和我呌:『死了許多時兒,就分散他房裏丫頭?』敎我就一聲兒再沒言語。這兩日你看他那媳婦子和兩個丫頭,狂的有些樣兒!我但開口,就説咱們擠撮他。」金蓮道:「娘,我也見這老婆這兩日有些别模改樣的。怕這賊沒廉耻貨,鎭日在那屋裏纏,要了這老婆也不定的。我聽見說,前日與了他兩對簪子,老婆戴在頭上,㧱與這個瞧,㧱與那個瞧。」月娘道:「荳芽菜兒,有甚綑兒!」衆人背地裏都不做喜歡。正是:遺踪堪入時人眼,不買胭脂畫牡丹。有詩為證:

\begin{myquote}
襄王臺下水悠悠,一種相思兩地愁。

月色不知人事改,夜深還照粉牆頭。
\end{myquote}

畢竟不知後來如何,且聽下囬分解。

