\includepdf[pages={83,84},fitpaper=false]{tst.pdf}
\chapter*{第四十二囬 \\豪家攔門玩煙火 貴客高樓醉賞燈}
\addcontentsline{toc}{chapter}{第四十二囬 豪家攔門玩煙火 貴客高樓醉賞燈}
\markboth{{\titlename}卷之五}{第四十二囬 豪家攔門玩煙火 貴客高樓醉賞燈}


\begin{myquote}
星月當空萬燭燒,人間天上兩元宵。

楽和春奏聲偏好,人蹈夜歸馬亦嬌。

易老韶光休浪度,最公白髮不相饒。

千金博得斯須刻,吩咐譙更仔細敲。
\end{myquote}

話説西門慶打發喬家去了,走來上房,和月娘大妗子李瓶兒商議。月娘道:「他家既先來與咱家孩子送節,咱少不的也買禮過去,與他家長姐送節,就權為插定一般,庶不差了禮數。」大妗子道:「咱這裏少不的立上個媒人,往來方便些。」月娘道:「他家是孔嫂兒,咱家安上誰好?」西門慶道:「一客不煩二主,就安上老馮罷。」於是連忙冩了請帖八個,就叫了老馮來,教他同玳安拿請帖盒兒,十五日請喬老親家母、喬五太太,并尚擧人娘子,朱序班娘子、崔親家母、段大姐、鄭三姐,來赴席,與李瓶兒做生日,并喫看燈酒。一面吩咐來興兒拿銀子早往糖餅鋪,早定下蒸酥點心,都用大方盤,要四盤蒸餅:兩盤菓餡團圓餅、兩盤玫瑰元宵餅;買四盤鮮菓:一盤李乾、一盤胡桃、一盤龍眼、一盤荔枝;四盤羹肴:一盤燒鵝、一盤燒鷄、一盤鴿子兒、一盤銀魚乾。又是兩套遍地錦羅緞衣服,一件大紅小袍兒、一頂金絲縐紗冠兒,兩盞雲南羊角珍燈,一盒衣翠,一對小金手鐲、四個金寳石戒指兒。十四日早裝盒擔,教女婿陳經濟和賁四穿青衣服,押送過去。喬大戶那邊,酒筵管待,重加答賀。囬盒中,囬了許多生活鞋脚。俱不必細説。

正亂着,應伯爵來講李智黄四關銀子事,看見問其所以。西門慶告訴與喬大户結親之事:「十五日好歹請令正來陪親家坐的。」伯爵道:「嫂子呼喚,房下必定來。」西門慶道:「今日請衆堂官娘子喫酒,咱們往獅子街房子内看燈去罷。」伯爵應諾去了。不題。

且説那日,院中吳銀兒先送了禮來,買了一盤壽桃、一盤壽麵、兩隻燒鴨、一副豕蹄、兩方銷金汗巾、一雙女鞋,來與李瓶兒上壽,就拜乾女兒相交。月娘收了禮物,打發轎子回去。李桂姐直到次日纔來,見吳銀兒在這裏,悄悄問月娘:「他多咱來了?」月娘如此這般告他説:「昨日送了禮來,拜認你六娘做乾女兒了。」李桂姐聽了,一聲兒沒言語,一日只和吴銀兒使性子,兩個不説話。

卻説前廳有王皇親家二十名小廝唱戲,挑了箱子來,有兩名師父領着,先與西門慶磕頭。西門慶吩咐西廂房做戲房,管待酒飯;堂客到時,吹打迎接。大廳上玳筵齊整,錦茵匝地。先是周守備娘子、荆都監母親荆太太,與張團練娘子先到了,俱是大轎,排軍喝道,家人媳婦跟隨。裏邊月娘衆姊妹都穿着袍出來迎接,至後廳敍禮,與衆親相見畢,讓坐遞茶。等着夏提刑娘子到纔擺茶。不料等到日中,還不見來。小廝邀了兩三遍,約午後時分,纔喝了道來。擡着衣匣,家人媳婦跟隨,許多僕從擁護。鼓楽接進去後廳,與衆堂客見畢禮數,依次序坐下。先在捲棚内擺茶,然後大廳上坐。春梅、玉簫、迎春、蘭香,都是雲髻珠子纓絡兒,金燈籠墜子,遍地錦比甲,大紅緞袍,翠藍織金裙兒,——惟春梅寳石墜子,大紅遍地錦比甲兒,——席上捧茶斟酒。那日,王皇親家楽扮的是《西廂記》。

不説畫堂深䖏,珠圍翠繞,歌舞吹彈飲酒。單表西門慶,那日打發堂客廳裏上茶,就騎馬約下應伯爵、謝希大,往獅子街房裏去了。吩咐四架煙火拿一架那裏去,晚夕堂客跟前放兩架。那裏樓上,設放圍屏桌席,掛上燈。旋叫了個廚子,生了火,家中擡了兩食盒下飯菜蔬、兩壜金華酒,叫了兩個唱的:董嬌兒、韓金釧兒。

原來西門慶先使玳安僱下轎子,請王六兒同往獅子街房裏去。玳安見婦人,道:「爹説請韓大嬸,那裏晚夕看放煙火。」那婦人笑道:「我羞剌剌怎麽好去哩!你韓大叔知道不嗔?」玳安道:「爹對韓大叔説了,教你老人家快收拾哩。若不是,使了老馮來請你老人家。今日各宅衆奶奶喫酒,六娘着他看哥兒,那裏抹嘴去。現爹巴巴使了我來。因叫了兩個唱的,沒人陪他。」那婦人聽了,還不動身。一囬,只見韓道國來家,玳安道:「這不是韓大叔來了?韓大嬸這裏不信我説哩!」婦子向他漢子説:「眞個教我去?」韓道國道:「老爹再三説,兩個唱的沒人陪他,請你過去,晚夕就看放煙火。等你,還不收拾哩!剛纔教我把舖子也收了,就晚夕一搭兒裏坐坐。保官兒也往家去了,晚夕該他上宿哩。」婦人道:「不知多咱纔散,你到那裏坐囬就來罷。家裏沒人,你又不該上宿。」説畢,打扮穿了衣服,玳安跟隨,逕到獅子街房裏。來昭妻一丈青又早將房裏收拾乾淨,牀炕帳幔褥被都是現成的,安息沉香薰的噴鼻香。房裏吊着兩盞紗燈,地平上火盆裏籠着一盆炭火。婦人走到裏面炕上坐下。良久,來昭妻一丈青走出來,道了萬福,拿茶喫了。

西門慶與應伯爵看了囬燈,纔到房子裏,兩個在樓上打雙陸。樓上除了六扇窗户,掛着簾子,下邊就是燈市,十分熱鬧。打了囬雙陸,收拾擺飯喫了,二人在簾裏觀看燈市。但見:
\begin{myquote}
萬井人煙錦綉圍,香車駿馬鬧如雷;

鰲山聳出青雲上,何䖏遊人不看來。
\end{myquote}

伯爵因問:「明日喬家那頭幾位人來?」西門慶道:「有他家做皇親家五太太。明日我又不在家,早晨赶廟中上元醮,又是府裏周南軒那裏請喫酒。」西門慶忽見人叢裏謝希大、祝日念,同一個戴方巾的在燈棚下看燈,指與伯爵瞧,因問:「那戴方巾這個人,你可認的他?如何跟着他一答兒裏走?」伯爵道:「此人眼熟,不認的他。」西門慶便叫玳安:「你去下邊悄悄請了謝爹來,休教祝麻子和那人看見。」玳安小廝眼裏説話,一直走下樓來,挨到人鬧裏,待祝日念和那人先過去了,從旁邊出來把謝希大拉了一把。慌的希大囬身觀看,卻是他。玳安道:「爹和應二爹在這樓上,請謝爹説話。」希大道:「你去,知道了。等陪他兩個到粘梅花處,就去見你爹。」玳安便一道煙去了。

不想到了粘梅花處,這希大向人鬧處就扠過一邊,由着祝日念和那一個人只顧那裏尋他。便走來樓上,見西門慶應伯爵兩個,作揖,因説道:「哥來此看燈,早晨就不説兄弟一聲!」西門慶道:「我早晨對衆人不好邀你們的,已托應二哥到你家請你去,說你不在家。剛纔祝麻子沒看見你這裏來?」因問:「那戴方巾的是誰?」希大道:「那戴方巾的是王昭宣府裏王三官兒。今日和祝麻子到我家,央我問許不與先生那裏借三百兩銀子,央我和老孫祝麻子作保,——要幹前程,入武學肄業。我那裏管他這閒帳!剛纔陪他燈市裏走了走,聽見哥使盛价呼喚,我只伴他到粘梅花䖏,教我乘人亂就扠開了,走來見哥。」因問伯爵:「你來多大囬了?」伯爵道:「哥使我先到你家,你不在,我就來了。和哥在這裏打了這囬雙陸。」西門慶問道:「你喫了飯不曾?叫小廝拿飯來你喫。」謝希大道:「可知好哩!早晨從家裏出來,和他兩個搭了這一日,誰喫飯來?」西門慶吩咐玳安:「廚下安排飯來,與你謝爹喫。」不一時,搽抹桌兒乾淨,就是春盤小菜、兩碗稀爛下飯、一碗𤆑肉粉湯、兩碗白米飯。希大獨自一個喫個裏外乾淨,剩下些汁湯兒,還泡了碗喫了。玳安收下家活去。希大在傍看着兩個打雙陸。

只見兩個唱的,門首下了轎子,擡轎的各提着衣裳包兒,笑進來。伯爵早已在窗裏看見,説道:「兩個小淫婦兒,這咱纔來。」吩咐玳安:「且别教他往後邊去,先叫他樓上來見我。」希大道:「今日叫的是那兩個?」玳安道:「是董嬌兒、韓玉釧兒。」忙下樓説道:「應二爹叫你説話。」兩個那裏肯來,一直往後走了。見了一丈青,拜了,引他入房中。看見王六兒頭上戴着時樣扭心䯼髻兒,羊皮金箍兒;身上穿紫潞紬襖兒,玄色一塊瓦領披襖兒,白挑線絹裙子;下邊顯着趫趫兩隻金蓮,穿老鴉緞子紗綠鎖線的平底鞋兒;描的水鬢長長的,紫膛色,不十分搽鉛粉;學個中人打扮,耳邊帶着丁香兒;進門只望着他拜了一拜,都在炕邊頭坐了。小鐵棍拿茶來,王六兒陪着喫了。兩個唱的上上下下把眼只看他身上,看一回,兩個笑一囬,更不知是什麽人。落後玳安進來,兩個唱的悄悄問他道:「房中那一位是誰?」玳安沒的囬答,只説:「是俺爹大姨人家,接來這看燈。」兩個唱的,進房中従新説道:「俺們頭裏不知是大姨,沒曾見的禮,休怪!」於是插燭磕了兩個頭。慌的王六兒連忙還下半禮。落後擺上湯飯來,陪着同喫。兩個拿楽器又唱與王六兒聽。

伯爵打了雙陸,下樓來小淨手,聽見後邊唱,點手兒叫過玳安,問道:「你告我説,兩個唱的在後邊唱與誰聽?」玳安只是笑,不做聲,説道:「你老人家曹州兵備好管事寬。唱不唱管他怎的?」伯爵道:「好賊小油嘴!你不和我説,愁我不知道?」玳安笑道:「你老人家知道罷了,又問怎的?」説畢,一直往後走了。伯爵上的樓來,西門慶又與謝希大打了三貼雙陸。只見李銘吴惠兩個驀地上樓來磕頭。伯爵道:「好呀!你兩個來的正好。在那裏來?怎知道俺們在這裏?」李銘跪下,掩口説道:「小的和吴惠先到宅裏來,宅裏説爹們在這邊房子裏擺酒,前來伏侍爹們。」西門慶道:「也罷!你起來伺候。玳安,快往對門請你韓大叔去。」不一時,韓道國到了,作了揖坐下。一面收拾放桌兒,廚下拿春盤案酒來,琴童便在旁邊用銅布甑兒篩酒。伯爵與希大居上,西門慶主位,韓道國打横,坐下,把酒來斟。一面使玳安後邊請唱的去。

少頃,韓玉釧兒董嬌兒兩個慢條廝禮上樓來,望上不當不正磕下頭去。伯爵罵道:「我道是誰來,原來是這兩個小淫婦兒!頭裏知道我在這裏,我叫着怎的不先來見我?這等大膽,到明日一家不與你個功德,你也不怕。」董嬌兒笑道:「哥兒,那裏隔墙掠鬼臉兒,可不把我唬殺!」韓玉釧道:「你知道愛奴兒掇着獸頭城外裏掠,好個丢醜兒的孩兒。」伯爵道:「哥,你今日忒多餘了。有了李銘吴惠在這裏唱罷了,又要這兩個小淫婦做什麽?還不趂早打發他去,大節夜還趕幾個錢兒。等住囬晚了,越發沒人要了!」韓玉釧兒道:「哥兒,你怎的沒羞?大爹叫了俺們來答應,又不伏侍你!哥,你怎的閑出氣?」伯爵道:「儍小歪剌骨兒,你現在這裏,不伏侍我,你説伏侍誰?」韓玉釧道:「唐胖子掉在醋缸裏——把你撅酸了。」伯爵道:「賊小淫婦兒,是撅酸了我?等住囬散了家去時,我和你答話!我左右有兩個法兒,你原出得我手!」董嬌兒問道:「哥兒,那裏兩個法兒,説來我聽!」伯爵道:「我頭一個兒,對巡捕説了,拿你犯夜。到第二日,我拿個拜帖兒對你周爺説,拶你一頓好拶子。十分不巧,只消三分銀子燒酒,把擡轎的灌醉了,隨你這小淫婦兒去。天晚,到家沒錢,不怕鴇子不打,管我腿事!」韓玉釧道:「十分晚了,俺們不去,在爹這房子裏睡。再不,教爹這裏差人送俺們。王媽媽支錢——一百文不於于你。好淡嘴女又十撇兒。」伯爵道:「我是奴才,如今年程欺保了!」拿三道三,説笑囬,兩個唱的在傍彈唱了春景之詞。

衆人纔拿起湯飯來喫,只見玳安兒走來,報道:「祝爹來了!」衆人都不言語。不一時,祝日念上的樓來,看見伯爵和謝希大在上面,説道:「你兩個好喫,可成個人!」因説:「謝子純,哥這裏請你,也對我説一聲兒。三不知就走的來了,教我只顧在粘梅花處那裏尋你。」希大道:「我也是誤行,纔撞見哥在樓上和應二哥打雙陸,走上來作揖,被哥留住了。」西門慶因令玳安兒:「拿椅兒來,我和祝兄弟在下邊坐罷。」於是安放鍾筯,在下席坐了。廚下拿了湯飯上來,一齊同喫。西門慶只喫了一個包兒,呷了一口湯,因見李銘在旁,都遞與李銘,遞下去喫了。那應伯爵、謝希大、祝日念、韓道國,每人青花白地喫一大深碗八寳攢湯,三個大包子,還零四個挑花燒賣,只留了一個包兒壓碟兒。左右收下湯碗去,斟上酒來飲酒。希大因問祝日念道:「你陪他還到那裏纔拆開了?怎知道我在這裏?」祝日念於是如此這般告説:「我因尋了你一囬,尋不着,就同王三官到老孫家會了,往許不與先生那裏借三百兩銀子去。乞孫寡嘴老油嘴把借契寫差了。」希大道:「你們休寫上我,我不管。左右是你與老孫作保,討保頭錢使。」因問:「怎的冩差了?」祝日念道:「我那等吩咐他,寫了文書滑着些,立與他三限纔還他這銀子。不依我,教我從新把文書又改了。」希大道:「你文書上怎麽寫着?念一遍我聽。」祝日念道:「依着了我,這等寫:

\begin{myquote}[\markfont]
立借契人王寀,係招宣府舍人。{\kaishu(休説『因為要錢使用』,只説)}要錢使用,憑中見人孫天化祝日念作保,借到許不與先生名下,{\kaishu(不要説『白銀』)}軟斯金三百兩,每月{\kaishu(休説『利錢』,只説)}出納梅兒五百文。{\kaishu(約至次年交還。別要題『次年』,只説)}約至三限交還。{\kaishu(那三限?)}頭一限,風吹轆軸打孤鴈;第二限,水底魚兒跳上岸;第三限,水裏石頭泡得爛;{\kaishu(這三限交還他。平白寫了『垓子點頭』那一年纔還他。我便説,垓子點頭,倘忽遇着一年地動怎了?教我改了兩句,説道)}如借債人東西不在,代保人門面南北躱閃。恐後無憑,立此文契不用。{\kaishu(到後又批了兩個字:)}後空。」
\end{myquote}

謝希大道:「你這等寫着,還説不滑哩?及到水裏石頭爛了時,知他和尚在也不在?」祝日念道:「你倒説的好,有一朝天旱水淺,朝廷挑河,把石頭乞做工的伕子兩三鐝頭砍得稀爛,怎了?那時少不的還他銀子。」衆人説笑了一囬。

看看天晚,西門慶吩咐樓上點起燈,又樓簷前一邊一盞羊角玲燈,甚是奇巧。不想家中月娘使棋童兒和排軍擡送了四個攢盒,都是美口糖食,細巧菓品:也有黄烘烘金橙、紅馥馥石榴、甜磂磂橄欖、青翠翠蘋婆、香噴噴水梨;又有純蜜蓋柿、透糖大棗、酥油松餅、芝蔴象眼、骨牌减煠、蜜潤縧環;也有柳葉糖、牛皮纏。端的世上稀奇,寰中少有。西門慶叫棋童兒向前問他:「家中衆奶奶們散了不曾?還在那裏喫酒?誰使你送來?」棋童道:「大娘使小的送來,與爹這邊下酒。衆奶奶們還未散哩。戯文扮了四摺,大娘留住,大門首喫酒看放煙煙火哩。」西門慶問:「有人看沒有?」棋童道:「擠圍滿街人看。」西門慶道:「我吩咐下平安兒,留下四名青衣排軍,拿欄杆在大門首攔人伺候,休放閑雜人挨擠。」棋童道:「小的與平安兒兩個,同排軍都看放了煙火。衆人七八散了,大娘纔使小的來了,並沒閑雜人攪擾。」西門慶聽了,吩咐把桌上飲饌都搬下去,將攢盒擺上。廚下拿上一道菓餡元宵來。兩個唱的在席前遞酒。西門慶吩咐棋童回家看去。一面重篩羙酒,再設珍饈,教李銘吴惠席前彈唱了一套燈詞〔雙調·新水令〕:

\begin{myquote}
「鳳城佳節賞元宵,遶鰲山瑞雲籠罩。見銀河星皎潔,看天塹月輪高。動一派簫韶,開玳宴儘歡笑。

{\markfont〔川撥棹〕}
「花燈兒兩邊挑,更那堪一天星月皎。我則見綉带風飄,寳蓋微搖;鰲山上燈光照耀,剪春蛾頭上挑。」

{\markfont〔七弟兄〕}
「一壁廂舞着,唱着共彈着,驚人的這百戯其實妙。動人的高戯怎生學,笑人的院本其實俏。」

{\markfont〔梅花酒〕}
「呀,一壁廂舞鮑老。仕女們打扮的清標,有萬種妖嬈,更百媚千嬌。一壁廂舞迓鼓,一壁廂躧高蹺,端的有笑樂。細氤氳蘭麝飄,笑吟吟飲香醪。」

{\markfont〔喜江南〕}
「呀,今日喜孜孜開宴賞元宵,玉纖慢撥紫檀槽。燈光明月兩相耀,照樓臺殿閣,今日個開懷沉醉楽淘淘。」
\end{myquote}

唱畢,喫了元宵,韓道國先往家去了。少頃,西門慶吩咐來昭將樓下開了兩間,吊掛上簾子,把煙火架擡出去。西門慶與衆人在樓上看,教王六兒陪兩個粉頭,和來昭妻一丈青,在樓下觀看。玳安和來昭將煙火安放在街心裏,須臾點着。那兩邊圍看的,挨肩擦膀,不知其數,都説西門大官人在此放煙火,誰人不來觀看?果然紮得停當好煙火!但見:

\begin{myquote}
一丈五高花樁,四圍下山棚熱鬧。最高處一雙僊鶴,口裏啣着一封丹書,乃是一枝起火。起火萃嵂一道寒光,直鑽透斗牛邊。然後正當中一個西瓜砲迸開,四下裏人物皆着,觱剝剝萬個轟雷皆燎徹。彩蓮舫,賽月明,一個趕一個,猶如金燈衝散碧天星;紫葡萄,萬架千株,好似驪珠倒挂水晶簾箔。霸王鞭,到處響亮;地老鼠,串遶人衣。瓊盞玉臺,端的旋轉得好看;銀蛾金蟬,施逞巧妙難移。八僊捧壽,各顯神通;七聖降妖,通身是火。黄煙兒,綠煙兒,氤氳籠罩萬堆霞;緊吐蓮,慢吐蓮,燦爛爭開十段錦。一丈菊與煙蘭相對,火梨花共落地桃爭春。樓臺殿閣,頃刻不見巍峨之勢;村坊社鼓,彷彿難聞歡鬧之聲。貨郎擔兒,上下光焰齊明;鮑老車兒,首尾迸得粉碎。五鬼鬧判,焦頭爛額見猙獰;十面埋伏,馬到人馳無勝負。總然費卻萬般心,只落得火滅煙消成煨燼!

玉漏銅壶且莫催,星橋火樹徹明開。

萬般傀儡皆成妄,使得遊人一笑囬。
\end{myquote}

那應伯爵見西門慶有酒了,剛看罷煙火下樓來,見王六兒在這裏,推小淨手,拉着謝希大、祝日念,也不辭西門慶就走了。玳安便道:「二爹那裏去?」伯爵便向他耳邊説道:「儍孩子,我頭裏説的那本帳,我若不起身,别人也只顧坐着,顯的就不趣了。等你爹問你,只説俺們都跑了。」落後西門慶見煙火放了,問伯爵等那裏去了?玳安道:「應二爹和謝爹都一路去了,小的攔不囬來,教上覆爹。」西門慶就不再問了。因叫過李銘吴惠來,每人賞了一大巨盃酒與他喫,吩咐:「我且不與你唱錢。你兩個到十六日,早來答應。還是應二爹三個,幷衆夥計當家兒,晚夕在門首喫酒。」李銘跪下道:「小的告禀爹,十六日和吴惠左順鄭奉三個,都往東平府,新陞的胡爺那裏到任,官身去,只到後晌纔得來。」西門慶道:「左右俺們晚夕纔喫酒哩,你只休悮了就是了。」二人道:「小的並不敢悮。」於是跪着喫畢酒,拜辭出門。西門慶吩咐:「明日家中堂客擺酒,李桂姐吴銀姐都在這裏,你兩個好歹來走一走。」兩個唱的應諾了,一同出門,不在話下。西門慶吩咐來昭、玳安、琴童,看着收家活,滅息了燈燭,就往後邊房裏去了。

且説來昭兒子小鐵棍兒,正在外邊看收了煙火,見西門慶進去了,於是來樓上。見他爹老子掉了一盤子雜合的肉菜、一甌子酒,和些元宵,拿到屋裏,就問他娘一丈青討,手裏拿着燒胡鬼子,被他娘打了兩下。不妨他走在後邊院子裏頑耍,只聽正面房子裏笑聲,只説唱的還沒去哩。見房門関着,於是眼裏望裏張看,見房裏掌着燈燭。原來西門慶和王六兒兩個,在床沿子上行房。西門慶已有酒的人,把老婆倒按在床沿上,燈下褪去小衣,那話上使着托子,幹後庭花。一上手一陣往來𢵞打,何止數百囬,𢵞打的連聲響亮,其喘息之聲,往來之勢,猶賽折床一般,無䖏不聽見。這小孩子正在那裏明覷,不防他娘一丈青走來後邊,看見他孩子,揪着頭角兒揪到那前邊,鑿了兩個栗爆。罵道:「賊祸根子!小奴才兒!你還少第二遭死!又往那裏聽他去。」於是與了他幾個元宵喫了,不放他出來,就嚇住他上炕睡了。西門慶和老婆足幹搗有兩頓飯時,纔了事。玳安打發擡轎的酒飯喫了,跟送他到家;然後纔來,同琴童兩個打着燈兒,跟西門慶家去。正是:不愁明月盡,自有暗香來。有詩為證:

\begin{myquote}
南樓玩賞頓忘歸,總有風流得幾時。

囬來明月三更轉,不覺歡娱醉似泥。
\end{myquote}

畢竟未知後來如何,且聽下囬分解。

