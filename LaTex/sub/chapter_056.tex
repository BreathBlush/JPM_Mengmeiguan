\includepdf[pages={111,112},fitpaper=false]{tst.pdf}
\chapter*{第五十六囬 \\西門慶周濟常時節 應伯爵擧荐水秀才}
\addcontentsline{toc}{chapter}{第五十六囬 西門慶周濟常時節 應伯爵擧荐水秀才}
\markboth{第五十六囬 西門慶周濟常時節 應伯爵擧荐水秀才}{第五十六囬 西門慶周濟常時節 應伯爵擧荐水秀才}


\begin{myquote}
斗積黄金侈素封,蘧蘧莊蝶夢魂中。

曾聞郿塢光難駐,不道銅山運可窮。

此日分籝推鮑子,當年沉水笑龐公。

悠悠末路誰知己,惟有夫君尚古風。
\end{myquote}

這八句單説人生世上,榮華富貴,不能常守。有朝無常到來,恁地堆金積玉,出落空手歸陰。因此西門慶仗義疎財,救人貧難,人人都是讚嘆他的,這也不在話下。當日西門慶留下兩個歌童祇候:「若遇有呼喚,不得有違。」兩人應諾去了。隨即打發苗家人囬書禮物,又賞了些銀錢。苗實苗秀磕頭謝了出門。後來兩個歌童,西門慶畢竟用他不着,都送太師府去了。正是:千金散盡教歌舞,留與他人楽少年。

却説常時節自那日席上求了西門慶的事情,還不得個到手,房主又日夜催逼了不的。恰遇西門慶自従在東京來家,今日也接風,明日也接風,一連過了十來日,只不得個會面。常言道:見面情難盡,一個不見,却告訴誰?每日央了應伯爵,只走到大官人門首,問聲説不在,就空囬了。回家又被渾家埋怨道:「你也是男子漢大丈夫,房子沒間住,吃這般懊惱氣!你平日只認的西門大官人,今日求些周濟,也做了瓶落水。」説的常時節有口無言,獃登登不敢做聲。到了明日,早起身尋了應伯爵,來到一個酒店内。只見小小茅簷兒,靠着一灣流水,門前綠樹陰中露出酒望子來。五七個火家,搬酒搬肉不住的走。店裏横着一張櫃檯,掛幾樣鮮魚鵝鴨之類,倒潔淨可坐。便請伯爵店裏吃三盃去。伯爵道:「這却不當生受。」常時節拉了到店裏坐下,量酒打上酒來,擺下一盤薰肉,一盤鮮魚。酒過兩巡,常時節道:「小弟向求哥和西門大官人説的事情,這幾日通不能够會面,房子又催逼的緊。昨晚被房下聒絮了半夜,耐不的,五更抽身,專求哥趂早;大官人還沒出門時慢慢地候他。不知哥意下如何?」應伯爵道:「受人之托,必當終人之事。我今日好歹要大官人助你些就是了。」兩個人又吃過幾盃。應伯爵便推:「早酒不吃罷。」常時節又勸一盃。算還酒錢,一同出門,逕奔西門慶屋裏來。

那時正是新秋時候,金風荐爽。西門慶連醉了幾日,覺精神減了幾分。正遇周内相請酒,便推事故不去,自在花園藏春塢遊玩。原來西門慶後園那藏春塢,有的是菓樹鮮花兒,四季不絶。這時雖是新秋,不知開着多少花朵在園裏。西門慶無事在家,只是和吴月娘、孟玉樓、潘金蓮、李瓶兒,五個在花園裏頑耍。只見西門慶頭戴着忠靖冠,身穿柳綠緯羅直身,粉頭靴兒。月娘上穿柳綠杭絹對衿襖兒,淺藍水紬裙子,金紅鳳頭高底鞋兒。孟玉樓上穿鴉青緞子襖兒,鵝黄紬裙子,桃紅素羅羊皮金滚口高底鞋兒。潘金蓮上穿着銀紅縐紗白絹裏對衿衫子,荳綠沿邊金紅心比甲兒,白杭絹畫拖裙子,粉紅花羅高底鞋兒。只有李瓶兒上穿素青杭絹大衿襖兒,月白熟絹裙子,淺藍玄羅高底鞋兒。四個妖妖嬈嬈,伴着西門慶尋花問柳,好不快活。

且説常時節和應伯爵來到廳上,問知大官人在屋裏,懽的坐着等了好半日,却不見出來。只見門外書童和畫童兩個擡着一隻箱子,都是綾絹衣服,氣吁吁走進門來,亂嚷道:「等了這半日,還只得一半!」就廳上歇下。應伯爵便問:「你爹在那裏?」書童道:「爹在園裏頑耍哩。」伯爵道:「勞你説聲。」兩個依舊擡着進去了。不一時,書童出來道:「爹請應二爹常二叔少待,便出來。」兩人坐着等了一回,西門慶纔走出來。二人作了揖,便請坐地。伯爵道:「連日哥吃酒忙,不得些空。今日却怎的在家裏?」西門慶道:「自従那日別後,整日被人家請去飲酒,醉的了不的,通沒些精神。今日又有人請酒,我只推有事不去。」伯爵道:「方纔那一箱衣服,是那裏擡來的?」西門慶道:「這目下交了秋,大家都要添些秋衣。方纔一箱是你大嫂子的,還做不完,纔夠一半哩。」常時節伸着舌道:「六房嫂子就六箱了,好不費事!小户人家,一疋布也難的。恁做着許多綾絹衣服,哥果是財主哩!」西門慶和應伯爵都笑起來。伯爵道:「這兩日杭州貨船怎地還不見到?不知他買賣貨物何如?這兩日不知李三黄四的銀子,曾在府裏頭關了些送來與哥麽?」西門慶道:「貨船不知在那裏擔閣着,書也没捎封寄來,好生放不下。李三黃四的,又説在出月纔關。」應伯爵挨到身邊坐下,乘間便説:「常二哥那一日在哥席上求的事情,一向哥又沒的空,不曾説的。常二哥被房主催逼慌了,每日被嫂子埋怨。二哥只麻做一團,沒個理會。如今又是秋涼了,身上皮襖兒又當在典舖裏。哥若有好心,常言道:救人湏救急時無。省的他嫂子日夜在屋裏絮絮叨叨。況且尋的房子住着了,人走動,也只是哥的體面。因此常二哥央小弟特地來求哥,早些周濟他吧。」西門慶道:「我當先曾許下他來。因為東京去了這番,費的銀子多了。本待等韓夥計到家,和他理會,要房子時,我就替他兑銀子買。如今又恁地要緊?」伯爵道:「不是常二哥要緊,當不的他嫂子聒絮,只得求哥早些便好。」西門慶躊躇了半晌道:「既這樣,也不難。且問他,要多少房子纔夠住了?」伯爵道:「他兩口兒也得一間門面,一間客坐,一間床房,一間廚灶:四間房子是少不得的。論着價銀,也得三四個多銀子。哥只早晚凑些,交他成就了這樁事罷。」西門慶道:「今日先把幾兩碎銀與他拿去,買件衣服,辦些家活,盤攪過來。待尋下房子,我自兑銀與你成交,可好麽?」兩個一齊謝道:「難得哥好心!」西門慶便呌書童:「去對你大娘説,皮匣内一包碎銀取了出來。」書童應諾去了。不一時取了一包銀子出來,遞與西門慶。西門慶對常時節道:「這一包碎銀,是那日東京太師府賞封剩下的十二兩,你拿去好雜用。」打開與常時節看,都是三五錢一塊的零碎紋銀。常時節接過,放在衣袖裏,就作揖謝了。西門慶道:「我這幾日不是要遲你,只等你尋下房子,一攪果和你交易。你又沒曾尋的。如今即忙便尋下,待我有銀,一起兑去便了。」常時節又稱謝不迭。三個依舊坐下。伯爵便道:「幾個古人輕財好施,到後來子孫高大門閭,把祖宗基業一發增的多了。慳吝的積下許多金寶,後來子孫不好,連祖宗墳土也不保。可知天道好還哩!」西門慶道:「兀那東西是好動不喜靜的,怎肯埋沒在一處?也是天生應人用的,一個人堆積,就有一個人缺少了。因此積下財寳,極有罪的。」有詩為證:

\begin{myquote}
積玉堆金始稱懷,誰知財寳祸根荄。

一文愛惜如膏血,仗義翻將笑作獃。

親友人人同陌路,存形心死定堪哀。

料他也有無常日,空手俜伶到夜臺。
\end{myquote}

正説着,只見書童托出飯來,三人吃了。常時節作謝起身,袖着銀子歡的走到家來。剛剛進門,只見那渾家鬧炒炒嚷將出來,罵道:「梧桐葉落滿身光棍的行貨子!出去一日,把老婆餓在家裏,尚兀自千歡萬喜到家來,可不害羞哩!房子没的住,受别人許多酸嘔氣,只教老婆耳朶裏受用。」那常二只是不開口。任老婆罵的完了,輕輕把袖裏銀子摸將出來,放在桌兒上,打開瞧着道:「孔方兄,孔方兄!我瞧你光閃閃響噹噹的無價之寳,滿身通麻了,恨没口水嚥你下去。你早些來時,不受這淫婦幾場合氣了!」那婦人明明看見包裏十二三兩銀子一堆,喜的搶近前來,就想要在老公手裏夺去。常二道:「你生世要駡漢子,見了銀子,就來親近哩!我明日把銀子去買些衣服穿,好自去别處過活,卻再不和你鬼混了。」那婦人陪着笑臉道:「我的哥,端的此是那裏來的這些銀子?」常二也不做聲。婦人又問道:「我的哥,難道你便怨了我?我只是要你成家。今番有了銀子,和你商量停當,買房子安身,卻不好?倒恁地喬張智!我做老婆的,不曾有失花兒,憑你怨我,也是枉了!」常二也不開口。那婦人只顧饒舌,又見常二不瞅不採,自家也有幾分慚愧了,禁不的掉下淚來。常二看了,嘆口氣道:「婦人家不耕不織,把老公恁地發作!」那婦人一發掉下淚來。兩個人都閉着口,又沒個人勸解,悶悶的坐着。常二尋思道:「婦人家也是難做。受了辛苦,埋怨人也怪他不的。我今日有了銀子,不瞅他,人就道我薄情,便大官人知道,也湏断我不是。」就對那婦人笑道:「我自耍你,誰怪你來!只你時常聒噪,我只得忍着出門去了。卻誰怨你來?我明白和你説,這銀子原是早上耐你不的,特地請了應二哥在酒店裏吃了三盃,一同往大官人宅裏等候。恰好大官人正在家,没曾去吃酒。多虧了應二哥,不知費許多唇舌,纔得這些銀子到手。還許我尋下房子,一頓兑銀與我成交哩!這十二兩,是先教我盤攪過日子的。」那婦人道:「原來正是大官人與你的。如今又不要花費開了,尋件衣服過冬,省的耐冷。」常二道:「我正要和你商量,十二兩紋銀,買幾件衣服,辦幾件家活在家裏,等有了新房子,搬進去也好看些。只是感不盡大官人恁好情,後日搬了房子,也索請他坐坐是。」婦人道:「且到那時,再作理會。」正是:惟有感恩并積恨,萬年千載不生塵。

常二與婦人兩個説了一回,那婦人道:「你那裏吃飯來没有?」常二道:「也是大官人屋裏吃來的,你没曾吃飯,就拿銀子買了米來。」婦人道:「仔細拴着銀子!我等你,就來。」常二取栲栳望街上便走。不一時,買了米,栲栳上又放着一大塊羊肉兒,笑哈哈跑進門來。那婦人迎門接住道:「這塊羊肉又買他做甚?」常二笑道:「剛纔説了許多辛苦,不爭這一些羊肉,就牛也該宰幾個請你。」那婦人笑指着常二罵道:「狠心的賊,今日便懷恨在心?看你怎的奈何了我。」常二道:「只怕有一日,叫我一萬聲:『親哥,饒我小淫婦罷。』我也只不饒你哩!試試手段着。」那婦人聽説,笑的走井邊打水去了。當下婦人做了飯,切了一碗羊肉,擺在桌兒上,便呌:「哥,吃飯。」常二道:「我纔在大官人屋裏吃的飯,不要吃了。你餓的慌,自吃些罷。」那婦人便一個自吃了。收了家活,打發常二去買衣服。

常二袖着銀子,一直奔到大街上來。看了幾家,都不中意。只買了一領青杭絹女襖,一條綠紬裙子,月白雲紬衫兒,紅綾襖子兒,白紬子裙兒,共五件;自家也對身買了件鵝黄綾襖子,丁香色紬直身兒,又有幾件布草衣服。共用去六兩五錢銀子。打做一包,背着來到家中,教婦人打開看看。那婦人忙打開來瞧着,便問:「多少銀子買的?」常二道:「六兩五錢銀子買來。」婦人道:「雖没的便宜,却值這些銀子。」一面收拾箱籠放好,明日去買家活。當日婦人歡天喜地過了一日,埋怨的話都掉在東洋大海去了,不在話下。

再表應伯爵和西門慶兩個,自打發常時節出門,依舊在廳上坐的。西門慶因説起:「我雖是個武職,恁的一個門面,京城内外也交結的許多官員。近日又拜在太師門下,那些通問的書柬,流水也似往來。我又不得細工夫,都不得料理。我一心要尋個先生兒在屋裏,好教他寫寫,省些力氣也好;只没個有才學的人。你看有時,便對我説。我須尋間空房與他住下,每年算還幾兩束脩與他養家。卻也要是你心腹之友便好。」伯爵道:「哥不説不知。你若要别樣卻有,要這個却難。怎的要這個倒没?第一要才學,第二就要人品了。又要好相處,沒些説是説非,翻唇弄舌,這就好了。若只是平平才學,又做慣搗鬼的,怎用的他?小弟只有祖父相處一個朋友生下來的孫子,他現是本州一個秀才。應擧過幾次,只不淂中。他胸中才學,果然班馬之上。就是他人品,也孔孟之流。他和小弟通家兄弟,極有情分的。曾記他十年前應擧兩道策,那一科試官極口贊他好。卻不想又有一個賽過他的,便不中了。後來連走了幾科不中,禁不的髮白鬢斑。如今他雖是飄零書劍,家裏也還有一百畝田,三四帶房子,整的潔淨住着。」西門慶道:「他家幾口兒也夠用了,卻怎的肯來人家坐館?」應伯爵道:「當先有的田房,都被那些大户人家買去了。如今只剩得雙手皮哩!」西門慶道:「原來是賣過的田,算甚麽數!」伯爵道:「這果是算不的數了。只他一個渾家,年紀只好二十左右,生的十分美貌。又有兩個孩子,纔三四歲。」西門慶道:「他家有了美貌渾家,那肯出來?」伯爵道:「喜的是兩年前,渾家專要偷漢,跟了個人上東京去了。兩個孩子又出痘死了。如今止存他一口,定然肯出來。」西門慶笑道:「恁地説的他好,都是鬼混!你且説他姓甚麽?」伯爵道:「姓水。他才學果然無比,哥若用他時,管情書柬詩詞歌賦,一件件增上哥的光輝哩。人看了時,都道西門大官人恁地才學哩!」西門慶道:「你纔説這兩樁都是調謊。我却不信你的調謊。你有記的他些書柬兒念來我聽。若好時,我便請他來家,撥間房子住下。只一口兒,也好看承的。尋個好日子,便請他也罷。」伯爵道:「曾記得他捎書來,要我替他尋個主兒。這一封書,略記的幾句,念與哥聽,〈黄鶯兒〉:

\begin{myquote}
『書寄應哥前:別來思,不待言。滿門兒托賴都康健。舍字在邊,傍立着官,有時一定求方便。羡如椽,往來言疏,落筆起雲煙。』」
\end{myquote}

西門慶聽畢,呵呵大笑將起來道:「他滿心正經,要你和他尋個主子,却怎的不捎封書來,倒寫着一隻曲兒!又做的不好。可知道他才學荒疎,人品散誕哩。」伯爵道:「這倒不要作難他。只為他與我是三世之交。小弟兩三歲時節,他也纔夠四五歲,那時就同吃糖糕餅果之類,也沒些兒爭論。後來大家長大了,上學堂讀書寫字,先生也道:『應二學生子和水學生子,一般的聰明伶俐,後來一定長進。』落後做文字,一樣同做,再没些妒忌。日裏同行同坐,夜裏有時也同一䖏歇。到了戴網子,尚兀自相厚的。因此是一個人一般極好兄弟,故此不拘形跡,便随意寫個曲兒。我一見了,也有幾分着惱。後想一想,他自托相知,纔敢如此,就不惱罷了。況且那隻曲兒,也到做的有趣。哥却看不出來。第一句説『書寄應哥前』,是啟口,就如人家寫『某人見字』一般,却不好哩?第二句説:『別來思,不待言』,這是敍寒温了,簡而文,又不好哩?第三句是『滿門兒托賴都康健』,這是説他家没事故了,後來一發好的緊了!」西門慶道:「第五句是甚麽説話?」伯爵道:「哥不知道,這正是拆白道字,尤人所難。『舍』字在邊旁,立着『官』字,不是個『舘』字?若有館時,千萬要擧荐,因此説:『有時定要求方便』。『羡如椽』,他説自家一筆如椽,做人家往來的書疏,筆兒落下去,其煙雲滿紙,因此説『落筆起雲煙。』哥,你看他詞裏,有一個字兒是閑話麽?只這幾句,穩穩把心窝裏事都寫在紙上,可不好哩!」西門慶被伯爵説了他恁地好處,到沒的説了,只得對伯爵道:「你既説他許多好處,且問你有甚正經的書札,拿些我看看,我就請了他。」伯爵道:「他做的詞賦也有在我處,只是不曾带得來哥看。我還記的他一篇文字,做得甚好。就念與哥聽着:

\begin{myquote}
『一戴頭巾心甚歡,豈知今日悮儒冠。

別人戴你三五載,偏戀我頭三十年。

要戴烏紗求閣下,做篇詩句別尊前。

此番非是吾情薄,白髮臨期太不堪!

今秋若不登高第,踹碎冤家學種田。
\end{myquote}

\begin{myquote}
維歲在大比之期,時到揭曉之候。訴我心事,告汝頭巾:為你青雲利器望榮身,誰知今日白髮盈頭戀故人。嗟乎!憶我初戴頭巾,青青子襟;承汝枉顧,昂昂氣新。既不許我少年早發,又不許我久屈待伸;上無公卿大夫之職,下非農工商賈之民。年年居白屋,日日走黌門。宗師案臨,膽怯心驚;上司迎接,東走西奔。思量為你,一世驚驚嚇嚇受了若干苦辛;一年四季零零碎碎被人賴了多少束修銀。告狀助貧分穀五斗,祭下領票支肉半斤。官府見了,不覺怒嗔;皂快通稱,盡道廣文。東京路上,陪人幾次;兩齋學霸,惟吾獨尊。你看我兩隻皂靴穿到底,一領藍衫剩布筋。埋頭有年,説不盡艱難悽楚;出身何日,空歷過冷淡酸辛。賺盡英雄,一生不得文章力;未沾恩命,數載猶懷霄漢心。嗟乎哀哉,哀此頭巾!看他形狀,其實可衿:後直前横,你是何物?七穿八洞,眞是祸根。嗚呼!冲霄鳥兮未垂翅,化龍魚兮已失鱗。豈不聞久不飛兮一飛登雲;久不鳴兮一鳴驚人。早求你脱胎換骨,非是我棄舊憐新。斯文名器,想是通神。従兹長别,方感洪恩。短詞薄奠,庶其來歆。理極數窮,不勝具懇。就此拜别,早早請行!』」
\end{myquote}

伯爵念罷,西門慶拍手大笑道:「應二哥把這樣才學就做了班揚了!」伯爵道:「他人品比才學又高,如今且説他人品罷。」西門慶道:「你且説來。」伯爵道:「前年他在一個李侍郎府裏坐館。那李家有幾十個丫頭,一個個都是美貌俊俏的;又有幾個伏侍的小廝,也一個個都標致龍陽的。那水秀才連住了四五年,再不起一些邪念。後來不想被幾個壞事的丫頭小廝,見是一個聖人一般,反去日夜刮他。那水秀才又極好慈悲的人,便口軟勾搭上了,因此被主人逐出門來,鬨動街坊,人人都説他無行。其實水秀才原是坐懷不亂的,若哥請他來家,憑你許多丫頭小廝同眠同宿,你看水秀才亂麽?再不亂的!」西門慶道:「他既前番被主人趕了出門,一定有些不停當哩。二哥雖與我相厚,那樁事不敢領教。前日敝僚友倪桂巖老先生,曾説他有個姓温的秀才,且待他來時再處。」

畢竟未知何如,且聽下囬分解。

