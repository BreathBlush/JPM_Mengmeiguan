\includepdf[pages={13,14},fitpaper=false]{tst.pdf}
\chapter*{第七囬 \\薛婆兒説娶孟玉樓 楊姑娘氣罵張四舅}
\addcontentsline{toc}{chapter}{第七囬 薛婆兒説娶孟玉樓 楊姑娘氣罵張四舅}
\markboth{{\titlename}卷之一}{第七囬 薛婆兒説娶孟玉樓 楊姑娘氣罵張四舅}


\begin{myquote}
我做媒人實可能,全憑兩腿走殷勤。

唇鎗慣把鰥男配,舌劔能調烈女心。

利市花紅頭上帶,喜筵餅錠袖中撑。

只有一件不堪處,半是成人半敗人。
\end{myquote}

話説西門慶家中,賣翠花兒的薛嫂兒,提着花箱兒,一地裏尋西門慶不着。因見西門慶使的小廝玳安兒,問:「大官人在那裏?」玳安道:「俺爹在舖子裏,和傅二叔算帳。」原來西門慶家開生藥舖,主管姓傅,名銘,字自新;排行第二,因此呼他做傅二叔。

這薛嫂一直走到舖子門首,掀開簾子,見西門慶正在裏面與主管算帳。一面點首兒,喚他出來。這西門慶見是薛嫂兒,連忙撇了主管出來,兩人走在僻靜處說話。薛嫂道了萬福,西門慶問他有甚話說。薛嫂道:「我來有一件親事,來對大官人說,管情中得你老人家意,就頂死了的三娘窝兒。方纔我在大娘房裏,買我的花翠,留我喫茶,坐了這一日,我就不曾敢提起。徑來尋你老人家,和你說。這位娘子,說起來你老人家也知道,是咱這南門外販布楊家的正頭娘子。手裏有一分好錢;南京拔步牀也有兩張。四季衣服、粧花袍兒,插不下手去也有四五隻箱子。珠子箍兒、胡珠環子、金寳石頭面、金鐲銀釧不消說,手裏現銀子,他也有上千兩;好三梭布也有三二佰筒。不幸他男子漢去販布,死在外邊。他守寡了一年多。身邊又沒子女,止有一個小叔兒,還小,纔十歲,青春年少,守他甚麽?有他家一個嫡親的姑娘,要主張着他嫁人。這娘子今年不上二十五六歲,生的長挑身材,一表人物。打扮起來,就是個燈人兒,風流俊俏,百伶百俐。當家立紀、針指女工、雙陸棋子,不消說。不瞞大官人說,他娘家姓孟,排行三姐,就住在臭水巷。又會彈了一手好月琴。大官人若見了,管情一箭就上垜。誰似你老人家有福,好得這許多帶頭,又得一個娘子!」西門慶只聽見婦人會彈月琴,便可在他心上。就問薛嫂兒:「幾時相會看去?」薛嫂道:「我和你老人家這等計議:相看不打緊。如今他家,一家子只是姑娘大。雖是他娘舅張四,山核桃差着一隔兒哩。這婆子原嫁與北邊半邊街徐公公房子裏住的孫歪頭。歪頭死了,這婆子守寡了三四十年,男花女花都無,只靠侄男侄女養活。今日已過,明日我來會大官人。咱只倒在他身上求他——求只求張良,拜只拜韓信。這婆子愛的是錢財。明知他侄兒媳婦有東西,隨問什麽人家他也不管,只指望要幾兩銀子。大官人多許他幾兩銀子。家裏有的是那囂緞子,拿上一段,買上一擔禮物,親去見他,和他講過。一拳打倒他,隨問傍邊有人説話,這婆子一力張主誰敢怎的?」這薛嫂兒一席話,說的西門慶歡従額角眉尖出,喜向腮邊笑臉生。看官聽說:世上這媒人們,原來只一味圖賺錢,不顧人死活。無官的說做有官,把偏房說做正房。一味瞞天大謊,全無半點兒真實。正是:
\begin{myquote}
媒妁殷勤說始終,孟姬愛嫁富家翁。

有緣千里能相會,無緣對面不相逢。
\end{myquote}

西門慶當日與薛嫂相約下,明日是好日期,就買禮往北邊他姑娘家去。薛嫂說畢話,提着花箱兒去了。西門慶進來和傅夥計算帳。一宿晚景不題。

到次日,西門慶早起,打選衣帽齊整,拿了一段尺頭,買了四盤羹果,僱了一個擡盒的。薛嫂領着,西門慶騎着頭口,小廝跟隨,徑來北邊半邊街徐公房子裏楊姑娘家門首。薛嫂先入去通報姑娘得知,説:「近邊一個財主,敬來門外,和大娘子說親。我說一家只姑奶奶是大,先來覿面,親見過你老人家,講了話,然後纔敢領去門外相看。今日小媳婦領來,現在門首下馬伺候。」婆子聽見,便道:「阿呀保山!你如何不先來説聲?」一面吩咐了丫鬟打掃客位,收拾乾淨,炖下好茶;一面道:「有請!」這薛嫂一力攛掇先把盒擔擡進去擺下,打發空盒擔兒出去,就請西門慶進來入見。這西門慶頭戴纏棕大帽,一撒鈎縧,粉底皂靴,進門見婆子拜四拜。婆子拄着拐,慌忙還下禮去。西門慶那裏肯,一口一聲只叫:「姑娘請受禮!」讓了半日,婆子受了半禮。分賓主坐下,薛嫂在傍打横。婆子便道:「大官人貴姓?」薛嫂道:「我纔對你老人家説,就忘了!便是咱清河縣數一數二的財主,西門慶大官人!在縣前開着個大生藥舖,又放官吏債。家中錢過北斗,米爛陳倉。沒個當家立紀娘子。聞得咱家門外大娘子要嫁,特來見姑奶奶講說親事。」因說:「你兩親家都在此,六眼不藏私,有話當面說,省得俺媒人們架謊。這裏是姑奶奶大,大官人有話不先來和姑奶奶說,再和誰說?」婆子道:「官人倘然要說俺侄兒媳婦,自恁來閑講便了,何必費煩,又買禮來?使老身却之不恭,受之有愧!」西門慶道:「姑娘在上,沒的禮物,惶恐!」那婆子一面拜了兩拜,謝了,收過禮物去。薛嫂托盤子出門,一囬走來陪坐。拿茶上來喫畢,婆子開口説道:「老身當言不言謂之懦。我侄兒在時,做人掙了一分錢。不幸死了,如今都落在他手裏,說少也有上千兩銀子東西。官人做小做大,我不管你,只要與我侄兒念上個好經。老身便是他親姑娘,又不隔従,就與上我一個棺材本,也不曾要了你家的。我破着老臉,和張四那老狗做臭老鼠,替你兩個硬張主。娶過門時,生辰貴降,官人放他來走走,就認俺這門窮親戚,也不過上你窮。」西門慶笑道:「你老人家放心,適間所言的話,我小人都知道了。你老人家既開口,休說一個棺材本,就是十個棺材本,小人也來得起!」說着,向靴桶裏取出六錠——三十兩雪花官銀,放在面前,說道:「這個不當甚麽,先與你老人家買盞茶喫。到明日娶過門時,還找七十兩銀子、兩疋緞子,與你老人家為送終之資。其四時八節,只照舊上門行走。」

看官聽說:世上錢財,乃是衆生腦髄,最能動人。這老虔婆黑眼睛珠,見了二三十兩白晃晃的官銀,滿面堆下笑來,說道:「官人在上,不當老身意小。自古先說斷,後不亂。」薛嫂在傍插口說:「你老人家忒多心,那裏這等計較!我們大老爹不是那等人,只恁還要掇着盒兒認親。你老人家不知,如今知府知縣相公也都來往,好不四海,結織人寬廣。你老人家能喫他多少!」一席話,說的婆子屁滾尿流。陪的坐一囬,喫了兩道茶,西門慶便要起身,婆子挽留不住。薛嫂道:「今日旣見了姑奶奶說過話,明日好往門外相看。」婆子道:「我家侄兒媳婦,不用大官人相。保山,你就說我說:不嫁這樣人家,再嫁甚樣人家?」西門慶作辭起身,婆子道:「官人,老身不知官人下降,匆忙不曾預備,空了官人,休怪。」拄拐送出。送了兩步,西門慶讓囬去了。薛嫂打發西門慶上馬,便說道:「還虧我主張的有理麽?寜可先在婆子身上倒,還強如别人說多。」因說道:「你老人家先囬去罷,我還在這裏和他說句話。咱已是會過。明日先往門外去了。」西門慶便拿出一兩銀子來,與薛嫂做驢子錢。薛嫂接了。西門慶便上馬來家。他便還在楊姑娘家説話飲酒,到日暮時分纔歸家去。

話休饒舌。到次日,西門慶打選衣帽齊整,袖着插戴,騎着大白馬,玳安平安兩個小廝跟隨,薛嫂兒便騎驢子,出的南門外,來到猪市街,到了楊家門首。原來門面四間,到底五層。坐南朝北,一間門樓,粉青照壁。西門慶勒馬在門首等候,薛嫂先入。去半日,出來說有請。西門慶下馬進去。裏面儀門紫牆,竹槍籬影壁,院内擺設榴樹盆景,臺基上靛缸一溜,打布凳兩條。薛嫂推開朱紅槅扇,三間倒坐,客位正面上供養着一軸水月觀音、善財童子,四面掛名人山水,大理石屏風,安着兩座投箭高壺,上下椅桌光鮮,簾櫳瀟灑。薛嫂請西門慶正面椅子上坐了,一面走入裏邊。片晌出來,向西門慶耳邊說:「大娘子梳粧未了,你老人家請先坐一坐。」只見一個小廝兒,拿出一盞福仁泡茶來,西門慶喫了,收下盞托去。這薛嫂兒倒還是媒人家,一面指手畫脚與西門慶說:「這家中除了那頭姑娘,只這位娘子是大。雖有他小叔,還小哩,不曉得什麽。當初有過世的他老公在舖子裏,一日不算銀子,搭錢也賣兩大簸籮。毛青鞋面布,俺們問他買,定要三分一尺。現一日常有二三十染的喫飯,都是這位娘子主張整理。手下使着兩個丫頭、一個小廝。長丫頭十五歲,吊起頭去,名喚蘭香;小丫頭纔十二歲,名喚小鸞。到明日過門時,都跟他來。我替你老人家說成這親事,指望典兩間房兒住,強如住在北邊那搭剌子裏,往宅裏去不方便。你老人家去年買春梅,許了我幾疋大布,還沒與我。到明日不管——一總謝罷了。」又道:「剛纔你老人家看見門首那兩座布架子,當初楊大叔在時,街道上不知使了多少錢。這房子也值七八百兩銀子。到底五層,通後街。到明日,丢與小叔罷了。」

正説着,只見使了個丫頭來叫薛嫂。良久,只聞環珮叮咚,蘭麝馥郁,婦人出來。上穿翠藍麒麟補子粧花紗衫,大紅粧花寬欄。頭上珠翠堆盈,鳳釵半卸。西門慶睜眼觀那婦人,但見:

\begin{myquote}
長挑身材,粉粧玉琢。模様兒不肥不瘦,身段兒不短不長,面上稀稀有幾點微麻,生的天然俏麗;裙下映一對金蓮小脚,果然周正堪憐。二珠金環,耳邊低掛;雙頭鸞釵,鬢後斜插。但行動,胸前搖響玉玲瓏;坐下時,一陣麝蘭香噴鼻。恰似嫦娥離月殿,猶如神女下瑤階。
\end{myquote}

西門慶一見,滿心歡喜。薛嫂忙去掀開簾子,婦人出來,望上不端不正道了個萬福,就在對面椅上坐下。西門慶把眼上下不轉睛看了一囬,婦人把頭低了。西門慶開言說:「小人妻亡已久,欲娶娘子入門爲正,管理家事。未知意下如何?」那婦人問道:「官人貴庚,沒了娘子多少時了?」西門慶道:「小人虚度二十八歲,七月二十八日子時建生。不幸先妻沒了,一年有餘。不敢請問娘子青春多少?」婦人道:「奴家青春是三十歲。」西門慶道:「原來長我二歲。」薛嫂在傍插口道:「妻大兩,黄金日日長;妻大三,黄金積如山。」説着,只見小丫鬟拿了三盞蜜餞金橙子泡茶,銀鑲雕漆茶鍾,銀杏葉茶匙。婦人起身,先取頭一盞,用纖手抹去盞邊水漬,遞與西門慶,忙用手接了。道了萬福,慌的還禮不迭。薛嫂向前用手掀起婦人裙子來,裙邊露出一對剛三寸、恰半扠,一對尖尖趫趫金蓮來,脚穿着大紅遍地金雲頭白綾高底鞋兒,與西門慶瞧。西門慶滿心歡喜。婦人取第二盞茶來,遞與薛嫂;他自取一盞陪坐。喫了茶,西門慶便叫玳安用方盒呈上錦帕二方、寳釵一對、金戒指六個,放在托盤内拿下去。薛嫂一面教婦人拜謝了,因問官人行禮日期:「奴這裏好做預備。」西門慶道:「既蒙娘子見允,今月二十四日,有些微禮過門來。六月初二日准娶。」婦人道:「既然如此,奴明日就使人來對北邊姑娘那裏說去。」薛嫂道:「大官人昨日已是到姑奶奶府上講過話了!」婦人道:「姑娘說甚來?」薛嫂道:「姑奶奶聽見大官人說此親事,好不歡喜,纔使我領大官人來這裏相見。說道:不嫁這等人家,再嫁那樣人家?我就做硬主媒,保這門親事。」婦人道:「既是姑娘恁的說,又好了!」薛嫂道:「好大娘子,莫不俺做媒敢這等搗謊!」

說畢,西門慶作辭起身。薛嫂送出巷口,向西門慶說道:「看了這娘子,你老人家心下如何?」西門慶道:「薛嫂,其實累了你。」薛嫂道:「你老人家請先行一步,我和大娘子說句話就來。」西門慶騎馬進城去了。薛嫂轉來向婦人說道:「娘子,你嫁得這位老公也罷了。」因問西門慶房裏有人沒有人,現作何生理。薛嫂道:「好奶奶,就有房裏人,那個是成頭腦的!我說是謊,你過去就看出來。他老人家名目,誰是不知道的?清河縣數一數二的財主,有名賣生藥放官吏債西門大官人!知縣知府都和他往來。近日又與東京楊提督結親,都是四門親家,誰人敢惹他!」

婦人安排酒飯,與薛嫂兒正喫着,只見他姑娘家使了小廝安童,盒子裏挎着鄉裏來的四塊黄米麵棗兒糕、兩塊糖、十幾個艾窝窝,就來問:「曾受了那人家插定不曾?奶奶說來:這人家不嫁,待嫁甚人家!」婦人道:「多謝你奶奶掛心,今已曾留下插定了。」薛嫂道:「天麽,天麽!早是俺媒人不說謊!姑奶奶家使了大官兒說將來了。」婦人收了糕,出了盒子,裝了滿滿一盒子點心臘肉,又與了安童五六十文錢:「到家多拜上奶奶。那家日子定下二十四日行禮,出月初二日准娶。」小廝去了。薛嫂道:「姑奶奶家送來什麽?與我些包了家去,捎與孩子喫。」婦人與了他一塊糖、十個艾窝窝。千恩萬謝出門,不在話下。

且說他母舅張四,倚着他小外甥楊宗保,要圖留婦人手裏東西,一心擧保與大街坊尚推官兒子尚擧人為繼室。若小可人家,還可有話說;不想聞得是縣前開生藥舖西門慶定了。他是把持官府的人,遂動不得秤了。尋思已久:「千方百計,不如破他為上計!」走來對婦人說:「娘子不該接西門慶插定,還依我嫁尚推官兒子尚擧人。他又是廝文詩禮人家,又有莊田地土,頗過得日子,強如嫁西門慶。那廝積年把持官府,刁徒潑皮。他家現有正頭娘子,乃是吳千户家女兒。過去做大是做小是?却不難為你了!況他房裏又有三四個老婆,並沒上頭的丫頭。到他家人多口多,你惹氣也!」婦人道:「自古船多不礙路。若他家有大娘子,我情願讓他做姐姐,奴做妹子。雖然房裏人多,漢子歡喜,那時難道你阻他?漢子若不歡喜,那時難道你去扯他?不怕一百,人單擢着。休說他富貴人家,那家沒四五個?着緊街上乞食的,㩦男抱女,也絜扯着三四個妻小。你老人家忒多慮了!奴過去自有個道理,不妨事。」張四道:「娘子,我聞得此人,單管挑販人口,慣打婦熬妻,稍不中意,就令媒人賣了。你願受他的這氣麽?」婦人道:「四舅,你老人家差矣!男子漢雖利害,不打那勤謹省事之妻。我在他家,把得家定,裏言不出,外言不入,他敢怎的?為女婦人家,好喫懶做,嘴大舌長,招是惹非,不打他,打狗不成?」張四道:「不是,我打聽他家,還有一個十四歲未出嫁的閨女。誠恐去到他家,三窝兩塊,他人多口多,惹氣怎了?」婦人道:「四舅說那裏話!奴到他家,大是大,小是小,凡事從上流看。待得孩兒們好,不怕男子漢不歡喜,不怕女兒們不孝順。休說一個,便是十個,也不妨事。」張四道:「我見此人,有些行止欠端,在外眠花臥柳。又裏虚外實,少人家債負,只怕坑陷了你。」婦人道:「四舅,你老人家又差矣!他就外邊胡行亂走,奴婦人家,只管得三層門内,管不得那許多三層門外的事。莫不成日跟着他走不成?常言道:世上錢財儻來物,那是長貧久富家?緊着起來,朝廷爺一時沒錢使,還問太僕寺借馬價銀子支來使。休說買賣的人家,誰肯把錢放在家裏?各人裙帯上衣食,老人家倒不消這様費心。」這張四見說不動這婦人倒喫他搶了幾句好話,好無顏色。喫了兩盞清茶,起身去了。有詩為證:

\begin{myquote}
張四無端散楚言,姻緣誰想是前緣!

佳人心愛西門慶,說破咽喉總是閑。
\end{myquote}

張四羞慚歸家,與婆子商議。單等婦人起身,指着外甥楊宗保,要攔奪婦人箱籠。

話休饒舌。到二十四日,西門慶行禮。請了他吴大妗來,坐轎押擔。衣服頭面、四季袍兒、羹果茶餅、布絹紬綿,約有二十餘擔。這邊請他姑娘並他姐姐,接茶陪侍,不必細說。到二十六日,請十二位高僧念經,做水陸燒靈,都是他姑娘一力張主。這張四,臨婦人起身那當日,請了幾位街坊衆鄉鄰,來和婦人講話。那日,薛嫂正引着西門慶家小廝伴當,僱了幾個閒漢,並守備府裏討的一二十名軍牢,正進來搬擡婦人牀帳、嫁粧箱籠。被張四攔住,説道:「保山,且休擡!有話講。」一面邀請了街坊鄰舍進來坐下。張四先開言說:「列位高鄰聽着!大娘子在這裏,不該我張龍說。你家男子漢楊宗錫,與你這小叔楊宗保,都是我外甥,是我的姐姐養的。今日不幸他死了。空掙了一場錢,有人主張着你。這是親戚難管你家務事。這也罷了!爭奈第二個外甥楊宗保年幼,一個業障都在我身上。他是你男子漢一母同胞所生,莫不家當沒他的份兒?今日對着列位高鄰在這裏,你手裏有東西、沒東西,嫁人去,也難管你。只把你箱籠打開,眼同衆人看一看,你還擡去,我不留下你的,只見個明白。娘子你意下如何?」婦人聽言,一面哭起來,説道:「衆位聽着,你老人家差矣!奴不是歹意謀死了男子漢,今日腆羞臉又嫁人。他手裏有錢沒錢,人所共知,就是積攢了幾兩銀子,都使在這房子上。房兒我沒帶去,都留與小叔。家活等件,分毫不動。就是外邊有三四百兩銀子欠帳,文書合同已都交與你老人家,陸續討來家中盤纏。再有甚麽銀兩來?」張四道:「你沒銀兩也罷。如今只對着衆位,打開箱籠,有沒有看一看,你還拿了去,我又不要你的。」婦人道:「莫不奴的鞋脚,也要瞧不成?」

正亂着,只見姑娘拄拐自後而出。衆人便道:「姑娘出來。」都齊聲唱喏。姑娘還了萬福,陪衆人坐下。姑娘開口:「列位高鄰在上,我是他的親姑娘,又不隔従,莫不沒我說處?死了的也是侄兒,活着的也是侄兒,十個指頭咬着都痛。如今休說他男子漢手裏沒錢,他就是有十萬兩銀子,你只好看他一眼罷了。他身邊又無出,少女嫩婦的,你攔着不教他嫁人,留着他做什麽?」衆街鄰高聲道:「姑娘見得有理!」婆子道:「難道他娘家陪的東西也留下他的不成?他背地又不曾私自與我什麽。說我護他,也要公道。不瞞列位說,我這侄兒媳婦平日有仁義,老身捨不得他,好溫克性兒。不然,老身也不管着他。」那張四在傍,把婆子瞅了一眼,說道:「你好失心兒,鳳凰無寳處不落!」只這一句話,道着這婆子真病。須臾怒起,紫漒了面皮,扯定張四大駡道:「張四,你休胡言亂語!我雖不能不才,是楊家正頭香主,你這老油嘴,是楊家那膫子㒲的?」張四道:「我雖是異姓,兩個外甥是我姐姐養的。你這老咬蟲,女生外向,行放火又一頭放水!」姑娘道:「賤沒廉耻老狗骨頭!他少女嫩婦的,留着他在屋裏,有何算計?既不是圖色慾,便欲起謀心,將錢肥己!」張四道:「我不是圖錢,爭奈楊宗保是我姐姐養的。有差遲,都是我!過不得日子,不是你!這老殺才,搬着大,引着小,黄貓兒黑尾!」姑娘道:「張四,你這老花根!老奴才!老粉嘴!你恁騙口張舌的,好淡扯!到明日死了時,不使個繩子扛子!」張四道:「你這嚼舌根老淫婦,掙將錢來焦尾靶!怪不得恁無兒無女!」姑娘急了,罵道:「張四賊!老娼根!老猪狗!我無兒無女,強似你家媽媽子穿寺院養和尚、㒲道士!你還在睡裏夢裏。」當下兩個差些兒不曾打起來。多虧衆鄰舍勸住,說道:「老舅,你讓姑娘一句兒罷。」薛嫂兒見他二人嚷打一團,領率西門慶家小廝伴當,並發來衆軍牢趕入,鬧裏七手八脚,將婦人牀帳、裝奩、箱籠,搬的搬,擡的擡,一陣風都搬去了。那張四氣的眼大大的,敢怒而不敢言。衆鄰舍見不是事,安撫了一囬,各人都散了。

到六月初二日,西門慶一頂大轎,四對紅紗燈籠,他這邊姐姐孟大姨送親。他小叔楊宗保,頭上扎着髻兒,穿着青紗衣,撒騎在馬上,送他嫂子成親。西門慶答賀了他一疋錦緞、一柄玉縧兒。蘭香小鸞兩個丫頭,都跟了來舖牀疊被。小廝琴童,方年十五歲,亦帶過來伏侍。到三日,楊姑娘家,並婦人兩個嫂子孟大嫂二嫂,都來做三日。西門慶與他楊姑娘七十兩銀子、兩疋尺頭。自此親戚來往不絶。西門慶就把西廂房裏收拾三間與他做房,排行第三,號玉樓。令家中大小,都隨着叫三娘。到晚,一連在他房中歇了三夜。正是:銷金帳裏,依然兩個新人;紅錦被中,現出兩般舊物。有詩為證:
\begin{myquote}
怎覩多情風月標,教人無福也難消。

風吹列子歸何處?夜夜嬋娟在柳梢。
\end{myquote}

畢竟未知後來何如,且聽下囘分解。

