\includepdf[pages={79,80},fitpaper=false]{tst.pdf}
\chapter*{第四十囬 \\抱孩童瓶兒希寵 粧丫鬟金蓮市愛}
\addcontentsline{toc}{chapter}{第四十囬 抱孩童瓶兒希寵 粧丫鬟金蓮市愛}
\markboth{\titlename}{第四十囬 抱孩童瓶兒希寵 粧丫鬟金蓮市愛}


\begin{myquote}
善事雖好做,無心近不得。

你若做好事,別人分不得。

經卷積如山,無緣看不得。

財錢過壁堆,臨危將不得。

靈前好供奉,起來吃不得。

兒孫雖滿堂,死來替不得。
\end{myquote}

話説當夜月娘和王姑子一炕睡,王姑子因問月娘:「你老人家怎的就沒見點喜事兒?」月娘道:「又説喜事哩!前日八月裏,因買了對過喬大户房子,平白俺們都過去看,上他那樓梯,一脚躡滑了,把個六七個月身扭掉了。至今再誰見什麽孩子來!」王姑子道:「我的奶奶,六七個月也成形了。」月娘道:「半夜裏掉在榪子裏,我和丫頭點燈撥着瞧,倒是個小廝兒。」王姑子道:「我的奶奶,可惜了。怎麽來扭着了?還是胎氣坐的不牢!」月娘道:「我自上他家樓,梯子窄趔,不知怎的一脚滑下來!還虧了孟三姐,一手扶住我,不然一直掉下來了。」王姑子道:「你老人家養出個兒來,強如别人。你看他前邊六娘,進門多少時兒,倒生了個兒子,何等的好!」月娘道:「他各人的兒女,隨天罷了。」王姑子道:「也不打緊。俺們同行一個薛師父,一紙好符水薬。前年陳郎中娘子,也是中年無子,常時小産了幾胎,白不存。也是吃了薛師父符薬,如今生了好不醜滿抱的小廝兒!一家兒歡喜的了不得。只是用着一件物件兒難尋。」月娘問道:「什麽物件兒?」王姑子道:「用着頭生孩子的衣胞。拿酒洗了,燒成灰兒,拌着符薬,揀壬子日,人不知鬼不覺,空心用黄酒吃了。算定日子兒不錯,至一個月,就坐胎氣,好不准!」月娘道:「這師父是男僧女僧?在那裏住?」王姑子道:「他也是俺女僧,也有五十多歲。原在地藏庵兒住來,如今搬在南首裏法華庵兒做首座。好不有道行!他好少經典兒!又會講説〈金剛科儀〉,各樣因果寳卷,成月説不了。專在大人家行走,要便接了去,十朝半月不放出來。」月娘道:「你到明日請他來走走。」王姑子道:「我知道。等我替你老人家討了這符薬來着!止是這一件兒難尋。這裏沒尋處,恁般如此,你不如把前頭這孩子的房兒,借情刨出來使了罷。」月娘道:「緣何損別人,安自己的!我與你銀子,你替我慢慢另尋便了。」王姑子道:「這個倒只是問老娘尋,他纔有。我替你整治這符水,你老人家吃了,管情就有。難得你明日另養出來,隨他多少,十個明星當不的月!」月娘吩咐:「你却休對人説。」王姑子道:「好奶奶,儍了我,肯對人説!」説了一囬,各人都睡了。一宿晚景題過。

到次日,西門慶打廟裏來家。月娘纔起來梳頭。玉蕭接了衣服,坐下。月娘因説:「昨日家裏六姐等你來上壽,怎的就不來了?」西門慶悉把醮事未了,「吳親家晚夕費心,擺了許多桌席。吴大舅先來了,留住我和花大哥、應二哥、謝希大,兩個小優兒彈唱着,俺們吃了半夜酒。今早我便先進城來了,應二哥他三個還吃酒哩。昨日甚是難為吴親家,破費了許多錢!」告訴了一回。玉蕭遞茶吃了,也没往衙門裏去,走到前邊書房裏,歪在床上就睡着了。落後潘金蓮李瓶兒梳了頭,抱着孩子出來,都到上房陪着吃茶。月娘向李瓶兒道:「他爹來了這一日,在前頭哩。我教他吃茶食,他不吃。丫頭有了飯了。你把你家小道士,替他穿上衣裳,抱到前頭與他爹瞧瞧去。」潘金蓮道:「我也去,等我替道士兒穿衣服。」於是戴上銷金道髻兒,穿上道衣,帶了項牌符索,套上小鞋襪兒,金蓮就要奪過去。月娘道:「教他媽媽抱罷,况是你這蜜褐色挑繡裙子不耐汚,撒上點子臢倒了不成!。」於是李瓶兒抱定官哥兒,潘金蓮便跟着,來到前邊西廂房内。書童見他二人掀簾,連忙就躱出來了。金蓮見西門慶臉朝裏睡炕床上,指着孩子説:「老花子,你好睡!小道士兒自家來請你來了。大媽媽房裏擺下飯,教你吃去。你還不快起來?還推睡兒!」那西門慶吃了一夜酒的人,丢倒頭,那顧天高地下,鼾睡如雷。金蓮與李瓶兒一邊一個,坐在床上,把孩子放在他面前。怎禁的鬼混,不一時,把西門慶弄醒了。睜開眼,看見官哥兒在面前,頭上戴着銷金道髻兒,身穿小道衣兒,項圈符索,喜歡的眉開眼笑。連忙接過來,抱到懷裏,與他親個嘴兒。金蓮道:「好乾淨嘴頭子,就來親孩兒!小道士兒吴應元,你噦他一口!你説:昨日在那裏使牛耕地來?今日乏困的你這樣的,大白日強覺!昨日叫五媽只顧等着你,你恁大膽,不來與五媽磕頭!」西門慶道:「昨日醮事散的晚。晚夕謝將,又整酒吃了一夜,今日到這咱時分還一頭酒。在這裏睡回,還要往尚擧人家吃酒去。」金蓮道:「你不吃酒去罷了。」西門慶道:「他家従昨日送了帖兒來,不去,惹人家不怪?」金蓮道:「你去,晚夕早些兒來家,我等着你哩。」李瓶兒道:「他大媽媽擺下飯了,又做了些酸笋湯,請你吃飯去哩。」西門慶道:「我心裏還不待吃,等我去呵些湯罷。」於是起來往後邊去了。

這潘金蓮見他去了,一屁股就坐在床上正中間,脚蹬着地爐子,説道:「這原來是個套炕子。」伸手摸了摸褥子裏,説道:「倒且是燒的滚熱的炕兒。」瞧了瞧旁邊桌上,放着個烘硯瓦的銅絲火籠兒,隨手取過來,叫:「李大姐,那邊香几兒上牙盒裏盛的甜香餅兒,你取些來與我。」一面揭開了,拿幾個在火爐内。一面夾在襠裏,拿裙子裹的嚴嚴的,且薰熱身上。坐了一囬,李瓶兒説道:「咱進去罷,只怕他爹吃了飯出來。」金蓮道:「他出來不是,怕他麽?」於是二人抱着官哥,進入後邊來。良久,西門慶吃了飯,吩咐排軍備馬,午後往尚擧人家吃酒去了。潘姥姥先去了。

且説晚夕王姑子要家去,月娘悄悄與了他一兩銀子,叫他休對大師父説,好歹請薛姑子帶了符薬來。王姑子接了銀子,和月娘説:「我這一去,只過十六日兒纔來罷。就替你尋了那件東西兒來。」月娘道:「也罷,你只替我幹的停當,我還謝你。」於是作辭去了。看官聽説:但凡大人家,似這樣僧尼牙婆,決不可擡擧在深宫大院相伴着婦女,俱以講天堂地獄、談經說典為由,背地裏説條念款,送暖偸寒,甚麽事兒不幹出來!十個九個,都被他送上災厄。有詩為證:

\begin{myquote}
最有緇流不可言,深宫大院哄嬋娟。

此輩若皆成佛道,西方依舊黑漫漫!
\end{myquote}

卻説金蓮,晚夕趁月娘房裏陪着衆人坐的。走到鏡臺前把䯼髻摘了,打了個盤頭揸髻;把臉搽的雪白,抹的嘴唇兒鮮紅;戴着兩個金燈籠墜子,貼着三個面花兒,帶着紫銷金箍兒;尋了一套大紅織金襖兒,下着翠藍緞子裙:要裝丫頭哄月娘衆人耍子。叫將李瓶兒來與他瞧,把李瓶兒笑的前仰後合,説道:「姐姐,你裝扮起來,活像個丫頭!等我往後邊去,——我那屋裏有紅布手巾,替你蓋着頭。——對他們只説他爹又尋了個丫頭,唬他們唬,管定就信了!」春梅打着燈籠,在頭裏走。走到儀門首,撞見陳經濟,笑道:「我道是誰來?這個就是五娘幹的營生。」李瓶兒叫道:「姐夫,你過來,等我和你説了着。你先進去,見他們只如此如此,這般這般。」經濟道:「我有法兒哄他。」於是先走到上房裏,衆人都在炕上坐着吃茶。經濟道:「娘,你看爹!平白裏叫薛嫂兒使了十六兩銀子,買了人家一個二十五歲會彈唱的姐兒,剛纔拿轎子送將來了。」月娘道:「眞個?薛嫂兒怎不先來對我説?」經濟道:「他怕你老人家駡他,送轎子到大門首,他就去了。丫頭便教他們領進來了。」大妗子還不言語。楊姑娘道:「官人有這幾房姐姐夠了,又要他來做什麽?」月娘道:「好奶奶,你禁的有錢,就買一百個,有什麽多?俺們都是老婆當軍,在這屋裏充數兒罷了!」玉簫道:「等我瞧瞧去。」只見月亮地裏,原來春梅打燈籠,落後叫了來安兒小廝打着,和李瓶兒後邊跟着,金蓮搭着蓋頭,穿着紅衣服進來。慌的孟玉樓李嬌兒都出來看。良久,進入房裏。玉簫挨在月娘邊,説道:「這個是主子,還不磕頭哩!」一面揭了蓋頭。那潘金蓮插燭也似磕下頭去,忍不住撲矻的笑了。玉樓道:「好丫頭,不與你主子磕頭,且笑!」月娘也笑了,説道:「這六姐成精死了罷!把俺們哄的信了。」玉樓道:「大娘,我不信。」楊姑娘道:「姐姐,你怎的見出來不信?」玉樓道:「俺六姐平昔磕頭,也學的那等,磕了頭起來,倒退兩步纔拜。」楊姑娘道:「還是姐姐看的出來,要着老身,就信了。」李嬌兒道:「我也就信了。剛纔不是揭蓋頭,他自家笑,還認不出來。」正説着,只見琴童兒抱進毡包來,説:「爹來家了。」孟玉樓道:「你且藏在明間裏,等爹進來,等我哄他哄。」

不一時,西門慶來到。楊姑娘、大妗子出去了。進入房内,椅子上坐下。月娘在旁不言語。玉樓道:「今日薛嫂兒轎子送人家一個二十歲丫頭來,説是你教他送來,要他的。你恁許大年紀,前程也在身上,還幹這勾當?」西門慶笑道:「我那裏教他買丫頭來?信那老淫婦哄你哩。」玉樓道:「你問大姐姐不是,丫頭也領在這裏。我不哄你,你不信,我就叫出來你瞧。」於是叫玉簫:「你拉進那新丫頭來見你爹。」那玉簫掩着嘴兒笑,又不敢去拉。前邊走了走兒,又囬來了,説道:「他不肯來。」玉樓道:「等我去拉。恁大膽子的奴才,頭兒沒動,就扭主子?也是個不聽指教的!」一面走到明間内。只聽説道:「怪行貨子!我不好罵的。人不進去,只顧拉人,拉的手脚兒不着地。」玉樓笑道:「好奴才,誰家使的你恁沒規矩,不進來見你主子磕頭?」一面拉進來。西門慶燈影下睜眼觀看,却是潘金蓮打着揸髻裝丫頭,笑的眼沒縫兒。那金蓮就坐在傍邊椅子上。玉樓道:「好大膽丫頭,新來乍到,就恁少調失教的,大剌剌對着主子坐着!還撅臭與他這個主子兒了?」月娘笑道:「你趁着你主子來家,與他磕個頭兒罷。」那金蓮也不動,走到月娘裏間屋裏,一頓把簪子拔下,戴上䯼髻出來。玉樓道:「好淫婦,討了誰上頭話,就戴上䯼髻了!」衆人又笑了一囬。

月娘告訴西門慶説:「今日喬親家那裏使喬通送了六個帖兒來,請俺們吃看燈酒。咱到明日,不先送些禮兒去?」教玉簫拿帖兒與西門慶瞧。見上面寫着:

\begin{myquote}[\markfont]
「十二日寒舍薄具菲酌,奉屈魚軒。仰冀賁臨,不勝榮幸。右啟大德望西門大親家老夫人粧次。

\raggedleft{{\kaishu(下書)}眷末喬門鄭氏歛袵拜。」}
\end{myquote}

西門慶看畢,説道:「明早叫來興兒買四樣餚品,一罈南酒,送了去就是了。到明日咱家發柬,十四日也請他娘子,并周守備娘子、荆都監娘子、夏大人娘子、張親家母。大妗子也不必家去了。教賁四叫將花兒匠來,做幾架煙火;王皇親家一起扮戲的小廝叫來扮〈西廂記〉的。你們往院中,再把吳銀兒李桂兒接了來。你們在家看燈吃酒,我和應二哥、謝子純,往獅子街樓上吃酒去。」説畢,不一時放下桌兒,安排酒上來。潘金蓮遞酒,衆姊妹相陪,吃了一囬。

西門慶因見金蓮裝扮丫頭,燈下艷粧濃抹,不覺淫心蕩漾,不住把眼色遞與他。這金蓮就知其意,行陪着吃酒,就到前邊房裏,去了冠兒,挽着杭州攢,重勻粉面,復點朱唇。原來早在房中,先預備下一桌酒,齊整菓菜,等西門慶進房,婦人還要私己與他遞酒。不一時,西門慶果然來到,見婦人還挽起雲髻來,心中甚喜。摟着他坐在椅子上,兩個説笑。不一時,春梅收拾上酒菜來,婦人従新與他遞酒。西門慶道:「小油嘴兒,頭裏已是遞過罷了,又教你費心!」金蓮笑道:「那個大夥裏酒兒不算,這個是奴家業兒,與你遞鍾酒兒,年年累你破費,你休抱怨。」把西門慶笑的没眼縫兒,連忙接了他酒,摟在懷裏膝蓋兒坐的。春梅斟酒,秋菊拿菜兒。金蓮道:「我問你,到十二日喬家請俺們都去,只教大姐姐去?」西門慶道:「他既是下帖兒都請你們,如何不去?到明日,叫奶子抱了哥兒也去走走,省的家裏尋他娘哭。」金蓮道:「大姐姐他們都有衣裳穿,我老道只是知數的那幾件子,沒件好當眼的。你把南邊新治來那衣服,一家分散幾件子,裁與俺們穿了罷。只顧放着,敢生小的兒也怎的?到明日咱家擺酒,請衆官娘子,俺們也好見他,不惹人笑話!我常時説着,你把臉兒憨着。」西門慶笑道:「既是恁的,明日叫了趙裁來,與你們裁了罷。」金蓮道:「及至明日叫裁縫做,只差兩日兒,做着還遲了哩。」西門慶道:「對趙裁説,多帶幾個人來,替你們趲造兩三件出來,就够了。剩下別的,慢慢再做也不遲。」金蓮道:「我早對你説過,好歹揀兩套上色兒的與我。我難像他們都有,我身上你沒與我做什麽大衣裳。」西門慶笑道:「賊小油嘴兒,随處掐個尖兒!」兩個説話飲酒,到一更時分,方上床。兩個如被底鴛鴦,帳中鸞鳳,畫樓燕語,不肯即休,整狂了半夜。

到次日,西門慶衙門中囬來,開了箱櫃,打開出南邊織造的夾板羅緞尺頭來。使小廝叫將趙裁來,每人做件粧花通袖袍兒,一套遍地錦衣服,一套粧花衣服。惟月娘是兩套大紅通袖遍地錦袍兒,四套粧花衣服。在捲棚内,一面使琴童兒叫趙裁去。這趙裁正在家中吃飯,聽的西門慶宅中呌,連忙丢下飯碗,帶着剪尺就走。時人有幾句誇讚這趙裁好䖏:

\begin{myquote}
我做裁縫姓趙,月月主顧來叫。

針線緊緊隨身,剪尺常掖靴靿。

幅摺趕空走儹,截彎病除手到。

不論上短下長,那管襟扭領拗?

每日肉飯三餐,兩頓酒兒是要。

剪截門首常空,一月不脱三廟。

有錢老婆嘴光,無時孩子亂呌。

不拘誰家衣裳,且交印鋪睡覺。

隨你催討終期,只拿口兒支調。

十分要緊騰挪,又將後來頂倒。

問你有甚高強?只是一味靠落!
\end{myquote}

不一時走到,見西門慶坐在上面,連忙磕了頭。桌上鋪着氈條,取出剪尺來,先裁月娘的:一件大紅遍地錦五彩粧花通袖百獸朝麒麟補子緞袍兒,一件玄色五彩遍地錦葫蘆樣鸞鳳穿花羅袍,一套大紅緞子遍地金通袖麒麟補子襖兒,翠藍寬拖遍地金裙,一套沉香色粧花補子遍地錦羅襖兒,大紅金枝緑葉百花拖泥裙。其餘李嬌兒、孟玉樓、潘金蓮、李瓶兒四個,都裁了一件大紅五彩通袖粧花錦鷄緞子袍兒,兩套粧花羅緞衣服。孫雪娥只是兩套,就沒與他袍兒。須臾,共裁剪三十件衣服,兑了五兩銀子,與趙裁做工錢。一面叫了十來個裁縫,在家趲造,不在話下。正是:金鈴玉墜裝閨女,錦綺珠翹飾嬌娃。

畢竟未知後來如何,且聽下囬分解。

\part*{夢梅館校本《金瓶梅詞話》卷之五}
\addcontentsline{toc}{part}{夢梅館校本《金瓶梅詞話》卷之五}

