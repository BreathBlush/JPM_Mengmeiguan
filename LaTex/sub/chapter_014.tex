\includepdf[pages={27,28},fitpaper=false]{tst.pdf}
\chapter*{第十四囬 \\花子虚因氣喪身 李瓶兒送奸赴會}
\addcontentsline{toc}{chapter}{第十四囬 花子虚因氣喪身 李瓶兒送奸赴會}
\markboth{{\titlename}卷之二}{第十四囬 花子虚因氣喪身 李瓶兒送奸赴會}


\begin{myquote}
眼意心期未即休,不堪撚弄玉搔頭。

春囬笑臉花含媚,淺蹙蛾眉柳带愁。

粉暈桃腮思伉儷,寒生蘭室盼綢繆。

何時得遂相如志,不讓文君詠白頭。
\end{myquote}

話説一日,吳月娘心中不快,吳大妗子來看,月娘留他住兩日。正陪着在房中坐的,忽見小廝玳安抱進毡包來説:「爹來家了。」吳大妗子便往李嬌兒房裏去了。少頃,西門慶進來,脱了衣服坐下。小玉拿茶來也不吃。月娘見他面帶幾分憂色,便問:「你今日會茶來家忒早。」西門慶道:「今該常時節會。他家沒地方,請了俺們在門外五里原永福寺去耍子。有花二哥,邀了應二哥,俺們四五個,往院裏鄭愛香兒家吃酒。正吃在熱鬧䖏,忽見幾個做公的進來,不由分説,把花二哥拿的去了,把衆人唬的吃了一驚。我便走到李桂姐家躱了半日。不放心,使人打聽,原來是花二哥内臣家,房族中花大花三花四告家財,在東京開封府遞了狀子,批下來着落本縣拿人。俺們纔放心,各人散歸家來。」月娘聞言便道:「正該!鎭日跟着這夥人喬神道,想着個家?只在外邊胡撞。今日只當弄出事來,纔是個了手。你如今還不心死,到明日不吃人爭鋒廝打,羣到那裏打個爛羊頭,你肯断絶了這條路兒!正經家裏老婆好言語説着,你肯聽?只是院裏淫婦,在你跟前説句話兒,你側着個驢耳朶聽他。正是:家人説着耳邊風,外人説着金字經。」西門慶笑道:「誰人敢七個頭八個膽打我?」月娘道:「你這行貨子,只好家裏説嘴頭子罷了。若上場兒,唬的看出那嘴舌來了。」

正説着,只見玳安走來説:「隔壁花二娘家使了天福兒來,請爹過那邊去説話。」這西門慶得不的一聲兒,趔趄脚兒就往外走。月娘道:「明日沒的敎人扯把你!」西門慶道:「切鄰間,不妨事。我去到那裏,看他有甚麽話説。」當下走過花子虚家來。李瓶兒使小廝請到後邊説話。只見婦人羅衫不整,粉面慵粧,従房裏出來,臉唬的蠟渣也似黄,跪着西門慶,再三哀告道:「大官人!沒奈何,不看僧面看佛面。常言道:家有患難,鄰保相助。因奴拙夫不聽人言,把着正經家事兒不理,只在外信着人,成日不着家。今日只當吃人暗算,弄出這等事來。着緊這時節方對小廝説將來,敎我尋人情救他。我一個女婦人家,沒脚蟹,那裏尋那人情去?發恨起將來,想着他恁不依説,拿到東京打的他爛爛的不虧。只是難為過世老公公的名子。奴沒奈何,請將大官人來,央及大官人,把他不要題起罷。千萬只看奴之薄面,有人情,好歹尋一個兒,只休敎他吃凌逼便了。」西門慶見婦人下禮,連忙道:「嫂子請起來,不妨!今日我還不知因為了甚勾當。俺們都在鄭家吃酒,只見幾個做公的人,把哥拿的到東京去了。」婦人道:「正是一言難盡。此是俺過世老公公連房大姪兒花大花三花四,與俺家都是叔伯兄弟。大哥喚做花子由,三哥喚花子光,第四個的叫花子華,俺這個名花子虚,都是老公公嫡親姪兒。雖然老公公掙下這一份家財,見俺這個兒不成器,従廣南囬來,把東西只交付與我手裏收着。着緊還打躺棍兒,那別的越發打的不敢上前。去年老公公死了,這花大花三花四也搶分了些床帳家伙去了,只是一分銀子兒沒曾得。我便説多少與他些也罷了。俺這個成日只在外邊胡幹,把正經事兒通不理一理兒。今日手暗不透風,卻教人弄下來了。」説畢,放聲大哭。西門慶道:「嫂子放心。我只道是甚麽事來,原來是房分中告家財事!這個不打緊處。既是嫂子吩咐,哥的事兒就是我的事,我的事就如哥的事一般,隨問怎的,我在下謹領。」婦人問道:「官人若肯下顧時,又好了。請問尋分上用多少禮兒,奴好預備。」西門慶道:「也用不多。聞得東京開封府楊府尹乃蔡太師門生。蔡太師與我這四門親家楊提督都是當朝天子面前説得話的人。拿兩個分上齊對楊府尹説,有個不依的?不拘多大事情也了了。如今倒是蔡太師用些禮物。那提督楊爺,與我舍下有親,他肯受禮?」

婦人便往房裏開箱子,搬出六十錠大元寳,共計三千兩,教西門慶收去,尋人情上下使用。西門慶道:「只消一半足矣,何消用得許多?」婦人道:「多的大官人收去。奴床後邊有四口描金箱櫃,蟒衣玉帶、帽頂縧環、提繫條脱、値錢珎寳翫好之物,一發大官人替我收去,放在大官人那裏,奴用時取去。趂早奴不思個防身之計,信着他,往後過不出好日子來。眼見得三拳敵不得四手,到明日沒的把這些東西兒吃人暗算搶奪了去,坑閃得奴三不歸。」西門慶道:「只怕花二哥來家尋問怎了?」婦人道:「這個都是老公公在時,梯己交與奴收着的,他一字不知。大官人只顧收去。」西門慶説道:「既是嫂子恁説,我到家叫人來取。」於是一直來家與月娘商議。月娘説:「銀子便用食盒叫小廝擡來。那箱籠東西若従大門裏來,敎兩邊街坊看着不惹眼?必須如此如此,夜晚打牆上過來,方隱密些。」西門慶聽言大喜,即令來旺兒、玳安兒、來興、平安,四個小廝,兩架食盒,把三千兩金銀先擡來家。然後到晚夕月上的時分,李瓶兒那邊同兩個丫鬟迎春綉春,放桌凳把箱櫃挨到牆上,西門慶這邊止是月娘金蓮春梅,用梯子接着。牆頭上鋪苫毡條,一個個打發過來,都送到月娘房中去。你説,有這等事?要得富,險上做。有詩為證:

\begin{myquote}
富貴自是福來投,利名還有利名憂。

命裏有時終須有,命裏無時莫強求。
\end{myquote}

西門慶收下他許多軟細金銀寳物,鄰舍街坊俱不得知道。連夜打點馱裝停當,求了他親家陳宅一封書,差家人上東京。一路朝登紫陌,暮踐紅塵,有日到了東京城内,交割楊提督書禮,轉求内閣蔡太師,柬帖下與開封府楊府尹。這府尹名喚楊時,別號龜山,乃陝西弘農縣人氏。由癸未進士陞大理寺卿,今推開封府尹,極是個清廉的官。况蔡太師是他舊時座主,楊戩又是當道時臣,如何不做分上?這裏西門慶又預星夜捎書花子虚知道説:「人情都到了。等當官問你家財下落,只説都花費無存,止是房産莊田見在。」

卻説一日楊府尹陞廳,六房官吏俱都祇候。但見:

\begin{myquote}
為官清正,作事廉明。每懷惻隠之心,常存仁慈之念。爭田奪地,辨曲直而後施行;鬬毆相爭,審輕重方使决断。閒則撫琴會客,忙應分理民情。雖然京兆宰臣官,果是一邦民父母。
\end{myquote}

當日楊府尹陞廳,監中提出花子虚來,傳一干人上廳跪下,審問他家財下落。那花子虚口口只説:「自従老公公死了,發送念經都花費了。止有宅舍兩所,莊田一處見在。其餘床帳家伙物件,俱被族人分搶一空。」楊府尹道:「你們内官家財無可稽考,得之易,失之易。既是花費無存,批仰清河縣,委官將花太監住宅二所、莊田一處,估價變賣,分給花子由等三人囬繳。」子由等還要當廳跪稟,還要監追子虚,要別項銀兩下落。被楊府尹大怒,都喝下來了,説道:「你這廝少打!當初你那内相一死之時,你們不告,做甚麽來?如今事情已往,又來騷擾,費去我紙筆。」於是把花子虚一下兒也沒打,批了一道公文,押發清河縣前來估計莊宅,不在話下。

早有西門慶家人來保打聽這消息,星夜囬來報知西門慶。西門慶聽的楊府尹見了分上,放出花子虚來家,滿心歡喜。這裏李瓶兒請過西門慶去計議,要敎西門慶:「拿幾兩銀子,買了這所住的宅子罷。到明日,奴不久也是你的人了。」西門慶歸家,與吳月娘商議。月娘道:「隨他當官估價賣多少,你不可承攬要他這房子。恐怕他漢子一時生起疑心來怎了?」這西門慶聽記在心。那消幾日,花子虚來家,清河縣委下楽縣丞丈估:計太監大宅一所,坐落大街安慶坊,値銀七百兩,賣與王皇親為業;南門外莊田一處,値銀六百五十五兩,賣與守備周秀為業。止有住居小宅,値銀五百四十兩,因在西門慶緊隔壁,沒人敢買。花子虚再三使人來説,西門慶只推沒銀子,延挨不肯上帳。縣中緊等要囬文書,李瓶兒急了,暗暗使過馮媽媽來對西門慶説,敎拿他寄放的銀子,兑五百四十兩買了罷。這西門慶方纔依允,當官交兌了銀兩。花大等都畫了字。連夜做文書,囬了上司。共該銀一千八百九十五兩,三人均分訖。

花子虚打了一場官司出來,沒分的絲毫,把銀兩房舍莊田又沒了,兩箱内三千兩大元寳又不見蹤影,心中甚是焦燥。因問李瓶兒查算西門慶那邊使用銀兩下落:「今剩下多少,還要凑着添買房子。」反吃婦人整罵了四五日,駡道:「呸!魍魎混沌!你成日放着正事兒不理,在外邊眠花臥柳不着家,只當被人所算,弄成圈套拿在牢裏,使將人來對我説,敎我尋人情。奴是個女婦人家,大門邊兒也没走,能走不能飛,曉得甚麽?認的何人?那裏尋人情?渾身是鐵打的多少釘兒?替你到處求爹爹告奶奶,甫能尋得人情。平昔不種下,急流之中誰人來管你?多虧了他隔壁西門慶,看日前相交之情,大冷天,刮得那黄風黑風,使了家下人往東京去,替你把事兒幹得停停當當的。你今日了畢官司出來,兩脚踏住平川地,得命思財,瘡好忘痛,來家還問老婆找起後帳兒來了,還説有也沒。你過眼:有你寫來的帖子見在!沒你的手字兒,我擅自拿出你的銀子尋人情——抵盗與人便難了。」花子虚道:「可知是我的帖子來説。實指望還剩下些,咱凑着買房子過日子,往後知數拳兒了。」婦人道:「呸,濁材料!我不好駡你的。你早仔細好來!囷頭兒上不算計,囷底兒下卻算計!千也説使多了,萬也説使多了。你那三千兩銀子,能到的那裏?蔡太師楊提督好小食腸兒?不是恁大情囑的話,平白拿了你一場,當官蒿條兒也沒曾打在你這王八身上,好好放出來,敎你在家裏恁説嘴!人家不屬你管轄,不是你甚麽着疼的親故,平白怎替你南上北下走跳,使錢救你?你來家該擺席酒兒,請過人來知謝人一知謝兒;還一掃箒掃得人光光的,問人找起後帳兒來了。」幾句連搽帶罵,駡的子虚閉口無言。

到次日,西門慶使了玳安送了一分禮來與子虚壓驚。子虚這裏安排了一席,叫了兩個妓者,請西門慶來知謝,就找着問他銀兩下落。依着他西門慶這邊還找過幾百兩銀子與他凑買房子。李瓶兒不肯,暗地使過馮媽媽子過來,對西門慶説:「休要來吃酒,開送了一篇花帳與他,只説銀子上下打點都使沒了。」花子虚不識時務,還使小廝再三邀請。西門慶一逕躱的往院裏去了,只囬不在家。花子虚氣的發昏,只是跌脚。看官聽説:大抵只是婦人更變,不與男子漢一心,隨你咬折釘子般剛毅之夫,也難防測其暗地之事。自古男治外而女治内,往往男子之名都被婦人壞了者為何?皆由御之不得其道故也。要之,在乎夫唱婦隨,容德相感,緣分相投,男慕乎女,女慕乎男,庶可以保其無咎。稍有微嫌,輒顯厭惡。若似花子虚終日落魄嫖風,謾無紀律,而欲其内人不生他意,豈可得乎!正是:自意得其墊,無風可動搖。有詩為證:

\begin{myquote}
功業如將智力求,當年盜跖卻封侯。

行藏有義眞堪羡,好色無仁豈不羞?

浪蕩貪淫西門子,背夫水性女嬌流。

子虚氣塞柔腸断,他日冥司必報仇!
\end{myquote}

話休饒舌。後來子虚只擯凑了二百五十兩銀子,買了獅子街一所房屋居住。得了這口重氣,剛搬到那裏,不幸害了一塲傷寒。従十一月初旬睡倒在床上,就不曾起來。初時,李瓶兒還請的大街坊胡太醫來看,後來怕使錢,只挨着。一日兩,兩日三,挨到二十頭,嗚呼哀哉,断氣身亡。亡年二十四歲。那手下的大小廝天喜兒,從子虚病倒之時,拐了五兩銀子,走的無蹤跡。子虚一倒了頭,李瓶兒就使了馮媽媽請了西門慶過去,與他商議,買棺入殮,念經發送子虚到墳上埋葬。那花大花三花四,一般兒男婦也都來吊孝。送殯囬來,各都散了。西門慶那日也敎吳月娘辦了一張桌席,與他山頭祭奠。當日婦人轎子歸家,也囬了一個靈位供養在房中。雖是守靈,一心只想着西門慶。従子虚在時,就把兩個丫頭敎西門慶要了,子虚死後,越發通家往還。

一日,正月初九日,李瓶兒打聽是潘金蓮生日,未曾過子虚五七,就買禮坐轎子,穿白綾襖兒,藍織金裙,白苧布䯼髻,珠子箍兒,來與金蓮做生日。馮媽媽抱毡包,天福兒跟轎,進門就先與月娘插燭也似磕了四個頭,説道:「前日山頭,多勞動大娘受餓,又多謝重禮!」拜了月娘,又請李嬌兒孟玉樓拜見了。然後潘金蓮來到,説道:「這個就是五娘。」又磕下頭,一口一聲稱呼:「姐姐,請受奴一禮兒!」金蓮那裏肯受,相讓了半日,兩個還平磕了頭。金蓮又謝了他壽禮。又有吳大妗子、潘姥姥,都一同見了。李瓶兒便請西門慶拜見。月娘道:「他今日往門外玉皇廟打醮去了。」一面讓坐下,喚茶來吃了。良久,只見孫雪娥走過來,李瓶兒見他粧飾少次於衆人,便立起身來問道:「此位是何人?奴不知,不曾請見的。」月娘道:「此是他姑娘哩。」這李瓶兒就要慌忙行禮,月娘道:「不勞起動二娘,只拜平拜兒罷。」於是二人彼此拜畢,月娘就讓到房中,換了衣裳,吩咐丫鬟明間内放桌兒擺茶。湏臾圍爐添炭,酒泛羊羔,安排上酒來。當下吳大妗子潘姥姥李瓶兒上坐。月娘和李嬌兒主席,孟玉樓和潘金蓮打横,孫雪娥囬廚下照管,不敢久坐。月娘見李瓶兒鍾鍾酒都不辭,於是親自遞了一遍酒。又令李嬌兒衆人各遞酒一遍,頗嘲問他話兒。便説道:「花二娘搬的遠了,俺姊妹們離多會少,好不思想!二娘狠心,就不説來看俺們看兒?」孟玉樓便道:「二娘今日不是因與六姐做生日,還不來哩!」李瓶兒道:「好大娘三娘,蒙衆娘擡擧,奴心裏也要來。一來熱孝在身;二者拙夫死了,家下沒人。昨日纔過了他五七,不是怕五娘怪,還不敢來。」因問:「大娘貴降在幾時?」月娘道:「賤日早哩!」潘金蓮接過來道:「大娘生日是八月十五,二娘好歹來走走。」李瓶兒道:「不消説,一定都來。」孟玉樓道:「二娘今日與俺姊妹相伴一夜兒呵,不往家去罷了。」李瓶兒道:「奴可知也要和衆位娘敍些話兒。不瞞衆位娘説,小家兒人家,初搬到那裏,自從拙夫没了,家下没人。奴那房子後牆,緊靠着喬皇親花園,好不空。晚夕常有孤狸打磚掠瓦,奴又害怕。原有兩個小廝,那個大小廝又走了,止是這個天福兒小廝看守前門,後半截通空落落的。倒虧了這個老馮,是奴舊時人,常來與奴漿洗些衣裳,與丫頭做鞋脚,累他。」月娘因問:「老馮多大年紀?且是好個恩實媽媽兒,高言兒也沒句兒。」李瓶兒道:「他今年五十六歲,屬狗兒。男花女花沒有,只靠説媒度日。我這裏常管他些衣裳兒。昨日拙夫死了,叫過他來與奴做伴兒,晚夕同丫頭一炕睡。」潘金蓮嘴快,説道:「可又來,既有老馮在家裏看家,二娘在這過一夜兒也罷了。左右那花爹沒了,有誰管着你?」玉樓道:「二娘只依我,敎老馮囬了轎子,不去罷。」那李瓶兒只是笑,不做聲。

説話中間,酒過數巡。潘姥姥先起身往前邊去了。潘金蓮隨跟着他娘,往房裏去了。李瓶兒再三辭:「奴的酒夠了。」李嬌兒道:「花二娘怎的在他大娘三娘手裏吃過酒,偏我遞酒二娘不肯吃,顯的有厚薄。」於是拿大盃,只顧斟上。李瓶兒道:「好二娘,奴委的吃不去了,豈敢做假!」月娘道:「二娘你吃過此盃,畧歇歇兒罷。」那李瓶兒方纔接了,放在面前,只顧與衆人説話。孟玉樓見春梅立在傍邊,便問春梅:「你娘在前邊做甚麽哩?你去連你娘潘姥姥快請來。你説大娘請來陪你花二娘吃酒哩。」春梅去不多時,囬來道:「俺姥姥害身上疼,睡哩。俺娘在房裏匀臉,就來。」月娘道:「我倒也沒見,你倒是個主人家,把客人丢下,三不知往房裏去了。俺姐兒一日臉不知匀多少遭數,要便走的匀臉去了。諸般都好,只是有這些孩子氣。」正説着,只見潘金蓮上穿丁香色潞紬雁啣蘆花樣對衿襖兒,白綾豎領,粧花眉子,溜金蜂趕菊鈕扣兒;下着一尺寬海馬潮雲、羊皮金沿邊挑線裙子;大紅緞子白綾高底鞋,粧花膝褲;青寳石墜子,珠子箍——與孟玉樓一樣打扮。惟月娘是大紅緞子襖,青素綾披襖,沙綠紬裙,頭上帶着䯼髻、貂鼠臥兔兒。玉樓在席上,看見金蓮豔抹濃粧,鬢嘴邊撇着一根金壽字簪兒,従外搖擺將來,戯道:「五丫頭,你好人兒!今日是你個驢馬畜,把客人丢在這裏,你躱房裏去了。你可成人養的?」那金蓮笑嘻嘻向他身上打了一下。玉樓道:「好大膽的五丫頭!你不來遞一鍾兒?」李瓶兒道:「奴在三娘手裏吃了好少酒兒,已都夠了。」金蓮道:「他的手裏是他手裏帳,我也敢奉二娘一鍾兒。」於是揎起袖子,滿斟一大盃,遞與李瓶兒,只顧放着,不肯吃。月娘陪吳大妗子從房裏出來,看見金蓮陪着李瓶兒坐的,問道:「他潘姥姥怎的不來陪花二娘坐?」金蓮道:「俺媽害身上疼,在房裏歪着哩,叫他,不肯來。」月娘因看見金蓮鬢上撇着那壽字簪兒,便問:「二娘,你與六姐這對壽字簪兒是那裏打造的?倒且是好樣兒。到明日俺每人照樣也配恁一對兒戴。」李瓶兒道:「大娘既要,奴還有幾對兒,到明日每位娘都補奉上一對兒。此是過世老公公宫裏御前带出來的,外邊那裏有這樣範!」月娘道:「奴取笑,鬬二娘耍子。俺姊妹們人多,那裏有這些相送!」

衆女眷飲酒歡笑,看看日西時分,馮媽媽在後邊雪娥房裏管待酒飯,吃的臉紅紅的出來,催逼李瓶兒起身,——不起身好打發轎子囬去。月娘道:「二娘不去罷,叫老馮囬了轎子家去罷。」李瓶兒只説:「家裏無人,改日再奉看列位娘,有日子住哩。」孟玉樓道:「二娘好執古,俺衆人就沒些分上兒?如今不打發轎子,等住囬他爹來,少不的也要㽞二娘。」只這説話,逼迫的李瓶兒就把房門鑰匙遞與馮媽媽説道:「既是他衆位娘再三留我,顯的奴不識敬重。吩咐轎子囬去,敎他明日來接罷。你和小廝在家仔細門戶。」又叫過馮媽媽,附耳低言:「敎大丫頭迎春拿鑰匙開我床房裏頭一個箱子,小描金頭面匣兒裏,拿四對金壽字簪兒。你明日早送來,我要送四位娘。」那馮媽媽得了話,拜辭了月娘。月娘道:「吃了酒去!」馮媽媽道:「我剛纔在後邊姑娘房裏,酒飯都吃了。明日老身早來罷。」一面千恩萬謝出門,不在話下。

少頃李瓶兒不肯吃酒,月娘請到上房同大妗子一處吃茶坐的。忽見玳安小廝抱進毡包,西門慶來家,掀開簾子進來,説道:「花二娘在這裏?」慌的李瓶兒跳起身來,兩個見了禮,坐下。月娘叫玉簫與西門慶接了衣裳。西門慶便對吳大妗子李瓶兒説道:「今日會門外玉皇廟聖誕打醮,該我年例做會首。要不是,過了午齋我就來了。因與衆人在吳道官房裏算帳,七擔八柳,纏到這早晚。」因問:「二娘今日不家去罷了?」玉樓道:「二娘這裏再三不肯,要去。被俺衆姊妹強着留下。」李瓶兒道:「家裏没人,奴不放心。」西門慶道:「没的扯淡!這兩日好不巡夜的甚緊,怕怎的?但有些風吹草動,拿我個帖送與周大人,點到奉行。」又道:「二娘怎的冷清清坐着?用了些酒兒不曾?」孟玉樓道:「俺衆人再三奉勸二娘,二娘只是推不肯吃。」西門慶道:「你們不濟,等我奉勸二娘。二娘好小量兒!」李瓶兒口裏雖説「奴吃不去了」,只不動身。一面吩咐丫鬟,従新房中放桌兒,都是留下伺候西門慶的整下飯菜蔬、細巧菓仁,擺了一張桌子。吳大妗子知局,趐趫推不用酒,因往李嬌兒那邊房裏去了。當下李瓶兒上坐,西門慶㧱椅子関席。吳月娘在炕上跐着爐壶兒,孟玉樓潘金蓮兩邊打横。五人坐定,把酒來斟。也不用小鍾兒,要大銀衢花鍾子,你一盃,我一盞。常言:風流茶説合,酒是色媒人。吃來吃去,吃的婦人眉黛低横,秋波斜視。正是:兩朶桃花上臉來,眉眼旋開眞色婦。月娘見他二人吃得餳成一塊,言頗涉邪,看不上,往那邊房裏陪吳大妗子坐去了,由着他三個陪着。吃到三更時分,李瓶兒星眼乜斜,身立不住,拉金蓮往後邊凈手。西門慶走到月娘這邊房裏,亦東倒西歪,問月娘打發他那裏歇。月娘道:「他來與那個做生日就在那個兒房裏歇。」西門慶道:「我在那裏歇宿?」月娘道:「隨你那裏歇宿。再不,你也跟了他一處去歇罷。」西門慶笑道:「豈有此禮。」因叫小玉來脱衣:「我在這房裏睡了。」月娘道:「就別要汗邪,休要惹我那沒好口的駡的出來!你在這裏,他大妗子那裏歇?」西門慶道:「罷罷!我往孟三兒房裏歇去罷。」於是往玉樓房中歇了。

潘金蓮引着李瓶兒淨了手,同往他前邊來,晚夕和姥姥一處歇臥。到次日起來,臨鏡梳頭,春梅與他討洗臉水,打發他梳粧。因見春梅伶變,知是西門慶用過的丫鬟,與了他一付金三事兒。那春梅連忙就對金蓮説了。金蓮謝了又謝,説道:「又勞二娘賞賜他。」李瓶兒道:「不枉了五娘有福,好個姐姐!」早晨,金蓮領着他同潘姥姥,叫春梅開了花園門,各處遊看了一遍。李瓶兒看見他那邊牆頭開了個便門,通着他那壁,便問:「西門爹幾時起蓋這房子?」金蓮道:「前者央陰陽看來,也只到這二月間興土動工,收拾起要蓋。把二娘那房子打開通做一處,前面蓋山子捲棚,展一個大花園;後面還蓋三間翫花樓,與奴這三間樓相連做一條邊。」這李瓶兒聽記在心。兩人正説話,只見月娘使了小玉來請後邊吃茶。三人同來到上房,吳月娘李嬌兒孟玉樓,陪着吳大妗子,擺下茶等着哩。

衆人正吃點心茶湯,只見馮媽媽驀地走來,衆人讓他坐、吃茶。馮媽媽向袖中取出一方舊汗巾,包着四對金壽字簪兒,遞與李瓶兒。接過來先奉了一對與月娘,然後李嬌兒孟玉樓孫雪娥,每人都是一對。月娘道:「多有破費二娘,這個卻使不得!」李瓶兒笑道:「好大娘,甚麽罕稀之物,胡亂與娘們賞人便了。」月娘衆人拜謝了,方纔各人插在頭上。月娘道:「只説二娘家門首就是燈市,好不熱鬧。到明日俺們看燈去,就到往二娘府上望望,休要推不在家。」李瓶兒道:「奴到那日,奉請衆位娘。」金蓮道:「姐姐還不知,奴打聽來,這十五日是二娘生日。」月娘道:「今日説過,若到二娘貴降的日子,俺姊妹一個也不少,來與二娘祝壽去。」李瓶兒笑道:「蜗居小舍,娘們肯下降,奴一定奉請。」不一時吃罷早飯,擺上酒來飲酒。看看留連到日西時分,轎子來接,李瓶兒告辭歸家,衆姊妹款留不住。臨出門請西門慶拜見。月娘道:「他今日早起身出門,與縣丞送行去了。」婦人千恩萬謝,方纔上轎來家。正是:合歡核桃眞堪笑,裏許原來別有人。

畢竟後來何如,且聽下囬分解。

