\includepdf[pages={75,76},fitpaper=false]{tst.pdf}
\chapter*{第三十八囬 \\西門慶夾打二搗鬼 潘金蓮雪夜弄琵琶}
\addcontentsline{toc}{chapter}{第三十八囬 西門慶夾打二搗鬼 潘金蓮雪夜弄琵琶}
\markboth{{\titlename}卷之四}{第三十八囬 西門慶夾打二搗鬼 潘金蓮雪夜弄琵琶}


\begin{myquote}
麗質溫柔更老成,玉壺明月適人情。

輕囬玉臉花含媚,淺蹙蛾眉雲髻鬆。

勾引蜂狂桃蕊綻,潛牽蝶亂柳ん新。

令人心地常相憶,莫學章臺贈淡情。
\end{myquote}

話説馮婆子走到前廳角門首,看見玳安在廳槅子前,拿着茶盤兒伺候。玳安望着馮媽媽𢫓嘴兒:「你老人家先往那裏去。俺爹和應二爹説話哩!説了話,打發去了,就起身。先使棋童兒送酒去了。」那婆子聽見,兩步做一步走的去了。

原來應伯爵來説:「攬頭李智黄四,派了年例三萬香蠟等料錢糧下來,該一萬兩銀子,也有許多利息。上完了批,就在東平府現關銀子。來和你計較,做不做?」西門慶道:「我那裏做他!攬頭以假充眞,買官誆官,我衙門裏搭了事件還要動他。我做他怎的?」伯爵道:「哥若不做,敎他另搭別人。在你借二千兩銀子與他,每月五分行利。教他關了銀子還你,你心下如何?計較定了,我對他説,教他兩個明日拿文書來。」西門慶道:「旣是你的分上,我挪一千銀子與他罷。如今我莊上收拾,還沒銀子哩。」伯爵見西門慶吐了口兒,説道:「哥,若十分沒銀子,看怎麽再撥五百兩銀子貨物兒,湊個千五兒與他罷。他不敢少下你的。」西門慶道:「他少下我的,我有灋兒處。又一件,應二哥,銀子便與他,只不呌他打着我的旗兒在外邊東誆西騙!我打聽出來,只怕我衙門監裏放不下他。」伯爵道:「哥説的什麽話!典守者不得辭其責。他若在外邊打哥的旗兒,常沒事罷了;若壞了事,要我做什麽?哥,你只顧放心,但有差錯,我就來對哥説。説定了,我明日教他好寫文書。」西門慶道:「明日不敎他來,我有勾當。敎他後日來。」説畢,伯爵去了。

西門慶叫玳安伺候馬,帶上眼紗,問:「棋童去沒有?」玳安道:「回來了,取挽手兒去了。」不一時,取了挽手兒來,打發西門慶上馬,逕往牛皮巷來。

不想韓道國兄弟韓二搗鬼,耍錢輸了。喫的光睜睜兒的走來哥家,問王六兒討酒喫。袖子裏掏出一條小腸兒來,説道:「嫂,我哥還沒來哩。我和你喫壺燒酒。」那婦人恐怕西門慶來,又見老馮在廚下,不去兜攬他,説道:「我是不喫。你要喫,拿過一邊喫去,我那裏耐煩!你哥不在家,招是招非的又來做什麽!」那韓二搗鬼把眼兒涎瞪着,又不去,看見桌底下一罈白泥頭酒,貼着紅紙帖兒,問道:「嫂子是那裏酒?打開篩壺來俺們喫。耶嚛,你自受用?」婦人道:「你趂早兒休動,是宅裏老爹送來的,你哥還沒見哩!等他來家,有便倒一甌子與你喫。」韓二道:「等什麽哥!就是皇帝爺的,我也喫一鍾兒。」纔待搬泥頭,被婦人劈手一推,奪過酒來,提到屋裏去了,把二搗鬼仰八叉推了一跤。半日爬起來,惱羞變成怒,口裏喃喃呐呐罵道:「賊淫婦,我好意帶將菜兒來,見你獨自一個冷落落,和你喫盃酒。你不理我,倒推我一跤!我教你不要慌,你另敍上了有錢的漢子,不理我了,要把我打開,故意的遠我、囂我、訕我又趍我。休敎我撞見,我教你這不値錢的淫婦白刀子進去,紅刀子出來!」婦人見他的話不防頭,一點紅従耳畔起,須臾紫脹了雙腮。便取棒槌在手,趕着打出來,駡道:「賊餓不死的殺才!倒了你,那裏𠳹醉了,來老娘這裏撒野火兒!老娘手裏饒你不過!」那二搗鬼口裏喇喇哩哩駡淫婦,直罵出門去。

不想西門慶正騎馬來,見了他,問是誰。婦人道:「情知是誰:是韓二那廝,見他哥不在家,要便耍錢輸了,喫了酒來毆我。有他哥在家,常時撞見打一頓。」那二搗鬼一溜煙跑了。西門慶又道:「這少死的花子,等我明日到衙門裏與他做功德!」婦人道:「又教爹惹惱。」西門慶道:「你不知,休要慣了他。」婦人道:「爹説的是,自古良善被人欺,慈悲生患害。」一面讓西門慶明間内坐。西門慶吩咐棋童囬馬家去。呌玳安兒:「你在門首看,但掉着那光棍的影兒,就與我鎖在這裏,明日帶衙門裏來。」玳安道:「他的魂兒聽見爹到了,不知走的那裏去了!」

西門慶坐下,婦人見畢禮,連忙屋裏呌丫鬟錦兒,拿了一盞菓仁茶出來與西門慶喫,就叫他磕頭。西門慶道:「也罷,倒好個孩子。你且將就使着罷。」又道:「老馮在這裏?怎的不替你拿茶?」婦人道:「馮媽媽他老人家我央及他廚下使着手哩。」西門慶又道:「頭裏我使小廝送來的那酒,是個内臣送我的竹葉青酒哩。裏頭有許多薬味,甚是峻利。我前日見你這裏打的酒,通喫不上口,我所以拿的這罈酒來。」婦人又道個萬福説:「多謝爹的酒!正是這般説,俺們不爭氣,住在這僻巷子裏,又沒個好酒店,那裏得上樣的酒來喫!只往大街上取去。」西門慶道:「等韓夥計來家,你和他計較。等於獅子街那裏替你破幾兩銀子買下房子,等你兩口子一發搬到那裏住去罷。舖子裏又近,買東西諸事方便。」婦人道:「爹説的是,若你老人家恁的可憐見!離了這塊兒也好,就是你老人家行走,也免了許多小人口嘴。喒行的正,也不怕他。爹心裏要去自情去,他在家和不在家一個樣兒,也少不的打這條路兒來。」説一回,房裏放下桌兒,請西門慶房裏寬了衣服坐。須臾,安排酒菜上來,桌上無非是些鷄鴨魚肉嗄飯點心之類。婦人陪定,把酒來斟。不一時,兩個並肩疊股而飲,喫的酒濃時,兩個脱剝上牀交歡,自在頑耍。

婦人早已牀炕上鋪的厚厚的被褥,被裏薰的噴鼻香。西門慶見婦人好風月,一徑要打動他,家中袖了一個錦包兒來,打開:裏面銀托子、相思套、硫黄圈、薬煮的白綾帶子、懸玉環、封臍膏、勉鈴,一弄兒淫器。那婦人仰臥枕上,玉腿高蹺,鷄舌内吐,西門慶先把勉鈴教婦人自放牝内,然後將銀托子束其根,硫黄圈套其首,封臍膏貼於臍上。婦人以手導入牝中,兩相迎湊,漸入大半。婦人呼道:「達達,我只怕你蹲的腿酸,拿過枕頭來,你墊着坐,等我淫婦自家動罷!」又道:「只怕你不自在,你把淫婦腿弔着㒲,你看好不好?」西門慶眞個把他脚帶解下一條來,拴他一足,弔在牀槅子上。低着拽,拽的婦人牝中之津如蜗之吐涎,綿綿不絶,又拽出好些白漿子來。西門慶問道:「你如何流這些白漿?」纔待要抹之。婦人道:「你休抹,等我吮咂了罷!」於是蹲跪他面前,吮吞數次,鳴咂有聲。咂的西門慶淫心頓起,掉過身子,兩個幹後庭花。龜頭上有硫黃圈,濡硏艱澀,婦人蹙眉隱忍,半晌僅沒其稜。西門慶於是頗作抽送,已而婦人用手摸之,漸入大半。把屁股坐在西門慶懷裏,回首流眸,作顫聲叫:「達達,慢着些!往後越發粗大,教淫婦怎生挨忍?」西門慶且扶起其股,觀其出入之勢。因叫婦人小名:「王六兒,我的兒!你達不知心裏怎的,只好這一樁兒。不想今日遇你,正可我之意。我和你明日生死難開。」婦人道:「達達,只怕後來耍的絮煩了,把奴不理,怎了?」西門慶道:「相交下來,纔見我不是這樣人。」説話之間,兩個幹够一頓飯時。西門慶令婦人沒高低淫聲浪語叫着纔過,婦人在下,一面用手擧股承受其精,樂極情濃,一泄如注。已而拽出那話來,帶着圈子,婦人還替他吮咂淨了。兩個方纔並頭交股而臥。正是:一般滋味美,好耍後庭花。有詩為證:

\begin{myquote}
美寃家,一心愛折後庭花。尋常只在門前裏走,又被開路先鋒把住了他。放在户中難禁受,轉絲韁,勒囬馬;親得勝,弄的我身上麻。蹴損了奴的粉臉,粉臉那丹霞。
\end{myquote}

西門慶與婦人抱到二鼓時分,小廝馬來接,方纔起身回家。到次日早,衙門裏差了兩個緝捕,把二搗鬼拿到提刑院,只當做掏摸土賊,不由分説,一夾二十板,打得順腿流血,睡了一個月,險不把命花了,往後嚇了連影再不敢上婦人門纏攪了。正是:恨小非君子,無毒不丈夫!

遲了幾日,來保韓道國一行人東京囬來,備將前事,對西門慶説:「翟管家見了女子,甚是歡喜,説費心。留俺在府裏住了兩日。討了囬書,送了爹一匹青馬,封了韓夥計女兒五十兩銀子禮錢,又與了小的二十兩盤纏。」西門慶道:「夠了。」看了囬書,書中無非是知感不盡之意。自此兩家都下「眷生」名字,稱呼親家,不在話下。韓道國與西門慶磕頭,拜謝回家。西門慶道:「韓夥計,你還把你女兒這禮錢收去,也是你兩口兒恩養孩兒一場。」韓道國再三不肯收,説道:「蒙老爹厚恩,禮錢已是前日有了。這銀子小人怎好又受得?従前累的老爹好少哩!」西門慶道:「你不依,我就惱了。你將囬家,不要花了,我有個處。」那韓道國就磕頭謝了,拜辭囬去。

老婆見他漢子來家,滿心歡喜。一面接了行李,與他拂了塵土,問他長短,「孩子到那裏好麽?」這道國把往回一路的話告訴一遍,説:「好人家。孩子到那裏,就與了三間房,兩個丫鬟伏侍。衣服頭面是不消説,第二日就領了後邊,見了太太。翟管家甚是歡喜,留俺們住了兩日,酒飯連下人都喫不了。又與了五十兩禮錢。我再三推辭,大官人又不肯,還教我拿囬來了。」因把銀子與婦人收了,婦人一塊石頭方落地。因和韓道國説:「喒到明日,還得一兩銀子謝老馮。你不在,虧他常來做伴兒。大官人那裏,也與了他一兩。」正説着,只見丫頭過來遞茶。韓道國道:「這個是那裏大姐?」婦人道:「這個是喒新買的丫頭,名喚錦兒。過來與你爹磕頭。」磕了頭,丫頭往廚下去了。老婆如此這般,把西門慶勾搭之事,告訴一遍:「自従你去了,來行走了三四遭,纔使四兩銀子,買了這個丫頭。但來一遭,帶一二兩銀子來。第二的不知高低,氣不憤,走來這裏放水,被他撞見了,拿到衙門裏打了個臭死,至今再不敢來了。大官人見不方便,許了要替喒們大街上買一所房子,教喒搬到那裏住去。」韓道國道:「嗔道他頭裏不受這銀子,教我拿囬來,休要花了,原來就是這些話了。」婦人道:「這不是有了五十兩銀子?他到明日,一定與喒多添幾兩銀子,看所好房兒。也是我輸了身一場,且落他些好供給穿戴!」韓道國道:「等我明日往舖子裏去了,他若來時,你只推我不知道。休要怠慢了他,凡事奉承他些兒!如今好容易賺錢,怎麽趕的這個道路!」老婆笑道:「賊強人,倒路死的!你倒會喫自在飯兒,你還不知老娘怎生受苦哩!」兩個又笑了一囬,打發他喫了晚飯,夫婦收拾歇下。到天明,韓道國宅裏討了鑰匙,開舖子去了。與了老馮一兩銀子謝他,俱不必細説。

一日,西門慶同夏提刑衙門囬來。夏提刑見西門慶騎着一匹高頭點子青馬,問道:「長官,那匹白馬怎的不騎,又換了這匹馬?倒好一匹馬,不知口裏如何?」西門慶道:「那馬在家歇他兩日兒。這馬是昨日東京翟雲峰親家送來的,是西夏劉參將送他的,口裏纔四個牙兒。脚程緊慢都由他的,只是有些毛病兒,快護槽踢蹬。初時着了路上走,把膘息跌了許多,這兩日纔喫的好些兒了。」夏提刑道:「這馬甚是會行,只好長官騎着每日躧街道兒罷了,不可走遠了他。論起在喒這裏,也値七八十兩銀子。我學生騎的那馬,昨日又瘸了,今早來衙門裏來,旋拿帖兒問舍親借了這匹馬騎來了,甚是不方便。」西門慶道:「不打緊,長官沒馬,我家中還有一疋黄馬,送與長官罷。」夏提刑擧手道;「長官下顧,學生奉價過來。」西門慶道:「不湏計較,學生到家就差人送來。」兩個走到西街口上,西門慶擧手,分路來家;到家就使玳安把馬送去。夏提刑見了大喜,賞了玳安一兩銀子,與了囬帖兒,説:「多上覆,明日到衙門裏面謝。」

過了兩月,乃是十月中旬時分。夏提刑家中做了些菊花酒,叫了兩名小優兒,請西門慶一敍,以酬送馬之情。西門慶家中喫了午飯,理了些事務,往夏提刑家飲酒。原來夏提刑備辦一席齊整酒餚,只為西門慶一人而設。見了他來,不勝歡喜,降階迎接,至廳上叙禮。西門慶道:「如何長官這等費心!」夏提刑道:「今年寒家做了些菊花酒,閒中屈執事一叙,再不敢請他客。」於是見畢禮數,寬去衣服,分賓主而坐。茶罷着棋,就席飲酒敍談。兩個小優兒在旁彈唱。正是:得多少金樽進酒浮香蟻,象板催箏唱鷓鴣。

不説西門慶在夏提刑家飲酒。單表潘金蓮,見西門慶許多時不進他房裏來,每日翡翠衾寒,芙蓉帳冷。那一日把角門兒開着,在房内銀燈高點,靠定幃屏,彈弄琵琶。等到二三更,便使春梅瞧數次,不見動靜。正是:銀箏夜久殷勤弄,寂寞空房不忍彈。在牀上和衣兒又睡不着,不免取過琵琶,横在膝上,低低彈了個〔二犯江兒水〕,以遣其悶:

\begin{myquote}
「悶把幃屏來靠,和衣強睡倒。」
\end{myquote}

猛聽的房簷上鐵馬兒一片聲響,只道西門慶來到,敲的門環兒響,連忙使春梅去瞧。他囬道:「娘錯了,是外邊風起落雪了!」婦人於是彈唱道:

\begin{myquote}
「聽風聲嘹亮,雪洒窗寮,任氷花片片飄。」
\end{myquote}

一囬兒,燈昏香盡,心裏欲待去剔續,見西門慶不來,又意兒懶的動彈了。唱道:

\begin{myquote}
「懶把寳燈挑,慵將香篆燒。{\marktext\small\color{mydarkgray}(只是捱一日似三秋,盼一夜如半夏。)}捱過今宵,怕到明朝。細尋思,這煩惱何日是了?{\marktext\small\color{mydarkgray}(暗想負心賊當初説的話兒,心中由不的我傷情兒。)}{\marktext\small(合)}想起來,今夜裏心兒内焦,悞了我青春年少。{\marktext\small\color{mydarkgray}(誰想你弄的我三不歸,四不着地。)}你撇的人有上梢來沒下梢!」
\end{myquote}

且説西門慶約一更時分,従夏提刑家喫了酒歸來,一路天氣陰晦,空中半雨半雪下來,落在衣服上都化了,不免打馬來家。小廝打着燈籠,就不到後邊,逕往李瓶兒房來。李瓶兒迎着,一面替他拂去身上雪霰。西門慶穿着青絨獅子補子、坐馬白綾襖子、忠靖緞巾、皂靴棕套、貂鼠風領。李瓶兒替他接了衣服,止穿綾敞衣,坐在牀上,就問:「哥兒睡了不曾?」李瓶兒道:「小官兒頑了這回,方睡下了。」西門慶吩咐:「呌孩兒睡罷,休要沉動着,只怕唬醒他。」迎春於是拿茶來喫了。李瓶兒問:「今日喫酒來的早。」西門慶道:「夏龍溪還是前日因我送了他那匹馬,今日全為我費心,治了一席酒請我;又呌了兩個小優兒。和他坐了這一回,見天氣下雪,來家早些。」李瓶兒道:「你喫酒?教丫頭篩酒來你喫。大雪裏來家,只怕冷哩。」西門慶道:「還有那葡萄酒,你篩來我喫。今日他家喫的是自造的菊花酒,我嫌他け香け氣的,我沒大好生喫。」於是迎春放下桌兒,就是幾碟醃鷄兒嗄飯,細巧菓菜之類。李瓶兒拿杌兒在旁邊坐下,桌下放着一架小火盆兒。

這裏兩個喫酒,潘金蓮在那邊屋裏冷清清,獨自一個兒坐在牀上,懷抱着琵琶,桌上燈昏燭暗。待要睡了,又恐怕西門慶一時來;待要不睡,又是那盹困,又是寒冷。不免除去冠兒,亂挽烏雲,把帳兒放下半邊來,擁衾而坐。正是:

\begin{myquote}
倦倚綉牀愁懶睡,低垂錦帳綉衾空;

早知薄倖輕抛棄,辜負奴家一片心。
\end{myquote}

又唱道:

\begin{myquote}
「懊恨薄情輕棄,離愁閒自惱。」
\end{myquote}

又喚春梅過來:「你去外邊再瞧瞧,你爹來了沒有?快來囬我話。」那春梅走去,良久囬來説道:「娘還認爹沒來呢!爹來家不耐煩了,在六娘屋裏喫酒的不是?」這婦人不聽罷了,聽了如同心上戳上幾把刀子一般,駡了幾句負心賊,由不得撲簌簌眼中流下淚來。一逕把那琵琶兒放得高高的,口中又唱道:

\begin{myquote}
「論殺人好恕,情理難饒,負心的天鑒表!{\marktext\small\color{mydarkgray}(好教我提起來,又是那疼他,又是那恨他。)}心癢痛難揉,愁懷悶自焦。{\marktext\small\color{mydarkgray}(呌了聲,賊狠心的寃家,我比他何如?鹽也是這般鹽,醋也是這般醋,磚兒能厚,瓦兒能薄,你一旦棄舊憐新!)}讓了甜桃,去尋酸棗。{\marktext\small\color{mydarkgray}(不合今日教你哄了!)}奴將你這定盤星兒錯認了。{\marktext\small(合)}想起來,心兒裏焦。誤了我青春年少,你撇的人有上梢來沒下梢! 

為人莫作婦人身,百般苦楽由他人。

癡心老婆負心漢,悔莫當初錯認眞!

常記的當初相聚,癡心兒望到老。{\marktext\small\color{mydarkgray}(誰想今日他把心變了,把奴來一旦輕抛不理,正如那日。)}被雲遮楚岫,水淹藍橋。打拆開鸞鳳交。{\marktext\small\color{mydarkgray}(到如今當面對語,心隔千山;隔着一堵牆,咫尺不淂相見。)}心遠路非遙,{\marktext\small\color{mydarkgray}(意散了,如鹽落水,如水落沙相似了。)}情疏魚雁杳。{\marktext\small\color{mydarkgray}(空教我有情難控訴。)}地厚天高。{\marktext\small\color{mydarkgray}(空教我無夢到陽臺。)}夢断魂勞。俏冤家這其間心變了!{\marktext\small(合)}想起來,心兒裏焦。誤了我青春年少。你撇的人有上梢來沒下梢!」
\end{myquote}

西門慶正在房中和李瓶兒喫酒,忽聽見這邊房裏,彈的琵琶之聲,便問:「是誰彈琵琶?」迎春答道:「是五娘在那邊彈琵琶響。」李瓶兒道:「原來你五娘還沒睡哩!綉春,你快去請你五娘來喫酒,你説俺娘請哩。」那綉春去了。李瓶兒忙教迎春那邊安下個坐兒,放個鍾筯在面前。良久,綉春走來説:「五娘摘了頭,不來哩。」李瓶兒道:「迎春,你再去請你五娘去。你説娘和爹請五娘哩。」不多時,迎春來説:「五娘把角門兒關了。説吹了燈,睡下了。」西門慶道:「休要信他小淫婦兒。等我和你兩個拉他去,務要把他拉了來,喒和他下盤棋耍子。」於是和李瓶兒同來打他角門。打了半日,春梅把角門子開了。西門慶拉着李瓶兒進入他房中,只見婦人坐在帳中,琵琶放在傍邊。西門慶道:「怪小淫婦兒,怎的兩三轉請着你不去?」金蓮坐在牀上紋絲兒不動,把臉兒沉着,半日説道:「那沒時運的人兒,丢在這冷屋裏隨我自生兒由活的,又來瞅睬我怎的?沒的空費了你這個心,留着別處使!」西門慶道:「怪奴才,八十歲媽媽沒牙——有那些唇舌的!李大姐那邊請你和他下盤棋兒,只顧等你不去了。」李瓶兒道:「姐姐,可不是的?我那屋裏擺下棋子了,喒們閑着下一盤兒,賭盃酒喫。」金蓮道:「李大姐,你們自去,我摘了頭。你不知我心裏不耐煩,我如今睡也,比不的你們心寬閑散。我這兩日,只有口遊氣兒。黄湯淡水誰嚐着來?我成日睜着臉兒過日子哩!」西門慶道:「怪奴才!你好好兒的,怎的不好?你若心内不自在,早對我説,我好請太醫來看你。」金蓮道:「你不信,教春梅拿過我的鏡子來,等我瞧。這兩日,瘦的像個人模樣哩!」春梅把鏡子眞個遞在婦人手裏,燈下觀看。正是:

\begin{myquote}
羞對菱花試新粧,為郎憔悴減容光;閉門不管閒風月,任您梅花自主張。

「羞把菱花來照,蛾眉懶去掃。暗消磨了精神,折損了丰標,瘦伶仃不甚好。」
\end{myquote}

西門慶拿過鏡子,也照了照,説道:「我怎麽不瘦?」金蓮道:「拿什麽比的你?每日碗酒塊肉,喫的肥胖胖的,專一只奈何人!」被西門慶不由分説,一屁股挨着他坐在牀上,摟過脖子來就親了個嘴。舒手被裏,摸見他還沒脫衣裳。兩隻手齊插在他腰裏去,説道:「我的兒,眞個瘦了些!」金蓮道:「怪行貨子,好冷手,冰的人慌!莫不我哄了你不成?」正是:

\begin{myquote}
「香褪了海棠嬌,衣愡了楊柳ん。{\marktext\small\color{mydarkgray}(説着,就沿香腮抛下珠淚來。我的苦惱,誰人知道?眼淚打肚裏流罷了。)}悶悶無聊,攘攘勞勞,淚珠兒到今滴盡了。{\marktext\small(合)}想起來,心裏亂焦。誤了我青春年少。你撇的人來有上梢來沒下梢!」
\end{myquote}

亂了一囬,西門慶還把他強死強活拉到李瓶兒房内,下了一盤棋,喫了一囬酒。臨起身,李瓶兒見他這等臉酸,把西門慶攛掇過他這邊歇了。正是:得多少腰瘦故知閒事惱,淚痕只為別情濃。有詩為證:

\begin{myquote}
自従別後減容光,萬轉千回懶下牀;

虧殺瓶兒成好事,得教巫女會襄王。
\end{myquote}

畢竟未知後來如何,且聽下囬分解。

