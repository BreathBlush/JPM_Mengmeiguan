\includepdf[pages={147,148},fitpaper=false]{tst.pdf}
\chapter*{第七十四囬 \\宋御史索求八僊鼎 吴月娘聽宣黄氏卷}
\addcontentsline{toc}{chapter}{第七十四囬 宋御史索求八僊鼎 吴月娘聽宣黄氏卷}
\markboth{第七十四囬 宋御史索求八僊鼎 吴月娘聽宣黄氏卷}{第七十四囬 宋御史索求八僊鼎 吴月娘聽宣黄氏卷}
\thispagestyle{empty}

\begin{myquote}
昔年南去得娱賓,頓遜盃前共好春。

螘泛羽觴蠻酒膩,鳳啣瑤句蜀箋新;

花憐遊騎紅隨轡,草戀征車碧繞輪。

别後清清鄭南陌,不知風月屬何人。
\end{myquote}

話説西門慶摟抱潘金蓮,一覺睡到次日天明。婦人見他那話還直竪一條棍相似,便道:「達達,你將就饒了我罷,我來不得了,待我替你咂咂罷!」西門慶道:「怪小淫婦兒,你不若咂咂,咂的過了,是你造化!」這婦人眞個蹲向他腰間,按着他一隻腿,用口替他吮弄那話。約吮夠一個時分,精還不過,這西門慶用手按着粉項,往來只顧没稜露腦搖撼,那話在口裏吞吐不絕,抽拽的婦人口邊白沫横流,殘脂在莖。精欲洩之際,婦人一面問西門慶:「二十八日,應二爹送了請帖來請,俺們去不去?」西門慶道:「怎的不去?都收拾了去。」婦人道:「我有樁事兒央你,依不依?」西門慶道:「怪小淫婦兒,你有甚事說不是?」婦人道:「把李大姐那皮襖㧱出來與我穿了罷,明日喫了酒囘來,他們都穿着皮襖,只奴沒件兒穿。」西門慶道:「有年時王招宣府中當的皮襖,你穿就是了。」婦人道:「當的我不穿他。你與了李嬌兒去;把李嬌兒那皮襖卻與雪娥穿,我穿李大姐這皮襖。你今日㧱出來與了我,我㩟上兩個大紅遍地金鶴袖,襯着白綾襖兒穿。也是我與你做老婆一場,沒曾與了別人。」西門慶道:「賊小淫婦兒,單管愛小便益兒。他那件皮襖,值六十兩銀子哩!油般大黑蜂毛兒,你穿在身上是會搖擺。」婦人道:「怪奴才,你是與了張三李四的老婆穿了?左右是你的老婆,替你裝門面的,沒的有這些聲兒氣兒的!好不好,我就不依了。」西門慶道:「你又求人,又做硬兒!」婦人道:「怪硶貨!我是你房裏丫頭,在你跟前服軟?」一面説着,把那話放在粉臉上,只顧偎㨪,良久又吞在口裏,挑弄蛙口;一囬又用舌尖舐其琴絃,攪其龜稜;然後將朱唇裹着,只顧動動的。西門慶靈犀灌頂,滿腔春意透腦,良久精來,連聲呼:「小淫婦兒,好生裹緊着,我待過也……」言未絶,其精邈了婦人一口,婦人一面一口口接着都咽了。正是:自有内事迎郎意,殷勤愛把紫簫吹。

當日卻是安郎中擺酒,西門慶起來梳頭淨面,出門,婦人還睡在被裏,便說道:「你趂閒尋尋兒出來罷。等一囬你又不得閒了。」這西門慶於是走到李瓶兒房中,奶子丫頭又早起來收拾乾凈,安頓下茶水伺候。見西門慶進來坐下,問「供養娘了?」如意兒道:「咱供養多時了。」西門慶見如意兒穿着玉色對衿襖兒,白布裙子,葱白緞子紗綠高底鞋兒,薄施朱粉,長畫蛾眉,油胭脂搽的嘴唇鮮紅的,耳邊帶着兩個金丁香兒,手上帶着李瓶兒與他四個烏金戒指兒,笑嘻嘻遞了茶,在旁邊說話兒。西門慶一面使迎春往後邊討牀房裏鑰匙去。那如意兒便問:「爹討來做什麽?」西門慶道:「我要尋皮襖與你五娘穿。」如意道:「是娘的那貂鼠皮襖?」西門慶道:「就是。他要穿穿,㧱與他罷。」迎春去了,把老婆就摟在懷裏,兩手就舒在胸前,摸他奶頭,説道:「我兒,你雖然生養了孩子,奶頭兒到還恁緊。」就兩個臉對臉兒親嘴,且咂舌頭做一䖏。如意兒道:「我見爹常在五娘身邊,没見爹往别的房裏去,他老人家别的罷了,只是心窄容不的人。前日爹不在,為了棒槌,好不和我大嚷了一場,多虧韓嫂兒和三娘來勸開了。落後爹來家,也沒敢和爹說。不知什麽多嘴的人對他說,又説爹要了我。他也告爹來不曾?」西門慶道:「他也告我來。你到明日替他陪個禮兒便了,他是恁行貨子,受不的人個甜棗兒就喜歡的!嘴頭子雖利害,倒也没什麽心。」如意兒道:「前日我和他嚷了,第二日爹到家,就和我說好話。説爹在他身邊偏的多,『就是別的娘都讓我幾分。你凡事只有個不瞞我,我放着河水不洗船,好做惡人?』」西門慶道:「旣是如此,大家取和些。」又許下老婆:「你們晚夕等我來這房裏睡。」如意道:「爹真個來?休哄俺們着!」西門慶道:「誰哄你來?」正說着,只見迎春取鑰匙來了。西門慶教開了牀房門,又開橱櫃,㧱出那皮襖來,抖了抖,還用包袱包了,教迎春㧱到那邊房裏去。如意兒悄悄向西門慶說:「我没件好裙襖兒,你趂着手兒,再尋出來與了我罷。有娘小衣裳兒,再與我一件兒。」西門慶連忙就敎他開箱子,尋出一套翠藍緞子襖兒,黄綿紬裙子;又是一件藍潞紬綿褲兒;又是一雙粧花膝褲腿兒,與了他。老婆磕頭謝了。西門慶鎖上門去了,就使他送皮襖與金蓮房裏來。

金蓮纔起來,在牀上裹脚,只見春梅説:「如意兒送皮襖來了。」婦人便知其意,説道:「你敎他進來。」問道:「爹使你來?」如意道:「是爹敎我送來與娘穿。」金蓮道:「也與了你些什麽兒没有?」如意道:「爹賞了我兩件紬絹衣裳年下穿,教我來與娘磕頭。」於是向前磕了四個頭。婦人道:「姐姐們,這般却不好?你主子旣愛你,常言船多不碍港,車多不碍路,那個好做惡人?你只不犯着我,我管你怎的,我這裏還多着個影兒哩!」如意兒道:「俺娘已是没了,雖是後邊大娘承攬,娘在前邊還是主兒,早晚望娘擡擧。小媳婦敢欺心,那裏是葉落歸根之䖏?」婦人道:「你這衣服,少不得還對你大娘說聲是的。」如意道:「小的前者也問大娘討來,大娘説,等爹開箱櫃時㧱兩件與你。」婦人道:「旣説知,罷了。」這如意就出來,還到那邊房裏。西門慶已往前廳去了,如意便問迎春:「你頭裏取鑰匙去,大娘怎的説?」迎春說:「大娘問,你爹要鑰匙做什麽?我也没說㧱皮襖與五娘,只説我不知道。大娘没言語。」

卻說西門慶走到廳上看着設席擺列,海鹽子弟張羙、徐順、苟子孝、生旦都挑戯箱到了。李銘等四名小優兒,又早來伺候,都磕頭見了。西門慶吩咐打發飯與衆人喫。吩咐李銘三個在前邊唱,左順後邊答應堂客。那日,韓道國娘子王六兒沒來,打發申二姐買了兩盒禮物,坐轎子,他家進財兒跟着,也來與玉樓做生日。王經送到後邊,打發轎子出去了。那日門外韓大姨、孟大妗子都到了,又是傅夥計、甘夥計娘子、崔本媳婦兒段大姐並賁四娘子。西門慶正在廳上,看見夾道内玳安領着那個五短身子,穿綠緞襖兒、紅裙子,勒着藍金綃箍兒,不搽胭粉,兩個密縫眼兒,一似鄭愛香模樣,便問:「是誰?」玳安道:「是賁四嫂。」西門慶就没言語。往後見了月娘,月娘擺茶。西門慶進來喫粥,遞與月娘鑰匙。月娘道:「你開門做什麽?」西門慶道:「六兒他說明日往應二哥家喫酒没皮襖,要李大姐那皮襖穿。」被月娘瞅了一眼,説道:「你自家把不住自家嘴頭子。他死了,嗔人分散他房裏丫頭;像你這等,就没的話兒說了。他現放皮襖不穿,巴巴兒只要這皮襖穿!早是他死了,你指望這皮襖;他不死,你只好看一眼兒罷了!」幾句說得西門慶閉口無言。忽報劉學官來還銀子,西門慶出去,陪坐在廳上説話。只見玳安㧱進帖兒説:「王招宣府送禮來了。」西門慶問:「是什麽禮?」玳安道:「是賀禮。一疋尺頭,一罈南酒,四樣下飯。」西門慶看帖兒,上寫着:「眷晚生王寀頓首拜。」西門慶即便呌王經㧱眷生囬帖兒謝了。賞了來人五錢銀子,打發出了門。

只見李桂姐門首下轎,保兒挑四方盒禮物,慌的玳安替他抱毡包,説道:「桂姨打夾道内進去罷,廳上有劉學官坐着哩。」那桂姐即向夾道内進裏邊去。來安兒把盒子挑進月娘房裏去。月娘道:「爹看見來不曾?」玳安道:「爹陪着客,還不見哩。」月娘便説道:「連盒放在明間内。」一囘,客去了,西門慶進來喫飯。月娘道:「李桂姐送禮在這裏。」西門慶道:「我不知道。」月娘令小玉揭開盒兒,見一盒果餡壽糕,一盒玫瑰八僊糕,兩隻燒鴨,一副豕蹄。只見桂姐従房内出來,滿頭珠翠,勒着白挑線汗巾,大紅對衿襖兒,藍緞裙子,望着西門慶磕了四個頭。西門慶道:「罷了,又買這禮來做什麽?」月娘道:「剛纔桂姐對我説,怕你惱他。不干他事。説起來都是他媽的不是。那日桂姐害頭疼來,只見這王三官領着一行人,往秦玉芝兒家請秦玉芝兒。打門首過,進來喫茶,就被人進來驚散了。桂姐也没出來見他。」西門慶道:「那一遭是没出來見他,這一遭又是没出來見他,自家也説不過。論起來我也難管。你這麗春院㧱燒餅砌着門不成?到處銀錢兒都是一樣,我也不惱!」那桂姐跪在地下,只顧不起來,説道:「爹惱的是。我若和他沾沾身子,就爛化了,一個毛孔兒裏生個天疱瘡!都是俺媽空老了一片皮幹的營生,没個主意,好的也招惹,歹的也招惹來家,平白敎爹惹惱!」月娘道:「你旣來了,說開就是了,又惱怎的?」西門慶道:「你起來,我不惱你便了。」那桂姐故作喬張致,説道:「爹笑一笑兒,我纔起來;你不笑,我就跪一年也不起來。」不防潘金蓮在傍插口道:「桂姐,你起來。只顧跪着他,求告他黄米頭兒,教他張致!如今在這裏你便跪着他,明日到你家他卻跪着你;你那時別要理他!」把西門慶月娘都笑了,桂姐纔起了來。

只見玳安慌慌張張來報:「宋老爹和安老爹來了。」這西門慶便敎拿衣服,穿了出去迎接去了。桂姐向月娘説道:「耶嚛嚛!従今後我也不要爹了,只與娘做女兒罷。」月娘道:「你虚頭願心,説過道過罷了。前日兩遭往裏頭去,没在你那裏?」桂姐道:「天麽天麽!可是殺人!爹沒往我家裏,若是到我家,見爹一面,沾沾身子兒,就促死了我,渾身生天疱瘡!娘,你錯打聽了,敢不是我那裏,都往鄭月兒家走了兩遭,請了他家小粉頭子了。我這篇是非,就是他氣不憤架的;不然爹如何惱我?」金蓮道:「各人衣飯,他平白怎麽架你是非?」桂姐道:「五娘,你不知,俺們這裏邊人,一個氣不憤一個,好不生分!」月娘接過來道:「你們裏邊與外邊怎的打偏别?也是一般,一個不憤一個。那一個有些時道兒,就要躧下去。」月娘擺茶與他喫,不在話下。

卻説西門慶迎接宋御史安郎中到廳上敍禮,每人一疋緞子,一部書奉賀西門慶。見了桌席齊整,甚是稱謝不盡。一面分賓主坐下,呌上戲子來參見。吩咐:「等蔡老爹到,用心扮演。」不一時喫了茶,宋御史道:「學生有一事奉凟四泉,今有巡撫侯石泉老先生,新陞太常卿,學生同兩司作東,二十九日借尊府,置盃酒奉餞,初二日就起行上京去了,未審四泉允諾否?」西門慶道:「老先生吩咐,敢不従命。但未知多少桌席?」宋御史道:「學生有分資在此。」即喚書吏上來,毡包内取出布按兩司連他共十二封分資來,每人一兩,共十二兩銀子。要一張大插桌,餘者六桌都是散桌,叫一起戲子。西門慶答應收了,宋御史又下席作揖致謝。少頃,請去捲棚聚景堂那裏坐的。不一時,鈔關錢主事也到了。三員官會在一處,換了茶,擺棋子下棋。安御史見西門慶堂廡寬廣,院中幽深,書畫文物,極一時之盛。又見挂着一幅□陽捧日横批古畫,正面螺鈿屏風,屏風前安着一座八僊捧壽的流金鼎,約數尺高,甚是做得奇巧。見爐内焚着沉檀香,煙從龜鶴鹿口中吐出,只顧近前觀看,誇獎不已。問西門慶:「這付爐鼎造得好!」因向二官說:「我學生寫書與淮安劉年兄那裏,替我捎帶這樣一付來送蔡老先,還不見到。四泉不知是那裏得來的?」西門慶道:「也是淮上一個人送學生的。」說畢,下棋。西門慶吩咐下邊,看了兩個桌盒,細巧菜蔬,菓餡點心上來,一面呌生旦在上唱南曲。宋御史道:「客尚未到,主人先喫得面紅,說不通。」安郎中道:「天寒飲一盃無碍。」原來宋御史已差公人船上邀蔡知府去了。近午時分,來人囬報:「邀請了,在磚廠黄老爹那裏下棋,便來也。」宋御史令起去伺候。一面下棋飲酒。安郎中喚戯子:「你們唱個〈宜春令〉奉酒。」於是貼旦唱道:

\begin{myquote}
「第一來為壓驚,第二來因謝誠。殺羊茶飯,來時早已安排定。断閑人,不會親鄰,請先生和俺鶯娘匹娉。我只見他歡天喜地,道謹依來命。」

\setlength{\hangindent}{4em}\hangafter=0{\markfont〈玉枝花〉}「來囬顧影,文魔秀士欠酸丁。下工夫將頭顱來掙,遲和疾擦倒蒼蝇。光油油耀花人眼睛,酸溜溜螫得牙根冷。天生這個後生,天生這個俊英!」

\setlength{\hangindent}{4em}\hangafter=0{\markfont〈玉嬌鶯〉}「今宵歡慶,我鶯娘何曾慣經,你須索要欵欵輕輕。燈兒下共交鴛頸,端詳可憎,誰無志誠。您兩人今夜親折證。謝芳卿,感紅娘錯愛,成就了這姻親。」

\setlength{\hangindent}{4em}\hangafter=0{\markfont〈解三醒〉}「玳筵開、香焚寳鼎,綉簾外、風掃閑庭。落紅滿地胭脂冷,碧玉欄杆花弄影。準備鴛鴦夜月銷金帳,孔雀春風軟玉屏。合歡令,更有那鳳簫象板,錦瑟鸞笙。」

\setlength{\hangindent}{4em}\hangafter=0{\markfont〈前腔〉(生唱)}「可憐我書劍飄零無厚聘,感不盡姻親事有成。新婚燕爾安排定,除非是折桂手報答前程。我如今博得個跨鳳乘鸞客,到晚來臥看牽牛織女星。非僥倖,受用的珠圍翠繞,結果了黄卷青燈。」

\setlength{\hangindent}{4em}\hangafter=0{\markfont〈尾聲〉}「老夫人專意等。{\marktext(生唱)}常言道恭敬不如従命。{\marktext(紅唱)}休使紅娘再來請。」
\end{myquote}

唱畢,忽吏典進報:「蔡老爹和黄老爹來了。」宋御史忙令收了桌席,各整衣冠,出來迎接。蔡九知府穿素服金带,跟着許多官吏。先令人投一「侍生蔡脩拜」帖與西門慶,進廳上。安郎中道:「此是主人西門大人,現在本䖏作千兵,也是京中老先生門下。」那蔡知府又作揖,稱道:「久仰,久仰!」西門慶亦道:「容當奉拜。」敍禮畢,各寬衣服坐下,左右上了茶,各人扳話。良久,就上坐,西門慶令小優兒在傍彈唱。蔡九知府居上,主位四坐。厨役割道湯飯,戯子呈遞手本,蔡九知府揀了〈雙忠記〉,演了兩摺,酒過數巡,宋御史令生旦上來遞酒。小優兒席前唱一套〈新水令〉「玉驄驕馬出皇都。」蔡知府笑道:「松原直得多。可謂御史青驄馬,三公乃劉郎舊家物耳。」安郎中道:「今日更不遣江州司馬青衫濕。」言罷,衆人都笑了。西門慶又令春鴻唱了一套「金門獻罷平胡表」,把宋御史喜歡的了不的。因向西門慶道:「此子可愛!」西門慶道:「此是小价,原是揚州人。」宋御史攜着他手兒,敎他遞酒,賞了他三錢銀子,磕頭謝了。正是:

\begin{myquote}
窗外日光彈指過,席前花影坐間移。

一盃未盡笙歌送,堦下申牌又報時。
\end{myquote}

不覺日色沉西,蔡九知府見天色晚了,即令左右穿衣,告辭。衆位欵留不住,俱送出大門而去,隨即差了兩名吏典,把桌席羊酒尺頭,擡送到新河口下處去訖,不題。宋御史於是亦作辭西門慶,因説道:「今日且不謝,後日還要取擾。」各上轎而去。

西門慶送了囬來,打發了戲子,吩咐:「後日仍是你們來,再唱一日,呌幾個會唱的來,宋老爹請巡撫侯爺哩。」戲子道:「小的知道了。」西門慶令攢上酒桌,使玳安:「去請溫相公來坐坐。」再教來安兒:「去請應二爹去。」不一時,次第而至,各行禮坐下。三個小優兒在傍彈唱,把酒來斟。西門慶問伯爵:「你娘們明日都去,你呌唱的?是雜耍的?」伯爵道:「哥到説得好,小人家那裏着放?將就叫了兩個唱女兒唱罷了。明日早些請衆嫂子下降。」這裏前廳喫酒,不題。

且說鄭金左順在後邊堂客席前唱了一日。孟大姨與孟二妗子先起身去了。落後楊姑娘也要去,月娘道:「姑奶奶,你再住一日兒家去不是?薛姑子使他徒弟取了卷來,咱晚夕敎他宣卷咱們聽。」楊姑娘道:「老身實和姐姐說,要不是我也住,明日俺門外第二個侄兒定親事,使孩子來請我,我要瞧瞧去。」於是作辭而去。只有傅夥計甘夥計娘子,與賁四娘子、段大姐、月娘還留在上房陪大妗子、潘姥姥、李桂姐、申二姐、郁大姐在傍,一遞一套彈唱,兩個小優兒都打發在前邊來了。又喫至掌燈已後,三位夥計娘子都作辭去了。止段大姐沒去,在後邊雪娥房中歇了。潘姥姥往金蓮房内去了。只有大妗子、李桂姐、申二姐和三個姑子、郁大姐和李嬌兒、孟玉樓、潘金蓮,在月娘房内坐的。忽聽前邊西門慶散了,小廝收進家活來。這金蓮慌忙抽身就往前走了,到前邊,黑影兒裏悄悄立在角門首。只見西門慶扶着來安兒,打着燈,趔趄着脚兒,就往李瓶兒那邊走,看見金蓮在門首立着,拉着手進入房來。那來安兒便往上房交鍾筯。

月娘只說西門慶進來,把申二姐、李大姐、郁大姐都打發往李嬌兒房内去了。問來安道:「你爹來沒有?在前邊做什麽?」來安道:「爹在五娘房裏去的不耐煩了!」月娘聽了,心内就有些惱,因向玉樓道:「你看恁没來頭的行貨子!我說他今日進來往你房裏去,如何三不知又摸到他那屋裏去了?這兩日又浪風發起來,只在他前邊纏!」玉樓道:「姐姐,隨他纏去!恰似咱們把這件事放在頭裏,爭他的一般。可是大師父說笑話兒的來頭,左右這六房裏由他串到。他爹心中所欲,你我管的他?」月娘道:「乾淨他有了話。剛纔聽見前頭散了就慌的奔命的往前走了。」因問小玉:「竃上沒人了,與我把儀門拴上了罷。後邊請三位師父來,咱們且聽他宣一囬卷着。」又把李桂姐、申二姐、段大姐、郁大姐,都請了來。月娘向大妗子道:「我頭裏旋呌他使小沙彌請了〈黄氏女卷〉來宣,今日可可兒楊姑娘又去了。」吩咐玉簫炖下好茶。玉樓對李嬌兒說:「咱兩家子輪替管茶,休要只顧累了大姐姐這屋裏。」於是各往房裏吩咐預備茶去。不一時,放下炕桌兒,三個姑子來到,盤膝坐在炕上。衆人俱各坐了,擠了一屋裏人,聽他宣卷。月娘洗手炷了香。這薛姑子展開〈黄氏女卷〉,高聲演說道:

\begin{myquote}
「盖聞法初不滅,故緣滅以歸空;道本無生,每因生而不用。由法身以垂八相,由八相以顯法身。朗朗慧燈,通開世戶;明明佛鏡,照破昏衢。百年光景賴刹那,四大幻身如泡影。每日塵勞碌碌,終朝業識忙忙。豈知一性圓明,徒逞六根貪慾。功名蓋世,無非大夢一塲;富貴驚人,難免無常二字。風火散時無老少,溪山磨盡幾英雄。我好十方傳句偈,八部會壇場:救火宅之蒸熬,發空門之扃鑰。偈曰:

富貴貧窮各有由,只緣分定不湏求。

未曾下的春時種,空手荒田望有秋?

衆菩薩們聽我貧僧演説佛法,這四句偈子,乃是老祖留下。如何說『富貴貧窮各有由?』像如今你這衆菩薩嫁得官人,高官厚祿,在這深宅大院,呼奴使婢,插金帶銀。在綾錦窝中長大,綺羅堆裏生成,思衣而綾錦千箱,思食而珍羞百味,享榮華,受富貴,盡皆是你前世因由,根基上有你的一般大緣份,不待求而自得。就是貧僧在此宣經念佛,也是喫着這羙口茶飯,受着發心布施,老大緣分,非同小可。都是龍華一會上的人,皆是前生修下的功果。你不修下時,就如春天不曾下種,到了秋成時候,一片荒田,那成熟結子従那裏來?正是:

淨埽靈臺好下工,得意歡喜不放鬆;

五濁六根爭洗淨,參透玄門見家風。

百歲光陰瞬息囬,此身必定化飛灰;

誰人肯向生前悟,悟却無生歸去來。

人命無常呼吸間,眼觀紅日墜西山。

寶山歷盡空囘首,一失人身萬劫難。

想這富貴榮華,如湯潑雪,仔細算來一件無,都做了虚花驚夢。我今得個人身,心中煩惱悲切,死後四大化作塵土,又不知這點靈魂往何處受苦去也。懼怕生死輪迴,往前再參一步。」唱:

\setlength{\hangindent}{4em}\hangafter=0{\markfont〈一封書〉}「生和死兩厢,嘆浮生終日忙。男和女滿堂,到無常祇自當。人如春夢終須短,命苦風燈不久長。自思量,可悲傷,題起敎人欲斷腸。」

「開卷曰:應身長救苦,並本無去亦無來,彌陀敎主大願弘深,四十八願度衆生,使人人悟本性。彌陀今惟心淨主渡苦海,苦海洪波,證菩提之妙果。持念者罪減河沙,稱揚者福增無量,書寫讀誦者當生華藏之天。見聞受持,臨命終時定往西方淨土。凡念佛者断有功無量,慈愍故,慈愍故,大慈愍故,信禮常住三寳,皈命十方一切佛法僧,法輪常轉度衆生。偈曰:

無上甚深微妙法,百千萬劫難遭遇。

我今見聞得受持,願解如來眞實意。」

\setlength{\hangindent}{4em}\hangafter=0{\markfont〈黄氏寳卷〉}「纔展開,諸佛菩薩降臨來。爐香遍滿虚空界,佛號聲名動九垓。

昔日漢王治世,雨順風調,國泰民安,感得一位善心娘子出世。家住曹州南華縣,黄員外所生一女,端嚴美色,年方七歲,喫齋把素,念〈金剛經〉報答父母深恩,每日不缺,感得觀世音菩薩半空中化魂。父母見他終日念經,苦勸不從。一日尋媒,吉日良時,把他嫁與一婿,姓趙名令方,屠宰為生。為夫婦一十二載,生下一男二女。一日黄氏告其夫曰:『我與你為夫妻一十二載,生下嬌兒嬌女,但貪戀恩愛,永墮沉淪。妾有小詞,勸與丈夫聽取。』詞曰:『宿緣夫婦得成雙,雖有男和女,誰會抵無常?伏望我夫主,定念與奴同,共修行,終天年。富貴也莫羡,莫貪名與利,隨分度時光。』這趙郎見詞,不能依隨。一日作别起身,往山東買猪去。黄氏女見丈夫去了,每日淨房寢歇,沐浴身體,燒香禮誦〈金剛經〉。」白:

「令方當下山東去,三個兒女在中堂。黄氏女,在西房,香湯沐浴;換衣裳,卸簪珥,淺淡梳粧。每日家,向西方,燒香禮拜。面念顔,並寳卷,持念〈金剛〉。看經文,猶未了,香煙衝散。念佛音,聲朗朗,貫徹穹蒼。地獄門,天堂界,豪光發現。閻羅王,一見了,喜悅龍顔。莫不是,陽世間,生下佛祖?急宣召,二鬼判,審問端詳。有鬼判,告吾王,聆音察理:曹州府,南華縣,有一善良;看經文,黄氏女,持齋把素;行善心,功行大,驚動天堂。」唱:

\setlength{\hangindent}{4em}\hangafter=0{\markfont〈金字經〉}「閰羅王,聞言心内忙,急點無常鬼一雙。一雙急趙家庄。黄氏女正看經卷,忽見僊童在面前。」白:

「善人便是童子請,惡人須遣夜叉郎。黄氏看經忙來問:『誰家童子到奴行?』僊童答告娘子道:『善心娘子你莫慌。不是凡間親眷屬,我是陰間童子郎。今因為你看經卷,閻王請你善心娘。』黄氏見說心煩惱,小心一一告無常:『同姓同名勾一個,如何勾我見閻王?千死萬死甘心死,怎捨嬌娃女一雙,大姐嬌姑方九歲,伴嬌六歲怎抛娘?長壽嬌兒年三歲,常抱懷中心怎忘?若放奴家魂一命,多將功德與你行。』僊童答告娘子道:『何人似你念〈金剛〉?』

黄氏哀告二童子,再三不肯赴幽冥,留戀孩兒難抛捨。僊童催促善心娘:『陰間取你三更死,定不容情到四更。不比陽間好轉限,違限你我罪不輕。』黄氏此時心意想,便喚女使去燒湯。香湯沐浴方纔了,將身便乃入佛堂。盤膝坐定不言語,一靈眞性見閻王。」唱:

\setlength{\hangindent}{4em}\hangafter=0{\markfont〈楚江秋〉}「人生夢一場,光陰不久長。臨危個個是風燈樣。看看囬步見閻王,急辦行粧。望鄉臺上把家鄉望,兒啼女哭好悽惶。排鈸打鼓作道場,披麻带孝安塋葬。」白:

「不説令方悽惶事,且言黄氏赴陰靈。看看來到奈何岸,一道金橋接路行。借問此橋作何用?單等看經念佛人。奈何兩邊血浪水,河中多少罪淹魂。悲聲哭泣紛紛鬧,四面毒蛇咬露筋。前到破錢山一座,黄氏向前問原因。是你陽間人化紙,殘燒未了便拋焚。因此挑番多破碎,積聚號作破錢山。又打枉死城下過,多少孤魂未托生。黄氏見說心慈愍,舉口便誦〈金剛經〉。河裏罪人都開眼,刀山劍樹盡成林。湯鑊火池蓮花現,無間地獄瑞雲籠。當下僊童忙不住,急忙便去奏閻君。」唱:

\setlength{\hangindent}{4em}\hangafter=0{\markfont〈山坡羊〉}「黄氏到了那森羅寳殿,有童子先奏說,請了看經人來見。閻羅王便傳召請,黄氏拜在金堦下,不由的跪在面前。有閻君問,你従幾年把〈金剛經〉念起?何年月日感得觀世音出現?這黄女叉手訴說前情來呵:自従七歲喫齋供養聖賢。望上聖聽言,従嫁了兒夫,看經心不減。」白:

「閻君當下忙傳旨,善心娘子你聽因。你念〈金剛〉多少字?幾多點畫接陰陰。甚字起頭甚字落?是何兩字在中間?你若念經無差錯,放你還魂囬世間。黄氏當時堦下立,願王聽奴念〈金剛〉:字有五千四十九,八萬四千點畫行,『如』起頭『行』字住,『荷擔』兩字在中央。黄氏説經猶未了,閻王殿前放毫光。擧手龍顏眞喜悦,放你還魂看世間。黄氏聞知忙便告,願王俯就聽奴言:第一不往屠家去,第二不要染衣行;只願作個善門子,看經念佛過時光。閻王取筆忙判断,曹州張家轉為男。他家積有家財廣,缺少墳前拜孝郎。員外夫妻俱修善,姓名四海廣傳揚。喫罷迷魂湯一盞,張家娘子腹懷躭。十月滿足生一子,左肋紅字有兩行:此是看經黄氏女,曾嫁觀水趙令方;此是看經多因果,得為男子壽延長。張家員外親看見,愛如珍寳喜開顔。」唱:

\setlength{\hangindent}{4em}\hangafter=0{\markfont〈皂羅袍〉}「黄氏在張家托化。轉男身,相凑無差。員外見了喜添花。三年就養成人大。年方七歲,聰明秀發。攻書習字,取名俊達。十八歲科擧登黄甲。」

「卻説張俊達十八歲登科應擧,陞授曹州南華縣知縣。忽然思憶是他本鄉,到縣中赴任之後,先完王糧國稅,然後理論公廳。差兩個公差,即去請趙郎令方:我和他說話。兩個公差不敢怠慢,即到趙家來請令方。」白:

「趙令方,在家中,看經念佛。兩公人,忙唱喏,聽說來因。即時間,忙打扮,來到縣裏。公廳上,忙施禮,且説家門。張知縣,起躬身,便令坐下。敍寒溫,分賓主,捧出茶湯。你是我,親夫主,令方姓趙。我是你,前妻子,黄氏之身。你不信,到靜臺,脱衣親見。左肋下,朱砂記,字寫原因。我大女,嬌姑兒,嫁人去了。第二女,伴嬌姐,嫁了曹眞。長壽兒,我掛牽,守我墳塋。咱兩個,同騎馬,前到先塋。

知縣同令方兒女五人,到黄氏墳前,開棺見屍,容顔不動。囬來做道場七日。令方看〈金剛經〉,瑞雪紛紛,男女五人,總駕祥雲昇天去了。〈臨江僊〉一首為證:

黄氏看經成正果,同日登極楽。五口盡昇天道,善人傳觀音,菩薩來度我。

寳卷已終,佛聖已知。法界有情,同生勝會。南無一乘宗,無量義,真空妙有如來救苦經。諸佛海會悉遙聞,普使河沙同淨土。伏願經聲佛號,上徹天堂,下透地府:念佛者出離苦海,作惡者永墮沉淪;得悟者諸佛引路,放光明照徹十方。東西下迴光返照,南北處親到家鄉。證無生漂舟到岸,小孩兒得見親娘。入母胎三灾不怕,八十劫永遠安康。」偈曰:

「衆等所造諸惡業,自從無始至如今。

靈山失散迷真性,一點靈光串四生。

一報天地蓋載恩,二報日月照臨恩。

三報皇天水土恩,四報爹娘養育恩。

五報祖師傳法恩,六報十類孤魂早超生。

摩訶般若波羅密。」
\end{myquote}

薛姑子宣畢卷,已有二更天氣。先是李嬌兒房内元宵兒㧱了一道茶來,衆人喫了。後孟玉樓房中蘭香㧱了幾樣精製菓菜,一坐壺酒來,又炖了一大壺好茶,與大妗子、段大姐、桂姐衆人喫。月娘又教玉簫㧱出四盒兒細茶食餅糖之類,與三位師父點茶。李桂姐道:「三位師父宣了這一囘卷,也該我唱個曲兒孝順。」月娘道:「桂姐,又起動你唱。」郁大姐道:「等我先唱。」月娘道:「也罷,郁大姐先唱。」申二姐道:「等姐姐唱了,等我也唱個兒與娘們聽。」桂姐不肯,道:「還是我先唱。」因問月娘:「要聽什麽?」月娘道:「你唱『更深靜悄』。」當下桂姐送衆人酒,取過琵琶來,輕舒玉笋,欵跨鮫綃,啟朱唇,露皓齒,唱道:

\begin{myquote}
「更深靜悄,把被兒熏了。看看等到月上花梢,靜悄悄全無消耗。敲殘了更鼓,你便纔來到。見我這臉兒不瞧,來跪在奴身邊告。我故意兒焦,他偸眼兒瞧,甫能咬定牙,其實忍不住笑。」
\end{myquote}

又:

\begin{myquote}
「勤兒推磨,好似飛蛾投火。他將我啞謎兒包籠,我手裏登時猜破。近新來把不住船兒舵,特故裏搬弄心腸軟,一似酥蜜果。者麽是誰,休道是我。便做鐵打人,其實強不過。」
\end{myquote}

又:

\begin{myquote}
「疏狂忒煞,薄情無奈,兩三夜不見你囬來。問着他便撒頑不睬,不由人轉尋思權寜耐。他笑吟吟將被兒伸開,半掩着羅幃待。我推繡鞋不去睬。你若是惱的人慌,只教氣得你害。」
\end{myquote}

又:

\begin{myquote}
「花街柳市,你戀着蜂媒蝶使。我這裏玉潔冰清,你那裏瓜甜蜜柿。恰囬來無酒佯裝醉,只顧裏打草驚蛇,到尋我些風流罪。我欲待撾了你面皮,又恐傷了就裏。待要隨順了他,其實受不的你氣。」
\end{myquote}

桂姐唱畢,郁大姐纔要接琵琶,早被申二姐要過去了。挂在胳膊上,先説道:「我唱個十二月兒〈掛真兒〉與大妗子和娘們聽罷。」於是唱道:「正月十五鬧元宵,滿把焚香天地也燒。……」一套唱畢,月娘笑道:「慢慢兒的說,左右夜長儘着你説。」那時大妗子害夜深困的慌,也沒等的郁大姐唱,喫了茶,就先往月娘房内睡去了。须臾唱完,都散歸各房内睡去了。桂姐便歸李嬌兒房内,段大姐便往孟玉樓房中,三位師父便往孫雪娥後邊房裏睡。郁大姐申二姐與玉簫小玉在那邊炕屋裏睡。月娘同大妗子在上房内睡。俱不在話下。正是:參横斗轉三更後,一鈎斜月到紗わ。

畢竟未知後來如何,且聽下囘分解。

