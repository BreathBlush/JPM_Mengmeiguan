\includepdf[pages={71,72},fitpaper=false]{tst.pdf}
\chapter*{第三十六囬 \\翟謙寄書尋女子 西門慶結交蔡狀元}
\addcontentsline{toc}{chapter}{第三十六囬 翟謙寄書尋女子 西門慶結交蔡狀元}
\markboth{{\titlename}卷之四}{第三十六囬 翟謙寄書尋女子 西門慶結交蔡狀元}


\begin{myquote}
富川遙望劍江西,一片孤雲對夕暉。

有淚應投煙樹断,無書堪寄雁鳞稀。

問安已負三千里,流落空懷十二時。

海闊天高都是念,憑誰為我説歸期!
\end{myquote}

話説次日,西門慶早與夏提刑出郊外,接了新巡按,又到莊上犒勞做活的匠人。至晚來家,有平安進門就禀:「今日有東昌府下文書快手往京裏,順便捎了一封書帕來,説是太師爺府裏翟大爹寄來的書與爹。小的接了,交進大娘房裏去了。那人明日午後來討囬書。」西門慶聽了,走到上房,取書拆開,觀看上面寫着什麽言詞:

\begin{myquote}[\markfont]
\hspace*{4em}「京都侍生翟謙頓首書拜

即擢大錦堂西門大人門下:久仰山斗,未接丰標;屢辱厚情,感媿何盡!前蒙馳諭,生銘刻在心,凡百於老爹左右,無不盡力扶持。所有瑣事,敢托盛价煩凟,想已為我處之矣。今因便鴻,薄具帖金十兩奉賀,兼候起居。伏望俯賜囬音,生不勝感激之至。外新狀元蔡一泉,乃老爺之假子。奉勅回籍省視,道經貴處,仍望留之一飯;彼亦不敢有忘也。至祝至祝。龝後一日信」
\end{myquote}

西門慶看畢,只顧咨嗟不已,説道:「快教小廝叫媒人去!我什麽營生就忘死了,再想不起來。」吴月娘便問:「什麽勾當?你對我説。」西門慶道:「東京太師老爺府裏翟管家,前日有書來,説無子,來央及我這裏替他尋個女子。不拘貧富,不限財禮,只要好的,他要圖生長。粧奩財禮該使多少,教我開了寫去,他一封封過銀子來。往後他在老爺面前,一力好扶持我做官。我一向亂着,上任七事八事,就把這事忘死了,想不起來。來保他又日逐往鋪子裏去了,又不提我。今日他老遠的又教人捎書來,問尋的親事怎樣的了。又寄了十兩折禮銀子賀我。明日原差人來討囬書,你教我怎樣囬答他?敎他就怪死了!呌了媒人,你吩咐他好歹上緊替他尋着。不拘大小人家,只要好女兒,或十五六,十七八的也罷!該多少財禮,我這裏與他。再不,把李大姐房裏綉春,倒好模樣兒,與他去罷。」月娘道:「我説你是個火燎腿行貨子!這兩三個月,你早做什麽來?人家央你一場,替他看個眞正女子去,他也好謝你。那丫頭你又收過他,怎好打發去的!你替他當個事幹,他到明日也替你用的力。如今旋捏佛旋燒香,急水裏怎麽下得槳?比不的買什麽兒,㧱了銀子到市上就買的來了。一個人家閨門女子,好歹不問,也等教媒人慢慢踏看將來。你到説的好容易自在話兒!」西門慶道:「明日他來要回書,怎麽囬答他?」月娘道:「虧你還断事!這些勾當兒便不會打發人?等那人明日來,你多與他些盤纏,寫在書上,囬覆了他去。只説女子尋下了,只是衣服粧奩未辦,還待幾時完畢,這裏差人送去。打發去了,你這裏敎人替他尋也不遲。此一擧兩得其便,纔幹出好事來,也是人家托你一場。」西門慶笑道:「説的有理。」一面呌將陳經濟來,隔夜修了囬書。

次日,下書人來到。西門慶親自出來,問了備細。又問:「蔡狀元幾時船到?好預備接他。」那人道:「小人來時,蔡老爹纔辭朝,京中起身。翟爹説,只怕蔡老爹囬鄉,一時缺少盤纏,煩老爹這裏多少只顧借與他。寫書去翟爹那裏,如數補還。」西門慶道:「你多上覆翟爹,隨他要多少,我這裏無不奉命。」説畢,命陳經濟讓去廂房内管待酒飯。臨去,交割囬書,又與了他五兩路費。那人拜謝,歡喜出門,長行去了。正是:意急欲搖飛虎䩞,心忙抨碎紫花鞭。

看官聽説:當初安忱取中頭甲,被言官論他是先朝宰相安惇之弟,係黨人子孫,不可以魁多士。徽宗御筆逼不得已把蔡蘊擢為第一,做了狀元。投在蔡京門下,做了假子,陞秘書省正字,給假省親。

且説月娘家中,使小廝呌了老馮、薛嫂兒,並別的媒人來,吩咐各處打聽,「人家有好女子,㧱帖兒來説。」不在話下。

一日,西門慶使來保往新河口,打聽蔡狀元船隻,原來和同榜進士安忱同船。這安進士亦因家貧未續親,東也不成,西也不就,辭朝還家續親,因此二人同船。來到新河口,來保㧱着西門慶拜帖來到船上拜見,就送了一分嗄程,酒麵鷄鵝嗄飯鹽醬之類。況且蔡狀元在東京,翟謙已是預先和他説了:「清河縣有老爺門下一個西門千户,乃是大巨家,富而好禮。亦是老爺擡擧,現做理刑官。你到那裏,他必然厚侍。」這蔡狀元牢記在心。見西門慶差人遠來迎接,又餽送如此大禮,心中甚喜。次日到了,就同安進士進城拜西門慶。西門慶已是呌廚子家裏預備下酒席。因在李知縣衙内喫酒,看見有一起蘇州戲子唱的好,問書童兒,説在南門外磨子營兒那裏住。旋呌了四個來答應。蔡狀元那日封了一端絹帕、一部書、一雙雲履;安進士亦是書帕二事、四袋芽茶、四柄杭扇。各具官袍烏紗,先投拜帖進去。西門慶冠冕迎接至廳上,敍禮交拜。家童獻畢贄儀,然後分賓主而坐。

先是蔡狀元擧手欠身説道:「京師翟雲峯甚是稱道賢公,閥閲名家,清河巨族,久仰德望,未能識荆。今得晉拜堂下,為幸多矣。」西門慶答道:「不敢。昨日雲峯書來,具道二位老先生華輈下臨,理當迎接。奈公事所覊,幸為寬恕。」因問:「二位老先生僊鄉、尊號?」蔡狀元道:「學生蔡蘊,本貫滁州之匡廬人也,賤號一泉。僥倖狀元,官拜秘書正字。給假省親,得蒙皇上兪允。不想雲峯先生稱道盛德,拜遲!」安進士道:「學生乃浙江錢塘縣人氏,賤號鳳山。現除工部觀政,亦給假還鄉續親。敢問賢公尊號?」西門慶道:「在下卑官武職,何得號稱。」詢之再三,方言:「賤號四泉。累蒙蔡老爺擡擧,雲峯扶持,襲錦衣千户之職。現任理刑,實為不稱。」蔡狀元道:「賢公抱負不凡,雅望素著,休得自謙。」叙畢禮,就請去花園捲棚内寬衣。蔡狀元辭道:「學生歸心匆匆,行舟在岸,就要囬去。旣見尊顔,又不遽捨,奈何奈何!」西門慶道:「蒙二公不棄蜗居,伏乞暫駐文旆,少留一飯,以盡芹獻之情。」蔡狀元道:「旣是雅情,學生領命。」一面脱去衣服,二人坐下。左右又換了一道茶上來。

蔡狀元以目瞻顧西門慶家園池臺館,花木森秀,一望無際。心中大喜,極口稱羡,誇道:「誠乃勝蓬瀛也!」於是擡過棋桌來下棋。西門慶道:「今日有兩個戯子在此伺候,以供燕賞。」安進士道:「在那裏,何不令來一見?」不一時,四個戯子跪下磕頭。蔡狀元問道:「那兩個是生旦?呌甚名字?」於是走向前説道:「小的是裝生的,呌苟子孝;那一個裝旦的,呌周順;一個貼旦,呌袁琰;那一個裝小生的,呌胡慥。」安進士問:「你們是那裏子弟?」苟子孝道:「小的都是蘇州人。」安進士道:「你等先粧扮了來,唱個我們聽。」四個戲子下邊粧扮去了。西門慶令後邊取女衣釵梳與他,教書童也粧扮起來。共三個旦、兩個生,在席上先唱《香囊記》。大廳正面設兩席,蔡狀元安進士居上,西門慶下邊主位相陪。飲酒中間,唱了一摺下來。安進士看見書童兒裝小旦,便道:「這個戲子是那裏的?」西門慶道:「此是小价書童。」安進士呌上去,賞他酒喫,説道:「此子絶妙,而無以加矣!」蔡狀元又呌別的生旦過來,亦賞酒與他喫。因吩咐:「你唱個〔朝元歌〕『花邊柳邊』。」苟子孝答應,在旁拍手唱道:

\begin{myquote}
「花邊柳邊,簷外晴絲捲。山前水前,馬上東風軟。自歎行踪,有如蓬轉;盼望家鄉留戀。雁杳魚沉,離愁滿懷,誰與傳日短北堂萱?空勞魂夢牽。{\marktext\small(合)}洛陽遙遠,幾時得上九重金殿!」
\end{myquote}

唱了一個,喫畢酒,又唱第二個:

\begin{myquote}
「十載,青燈黄卷。螢窗苦勉旃,雪案費精硏。指望榮親,姓揚名顯;試向文場鏖戰。禮樂三千,英雄五百爭後先。快着祖生鞭,行瞻尺五天。{\marktext\small(合前)}」
\end{myquote}

安進士令苟子孝:「你們可記的《玉環記》『恩德浩無邊』?」苟子孝答道:「此是〔畫眉序〕,小的記得。」

\begin{myquote}
「恩德浩無邊,父母重逢感非淺。幸終身托與,又與姻緣。風雲際會異日飛騰,鸞鳳配今諧繾綣。{\marktext\small(合)}料應夫婦非今世,前生玉種藍田。」
\end{myquote}

書童兒把酒斟上,拍手唱道:

\begin{myquote}
「弱質始笄年,父母恩深浩如天。報無由媿赧,此心縈牽。鴛鴦配深沐親恩,箕箒婦願夫榮顯。{\marktext\small(合前)}」
\end{myquote}

原來安進士杭州人,喜尚南風。見書童兒唱的好,拉着他手兒,兩個一遞一口喫酒。良久,酒闌上來,西門慶陪他復遊花園,向捲棚内下棋。今小廝㧱兩桌盒,三十樣都是細巧菓菜、鮮物下酒。蔡狀元道:「學生們初會,不當深擾潭府。天色晚了,告辭罷。」西門慶道:「豈有此理。」因問:「二公此囬去,還到船上?」蔡狀元道:「暫借門外永福佛寺寄居。」西門慶道:「如今就門外去也晚了。不如老先生把手下従者留下一二人答應,餘者都吩咐囬去,明日來接,庶可兩盡其情。」蔡狀元道:「賢公雖是愛客之意,其如過擾何?」當下二人一面吩咐手下:「都囬門外寺裏歇去,明日早㧱馬來接。」衆人應諾去了,不在話下。二人在捲棚内下了兩盤棋,子弟唱了兩摺。恐天晚,西門慶與了賞錢,打發去了。止是書童一人,席前遞酒伏侍。看看喫至掌燈,二人出來更衣。蔡狀元拉西門慶説話:「此去學生囬鄉省親,路費缺少……」西門慶道:「不勞老先生吩咐,雲峯尊命,一定謹領。」良久,讓二人到花園,「還有一處小亭請看。」把二人一引,轉過粉牆,來到藏春塢,——乃一邊僻靜所。雪洞内裏面曉騰騰掌着燈燭,小琴桌兒早已陳設綺席菓酌之䫫。牀榻依然,琴書瀟灑。従新復飲,書童在旁歌唱。蔡狀元問道:「大官,你會唱『紅入僊桃』?」書童道:「此是〔錦堂月〕,小的記的。」蔡狀元道:「旣是記的,大官你唱。」於是把酒都斟上。那書童㧱住南腔,拍手唱道:

\begin{myquote}
「紅入僊桃,青歸御柳,鶯啼上林春早。簾捲東風,羅襟曉寒猶峭。喜僊姑書付青鸞,念慈母恩同烏鳥。{\marktext\small(合)}風光好,但願人景長春,醉遊蓬島。」
\end{myquote}

安進士聽了,喜之不勝。向西門慶稱道:「此子可敬!」將盃中之酒一吸而飲之。那書童席前穿着翠袖紅裙,勒着銷金箍兒,高擎玉斝,捧上酒去,又唱道:

\begin{myquote}
「難報母氏劬勞,親恩罔極,只願壽比松喬。定省晨昏,連枝尚有兄嫂。喜春風棠棣聯芳,娱晚景松柏同操。{\marktext\small(合前)}」
\end{myquote}

當日飲至夜分,方纔歇息。西門慶藏春塢翡翠軒兩處俱設牀帳,鋪陳綾錦被褥,就派書童玳安兩個小廝答應。西門慶道了安置,囬後邊去了。

到次日,蔡狀元安進士跟従人夫轎馬來接。西門慶廳上擺酒伺候;攢盤酒飯,與脚下人喫。教兩個小廝,方盒捧出禮物:蔡狀元是金緞一端、領絹二端、合香五百、白金一百兩;安進士是色緞一端、領絹一端、合香三百、白金三十兩。蔡狀元固辭再三,説道:「但假十數金足矣,何勞如此太多,又蒙厚腆!」安進士道:「蔡年兄領受,學生不當。」西門慶笑道:「些須微贐,表情而已。老先生榮歸續親,在下此意,少助一茶之需。」於是二人俱席上出來謝道:「此情此德,何日忘之!」一面令家人各收下去,入氈包内。與西門慶相別,説道:「生輩此去,天各一方,暫違台教。不日旋京,倘得寸進,自當圖報。」安進士道:「今日相别,何年再得奉接尊顔!」西門慶道:「學生蜗居屈尊,多有褻慢,幸惟情恕!本當遠送,奈官守在身,先此告過。」送二人到門首,看着上馬而去。正是:博得錦衣歸故里,功名方信是男兒。

畢竟未知後來何如,且聽下囬分解。

