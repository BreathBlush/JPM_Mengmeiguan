\includepdf[pages={185,186},fitpaper=false]{tst.pdf}
\chapter*{第九十三囘 \\王杏庵仗義賙貧 任道士因財惹祸}
\addcontentsline{toc}{chapter}{第九十三囘 王杏庵仗義賙貧 任道士因財惹祸}
\markboth{{\titlename}卷之十}{第九十三囘 王杏庵仗義賙貧 任道士因財惹祸}


\begin{myquote}
誰道人生運不通,吉兇祸福並肩行。

只因風月將身陷,未許人心直似針。

自課官途無枉屈,豈知天道不昭明。

早知成敗皆由命,信步而行暗黑中。
\end{myquote}

話説陳經濟自従西門大姐死了,被吳月娘告了一狀,打了一場官司出來,唱的馮金寳又歸院中去了。剛刮剌出個命兒來,房兒也賣了,本錢兒也沒了,頭面也使了,家伙也沒了。又說陳定在外邊打發人尅落了錢,把陳定也攆去了。家中日逐盤費不週,坐喫山空,不免往楊大郎家中,問他這半船貨的下落。一日來到楊大郎門首,呌聲:「楊大郎在家不在?」不想楊光彦拐了他半船貨物,一向在外,賣了銀兩,四散躱閃。及打聽得他家中吊死了老婆,他丈母縣中告他,坐了半個月監房,這楊大郎驀地來家,住着不出來。聽見經濟上門呌他,問貨船下落,一徑使兄弟楊二風出來,反問經濟要人:「你把我哥哥呌的外邊做買賣,這幾個月通無音信,不知拋在江中,推在河内,害了性命。你倒還來我家尋貨船下落!人命要緊?你那貨物要緊?」這楊二風平時是個刁徒潑皮,耍錢搗子,胳膊上紫肉横生,胸前上黄毛亂長,是一條直率之光棍。走出來一把手扯住經濟,就問他要人。那經濟慌忙掙開手,跑囬家來。這楊二風故意拾了塊三尖瓦楔,將頭顱礸破,血流滿面,趕將經濟來罵道:「我㒲你娘眼!我見你家甚麽銀子來,你來我屋裏放屁,喫我一頓好拳頭!」那陳經濟金命水命,走投無命,奔到家把大門關閉,如鐵桶相似,就是樊噲也撞不開。由着楊二風摔爹娘罵父母,拿大磚砸門,只是鼻口内不聽見氣兒。又况纔打了官司出來,夢條䋲蛇也害怕,只得含忍過了。正是:嫩草怕霜霜怕日,惡人自有惡人磨。

不消幾時,把大房賣了,找了七十兩銀子,典了一所小房,在僻巷内居住。落後兩個丫頭,賣了一個重喜兒,只留着元宵兒和他同舖歇。又過了不上半月,把小房倒騰了,卻去賃房居住。陳安也走了,家中沒營運,元宵兒也死了,止是單身獨自。家伙桌椅都變賣了,只落得一貧如洗。未幾,房錢不給,鑽入冷舖内存身。花子見他是個富家勤兒,生的清俊,呌他在熱坑上睡,與他燒餅兒喫。有當夜的過來,敎他頂火夫,打梆子搖鈴。那時正値臘月殘冬時分,天降大雪,吊起風來,十分嚴寒。這陳經濟打了囬梆子,打發當夜的兵牌過去,不免手提鈴串了幾條街巷。又是風雪,地下又踏着那寒冰,凍得聳肩縮背,戰戰兢兢。臨五更鷄呌,只見個病花子躺在牆底下,恐怕死了,總甲吩咐他看守着他,尋了把草敎他烤。這經濟支更一夜沒曾睡,就歪下睡着了。不想做了一夢,夢見那時在西門慶家,怎生受榮華富貴,和潘金蓮勾搭頑耍戯謔,從睡夢中就哭醒了。衆花子説:「你哭怎的?」這經濟便道:「你衆位哥哥,聽我訴說一遍。」有〔粉蝶兒〕為證:

\begin{myquote}
「九臘深冬,雪漫天凉然冰凍。更搖天撼地狂風。凍得我體僵麻,心膽戰,實難扎掙。挨不過肚中饑,又難禁身上冷。住着這半邊天,端的是冷。挨不過凄凉要尋死路,百忙裏捨不的頽命!」

{\markfont〔耍孩兒一煞〕}「不覺撞昏鐘,昏鐘人初定。是誰人叫我?原來是總甲張成!他那裏急急呼,我這裏連連應。趂今宵誰肯與我支更?也是我一時僥倖,他先遞與我幾個燒餅。」

{\markfont〔二煞〕}「多承總甲憐咱冷,教我敲梆守守更,由着他調用。但得些濟饑錢米,那裏管人貧下賤!一任教喝號提鈴。」

{\markfont〔三煞〕}「坐一囘脚手麻,立一回肚裏疼。冷燒餅乾嚥無茶送。剛然未到三更後,下夜的兵牌叫點燈歪踢弄。與了他四十文,方纔得買一個姑容。」

{\markfont〔四煞〕}「到五更鷄打鳴,大街上人漸行。衆人各去都不等。只見病花子躺在牆根下,敎我煨着他不暫停。得他口煖氣兒心纔定。剛合眼一場幽夢,猛驚囘哭到天明。」

{\markfont〔五煞〕}「花子說你哭怎的?我従頭兒訴始終:我家積祖根基兒重。說聲賣松槁陳家誰不怕?名姓多居仕宦中。我祖爺爺曾把淮鹽種。我父親專結交勢耀,生下我喫酒行兇!」

{\markfont〔六煞〕}「先亡了打我的爺,後亡了我父親。我娘疼,專隨従。喫酒耍錢般般會,酒肆窠窝處處通。所事兒都相稱。娶了親就遭官事,丈人家躱重投輕。」

{\markfont〔七煞〕}「我也曾在西門家做女婿,調風月把丈母淫。錢場裏信着人鑽狗洞。也曾黄金羙玉當場賭,也曾馱米擔柴往院裏供。毆打妻兒病死了,死了時他家告狀。使了許多錢,方得頭輕。」

{\markfont〔八煞〕}「賣大房買小房,贖小房又倒騰。不思久遠含餘剩。饑寒苦惱妾成病,死在房簷不許停。所有都乾淨。嘴頭饞不離酒肉,沒攪計拆賣坟塋!」

{\markfont〔九煞〕}「掇不的輕負不的重,做不的傭務不的農,未曾幹事兒先愁動。閒中無事思量嘴,睡起須教日頭紅。狗性子生鐵般硬。惡盡了十親九眷,凍餓死有那個憐憫!」

{\markfont〔十煞〕}「討房錢不住催,他料我也住不成。沙鍋破碗全無用。幾推趕出門兒外,凍骨淋皮無處存。不免冷舖將身奔。但得個時通運轉,我那其間忘不了恩人。」

「頻年困苦痛妻亡,身上無衣口絶糧;

馬死奴逃房又賣,隻身獨自走他鄉。

朝依肆店求遺饌,暮宿莊園倚敗牆。

只有一條身後路,冷舖之中去打梆。」
\end{myquote}

卻説陳經濟晚夕在冷舖存身,白日間街頭乞食。清河縣城内有一老者,姓王名宣,字廷用,年六十餘歲,家道殷實。為人心慈,好仗義疎財,廣結交,楽施捨,專一濟貧拔苦,好善敬神。所生二子,皆當家成立,長子王乾,襲祖職為牧馬所掌印正千戶;次子王震,現為府學庠生。老者門首搭了個主管,開着個解當舖兒。每日豐衣足食,閒散無拘,在梵宇聽經,琳宫講道。無事在家門首施薬救人,撚素珠念佛。因後園中有兩株杏樹,道號為杏庵居士。

一日,杏庵頭戴重簷幅巾,身穿水合道服,在門首站立。只見陳經濟打他門首過,向前趴在地下磕了個頭。慌的杏庵還禮不迭,説道:「我的哥,你是誰?老拙眼昏不認得你。」這經濟戰戰兢兢站立在旁邊,說道:「不瞞你老人家,小人是賣松槁陳洪兒子。」老者想了半日說:「你莫不是陳大寬的令郎麽?」因見他衣服襤褸,形容憔悴,說道:「我賢侄,你怎的弄得這等模樣?」便問:「你父親母親可安麽?」經濟道:「我爹死在東京,我母親也死了。」杏庵道:「我聞得你在丈人家住來?」經濟道:「家外父死了,外母把我攆出來。他女兒死了,告我到官,打了一場官司,把房兒也賣了。有些本錢兒,都喫人坑了。一向閒着,沒有營運。」杏庵道:「賢姪,你如今在那裏居住?」經濟半日不言語,說:「不瞞你老人家說,如此如此。」杏庵道:「可憐。賢姪,你原來討喫哩!想着當初,你府上那樣根基人家!我與你父親相交,賢姪你那咱還小哩,纔扎着總角上學哩。一向流落到此地位,可傷可傷!你還有甚親家,也不看顧你看顧兒?」經濟道:「正是。俺張舅那裏,一向也久不上門,不好去的。」問了一囘話,老者把他讓到裏面客位裏,令小廝放桌兒,擺出點心嗄飯來,教他儘力喫了一頓。見他身上單寒,拿出一件青布綿道袍兒,一頂毡帽,又一雙毡襪綿鞋,又秤一兩銀子、五百銅錢,遞與他,吩咐說:「賢侄,這衣服鞋襪,與你身上穿;那銅錢與你盤纏,賃半間房兒住。這一兩銀子,你拿着做上些小買賣兒,也好糊口過日子,強如在冷舖中,學不出好人來。每月該多少房錢,來這裏,老拙與你。」

這陳經濟趴在地下磕頭謝了,說道:「小姪知會!」拿着銀錢,出離了杏庵門首。也不尋房子,也不做買賣,把那五百文錢,每日只在酒店麵店以了其事;那一兩銀子,搗了些白銅炖罐,在街上行使。喫巡邏的當土賊拿到該坊節級處,一頓拶打,使的罄盡,還落了一屁股瘡。不消兩日,把身上綿衣也輸了,襪兒也換來嘴喫了,依舊原在街上討喫。

一日,又打王杏庵門首所過。杏庵正在門首,只見經濟走來磕頭,身上衣襪都沒了,止戴着那毡帽,精脚靸鞋,凍的乞乞縮縮。老者便問:「陳大官,做得買賣如何?房錢到了,來取房錢來了?」那陳經濟半日無言可對。問之再三,方說:「如此這般,都没了。」老者便道:「阿呀,賢姪!你這等就不是過日子的道理。你又撚不的輕,負不的重,但做了些小活路兒,還強如乞食,免教人耻笑,有玷你父祖之名。你如何不依我説?」一面又讓到裏面,敎安童拿飯來與他喫飽了。又與了他一條袷褲,一領白布衫,一雙裹脚,一吊銅錢,一斗米:「你拿去務要做上个小買賣,賣些柴炭荳兒、瓜子兒,也過了日子,強似這等討喫。」這經濟口雖答應,拿錢米在手,出離了老者門,那消數日,熟食肉麵,都在冷舖内和花子打夥兒都喫了。耍錢又把白布衫袷褲都輸了。大正月裏,又抱着肩兒在街上走。不好來見老者,走在他門首房山牆底下,向日陽站立。老者冷眼看見他,不呌他。他挨挨搶搶,又到跟前,趴在地下磕頭。老者見他還依舊如此,說道:「賢姪,這不是常策。咽喉深似海,日月快如梭,無底坑如何塡得起?你進來,我與你說。有一個去䖏,又清閒,又安得你身,只怕你不去。」經濟跪下哭道:「若得老伯見憐,不拘那裏,但安下身,小的情願就去。」杏庵道:「此去離城不遠,臨清馬頭上,有座晏公廟。那裏魚米之鄉,舟船輻輳之地,錢糧極廣,清幽瀟灑。廟主任道士,與老拙相交極厚,他手下也有兩三個徒弟徒孫。我備分禮物,把你送與他做個徒弟出家,學些經典吹打,與人家應福,也是好處。」經濟道:「老伯看顧,可知好哩。」杏庵道:「旣然如此,你去。明日是個好日子,你早來,我送你去。」經濟去了,這王老連忙叫了裁縫來,就替經濟做了兩件道衣,一頂道髻,鞋襪俱全。

次日,經濟果然來到。王老敎他空屋裏洗了澡,梳了頭,戴上道髻,裏外換了新襖新褲。上蓋青絹道衣,下穿雲履毡襪。備了四盤羹果,一罈酒,一疋尺頭,封了五兩銀子,他便乘馬,僱了一疋驢兒與經濟騎着。安童喜童跟隨,兩個人擡了盒擔,出城門逕往臨清馬頭晏公廟來。止七十里,一日路程。比及到晏公廟,天色已晚。但見:

\begin{myquote}
日影將沉,繁陰已轉。斷霞映水散紅光,落日薄山生碧霧。綠楊影裏,時聞鳥雀歸林;紅杏村中,每見牛羊入圈。正是:溪邊漁父投林去,野外牧童跨犢歸。
\end{myquote}

王老到於馬頭上,過了廣濟閘大橋,見無數舟船,停泊在河下。來到晏公廟前下馬,進入廟來。只見青松欝欝,翠柏森森。兩邊八字紅牆,正面三間朱戶。端的好座廟宇!但見:

\begin{myquote}
山門高聳,殿閣崚層。高懸勅額金書,彩畫出朝入相。五間大殿,塑龍王一十二尊;兩下長廊,刻水族百千萬衆。旗竿凌漢,帥字招風。四通八逹,春秋社稷享依時;雨順風調,河道民間皆祭賽。萬年香火威靈在,四境官民仰賴安。
\end{myquote}

山門下,早有小童看見,報入方丈。任道士忙整衣出迎。王杏庵令經濟和禮物且在外邊伺候。不一時,任道士把杏庵讓入方丈松鶴軒叙禮,説:「王老居士,怎生一向不到敝廟隨喜?今日何幸,得蒙下顧!」杏庵道:「只因家中俗冗所覊,久失拜望。」叙禮畢,分賓主而坐,小童獻茶。茶罷,任道士道:「老居士今日天色已晚,你老人家不去罷了。」吩咐把馬牽入後槽餵息。杏庵道:「沒事不登三寳殿。老拙敬來有一事干凟,未知尊意肯容納否?」任道士道:「老居士有何見教?只顧吩咐,小道無不領命。」杏庵道:「今有故人之子,姓陳名經濟,年方二十四歲。生的資格清秀,倒也伶俐。只是父母去世太早,自幼失學。若說他父祖根基,也不是無名少姓人家子孫,有一分家當。只因不幸遭官事沒了,今無處棲身。老拙念他乃尊舊日相交之情,欲送他來貴宫作一徒弟,未知尊意如何?」任道士便道:「老居士吩咐,小道怎敢違阻!奈因小道命蹇,手下雖有兩三個徒弟,都不省事,沒一個成立的,小道常時惹氣。未知此人誠實不誠實?」杏庵道:「這個小的,不瞞尊師說,只顧放心!一味老實本分,膽兒又小,所事兒伶範,堪可作一徒弟。」任道士問:「幾時送來?」杏庵道:「現在山門外伺候。還有些薄禮,伏乞笑納。」慌的任道士道:「老居士何不早說?」一面道:「有請!」於是擡盒人擡進禮物。任道士見帖兒上寫着:「謹具粗緞一端,魯酒一樽,豚蹄一副,燒鴨二隻,樹果二盒,白金五兩。知生王宣頓首拜。」連忙稽首謝道:「老居士何以遠勞見賜許多重禮!使小道卻之不恭,受之有愧!」只見陳經濟頭戴着金梁道髻,身穿青絹道衣,脚下雲履淨襪,腰繫絲縧,生的眉清目秀,齒白唇紅,面如傅粉,走進來向任道士倒身下拜,拜了四雙八拜。任道士因問:「多少青春?」經濟道:「屬馬,交新春二十四歲了。」任道士見他果然伶俐,取了他個法名,叫做「陳宗羙」。原來任道士手下,有兩個徒弟:大徒弟姓金,名宗明;二徒弟姓徐,名宗順。他便叫陳宗羙。王杏庵都請出來,見了禮數。一面收了禮物,小童掌上燈來,放桌兒,先擺飯,後喫酒。餚品盃盤,堆滿桌上,無非是鷄蹄鵝鴨魚蝦之類。王老喫不多酒。師徒輪番勸夠幾巡,王老不勝酒力,告辭。房中自有牀舖,安歇一宿。

到次日清辰,小童舀水淨面,梳洗盥漱畢,任道士又早來遞茶。不一時擺飯,又喫了兩盃酒,餵飽頭口,與了擡盒人力錢。王老臨起身,呌過經濟來吩咐:「在此好生用心,習學經典,聽師父指敎。我常來看你,按季送衣服鞋脚來與你。」又向任道士說:「他若不聽敎訓,一任責治,老拙並不護短。」一面背地又囑付經濟:「我去後,你要洗心改正,習本等事業。你若再不安分,我不管你了。」那經濟應諾道:「兒子理會了。」王老當下作辭任道士,出山門上馬,離晏公廟囘家去了。

經濟自此就在晏公廟做了道士。因見任道士年老赤鼻,身體魁偉,聲音洪亮,一部髭髯,能談善飲,只專迎賓送客,凡一應大小事,都在大徒弟金宗明手裏。那時朝廷運河初開,臨清設二閘,以節水利。不拘官民,船到閘上,都來廟裏或求神福,或來祭願,或討卦與笤,或做好事。也有布施錢米的,也有餽送香油紙燭的,也有留松篙蘆蓆的。這任道士將常住裏多餘錢糧,都令手下徒弟在馬頭上開設錢米舖,賣將銀子來,積攢私囊。

他這大徒弟金宗明也不是個守本分的。年約三十餘歲,常在娼樓包占楽婦,是個酒色之徒。手下也有兩個清潔年小徒弟,同鋪歇臥,日久絮煩。因見經濟生的齒白唇紅,面如傅粉,清俊乖覺,眼裏說話,就纏他同房居住。晚夕和他喫半夜酒,把他灌醉了,在一舖歇臥。初時兩頭睡,便嫌經濟脚臭,叫過一個枕頭上睡。睡不多囬,又說他口氣噴着,令他掉轉身子,屁股貼着肚子。那經濟推睡着不理他,他把那話弄得硬硬的,直豎一條棍,抹了些唾津在頭上,往他糞門裏只一頂。原來經濟在冷舖中,被花子飛天鬼侯林兒弄過的,眼子大了,那話不覺就進去了。這經濟口中不言,心内暗道:「這廝合敗!他討得十分便益多了,把我不知當做甚麽人兒,也來托大。與他個甜頭兒,且教他在我手内納些敗缺!」一面故意聲呌起來。這金宗明恐怕老道士聽見,連忙掩住他口,説:「好兄弟,禁聲!隨你要的,我都依你。」經濟道:「你旣要勾搭我,我不言語,須依我三件事。」宗明道:「好兄弟,休說三件,就是十件事,我也依你。」經濟道:「第一件,你旣要我,不許你再和那兩個徒弟睡。第二件,大小房門上鑰匙,我要執掌。第三件,隨我往那裏去,你休嗔我。你都依了我,我方依你此事。」金宗明道:「這個不打緊,我都依你。」當夜兩個顛來倒去,整狂了半夜。這陳經濟自幼風月中撞,甚麽事不知道!當下被底山盟,枕邊海誓,淫聲艷語,摳吮㖭品,把這金宗明哄得歡喜無盡。到第二日,果然把各䖏鑰匙都交與他手内;就不和那兩個徒弟在一䖏,每日只同他一舖歇臥。

一日兩,兩日三,忽一日任道士師徒三個,都往人家應福做好事去。任道士留下他看家,徑智賺他,「王老居士只說他老實,看老實不老實!」臨出門吩咐:「你在家好生看着。」那後邊養的一羣鷄,說道是鳳凰:「我不久功成行滿,騎他上昇,朝參玉帝。那房内做的幾缸,都是毒薬汁,若是徒弟壞了事,我也不打他,只與他這毒薬汁喫了,直敎他立化。你須用心看守!我午齋囘來,帶點心與你喫。」說畢,師徒去了。這經濟關上門笑道:「豈有我這些事兒不知道?那房内幾缸黄米酒,哄我是甚毒薬汁;那後邊養的幾隻鷄,說是鳳凰,要騎他上昇。」於是揀肥的宰了一隻,退的淨淨,煮在鍋裏。把缸内酒用鏇子舀出來,火上篩熱了。手撕鷄肉,蘸着蒜醋,喫了個不亦楽乎!還說了四句:「黄銅鏇舀清酒,煙籠皓月;白ゐ鷄蘸爛蒜,風捲殘雲。」正喫着,只聽師父任道士外邊叫門。這經濟連忙收拾了家伙,走出來開門。任道士見他臉紅,問他怎的來?這經濟徑低頭不言語。師父問:「你怎的不言語?」經濟道:「告禀師父得知:師父去後,後邊那鳳凰不知怎的飛了去一隻。敎我慌了,上房尋了半日,沒有。怕師父來家打,待要拿刀子抹,恐怕疼;待要上吊,恐怕斷了繩子跌着;待要投井,又怕井眼小掛脖子。算計的沒䖏去了,把師父缸内的毒薬汁,舀了兩碗來喫了。」師父便問:「你喫下去覺怎樣的?」經濟道:「喫下去半日不死不活的,倒像醉了的一般。」任道士聽言,師徒們都笑了,說:「還是他老實。」又替他使錢討了一張度牒。以此往後凡事並不防範。正是:三日賣不得一擔眞,一日賣了三擔假。

這陳經濟因此常拿着銀錢,往馬頭上遊翫。看見院中架兒陳三兒,説:「馮金寳兒他鴇子死了,他又賣在鄭家,叫鄭金寳兒。如今又在大酒樓上趕趂哩,你不看他看去?」這小夥兒舊情不改,拿着銀錢跟定陳三兒,逕往馬頭大酒樓上來。此不來倒好,若來,正是:五百載寃家來聚會,數年前姻眷又相逢。有詩為證:

\begin{myquote}
人生莫惜金縷衣,人生莫負少年時。

見花欲折須當折,莫待無花空折枝!
\end{myquote}

原來這座酒樓,乃是臨清第一座酒樓,名喚謝家酒樓。裏面有百十座閣兒,周そ都是綠欄杆。就緊靠着山崗,前臨官河,極是人煙熱鬧去處,舟船往來之所。怎見得這座酒樓齊整?

\begin{myquote}
雕簷映日,畫棟飛雲,綠欄杆低接軒窻,翠簾櫳高懸戶牖。吹笙品笛,盡都是公子王孫;執盞擎盃,擺列着歌姬舞女。消磨醉眼,倚青天萬疊雲山;勾惹吟魂,翻瑞雪一河煙水。白蘋渡口,時聞漁父鳴榔;紅蓼灘頭,每見釣翁擊楫。樓畔綠楊啼野鳥,門前翠柳繫花驄。
\end{myquote}

這陳三兒引經濟上樓,到一個閣兒裏坐下。烏木春檯,紅漆凳子。便叫店小二連忙打抹了春檯,拿一付鍾筯,安排一分上品酒菓下飯來擺着,使他下邊叫粉頭去了。須臾,只聽樓梯響,馮金寳上來,手中拿着個廝鑼兒,見了經濟,深深道了萬福。常言情人見情人,不覺簇地兩行淚下。正是:數聲嬌語如鶯囀,一串珍珠落線頭!經濟一見,便拉他一䖏坐,問道:「姐姐,你一向在那裏來,不見你!」這馮金寳收淚道:「自従縣中打斷出來,我媽着了驚唬,不久得病死了。把我賣在鄭五媽兒家做粉頭。這兩日子弟稀少,不免又來在臨清馬頭上趕趁酒客。昨日聽見陳三兒説,你在這裏開錢舖,要見你一見。不期你今日在此樓上喫酒,會見一面,可不想殺我也。」說畢,又哭了。經濟便取袖中帕兒,替他抹了眼淚,說道:「我的姐姐,你休煩惱,我如今又好了。自従打出官司來,家業都沒了。投在這晏公廟,一向出家做了道士。師父甚是重托我。往後我常來看你。」因問:「你如今在那裏安下?」金寳便說:「奴就在這橋西洒家店劉二那裏,有百十間房子,四外行院窠子妓女,都在那裏安下。白日裏便來這各酒樓趕趁。」說着,兩個挨身做一處飲酒。陳三兒盪酒上樓,拿過琵琶來。金寳彈唱了個曲兒與經濟下酒,名〔普天楽〕:

\begin{myquote}
「淚雙垂,垂雙淚。三盃别酒,別酒三盃。鸞鳳對拆開,拆開鸞鳳對。嶺外斜暉看看墜,看看墜嶺外斜暉。天昏地暗,地暗天昏。徘徊不捨,不捨徘徊!」
\end{myquote}

兩人喫得酒濃時,未免解衣雲雨,下個房兒。這陳經濟一向不曾近婦女,久渴的人,今得遇金寳,儘力盤桓。尤雲殢雨,未肯即休。但見:

\begin{myquote}
一個玉臂忙搖,一個柳腰款擺。雙睛噴火,星眼郎當。一個汗浹胸膛,發狠要贏三五陣;一個香消粉黛,呻吟叫夠數千聲。戰良久,靈龜深入性偏剛;鬬多時,一股清泉往裏邈。幾番鏖戰煙蘭妓,不似今番這一遭。
\end{myquote}

須臾事畢,各整衣衫。經濟見天色晚來,與金寳作别,與了金寳一兩銀子,與了陳三兒三百文銅錢。囑咐:「姐姐,我常來看你,咱在這搭兒裏相會。你若想我,使陳三兒叫我去!」下樓來,又打發了店主人謝三郎三錢銀子酒錢。經濟囘廟中去了。這馮金寳送至橋邊方囘。正是:盼穿秋水因錢鈔,哭損花容為鄧通!

畢竟未知如何,且聽下囘分解。

