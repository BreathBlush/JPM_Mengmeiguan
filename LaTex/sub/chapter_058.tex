\includepdf[pages={115,116},fitpaper=false]{tst.pdf}
\chapter*{第五十八囬 \\懷妒忌金蓮打秋菊 乞臘肉磨鏡叟訴寃}
\addcontentsline{toc}{chapter}{第五十八囬 懷妒忌金蓮打秋菊 乞臘肉磨鏡叟訴寃}
\markboth{{\titlename}卷之六}{第五十八囬 懷妒忌金蓮打秋菊 乞臘肉磨鏡叟訴寃}


\begin{myquote}
綉幃寂寂思懨懨,萬種新愁日夜添。

一雁呌羣秋度塞,亂蛩吟苦月當簷。

藍橋失路悲紅線,金屋無人下翠簾。

何似湘江江上竹,至今猶被淚痕沾。
\end{myquote}

話説當日西門慶前廳陪親朋飲酒,吃的酩酊大醉,走入後邊孫雪娥房裏來。雪娥正顧灶上看收拾家伙。聽見西門慶往後邊去,慌的兩步做一步走。先前郁大姐正在他炕上坐的,一面攛掇他往月娘炕屋裏和玉簫小玉一處睡去了。原來孫雪娥在後邊,也住着一明兩暗三間房,一間床房,一間炕房。西門慶也有一年多沒進他房中來,聽見今日進來,連忙向前替西門慶接了衣服,安頓中間椅子上坐的。一面在房中揩抹涼蓆,收拾床鋪,薰香澡牝。走來遞茶與西門慶吃了,攙扶進房中,上床脱靴解帶,打發安歇。一宿無話。

到次日廿八,乃西門慶正生日。剛燒畢紙,只見韓道國後生胡秀到了門首下頭口,左右禀報與西門慶。西門慶叫胡秀到廳上,磕頭見了,問他:「貨船在那裏?」這胡秀遞上書帳,悉把「韓大叔在杭州置了一萬兩銀子緞絹貨物,現今直抵臨清鈔關,缺少税鈔銀兩。討了銀兩方纔納稅起脚,裝載進城」,具禀一遍。這西門慶一面看了書帳,心中大喜。吩咐棋童看飯與胡秀喫了,敎他往喬親家爹那裏見見去。不一時,胡秀喫畢飯去了。西門慶進來對吴月娘説:「如此這般,韓夥計貨船到了臨清,使了後生胡秀送書帳上來。如今少不的把對門房子打掃,卸到那裏,尋夥計,收拾裝鑲土庫,開舖子發賣。」月娘聽了,便説:「你上緊尋着。也不早了,還要慢慢的?」西門慶道:「如今等應二哥來,我就對他説,教他上緊尋覓。」不一時,應伯爵來了。西門慶在廳上陪着他坐,對他説:「韓夥計杭州貨船到了,缺少個夥計發賣。」伯爵就説:「哥,恭喜!今日華誕的日子貨船到,决增十倍之利,喜上加喜!哥若尋賣手,不打緊,我有一相識,卻是父交子往的朋友,原是這緞子行賣手,連年運拙,閒在家中。今年纔四十多歲,正是當年漢子。眼力看銀水是不消説,寫算皆精,又會做買賣。此人姓甘,名潤,字出身,現在石橋兒巷住,倒是自己房兒。」西門慶道:「若好,你明日請他見我。」

正説着,只見李銘吳惠鄭奉三個先來,趴在地下磕頭,起來旁邊站立。不一時,雜耍樂工都到了。廂房中打發喫飯,就把桌子擺下,與李銘吳惠鄭奉三個同喫。只見答應的節級拿票來回話:「小的叫了唱的,止有鄭愛月兒不到。他家鴇子説,收拾了纔待來,被王皇親家人攔的往宅裏唱去了。小的只叫了齊香兒、董嬌兒、洪四兒三個,收拾了便來也。」西門慶聽見他不來,便道:「胡説,怎的不來?」便叫過鄭奉問:「怎的你妹子我這裏叫他不來?果係是被王皇親家攔了去?」那鄭奉跪下便道:「小的另住,不知道。」西門慶道:「你説往王皇親家唱就罷了?敢量我就拿不得來!」便叫玳安兒近前吩咐:「你多帶兩個排軍,就拿我個侍生帖兒,到王皇親家宅内,見你王二老爹,就説是我這裏請幾位客人吃酒,這鄭月兒答應下兩三日了,好歹放了他來。倘若推辭,連那鴇子都與我鎖了,墩在門房兒裏!這等可惡,叫不得來就罷了?」一面呌鄭奉:「你也跟了去。」那鄭奉又不敢不去。走出外邊來,央及玳安兒説道:「安哥,你進去,我在外邊等着罷。一定是王二老爹府裏叫,怕不的還没收拾去哩。有累安哥,若是没動身,看怎的將就教他好好的來罷。」玳安道:「若果然往王家宅裏去了,等我拿帖兒討去。若是在家藏着,你進去對他媽説,教他快收拾一答兒來。俺就與你替他回護兩句言語兒,爹就罷了。你們不知道他性格。他従夏老爹宅定下,你不來,他可知惱了哩。」這鄭奉一面先往家中説去了。玳安同兩個排軍,一名節級,後邊走着。

且説西門慶打發玳安鄭奉去了,因向伯爵道:「這個小淫婦兒,這等可惡!在別人家唱,我這裏叫他不來。」伯爵道:「小行貨子,他曉的甚麽?他還不知你的手段哩。」西門慶道:「我倒見他酒席上説話兒伶俐,叫他來唱兩日試他,倒這等可惡!」伯爵道:「哥今日揀的這四個粉頭,都是出類拔萃的尖兒了,再無有出在他上的了。」李銘道:「二爹,你還没見愛月兒哩。」伯爵道:「我跟你爹在他家喫酒,他還小哩。這幾年倒沒曾見,不知出落的怎樣的了。」李銘道:「這小粉頭子,雖故好個身段兒,光是一味粧飾。唱曲也會,怎生趕的上桂姐的一半兒唱。爹這裏是那裏,叫着敢不來?就是來了,虧了你?還是不知輕重。」只見胡秀來囬話:「小的到喬爹那邊見了來了,伺候老爺示下。」西門慶叫陳經濟:「後邊討五十兩銀子來。令書童寫一封書,使了印色,差一名節級,明日早起身一同去下與你鈔關上錢老爹,教他過稅之時青目一二。」湏臾,陳經濟取了一封銀子來交與胡秀。胡秀禀道:「小的往韓大叔家歇去。」便領了文書并税帖,次日早同節級起身,不在話下。

忽聽喝的道子响,平安來報:「劉公公與薛公公來了。」西門慶即冠帶迎接至大廳,見畢禮數,請至捲棚内,寬去上蓋蟒衣,上面設兩張校椅坐下。應伯爵在下,與西門慶關席陪坐。薛内相便問:「此位是何人?」西門慶道:「去年老太監會過來,乃是學生故友應二哥。」薛内相道:「卻是那快耍笑的應先兒麽?」那應伯爵欠身道:「老公公還記的,就是在下。」須臾,拿茶上來喫了。只見平安走來禀道:「府裏周爺差人拿帖兒來,説今日還有一席,來遲些。教老爹這裏先坐,不須等罷。」西門慶看了帖兒,便説:「我知道了。」薛内相因問:「西門大人,今日誰來遲?」西門慶道:「周南軒那邊還有一席,使人來説,上坐休等他哩,只怕來遲些。」薛内相道:「既來説,咱虚着他席面就是。」上面只見兩個小廝上來,一邊一個打扇。

正説話之間,王經拿了兩個帖兒進來:「兩位秀才來了。」西門慶見帖兒上一個是侍生倪鵬、一個溫必古。西門慶就知倪秀才擧薦了他同窗朋友來了,連忙出來迎接。見都穿着衣巾進來,且不看倪秀才,觀看那溫必古:年紀不上四旬,生的明眸皓齒,三牙鬚;丰姿洒落,擧止飄逸。未知行藏何如,先觀動靜若是。有幾句道得他好:

\begin{myquote}
雖抱不羈之才,慣遊非禮之地。功名蹭蹬,豪傑之志已灰;家業凋零,浩然之氣先喪。把文章道學,一併送還了孔夫子;將致君澤民的事業,及榮身顯親的心念,都撇在東洋大海。和光混俗,惟其利欲是前;隨方逐圓,不以廉耻為重。峨其冠,博其帶,而眼底旁若無人;席上闊其論,高其談,而胸中實無一物。三年叫案,而小考尚難,豈望月桂之高攀;廣坐啣盃,遯世無悶,且作岩穴之隱相。
\end{myquote}

西門慶讓至廳上敍禮,每人遞書帕二事與西門慶祝壽。交拜畢,分賓主而坐。西門慶問道:「久仰溫老先生大才,敢問尊號?」溫秀才道:「學生賤名必古,字日新,號葵軒。」西門慶道:「葵軒老先生。」又問:「貴庠?魁經?」溫秀才道:「學生不才,府學備數,初學《易經》。一向久仰尊府大名,未敢進拜。昨因我這敝同窗倪桂岩道及老先生盛德,敢來登堂恭謁。」西門慶道:「不敢。承老先生先施,學生容日奉拜。只因學生一個武官,粗俗不知文理,往來書柬無人代筆。前者因在我這敝同僚府上,會遇桂岩老先生,甚是稱道老先生大才盛德。正欲趨拜請教,不意老先生下降,兼承厚貺,感激不盡。」温秀才道:「學生匪才薄德,繆承過譽。」茶罷,西門慶讓至捲棚内。有薛劉二老太監在座,薛内相道:「請二位老先生寬衣進來。」西門慶一面請寬了青衣,進裏面,各遜讓再四,方纔一邊一位垂首坐下。

正敍談間,吴大舅范千户到了,叙禮坐定。不一時,玳安與同答應的和鄭奉都來回話道:「四個唱的都叫來了。」西門慶問:「是王皇親那裏不是?」玳安道:「是王皇親宅内叫。還沒起身,小的要拴他鴇子墩鎖,他慌了,纔上轎,都一答兒來了。」西門慶即出來,到廳臺基上站立。只見四個唱的一齊進來,向西門慶花枝颭招,綉帶飄飄,都插燭也似磕下頭去。那鄭愛月兒穿着紫紗衫兒,白紗挑線裙子,頭上鳳釵半卸,寳髻玲瓏,腰肢嬝娜,猶如楊柳輕盈;花貌娉婷,好似芙蓉豔麗。正是:萬種風流無䖏買,千金良夜實難消。西門慶便向鄭愛月兒道:「我叫你,如何不來?這等可惡,敢量我拿不得你來!」那鄭愛月兒磕了頭起來,一聲兒也不言語,笑着同衆人一直往後邊去了。

到後邊,與月娘衆人都磕了頭。看見李桂姐吳銀兒都在跟前,各道了萬福,説道:「你二位來的早。」李桂姐道:「俺們兩日没家去了。」因説:「你四個怎的這咱纔來?」董嬌兒道:「都是月姐帶累的俺們來遲了!收拾下,只顧等着他,白不起身。」那鄭愛月兒用扇兒遮着臉兒,只是笑,不做聲。月娘便問:「這位大姐是誰家的?」董嬌兒道:「娘不知道,他是鄭愛香兒的妹子鄭愛月兒,纔成人還不上半年光景。」月娘道:「可倒好個身段兒。」説畢,看茶吃了。一面放桌兒擺茶與衆人吃。那潘金蓮且只顧揭起他裙子,撮弄他的脚看,説道:「你們這裏邊的樣子,只是忒直尖了。不像俺外邊的樣子趫。俺外邊尖的停匀,你裏邊的後跟子大。」月娘向大妗子道:「偏他恁好百勝,問他怎的!」一回又取下他頭上金魚撇杖兒來瞧,因問:「你這樣兒是那裏打的?」鄭愛月兒道:「是俺裏邊銀匠打的。」湏臾擺下茶,月娘便叫:「桂姐、銀姐,你陪他四個吃茶。」不一時,六個唱的做一處,同吃了茶。李桂姐吴銀兒便向董嬌兒四個説:「你們來花園裏走走。」董嬌兒道:「等我們到後邊就來。」

這李桂姐和吴銀兒就跟着潘金蓮孟玉樓出儀門,往花園中來。因有人在大捲棚内,就不曾過那邊去。只在這邊,看了回花草,就往李瓶兒房裏看官哥兒。官哥心中又有些不自在,睡夢中驚哭,吃不下奶去。李瓶兒在屋裏守着不出來,看見李桂姐吴銀兒和孟玉樓潘金蓮進來,連忙讓坐的。桂姐問道:「哥兒睡哩?」李瓶兒道:「他哭了這一日,我打發他面朝裏床纔睡下了。」玉樓道:「大娘説請劉婆子來看他看,你怎的不使小廝快請去?」李瓶兒道:「今日他爹的好日子,明日請他去罷。」正説話中間,只見四個唱的和西門大姐小玉走來。大姐道:「原來你們都在這裏,卻教俺花園内尋你。」玉樓道:「花園内有人在那裏,咱們不好去的。瞧了瞧兒就來了。」李桂姐問洪四兒:「你們四個在後邊,做甚麽這半日纔來?」洪四兒道:「俺們在後邊四娘房裏吃茶來,坐了這一回。」潘金蓮聽了,望着玉樓李瓶兒笑,問洪四兒:「誰對你説是四娘來?」董嬌兒道:「他㽞俺們在房裏吃茶來,他們問來:『還不曾與你老人家磕頭,不知娘是幾娘?』他便説:『我是你四娘哩。』」金蓮道:「沒廉耻的小婦人,別人稱道你便好,誰家自己稱是四娘來?這一家大小,誰興你?誰數你?誰叫你是四娘?漢子在屋裏睡了一夜兒,得了些顔色兒,就開起染房來了!若不是大娘房裏有他大妗子,他二娘房裏有桂姐,你房裏有楊姑奶奶,李大姐便有銀姐在這裏,我那屋裏有他潘姥姥,且輪不到往你那屋裏去哩。」玉樓道:「你還沒曾見哩,今日早晨起來,打發他爹往前邊去了。在院子裏呼張喚李的,便那等花哨起來!」金蓮道:「常言道:奴才不可逞,小孩兒不宜哄。」又問小玉:「我聽見你爹對你奶奶説,替他尋丫頭子與他。説你爹昨日到他屋裏,見他只顧收拾不完,問他到底怎麽,那小淫婦做勢兒對你爹説:『我白日不得個閑收拾屋裏,只好晚夕來這屋裏睡罷了。』你爹説:『不打緊,到明日對你娘説,尋一個丫頭子與你使便了。』真個有此話?」小玉道:「我不曉的,敢是玉簫他聽見來。」金蓮向桂姐道:「你爹不是你各房裏有人,等閒不往他後邊去。莫不俺們背地説他,本等他嘴頭子不達時務,慣傷犯人。俺們急切不和他説話。」正説着,綉春拿了茶上來,每人一盞果仁泡茶。正吃間,忽聽前邊鼓楽響動,荆都監衆人都到齊了,遞酒上坐。玳安兒來叫,四個唱的就往前邊去了。

那日喬大户沒來。先是雜耍百戯,吹打彈唱,隊舞弔罷,做了個笑樂院本。割切上來,獻頭一道湯飯。只見任醫官到了,冠帶着進來。西門慶迎接至廳上叙禮。任醫官令左右氈包内取出一方壽帕、二星白金來,與西門慶拜壽。説道:「昨日韓明川纔説老先生華誕,恕學生來遲。」西門慶道:「豈敢動勞車駕,又兼謝盛儀。外日多謝妙薬。」彼此拜畢,任醫官還要把盞。西門慶道:「不消了。剛纔已見過禮就是了。」一面脱了衣服,安在左手第四席,與吴大舅相近而坐。獻上湯飯,並手下攢盤。任醫官道:「多謝了。」令僕從領下去,告坐坐下。四個唱的彈着楽器,在旁唱了一套壽詞。西門慶令上席各分頭遞酒。下邊楽工呈上揭帖。到劉薛二内相席前,令揀一段「韓湘子度陳半街」:《升僊會》雜劇。纔唱了一摺,只聽喝道之聲漸近,平安進來禀報:「守備府周爺來了。」西門慶冠帶迎接,未曾相見,就先請寬盛服。周守備道:「我來非為別務,要與四泉把一盞。」薛内相向前來説道:「周大人不消把盞,只見禮兒罷。」於是二人交拜。又道:「我學生來遲,恕罪,恕罪!」叙畢禮數,方寬衣解帶,纔與衆人作揖。左首第三席安下鍾筯,下邊就是湯飯割切,一道添換拿上來。席前打發馬上人兩盤點心、兩盤熟肉、兩瓶酒。周守備擧手謝道:「忒多了。」令左右上來領下去,然後坐下。一面劉薛二内相,每人送周守備一大盃。觥籌交錯,歌舞吹彈,花攢錦簇飲酒。正是:舞低楊柳樓心月,歌罷桃花扇底風。

吃至日暮時分。先是任醫官隔門,去的早,西門慶送出來。任醫官因問:「老夫人貴恙覺好了?」西門慶道:「拙室服下良劑,已覺好些。這兩日不知怎的,又有些不自在。明日還望老先生過來看看。」説畢,任醫官作辭,上馬而去。落後又是倪秀才溫秀才起身。西門慶再三款留不住,送出大門,説道:「容日奉拜請教。寒家就在對門收拾一所書院,與老先生居住,連寳眷都搬來一䖏方便。學生每月奉上束脩,以備薪水之需。」温秀才道:「多承盛愛,感激不盡。」倪秀才道:「觀此,是老先生崇尚斯文之雅意矣!」打發二秀才去了,西門慶陪客飲酒,吃至更闌方散。四個唱的都歸在月娘房内,唱與月娘大妗子楊姑娘衆人聽。

西門慶還在前邊,留下吴大舅應伯爵復坐飲酒,看着打發楽工酒飯吃了,先去了。其餘席上家伙都收了,鮮菓殘饌,都令手下人分散吃了。吩咐從新後邊拿菓碟兒上來,教李銘吴惠鄭奉上來彈唱,拿大盃賞酒與他喫。應伯爵道:「哥今日華誕設席,列位都是喜歡。」李銘道:「今日薛爺和劉爺也費了許多賞賜。落後見桂姐銀姐又出來,每人又遞了一包與他。只是薛爺比劉爺年小,快頑些。」不一時,畫童兒拿上添換菓碟兒來,都是蜜餞减碟、榛松菓仁、紅菱雪藕、蓮子荸薺、酥油蚫螺、冰糖霜梅、玫瑰餅之類。這應伯爵看見酥油蚫螺渾白與粉紅兩樣,上面都沾着飛金。就先揀了一個放在口内,如甘露洒心,入口而化。説道:「倒好喫!」西門慶道:「我的兒,你倒肯吃,此是你六娘親手揀的。」伯爵笑道:「也是我女兒孝順之心。」説道:「老舅,你也請個兒。」於是揀了一個,放在吴大舅口内。又叫李銘吴惠鄭奉近前,每人揀了一個賞他。

正飲酒間,伯爵向玳安道:「你去後邊叫那四個小淫婦出來。我便罷了,也教他唱個兒與老舅聽。再遲一回兒便好去。今日連轎錢四錢,他只唱了兩套。休要便宜了他。」那玳安不動身,説道:「小的叫了他了。在後邊唱與妗子和娘們聽哩,便來。」伯爵道:「賊小油嘴,你幾時去哩?還哄我。」因叫王經:「你去。」那王經又不動。伯爵道:「我使着你們都不去,等我去罷。」於是就往後走。玳安道:「你老人家趁早休進去。後邊有狗哩,好不利害,只咬大腿。」伯爵道:「若咬了我,我直賴到你娘那炕頭子上。」玳安纔入後邊,良久,只聽一陣香風過,覺有笑聲,四個粉頭,都用汗巾兒搭着頭出來。伯爵看見道:「我的兒,誰養的你恁乖?搭上頭兒,心裏要去的情。好自在性兒!不唱個曲兒與俺們聽,就指望去?好容易!連轎子錢,就是四錢銀子。買紅梭兒米,買一石七八斗。夠你家鴇子和你一家大小喫一個月。」董嬌兒道:「哥兒,恁便益衣飯兒,你也入了籍罷了!」洪四兒道:「大爺,這早晚七八有二更,放了俺們去罷了。」齊香兒道:「俺們明日還要起早往門外送殯去哩。」伯爵道:「誰家?」齊香兒道:「是房簷底下開門兒那家子。」伯爵道:「莫不又是王三官兒家?前日被他連累你那場事,多虧你大爹這裏人情,替李桂兒説,連你也饒了。這一遭,雀兒不在那窝兒罷了。」齊香兒笑罵道:「怪老油嘴!汗邪了你,恁胡說。」伯爵道:「你笑話我老,我那些兒放着老?我半邊俏,把你這四個小淫婦兒還不夠擺布!」洪四兒笑道:「哥兒,我看你行頭不怎麽的,光一味好撇!」伯爵道:「我那兒,到跟前看手段還錢。」又道:「鄭家那賊小淫婦兒,喫了糖五老座子兒,百不言語,有些出神的模樣。敢記掛着那孤老兒在家裏?」董嬌兒道:「他剛纔聽見你説,在這裏有些怯牀。」伯爵道:「怯牀不怯牀,拿楽器來,每人唱一套,你們去罷。我也不留你了。」西門慶道:「也罷,你們叫兩個遞酒,兩個唱一套與他聽罷。」齊香兒道:「等我和月姐唱。」當下鄭月兒琵琶,齊香兒彈箏,坐在校牀上,兩個輕舒玉指,款跨鮫綃,啟朱唇,露皓齒,歌美韻,放嬌聲,唱了一套〔越調·鬦鵪鶉〕:「夜去明來,倒有個天長地久。」當下董嬌兒遞吴大舅酒,洪四兒遞應伯爵酒,在席上交盃換盞,倚翠偎紅,翠袖殷勤,金盃瀲灧。正是:

\begin{myquote}
朝赴金谷宴,暮伴綺樓娃,

休道歡娱處,流光逐落霞。
\end{myquote}

當下酒進數巡,歌吟兩套,打發四個唱的去了。西門慶還留吴大舅坐,教春鴻上來唱南曲與大舅聽。吩咐棋童:「備馬來,拿燈籠送大舅。」大舅道:「姐夫,不消備馬,我同應二哥一路走罷。天色晚了。」西門慶道:「無是理。如此,教棋童打燈籠送到家。」當下唱了一套,吴大舅與伯爵起身作别道:「深擾姐夫。」西門慶送至大門首,因和伯爵説:「你明日好歹上心,約會了那位甘夥計來見了,批合同。我會了喬親家,好收拾那邊房子。一兩日卸貨。」伯爵道:「哥不消吩咐,我知道。」一面作辭,與大舅同行,棋童打着燈籠。吴大舅便問:「剛纔姐夫説收拾那裏房子?」伯爵悉把「韓夥計貨船到,無人發賣,他心内要開個緞子舖,收拾對門房子,教我替他尋個夥計」一節,對大舅説了。大舅道:「幾時開張?咱們親朋會定,少不的具菓盒花紅來作賀作賀。」須臾出大街,到伯爵小衚衕口上。大舅要棋童:「打燈籠送你應二叔到家。」伯爵不肯,説道:「棋童,你送大舅,我不消燈籠。進巷内就是了!」一面作辭,分路回家。棋童便送大舅去了。

西門慶打發李銘等唱錢,關門,回後邊月娘房中歇了一夜。到次日,果然伯爵領了甘出身,穿青衣,走來拜見,講説了回買賣之事。西門慶叫將崔本來,會喬大户,那邊收拾房子卸貨,修蓋土庫門面,擇日開張擧事。喬大户對崔本説:「將來凡一應大小事,隨你親家爹這邊只顧䖏,不消多計較。」當下就和甘夥計批立了合同,就立伯爵作保。譬如得利十分為率,西門慶分五分,喬大户分三分,其餘韓道國、甘出身,與崔本三份均分。一面收卸磚瓦木石,修蓋土庫,裏面裝畫牌面。待貨車到日,堆卸貨物。後邊獨自收拾一所書院,請將溫秀才來作西賓,專修書柬,回答往來士夫。每月三兩束修,四時禮物不缺。又撥了畫童兒小廝伏侍他半晚,替他拿茶飯,舀硯水。他若出門望朋友,跟他拿拜帖匣兒。西門慶家中常筵客,就請過來陪侍飲酒,俱不必細説。

不覺過了西門慶生辰,第二日早晨,就請了任醫官來看李瓶兒,討薬,又在對門看看收拾。楊姑娘先家去了,李桂姐吴銀兒還沒家去。吴月娘買了三錢銀子螃蠏,午間煮了,來在後邊院内請大妗子、李桂姐、吴銀兒衆人,都圍着喫了一囬。只見月娘請的劉婆子來看官哥兒,喫了茶,李瓶兒就陪他往前邊房裏去了。劉婆子説:「哥兒驚了,住了奶奶。」又留下幾服薬。月娘與了他三錢銀子,打發去了。孟玉樓、潘金蓮,和李桂姐、吴銀兒、大姐,都在花架底下,放小桌兒、舖氈條,同抹骨牌,賭酒頑耍。那個輸一牌,喫一大盃酒。孫雪娥喫衆人贏了七八鍾酒,又不敢久坐,坐一囬又去了。西門慶在對門房子内,看着收拾打掃,和應伯爵崔本甘夥計喫酒,又使小廝來家要菜兒。慌的雪娥往廚下打發,只拿李嬌兒頂缺。金蓮教吳銀兒、桂姐:「你唱『慶七夕』俺們聽。」當下彈着琵琶,唱〔商調·集賢賓〕:

\begin{myquote}
「暑纔消大火即漸西,斗柄往坎宫移。一葉梧桐飄墜,萬方秋意皆知。暮雲閑聒聒蟬鳴,晚風輕點點螢飛。天階夜涼清似水,鵲橋圖高掛偏宜。金盤内種五生,瓊樓上設筵席。」
\end{myquote}

當日衆姊妹飲酒至晚,月娘裝了盒子,相送李桂姐吴銀兒家去了。潘金蓮喫的大醉歸房。因見西門慶夜間在李瓶兒房裏歇了一夜,早晨請任醫官又來看他,都惱在心裏。知道他孩子不好,進門,不想天假其便,黑影中躧了一脚狗尿。到房中呌春梅點燈來看,大紅緞子新鞋兒上,滿幫子都展汚了。登時柳眉剔竪,星眼圓睜。叫春梅打着燈,把角門關了。拿大棍把那狗沒高低只顧打,打的怪叫起來。李瓶兒那邊使過迎春來説:「俺娘説,哥兒纔喫了老劉的薬,睡着了,教五娘這邊休打狗罷。」這潘金蓮坐着,半日不言語。一面把那狗打了一回,開了門放出去了,又尋起秋菊的不是來。看着那鞋,左也惱,右也惱。因把秋菊喚至跟前説:「論起這早晚,這狗也該打發去了,只顧還放在這屋裏做甚麽?是你這奴才的野漢子?你不打發他出去,教他恁遍地撒屎,把我恁雙新鞋兒,連今日纔三四日兒,躧了恁一鞋幫子屎!知道了我來,你與我點個燈兒出來!你如何恁推聾粧啞裝憨兒?」春梅道:「我頭裏纔對他説,你趂娘不來,早喂他些飯,關到後邊院子裏去罷。他佯打耳睜的不理我,還㧱眼兒瞟着我!」婦人道:「可又來,賊膽大萬殺的奴才!怎麽恁把屁股兒懶待動彈?我知道你在這屋裏成了把頭,便説你恁久慣牢頭,把這打來不作理。」因叫他到跟前,叫春梅:「拿過燈來,教他瞧躧的我這鞋上的齷齪!我纔做的恁雙心愛的鞋兒,就教你這奴才遭塌了我的!」哄得他低頭瞧,提着鞋拽巴兜臉就是幾鞋底子。打的秋菊嘴唇都破了,只顧搵着搽血。那秋菊走開一邊,婦人駡道:「好賊奴才,你走了!」教春梅:「與我採過跪着。取馬鞭子來,把他身上衣服與我扯了,好好教我打三十馬鞭子便罷,但扭一扭兒,我亂打了不算!」春梅於是扯了他衣裳。婦人教春梅把他手拴住,雨點般鞭子輪起來,打的這丫頭殺豬也似叫。那邊官哥纔合上眼兒,又驚醒了。又使了綉春來説:「俺娘上覆五娘,饒了秋菊,不打他罷。只怕唬醒了哥哥。」

那潘姥姥正歪在裏間屋裏炕上,聽見金蓮打的秋菊叫,一𥑮碌子爬起來,在旁邊勸解。見金蓮不依,落後又見李瓶兒使過綉春來説,又走向前奪他女兒手中鞭子,説道:「姐姐,少打他兩下兒罷。惹的他那邊姐姐説,只怕唬了哥哥。為驢紂棍不打緊——倒沒的傷了紫荆樹。」金蓮緊自心裏惱,又聽見他娘説了這一句,越發心中攛上把火一般。須臾,紫漒了面皮,把手只一推,險些兒不把潘姥姥推了一跤。便道:「怪老貨,你不知道,與我過一邊坐着去!不干你事,來勸甚麽膫子?甚麽紫荆樹,驢紂棍,單管外合裏應!」潘姥姥道:「賊作死的短壽命!我怎的外合裏應?我來你家討冷飯喫?教你恁頓摔我!」金蓮道:「你明日就與我夾着那老ず走,恒是他家不敢拿長鍋煮喫了我。」那潘姥姥聽見女兒這等訌他,走那裏邊屋裏嗚嗚咽咽哭起來了。由着婦人打秋菊,打夠約二三十馬鞭子,然後又蓋了十闌杆,打得皮開肉綻,纔放起來。又把他臉和腮頰,都用尖指甲掐的稀爛。李瓶兒在那邊,只是雙手摀着孩子耳朶,腮頰淌淚,敢怒而不敢言。

不想那日西門慶在對門房子裏喫酒,散了,逕往玉樓房中歇了一夜。到次日,周守備家請喫補生日酒,不在家。李瓶兒見官哥兒喫了劉婆子薬不見動靜,夜間又着驚唬,一雙眼只是往上吊吊的。因那日薛姑子王姑子家去,來對月娘説;向房中拿出他壓被的銀獅子一對來,要教薛姑子印造《佛頂心陀羅經》,趕八月十五日嶽廟裏去捨。那薛姑子就要拿着走,被孟玉樓在旁説道:「師父,你且住。大娘,你還使小廝叫將賁四來,替他兑兑多少分兩,就同他往經舖裏講定個數兒來。每一部經多少銀子?咱們捨多少,到幾時有,纔好。你教薛師父去,他獨自一個,怎弄的過來?」月娘道:「你也説的是。」一面使來安兒:「你去瞧賁四來家不曾?你叫了他來。」來安兒一直去了。不一時,賁四來到。向月娘衆人作了揖,把那一對銀獅子上天平兑了,重四十一兩伍錢。月娘吩咐同薛師父往經舖,講印造經數去了。潘金蓮隨即叫孟玉樓:「咱送送他兩位師父去。就前邊看看大姐,他在屋裏做鞋哩。」兩個携着手兒,往前邊來。賁四同來安兒、薛姑子、王姑子,往經舖裏去了。

金蓮與玉樓走出大廳前,來東廂房門首,見大姐正守着針線筐兒,在簷下衲鞋。金蓮拿起來看,卻是沙綠潞紬子鞋面。玉樓道:「大姐,你不要這紅鎖線子,爽利着藍鎖線兒卻不老作些?你明日還要大紅提跟子。」大姐道:「我有一雙是大紅提跟子的。這個我心裏要藍提跟子,所以使大紅線鎖口。」金蓮瞧了一囬,三個都在廳臺基上坐的。玉樓問大姐:「你女婿在屋裏不在?」大姐道:「他不知那裏喫了兩鍾酒,在屋裏睡哩。」孟玉樓便向金蓮説:「剛纔若不是我在旁邊説着,李大姐恁瞎帳行貨,就要把銀子交姑子拿了印經去。經也印不成,沒脚蠏行貨子,藏在那大人家,你那裏尋他去?早是我説,叫將賁四來,同他去了。」金蓮道:「你看麽,你教我幹,恁有錢的姐姐,不賺他些兒是儍子,只像牛身上拔一根毛了!你孩兒若沒命,休説捨經,隨你把萬里江山捨了,也成不的!正是:饒你有錢拜北斗,誰人買得不無常?如今這屋裏,只許人放火,不許俺們點燈。大姐聽着,也不是別人。偏染的白兒不上色,偏你會那等輕狂百勢,大清早晨,刁蹬着漢子請太醫看。他亂他的,俺們又不管。每當在人前,會那等撇清兒説話:『我心裏不耐煩。他爹要便進我屋裏,推看孩子,雌着和我睡。誰耐煩?教我就攛掇往別人屋裏睡去了。』俺們自恁的罷了,背地還嚼説俺們。那大姐姐偏聽他一面詞兒説話。不是俺們爭這個事,怎麽昨日漢子不進你屋裏去,你使丫頭在角門子首叫進屋裏,推看孩子,你便喫薬,一徑把漢子作成在那屋裏和吴銀兒睡了一夜去了。一徑顯你那乖覺,教漢子喜歡你。那大姐姐就沒的話兒説了。昨日晚夕,人進屋裏躧了一鞋狗屎,打丫頭趕狗,也嗔起來。使丫頭過來説,唬了他孩子了。俺娘那老貨,又不知道,㨪他那嘴吃,教他拿小買住,走來勸甚麽的『驢紂棍傷了紫荆樹』。我惱他那等輕聲浪氣,他又來我跟前説長話短,教我墩了他兩句,他今日使性子家去了。去了罷,教我説,他家有你這樣窮親戚也不多,沒你也不少!比是恁地快使性子,到明日不要來他家。怕他拿長鍋煮喫了我?隨我和他家纏去。」玉樓笑道:「你這個沒訓教的子孫,你一個親娘母兒,你這等訌他?」金蓮道:「不是這等説,惱人腸子了!單管黄貓黑尾,外合裏應,只替人説話!喫人家碗半,被人家使喚。得不的人家一個甜棗兒,千也説好,萬也説好。想着迎頭兒養了這個孩子,把漢子調唆的生根也似的,把他便扶的正正兒的,把人恨不的躧到那泥裏頭還躧!今日怎的天也有眼,你的孩兒生出病來了!我只説日頭常晌午,如何也有個錯了的時節兒!」

正説着,只見賁四和來安兒往經舖裏交了銀子,來回月娘話。看見玉樓金蓮和大姐都在廳臺基上坐的,只顧在儀門外立着,不敢進來。來安走來說道:「娘們閃閃兒,賁四來了。」金蓮道:「怪囚根子!你教他進去不是,纔乍見他來?」來安説了,賁四於是低着頭,一直到後邊見月娘、李瓶兒,把上項説了:「銀子四十一兩五錢,眼同兩個師父,交付與翟經兒家收了。講定印造綾殼《陀羅經》五百部,每部五分;絹売經一千部,每部三分。算共該五十五兩銀子。除收過四十一兩五錢,還找與他十三兩五錢。准在十四日早擡經來。」李瓶兒連忙向房裏取出一個銀香毬來,教賁四上天平兑了十五兩。李瓶兒道:「你拿了去。除找與他,別的你收着。換下些錢,到十五日廟上捨經,與你們做盤纏就是了,省的又來問我要。」賁四於是揝香毬出門。月娘使來安送賁四出去。李瓶兒道:「四哥,多累你。」賁四躬着身説道:「小人不敢。」走到前邊,金蓮玉樓又叫住問他:「銀子交付與經舖了?」賁四道:「已交付明白,共一千五百部經,共該給五十五兩銀子。除收過那四十一兩五錢,剛纔六娘又與了這件銀香毬。」玉樓金蓮瞧了瞧,沒言語。賁四便回家去了。玉樓向金蓮説道:「李大姐像這等都枉費了錢。他若是你的兒女,就是榔頭也樁不死。他若不是你兒女,你捨經造像,隨你怎的也留不住他!信着姑子,甚麽繭兒幹不出來。剛纔不是我説着,把這些東西就託他拿的去了。這等着咱家個人兒去,卻不好?」金蓮道:「縱然他背地落,也落不多兒。」兩個説了一回,都立起來。金蓮道:「咱們往前邊大門首走走去。」因問大姐:「你不出去?」大姐道:「我不去。」

這潘金蓮便拉着玉樓手兒,兩個同來到大門裏首站立。因問平安兒:「對門房子都收拾了?」平安道:「這咱哩!従昨日,爹看着都打掃乾淨了。後邊樓上堆貨。昨日教陰陽來破土,樓底下要裝鑲三間土庫擱緞子。門面打開一溜三間,舖子門面都教漆匠裝新油漆。地下墁磚,鑲地平,打架子,要在出月開張。」玉樓又問:「那寫書溫秀才家小,搬過來了不曾?」平安道:「従昨日就過來了。今早爹吩咐,把後邊堆放的那一張涼牀子拆了與他。又搬了兩張桌子,四張椅子與他坐。」金蓮道:「你沒見他老婆,怎的模樣兒?」平安道:「黑影子坐着轎子來,誰看見他來?」

正説着,只聽見遠遠一個老頭兒,斯琅琅搖着驚閨葉過來。潘金蓮便道:「磨鏡子的過來了。」教平安兒:「你叫住他,與俺們磨磨鏡子。我的鏡子這兩日都使的昏了,吩咐你這囚根子看着,過來再不叫!俺們出來站了多大囬,怎的就有磨鏡子的過來了?」那平安一面叫住,磨鏡老兒放下擔兒。見兩個婦人在門裏首,向前唱了兩個喏,立在傍邊。金蓮便問玉樓道:「你也磨?都教小廝帶出來,一答兒裏磨了罷。」於是使來安兒:「你去我屋裏,問你春梅姐討我的照臉大鏡子,兩面小鏡子兒;就把那大四方穿衣鏡也帶出來,教他好生磨磨。」玉樓吩咐來安:「你到我屋裏,教蘭香也把我的鏡子拿出來。」那來安兒去不多時,兩隻手提着大小七面鏡子,懷裏又抱着四方穿衣鏡出來。金蓮道:「賊小囚兒,你拿不了,做兩遭兒拿,如何恁拿出來?一時叮噹了我這鏡子,怎了?」玉樓道:「我沒見你這面大鏡子,是那裏的?」金蓮道:「是舖子人家當的。我愛他且是亮,安在屋裏早晚照照。」因問:「我的鏡子只三面?」玉樓道:「我的大小只兩面。」金蓮道:「這兩面是誰的?」來安道:「這兩面是俺春梅姐的,捎出來也教磨磨。」金蓮道:「賊小肉兒,他放着他的鏡子不使,成日只撾着我的鏡子照。弄的恁昏昏的!」共大小八面鏡子,交付與磨鏡老叟,教他磨。當下絆在坐架上,使了水銀,那消頓飯之間,睜磨的耀眼爭光。婦人拿在手内,對照花容,猶如一汪秋水相似。有詩為證:

\begin{myquote}
蓮萼菱花共照臨,風吹貌動影沉沉。

一池秋水芙蓉現,好似嫦娥入月宫。

翠袖拂塵霜暈退,朱唇呵氣碧雲深,

従教粉蝶飛來撲,始信花香在畫中。
\end{myquote}

那磨鏡老子,須臾將鏡子磨畢,交與婦人看了,付與來安兒收進去了。玉樓便令平安問舖子裏傅夥計櫃上要五十文錢兒與磨鏡的。那老子一手接了錢,只顧立着不去。玉樓教平安問那老子:「你怎的不去,敢嫌錢少?」那老子不覺眼中撲簌簌流下淚來,哭了。平安道:「俺當家的奶奶問你,怎的煩惱?」老子道:「不瞞哥哥説,老漢今年癡長六十一歲。老漢前妻丢下個兒子,二十二歲,尚未娶妻,專一狗油,不幹生理。老漢日逐出來掙錢,便養活他。他又不守本分,常與街上搗子耍錢。昨日惹了祸,同拴到守備府中,當土賊打了他二十大棍。歸來把媽媽的裙襖都去當了。媽媽便氣了一場病,打了寒,睡在炕上半個月。老漢説了他兩句,他便走出來,不往家去。敎老漢日逐找尋他,不着個下落。待要賭氣不尋他,况老漢恁大年紀,止生他一個兒子,往後無人送老。有他在家,見他不成人,又要惹氣。似這等,乃老漢的業障!有這等負屈銜冤,没䖏告訴,所以這等淚出痛腸。」玉樓敎平安兒:「你問他,你這後娶婆兒是今年多大年紀了?」老子道:「他今年癡長五十五歲了,男女花兒没有。如今打了寒纔好些,只是沒將養的,心中想塊臘肉兒喫。老漢在街上恁問了兩三日,走了十數條街巷,白討不出塊臘肉兒來!甚可嗟歎人子!」玉樓笑道:「不打緊處,我屋裏抽替内,有塊臘肉兒哩。」即令來安兒:「你去對蘭香説,還有兩個餅錠,教他拿與你來。」金蓮叫那老頭子:「問你家媽媽兒,喫小米兒粥不喫?」老漢子道:「怎的不喫?那裏有?可知好哩!」金蓮於是叫過來安兒來:「你對春梅説,把昨日你姥姥捎來的新小米兒量二升,就拿兩個醬瓜茄出來,與他媽媽兒喫。」那來安去不多時,拿出半腿臘肉,兩個餅錠,二升小米,兩個醬瓜茄 ,叫道:「老頭子過來,造化了你。你家媽媽子不是害病想喫,只怕害孩子坐月子,想定心湯喫。」那老子連忙雙手接了,安放在擔内,望着玉樓金蓮唱了個喏,揚長挑着擔兒,搖着驚閨葉去了。平安道:「二位娘不該與他這許多東西,被這老油嘴設智誆的去了!他媽媽子是個媒人,昨日打這街上走過去不是,幾時在家不好來?」金蓮道:「賊囚!你不早説,做甚麽來?」平安道:「罷了,也是他的造化!可可二位娘出來看見,叫住他,照顧了他這些東西去了。」正是:

\begin{myquote}
閒來無事倚門楣,正是驚閨一老來;

不獨纖微能濟物,無緣滴水也難為。
\end{myquote}

畢竟未知後來何如,且聽下囬分解。

