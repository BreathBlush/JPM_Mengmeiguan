\includepdf[pages={153,154},fitpaper=false]{tst.pdf}
\chapter*{第七十七囬 \\西門慶踏雪訪愛月 賁四嫂倚牖盼佳期}
\addcontentsline{toc}{chapter}{第七十七囬 西門慶踏雪訪愛月 賁四嫂倚牖盼佳期}
\markboth{{\titlename}卷之八}{第七十七囬 西門慶踏雪訪愛月 賁四嫂倚牖盼佳期}


\begin{myquote}
飛彈參差拂早梅,強欺寒色尚低囬。

風憐落溷留香與,月令深情借艷開。

梁殿得非蕭帝瑞,齊宫應是玉兒媒。

不知謝客離腸醒,臨水應添萬恨來。
\end{myquote}

話説溫秀才求見西門慶不得,自知慚愧,隨携家小搬移原舊家去了。西門慶收拾書院,做了客座,不在話下。

一日尚擧人來拜辭,起身上京會試,問西門慶借皮箱、毡衫。西門慶陪他坐的待茶,又送贐禮與他。因說起:「喬大户雲離守兩位舍親,一授義官,一襲祖職,現任管事。欲求兩篇軸文奉賀,不知老翁可有相知否?借重一言,學生具幣禮拜求。」尚擧人笑道:「老翁何用禮為?學生敝同窗聶兩湖,現在武庫肆業,與小兒為師在舍,本領雜作極富。學生就與他說,老翁差盛使持軸,送到學生那邊。」西門慶連忙致謝,茶畢起身。西門慶這裏隨即封了兩方手帕、五錢白金,差琴童送軸子並毡衫、皮箱,到尚擧人處收下。那消兩日光景,寫成軸文,差人送來。西門慶挂在壁上,但見青緞錦軸,金字輝煌,文不加點,心中大喜。只見應伯爵來問:「喬大户與雲二哥的事幾時擧行?軸文做了不曾?溫老先兒怎的連日不見?」西門慶道:「又題甚麽溫老先生兒,通是個狗類之人!」如此這般,告訴伯爵一遍。伯爵道:「哥,我說此人言過其實,虚浮之甚!早是你有後眼,不然教調壞了咱家小兒們了!」又問:「他二公賀軸,何人寫了?」西門慶道:「昨日尚小塘來拜我,説他朋友聶兩湖善於詞藻,央求聶兩湖作了文章,已寫了來,你瞧。」於是引伯爵到廳上,觀看一遍,喝采不已。說道:「人情都全了。哥,你早送與人家預備。」西門慶道:「明日好日期,備羊酒花紅菓盒,早差人送去。」

正說着,忽報:「夏老爹兒子來拜辭,明日初六日早起身去也。小的答應爹不在家,他說教對何老爹那裏說聲,明早差人那邊看守去。」西門慶觀見六摺帖兒上寫着:「寅家晚生夏承恩頓首拜,謝辭。」西門慶道:「連尚擧人搭他家,就是兩分香絹贐儀。」吩咐琴童:「連忙買了,教你姐夫封了,寫帖子送去。」

正在書房中留伯爵喫飯,忽見平安兒慌慌張張,拿進三個帖兒來報:「參議汪老爹、兵備雷老爹、郎中安老爹來拜。」西門慶看帖兒,「江伯彦、雷起元、安忱拜」,連忙穿衣裳繫帶。伯爵道:「哥,你有事,我喫了飯去罷。」西門慶道:「我明日會你哩。」一面整衣出迎,三員官皆相讓而入,一個白鷴,一個雲鷺,一個穿豸補子,手下跟従許多官吏。進入大廳敍禮,道及向日厚擾之事。少頃茶罷,坐話間,安郎中便道:「雷東谷汪少華並學生又來干凟,有浙江本府趙大尹,新陞大理寺丞,學生三人借尊府奉請。已發柬,定初九日赴會。主席共五席,戲子學生那裏呌來。未知肯允諾否?」西門慶道:「老先生吩咐,學生掃門拱候。」安郎中令吏取分資三兩遞上。西門慶令左右收了,相送出門。雷東谷向西門慶道:「前日錢龍野書到,說那孫文相乃是令夥計,學生一並除他開了。曾來相告不曾?」西門慶道:「正是。多承老先生費心,容當叩拜。」雷兵備道:「你我相愛厚間,何為多較!」言畢,相揖上轎而去。

原來潘金蓮自従當家管理銀錢,另預了一把新等子,每日小廝買進菜蔬來,教拿到跟前,與他瞧過,方數錢與他。他又不數,只教春梅數錢、提等子。小廝被春梅罵的狗血噴了頭,皆出生入死,行動就說「落」,教西門慶打。以此衆小廝皆互相抱怨,都說:「在三娘手裏使錢好,五娘行動沒「打」不說話。」

却說次日,西門慶早往衙門,中午散了,對何千户說:「夏龍溪家小已起身去了,長官沒曾委人那裏看守門户去?」何千户道:「正是。昨日那邊着人來說,學生原差小价去了。」西門慶道:「今日同長官到那裏看看去。」於是出衙門,並馬兩個到了夏家。宅内家小,已是去盡了,伴當在門首伺候。兩位官府下馬,進到廳上。西門慶引着何千户前後觀看了。又到他前邊花亭,見一片空地,無甚花草。西門慶道:「長官來到,明日還收拾了耍子所在,裁些花草,把這座亭子修理修理。」何千户道:「這個一定。學生開春従新修整修整,添些磚瓦木石,蓋三間捲棚,早晚請長官來消閑散悶。」西門慶因問:「府上寳眷有多少來住?」何千户道:「學生這房頭不上數口,還有幾房家人並伴當,不過十數人而已。」西門慶道:「似此還住不了。這宅子,前後五十餘間房。」看了一囬,吩咐家人收拾打掃,関閉門户,不日寫書往東京回老公公話,趕年裏搬取家眷。當日西門慶作別囬家,何千戶看了一囬,還歸衙門裏去了。次日纔搬行李來住,不在言表。

西門慶剛到家下馬,只見何九買了一疋尺頭、四樣下飯、鷄鵝、一罈酒,來謝西門慶。又是劉内相差家人送了一食盒大小純紅挂黄蠟燭、二十張桌圍、八十股官香、一盒沉速料香、一罈自造内酒、一口鮮猪。西門慶進門,劉公公家人就磕頭說道:「家公公多上覆,這些微禮,與老爹賞人。」西門慶道:「前日空過老公公,怎又送這厚禮來?」便令左右:「快收了,請管家等等兒。」少頃,畫童兒㧱出一鍾茶來,打發喫了。西門慶封了五錢銀子賞錢,拿囬帖打發去了。一面請何九進去。見西門慶在廳上站立,換了冠帽,戴着白毡忠靖冠,見何九,一把扯往廳上來。何九連忙倒身磕下頭去:「向蒙老爹天心,超生小人兄弟,感恩不淺!」請西門慶受禮。西門慶不肯受磕頭,拉起還說:「老九,你我舊人,快休如此!」就讓他坐。何九説道:「老爹今非昔比,小人微末之人,豈敢僭坐?」只站立在傍邊。西門慶坐上陪着喫了一盞茶,說道:「老九,你如何又費心送禮來?我断然不受。若有甚麽人欺負你,只顧來說,我親替你出氣。倘縣中派你甚差事,我拿帖兒與你李老爹說。」何九道:「蒙老爹恩典,小人知道。小人如今也老了,差事已告與小兒何欽頂替着哩。」西門慶道:「也罷,也罷!你清閑些了。」說道:「旣你不肯,我把這酒禮收了。那尺頭你還拿去,我也不留你坐了。」那何九千恩萬謝,拜辭去了。

西門慶坐廳上,看着打點禮物:菓盒、花紅、羊酒、軸文並各人分資,先差玳安送往喬大户家去,後呌王經送雲離守家去。玳安囬來,喬家與了五錢銀子。王經到雲離守家管待了茶食,與了一疋眞青大布,一雙琴鞋,囬「門下辱愛生」雙帖兒:「多上覆老爹,改日奉請。」

西門慶滿心歡喜。到後邊月娘房中,擺飯喫,因向月娘說:「賁四去了,吴二舅在獅子街賣貨,我今日倒閒,往那裏看看去。」月娘道:「你去不是。若是要酒菜兒,早使小廝來家說。」西門慶道:「我知道。」一面吩咐備馬,就戴着毡忠靖巾,貂鼠暖耳,綠絨補子𧜽褶,粉底皂靴,琴童玳安跟隨,逕往獅子街來。到房子内,吴二舅與來昭正挂着花栲栳兒發賣紬絹絨線絲綿,擠一舖子人做買賣,打發不開。西門慶下馬,看了看,走到後邊暖房内坐下。吴二舅走來作揖,囬説:「一日也攢銀錢二十兩。」西門慶又吩咐來昭妻一丈青:「二舅茶飯,每日這裏依舊打發,休要悮了!」來昭妻道:「逐日炖茶酒飯,都是我自整理。」

西門慶見天陰晦上來,但見彤雲密布,冷氣侵人,作雪的模樣。忽然想起要往院中鄭月兒家去,即令琴童:「騎馬家中取我的皮襖來。問你大娘,有酒菜兒,捎一盒與你二舅喫。」琴童應諾,到家,不一時取了西門慶長身貂鼠皮襖,後面排軍拿了一盒酒菜,裏面四碟醃鷄下飯,煎炒鵓鴿,四碟海味案酒,一盤韮盒兒,一錫瓶酒。西門慶陪二舅在房中喫了三盃,吩咐二舅:「你晚夕在此上宿,自在用,我家去罷。」於是帶上眼紗,騎馬,玳安琴童跟隨,逕進勾欄,往鄭愛月兒家來。轉過東街口,只見天上紛紛揚揚,飄下一天瑞雪來。正是:拳頭大塊空中舞,路上行人只叫苦。但見:

\begin{myquote}
漠漠嚴寒匝地,這雪兒下得正好;扯絮撏綿裁織,片片大如栲栳。見林間竹屋茅茨,爭些被他壓倒。富室豪家却言,消灾瘴猶嫌小,圍向那紅爐獸炭,穿的是貂裘繡襖。手撚梅花,唱道是國家祥瑞,不念貧民些小。高臥有幽人,吟詠多詩草。
\end{myquote}

西門慶隨路踏着那亂瓊碎玉,貂襖沾濡粉蝶,馬蹄蕩滿銀花。進入勾欄,到於鄭愛月兒家門首下馬。只見丫鬟看見,飛報進來說:「老爹來了。」鄭媽媽出來迎接,至於中堂見禮。説道:「前日多謝老爹重禮,姐兒又在宅内打擾;又教他大娘三娘賞他花翠汗巾。」西門慶道:「那日空了他來。」一面坐下。西門慶令玳安把馬牽進來,自有院落安放。老媽道:「請爹後邊明間坐罷,月姐纔起來梳頭,只說老爹昨日來,倒伺候了一日。今日他心中有些不快,起來的遲些。」這西門慶一面進入他後邊往房明間内,但見綠窗半啟,毡幙低張。地平上黄銅火盆生着炭火。西門慶坐在正面椅上。先是鄭愛香兒出來相見了,遞了茶,然後愛月兒纔出來。頭挽一窝絲杭州攢,翠梅花鈿兒,金鈒釵梳,海獺臥兔兒。打扮的霧靄雲鬟,粉粧玉琢。上穿白綾襖兒,緑遍地錦比甲,下着大幅湘紋裙子。高高顯一對小小金蓮,猶如新月,狀若蛾眉;好似羅浮僊子臨凡境,巫山神女降世間。粉頭出來笑嘻嘻的向西門慶道了萬福,說道:「爹,我那一日來晚了。緊自前邊人散的遲,到後邊大娘又只顧不放俺們,留着喫飯,來家有三更天了。」西門慶笑道:「小油嘴兒,你倒和李桂姐兩個,把應花子打的好響瓜兒。」鄭愛月兒道:「誰教他怪物勞,在酒席上屎口兒傷俺們來。那一日,祝麻子也醉了,哄我,要送俺們來。我便說,沒爹這裏燈籠送俺們?蔣胖子掉在陰溝裏——缺臭了你了!」西門慶道:「我昨日聽見洪四兒說,祝麻子又會着三官兒,大街上請了榮嬌兒。」鄭月兒道:「只在榮嬌兒家歇了一夜,燒了一炷香,不去了。如今還在秦玉芝兒家走着哩。」說了一囬話,道:「爹,只怕你冷,往房裏坐的。」

這西門慶到於房中,脫去貂裘,和粉頭圍爐共坐。房中香氣襲人。只見丫鬟來放桌兒,四碟細巧菜蔬,安下三個薑碟兒。須臾,拿了三甌兒黄芽韮菜肉包的一寸大的水角兒來。姊妹二人,陪西門慶每人喫了一甌兒。愛月兒又撥了上半甌兒,添與西門慶。西門慶道:「我夠了,纔在那邊房子線舖,陪你吴二舅喫了兩個點心來了。心裏要來你這裏走走,不想天氣落雪,家中使小廝取了皮襖,穿上就來了。」愛月兒道:「爹前日不會下我?教昨日等了一日,不見爹。不想爹今日來了!」西門慶道:「昨日家中有兩位士夫來望,亂着,就不曾來得。」愛月兒道:「我要問爹,有貂鼠買個兒與我,我要做個圍脖兒戴。」西門慶道:「不打緊。打巧昨日舍夥計打遼東來,送了我十個好貂鼠。你娘們都沒圍脖兒,到明日一總做了,送一個來與你。」愛香兒道:「爹只認的月姐,就不送與我一個兒!」西門慶道:「你姊妹兩個一家一個。」於是愛香愛月兒連忙起身道了萬福。西門慶吩咐:「休見了桂姐銀姐說。」鄭月兒道:「我知道。」因說到:「明日李桂姐見吳銀兒在那裏過夜,問我他幾時來了?我沒瞞他,教我說昨日請周爺,俺們四個都在這裏唱了一日。爹說有王三官兒在這裏,不敢請你的。今日是親朋會中人喫酒,纔請你來來。他一聲兒也沒言語。」西門慶道:「你這個囬的他好。前日李銘我也不要他唱來,再三央及你應二爹來說;落後你三娘生日,桂姐買了一分禮來,再三與我賠不是,不是你娘們說着,我不理他。昨日我竟留下銀姐,使他知道。」愛月兒道:「不知三娘生日,我失悮了人情。」西門慶道:「等明日你雲老爹擺酒,我請你和銀姐那裏唱一日。」愛月兒道:「爹吩咐,我去。」不一時,丫鬟收拾飯桌下去。粉頭取出個鸂鶒木匣兒,傾出三十二扇象牙牌來,和西門慶在炕毡條上抹牌頑耍,愛香兒也坐在傍邊看牌。院内雪如風舞梨花,紛紛只顧下。但見:

\begin{myquote}
恍惚漸迷鴛甃,頃刻拂滿蜂鬚。似玉龍鱗甲遶空飛,白鶴羽毛搖地落。好若數蠏行沙上,猶賽亂瓊堆砌間。正是:

盡道豐年瑞,豐年瑞若何?長安有貧者,為瑞不宜多!
\end{myquote}

當下三人抹了囬牌,須臾,擺上酒來飲酒。桌上盤堆異菓,肴列珍羞,茶煮龍團,酒斟琥珀,詞歌〔金縷〕,笑啟朱唇。愛香與愛月兒一邊一個捧酒,不免箏排雁柱,款跨鮫綃,姊妹兩個彈着,唱了一套〔青衲襖〕:

\begin{myquote}
「想多嬌情性兒標,想多嬌恩意兒好。想起携手同行共歡笑,吟風詠月將詩句兒嘲。女溫柔,男俊俏,正青春年紀小。誰承望將比目魚分開,瓶墜簪折,今日早魚沉雁杳。」

{\markfont〔罵玉郎〕}「想着俺那多嬌,一去無消耗。想着俺情似漆意如膠,常記的共枕同歡樂。想着他花樣嬌柳樣柔,傾國傾城貌。」

{\markfont〔大迓鼓〕}「千般丰韵嬌。風流俊俏,體態妖嬈,所為諸般妙:搊箏撥阮,歌舞吹簫。縱有丹青難畫描。」

{\markfont〔感皇恩〕}「呀,好敎我無緒無聊,意攘心勞。懶將這杜詩溫,韓文續。我這裏愁懷越焦,這些時容貌添憔。不能够同歡樂成配偶,到有分受煎熬。」

{\markfont〔東歐令〕}「潘郎貌,沈郎腰,可惜相逢無下梢。心腸懊惱傷懷抱,烈火燒祅廟,滔滔綠水淹藍橋,想思病怎生逃!」

{\markfont〔採茶歌〕}「相思病怎生逃,離愁陣擺的堅牢,鐵石人見了也魂消!愁似南山堆積積,悶如東海水滔滔!」

{\markfont〔賺〕}「誰想今朝!自古書生多命薄,傷懷抱。癡心惹的傍人笑,對誰陳告?」

{\markfont〔烏夜啼〕}「想當初偎紅倚翠,踏青鬦草。相逢對景同歡楽。到春來,語呢喃燕子尋巢;到夏來,荷蓮香開滿池沼;到秋來,菊滿荒郊;到冬來,瑞雪飄飄。想當初畫堂歌舞列着佳肴,今日個孤眠旅館無着落,鬼病侵難醫療。好教我情牽意惹,心痒難撓。」

{\markfont〔節節高〕}「悶懨懨睡不着,想多嬌:知音解吕明宫調,諸般妙;閉月容羞花貌,言語嬌媚心聰俏。恰似僊子行來到,金蓮款步鳳頭翹,朱唇皓齒微微笑。」

{\markfont〔鵪鶉兒〕}你看他體態輕盈,更那堪衣穿素縞;你看他脂粉勻施,蛾眉淡掃。看了他萬種妖嬈難畫描,難畫描。酒泛羊羔,寳鴨香飄,銀燭高燒。成就了美滿夫妻,穩取同心到老。」

{\markfont〔尾聲〕}「青霄有路終須到,生前無分也難消,把佳期叮嚀休忘了!」
\end{myquote}

唱一套,姐兒兩個拿上骰盆兒來,和西門慶搶紅頑笑。盃來盞去,各添春色。西門慶忽把眼看見鄭愛月兒房中牀傍側首錦屏風上,挂着一軸《愛月羙人圖》,題詩一首:

\begin{myquote}
「有羙人兮迥出羣,輕風斜拂石榴裙。

花開金谷春三月,月轉花陰夜十分。

玉雪精神聯仲琰,瓊林才貌過文君。

少年情思應須慕,莫使無心托白雲。」

\raggedleft{{\marktext\small(下書)}「三泉主人醉筆。」}
\end{myquote}

西門慶看了,便問:「三泉主人是王三官兒的號?」慌的鄭愛月兒連忙摭説道:「這還是他舊時寫下的。他如今不號三泉了,號小軒了。他告人説,學爹說:『我號四泉,他怎的號三泉?』他恐怕爹惱,因此改了號小軒。」一面走向前,取筆過來,把那「三」字就塗抹了。西門慶滿心歡喜,說道:「我並不知他改號一節。」粉頭道:「我聽見他對一個人說來,我纔曉的。他去世的父親號逸軒,他故此改號小軒。」說畢,鄭愛香兒往下邊去了,獨有愛月兒陪西門慶在房内,兩個並肩疊股,搶紅飲酒。因說起林太太來,怎的大量,好風月:「我在他家喫酒那日,王三官請我到後邊拜見。還是他主意,教三官拜認我做義父,教我受他禮,委托我指教他成人。」粉頭拍手大笑道:「還虧我指與爹這條路兒,到明日,連三官兒娘子不怕不屬了爹!」西門慶道:「我到明日,我先燒與他一炷香;到正月裏,請他和三官娘子往我家看燈喫酒。看他去不去。」粉頭道:「爹,你還不知三官娘子生的怎樣標致,就是個燈人兒沒他那一段兒風流妖艷!今年十九歲兒,只在家中守寡,王三官兒通不着家。爹,你若用個工夫兒,不愁不是你的人。」

兩個說話之間,相挨相凑。只見丫鬟㧱上幾樣細菓碟兒來,都是减碟,菓仁、風菱、鮮柑、螳螂、雪梨、蘋婆、蚫螺、冰糖橙丁之類。粉頭親手奉與西門慶下酒。又用舌尖噙鳳香餅密送入他口中,又用纖手掀起西門慶藕合緞𧜽子,看見他白綾褲子。西門慶一面解開褲帶,露出那話來教他弄。粉頭見根下束着銀托子,那話猙獰跳腦,紫漒光鮮。西門慶令他品之。這粉頭眞個低垂粉項,輕啓朱唇,半吞半吐,或進或出,嗚咂有聲。品弄了一囘,靈犀已透,淫心似火,欲求媾歡。粉頭便往後邊去了。西門慶出房更衣,見雪越下得甚緊。囘到房中,丫鬟向前挂起錦幔,款設鴛枕,展放鮫綃,薰熱香球,牀上舖得被褥甚厚,打發脱靴解帶,先上牙牀。粉頭澡牝囬來,掩上雙扉,共入鴛帳。正是:得多少動人春色嬌還媚,惹蝶芳心轉意濃。有詩為証:

\begin{myquote}
聚散無憑在夢中,起來殘燭映紗紅。

鍾情自古多神會,誰道陽臺路不通。
\end{myquote}

兩個雲雨歡娱,到一更時分起來。丫鬟掌燈進房,整衣理鬢,復篩羙酒,重整佳肴,又飲夠幾盃。問玳安:「有燈籠傘没有?」玳安道:「琴童家去取燈籠傘來了。」這西門慶方纔作别了。鴇子粉頭相送出門,看着上馬。鄭月兒揚聲呌道:「爹若呌我,早些來說。」西門慶道:「我知道。」一面上馬,打着傘,出院門,一路踏雪到家中。對着吴月娘,只說在獅子街和吴二舅飲酒,不在話下。一宿晚景題過。

到次日,卻是初八日,打聽何千戶行李都搬過夏家房子内去了。西門慶這邊送了四盒細茶食、五錢折帕慶房賀儀過去。只見應伯爵驀地走來,西門慶見雪晴天有風色甚冷,留他前邊書房中向火,呌小廝放桌兒,拿菜兒留他喫粥。因說起:「昨日喬親家雲二哥禮並折帕,都送過去了。你的人情,我這邊已是替你每家封了二錢,出上了,你那裏不消與他罷,只等發柬請喫酒。」那應伯爵擧手謝了。西門慶道:「何大人已搬過去了。今日我送茶並慶房人情,你不送些茶兒與他?」伯爵道:「他請人?」又問:「昨日安大人三位來做甚麽?那兩位是何人?」西門慶道:「那兩位一個雷兵備,一個是汪參議,都是浙江人。因在我這裏擺酒,明日要請杭州趙霆知府,新陞京堂大理寺丞,是他們本府父母官,如何不敬?代一張桌面,餘者散席。戲子他那裏呌來,俺這裏少不的呌兩個小優兒答應便了。通身只三兩分資。」伯爵道:「大凡文職,好細。三兩銀子夠做甚麽,哥少不得賠些兒。」西門慶道:「這雷兵備就是問黄四小舅子孫文相的,昨日沒曾對我提起開除他罪名來了?」伯爵道:「你說他不仔細?如今還記着,折准擺這席酒纔罷了!」

說話之間,伯爵呌應寳:「你叫那個人來見你大爹。」西門慶便問:「是何人?」伯爵道:「我那邊左近住一個小後生,倒也是舊人家出身,父母都沒了,自幼在王皇親家宅内答應,好幾年了,也有了媳婦兒了。因在莊子上和一般家人不和,出來了。如今閒着,做不的甚麽買賣兒。他與應寳是朋友,央及應寳,要尋個人家,做房家人。今早應寳對我說:『爹倒好擧薦與大爹宅内答應,只怕大爹少人使。』我便說:『不知你大爹用不用。』」因問應寶:「他呌甚麽名字?你呌他進來。」應寳道:「他姓來,叫來友兒。」只見那來友兒穿着青布四塊瓦,布襪靸鞋,趴在地上磕了個頭,起來簾外站立。伯爵道:「若論這軀勞的,膂力儘有,掇輕服重,都去的。」因問:「你多少年紀了?」那人道:「小的二十歲了。」又問:「你媳婦沒子女?」那人道:「只光兩口兒。」應寳道:「不瞞爹説,他媳婦纔十九歲兒,厨竈針線,大小衣裳,都會做。」西門慶見那人低頭並足,為人樸實,便道:「既是你應二爹來說,用心在我這裏答應。」吩咐:「揀個好日期,寫紙文書,兩口兒搬進來罷。」那來友兒磕了個頭,西門慶教琴童兒領着,後邊見月娘衆人磕頭去,對月娘說:「就把來旺兒原住的那一間房,與他居住。」伯爵坐了囬,家去了。應寳同他寫了一紙投身文書,交與西門慶收了,改名來爵,不在話下。

初九日,西門慶與安郎中汪參議雷兵備擺酒請趙知府。那日早晨,來爵兒兩口兒就搬進來。他媳婦兒後邊見月娘衆人磕頭。月娘見他穿着紫紬襖、青布披襖、綠布裙子,生的五短身材,瓜子面皮兒,搽胭抹粉,施點朱唇,纏的兩隻脚趫趫的。問起來,諸般針指都會做。起了他個名字,叫做惠元,與惠秀、惠祥,一遞三日上竈,不題。

卻説賁四娘子,自従他家長兒與了夏家,每日買東買西,只央及平安兒和來安、畫童兒,或是隔壁韓嫂兒的兒子小雨兒。西門慶家中這些大官兒,常在他屋裏坐的,打平和兒喫酒;賁四娘子兒和氣,就定出菜兒來,或要茶水,應手而至。就是賁四一時舖中歸來撞見,亦不見怪。以此今日他不在家,使着那個不替他動彈?玳安與平安,常在他屋裏坐的多。

一日,門外楊姑娘没了,安童兒來報丧。西門慶這邊整治了一張插桌,三牲湯飯,又封了五兩香儀。吴月娘、李嬌兒、孟玉樓、潘金蓮,四頂轎子起身,都往北邊與他燒紙弔孝。琴童兒、棋童兒、來爵兒、來安兒四個,都跟轎子,不在家。西門慶在對過緞舖子書房内,看着毛襖匠與月娘做貂鼠圍脖,先趲出一個圍脖兒,使玳安送與院中鄭月兒去,封了十兩銀子,與他過節。鄭家管待玳安酒饌,與了他三錢銀子買瓜子兒嗑。走來囬西門慶話,說:「月姨多上覆,多謝了,前日空過了爹來。與了小的三錢銀子。」西門慶道:「你收了罷。」因問他:「賁四不在家,你頭裏従他屋裏出來,做甚麽來?」玳安道:「賁四娘子,従他女孩兒嫁了,没人使。常央及小的們替他買買甚麽兒。」西門慶道:「他旣沒人使,你們替他勤勤兒也罷。」又悄悄向玳安道:「你慢慢和他說,如此這般:『爹要來你這屋裏來看你看兒,你心如何?』看他怎的說。他若肯了,你問他討個汗巾兒來與我。」玳安道:「小的知道了。」領了西門慶言語,應諾下去。

西門慶使陳經濟看着裁貂鼠,就走到家中來。只見王經向顧銀舖内,取了金赤虎,又是四對金頭銀簪兒,交與西門慶。西門慶留下兩對在書房内,餘者袖進李瓶兒房内。坐下,與了如意兒那赤虎,又與他一對簪兒;把那一對簪兒,就與了迎春。二人接了,連忙插燭也似磕了頭。西門慶令迎春取飯去。須臾,拿了飯來。喫了飯,出來在書房内坐下。只見玳安慢走到跟前,見王經在傍,不言語。西門慶使王經後邊取茶去。那玳安方説:「小的將爹言語對他説了,他笑了。約會晚上些,伺候等爹過去坐坐。叫小的拿了這汗巾兒來。」西門慶見紅綿紙兒包着一方紅綾織錦づ紋汗巾兒,聞了聞,噴鼻香,滿心歡喜,連忙袖了。只見王經拿茶來,喫了,又走過對門,看着匠人做生活去。

忽報花大舅來了。西門慶道:「請過來這邊坐。」花子由走到書房暖閣兒裏,作揖坐下,致謝外日多有相擾。叙話間,書童兒對門拿過茶來喫了。花子由悉言:「門外客人有五百包無錫米,凍了河,緊等要賣了囘家去。我想着,姐夫倒好買下等價錢。」西門慶道:「我平白要他做甚麽?凍河還沒人要,到開河船來了,越發價錢跌了。如今家中也沒銀子。」即吩咐玳安:「收拾放桌兒,家中說看菜兒來。」一面使畫童兒:「請你應二爹來陪你花爹坐。」不一時,伯爵來到。三人共坐在一處,圍爐飲酒,桌上擺設四盤四碟,都是煎炒鷄魚,燒爛下飯。又叫孫雪娥烙了兩筯餅,又是四碗肚肺乳線湯。良久,只見吴道官徒弟應春,送節禮疏誥來。西門慶請來同坐喫酒,攬李瓶兒百日經,與他銀子罷。喫至日落時分,二人先起身去了。次後甘夥計收了舖子,又請來坐,與伯爵擲骰猜枚,談話,不覺到掌燈已後,吴月娘衆人轎子到了,來安走來囬話。伯爵道:「嫂子們今日都往那裏去了?」西門慶道:「北邊他楊姑娘沒了。今日三日念經,我這裏備了張插桌祭祀,又封了香儀兒,都去弔問弔兒。」伯爵道:「他老人家也高壽了。」西門慶道:「敢也有七十五六了,男花女花都没有,只靠他門外侄兒那裏養活。材兒也是我這裏替他備下的,這幾年了。」伯爵道:「好,好兒!老人家有了黄金入櫃,就是一塲事了。哥的大陰騭!」說畢,酒過數巡,伯爵與甘夥作辭去了。西門慶道:「十一日該姐夫這裏上宿。」玳安道:「那邊舖子裏,傅二叔也家去了,只小的一個在舖子裏睡。」西門慶就起身走過來,吩咐後生王顯:「仔細火燭。」王顯道:「小的知道。」看着把門關上了。

這西門慶見沒人,兩三步就走入賁四家來。只見賁四娘子兒,在門首獨自站立已久,見對門関的門響,西門慶従黑影中走至跟前。這婦人連忙把封門一開,西門慶鑽入裏面。婦人還扯上封門,說道:「爹請裏邊紙門内坐罷。」原來裏間槅扇鑲着後半間,紙門内又有個小炕兒,籠着旺旺的火,桌上點着燈,兩邊護炕,従新糊的雪白,挂着四扇弔屏兒。那婦人頭上勒着翠藍銷金箍兒,䯼髻插着四根金簪兒,耳朶上兩個丁香兒,上穿紫紬襖,青綃絲披襖,玉色綃裙子。向前與西門慶道了萬福,連忙遞了一盞茶兒與西門慶喫。因悄悄說:「只怕隔壁韓嫂兒知道。」西門慶道:「不妨事,黑影子他那裏曉的。」於是不由分說,把婦人摟到懷中就親嘴。拉近枕頭來,解衣按在炕沿子上,扛起腿來就聳。那話上已束着托子,剛插入牝中,纔拽了幾拽,婦人下邊淫水直流,把一條藍布褲子都濕了。西門慶拽出那話來,向順袋内取出包兒顫聲嬌來,蘸了些在龜頭上,攮進去,方纔澀住淫津,肆行抽拽。婦人雙手扳着西門慶肩膊,兩相迎湊,在下柔聲顫語,呻吟不絶。這西門慶乘着酒興,架其兩腿在胳膊上,只顧没稜露腦,鋭進長驅,肆行𢵞磞,何止二三百度。须臾,弄的婦人雲髻鬅鬆,舌尖冰冷,口不能言。西門慶則氣喘吁吁,靈龜暢羙,一泄如注。良久拽出那話來,淫水隨出,用帕搽之。兩個整衣繫帶,復理殘粧。西門慶向袖中掏出五六兩一包碎銀子,又是兩對金頭簪兒,遞與婦人:「節間買花翠帶。」婦人拜謝了,悄悄打發出來。那邊玳安在舖子裏,專心只聽這邊門環兒響,便開大門,放西門慶進來。自知更無一人曉的。後次朝來暮往,也入港一二次。正是若要人不知,除非己莫為,不想被韓嫂兒冷眼睃見,傳的後邊金蓮知道了。這金蓮亦不說破他。

一日,臘月十五日,喬大户家請喫酒。西門慶這裏會同應伯爵、吴大舅,一齊起身。那日有許多親朋,做戯飲酒,至二更方散。第二日每家一張桌面,俱不必細說。

單表崔本治了二千兩湖州紬絹貨物,臘月初旬起身,雇船裝載,趕至臨清馬頭,教後生榮海看守貨物,便雇頭口來家取車稅銀兩。到門首下頭口,琴童道:「崔大哥來了,請廳上坐。爹在對門房子裏,等我請去。」一面走到對門,不見西門慶。因問平安兒,平安兒道:「爹敢進後邊去了。」這琴童兒走到上房問月娘。月娘道:「見鬼的賊囚,你爹従早晨出去,再幾時進來!」又到各房裏並花園書房都瞧遍了,沒有。琴童在大門首揚聲道:「着恐殺人!不知爹往那裏去了,白尋不着。大白日裏把爹來不見了!崔大哥來了這一日,只顧教他坐着。」那玳安分明知道,不言語,不想西門慶従前邊進來,把衆小廝喫了一驚。原來西門慶在賁四屋裏入港,纔出來。那平安打發西門慶進去了,望着琴童兒吐舌頭兒,都替他捏兩把汗,都道:「管情崔大哥去了,有幾下子打。」不想西門慶走到廳上,崔本見了,磕頭畢,交了書帳說:「船到馬頭,少車税銀兩。我従臘月初一日起身,在揚州與他兩個分路,他們往杭州去了。俺們都到苗青家住了兩日。」因説:「苗青替老爹使了十兩銀子,招了揚州衛一個千户家女子,十六歲了,名喚楚雲。說不盡的花如臉,玉如肌,星如眼,月如眉,腰如柳,襪如鈎,兩隻脚兒恰剛三寸。端的有沉魚落雁之容,閉月羞花之貌。腹中有三千小曲、八百大曲。端的風流如水晶盤内走明珠,態度似紅杏枝頭籠曉日。苗青如今還養在家,替他打箱奩、治衣服,待開春韓夥計保官兒船上带來,伏侍老爹,消愁解悶。」西門慶聽了,滿心歡喜。説道:「你船上捎了來也罷,又費煩他治甚衣服,打甚粧奩,愁我家沒有?」於是恨不的騰雲展翅,飛上揚州搬取嬌姿,賞心楽事。正是:鹿分鄭相應難辨,蝶化莊周未可知。有詩為証:

\begin{myquote}
聞道揚州一楚雲,偶憑幽鳥語來眞。

不知好物都離隔,試把梅花問主人。
\end{myquote}

西門慶陪崔本喫了飯,兑了五十兩銀子做車税錢,又寫書與錢主事,令煩青目。崔本言訖,當下作辭,往喬大户家回話去了。平安見西門慶不尋琴童兒,都說:「我兒,你不知有多少造化!爹進來若不是喜歡,綁着鬼有幾下打。」琴童笑道:「只你知爹性兒?」

比及起了貨來,獅子街卸下,就是下旬時分。西門慶正在家打發送節禮,忽見荆都監差人拿帖兒來問:「宋大巡題本已上京數日,未知旨意下來不曾。伏惟老翁差人察院衙門一打聽為妙。」這西門慶即差答應節級,㧱着五錢銀子,往巡按公衙書辦打聽。果然昨日東京邸報下來,寫抄得一紙全報來與西門慶觀看。上面寫着甚的?

\begin{myquote}[\markfont]
「山東巡按監察御史宋喬年一本,循例擧劾地方文武官員,以勵人心,以隆

聖治事。竊惟吏以撫民,武以禦亂,所以保障地方,以司民命者也。苟非其人,則䖏置乖方,民受其害,國何賴焉!此國家莫急於文武兩途,而激勸之典不容不亟擧也。臣奉

命按臨山東等處,親歷省察風俗,至於吏政民瘼,監司守禦,無不留心咨訪;復令安撫大臣,詳加鑒别,各官賢否,頗得其實。茲當差滿之期,敢不一一陳之:山東左布政陳四箴,操履忠貞,撫民有方;廉使趙訥,綱紀肅清,士民服習;提學副使陳正彙,操砥礪之行,嚴督率之條。又訪得兵備副使雷起元,軍民咸服其恩威,僚幕悉推其練達;濟南府知府張叔夜,經濟可觀,才堪司牧;東平府知府胡師文,居任清愼,視民如傷;徐州府知府韓邦奇,志務清修,才堪廊廟;萊州府知府葉照,屏海寇而道不拾遺,惠民疇而懇田不滷。此數臣者,皆當薦獎而優擢者也。又訪得左參議馮廷鵠,傴僂之形,桑楡之景,形若木偶,尚肆貪婪。東昌府知府徐松,縱妾父而通賄,謗聞致騰於公堂;慕羨餘而誅求,詈聲輙遍於閭閻。此二臣者,所當亟賜罷斥者也。再訪得左軍院僉書守禦周秀,器宇恢弘,操持老練,得將帥之體,軍心允服,□□□□□賊盗潛消。濟州兵馬都監荆忠,年力精強,才猷練達,冠武科而稱為儒將,勝算可以臨戎,肅號令而極其嚴明,長策卒能禦侮;兖州兵馬都監溫璽,夙閑韜略,熟習弓馬,休養騎卒以備不虞,併力設險以防不測。此三臣者,所當亟賜遷擢者也。清河縣千户吴鎧,以練達之才,得衛守之法。驅兵以擣中堅,靡攻不克;儲食以資糧餉,無人不飽。推心置腹,人思効命。實一方之保障,為國家之屏藩。宜特加超擢,鼓舞臣僚。陛下誠以臣言可採,擧而行之,庶幾官爵不濫,而人心思奮,守牧得人而

聖治有賴矣!等因。奉

欽依:該部知道。續該吏兵二部題前事:看得御史宋喬年所奏内,劾擧地方文武官員,無非體國之忠,出于公論。詢訪得實,以裨

聖治之事。伏乞

聖明俯賜施行,天下幸甚,生民幸甚。奉欽依:擬行。」
\end{myquote}

西門慶一見,滿心歡喜,拿着邸報,走到後邊對月娘說:「宋道長本下來了。已是保擧你哥陞指揮僉事,現任管屯。周守禦與荆大人都有獎勵,轉副參統制之任。如今快使小廝請他來,對他説聲。」月娘道:「你使人請去,我教丫鬟看下酒菜兒。我愁他這一上任,也要銀子使。」西門慶道:「不打緊,我借與他幾兩銀子也罷了。」不一時,請得吴大舅到了。西門慶送那題奏旨意與他瞧。吳大舅連忙拜謝西門慶與月娘,說道:「多累姐夫姐姐扶持,恩當重報,不敢有忘。」西門慶道:「大舅,你若上任擺酒沒銀子使,我這裏兑二十兩銀子,你那裏使着。」那吴大舅又作揖謝了。於是就在月娘房中,安排上酒來喫酒。月娘也在旁邊陪坐。西門慶即令陳經濟把全抄寫了一本,與大舅㧱着。即差玳安拿帖,送邸報往荆都監周守禦兩家報喜去。正是:勸君不費鐫研石,路上行人口是碑。

畢竟未知後來如何,且聽下囬分解。

