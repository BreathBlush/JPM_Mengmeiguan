\includepdf[pages={3,4},fitpaper=false]{tst.pdf}
\chapter*{第二囬 \\西門慶簾下遇金蓮 王婆子貪賄說風情}
\addcontentsline{toc}{chapter}{第二囬 西門慶簾下遇金蓮 王婆子貪賄說風情}
\markboth{{\titlename}卷之一}{第二囬 西門慶簾下遇金蓮 王婆子貪賄說風情}


\begin{myquote}
月老姻缘配未眞,金蓮賣俏逞花容。
只因月下星前意,惹起門旁簾外心。

王婆誘財施巧計,鄆哥賣果被嫌嗔。
那知後日蕭牆祸,血濺屏幃滿地紅。
\end{myquote}

話説武松自從搬離哥家,撚指不覺雪晴,過了十數日光景。

却說本縣知縣,自從到任以來,却得二年有餘,賺得許多金銀。要使一心腹人,送上東京親眷處收寄,三年任滿朝覲,打點上司。一來却怕路上小人,須得一個有力量的人去方好。猛可想起都頭武松:「須得此人英雄膽力,方了得此事。」當日就喚武松到衙内商議道:「我有個親戚,在東京城内做官,姓朱名勔,現做殿前太尉之職。要送一擔禮物,捎封書去問安。只恐途中不好行,須得你去方可。你休推辭辛苦,囬來我自重賞你。」武松應道:「小人得蒙恩相擡擧,安敢推辭?旣蒙差遣,只得便去。小人自來也不曾到東京,就那裏觀光上國景緻,走一遭,也是恩相擡擧。」知縣大喜,賞了武松三盃酒,十兩路費,不在話下。

且說武松領了知縣的言語,出的縣門來,到下處呌個土兵,却來街上買了一瓶酒並菜蔬之類,逕到武大家。武大恰街上囬來,見武松在門前坐地,教土兵去廚下安排。那婦人餘情不斷,見武松把將酒食來,心中自思:「莫不這廝思想我了?不然却又囬來!那廝一定強我不過,我且慢慢問他。」婦人便上樓去,重勻粉面,再挽雲鬟,換了些顔色衣服穿了,來到門前迎接武松。婦人拜道:「叔叔!不知怎的錯見了,好幾日並不上門,教奴心裏沒理會處。每日教你哥哥去縣裏尋叔叔陪話,歸來只説沒尋處。今日再喜得叔叔來家。没事壞鈔做甚麽!」武松道:「武二有句話,特來要和哥哥說知。」婦人道:「旣如此,請樓上坐。」

三個人來到樓上,武松讓哥嫂上首坐了,他便掇杌子打横。土兵擺上酒來,熱下飯,一齊拿上來。武松勸哥嫂喫。婦人便把眼來睃武松,武松只顧喫酒。酒至數巡,武松問迎兒討副勸盃,叫土兵篩一盃酒,拿在手裏,看着武大道:「大哥在上,武二今日蒙知縣相公差往東京幹事,明日便要起程。多是兩三個月,少是一個月便囬。有句話特來和你說:你從來為人懦弱,我不在家,恐怕外人來欺負。假如你每日賣十扇籠炊餅,你從明日為始,只做五扇籠炊餅出去賣。每日遲出早歸,不要和人喫酒。歸家便下了簾子,早閉門,省了多少是非口舌。若是有人欺負你,不要和他爭執,待我囬來,自和他理論。大哥,你依我時,滿飲此盃!」武大接了酒道:「我兄弟見得是,我都依你説。」喫過了一盃,武松再斟第二盞酒,對那婦人說道:「嫂嫂是個精細的人,不必要武松多說。我的哥哥,為人質朴,全靠嫂嫂做主。常言表壯不如裏壯。嫂嫂把得家定,我哥哥煩惱做甚麽?豈不聞古人云:籬牢犬不入!」那婦人聽了這幾句話,一點紅從耳畔起,須臾紫漒了面皮,指着武大駡道:「你這個混沌東西,有甚言語在别人䖏說,來欺負老娘!我是個不帶頭巾的男子漢,叮叮噹噹響的婆娘,拳頭上也立的人,胳膊上走得馬,人面上行的人,不是那腲膿血搠不出來鱉老婆!自従嫁了武大,眞個螻蟻不敢入屋裏來,有甚麽「籬笆不牢犬兒鑽得入來」?你休胡言亂語,一句句都要下落。丢下塊磚兒,一個個也要着地!」武松笑道:「若得嫂嫂這般做主,最好。只要心口相應。却不應心頭不似口頭。旣然如此,我武松都記得嫂嫂說的話了,請過此盃!」那婦人一手推開酒盞,一直跑下樓來,走到半胡梯上,發話道:「旣是你聰明伶俐,却不道長嫂如母?我初嫁武大時,不曾聽得有甚小叔。那裏走得來,是親不是親,便要做喬家公!自是老娘晦氣了,偏撞着這許多鳥事!」一面哭下樓去了。有詩為證:

\begin{myquote}
苦口良言諫勸多,金蓮懷恨起風波。

自家惶愧難存坐,氣殺英雄小二哥。
\end{myquote}

那婦人做出許多喬張致來。武大武松喫了幾杯酒,坐不住,都下的樓來,弟兄洒淚而别。武大道:「兄弟去了,早早囬來,和你相見。」武松道:「哥哥,你便不做買賣也罷,只在家裏坐的。盤纏兄弟自差人送與你。」臨行,武松又吩咐道:「哥哥,我的言語,休要忘了,在家仔細門户!」武大道:「理會得了。」武松辭了武大,回到縣前下處,收拾行裝並防身器械。次日,領了知縣禮物,金銀馱垜,討了脚程,起身上路,往東京去了。不題。

只說武大,自從兄弟武松説了去,整整乞那婆娘駡了三四日。武大忍氣吞聲,由他自罵,只依兄弟言語,每日只做一半炊餅出去,未晚便囬家。歇了擔兒,先便去除了簾子,關上大門,却來屋裏動彈。那婦人看了這般,心内焦躁起來,罵道:「不識時濁物!我倒不曾見日頭在半天裏,便把牢門關了,也喫鄰舍家笑話,說我家怎生禁鬼!聽信你兄弟說空生有卵鳥嘴,也不怕别人笑恥!」武大道:「由他笑也罷,我兄弟說的是好話,省了多少是非。」被婦人噦在臉上道:「呸!濁東西,你是個男子漢,自不做主,却聽別人調遣!」武大搖手道:「由他,我兄弟説的是金石之語!」原來武松去後,武大每日只是晏出早歸,到家便關門。那婦人氣生氣死,和他合了幾場氣。落後鬧慣了,自此婦人約莫武大歸來時分,先自去收簾子,關上大門。武大見了,心裏自也暗喜,尋思道:「恁的却不好?」有詩爲證:

\begin{myquote}
愼事關門幷早歸,眼前恩愛隔崔嵬。

春心一點如絲亂,任鎖牢籠總是虚。
\end{myquote}

白駒過隙,日月攛梭,纔見梅開臘底,又早天氣囬陽。一日,三月春光明媚時分,金蓮打扮光鮮,單等武大出門,就在門前簾下站立,——約莫將及他歸來時分,便下了簾子,自去房内坐的。一日,也是合當有事,却有一個人從簾子下走過來。自古没巧不成話,姻緣合當凑着:婦人正手裏拿着叉竿放簾子,忽被一陣風將叉竿刮倒,婦人手擎不牢,不端不正,却打在那人頭巾上。婦人便慌忙陪笑,把眼看那人,也有二十五六年紀,生得十分博浪:

\begin{myquote}
頭上戴着纓子帽兒,金玲瓏簪兒,金井玉欄杆圈兒;長腰身,穿綠羅褶兒;脚下細結底陳橋鞋兒,清水布襪兒;腿上勒着兩扇玄色挑絲護膝兒;手裏搖着洒金川扇兒,越顯出張生般龐兒,潘安的貌兒,——可意的人兒,風風流流従簾子下丢與奴個眼色兒。
\end{myquote}

這個人被叉竿打在頭上,便立住了脚。待要發作時,囬過臉來看,却不想是個羙貌妖嬈的婦人。但見他:

\begin{myquote}
黑鬒鬒賽鴉翎的鬢兒,翠彎彎的新月的眉兒,清泠泠杏子眼兒,香噴噴櫻桃口兒,直隆隆瓊瑤鼻兒,粉濃濃紅豔腮兒,嬌滴滴銀盆臉兒,輕嬝嬝花朵身兒,玉纖纖葱枝手兒,一捻捻楊柳腰兒,軟濃濃白麵臍肚兒,窄多多尖趫脚兒,肉奶奶胸兒,白生生腿兒;更有一件緊揪揪、紅縐縐、白鮮鮮、黑裀裀,正不知是什麽東西!
\end{myquote}

觀不盡這婦人容貌,且看他怎生打扮?但見:

\begin{myquote}
頭上戴着黑油油頭髮䯼髻,四面上貼着飛金。一逕裏墊出香雲一結,周圍小簪兒齊插。六鬢斜插一朶並頭花,排草梳兒後押。難描八字彎彎柳葉,襯在腮兩朶桃花。玲瓏墜兒最堪誇,露賽玉酥胸無價。毛青布大袖衫兒褶兒又短,襯湘裙碾絹綾紗。通花汗巾兒袖中兒邊搭剌,香袋兒身邊低掛。抹胸兒重重紐扣。褲腿兒臟頭垂下。往下看:尖趫趫金蓮小脚,雲頭巧緝山牙;老鴉鞋兒白綾高底,步香塵偏襯登踏。紅紗膝褲扣鶯花,行坐處風吹裙袴。口兒裏常噴出異香蘭麝,櫻桃初笑臉生花。人見了魂飛魄散,賣弄殺偏俏的寃家!
\end{myquote}

那人見了,先自酥了半邊,那怒氣早已鑽入爪哇國去了,變做笑吟吟臉兒。這婦人情知不是,叉手望他深深拜了一拜,說道:「奴家一時被風失手,誤中官人,休怪。」那人一面把手整頭巾,一面把腰曲着地,還喏道:「不妨!娘子請方便。」却被這間壁住的賣茶王婆子看見。那婆子笑道:「兀的誰家大官人打這屋簷下過?打的正好!」那人笑道:「倒是我的不是,一時衝撞,娘子休怪!」婦人答道:「官人不要見責!」那人又笑着大大地唱個喏,囬應道:「小人不敢!」那一雙積年招花惹草,慣覷風情的賊眼,不離這婦人身上。臨去也囬頭了七八遍,方一直搖搖擺擺,遮着扇兒去了。有诗為證:

\begin{myquote}
風日清和漫出遊,偶従簾下識嬌羞。

只因臨去秋波轉,惹起春心不肯休。
\end{myquote}

當時婦人見了那人生的風流浮浪,語言甜淨,更加幾分留戀:「倒不知此人姓甚名誰,何處居住。他若沒我情意時,臨去也不囬頭七八遍了。不想這段姻緣,却在他身上!」却是在簾下眼巴巴的,看不見那人,方纔收了簾子,關上大門歸房去了。

看官聽説:莫不這人無有家業的?原是清縣一個破落户財主,就縣門前開着個生薬舖。從小兒也是個好浮浪子弟,使得些好拳棒,又會賭博,雙陸象棋,拆白道字,無不通曉。近來發跡有錢,專在縣裏管些公事,與人把攬說事過錢,交通管吏。因此滿縣人都懼怕他。那人複姓西門,單名一個慶字,排行第一,人都叫他做西門大郎。近來發跡有錢,人都稱他做西門大官人。他父母雙亡,兄弟俱無,先頭渾家已早逝,身邊止有一女。新近又娶了清河左衛吴千户之女,填房為繼室。房中也有四五個丫鬟婦女。又常與勾欄裏的李嬌兒打熱,今也娶在家裏。南街子又占着窠子卓二姐,名卓丢兒,包了些時,也娶來家居住。專一嫖風戲月,調占良人婦女。娶到家中,稍不中意,就令媒人賣了;一個月倒在媒人家去二十餘遍。人都不敢惹他。

這西門大官人自從簾下見了那婦人一面,到家尋思道:「好一個雌兒,怎能够得手?」猛然想起那間壁賣茶王婆子來:「堪可如此如此,這般這般。撮合得此事成,我破幾兩銀子謝他,也不値甚的。」於是連飯也不喫,走出街上閑遊,一直逕踅入王婆茶坊裏來,便去裏邊水簾下坐了。王婆笑道:「大官人,却纔唱得好個大肥喏!」西門慶道:「乾娘,你且來,我問你:間壁這個雌兒是誰的娘子?」王婆道:「他是閻羅大王的妹子,五道將軍的女兒,問他怎的?」西門慶說:「我和你説正話,休取笑。」王婆道:「大官人怎的不認的?他老公便是縣前賣熟食的。」西門慶道:「莫不是賣棗糕徐三的老婆?」王婆搖手道:「不是!若是他,也是一對兒。大官人再猜!」西門慶道:「敢是賣餶飿的李三娘子兒?」王婆搖手道:「不是!若是他,倒是一雙。」西門慶道:「莫不是花胳膊劉小二的婆兒?」王婆大笑道:「不是!若是他時,又是一對兒。大官人再猜!」西門慶道:「乾娘,我其實猜不着了。」王婆哈哈笑道:「好敎大官人得知了罷,笑一聲。他的蓋老,便是街上賣炊餅的武大郎!」西門慶聽了,跌脚笑道:「莫不是人叫他三寸丁谷樹皮的武大郎麽?」王婆道:「正是他。」西門慶聽了,叫起苦來,説道:「好一塊羊肉,怎生落在狗口裏!」王婆道:「便是這般苦事。自古駿馬却馱癡漢走,羙妻常伴拙夫眠。月下老偏這等配合!」西門慶道:「乾娘,我少你多少茶果錢?」王婆道:「不多。由他,歇些時却算不妨。」西門慶又道:「你兒子王潮,跟誰出去了?」王婆道:「說不的,跟了一個淮上客人,至今不歸,又不知死活。」西門慶道:「却不叫他跟我?那孩子倒乖覺伶俐!」王婆道:「若得大官人擡擧他時,十分之好。」西門慶道:「待他歸來,却再計較。」說畢,作謝起身去了。約莫未及兩個時辰,又踅將來王婆門首簾邊坐的,朝着武大門前。半歇,王婆出來道:「大官人,喫個梅湯?」西門慶道:「最好,多加些酸味兒。」王婆做了個梅湯,雙手遞與西門慶,喫了,將盞子放下。西門慶道:「乾娘,你這梅湯做得好,有多少在屋裏?」王婆笑道:「老身做了一世媒,那討得一個在屋裏?」西門慶笑道:「我問你這梅湯,你却說做媒,差了多少!」王婆道:「老身只聽得大官人問這媒做得好,老身只道說做媒。」西門慶道:「乾娘,你旣是撮合山,也與我做頭媒!說頭好親事,我自重重謝你。」王婆道:「看這大官人作戲!你宅上大娘子得知,老婆子這臉上怎吃得那等刮子!」西門慶道:「我家大娘子最好性格。現今也有幾個身邊人在家,只是沒一個中得我意的。你有這般好的,與我主張一個,便來說也不妨。若是囬頭人兒也好,只是要中得我意。」王婆道:「前日有一個倒好,只怕大官人不要。」西門慶道:「若是好時,與我說成了,我自重謝你。」王婆道:「生的十二分人才,只是年紀大些。」西門慶道:「自古半老佳人可共,便差一兩歲也不打緊。眞個多少年紀?」王婆子道:「那娘子是丁亥生,屬猪的,交新年恰九十三歲了。」西門慶笑道:「你看這風婆子,只是扯着風臉取笑!」說畢,西門慶笑了起身去。

看看天色晚了,王婆恰纔點上燈來,正要關門。只見西門慶又踅將來,俓去簾子底下,那凳子上坐了,朝着武大門前,只顧將眼睃望。王婆道:「大官人,喫個和合湯?」西門慶道:「最好,乾娘放甜些。」王婆連忙取一鍾來,與西門慶喫了。坐到晚夕,起身道:「乾娘,記了帳目,明日一發還錢。」王婆道:「由他,伏惟安置,來日再請過访。」西門慶笑了去。到家甚是寢食不安,一片心只在婦人身上。當晚無話。

次日清晨,王婆恰纔開門,把眼看外時,只見西門慶又早在街前來囬踅走。王婆道:「這刷子踅得緊。你看我着些甜糖,抹在這廝鼻子上,教他舐不着!那廝會討縣裏人便益,且敎他來老娘手裏納些財鈔,賺他幾貫風流錢使。」原來這開茶坊的王婆子,也不是守本分的。便是積年通殷勤,做媒婆,做賣婆,做牙婆,又會收小的,也會抱腰,又善放刁。還有一件不可說,䯼髻上着綠,陽臘灌腦袋。端的看不出這婆子的本事來!但見:

\begin{myquote}
開言欺陸賈,出口勝隨何。只憑說六國唇鎗,全仗話三齊舌劍:隻鸞孤鳳,霎時間交仗成雙;寡婦鰥男,一席話搬唆擺對。解使三重門内女,遮麽九級殿中僊。玉皇殿上侍香金童,把臂拖來;王母宫中傳言玉女,攔腰抱住。略施奸計,使阿羅漢抱住比丘尼;纔用機關,教李天王摟定鬼子母。甜言說誘,男如封陟也生心;軟語調和,女似麻姑須亂性。藏頭露尾,攛掇淑女害相思;送暖偸寒,調弄嫦娥偸漢子。這婆子,端的慣調風月巧安排,常在公門遭鬦毆。
\end{myquote}

這婆子正開門,在茶局子裏整理茶鍋。張見西門慶踅過幾遍,奔入茶局子水簾下,對着武大門首,不住把眼只望簾子裏瞧。王婆只推不看見,只顧在茶局子内搧火,不出來問茶。西門慶叫道:「乾娘,點兩盃茶來我喫。」王婆應道:「大官人來了?連日少見,且請坐。」不多時,便濃濃點兩盞稠茶,放在桌子上。西門慶道:「乾娘,相陪我喫個茶。」王婆哈哈笑道:「我又不是你影射的,緣何陪着你喫茶?」西門慶也笑了。一會,便問:「乾娘,間壁賣的是甚麽?」王婆道:「他家賣的拖煎河漏子、軟巴子肉、翻包着菜肉匾食、餃窝窝、蛤蜊麵,熱盪溫和大辣酥。」西門慶笑道:「你看這風婆子,只是風!」王婆笑道:「我不是風,他家自有親老公!」西門慶道:「我和你説正經話。他家如灋做得好炊餅,我要問他買四五十個,拿的家去。」王婆道:「若要買他炊餅,少間等他街上囬來買,何消上門上户?」西門慶道:「乾娘說的是。」喫了茶,坐了一會,起身去了。

良久,王婆只在茶局裏張時,冷眼張見他在門前,踅過東,看一看,又轉西去,又睃一睃,一連走了七八遍。少頃,逕入茶房裏來。王婆道:「大官人稀行,好幾日不見面了!」西門慶便笑將起來,去身邊摸出一兩一塊銀子,遞與王婆,說道:「乾娘,權且收了做茶錢。」王婆笑道:「何消得許多!」西門慶道:「多者乾娘只顧收着。」婆子暗道:「來了,這刷子當敗。且把銀子收了,到明日與老娘做房錢!」便道:「老身看大官人有些渴,喫個寬煎茶兒如何?」西門慶道:「如何乾娘便猜得着?」婆子道:「有甚難猜處?自古入門休問榮枯事,觀看形容便得知。老身異樣蹺蹊古怪的事不知猜够多少。」西門慶道:「我一件心上的事,乾娘若猜得着時,便輸與你五兩銀子。」王婆笑道:「老身也不消三智五猜,只一智便猜個中節。大官人,你將耳朶來:你這兩日脚步兒勤趕趁得頻,一定是記掛着間壁那個人。我這猜如何?」西門慶笑將起來道:「乾娘端的智賽隨何,機強陸賈。不瞞乾娘說,不知怎的,喫他那日叉簾子時見了一面,恰似收了我三魂六魄的一般,日夜只是放他不下。到家茶飯懶喫,做事没入脚處。不知你會弄手段麽?」王婆哈哈笑道:「老身不瞞大官人說,我家賣茶,呌做鬼打更。三年前六月初三日下大雪那一日,賣了一個泡茶,直到如今不發市,只靠些雜趁養口。」西門慶道:「乾娘,如何叫做雜趂?」王婆笑道:「老身自従三十六歲沒了老公,丢下這個小廝,無得過日子。迎頭兒跟着人說媒,次後攬人家些衣服賣,又與人家抱腰收小的,閑常也會做牽頭,做馬泊六,也會針灸看病,也會做貝戎兒。」西門慶聽了,笑將起來:「我並不知乾娘有如此手段!端的與我說這件事成,便送十兩銀子與你做棺材本。你好敎這雌兒會我一面。」王婆便哈哈笑了。有詩為證:

\begin{myquote}
西門浪子意猖狂,死下功夫戲女娘。

虧殺賣茶王老母,生敎巫女會襄王。
\end{myquote}

畢竟婆子有甚計策說來,要知後項事情,且聽下囬分解。

